\begin{figure}[!tb]
  \centering
  \begin{subfigure}{0.55\linewidth}
    \centering
    \caption{}
    \label{subfig:visual_abstract_1}
  \vspace{-1.3\baselineskip}
  \begin{minipage}[t]{0.33\linewidth}
    \centering
    % load "spectrumdefault{left, center, right}" styles
    \input{fig/spectrum_default_style_main.tex}
    % customize "zmystyle" as you wish
    \pgfkeys{/pgfplots/zmystyle/.style={spectrumdefaultleft,
        title={mb, exact}}}
    % This file was created by tikzplotlib v0.9.7.
\begin{tikzpicture}

\begin{axis}[
axis line style={white!80!black},
log basis x={10},
tick pos=left,
title={full\_batch\_exact, one\_group, N=128, D=895210},
xlabel={eigenvalues},
xmin=0.0001, xmax=35.8794860839844,
xmode=log,
ylabel={density},
ymin=1.11705508533742e-06, ymax=1,
ymode=log,
zmystyle
]
\draw[draw=white,fill=black] (axis cs:0.0001,1.11705508533742e-06) rectangle (axis cs:0.000155434142340701,0.99871426815899);
\draw[draw=white,fill=black] (axis cs:0.000155434142340701,1.11705508533742e-06) rectangle (axis cs:0.000241597726051892,1.11705508533742e-06);
\draw[draw=white,fill=black] (axis cs:0.000241597726051892,1.11705508533742e-06) rectangle (axis cs:0.000375525353403394,1.11705508533742e-06);
\draw[draw=white,fill=black] (axis cs:0.000375525353403394,1.11705508533742e-06) rectangle (axis cs:0.00058369461233445,5.58528041795857e-06);
\draw[draw=white,fill=black] (axis cs:0.00058369461233445,1.11705508533742e-06) rectangle (axis cs:0.000907260714570931,2.68093507478535e-05);
\draw[draw=white,fill=black] (axis cs:0.000907260714570931,1.11705508533742e-06) rectangle (axis cs:0.00141019291048744,4.80334210778594e-05);
\draw[draw=white,fill=black] (axis cs:0.00141019291048744,1.11705508533742e-06) rectangle (axis cs:0.00219192125576551,7.26086604072757e-05);
\draw[draw=white,fill=black] (axis cs:0.00219192125576551,1.11705508533742e-06) rectangle (axis cs:0.00340699400468264,9.04815617377603e-05);
\draw[draw=white,fill=black] (axis cs:0.00340699400468264,1.11705508533742e-06) rectangle (axis cs:0.00529563191077756,0.000105003294068835);
\draw[draw=white,fill=black] (axis cs:0.00529563191077756,1.11705508533742e-06) rectangle (axis cs:0.00823122004203757,0.000123993251732419);
\draw[draw=white,fill=black] (axis cs:0.00823122004203757,1.11705508533742e-06) rectangle (axis cs:0.0127941262765169,0.000138514984063382);
\draw[draw=white,fill=black] (axis cs:0.0127941262765169,1.11705508533742e-06) rectangle (axis cs:0.0198864404478903,0.000137397927730282);
\draw[draw=white,fill=black] (axis cs:0.0198864404478903,1.11705508533742e-06) rectangle (axis cs:0.0309103181522725,0.000130695589731351);
\draw[draw=white,fill=black] (axis cs:0.0309103181522725,1.11705508533742e-06) rectangle (axis cs:0.0480451879147667,0.000115056801067177);
\draw[draw=white,fill=black] (axis cs:0.0480451879147667,1.11705508533742e-06) rectangle (axis cs:0.0746786257712956,9.27156744040709e-05);
\draw[draw=white,fill=black] (axis cs:0.0746786257712956,1.11705508533742e-06) rectangle (axis cs:0.116076081479435,7.26086604072757e-05);
\draw[draw=white,fill=black] (axis cs:0.116076081479435,1.11705508533742e-06) rectangle (axis cs:0.180421861710252,5.5852815410002e-05);
\draw[draw=white,fill=black] (axis cs:0.180421861710252,1.11705508533742e-06) rectangle (axis cs:0.280437173344456,3.79799140794064e-05);
\draw[draw=white,fill=black] (axis cs:0.280437173344456,1.11705508533742e-06) rectangle (axis cs:0.435895115192459,2.23411254153434e-05);
\draw[draw=white,fill=black] (axis cs:0.435895115192459,1.11705508533742e-06) rectangle (axis cs:0.677529833804408,1.45217310832009e-05);
\draw[draw=white,fill=black] (axis cs:0.677529833804408,1.11705508533742e-06) rectangle (axis cs:1.05311268627626,5.58528041795857e-06);
\draw[draw=white,fill=black] (axis cs:1.05311268627626,1.11705508533742e-06) rectangle (axis cs:1.63689667179461,2.2341114184372e-06);
\draw[draw=white,fill=black] (axis cs:1.63689667179461,1.11705508533742e-06) rectangle (axis cs:2.54429630280743,1.11705508533742e-06);
\draw[draw=white,fill=black] (axis cs:2.54429630280743,1.11705508533742e-06) rectangle (axis cs:3.95470513687488,2.23411141854822e-06);
\draw[draw=white,fill=black] (axis cs:3.95470513687488,1.11705508533742e-06) rectangle (axis cs:6.1469620116051,3.351167751648e-06);
\draw[draw=white,fill=black] (axis cs:6.14696201160511,1.11705508533742e-06) rectangle (axis cs:9.55447768274709,4.46822408474777e-06);
\draw[draw=white,fill=black] (axis cs:9.55447768274709,1.11705508533742e-06) rectangle (axis cs:14.8509204413116,4.46822408485879e-06);
\draw[draw=white,fill=black] (axis cs:14.8509204413116,1.11705508533742e-06) rectangle (axis cs:23.0834008176524,1.11705508533742e-06);
\draw[draw=white,fill=black] (axis cs:23.0834008176524,1.11705508533742e-06) rectangle (axis cs:35.8794860839844,1.11705508533742e-06);
\end{axis}

\end{tikzpicture}

  \end{minipage}
  \hspace{1.9ex}
  \begin{minipage}[t]{0.33\linewidth}
    \centering
    % load "spectrumdefault{left, center, right}" styles
    \input{fig/spectrum_default_style_main.tex}
    % customize "zmystyle" as you wish
    \pgfkeys{/pgfplots/zmystyle/.style={spectrumdefaultcenter,
        title={sub, exact}}}
    % This file was created by tikzplotlib v0.9.7.
\begin{tikzpicture}

\definecolor{color0}{rgb}{0.274509803921569,0.6,0.564705882352941}

\begin{axis}[
axis line style={white!80!black},
log basis x={10},
tick pos=left,
title={frac\_batch\_exact, one\_group, N=128, D=895210},
xlabel={eigenvalues},
xmin=0.0001, xmax=35.8794860839844,
xmode=log,
ylabel={density},
ymin=1.11705508533742e-06, ymax=1,
ymode=log,
zmystyle
]
\draw[draw=white,fill=color0] (axis cs:0.0001,1.11705508533742e-06) rectangle (axis cs:0.000155434142340701,0.999840260942811);
\draw[draw=white,fill=color0] (axis cs:0.000155434142340701,1.11705508533742e-06) rectangle (axis cs:0.000241597726051892,1.11705508533742e-06);
\draw[draw=white,fill=color0] (axis cs:0.000241597726051892,1.11705508533742e-06) rectangle (axis cs:0.000375525353403394,1.11705508533742e-06);
\draw[draw=white,fill=color0] (axis cs:0.000375525353403394,1.11705508533742e-06) rectangle (axis cs:0.00058369461233445,1.11705508533742e-06);
\draw[draw=white,fill=color0] (axis cs:0.00058369461233445,1.11705508533742e-06) rectangle (axis cs:0.000907260714570931,1.11705508533742e-06);
\draw[draw=white,fill=color0] (axis cs:0.000907260714570931,1.11705508533742e-06) rectangle (axis cs:0.00141019291048744,1.11705508533742e-06);
\draw[draw=white,fill=color0] (axis cs:0.00141019291048744,1.11705508533742e-06) rectangle (axis cs:0.00219192125576551,1.11705508533742e-06);
\draw[draw=white,fill=color0] (axis cs:0.00219192125576551,1.11705508533742e-06) rectangle (axis cs:0.00340699400468264,1.11705508533742e-06);
\draw[draw=white,fill=color0] (axis cs:0.00340699400468264,1.11705508533742e-06) rectangle (axis cs:0.00529563191077756,1.11705508533742e-06);
\draw[draw=white,fill=color0] (axis cs:0.00529563191077756,1.11705508533742e-06) rectangle (axis cs:0.00823122004203757,1.11705508533742e-06);
\draw[draw=white,fill=color0] (axis cs:0.00823122004203757,1.11705508533742e-06) rectangle (axis cs:0.0127941262765169,1.11705508533742e-06);
\draw[draw=white,fill=color0] (axis cs:0.0127941262765169,1.11705508533742e-06) rectangle (axis cs:0.0198864404478903,5.58528041795857e-06);
\draw[draw=white,fill=color0] (axis cs:0.0198864404478903,1.11705508533742e-06) rectangle (axis cs:0.0309103181522725,6.70233675105834e-06);
\draw[draw=white,fill=color0] (axis cs:0.0309103181522725,1.11705508533742e-06) rectangle (axis cs:0.0480451879147667,1.00535057505797e-05);
\draw[draw=white,fill=color0] (axis cs:0.0480451879147667,1.11705508533742e-06) rectangle (axis cs:0.0746786257712956,1.11705620837905e-05);
\draw[draw=white,fill=color0] (axis cs:0.0746786257712956,1.11705508533742e-06) rectangle (axis cs:0.116076081479435,1.22876184168903e-05);
\draw[draw=white,fill=color0] (axis cs:0.116076081479435,1.11705508533742e-06) rectangle (axis cs:0.180421861710252,1.78729000826112e-05);
\draw[draw=white,fill=color0] (axis cs:0.180421861710252,1.11705508533742e-06) rectangle (axis cs:0.280437173344456,2.45752380816539e-05);
\draw[draw=white,fill=color0] (axis cs:0.280437173344456,1.11705508533742e-06) rectangle (axis cs:0.435895115192459,2.23411254152324e-05);
\draw[draw=white,fill=color0] (axis cs:0.435895115192459,1.11705508533742e-06) rectangle (axis cs:0.677529833804408,2.23411254153434e-05);
\draw[draw=white,fill=color0] (axis cs:0.677529833804408,1.11705508533742e-06) rectangle (axis cs:1.05311268627626,1.56387874163006e-05);
\draw[draw=white,fill=color0] (axis cs:1.05311268627626,1.11705508533742e-06) rectangle (axis cs:1.63689667179461,1.00535057505797e-05);
\draw[draw=white,fill=color0] (axis cs:1.63689667179461,1.11705508533742e-06) rectangle (axis cs:2.54429630280743,5.58528041795857e-06);
\draw[draw=white,fill=color0] (axis cs:2.54429630280743,1.11705508533742e-06) rectangle (axis cs:3.95470513687488,1.11705508533742e-06);
\draw[draw=white,fill=color0] (axis cs:3.95470513687488,1.11705508533742e-06) rectangle (axis cs:6.1469620116051,3.351167751648e-06);
\draw[draw=white,fill=color0] (axis cs:6.14696201160511,1.11705508533742e-06) rectangle (axis cs:9.55447768274709,4.46822408485879e-06);
\draw[draw=white,fill=color0] (axis cs:9.55447768274709,1.11705508533742e-06) rectangle (axis cs:14.8509204413116,3.351167751648e-06);
\draw[draw=white,fill=color0] (axis cs:14.8509204413116,1.11705508533742e-06) rectangle (axis cs:23.0834008176524,3.351167751648e-06);
\draw[draw=white,fill=color0] (axis cs:23.0834008176524,1.11705508533742e-06) rectangle (axis cs:35.8794860839844,1.11705508533742e-06);
\end{axis}

\end{tikzpicture}

  \end{minipage}
  \hspace{-3.5ex}
  \begin{minipage}[t]{0.33\linewidth}
    \centering
    % load "spectrumdefault{left, center, right}" styles
    \input{fig/spectrum_default_style_main.tex}
    % customize "zmystyle" as you wish
    \pgfkeys{/pgfplots/zmystyle/.style={spectrumdefaultright,
        title={mb, mc}}}
    % This file was created by tikzplotlib v0.9.7.
\begin{tikzpicture}

\definecolor{color0}{rgb}{0.4,0.4,0.4}

\begin{axis}[
axis line style={white!80!black},
log basis x={10},
tick pos=left,
title={full\_batch\_mc, one\_group, N=128, D=895210},
xlabel={eigenvalues},
xmin=0.0001, xmax=35.8794860839844,
xmode=log,
ylabel={density},
ymin=1.11705508533742e-06, ymax=1,
ymode=log,
zmystyle
]
\draw[draw=white,fill=color0] (axis cs:0.0001,1.11705508533742e-06) rectangle (axis cs:0.000155434142340701,0.999858133844141);
\draw[draw=white,fill=color0] (axis cs:0.000155434142340701,1.11705508533742e-06) rectangle (axis cs:0.000241597726051892,1.11705508533742e-06);
\draw[draw=white,fill=color0] (axis cs:0.000241597726051892,1.11705508533742e-06) rectangle (axis cs:0.000375525353403394,1.11705508533742e-06);
\draw[draw=white,fill=color0] (axis cs:0.000375525353403394,1.11705508533742e-06) rectangle (axis cs:0.00058369461233445,1.11705508533742e-06);
\draw[draw=white,fill=color0] (axis cs:0.00058369461233445,1.11705508533742e-06) rectangle (axis cs:0.000907260714570931,1.11705508533742e-06);
\draw[draw=white,fill=color0] (axis cs:0.000907260714570931,1.11705508533742e-06) rectangle (axis cs:0.00141019291048744,1.11705508533742e-06);
\draw[draw=white,fill=color0] (axis cs:0.00141019291048744,1.11705508533742e-06) rectangle (axis cs:0.00219192125576551,1.11705508533742e-06);
\draw[draw=white,fill=color0] (axis cs:0.00219192125576551,1.11705508533742e-06) rectangle (axis cs:0.00340699400468264,1.11705508533742e-06);
\draw[draw=white,fill=color0] (axis cs:0.00340699400468264,1.11705508533742e-06) rectangle (axis cs:0.00529563191077756,1.11705508533742e-06);
\draw[draw=white,fill=color0] (axis cs:0.00529563191077756,1.11705508533742e-06) rectangle (axis cs:0.00823122004203757,1.11705508533742e-06);
\draw[draw=white,fill=color0] (axis cs:0.00823122004203757,1.11705508533742e-06) rectangle (axis cs:0.0127941262765169,1.11705508533742e-06);
\draw[draw=white,fill=color0] (axis cs:0.0127941262765169,1.11705508533742e-06) rectangle (axis cs:0.0198864404478903,1.11705508533742e-06);
\draw[draw=white,fill=color0] (axis cs:0.0198864404478903,1.11705508533742e-06) rectangle (axis cs:0.0309103181522725,4.46822408474777e-06);
\draw[draw=white,fill=color0] (axis cs:0.0309103181522725,1.11705508533742e-06) rectangle (axis cs:0.0480451879147667,6.70233675116937e-06);
\draw[draw=white,fill=color0] (axis cs:0.0480451879147667,1.11705508533742e-06) rectangle (axis cs:0.0746786257712956,1.34046747499901e-05);
\draw[draw=white,fill=color0] (axis cs:0.0746786257712956,1.11705508533742e-06) rectangle (axis cs:0.116076081479435,1.22876184168903e-05);
\draw[draw=white,fill=color0] (axis cs:0.116076081479435,1.11705508533742e-06) rectangle (axis cs:0.180421861710252,1.45217310832009e-05);
\draw[draw=white,fill=color0] (axis cs:0.180421861710252,1.11705508533742e-06) rectangle (axis cs:0.280437173344456,1.8989956415822e-05);
\draw[draw=white,fill=color0] (axis cs:0.280437173344456,1.11705508533742e-06) rectangle (axis cs:0.435895115192459,2.01070127490328e-05);
\draw[draw=white,fill=color0] (axis cs:0.435895115192459,1.11705508533742e-06) rectangle (axis cs:0.677529833804408,2.34581817484432e-05);
\draw[draw=white,fill=color0] (axis cs:0.677529833804408,1.11705508533742e-06) rectangle (axis cs:1.05311268627626,1.56387874163006e-05);
\draw[draw=white,fill=color0] (axis cs:1.05311268627626,1.11705508533742e-06) rectangle (axis cs:1.63689667179461,1.22876184168903e-05);
\draw[draw=white,fill=color0] (axis cs:1.63689667179461,1.11705508533742e-06) rectangle (axis cs:2.54429630280743,3.351167751648e-06);
\draw[draw=white,fill=color0] (axis cs:2.54429630280743,1.11705508533742e-06) rectangle (axis cs:3.95470513687488,1.11705508533742e-06);
\draw[draw=white,fill=color0] (axis cs:3.95470513687488,1.11705508533742e-06) rectangle (axis cs:6.1469620116051,3.351167751648e-06);
\draw[draw=white,fill=color0] (axis cs:6.14696201160511,1.11705508533742e-06) rectangle (axis cs:9.55447768274709,5.58528041795857e-06);
\draw[draw=white,fill=color0] (axis cs:9.55447768274709,1.11705508533742e-06) rectangle (axis cs:14.8509204413116,2.23411141854822e-06);
\draw[draw=white,fill=color0] (axis cs:14.8509204413116,1.11705508533742e-06) rectangle (axis cs:23.0834008176524,3.351167751648e-06);
\draw[draw=white,fill=color0] (axis cs:23.0834008176524,1.11705508533742e-06) rectangle (axis cs:35.8794860839844,1.11705508533742e-06);
\end{axis}

\end{tikzpicture}

  \end{minipage}
  \end{subfigure}
  \hfill
  \begin{subfigure}{0.44\linewidth}
    \centering
    \vspace{-4.7ex}
    \caption{}
    \label{subfig:visual_abstract_2}
    \vspace{-0.8\baselineskip}
    \begin{tikzpicture}
% PDF Picture. Set the origin to the south west corner
\node[inner sep=0pt] (pdf_plot) at (0,0){\includegraphics[width=\the\columnwidth]{fig/visual_abstract/vivit_quantities.pdf}};

\coordinate (origin) at (pdf_plot.south west);

% Text
\node[anchor=south west, inner sep=0pt] at ($ (origin) + (22mm, 23mm) $)
{$\footnotesize \vtheta_t$};

\node[anchor=south west, inner sep=0pt] at ($ (origin) + (42mm, 23mm) $)
{$\textcolor{oo_gelb}{\footnotesize \ve_k}$};

\node[anchor=south west, inner sep=0pt] at ($ (origin) + (69mm, 47mm) $)
{$\textcolor{oo_rot}{\footnotesize \gL}$};

\node[anchor=south west, inner sep=0pt] at ($ (origin) + (69mm, 28mm) $)
{$\textcolor{oo_blau}{\footnotesize m_{\vtheta_t}}$};

\node[anchor=south west, inner sep=0pt] at ($ (origin) + (23mm, 50mm) $)
{$\textcolor{oo_gelb}{\footnotesize \mathcal{E}}$};

\node[anchor=south west, inner sep=0pt, rotate=21] at ($ (origin) + (55mm, 3mm) $)
{\footnotesize $\textcolor{oo_blau}{\footnotesize \gamma_k} ,\textcolor{oo_blau}{\footnotesize \lambda_k}$};

\node[anchor=south west, inner sep=0pt, rotate=30] at ($ (origin) + (66mm, -1mm) $)
{\footnotesize $\textcolor{oo_gruen}{\footnotesize \gamma_{nk}} ,\textcolor{oo_gruen}{\footnotesize \lambda_{nk}}$};
\end{tikzpicture}

%%% Local Variables:
%%% mode: latex
%%% TeX-master: "../../thesis"
%%% End:

    \vspace{-0.5\baselineskip}
  \end{subfigure}

  \vspace{-3.0ex}

  \caption{
  \textbf{Overview of \vivittitle's quantities:}
\textbf{(a)} \ggn eigenvalue distribution of \deepobs' \threecthreed architecture on
\cifarten \cite{schneider2019deepobs}
% ($D = 895,210$, $C = 10$)
for settings with different costs on a mini-batch of size $N = 128$.
From left to right: Exact \ggn,
% on the mini-batch,
exact GGN on a mini-batch fraction,
% ($\nicefrac{1}{8}$, as in \cite{zhang2017blockdiagonal}),
\mc approximation of the \ggn.
% on the mini-batch.
%
\textbf{(b)} Pictorial illustration: Loss function $\textcolor{oo_rot}{\gL}$
from \Cref{eq:objective-function} and quadratic model $\textcolor{oo_blau}{q}$
around $\vtheta_t \in \mathbb{R}^2$ from \Cref{eq:quadratic_model} (both
represented by their contour lines). The low-rank structure provides efficient
access to the \ggn{}'s eigenvectors $\{\textcolor{oo_gelb}{\ve_k}\}$, along
which $\textcolor{oo_blau}{q}$ decouples into one-dimensional parabolas
characterized by the directional derivatives $\textcolor{oo_blau}{\gamma_k},
\textcolor{oo_blau}{\lambda_k}$ and per-sample contributions
$\textcolor{oo_gruen}{\gamma_{nk}}, \textcolor{oo_gruen}{\lambda_{nk}}$
(\Cref{eq:gammas-lambdas}). $\textcolor{oo_gelb}{\mathcal{E}}$ is the \ggn{}'s
top-$1$ eigenspace.}
  \label{fig:visual_abstract}
\end{figure}

%%% Local Variables:
%%% mode: latex
%%% TeX-master: "../main"
%%% End:
