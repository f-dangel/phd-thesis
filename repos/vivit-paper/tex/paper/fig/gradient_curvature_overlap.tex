
\begin{figure}[!b]
  \centering

  % pgfplots style "overlapdefault"
  \pgfkeys{/pgfplots/overlapdefault/.style={
      width=\linewidth,
      height=0.18\textheight,
      every axis plot/.append style={line width = 1.2pt},
      every axis plot post/.append style={
        mark size=1, mark options={opacity=0.3}
      },
      tick pos = left,
      xmajorticks = true,
      ymajorticks = true,
      ylabel near ticks,
      xlabel near ticks,
      xtick align = inside,
      ytick align = inside,
      legend cell align = left,
      legend columns = 1,
      legend pos = south east,
      legend style = {
        fill opacity = 0.9,
        text opacity = 1,
        font = \small,
      },
      xticklabel style = {font = \small, inner xsep = -5ex},
      xlabel style = {font = \small},
      axis line style = {black},
      yticklabel style = {font = \small, inner ysep = -4ex},
      ylabel style = {font = \small, inner ysep = -4ex},
      title style = {font = \small, inner ysep = -3ex},
      grid = major,
      grid style = {dashed}
    }
  }

  % pgfplots style for first few rows "overlapdefaultfirst"
  \pgfkeys{/pgfplots/overlapdefaultfirst/.style={
      overlapdefault,
      xlabel=\empty,
      xticklabels=\empty,
      title=\empty,
  }}

  % pgfplots style for first last row "overlapdefaultfirst"
  \pgfkeys{/pgfplots/overlapdefaultlast/.style={
      overlapdefault,
      title=\empty,
    }}

  \begin{flushleft}
    \vspace{1ex}
    (a)
  \end{flushleft}
  \vspace{-2\baselineskip}

  \hfill
  \begin{minipage}[t]{0.495\linewidth}
    \centering
    % customize "zmystyle" as you wish
    \pgfkeys{/pgfplots/zmystyle/.style={overlapdefaultfirst,
        title = {Gradient-\ggn overlap},
      }}
    \input{../../fig/exp02_gradient_curvature_overlap/fig/overlap_000000/Overlap.tex}
  \end{minipage}
  \begin{minipage}[t]{0.495\linewidth}
    \centering
    % customize "zmystyle" as you wish
    \pgfkeys{/pgfplots/zmystyle/.style={overlapdefaultfirst,
        title = {Overlap histogram},
      }}
    \input{../../fig/exp02_gradient_curvature_overlap/fig/overlap_000000/OverlapHistogram.tex}
  \end{minipage}
  \hfill

  \vspace{-2.5\baselineskip}

  \begin{flushleft}
    \vspace{1ex}
    (b)
  \end{flushleft}
  \vspace{-2\baselineskip}

  \hfill
  \begin{minipage}[t]{0.495\linewidth}
    \centering
    % customize "zmystyle" as you wish
    \pgfkeys{/pgfplots/zmystyle/.style={overlapdefaultfirst,
      }}
    \input{../../fig/exp02_gradient_curvature_overlap/fig/overlap_000010/Overlap.tex}
  \end{minipage}
  \begin{minipage}[t]{0.495\linewidth}
    \centering
    % customize "zmystyle" as you wish
    \pgfkeys{/pgfplots/zmystyle/.style={overlapdefaultfirst,
      }}
    \input{../../fig/exp02_gradient_curvature_overlap/fig/overlap_000010/OverlapHistogram.tex}
  \end{minipage}
  \hfill

  \vspace{-2.5\baselineskip}

  \begin{flushleft}
    \vspace{1ex}
    (c)
  \end{flushleft}
  \vspace{-2\baselineskip}

  \hfill
  \begin{minipage}[t]{0.495\linewidth}
    \centering
    % customize "zmystyle" as you wish
    \pgfkeys{/pgfplots/zmystyle/.style={overlapdefaultlast,
      }}
    \input{../../fig/exp02_gradient_curvature_overlap/fig/overlap_000200/Overlap.tex}
  \end{minipage}
  \begin{minipage}[t]{0.495\linewidth}
    \centering
    % customize "zmystyle" as you wish
    \pgfkeys{/pgfplots/zmystyle/.style={overlapdefaultlast,
      }}
    \input{../../fig/exp02_gradient_curvature_overlap/fig/overlap_000200/OverlapHistogram.tex}
  \end{minipage}
  \hfill
  \vspace{-2\baselineskip}

  \caption{\textbf{Mini-batch gradient overlap along \ggn curvature directions:}
    Left column shows curvature-overlap pairs $(\lambda_i, p_i)$. Right column
    shows the gradient overlap with bins of EVecs. The net (a simple, but
    deep convolutional neural net on \mnist) is trained with \sgd ($N=128$).
    Rows show the distributions at \textbf{(a)} initialization,
    \textbf{(b)} early phase (step 10), and \textbf{(c)} end of training
    (step 200).}
  \label{fig:gradient-curvature-overlap}
\end{figure}

%%% Local Variables:
%%% mode: latex
%%% TeX-master: "../main"
%%% End: