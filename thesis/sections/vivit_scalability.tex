We now complement the theoretical computational complexity analysis from
\Cref{vivit::sec:method-complexity} with empirical studies. Results were generated on a
workstation with an Intel Core i7-8700K CPU (32\,GB) and one NVIDIA GeForce RTX
2080 Ti GPU (11\,GB). We use $M=1$ in the following.
% Furthermore, we set the number of \mc
% samples to $M=1$ in the following.


\begin{figure}[tb]
  \begin{subfigure}[t]{0.35\linewidth}
    \centering
    \caption{}
    \label{subfig:performance-cifar10-3c3d-cuda_main_1}

    \vspace{-\baselineskip}
    \begin{normalsize}
      $N_{\text{crit}}$ (eigenvalues)
    \end{normalsize}
    \vspace{0.15\baselineskip}

    \begin{normalsize}
      \begin{tabular}{lll}
    \toprule
    $_{\text{\tiny{\ggn}}}$$^{\text{\tiny{Data}}}$ & mb & sub \\
    \midrule
    exact & 677
              & 3184 \\
    mc   & 3060
              & 6029 \\
    \bottomrule
\end{tabular}
    \end{normalsize}

    \vspace{\baselineskip}

    \begin{normalsize}
      $N_{\text{crit}}$ (top eigenpair)
    \end{normalsize}
    \vspace{0.15\baselineskip}

    \begin{normalsize}
      \begin{tabular}{lll}
    \toprule
    $_{\text{\tiny{\ggn}}}$$^{\text{\tiny{Data}}}$ & mb & sub \\
    \midrule
    exact & 677
              & 3184 \\
    mc   & 3060
              & 6029 \\
    \bottomrule
\end{tabular}
    \end{normalsize}

    \vspace{3.6\baselineskip}

  \end{subfigure}
  \begin{subfigure}[t]{0.64\linewidth}
    \centering
    \caption{}
    \label{subfig:performance-cifar10-3c3d-cuda_main_2}

    \vspace{-1.5\baselineskip}

    % load "performancedefault" style
    % defines the pgfplots style "performancedefault"
\pgfkeys{/pgfplots/performancedefault/.style={
    width=1.03\linewidth,
    height=\goldenRatioInv*1.03\linewidth,
    every axis plot/.append style={line width = 1.5pt},
    every axis plot post/.append style={
      mark size=2, mark options={opacity=0.9, solid, line width = 1pt}
    },
    tick pos = left,
    xmajorticks = true,
    ymajorticks = true,
    ylabel near ticks,
    xlabel near ticks,
    xtick align = inside,
    ytick align = inside,
    legend cell align = left,
    legend columns = 3,
    % legend pos = north east,
    legend style = {
      fill opacity = 0.7,
      text opacity = 1,
      font = \footnotesize,
      at={(1, 1.025)},
      anchor=south east,
      column sep=0.25cm,
    },
    xticklabel style = {font = \footnotesize, inner xsep = 0ex},
    xlabel style = {font = \footnotesize},
    axis line style = {black},
    yticklabel style = {font = \footnotesize, inner ysep = 0ex},
    ylabel style = {font = \footnotesize, inner ysep = 0ex},
    title style = {font = \footnotesize, inner ysep = 0ex, yshift = -0.75ex},
    grid = major,
    grid style = {dashed},
    title = {},
  }
}

%%% Local Variables:
%%% mode: latex
%%% TeX-master: "../../thesis"
%%% End:

    % customize "zmystyle" as you wish
    \pgfkeys{/pgfplots/zmystyle/.style={performancedefault, height=0.5\linewidth}}
    % This file was created by tikzplotlib v0.9.7.
\begin{tikzpicture}

\definecolor{color0}{rgb}{0.937254901960784,0.231372549019608,0.172549019607843}
\definecolor{color1}{rgb}{0.274509803921569,0.6,0.564705882352941}
\definecolor{color2}{rgb}{0.870588235294118,0.623529411764706,0.0862745098039216}
\definecolor{color3}{rgb}{0.501960784313725,0.184313725490196,0.6}

\begin{axis}[
axis line style={white!80!black},
legend style={fill opacity=0.8, draw opacity=1, text opacity=1, at={(0.03,0.97)}, anchor=north west, draw=white!80!black},
tick pos=left,
title={cifar10\_3c3d, N=128, cuda, one\_group},
xlabel={top eigenpairs (\(\displaystyle k\))},
xmin=0.55, xmax=10.45,
ylabel={time [s]},
ymin=0.00910443848057122, ymax=1.89239674986061,
ymode=log,
zmystyle
]
\addplot [, color0, dashed, mark=pentagon*, mark size=3, mark options={solid}]
table {%
1 0.10646684000676
2 0.263332082999113
3 0.41296657500061
4 0.562371585998335
5 0.961040356996818
6 1.08828064100089
7 1.20301098700293
8 1.29626815899974
9 1.36217651501647
10 1.48477111599641
};
\addlegendentry{power iteration}
\addplot [, black, dashed, mark=*, mark size=3, mark options={solid}]
table {%
1 0.185383651005395
2 0.186310245997447
3 0.186981072998606
4 0.187526319001336
5 0.18765962299949
6 0.18803448399558
7 0.187778580999293
8 0.186061766995408
9 0.189352801004134
10 0.188033979997272
};
\addlegendentry{mb, exact}
\addplot [, color1, dashed, mark=diamond*, mark size=3, mark options={solid}]
table {%
1 0.0260587410011794
2 0.0255732080040616
3 0.0259397419940797
4 0.0259047899962752
5 0.0255934300002991
6 0.0258599759981735
7 0.0261660430041957
8 0.02605828599917
9 0.0260880819987506
10 0.0260066910050227
};
\addlegendentry{sub, exact}
\addplot [, color2, dashed, mark=square*, mark size=3, mark options={solid}]
table {%
1 0.0172249570023268
2 0.0170504369962146
3 0.017168151003716
4 0.0170781389970216
5 0.0171006899981876
6 0.0171182029953343
7 0.0170811919961125
8 0.0169726210006047
9 0.017144368001027
10 0.0171162970000296
};
\addlegendentry{mb, mc}
\addplot [, color3, dashed, mark=triangle*, mark size=3, mark options={solid,rotate=180}]
table {%
1 0.0127098220036714
2 0.0126403089961968
3 0.0126599400027771
4 0.0123849169976893
5 0.011915481001779
6 0.0116139689998818
7 0.0116039499989711
8 0.0117499150001095
9 0.0118982629937818
10 0.0121629770001164
};
\addlegendentry{sub, mc}
\end{axis}

\end{tikzpicture}

  \end{subfigure}

  \vspace{-7ex}
  \caption{\textbf{GPU memory and run time performance:}
  Performance measurements for the
  \threecthreed architecture ($D = 895,\!210$)
  on \cifarten ($C=10$).
  \textbf{(a)} Critical batch sizes $N_{\text{crit}}$
    for computing eigenvalues and the top eigenpair.
    \textbf{(b)} Run time comparison with a power iteration for extracting
    the $k$ leading eigenpairs using a batch of size $N=128$.
  }
  \label{fig:performance-cifar10-3c3d-cuda_main}
\end{figure}

%%% Local Variables:
%%% mode: latex
%%% TeX-master: "../main"
%%% End:



\subsubsection{Memory Performance}

% Describe computation
We consider two tasks:
\begin{enumerate}
\item \textbf{Computing eigenvalues:} The nontrivial eigenvalues
  $\{\lambda_{k}\,|\, (\lambda_{k}, \vetilde_{k}) \in \tilde{\sS}_+\}$ are
  obtained by forming and eigen-decomposing the Gram matrix $\mGtilde$, allowing
  stage-wise discarding of $\mV$ (see
  \Cref{vivit::sec:computing-full-ggn-eigenspectrum,vivit::sec:method-complexity}).
  \label{vivit::item:task-eigenvalues}

\item \textbf{Computing the top eigenpair:} For $(\lambda_{1}, \ve_{1})$, we
  compute the Gram matrix spectrum $\tilde{\sS}_{+}$, extract its top eigenpair
  $(\lambda_{1}, \vetilde_{1})$, and transform it into parameter space by
  \Cref{vivit::eq:ggn-eigenvectors}, \ie $(\lambda_{1}, \ve_{1} =
  \nicefrac{1}{\sqrt{\lambda_{1}}} \mV \vetilde_{1} )$. This requires more
  memory than task~\ref{vivit::item:task-eigenvalues} as $\mV$ must be stored.
  \label{vivit::item:task-eigenvectors}
\end{enumerate}
As a comprehensive memory performance measure, we use the largest batch size
before our system runs out of memory---we call this the \emph{critical batch size}
$N_{\text{crit}}$.

% Describe and explain results
\Cref{vivit::subfig:performance-cifar10-3c3d-cuda_main1} tabularizes the critical batch
sizes on GPU for the \threecthreed architecture on \cifarten. As expected,
computing eigenpairs requires more memory and leads to consistently smaller
critical batch sizes in comparison to computing only eigenvalues. Yet, they all
exceed the traditional batch size used for training ($N=128$, see
\cite{schneider2019deepobs}), even when using the exact \ggn. With \vivit{}'s
approximations, the memory overhead can be reduced to significantly increase the
applicable batch size.

We report similar results for more architectures, a block-diagonal approximation
(as in \citet{zhang2017blockdiagonal}), and on CPU in
\Cref{vivit::sec:performance-experiments}, where we also benchmark a third
task---computing damped Newton steps.

% Describe procedure
\subsubsection{Run Time Performance}

Next, we consider computing the $k$ leading eigenvectors and eigenvalues of a
matrix. A power iteration that computes eigenpairs iteratively via matrix-vector
products serves as a reference. For a fixed value of $k$, we repeat both
approaches $20$ times and report the shortest time.

% Describe computation
For the power iteration, we adapt the implementation from the \pyhessian library
\cite{yao2020pyhessian} and replace its Hessian-vector product by a matrix-free
\ggn-vector product \cite{schraudolph2002fast} through \pytorch's AD. We use the
same default hyperparameters for the termination criterion.
%
Similar to task~\ref{vivit::item:task-eigenvalues}, our method obtains the top-$k$
eigenpairs\sidenote{In contrast to the power iteration that is restricted to
  dominating eigenpairs, our approach allows choosing arbitrary eigenpairs.}
% $\{(\lambda_{1}, \ve_{1}), (\lambda_{2}, \ve_{2}), \ldots, (\lambda_{k},\ve_{k})\}$
by computing $\tilde{\sS}_{+}$, extracting its leading eigenpairs
% $\{(\lambda_{1}, \vetilde_{1}), (\lambda_{2},
% \vetilde_{2}), \ldots, (\lambda_{k}, \vetilde_{k})\}$,
and transforming the eigenvectors $\vetilde_{1}, \vetilde_{2}, \ldots,
\vetilde_{k}$ into parameter space by application of $\mV$ (see
\Cref{vivit::eq:ggn-eigenvectors}).

\begin{figure}
  \centering
  % defines the pgfplots style "eigspacedefault"
\pgfkeys{/pgfplots/eigspacedefault/.style={
    width=1.03\linewidth,
    height=\goldenRatioInv*1.03*\linewidth,
    every axis plot/.append style={line width = 1pt},
    tick pos = left,
    ylabel near ticks,
    xlabel near ticks,
    xtick align = inside,
    ytick align = inside,
    legend cell align = left,
    legend columns = 1,
    legend pos = north east,
    legend style = {
      fill opacity = 0.9,
      text opacity = 1,
      font = \tiny,
      % column sep=0.1cm,
    },
    legend image post style={scale=2},
    xticklabel style = {font = \small},
    xlabel style = {font = \small},
    axis line style = {black},
    yticklabel style = {font = \small},
    ylabel style = {font = \small},
    title style = {font = \small},
    grid = major,
    grid style = {dashed}
  }
}

\pgfkeys{/pgfplots/eigspacedefaultapp/.style={
    eigspacedefault,
    height=0.6\linewidth,
    legend columns = 2,
  }
}

\pgfkeys{/pgfplots/eigspacenolegend/.style={
    legend image post style = {scale=0},
    legend style = {
      fill opacity = 0,
      draw opacity = 0,
      text opacity = 0,
      font = \small,
      at={(1, 1.025)},
      anchor=south east,
      column sep=0.25cm,
    },
  }
}
%%% Local Variables:
%%% mode: latex
%%% TeX-master: "../main"
%%% End:

  \pgfkeys{/pgfplots/zmystyle/.style={
      eigspacedefault
    }}
  \tikzexternalenable
  % This file was created by tikzplotlib v0.9.7.
\begin{tikzpicture}

\definecolor{color0}{rgb}{0.145098039215686,0.490196078431373,0.349019607843137}

\begin{axis}[
axis line style={white!10!black},
log basis x={10},
tick pos=left,
xlabel={epoch (log scale)},
xmajorgrids,
xmin=0.794328234724281, xmax=125.892541179417,
xmode=log,
ylabel={overlap},
ymajorgrids,
ymin=0.842878600955009, ymax=1.00748197138309,
zmystyle
]
\addplot [, white!10!black, dashed]
table {%
0.794328234724281 1
125.892541179417 1
};
\addplot [, color0, mark=*, mark size=0.5, mark options={solid}, only marks]
table {%
1 0.850360572338104
1.04487179487179 0.878133177757263
1.09615384615385 0.91087943315506
1.1474358974359 0.980224907398224
1.20192307692308 0.993440926074982
1.25961538461538 0.988232254981995
1.32051282051282 0.987772941589355
1.38461538461538 0.996818244457245
1.44871794871795 0.992499232292175
1.51923076923077 0.993051826953888
1.58974358974359 0.988804459571838
1.66666666666667 0.993102371692657
1.74679487179487 0.994531810283661
1.83012820512821 0.990985095500946
1.91666666666667 0.989441215991974
2.00641025641026 0.996042609214783
2.1025641025641 0.990116953849792
2.20512820512821 0.996223270893097
2.30769230769231 0.996023297309875
2.41987179487179 0.992021441459656
2.53525641025641 0.994040310382843
2.65384615384615 0.990503132343292
2.78205128205128 0.994559407234192
2.91346153846154 0.994334101676941
3.05128205128205 0.996053695678711
3.19871794871795 0.994520008563995
3.34935897435897 0.997268378734589
3.50961538461538 0.997389912605286
3.67628205128205 0.992188334465027
3.8525641025641 0.993352770805359
4.03525641025641 0.995678246021271
4.2275641025641 0.992665469646454
4.42948717948718 0.99213582277298
4.64102564102564 0.997660756111145
4.86217948717949 0.967168152332306
5.09294871794872 0.993785679340363
5.33653846153846 0.993943214416504
5.58974358974359 0.993542373180389
5.85576923076923 0.991654217243195
6.13461538461539 0.969390213489532
6.42628205128205 0.99034708738327
6.73397435897436 0.990933060646057
7.05448717948718 0.99559211730957
7.38782051282051 0.995241284370422
7.74038461538461 0.990916728973389
8.10897435897436 0.990530490875244
8.49679487179487 0.994798362255096
8.90064102564103 0.997474670410156
9.32371794871795 0.995593726634979
9.76923076923077 0.993783950805664
10.2339743589744 0.992836356163025
10.7211538461538 0.983698666095734
11.2307692307692 0.99790632724762
11.7660256410256 0.995324611663818
12.3269230769231 0.997409164905548
12.9134615384615 0.989908039569855
13.5288461538462 0.99628621339798
14.1730769230769 0.9957315325737
14.849358974359 0.996454894542694
15.5544871794872 0.997333228588104
16.2948717948718 0.994928538799286
17.0705128205128 0.989782631397247
17.8846153846154 0.997422814369202
18.7371794871795 0.997152030467987
19.6282051282051 0.99539452791214
20.5641025641026 0.995177149772644
21.5416666666667 0.993471741676331
22.5673076923077 0.994047284126282
23.6442307692308 0.997011065483093
24.7692307692308 0.99792355298996
25.9487179487179 0.995034217834473
27.1826923076923 0.994511604309082
28.4775641025641 0.996814131736755
29.8333333333333 0.99793940782547
31.2564102564103 0.997101902961731
32.7435897435897 0.995478749275208
34.3044871794872 0.99685537815094
35.9358974358974 0.997273564338684
37.6474358974359 0.994191527366638
39.4391025641026 0.99822723865509
41.3173076923077 0.995255470275879
43.2852564102564 0.996559143066406
45.3461538461538 0.992474675178528
47.5064102564103 0.994227528572083
49.7692307692308 0.998042583465576
52.1378205128205 0.996337890625
54.6217948717949 0.995224297046661
57.2211538461538 0.998927772045135
59.9455128205128 0.99811202287674
62.8012820512821 0.997424483299255
65.7916666666667 0.998414218425751
68.9230769230769 0.9983189702034
72.2051282051282 0.9963658452034
75.6442307692308 0.994187355041504
79.2467948717949 0.998039245605469
83.0192307692308 0.99797123670578
86.974358974359 0.996883273124695
91.1153846153846 0.997422218322754
95.4519230769231 0.995517551898956
100 0.998668372631073
};
\end{axis}

\end{tikzpicture}

  \tikzexternaldisable

  \caption{ \textbf{Full-batch \ggn versus full-batch Hessian:} Overlap between the
    top-$C$ eigenspaces of the full-batch \ggn and full-batch Hessian during
    training of the \threecthreed network on \cifarten with \sgd{}. }
  \label{vivit::fig:approx_GGN_Hessian}
\end{figure}

% Describe and explain results
\Cref{vivit::subfig:performance-cifar10-3c3d-cuda_main2} shows the GPU run time
for the \threecthreed architecture on \cifarten, using a mini-batch of size
$N=128$. Without any approximations to the \ggn, our method already outperforms
the power iteration for $k>1$ and increases \textit{much} slower in run time as
more leading eigenpairs are requested. This means that, relative to the
transformation of each eigenvector from the Gram space into the parameter space
through $\mV$, the run time mainly results from computing $\mV,\mGtilde$, and
eigendecomposing the latter. This is consistent with the computational
complexity of those operations in $NC$ (compare
\Cref{vivit::sec:method-complexity}) and allows for efficient extraction of a
large number of eigenpairs. The run time curves of the approximations confirm
this behavior by featuring the same flat profile. Additionally, they require
significantly less time than the exact mini-batch computation. Results for more
network architectures, a block-diagonal approximation and on CPU are reported in
\Cref{vivit::sec:performance-experiments}.

%%% Local Variables:
%%% mode: latex
%%% TeX-master: "../thesis"
%%% End:
