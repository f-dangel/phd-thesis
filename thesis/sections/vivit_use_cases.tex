Aiming to provide a well-founded, theoretical and empirical evaluation, we have
consciously focused on studying the approximation quality of \vivit's
quantities, as well as on demonstrating the efficiency of their computation. We
believe it is interesting in itself that the low-rank structure provides access
to quantities that would otherwise be costly. Still, we want to briefly address
possible use cases---their full development and assessment, however, will amount
to separate paper(s):
\begin{itemize}
\item \textbf{Monitoring tool:} Our computationally efficient curvature model
  provides geometric \textit{and} stochastic information about the local loss
  landscape and can be used by tools like \cockpit \citep{schneider2021cockpit}
  to debug optimizers or to gain insights into the optimization problem itself
  (as in \Cref{vivit::subsec:approx_quality,vivit::subsec:directional_derivatives}).

\item \textbf{Second-order optimization:} The quantities provided by \vivit{},
  in particular the first- and second-order directional derivatives, can be used
  to build a stochastic quadratic model of the loss function and perform
  Newton-like parameter updates. In contrast to existing second-order methods,
  \textit{per-sample} quantities contain information about the reliability of
  that quadratic model. This offers a new dimension for improving second-order
  methods through statistics on the mini-batch \textit{distribution} of the
  directional derivatives (\eg for variance-adapted step sizes), potentially
  increasing the method's performance and stability.
\end{itemize}

%%% Local Variables:
%%% mode: latex
%%% TeX-master: "../thesis"
%%% End:
