\subsubsection{Hardware Details}

Results were generated on a workstation with an Intel Core i7-8700K CPU (32\,GB)
and one NVIDIA GeForce RTX 2080 Ti GPU (11\,GB).

\subsubsection{Note}

\vivit{}'s quantities are implemented through \backpack, which is triggered by
\pytorch's gradient computation. Consequently, they can only be computed
together with \pytorch{}'s mini-batch gradient.

\subsubsection{Architectures}

We use untrained deep convolutional and residual networks from \deepobs
\cite{schneider2019deepobs} and \cite{idelbayev2018proper}. If a net has batch
normalization layers, we set them to evaluation mode. Otherwise, the loss would
not obey the sum structure of \Cref{vivit::eq:objective-function}. The batch
normalization layers' internal moving averages, required for evaluation mode,
are initialized by performing five forward passes with the current mini-batch in
training mode before.

In experiments with fixed mini-batches the batch sizes correspond to \deepobs'
default value for training where possible (\cifarten: $N=128$, \fmnist:
$N=128$). The ResNets use a batch size of $N=128$. On \cifarhun
(trained with $N=256$), we reduce the batch size to $N=64$ to fit the exact
computation on the full mini-batch, used as baseline, into memory. If the \ggn
approximation is evaluated on a subset of the mini-batch (\textbf{sub}),
$\floor{\nicefrac{N}{8}}$ of the samples are used (as in
\cite{zhang2017blockdiagonal}). The \mc approximation is always evaluated with a
single sample ($M=1$).

\subsubsection{Memory Performance (Critical Batch Sizes)}

Two tasks are considered (see \Cref{vivit::subsec:scalability}):
\begin{enumerate}
\item \textbf{Computing eigenvalues:} Compute the nontrivial eigenvalues
  $\{\lambda_{k}\,|\, (\lambda_{k}, \vetilde_{k}) \in \tilde{\sS}_+\}$ .
\item \textbf{Computing the top eigenpair:} Compute the top eigenpair
  $(\lambda_{1}, \ve_{1})$.
\end{enumerate}

We repeat the tasks above and vary the mini-batch size until the device runs out
of memory. The largest mini-batch size that can be handled by our system is
denoted as $N_{\text{crit}}$, the critical batch size. We determine this number
by bisection on the interval $[1; 32768]$.

Subfigures (a) and (b) of
\Cref{vivit::fig:performance-cifar10-3c3d-cpu,vivit::fig:performance-cifar10-3c3d-cuda,vivit::fig:performance-fmnist-2c2d-cpu,vivit::fig:performance-fmnist-2c2d-cuda,vivit::fig:performance-cifar100-allcnnc-cpu,vivit::fig:performance-cifar100-allcnnc-cuda,vivit::fig:performance-cifar10-resnet32-cpu,vivit::fig:performance-cifar10-resnet32-cuda,vivit::fig:performance-cifar10-resnet56-cpu,vivit::fig:performance-cifar10-resnet56-cuda}
present the results. As described in \Cref{vivit::sec:method-complexity},
computing eigenvalues is more memory-efficient than computing eigenvectors and
exhibits larger critical batch sizes. In line with the description in
\Cref{vivit::sec:approximations}, a block-diagonal approximation is usually more
memory-efficient and results in a larger critical batch size. Curvature
sub-sampling and \mc approximation further increase the applicable batch sizes.

In summary, there always exists a combination of approximations which allows for
critical batch sizes larger than the traditional size used for training (some
architectures even permit exact computation). Different accuracy-cost trade-offs
may be preferred, depending on the application and the computational budget. By
the presented approximations, \vivit's representation is capable to adapt over a
wide range.

\subsubsection{Run Time Performance}

Here, we consider the task of computing the $k$ leading eigenvectors and
eigenvalues of a matrix. \vivit{}'s eigenpair computation is compared with a
power iteration that computes eigenpairs iteratively via matrix-vector products.
The power iteration baseline is based on the \pyhessian library
\cite{yao2020pyhessian} and uses the same termination criterion (at most 100
matrix-vector products per eigenvalue; stop if the eigenvalue estimate's
relative change is less than $10^{-3}$). In contrast to \pyhessian, we use a
different data format and stack the computed eigenvectors. This reduces the
number of \texttt{for}-loops in the orthonormalization step. We repeat each run
time measurement $20$ times and report the shortest execution time as result.

Subfigures (c) and (d) of
\Cref{vivit::fig:performance-cifar10-3c3d-cpu,vivit::fig:performance-cifar10-3c3d-cuda,vivit::fig:performance-fmnist-2c2d-cpu,vivit::fig:performance-fmnist-2c2d-cuda,vivit::fig:performance-cifar100-allcnnc-cpu,vivit::fig:performance-cifar100-allcnnc-cuda,vivit::fig:performance-cifar10-resnet32-cpu,vivit::fig:performance-cifar10-resnet32-cuda,vivit::fig:performance-cifar10-resnet56-cpu,vivit::fig:performance-cifar10-resnet56-cuda}
show the results. For most architectures, our exact method outperforms the power
iteration for $k>1$ and increases only marginally in run time as the number of
requested eigenvectors grows. The proposed approximations share this property,
and further reduce run time.

\subsubsection{Note On \cifarhun (Large $C$)}

For datasets with a large number of classes, like \cifarhun ($C=100$),
computations with the exact \ggn are costly. In particular, constructing the
Gram matrix $\mGtilde$ has quadratic memory cost in $C$, and its
eigendecomposition has cubic cost in time with $C$ (see
\Cref{vivit::sec:method-complexity}).

As a result, the exact computation only works with batch sizes smaller than
\deepobs' default ($N=256$ for \cifarhun, see subfigures (a) and (b) of
\Cref{vivit::fig:performance-cifar100-allcnnc-cpu,vivit::fig:performance-cifar100-allcnnc-cuda}).
For the \ggn block-diagonal approximation, which fits into CPU memory for
$N=64$, the exact computation of top eigenpairs is slower than a power iteration
and only becomes comparable if a large number of eigenpairs is requested, see
\Cref{vivit::subfig:performance-cifar100-allcnnc-cpu-4}.

For such datasets, the approximations proposed in \Cref{vivit::sec:approximations} are
essential to reduce costs. The most effective approximation to eliminate the
scaling with $C$ is using an \mc approximation.
\Cref{vivit::fig:performance-cifar100-allcnnc-cpu,vivit::fig:performance-cifar100-allcnnc-cuda}
confirm that the approximate computations scale to batch sizes used for training
and that computing eigenpairs takes less time than a power iteration.

% expects (dataset, model, device, tables(0 or 1), caption)
\newcommand{\plotViViTPerformance}[5]{
\begin{figure*}[tb]
  \centering
  \begin{minipage}[t]{0.49\linewidth}
    \centering
    \begin{footnotesize}
      \textbf{Full network}
    \end{footnotesize}
  \end{minipage}
  \hfill
  \begin{minipage}[t]{0.49\linewidth}
    \centering
    \begin{footnotesize}
      \textbf{Block-diagonal approximation}
    \end{footnotesize}
  \end{minipage}

  \vspace{-2ex}
\ifthenelse{\equal{#4}{1}}{
  \begin{subfigure}[t]{0.49\linewidth}
    \centering
    \caption{Memory performance}\label{vivit::subfig:performance-#1-#2-#3-1}
    \begin{minipage}[t]{0.49\linewidth}
      \centering
      \begin{footnotesize}
        $N_{\text{crit}}$ (eigenvalues)

        \vspace{0.15\baselineskip}
        \input{../repos/vivit-paper/fig/exp10_performance_ggn_evals/fig/N_crit/evals/tab_#1_#2_#3_one_group}
      \end{footnotesize}
    \end{minipage}
    \hfill
    \begin{minipage}[t]{0.49\linewidth}
      \centering
      \begin{footnotesize}
        $N_{\text{crit}}$ (top eigenpair)

        \vspace{0.15\baselineskip}
        \input{../repos/vivit-paper/fig/exp10_performance_ggn_evals/fig/N_crit/evecs/tab_#1_#2_#3_one_group}
      \end{footnotesize}
    \end{minipage}
  \end{subfigure}
  \hfill
  \begin{subfigure}[t]{0.49\linewidth}
    \centering
    \caption{Memory performance}\label{vivit::subfig:performance-#1-#2-#3-2}
    \begin{minipage}[t]{0.49\linewidth}
      \centering
      \begin{footnotesize}
        $N_{\text{crit}}$ (eigenvalues)

        \vspace{0.15\baselineskip}
        \input{../repos/vivit-paper/fig/exp10_performance_ggn_evals/fig/N_crit/evals/tab_#1_#2_#3_layerwise_group}
      \end{footnotesize}
    \end{minipage}
    \hfill
    \begin{minipage}[t]{0.49\linewidth}
      \centering
      \begin{footnotesize}
        $N_{\text{crit}}$ (top eigenpair)

        \vspace{0.15\baselineskip}
        \input{../repos/vivit-paper/fig/exp10_performance_ggn_evals/fig/N_crit/evecs/tab_#1_#2_#3_layerwise_group}
      \end{footnotesize}
    \end{minipage}
  \end{subfigure}
}{}

  \begin{subfigure}[t]{0.49\linewidth}
    \centering
    \ifthenelse{\equal{#4}{1}}{}{
      \addtocounter{subfigure}{2}
    }
    \caption{Run time performance}\label{vivit::subfig:performance-#1-#2-#3-3}
    % load "performancedefault" style
    % defines the pgfplots style "performancedefault"
\pgfkeys{/pgfplots/performancedefault/.style={
    width=1.04\linewidth,
    height=\goldenRatioInv*1.04\linewidth,
    every axis plot/.append style={line width = 1.2pt},
    every axis plot post/.append style={
      mark size=2, mark options={opacity=0.9, solid, line width = 1pt}
    },
    tick pos = left,
    xmajorticks = true,
    ymajorticks = true,
    ylabel near ticks,
    xlabel near ticks,
    xtick align = inside,
    ytick align = inside,
    legend cell align = left,
    legend columns = 3,
    % legend pos = north east,
    legend style = {
      fill opacity = 0.9,
      text opacity = 1,
      font = \small,
      at={(1, 1.025)},
      anchor=south east,
    },
    xticklabel style = {font = \small, inner xsep = 0ex},
    xlabel style = {font = \small},
    axis line style = {black},
    yticklabel style = {font = \small, inner ysep = 0ex},
    ylabel style = {font = \small, inner ysep = 0ex},
    title style = {font = \small, inner ysep = 0ex, yshift = -0.75ex},
    grid = major,
    grid style = {dashed},
    title = {},
  }
}
%%% Local Variables:
%%% mode: latex
%%% TeX-master: "../main"
%%% End:

    % customize "zmystyle" as you wish
    \pgfkeys{/pgfplots/zmystyle/.style={performancedefault, legend columns = 2}}
    \input{../repos/vivit-paper/fig/exp10_performance_ggn_evals/fig/time/evecs/#1_#2_#3_one_group}
  \end{subfigure}
  \hfill
  \begin{subfigure}[t]{0.49\linewidth}
    \centering
    \caption{Run time performance}\label{vivit::subfig:performance-#1-#2-#3-4}
    % load "performancedefault" style
    % defines the pgfplots style "performancedefault"
\pgfkeys{/pgfplots/performancedefault/.style={
    width=1.04\linewidth,
    height=\goldenRatioInv*1.04\linewidth,
    every axis plot/.append style={line width = 1.2pt},
    every axis plot post/.append style={
      mark size=2, mark options={opacity=0.9, solid, line width = 1pt}
    },
    tick pos = left,
    xmajorticks = true,
    ymajorticks = true,
    ylabel near ticks,
    xlabel near ticks,
    xtick align = inside,
    ytick align = inside,
    legend cell align = left,
    legend columns = 3,
    % legend pos = north east,
    legend style = {
      fill opacity = 0.9,
      text opacity = 1,
      font = \small,
      at={(1, 1.025)},
      anchor=south east,
    },
    xticklabel style = {font = \small, inner xsep = 0ex},
    xlabel style = {font = \small},
    axis line style = {black},
    yticklabel style = {font = \small, inner ysep = 0ex},
    ylabel style = {font = \small, inner ysep = 0ex},
    title style = {font = \small, inner ysep = 0ex, yshift = -0.75ex},
    grid = major,
    grid style = {dashed},
    title = {},
  }
}
%%% Local Variables:
%%% mode: latex
%%% TeX-master: "../main"
%%% End:

    % customize "zmystyle" as you wish
    \pgfkeys{/pgfplots/zmystyle/.style={performancedefault, legend columns = 2}}
    \input{../repos/vivit-paper/fig/exp10_performance_ggn_evals/fig/time/evecs/#1_#2_#3_layerwise_group}
  \end{subfigure}
  \caption{#5}\label{vivit::fig:performance-#1-#2-#3}
\end{figure*}
}

%%% Local Variables:
%%% mode: latex
%%% TeX-master: "../../thesis"
%%% End:


% ---------------------------------
% FMNIST 2C2D
% ---------------------------------

\tikzexternalenable
\plotViViTPerformance{fmnist}{2c2d}{cuda}{1}{ \textbf{GPU memory and run time
    performance for the \twoctwod architecture on \fmnist.} Left and right
  columns show results with the full network's \ggn ($D = 3,\!274,\!634$,
  $C=10$) and a per-layer block-diagonal approximation, respectively.
  \subfigref{vivit::subfig:performance-fmnist-2c2d-cuda-1},\subfigref{vivit::subfig:performance-fmnist-2c2d-cuda-2}
  Critical batch sizes $N_{\text{crit}}$ for computing eigenvalues and the top
  eigenpair.
  \subfigref{vivit::subfig:performance-fmnist-2c2d-cuda-3},\subfigref{vivit::subfig:performance-fmnist-2c2d-cuda-4}
  Run time comparison with a power iteration for extracting the $k$ leading
  eigenpairs using a mini-batch of size $N=128$.}
\tikzexternaldisable

\tikzexternalenable
\plotViViTPerformance{fmnist}{2c2d}{cpu}{1}{ \textbf{CPU memory and run time
    performance for the \twoctwod architecture on \fmnist.} Left and right
  columns show results with the full network's \ggn ($D = 3,\!274,\!634$,
  $C=10$) and a per-layer block-diagonal approximation, respectively.
  \subfigref{vivit::subfig:performance-fmnist-2c2d-cpu-1},\subfigref{vivit::subfig:performance-fmnist-2c2d-cpu-2}
  Critical batch sizes $N_{\text{crit}}$ for computing eigenvalues and the top
  eigenpair.
  \subfigref{vivit::subfig:performance-fmnist-2c2d-cpu-3},\subfigref{vivit::subfig:performance-fmnist-2c2d-cpu-4}
  Run time comparison with a power iteration for extracting the $k$ leading
  eigenpairs using a mini-batch of size $N=128$.}
\tikzexternaldisable

% ---------------------------------
% CIFAR10 3C3D
% ---------------------------------

\tikzexternalenable
\plotViViTPerformance{cifar10}{3c3d}{cuda}{1}{ \textbf{GPU
    memory and run time performance for the \threecthreed architecture on
    \cifarten.} Left and right columns show results with the full network's \ggn
  ($D = 895,\!210$, $C=10$) and a per-layer block-diagonal approximation,
  respectively.
  \subfigref{vivit::subfig:performance-cifar10-3c3d-cuda-1},\subfigref{vivit::subfig:performance-cifar10-3c3d-cuda-2}
  Critical batch sizes $N_{\text{crit}}$ for computing eigenvalues and the top
  eigenpair.
  \subfigref{vivit::subfig:performance-cifar10-3c3d-cuda-3},\subfigref{vivit::subfig:performance-cifar10-3c3d-cuda-4}
  Run time comparison with a power iteration for extracting the $k$ leading
  eigenpairs using a mini-batch of size $N=128$.}
\tikzexternaldisable

\tikzexternalenable
\plotViViTPerformance{cifar10}{3c3d}{cpu}{1}{ \textbf{CPU
    memory and run time performance for the \threecthreed architecture on
    \cifarten.} Left and right columns show results with the full network's \ggn
  ($D = 895,\!210$, $C=10$) and a per-layer block-diagonal approximation,
  respectively.
  \subfigref{vivit::subfig:performance-cifar10-3c3d-cpu-1},\subfigref{vivit::subfig:performance-cifar10-3c3d-cpu-2}
  Critical batch sizes $N_{\text{crit}}$ for computing eigenvalues and the top
  eigenpair.
  \subfigref{vivit::subfig:performance-cifar10-3c3d-cpu-3},\subfigref{vivit::subfig:performance-cifar10-3c3d-cpu-4}
  Run time comparison with a power iteration for extracting the $k$ leading
  eigenpairs using a mini-batch of size $N=128$.}
\tikzexternaldisable

% ---------------------------------
% CIFAR10 RESNET32
% ---------------------------------

\tikzexternalenable
\plotViViTPerformance{cifar10}{resnet32}{cuda}{1}{ \textbf{GPU
    memory and run time performance for the \resnetthirtytwo architecture on \cifarten.}
  Left and right columns show results with the full network's \ggn ($D =
  464,\!154$, $C=10$) and a per-layer block-diagonal approximation,
  respectively.
  \subfigref{vivit::subfig:performance-cifar10-resnet32-cuda-1},\subfigref{vivit::subfig:performance-cifar10-resnet32-cuda-2}
  Critical batch sizes $N_{\text{crit}}$ for computing eigenvalues and the top
  eigenpair.
  \subfigref{vivit::subfig:performance-cifar10-resnet32-cuda-3},\subfigref{vivit::subfig:performance-cifar10-resnet32-cuda-4}
  Run time comparison with a power iteration for extracting the $k$ leading
  eigenpairs using a mini-batch of size $N=128$. }
\tikzexternaldisable

\tikzexternalenable
\plotViViTPerformance{cifar10}{resnet32}{cpu}{0}{ \textbf{CPU
    memory and run time performance for the \resnetthirtytwo architecture on \cifarten.}
  Left and right columns show results with the full network's \ggn ($D =
  464,\!154$, $C=10$) and a per-layer block-diagonal approximation,
  respectively.
  % \subfigref{vivit::subfig:performance-cifar10-resnet32-cpu-1},\subfigref{vivit::subfig:performance-cifar10-resnet32-cpu-2}
  % Critical batch sizes $N_{\text{crit}}$ for computing eigenvalues and the top
  % eigenpair.
  \subfigref{vivit::subfig:performance-cifar10-resnet32-cpu-3},\subfigref{vivit::subfig:performance-cifar10-resnet32-cpu-4}
  Run time comparison with a power iteration for extracting the $k$ leading
  eigenpairs using a mini-batch of size $N=128$. }
\tikzexternaldisable

% ---------------------------------
% CIFAR10 RESNET56
% ---------------------------------

\tikzexternalenable
\plotViViTPerformance{cifar10}{resnet56}{cuda}{1}{\textbf{GPU memory and run
    time performance for the \resnetfiftysix architecture on \cifarten.} Left and right
  columns show results with the full network's \ggn ($D = 853,\!018$, $C=10$)
  and a per-layer block-diagonal approximation, respectively.
  % \subfigref{vivit::subfig:performance-cifar10-resnet56-cuda-1},\subfigref{vivit::subfig:performance-cifar10-resnet56-cuda-2}
  % Critical batch sizes $N_{\text{crit}}$ for computing eigenvalues and the top
  % eigenpair.
  \subfigref{vivit::subfig:performance-cifar10-resnet56-cuda-3},\subfigref{vivit::subfig:performance-cifar10-resnet56-cuda-4}
  Run time comparison with a power iteration for extracting the $k$ leading
  eigenpairs using a mini-batch of size $N=128$.}
\tikzexternaldisable

\tikzexternalenable
\plotViViTPerformance{cifar10}{resnet56}{cpu}{0}{\textbf{CPU memory and run
    time performance for the \resnetfiftysix architecture on \cifarten.} Left and right
  columns show results with the full network's \ggn ($D = 853,\!018$, $C=10$)
  and a per-layer block-diagonal approximation, respectively. (a, b)
  % \subfigref{vivit::subfig:performance-cifar10-resnet56-cpu-1},\subfigref{vivit::subfig:performance-cifar10-resnet56-cpu-2}
  % Critical batch sizes $N_{\text{crit}}$ for computing eigenvalues and the top
  % eigenpair.
  \subfigref{vivit::subfig:performance-cifar10-resnet56-cpu-3},\subfigref{vivit::subfig:performance-cifar10-resnet56-cpu-4}
  Run time comparison with a power iteration for extracting the $k$ leading
  eigenpairs using a mini-batch of size $N=128$.}
\tikzexternaldisable

% ---------------------------------
% CIFAR100 ALLCNNC
% ---------------------------------

\tikzexternalenable
\plotViViTPerformance{cifar100}{allcnnc}{cuda}{1}{
  \textbf{GPU memory and run time performance for the \allcnnc
    architecture on \cifarhun.} Left and right columns show results with the
  full network's \ggn ($D = 1,\!387,\!108, C=100$) and a per-layer
  block-diagonal approximation, respectively.
  \subfigref{vivit::subfig:performance-cifar100-allcnnc-cuda-1},\subfigref{vivit::subfig:performance-cifar100-allcnnc-cuda-2}
  Critical batch sizes
  $N_{\text{crit}}$ for computing eigenvalues and the top eigenpair.
  \subfigref{vivit::subfig:performance-cifar100-allcnnc-cuda-3},\subfigref{vivit::subfig:performance-cifar100-allcnnc-cuda-4}
  Run time comparison with a power iteration for extracting the $k$ leading
  eigenpairs using a mini-batch of size $N=64$.
}
\tikzexternaldisable

\tikzexternalenable
\plotViViTPerformance{cifar100}{allcnnc}{cpu}{1}{
  \textbf{CPU memory and run time performance for the \allcnnc
    architecture on \cifarhun.} Left and right columns show results with the
  full network's \ggn ($D = 1,\!387,\!108, C=100$) and a per-layer
  block-diagonal approximation, respectively.
  \subfigref{vivit::subfig:performance-cifar100-allcnnc-cpu-1},\subfigref{vivit::subfig:performance-cifar100-allcnnc-cpu-2}
  Critical batch sizes
  $N_{\text{crit}}$ for computing eigenvalues and the top eigenpair.
  \subfigref{vivit::subfig:performance-cifar100-allcnnc-cpu-3},\subfigref{vivit::subfig:performance-cifar100-allcnnc-cpu-4}
  Run time comparison with a power iteration for extracting the $k$ leading
  eigenpairs using a mini-batch of size $N=64$.
}
\tikzexternaldisable

%%% Local Variables:
%%% mode: latex
%%% TeX-master: "../../thesis"
%%% End:


\subsubsection{Computing Damped Newton Steps}

A Newton step $-(\mG + \delta \mI)^{-1} \vg$ with damping $\delta > 0$ can be
decomposed into updates along the eigenvectors of the \ggn $\mG$,
\begin{equation}
  \label{vivit::eq:newton-step}
  -(\mG + \delta \mI)^{-1} \vg
=
   \sum_{k=1}^{K} \frac{- \gamma_{k}}{\lambda_{k} + \delta} \ve_{k}
  + \sum_{k = K + 1}^{D} \frac{-\gamma_{k}}{\delta} \ve_{k}\,.
\end{equation}
It corresponds to a Newton update along nontrivial eigendirections that uses the
first- and second-order directional derivatives described in
\Cref{vivit::sec:comp-direct-deriv} and a gradient descent step with learning rate
$\nicefrac{1}{\delta}$ along trivial directions (with $\lambda_k = 0$). In the
following, we refer to the first summand of \Cref{vivit::eq:newton-step} as Newton
step. As described in \Cref{vivit::sec:method-complexity}, we can perform the weighted
sum in the Gram matrix space, rather than the parameter space, by computing
\begin{equation*}
  \sum_{k=1}^{K} \frac{- \gamma_{k}}{\lambda_{k} + \delta} \ve_{k}
  =
  \sum_{k=1}^{K} \frac{- \gamma_{k}}{\lambda_{k} + \delta} \frac{1}{\sqrt{\lambda_{k}}} \mV \vetilde_{k}
  =\mV \left(
    \sum_{k=1}^{K} \frac{- \gamma_{k}}{(\lambda_{k} + \delta)\sqrt{\lambda_{k}}} \vetilde_{k}
  \right)\,.
\end{equation*}
This way, only a single vector needs to be transformed from Gram space into
parameter space.

\Cref{vivit::tab:performance} shows critical batch sizes for the Newton step
computation (first term on the right side of \Cref{vivit::eq:newton-step}),
using Gram matrix eigenvalues larger than $10^{-4}$ and constant damping
$\delta=1$. Second-order directional derivatives $\lambda_{k}$ are evaluated on
the same samples as the \ggn eigenvectors, but we \emph{always} use all
mini-batch samples to compute the directional gradients $\gamma_{n}$. Using our
approximations, the Newton step computation scales to batch sizes beyond the
traditional sizes used for training.

% https://texblog.org/2007/08/01/placing-figurestables-side-by-side-minipage/
\begin{table}[tb]
  \centering
  \caption{\textbf{Memory performance for computing damped Newton steps:} Left
    and right columns show the critical batch sizes with the full network's \ggn and
    a per-layer block-diagonal approximation, respectively.}
  \label{tab:performance}
  
%==========================================================  
  
  \begin{small}
    \textbf{\fmnist \twoctwod}
  \end{small}

  \begin{minipage}{0.49\linewidth}
    \centering
    \begin{small}
      \textbf{Full network}
    \end{small}
  \end{minipage}
  \hfill
  \begin{minipage}{0.49\linewidth}
    \centering
    \begin{small}
      \textbf{Block-diagonal approximation}
    \end{small}
  \end{minipage}
  \vspace{1ex}

  \begin{minipage}{0.245\linewidth}
    \centering
    \begin{small}
      $N_{\text{crit}}$ (GPU)
    \end{small}
    \vspace{0.15\baselineskip}

    \begin{small}
      \begin{tabular}{lll}
    \toprule
    $_{\text{\tiny{\ggn}}}$$^{\text{\tiny{Data}}}$ & mb & sub \\
    \midrule
    exact & 872
              & 6439 \\
    mc   & 6708
              & 12162 \\
    \bottomrule
\end{tabular}
    \end{small}
  \end{minipage}
  \hfill
  \begin{minipage}{0.245\linewidth}
    \centering
    \begin{small}
      $N_{\text{crit}}$ (CPU)
    \end{small}
    \vspace{0.15\baselineskip}

    \begin{small}
      \begin{tabular}{lll}
    \toprule
    $_{\text{\tiny{\ggn}}}$$^{\text{\tiny{Data}}}$ & mb & sub \\
    \midrule
    exact & 3276
              & 21295 \\
    mc   & 24064
              & > 32768 \\
    \bottomrule
\end{tabular}
    \end{small}
  \end{minipage}
  \hfill
  \begin{minipage}{0.245\linewidth}
    \centering
    \begin{small}
      $N_{\text{crit}}$ (GPU)
    \end{small}
    \vspace{0.15\baselineskip}

    \begin{small}
      \begin{tabular}{lll}
    \toprule
    $_{\text{\tiny{\ggn}}}$$^{\text{\tiny{Data}}}$ & mb & sub \\
    \midrule
    exact & 68
              & 159 \\
    mc   & 368
              & 528 \\
    \bottomrule
\end{tabular}
    \end{small}
  \end{minipage}
  \hfill
  \begin{minipage}{0.245\linewidth}
    \centering
    \begin{small}
      $N_{\text{crit}}$ (CPU)
    \end{small}
    \vspace{0.15\baselineskip}

    \begin{small}
      \begin{tabular}{lll}
    \toprule
    $_{\text{\tiny{\ggn}}}$$^{\text{\tiny{Data}}}$ & mb & sub \\
    \midrule
    exact & 3276
              & 20378 \\
    mc   & 23457
              & > 32768 \\
    \bottomrule
\end{tabular}
    \end{small}
  \end{minipage}

  \vspace{5ex}


%==========================================================


\begin{small}
    \textbf{\cifarten \threecthreed}
  \end{small}

  \begin{minipage}{0.49\linewidth}
    \centering
    \begin{small}
      \textbf{Full network}
    \end{small}
  \end{minipage}
  \hfill
  \begin{minipage}{0.49\linewidth}
    \centering
    \begin{small}
      \textbf{Block-diagonal approximation}
    \end{small}
  \end{minipage}
  \vspace{1ex}

  \begin{minipage}{0.245\linewidth}
    \centering
    \begin{small}
      $N_{\text{crit}}$ (GPU)
    \end{small}
    \vspace{0.15\baselineskip}

    \begin{small}
      \begin{tabular}{lll}
    \toprule
    $_{\text{\tiny{\ggn}}}$$^{\text{\tiny{Data}}}$ & mb & sub \\
    \midrule
    exact & 208
              & 727 \\
    mc   & 1055
              & 1816 \\
    \bottomrule
\end{tabular}
    \end{small}
  \end{minipage}
  \hfill
  \begin{minipage}{0.245\linewidth}
    \centering
    \begin{small}
      $N_{\text{crit}}$ (CPU)
    \end{small}
    \vspace{0.15\baselineskip}

    \begin{small}
      \begin{tabular}{lll}
    \toprule
    $_{\text{\tiny{\ggn}}}$$^{\text{\tiny{Data}}}$ & mb & sub \\
    \midrule
    exact & 2688
              & 12768 \\
    mc   & 14330
              & 20828 \\
    \bottomrule
\end{tabular}
    \end{small}
  \end{minipage}
  \hfill
  \begin{minipage}{0.245\linewidth}
    \centering
    \begin{small}
      $N_{\text{crit}}$ (GPU)
    \end{small}
    \vspace{0.15\baselineskip}

    \begin{small}
      \begin{tabular}{lll}
    \toprule
    $_{\text{\tiny{\ggn}}}$$^{\text{\tiny{Data}}}$ & mb & sub \\
    \midrule
    exact & 1085
              & 4415 \\
    mc   & 3854
              & 6624 \\
    \bottomrule
\end{tabular}
    \end{small}
  \end{minipage}
  \hfill
  \begin{minipage}{0.245\linewidth}
    \centering
    \begin{small}
      $N_{\text{crit}}$ (CPU)
    \end{small}
    \vspace{0.15\baselineskip}

    \begin{small}
      \begin{tabular}{lll}
    \toprule
    $_{\text{\tiny{\ggn}}}$$^{\text{\tiny{Data}}}$ & mb & sub \\
    \midrule
    exact & 2978
              & 13312 \\
    mc   & 14848
              & 20732 \\
    \bottomrule
\end{tabular}
    \end{small}
  \end{minipage}

  \vspace{5ex}


%==========================================================


  \begin{small}
    \textbf{\cifarten \resnetthirtytwo}
  \end{small}

  \begin{minipage}{0.49\linewidth}
    \centering
    \begin{small}
      \textbf{Full network}
    \end{small}
  \end{minipage}
  \hfill
  \begin{minipage}{0.49\linewidth}
    \centering
    \begin{small}
      \textbf{Block-diagonal approximation}
    \end{small}
  \end{minipage}
  \vspace{1ex}

  \begin{minipage}{0.245\linewidth}
    \centering
    \begin{small}
      $N_{\text{crit}}$ (GPU)
    \end{small}
    \vspace{0.15\baselineskip}

    \begin{small}
      \begin{tabular}{lll}
    \toprule
    $_{\text{\tiny{\ggn}}}$$^{\text{\tiny{Data}}}$ & mb & sub \\
    \midrule
    exact & 364
              & 1360 \\
    mc   & 1536
              & 2176 \\
    \bottomrule
\end{tabular}
    \end{small}
  \end{minipage}
  \hfill
  \begin{minipage}{0.245\linewidth}
  \textcolor{white}{-}
    % \centering
    % \begin{small}
    %   $N_{\text{crit}}$ (CPU)
    % \end{small}
    % \vspace{0.15\baselineskip}

    % \begin{small}
    %   % \input{../../fig/exp10_performance_ggn_evals/fig/N_crit/newton/tab_cifar10_resnet32_cpu_one_group.tex}
    % \end{small}
  \end{minipage}
  \hfill
  \begin{minipage}{0.245\linewidth}
    \centering
    \begin{small}
      $N_{\text{crit}}$ (GPU)
    \end{small}
    \vspace{0.15\baselineskip}

    \begin{small}
      \begin{tabular}{lll}
    \toprule
    $_{\text{\tiny{\ggn}}}$$^{\text{\tiny{Data}}}$ & mb & sub \\
    \midrule
    exact & 1051
              & 1851 \\
    mc   & 2048
              & 2208 \\
    \bottomrule
\end{tabular}
    \end{small}
  \end{minipage}
  \hfill
  \begin{minipage}{0.245\linewidth}
  \textcolor{white}{.}
    % \centering
    % \begin{small}
    %   $N_{\text{crit}}$ (CPU)
    % \end{small}
    % \vspace{0.15\baselineskip}

    % \begin{small}
    %   % \input{../../fig/exp10_performance_ggn_evals/fig/N_crit/newton/tab_cifar10_resnet32_cpu_layerwise_group.tex}
    % \end{small}
  \end{minipage}
  

  \vspace{5ex}


%==========================================================


  \begin{small}
    \textbf{\cifarten \resnetfiftysix}
  \end{small}

  \begin{minipage}{0.49\linewidth}
    \centering
    \begin{small}
      \textbf{Full network}
    \end{small}
  \end{minipage}
  \hfill
  \begin{minipage}{0.49\linewidth}
    \centering
    \begin{small}
      \textbf{Block-diagonal approximation}
    \end{small}
  \end{minipage}
  \vspace{1ex}

  \begin{minipage}{0.245\linewidth}
    \centering
    \begin{small}
      $N_{\text{crit}}$ (GPU)
    \end{small}
    \vspace{0.15\baselineskip}

    \begin{small}
      \begin{tabular}{lll}
    \toprule
    $_{\text{\tiny{\ggn}}}$$^{\text{\tiny{Data}}}$ & mb & sub \\
    \midrule
    exact & 217
              & 765 \\
    mc   & 896
              & 1240 \\
    \bottomrule
\end{tabular}
    \end{small}
  \end{minipage}
  \hfill
  \begin{minipage}{0.245\linewidth}
  \textcolor{white}{.}
    % \centering
    % \begin{small}
    %   $N_{\text{crit}}$ (CPU)
    % \end{small}
    % \vspace{0.15\baselineskip}

    % \begin{small}
    %   % \input{../../fig/exp10_performance_ggn_evals/fig/N_crit/newton/tab_cifar10_resnet56_cpu_one_group.tex}
    % \end{small}
  \end{minipage}
  \hfill
  \begin{minipage}{0.245\linewidth}
    \centering
    \begin{small}
      $N_{\text{crit}}$ (GPU)
    \end{small}
    \vspace{0.15\baselineskip}

    \begin{small}
      \begin{tabular}{lll}
    \toprule
    $_{\text{\tiny{\ggn}}}$$^{\text{\tiny{Data}}}$ & mb & sub \\
    \midrule
    exact & 688
              & 1247 \\
    mc   & 1232
              & 1259 \\
    \bottomrule
\end{tabular}
    \end{small}
  \end{minipage}
  \hfill
  \begin{minipage}{0.245\linewidth}
  \textcolor{white}{.}
    % \centering
    % \begin{small}
    %   $N_{\text{crit}}$ (CPU)
    % \end{small}
    % \vspace{0.15\baselineskip}

    % \begin{small}
    %   % \input{../../fig/exp10_performance_ggn_evals/fig/N_crit/newton/tab_cifar10_resnet56_cpu_layerwise_group.tex}
    % \end{small}
  \end{minipage}
  
  
\vspace{5ex}
  
%==========================================================  
  
  
  \begin{small}
    \textbf{\cifarhun \allcnnc}
  \end{small}

  \begin{minipage}{0.49\linewidth}
    \centering
    \begin{small}
      \textbf{Full network}
    \end{small}
  \end{minipage}
  \hfill
  \begin{minipage}{0.49\linewidth}
    \centering
    \begin{small}
      \textbf{Block-diagonal approximation}
    \end{small}
  \end{minipage}
  \vspace{1ex}

  \begin{minipage}{0.245\linewidth}
    \centering
    \begin{small}
      $N_{\text{crit}}$ (GPU)
    \end{small}
    \vspace{0.15\baselineskip}

    \begin{small}
      \begin{tabular}{lll}
    \toprule
    $_{\text{\tiny{\ggn}}}$$^{\text{\tiny{Data}}}$ & mb & sub \\
    \midrule
    exact & 14
              & 111 \\
    mc   & 745
              & 1402 \\
    \bottomrule
\end{tabular}
    \end{small}
  \end{minipage}
  \hfill
  \begin{minipage}{0.245\linewidth}
    \centering
    \begin{small}
      $N_{\text{crit}}$ (CPU)
    \end{small}
    \vspace{0.15\baselineskip}

    \begin{small}
      \begin{tabular}{lll}
    \toprule
    $_{\text{\tiny{\ggn}}}$$^{\text{\tiny{Data}}}$ & mb & sub \\
    \midrule
    exact & 43
              & 309 \\
    mc   & 2015
              & 2865 \\
    \bottomrule
\end{tabular}
    \end{small}
  \end{minipage}
  \hfill
  \begin{minipage}{0.245\linewidth}
    \centering
    \begin{small}
      $N_{\text{crit}}$ (GPU)
    \end{small}
    \vspace{0.15\baselineskip}

    \begin{small}
      \begin{tabular}{lll}
    \toprule
    $_{\text{\tiny{\ggn}}}$$^{\text{\tiny{Data}}}$ & mb & sub \\
    \midrule
    exact & 36
              & 255 \\
    mc   & 1119
              & 1536 \\
    \bottomrule
\end{tabular}
    \end{small}
  \end{minipage}
  \hfill
  \begin{minipage}{0.245\linewidth}
    \centering
    \begin{small}
      $N_{\text{crit}}$ (CPU)
    \end{small}
    \vspace{0.15\baselineskip}

    \begin{small}
      \begin{tabular}{lll}
    \toprule
    $_{\text{\tiny{\ggn}}}$$^{\text{\tiny{Data}}}$ & mb & sub \\
    \midrule
    exact & 97
              & 703 \\
    mc   & 3759
              & 4918 \\
    \bottomrule
\end{tabular}
    \end{small}
  \end{minipage}

\end{table}

%%% Local Variables:
%%% mode: latex
%%% TeX-master: "../main"
%%% End:


%%% Local Variables:
%%% mode: latex
%%% TeX-master: "../thesis"
%%% End:
