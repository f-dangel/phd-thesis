This section uses the notation from \Cref{vivit::sec:experiments} (see
\Cref{vivit::tab:notation_cases}).

\begin{table}[ht]
  \centering
  \caption{ \textbf{Notation for curvature approximations.} The notation is
    introduced in \Cref{vivit::sec:experiments}. This table recapitulates the
    abbreviations (referring to the approximations introduced in
    \Cref{vivit::sec:approximations}) and provides corresponding explanations. }
  \label{vivit::tab:notation_cases}
  \vspace{1ex}
  \begin{footnotesize}
    \begin{tabular}{ll}
      \toprule
      Abbreviation
      & Explanation \\
      \midrule
      \textbf{mb, exact}
      & \makecell[tl]{Exact \ggn with all mini-batch samples.\\
      Backpropagates $NC$ vectors.}
      \\
      \textbf{mb, mc}
      & \makecell[tl]{ \mc-approximated \ggn with all mini-batch samples.\\
      Backpropagates $N M$ vectors with $M$ the number of \mc{}-samples.}
      \\
      \textbf{sub, exact}
      & \makecell[tl]{Exact \ggn on a subset of mini-batch samples ($\floor{\nicefrac{N}{8}}$ as in \cite{zhang2017blockdiagonal}).\\
      Backpropagates $\floor{\nicefrac{N}{8}} C$ vectors.}
      \\
      \textbf{sub, mc}
      & \makecell[tl]{\mc-approximated \ggn on a subset of mini-batch samples.\\
      Backpropagates $\floor{\nicefrac{N}{8}} M$ vectors with $M$ the number of \mc{}-samples.}
      \\
      \bottomrule
    \end{tabular}
  \end{footnotesize}
\end{table}

%%% Local Variables:
%%% mode: latex
%%% TeX-master: "../thesis"
%%% End:
