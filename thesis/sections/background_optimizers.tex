\begin{figure*}[!p]
  \centering
  \begin{minipage}[t]{0.495\linewidth}
    \begin{updaterule}[\textbf{SGD~\cite{robbins1951stochastic}}]\label{opt:background::SGD}
      \begin{align*}
        \vtheta_{t+1} &= \vtheta_t - \eta \vg_{\sB_t}(\vtheta_t)
                        \shortintertext{with}
        &\text{learning rate $\eta \in \sR^+$}
      \end{align*}
    \end{updaterule}
    \vspace{-1.1ex}
    \begin{updaterule}[\textbf{Momentum~\cite{polyak1964some}}]\label{opt:background::Momentum}
      \begin{align*}
        \vtheta_{t+1} &= \vtheta_t - \vv_{t}
                        \shortintertext{where}
                        \vv_t &= \rho \vv_{t-1} + \eta \vg_{\sB_t}(\vtheta_t)
                                \shortintertext{with}
        &\text{learning rate $\eta \in \sR^+$}
        \\
                      &\text{momentum factor $\rho \in [0;1)$}
        \\
                      &\text{initial momentum $\vv_{0} \in \sR^{D}$}
      \end{align*}
    \end{updaterule}
    \vspace{-1.1ex}
    \begin{updaterule}[\textbf{NAG~\cite{nesterov1983method}}]\label{opt:background::NAG}
      \begin{align*}
        \vtheta_{t+1} &= \vtheta_t - \vv_{t}
                        \shortintertext{where}
                        \vv_t &= \rho \vv_{t-1} + \eta \vg_{\sB_t}(\vtheta_t - \rho \vv_{t-1})
                                \shortintertext{with}
        &\text{learning rate $\eta \in \sR^+$}
        \\
                      &\text{momentum factor $\rho \in [0;1)$}
        \\
                      &\text{initial momentum $\vv_{0} \in \sR^{D}$}
      \end{align*}
    \end{updaterule}
    \vspace{-1.1ex}
    \begin{updaterule}[\textbf{AdaGrad~\cite{duchi2011adaptive}}]\label{opt:background::AdaGrad}
      \begin{align*}
        \vtheta_{t+1} &= \vtheta_t - \eta \vg_{\sB_t}(\vtheta_t) \oslash (\vs_{t}^{\circ \nicefrac{1}{2}} + \epsilon \vone)
                        \shortintertext{where}
                        \vs_t &= \vs_{t-1} + (\vg_{\sB_t}(\vtheta_t))^{\circ 2}
                                \shortintertext{with}
        &\text{learning rate $\eta \in \sR^+$}
        \\
                      &\text{divide-by-zero safe guard $\epsilon \in \sR^+$}
        \\
                      &\text{initial $\vs_{0} \in \sR^{D}$}
      \end{align*}
    \end{updaterule}
    \vspace{-1.1ex}
    \begin{updaterule}[\textbf{RMSProp~\cite{tieleman2012lecture}}]\label{opt:background::RMSProp}
      \begin{align*}
        \vtheta_{t+1} &= \vtheta_t - \eta \vg_{\sB_t}(\vtheta_t) \oslash (\vs_{t}^{\circ \nicefrac{1}{2}} + \epsilon \vone)
                        \shortintertext{where}
                        \vs_t &= \rho \vs_{t-1} + (1-\rho) (\vg_{\sB_t}(\vtheta_t))^{\circ 2}
                                \shortintertext{with}
        &\text{learning rate $\eta \in \sR^+$}
        \\
                      &\text{decay rate $\rho \in [0;1)$}
        \\
                      &\text{divide-by-zero safe guard $\epsilon \in \sR^+$}
        \\
                      &\text{initial $\vs_{0} \in \sR^{D}$}
      \end{align*}
    \end{updaterule}
  \end{minipage}
  \hfill
  \begin{minipage}[t]{0.495\linewidth}
    \begin{updaterule}[\textbf{Adadelta~\cite{zeiler2012adadelta}}]\label{opt:background::Adadelta}
      \begin{align*}
        \vtheta_{t+1} &= \vtheta_t - \eta \vDelta_{t}
                        \shortintertext{where}
                        \vDelta_{t} &= (\vd_{t} + \epsilon \vone)^{\circ \nicefrac{1}{2}} \oslash (\vs_{t} + \epsilon \vone)^{\circ \nicefrac{1}{2}} \odot \vg_{\sB_t}(\vtheta_t)
                                      \shortintertext{\quad \ \, where}
        &\vs_t = \rho \vs_{t-1} + (1-\rho) (\vg_{\sB_t}(\vtheta_t))^{\circ 2}
        \\
                      &\vd_t = \rho \vd_{t-1} + (1-\rho) \vDelta_{t-1}^{\circ 2}
                        \shortintertext{with}
                                    &\text{learning rate $\eta \in \sR^+$}
        \\
                      &\text{decay rate $\rho \in [0;1)$}
        \\
                      &\text{divide-by-zero safe guard $\epsilon \in \sR^+$}
        \\
                      &\text{initial $\vs_{0}, \vd_{0}, \vDelta_{0} \in \sR^{D}$}
      \end{align*}
    \end{updaterule}
    \vspace{-1.1ex}
    \begin{updaterule}[\textbf{Adam~\cite{kingma2015adam}}]\label{ex:background::Adam}
      \begin{align*}
        \vtheta_{t+1} &= \vtheta_t - \eta \vmhat_{t}\oslash ( \vvhat_{t}^{\circ \nicefrac{1}{2}} + \epsilon \vone)
                        \shortintertext{where}
                        \vmhat_{t} &= \nicefrac{\vm_{t}}{(1 - \beta_{1}^{t})}
                                     \shortintertext{\quad \ \, where}
        &\vm_{t} = \beta_1 \vm_{t-1} + (1 - \beta_{1}) \vg_{\sB_t}(\vtheta_{t})
        \\
        \vvhat_t &= \nicefrac{\vv_t}{(1 - \beta_2^t)}
                   \shortintertext{\quad \ \, where}
                                   &\vv_t = \beta_2 \vv_{t-1} + (1 - \beta_2) (\vg_{\sB_t}(\vtheta_t))^{\circ 2}
                                     \shortintertext{with}
        &\text{learning rate $\eta \in \sR^+$}
        \\
                      &\text{decay rates $\beta_1, \beta_2 \in [0;1)$}
        \\
                      &\text{divide-by-zero safe guard $\epsilon \in \sR^+$}
        \\
                      &\text{initial $\vm_{0}, \vv_{0}, \in \sR^{D}$}
      \end{align*}
    \end{updaterule}
    \vspace{-1.1ex}
    \begin{updaterule}[\textbf{AMSGrad~\cite{reddi2018on}}]\label{opt:background::AMSGrad}
      \begin{align*}
        \vtheta_{t+1} &= \vtheta_t - \eta \vmhat_{t}\oslash ( \vvhat_{t}^{\circ \nicefrac{1}{2}} + \epsilon \vone)
                        \shortintertext{where}
                        \vmhat_{t} &= \nicefrac{\vm_{t}}{(1 - \beta_{1}^{t})}
                                     \shortintertext{\quad \ \, where}
        &\vm_{t} = \beta_1 \vm_{t-1} + (1 - \beta_{1}) \vg_{\sB_t}(\vtheta_{t})
        \\
        \vvhat_t &= \nicefrac{\vv_t}{(1 - \beta_2^t)}
                   \shortintertext{\quad \ \, where}
                                   &\vv_t = \max( \beta_2 \vv_{t-1} + (1 - \beta_2) (\vg_{\sB_t}(\vtheta_t))^{\circ 2}, \vv_{t-1})
                                     \shortintertext{with}
        &\text{learning rate $\eta \in \sR^+$}
        \\
                      &\text{decay rates $\beta_1, \beta_2 \in [0;1)$}
        \\
                      &\text{divide-by-zero safe guard $\epsilon \in \sR^+$}
        \\
                      &\text{initial $\vm_{0}, \vv_{0}, \in \sR^{D}$}
      \end{align*}
    \end{updaterule}
  \end{minipage}
  \vspace{-1ex}
  \caption{\textbf{Popular deep learning optimizers rely on the average
      mini-batch gradient.} At iteration $t$, they incorporate information in
    form of the mini-batch average gradient $\vg_{\sB_t}$. This is a
    representative subset of popular methods; see \cite{schmidt2021descending}
    for a more complete
    overview.}\label{fig:background::popularDeepLearningOptimizers}
\end{figure*}

%%% Local Variables:
%%% mode: latex
%%% TeX-master: "../../thesis"
%%% End:


\Cref{sec:background::SupervisedLearning} formulates learning as minimizing the
empirical risk, for which an optimization algorithm seeks to find a solution.
Deep learning optimizers are iterative: they start from an initial parameter
$\vtheta_{0}$ and aim to improve the parameters in iterations, leading
to a trajectory $\vtheta_{0}, \vtheta_{1}, \dots$ in parameter space. An
iteration is described by an update rule $\gM: \vtheta_{t} \mapsto
\vtheta_{t+1}$ that relies on internal states of the optimizer from the
observation history $H_t$, local observations at the current iterate
$\vtheta_t$, and hyperparameters $\phi_t$~\cite{choi2020on}. The next iterate
results from a simplified optimization problem where the actual objective is
replaced by a computationally cheap approximation. Often, this direction is
further adapted by outer-loop mechanisms, like line searches, or hyperparameters
that require manual tuning to achieve good performance.

The update quality is influenced by the quality of the local approximation.
Computing exact information about the objective $\gL_{\sD}$ is expensive, but
yields a more accurate local proxy. Due to the large-scale nature of deep
learning, algorithms only rely on stochastic information from a mini-batch
$\sB_t$ in form of the mini-batch loss $\gL_{\sB_t}$
(\Cref{sec:background::MiniBatching}). In addition to locality, noise further
complicates deriving a precise local approximation.

\subsubsection{Stochastic Gradient Descent}
One of the most popular methods, according to
\Cref{fig:background::ArxivMentions}, is stochastic gradient descent (\sgd,
\citep{robbins1951stochastic}). Its update derives from
\begin{subequations}
  \begin{align}
    \gM:\qquad
    \vtheta_{t+1}
    =
    \argmin_{\vtheta} m_{\vtheta_t}(\vtheta)
    =
    \vtheta_t - \eta \vg_{\sB_t}(\vtheta_t)
  \end{align}
  where $m_{\vtheta_t}(\vtheta)$ is a first-order Taylor around the current
  iterate $\vtheta_t$ based on the mini-batch gradient, regularized by a quadratic
  $L_2$ penalty to discourage large steps with a learning rate $\eta$,
  \begin{align}\label{eq:background::SGDLocalApproximation}
    m_{\vtheta_t}(\vtheta)
    =
    \gL_{\sB_t}(\vtheta_t)
    +
    {\vg_{\sB_t}(\vtheta_t)}^{\top}
    (\vtheta - \vtheta_t)
    +
    \frac{1}{2 \eta}
    \lVert \vtheta - \vtheta_t \rVert_2^2
  \end{align}
\end{subequations}
This leads to \Cref{opt:background::SGD}, which updates the parameters with the
scaled negative mini-batch gradient. The negative gradient also corresponds to
the direction of steepest descent, \ie along which the loss decreases most
rapidly around $\vtheta_t$ (see
\Cref{sec:background::naturalGradientDescent,eq:background::negativeGradientSteepestDescent}).

\subsubsection{Momentum Methods (Incorporating Past Knowledge)}

The descent direction in \sgd becomes poorer as the mini-batch gradient is
subject to more noise. Incorporating previous gradient observations can help
reduce noise and find a better descent direction~\cite[][Chapter
8.3]{goodfellow2016deep}. From a physical interpretation, the optimizer builds
up a velocity (referred to as \emph{momentum}) that is adapted with new gradient
observation. Two variations are the heavy ball~\cite{polyak1964some} and
Nesterov~\cite{nesterov1983method} momentum methods
(\Cref{opt:background::Momentum,opt:background::NAG}). Another intuition for
momentum is that it suppresses \sgd's oscillating behavior in the presence of
two directions with differing curvature by averaging out gradients that point in
opposite directions (see \eg central panel of \Cref{cockpit::fig:LINE}).

\subsubsection{Adaptive Methods (Per-parameter Learning Rate)}

Adaptive methods like AdaGrad~\cite{duchi2011adaptive},
RMSProp~\cite{tieleman2012lecture}, Adadelta~\cite{zeiler2012adadelta},
Adam~\cite{kingma2015adam}~\&~AMSGrad~\cite{reddi2018on} keep a learning rate
for each parameter that is adapted over time. This rescaling is driven by the
elementwise square of the mini-batch gradient, $\vg_{\sB_t}(\vtheta_t)^{\odot
  2}$
(\Cref{opt:background::AdaGrad,opt:background::RMSProp,opt:background::Adadelta,ex:background::Adam,opt:background::AMSGrad}).
These algorithms use this quantities in different ways, motivated by
shortcomings of predecessors (\eg too aggressive learning rate
shrinking~\cite{tieleman2012lecture,zeiler2012adadelta} and convergence
problems~\cite{reddi2018on}).

\subsubsection{No Clear Winner \& Mini-batch Gradient as Central Object}

\Cref{fig:background::popularDeepLearningOptimizers} summarizes the update rules
of the above optimization methods. Their update rules to derive $\vtheta_{t+1}$
rely on three ingredients:
\begin{itemize}
\item The history of iterates $\{\vtheta_{t'}\}_{t'=0}^{t}$
\item The history of gradients\sidenote{%
    One mild exception is Nesterov momentum (\Cref{opt:background::NAG}). It
    relies on the ``lookahead'' gradient history rather than the gradient
    history of iterates.}%
  $\{\vg_{\sB_{t'}}(\vtheta_{t'})\}_{t'=0}^{t}$

\item The history of elementwise gradient squares
  $\{\vg_{\sB_{t'}}(\vtheta_{t'})^{\odot 2}\}_{t'=0}^{t}$
  % , which can be derived from the gradient history
\end{itemize}
In summary, the main ingredient in all of these methods is the average
mini-batch gradient (and its elementwise square). Their updates are cheap and
only require computation of average mini-batch gradients, as efficiently
provided by AD in popular ML libraries
(\Cref{sec:background::GradientBackpropagation}):
\begin{align}\label{eq:background::PopularOptimizers}
  \begin{split}
    \gM
    &=
      \gM(
      H_t,
      \phi_t)
    \\
    \text{with}\qquad
    H_t
    &=
      \left\{\vtheta_{t'}\right\}_{t'=0}^{t},
      \cup
      \left\{\vg_{\sB_{t'}}(\vtheta_{t'}) \right\}_{t'=0}^{t},
      \cup
      \left\{\vg_{\sB_{t'}}(\vtheta_{t'})^{\odot 2} \right\}_{t'=0}^{t},
  \end{split}
\end{align}
While the mentioned methods are a representative subset of popular algorithms,
there are more than one hundred algorithms with structurally similar update
rules (see Table 2 in~\cite{schmidt2021descending} and references therein). Some
adaptive methods like Adam and RMSProp contain simpler methods like SGD and
Momentum as special cases~\cite{choi2020on}. Therefore, they should, in
principle, be able to perform better. However, recent work that compares deep
learning optimizers through benchmarks to identify the best-performing method
finds that the currently existing methods perform quite
similarly~\cite{schmidt2021descending}.

One reason why newly developed optimizers of the structure
\Cref{eq:background::PopularOptimizers} seem to not be able to clearly improve
over existing methods could be that their update rules are constrained to using
the average gradient, as it is readily available in software. To overcome these
limitations, it might be helpful to study the potential of methods that leverage
information beyond the gradient. Second-order optimization incorporate curvature
information and represent the state-of-the art in ``classical'' optimization
problems (convex optimization, generalized linear models). Noise-reduction
methods focus on improving the gradient estimator $\vg_{\sB_t}$ by reducing its
variance through per-sample information. The rest of this chapter provides
definitions for the information used by such methods, and provides additional
motivation for them by showcasing applications outside optimization.

%%% Local Variables:
%%% mode: latex
%%% TeX-master: "../thesis"
%%% End:
