Together with empirical risk minimization and neural networks, the last
ingredient in the ML pipeline, \Cref{alg:background::trainingLoop}, is computing
the gradient. Contemporary methods to train neural networks
(\Cref{sec:background::PopularOptimizers}) rely on this quantity. ML libraries
compute it via their built-in automatic differentiation (AD), built around the
famous backpropagation algorithm~\cite{rumelhart1986learning}, described in more
detail here.

\subsection{Foundations}\label{sec:background::ADFoundations}

\subsubsection{An Example \& Path Interpretation}

Given a program that evaluates a function, AD produces a program to evaluate its
derivative. The general idea is to consider the function as composition of
atomic operations (\eg addition, multiplication, \dots) with known derivatives.
To automatically compute the derivatives, one needs to track the dependencies
between intermediate variables and combine their derivatives using the chain
rule.

\tikzexternalenable
\definecolor{autodiffSketchForwardFill}{RGB}{255,255,204}
\definecolor{autodiffSketchBackwardFill}{RGB}{161,218,180}

\tikzset{autodiffSketchForward/.style={
    font=\footnotesize,
    fill=autodiffSketchForwardFill,
    thick,
    rounded corners=1ex,
    draw=black,
    minimum width=3.5ex,
    minimum height=3ex,
  }
}

\tikzset{autodiffSketchBackward/.style={
    font=\footnotesize,
    fill=autodiffSketchBackwardFill,
    thick,
    rounded corners=1ex,
    draw=black,
    minimum width=3.5ex,
    minimum height=3ex,
  }
}

\tikzset{autodiffSketchArrow/.style={
    ->,
    >=stealth,
    ultra thick,
    black,
  }
}

\newcommand{\autodiffSketchDrawNodes}{
  \node [anchor = south west, autodiffSketchForward] (z1)
  at (0, 0) {$z^{(1)}$};
  \node [anchor = south east, autodiffSketchForward] (z2)
  at (\figwidth pt, 0) {$z^{(2)}$};
  \node [anchor = center, autodiffSketchForward] (z3)
  at (0.5*\figwidth pt, 0.25*\figheight pt) {$z^{(3)}$};
  \node [anchor = west, autodiffSketchForward] (z4)
  at (0, 0.575*\figheight pt) {$z^{(4)}$};
  \node [anchor = east, autodiffSketchForward] (z5)
  at (\figwidth pt, 0.425*\figheight pt) {$z^{(5)}$};
  \node [anchor = east, autodiffSketchForward] (z6)
  at (\figwidth pt, 0.75*\figheight pt) {$z^{(6)}$};
  \node [anchor = north, autodiffSketchForward] (z7)
  at (0.5*\figwidth pt, \figheight pt) {$z^{(7)}$};
}

\newcommand{\autodiffSketchConnectNodes}{
  \draw[autodiffSketchArrow] (z1.north) to (z4.south);
  \draw[autodiffSketchArrow] (z2.north west) to (z3.south east);
  \draw[autodiffSketchArrow] (z2.north) to (z5.south);
  \draw[autodiffSketchArrow] (z3.north east) to (z5.south west);
  \draw[autodiffSketchArrow] (z4.north) to (z7.south west);
  \draw[autodiffSketchArrow] (z4.north east) to (z6.south west);
  \draw[autodiffSketchArrow] (z5.north) to (z6.south);
  \draw[autodiffSketchArrow] (z6.north west) to (z7.south east);
}

\newcommand{\autodiffSketchDrawBackwardNodes}{
  \draw[autodiffSketchArrow, draw=none] (z1.north) to
  node [midway, autodiffSketchBackward, xshift=2ex] {$\exp(z^{(1)})$}
  (z4.south);
  \draw[autodiffSketchArrow, draw=none] (z2.north west) to
  node [midway, autodiffSketchBackward] {$\cos(z^{(2)})$}
  (z3.south east);
  \draw[autodiffSketchArrow, draw=none] (z2.north) to
  node [midway, autodiffSketchBackward] {$1$}
  (z5.south);
  \draw[autodiffSketchArrow, draw=none] (z3.north east) to
  node [midway, autodiffSketchBackward] {$1$}
  (z5.south west);
  \draw[autodiffSketchArrow, draw=none] (z4.north)
  to
  node [midway, autodiffSketchBackward] {$1$}
  (z7.south west);
  \draw[autodiffSketchArrow, draw=none] (z4.north east) to
  node [midway, autodiffSketchBackward] {$z^{(5)}$}
  (z6.south west);
  \draw[autodiffSketchArrow, draw=none] (z5.north) to
  node [midway, autodiffSketchBackward] {$z^{(4)}$}
  (z6.south);
  \draw[autodiffSketchArrow, draw=none] (z6.north west) to
  node [midway, autodiffSketchBackward] {$1$}
  (z7.south east);
}

\newcommand{\autodiffSketchDefineSizeClip}{
  \pgfmathsetmacro{\figwidth}{0.95\linewidth}
  \pgfmathsetmacro{\figheight}{1.3\linewidth}
  \clip (0,0) rectangle (\figwidth pt, \figheight pt);
}

\begin{figure}
  \centering
  \begin{subfigure}[t]{0.45\linewidth}
    \centering
    \caption{Function as Python program}
    \label{subfig:background::AutodiffSketch1}
\begin{lstlisting}[language=Python]
from math import exp, sin

def z7(z1: float, z2: float):
    """Example function."""

    # intermediate variables
    z3 = sin(z2)
    z5 = z3 + z2
    z4 = exp(z1)
    z6 = z4 * z5

    # output variable
    z7 = z4 + z6

    return z7
\end{lstlisting}
  \end{subfigure}
  \hfill
  \begin{subfigure}[t]{0.45\linewidth}
    \centering
    \caption{Computation graph}
    \label{subfig:background::AutodiffSketch2}
    \begin{tikzpicture}
      \autodiffSketchDefineSizeClip
      \autodiffSketchDrawNodes
      \autodiffSketchConnectNodes

      \node [font=\footnotesize, anchor=west] at (z4.east) {$\exp(z^{(1)})$};
      \node [font=\footnotesize, anchor=north east] at (z3.south) {$\sin(z^{(2)})$};
      \node [font=\footnotesize, anchor=east] at (z5.west) {$z^{(3)} + z^{(2)}$};
      \node [font=\footnotesize, anchor=east] at (z6.west) {$z^{(4)} \cdot z^{(5)}$};
      \node [font=\footnotesize, anchor=west] at (z7.east) {$z^{(4)} + z^{(6)}$};
    \end{tikzpicture}
  \end{subfigure}

  \vspace{1ex}

  \begin{subfigure}[t]{0.45\linewidth}
    \centering
    \caption{Local derivatives}
    \label{subfig:background::AutodiffSketch3}
    \begin{tikzpicture}
      \autodiffSketchDefineSizeClip
      \autodiffSketchDrawNodes
      \autodiffSketchConnectNodes
      \autodiffSketchDrawBackwardNodes
    \end{tikzpicture}
  \end{subfigure}
  \hfill
  \begin{subfigure}[t]{0.45\linewidth}
    \centering
    \caption{Bauer paths}
    \label{subfig:background::AutodiffSketch4}
    \begin{tikzpicture}
      \autodiffSketchDefineSizeClip

      \begin{scope}[transparency group, opacity=0.2]
        \autodiffSketchDrawNodes
        \autodiffSketchConnectNodes
      \end{scope}
      \begin{scope}[transparency group, opacity=0.66]
        \autodiffSketchDrawBackwardNodes
      \end{scope}

      \begin{pgfonlayer}{background}
        \draw [maincolor, line width = 2.5pt] plot [smooth] coordinates
        {($(z7.east)!0.8!(z7.west)$)
          ($(z4.east)!0.66!(z4.west)$)
          ($(z1.east)!0.66!(z1.west)$)};

        \draw [secondcolor, line width = 2.5pt] plot [smooth] coordinates
        {($(z7.east)!0.6!(z7.west)$)
          ($(z6.east)!0.75!(z6.west)$)
          ($(z4.east)!0.33!(z4.west)$)
          ($(z1.east)!0.33!(z1.west)$)};

        \draw [thirdcolor, line width = 2.5pt] plot [smooth] coordinates
        {($(z7.east)!0.4!(z7.west)$)
          ($(z6.east)!0.5!(z6.west)$)
          ($(z5.east)!0.66!(z5.west)$)
          (z3)
          ($(z2.east)!0.66!(z2.west)$)};

        \draw [darkgraycolor, line width = 2.5pt] plot [smooth] coordinates
        {($(z7.east)!0.2!(z7.west)$)
          ($(z6.east)!0.25!(z6.west)$)
          ($(z5.east)!0.33!(z5.west)$)
          ($(z2.east)!0.33!(z2.west)$)};
      \end{pgfonlayer}
    \end{tikzpicture}
  \end{subfigure}
  \caption{\textbf{Basic AD principles~\cite{oktay2021randomized}.}
    \subfigref{subfig:background::AutodiffSketch1} Example input program
    represented by Python code. The atomic operations combined via
    composition are addition, multiplication, exponential function, and sine.
    The result $z^{(7)}$ is computed from the inputs $z^{(1)}, z^{(2)}$ through
    intermediates $z^{(3)}, z^{(4)}, z^{(5)}, z^{(6)}$,
    \begin{align*}
      z^{(7)}
      &=
        \exp(z^{(1)})
      \\
      &\phantom{=}\,
        + \exp(z^{(1)}) \left( \sin(z^{(2)}) + z^{(2)} \right)\,.
    \end{align*}
    \subfigref{subfig:background::AutodiffSketch2}
    Representation as computation graph to track dependencies between the
    intermediate variables on the level of atomic operations.
    \subfigref{subfig:background::AutodiffSketch3} Computing derivatives relies
    on local derivatives $\nicefrac{\partial z^{(j)}}{\partial z^{(i)}}$ on edges
    $(z^{(i)}, z^{(j)})$, which need to be accumulated according to the chain rule.
    \subfigref{subfig:background::AutodiffSketch4} Interpretation of the chain
    rule as sum over path products. Computing the derivatives of a node \wrt to
    another node in the graph requires summing the path product of local
    derivatives for all paths that connect them. In detail:
    \begin{align*}
      \frac{\partial z^{(7)}}{\partial z^{(1)}}
      &=
        \textcolor{maincolor}{
        \frac{\partial z^{(7)}}{\partial z^{(4)}}
        \frac{\partial z^{(4)}}{\partial z^{(1)}}
        }
        +
        \textcolor{secondcolor}{
        \frac{\partial z^{(7)}}{\partial z^{(6)}}
        \frac{\partial z^{(6)}}{\partial z^{(4)}}
        \frac{\partial z^{(4)}}{\partial z^{(1)}}
        }
      \\
      &=
        \textcolor{maincolor}{
        \exp(z^{(1)})
        }
        +
        \textcolor{secondcolor}{
        z^{(5)}
        \exp(z^{(1)})
        }
      \\
      &=
        \textcolor{maincolor}{
        \exp(z^{(1)})
        }
      \\
      &\phantom{=}\,
        +
        \textcolor{secondcolor}{
        \left( \sin(z^{(2)}) + z^{(2)} \right)
        \exp(z^{(1)})
        }\,,
      \\
      \frac{\partial z^{(7)}}{\partial z^{(2)}}
      &=
        \textcolor{thirdcolor}{
        \frac{\partial z^{(7)}}{\partial z^{(6)}}
        \frac{\partial z^{(6)}}{\partial z^{(5)}}
        \frac{\partial z^{(5)}}{\partial z^{(3)}}
        \frac{\partial z^{(3)}}{\partial z^{(2)}}
        }
      \\
      &\phantom{=}\,
        +
        \textcolor{darkgraycolor}{
        \frac{\partial z^{(7)}}{\partial z^{(6)}}
        \frac{\partial z^{(6)}}{\partial z^{(5)}}
        \frac{\partial z^{(5)}}{\partial z^{(2)}}
        }
      \\
      &=
        \textcolor{thirdcolor}{
        z^{(4)}
        \cos(z^{(2)})
        }
        +
        \textcolor{darkgraycolor}{
        z^{(4)}
        }
      \\
      &=
        \textcolor{thirdcolor}{
        \exp(z^{(1)})
        \cos(z^{(2)})
        }
        +
        \textcolor{darkgraycolor}{
        \exp(z^{(1)})
        }\,.
    \end{align*}}\label{fig:background::AutodiffSketch}
\end{figure}

%%% Local Variables:
%%% mode: latex
%%% TeX-master: "../../thesis"
%%% End:

\tikzexternaldisable

\Cref{fig:background::AutodiffSketch} illustrates the basic principles of AD for
an example from~\cite{oktay2021randomized}. The starting point is a function
defined by code (\Cref{subfig:background::AutodiffSketch1}). Such a function
transforms input to output variables through atomic operations and builds up
intermediate variables along its execution. The relation between these
operations are described by a directed graph $\gG = (\gV, \gE)$ with a set of
nodes $\gV = \{z^{(1)}, z^{(2)}, \dots\}$ and a set of edges $\gE =
\{(z^{(i_1)}, z^{(j_1)}), (z^{(i_2)}, z^{(j_2)}), \dots\}$ where $(z^{(i)},
z^{(j)})$ denotes a directed edge from node $z^{(i)}$ to $z^{(j)}$
(\Cref{subfig:background::AutodiffSketch2}).

To compute derivatives, the local derivatives on edges
(\Cref{subfig:background::AutodiffSketch3}) are combined according to the chain
rule. For $\nicefrac{\partial z^{(j)}}{ \partial z^{(i)}}$ between two variables
in the graph, all paths connecting them need to be considered. A path between
two nodes $z^{(i)}$ and $z^{(j)}$ is a sequence of edges that connect them:
starting from $z^{(i)}$, following the edges in a path leads to $z^{(j)}$. Let
$[z^{(i)} \to z^{(j)}]$ denote the set of paths connecting $z^{(i)}$ to
$z^{(j)}$. Then the derivative is the sum of path products of local derivatives
(\Cref{subfig:background::AutodiffSketch4}),
\begin{align}
  \label{eq:background::BauersFormula}
  \frac{\partial z^{(j)}}{\partial z^{(i)}}
  =
  \sum_{p \in [z^{(i)} \to z^{(j)}]}
  \prod_{(z^{(k)}, z^{(l)}) \in p}
  \frac{\partial z^{(l)}}{\partial z^{(k)}}\,.
\end{align}
The path formulation goes back to~\citet{bauer1974computational}.

\subsubsection{The Jacobian Matrix \& Its Chain Rule}
For the computation graph of a neural network's empirical risk, tracking
dependencies between variables at a scalar level would result in a considerable
book-keeping overhead due to the large number of connections. This can be
circumvented by using vector-valued (or tensor-valued) nodes $ \vz^{(1)},
\vz^{(2)}, \dots$ and tracking edges between vectors (or tensors), see
\Cref{fig:background::neuralNetworkLoss}. However, this requires a
generalization of \Cref{eq:background::BauersFormula} to multi-variate nodes.
The accumulation is efficiently expressed as matrix multiplication by arranging
partial derivatives into \emph{Jacobians}.

\begin{definition}[\textbf{Jacobian}]\label{def:background::JacobianVectorVector}
  Let $\vb:\mathbb{R}^{n} \to \mathbb{R}^{m}, \va \mapsto \vb(\va)$ be a
  differentiable vector-to-vector function. The Jacobian $\jac_{\va} \vb(\va)$
  of $\vb$ \wrt $\va$ is an $m \times n$ matrix with partial derivatives,
  \begin{align}
    \label{eq:background::JacobianVector}
    \jac_{\va} \vb(\va) = \frac{\partial \vb(\va)}{\partial \va^\top}\,,
    \quad\text{with}\quad
    [\jac_{\va} \vb(\va)]_{i,j}
    &=
      \frac{
      \partial \evb_i(\va)
      }{
      \partial \eva_j
      }\,.
  \end{align}
  Matrix- and tensor-valued functions require flattening their arguments into
  vectors\sidenote[][-15em]{\Cref{def:background::JacobianVectorVector} assumes
    vector-valued functions. With the flattening convention
    \Cref{def:background::Flattening}, it generalizes to tensor-valued functions
    as follows:\vspace{1ex}
    \begin{definition}[\textbf{Generalized Jacobian}]\label{hbp::def:generalizedJacobian}
      Let $\mB:\mathbb{R}^{n\times q}\to\mathbb{R}^{m\times p},
      \mA\mapsto\mB(\mA)$ be a differentiable matrix-to-matrix function. The
      Jacobian $\jac_{\mA} \mB(\mA)$ of $\mB$ \wrt $\mA$ is an $mp \times nq$
      matrix
      \begin{subequations}\label{hbp::equ:generalizedJacobian}
        \begin{align}
          \jac_{\mA} \mB(\mA) &= \frac{\partial \vec \mB(\mA)}{\partial (\vec \mA)^\top}
                                \shortintertext{with entries}
                                [\jac_{\mA} \mB(\mA)]_{i,j}
          &=
            \frac{
            \partial \left[ \vec \mB(\mA)\right]_i
            }{
            \partial \left[\vec \mA\right]_j
            }
        \end{align}
      \end{subequations}
      and the flattening operation $\vec$ from \Cref{def:background::Flattening}
      \citep[Chapter 9.4]{magnus1999MatrixDifferentialCalculus}. The analogous
      tensor case $(\mA, \mB) \to (\tA, \tB)$ requires lengthy notation and is
      therefore omitted.
    \end{definition}
    In the context of neural networks, the most common occurrences of
    \Cref{hbp::def:generalizedJacobian} involve vector-to-vector functions $f:
    \mathbb{R}^n \to \mathbb{R}^m, \vx \mapsto f(\vx)$ with
    \begin{align*}
      \jac_{\vx} f(\vx) = \frac{\partial f(\vx)}{\partial \vx^\top}\,.
    \end{align*}
    For instance, $\vx$ can be considered the input or bias vector of a layer
    applying an affine transformation. Other cases involve matrix-to-vector mappings
    $f: \mathbb{R}^{n\times q} \to \mathbb{R}^m, \mX \mapsto f(\mX)$ with
    \begin{align*}
      \jac_{\mX} f(\mX) = \frac{\partial f(\mX)}{\partial (\vec \mX)^\top}\,,
    \end{align*}
    where $\mX$ might correspond to the $\mathbb{R}^{m\times q}$ weight matrix
    of a linear layer. See \Cref{tab:background::Jacobians} for an overview.
  }.
  For a vector-to-scalar function $b(\va)$, \ie $m=1$, the Jacobian has one row
  that contains the gradient, $[\jac_{\va} b(\va)]^{\top} = \grad{\va}b$. The
  gradient will often be denoted by $\vg$. \Eg $\vg_{\pdata}(\vtheta) :=
  \grad{\vtheta} \gL_{\pdata}(\vtheta)$ for the gradient of the population risk
  \Cref{eq:background::expectedRisk}, and $\vg_{\sD}(\vtheta) := \grad{\vtheta}
  \gL_{\sD}(\vtheta)$ for the gradient of the empirical risk
  \Cref{eq:background::empiricalRisk} on a dataset $\sD$ (with $\sD = \Dtrain, \sB$
  for the train loss and the mini-batch gradient).
\end{definition}

In the vector-valued case, one must accumulate Jacobians through matrix
multiplies instead of scalar multiplications to compute derivatives,
\begin{align}\label{eq:background::BauersFormulaVector}
  \jac_{\vz^{(j)}}\vz^{(i)}
  =
  \sum_{p \in [\vz^{(i)} \to \vz^{(j)}]}
  \prod_{(\vz^{(k)}, \vz^{(l)}) \in p}
  \jac_{\vz^{(k)}}\vz^{(l)}(\vz^{(k)})\,.
\end{align}
The product term generalizes the chain rule to vector-valued functions.

\begin{theorem}[\textbf{Jacobian chain
    rule}]\label{def:background::JacobianChainRuleVector} Let $\vb: \mathbb{R}^{n} \to
  \mathbb{R}^{m}, \va \mapsto \vb(\va)$ and $\vc: \mathbb{R}^{m} \to
  \mathbb{R}^{r}, \vb \mapsto \vc(\vb)$ be differentiable vector-to-vector
  functions. Consider their composition $\vd = \vc \circ \vb: \mathbb{R}^{n} \to
  \mathbb{R}^{r}, \va \mapsto \vd(\va) = \vc(\vb(\va))$. The composition's
  Jacobian $\jac_{\va} \vd(\va) \in \sR^{r \times n}$ is related to the composite
  Jacobians via
  \begin{align}
    \label{eq:background::JacobianChainRuleVector}
    \jac_{\va} \vd(\va)=  \left[ \jac_{\vb} \vc(\vb) \right] \jac_{\va} \vb(\va)\,.
  \end{align}
  This can be generalized to tensor-valued functions\sidenote[][1em]{Proper
    arrangement of partial derivatives leads to a generalized Jacobian chain
    rule for matrices/tensors:\vspace{1ex}
    \begin{theorem}[\textbf{Generalized Jacobian chain rule}]
      \label{hbp::the:chainRuleJacobians}
      Let $\mB: \mathbb{R}^{n\times q} \to \mathbb{R}^{m \times p}$ and $\mC:
      \mathbb{R}^{m\times p} \to \mathbb{R}^{r\times s}$ be differentiable
      matrix-to-matrix functions. Let $\mD = \mC \circ \mB: \mathbb{R}^{n\times q}
      \to \mathbb{R}^{r\times s}, \mA \to \mD(\mA) = \mC(\mB(\mA))$ be their
      composition. Then,
      \begin{align}
        \label{hbp::equ:chainRuleJacobians}
        \jac_{\mA} \mD(\mA)=  \left[ \jac_{\mB} \mC(\mB) \right] \jac_{\mA} \mB(\mA)
      \end{align}
      with the generalized Jacobian \Cref{hbp::def:generalizedJacobian}
      \citep[Chapter 5.15]{magnus1999MatrixDifferentialCalculus}. The tensor case
      $(\mD, \mC, \mB, \mA) \to (\tD, \tC, \tB, \tA)$ is analogous.
    \end{theorem}
  }.
\end{theorem}

\subsubsection{Jacobian Accumulation (Automatic Differentiation Modes)}

Given the computation graph $\gG$ of a function to be differentiated,
\Cref{eq:background::BauersFormulaVector} describes the operations that need to
be performed. But there are different schedules for carrying out these
computations, with differing performance: \eg, it is possible to share
accumulated derivative products between paths that share subpaths (like paths
\tikz[baseline=-0.5ex]{\draw[fill=thirdcolor, draw=none] circle (0.75ex);} and
\tikz[baseline=-0.5ex]{\draw[fill=darkgraycolor, draw=none] circle (0.75ex);} in
\Cref{subfig:background::AutodiffSketch4}). And for a single path in the vector
case, the optimal contraction order of the Jacobian matrix chain (one summand of
\Cref{eq:background::BauersFormulaVector}) depends on the dimension of the nodes
(\Cref{eq:background::MatrixChain}). The following Jacobian accumulation
schedules are of specific interest for AD:
\begin{itemize}
\item \textbf{Forward accumulation, or forward mode AD,} starts at the leafs,
  \ie the nodes \wrt which the function is differentiated. Jacobians are
  accumulated in the same order as the function evaluation.
\item \textbf{Reverse accumulation, or reverse mode
    AD~\cite{griewank2012invented,linnainmaa1976taylor},} starts at the root,
  \ie the variable that is differentiated. Jacobians are accumulated from root
  to leaf nodes, traversing the graph backwards. This is often called a
  \emph{backward pass}.
\item \textbf{Optimal Jacobian accumulation} computes derivatives according to
  the optimal schedule which usually traverses the computation graph in a
  nontrivial fashion. For arbitrary computation graphs, finding this schedule is
  NP-hard \cite{naumann2008optimal}.
\end{itemize}
Due to the specific structure of computation graphs in deep learning, reverse
mode AD is often more practical than forward accumulation. This is outlined in
in the following section, that illustrates reverse mode for differentiation of a
neural network's loss \wrt its parameters, leading to the famous backpropagation
algorithm~\cite{rumelhart1986learning}.

\subsection{Gradient Backpropagation}\label{sec:background:GradientBackpropagationInMLLibraries}

Gradient backpropagation~\cite{rumelhart1986learning} enables efficient
differentiation of the training objective in deep learning. It is the central
algorithm of popular ML libraries with built-in AD. This section presents
backpropagation for chain-structured computation graphs (see~\cite[Chapter
6]{goodfellow2016deep} for the general case) like the loss of a sequential
feedforward neural network. Starting from the loss of a single datum, the goal
is to show that ML libraries combine AD and batching to maximize efficiency. But
this limits their functionality to computing the gradient, ignoring \eg the
per-sample structure in the loss. Alleviating this limitation to compute richer
information (\Cref{chap:background::HigherOrder}) using the existing
implementation of gradient backpropagation is a main goal of \Cref{part:papers}
in this thesis.

\tikzexternalenable
\begin{figure*}[t]
  \centering \resizebox{\linewidth}{!}{ {\footnotesize
      % basic setting of a fully-connected neural network with data flow for
% forward pass

\begin{tikzpicture}
  % first two layers
  \node (in1)
  [inner sep=0]
  {\tikz \drawMessagesWithArrows{$\vz^{(0)}$}{ }{ }{\hNodeDistance};};
  \node (layer1)
  [anchor=south west, inner sep=0]
  at (in1.south east)
  {\tikz \drawModuleWithParams{$f^{(1)}_{\vtheta^{(1)}}$}{16}{$\vtheta^{(1)}$}{ }{ };};
  \node (out1)
  [inner sep=0, anchor=south west]
  at (layer1.south east)
  {\tikz \drawMessagesWithArrows{$\vz^{(1)}$}{ }{ }{\hNodeDistance};};
  \node (layer2)
  [inner sep=0pt, anchor=south west]
  at (out1.south east)
  {\tikz \drawModuleWithParams{$f^{(2)}_{\vtheta^{(2)}}$}{16}{$\vtheta^{(2)}$}{ }{ };};

  % dots with messages
  \node (in2)
  [inner sep=0, anchor=south west]
  at (layer2.south east)
  {\tikz \drawMessagesWithArrows{$\vz^{(2)}$}{ }{ }{\hNodeDistance};};
  \node (dots)
  [xshift=0.75ex, inner sep=0pt, anchor=west]
  at (in2.east)
  {$\dots$};

  \node (inLast)
  [xshift=0.75ex, inner sep=0pt, anchor=west]
  at (dots.east)
  {\tikz \drawMessagesWithArrows{$\vz^{(L-1)}$}{ }{ }{\hNodeDistance};};

  \node (layerLast)
  [anchor=south west, inner sep=0]
  at (inLast.south east)
  {\tikz \drawModuleWithParams{$f^{(L)}_{\vtheta^{(L)}}$}{16}{$\vtheta^{(L)}$}{ }{ };};
  \node (outLast)
  [inner sep=0, anchor=south west]
  at (layerLast.south east)
  {\tikz \drawMessagesWithArrows{$\vz^{(L)}$}{ }{ }{\hNodeDistance};};

  % loss layer
  \node (lossLayer)
  [inner sep=0pt, anchor=south west]
  at (outLast.south east)
  {\tikz\drawModuleNoParams{$\ell$}{5};};
  \node (loss)
  [inner sep=0, anchor=south west]
  at (lossLayer.south east)
  {\tikz \drawMessagesWithArrows{$\ell$}{ }{ }{\hNodeDistance};};

  % label
  \node (label)
  [inner sep=0, anchor=south west, rotate=-90, xshift = -96, yshift = -50]
  at (lossLayer.south east)
  {\tikz \drawMessagesWithArrows{$\vy$}{ }{ }{\hNodeDistance};};
\end{tikzpicture}

%%% Local Variables:
%%% mode: latex
%%% TeX-master: "../../thesis"
%%% End:
}}
  \caption{\textbf{Computation graph of a sequential feedforward neural
      network's loss for a single datum from
      \Cref{eq:background::neuralNetworkLoss}.}}\label{fig:background::neuralNetworkLoss}
\end{figure*}
\tikzexternaldisable

\subsubsection{Loss of a Single Datum}
Consider the loss implied by a single datum $(\vx, \vy)$, a loss function
$\ell$, and a neural network $f_{\vtheta}$ depicted in
\Cref{fig:background::neuralNetworkLoss},
\begin{subequations}
  \begin{align}
    \label{eq:background::neuralNetworkLoss}
    \ell(\vtheta)
    =
    \ell(f_{\vtheta}(\vx), \vy)
    \qquad
    \text{with}
    \qquad
    f_{\vtheta}
    =
    f^{(L)}_{\vtheta^{(L)}}
    \circ
    f^{(L-1)}_{\vtheta^{(L-1)}}
    \circ
    \ldots
    \circ
    f^{(1)}_{\vtheta^{(1)}}\,.
  \end{align}
  Its computation graph $\gG = (\gV, \gE)$ has nodes
  \begin{align}
    \gV &=
          \left\{
          \vtheta^{(l)}
          \right\}_{l=1}^L
          \cup
          \left\{
          \vz^{(l)}
          \right\}_{l=0}^L
          \cup
          \left\{
          \vy, \ell
          \right\}
          \shortintertext{and edges}
          \gE &=
                \left\{
                (\vz^{(l-1)}, \vz^{(l)})
                \right\}_{l=1}^L
                \cup
                \left\{
                (\vtheta^{(l)}, \vz^{(l)})
                \right\}_{l=1}^L
                \cup
                \left\{
                (\vz^{(L)}, \ell), (\vy, \ell)
                \right\}\,.
  \end{align}
\end{subequations}
Each edge implies a Jacobian, categorized as one of the following:
\begin{itemize}
\item The \emph{input-output} Jacobian $\jac_{\vz^{(l-1)}}\vz^{(l)}(\vz^{(l-1)})$ of
  a module $l$.
\item The \emph{parameter-output} Jacobian
  $\jac_{\vtheta^{(l)}}\vz^{(l)}(\vtheta^{(l)})$ of a module $l$.
\item The \emph{prediction-loss} Jacobian $\jac_{\vz^{(L)}}\ell(\vz^{(L)})$ has
  one column and will be written as gradient of the loss \wrt the model
  prediction, $\jac_{\vz^{(L)}}\ell(\vz^{(L)}) = [\grad{\vz^{(L)}}
  \ell(\vz^{(L)})]^{\top}$.
\end{itemize}
The goal is to compute \emph{parameter-loss} Jacobians, \ie gradients of the
loss \wrt parameters $\jac_{\vtheta^{(l)}}\ell(\vtheta^{(l)}) =
[\grad{\vtheta^{(l)}} \ell(\vtheta^{(l)})]^{\top}$ for all layers $l=1,\dots,
L$.

First, consider only the gradient $\grad{\vtheta^{(l)}} \ell$ of one parameter
$\vtheta^{(l)}$. \Cref{eq:background::BauersFormulaVector} requires identifying
all paths connecting $\vtheta^{(l)}$ to $\ell$. Due to the computation graph's
chain structure, this is only a single path,
\begin{subequations}
  \begin{gather}
    [\vtheta^{(l)} \to \ell]
    =
    \left\{
      p
    \right\}
    \shortintertext{with}
    p
    =
    \left(
      \left( \vtheta^{(l)}, \vz^{(l)} \right),
      \left( \vz^{(l)}, \vz^{(l+1)} \right),
      \dots
      \left( \vz^{(L-1)}, \vz^{(L)} \right),
      \left( \vz^{(L)}, \ell \right)
    \right)\,.
  \end{gather}
  Plugging this into \Cref{eq:background::BauersFormulaVector} simplifies to
  \begin{align}
    \label{eq:background::ParameterJacobian}
    \underbrace{
    \jac_{\vtheta^{(l)}}\ell(\vtheta)
    }_{
    1 \times d^{(l)}
    }
    =
    \underbrace{
    \left[
    \jac_{\vz^{(L)}}\ell
    \right]
    }_{
    1 \times h^{(L)}
    }
    \underbrace{
    \left[
    \jac_{\vz^{(L-1)}}\vz^{(L)}
    \right]
    }_{
    h^{(L)} \times h^{(L-1)}
    }
    \cdots
    \underbrace{
    \left[
    \jac_{\vz^{(l)}}\vz^{(l+1)}
    \right]
    }_{
    h^{(l+1)} \times h^{(l)}
    }
    \underbrace{
    \left[
    \jac_{\vtheta^{(l)}}\vz^{(l)}
    \right]
    }_{
    h^{(l)} \times d^{(l)}
    }
    \,,
    \intertext{and in gradient notation}
    \label{eq:background::ParameterGradient}
    \underbrace{
    \grad{\vtheta^{(l)}}\ell(\vtheta)
    }_{
    d^{(l)}
    }
    =
    \underbrace{
    \left[
    \jac_{\vtheta^{(l)}}\vz^{(l)}
    \right]^{\top}
    }_{
    d^{(l)}
    \times
    h^{(l)}
    }
    \underbrace{
    \left[
    \jac_{\vz^{(l)}}\vz^{(l+1)}
    \right]^{\top}
    }_{
    h^{(l)}
    \times
    h^{(l+1)}
    }
    \cdots
    \underbrace{
    \left[
    \jac_{\vz^{(L-1)}}\vz^{(L)}
    \right]^{\top}
    }_{
    h^{(L-1)}
    \times
    h^{(L)}
    }
    \underbrace{
    \grad{\vz^{(L)}}\ell
    }_{
    h^{(L)}
    }
    \,.
  \end{align}
\end{subequations}
In comparison to the general formulation
\Cref{eq:background::BauersFormulaVector}, the rather simple graphs of a neural
network's loss yield much simpler expressions
(\Cref{eq:background::ParameterJacobian,eq:background::ParameterGradient}) that
are a result of the Jacobian chain rule
(\Cref{def:background::JacobianChainRuleVector}) applied to the loss
\Cref{eq:background::neuralNetworkLoss}. They also illustrate the impact of
contraction order on performance due to the connection to matrix
chains\sidenote[][-15em]{Assuming no cost to compute a Jacobian, the optimal
  Jacobian contraction of
  \Cref{eq:background::ParameterGradient,eq:background::ParameterJacobian} are
  matrix chain problems that can be solved with dynamic programming: given $n$
  matrices $\mA_1, \mA_2, \dots, \mA_n$, the task is to find the optimal
  contraction schedule of $\mA_1 \mA_2 \cdots \mA_n$. This is crucial for
  performance, as this example from \citet[Chapter 15.2]{cormen2001introduction}
  illustrates:
  \begin{example}[\textbf{Matrix chain contraction}]\label{eq:background::MatrixChain}
    Let $\mA_1 \in \sR^{10 \times 100}, \mA_2 \in \sR^{100 \times 5}, \mA_3 \in
    \sR^{5 \times 50}$. There are two schedules to evaluate the chain $\mA_1
    \mA_2 \mA_3$ (cost for addition neglected for simplicity):
    \begin{itemize}[leftmargin=*]
    \item \textbf{$(\mA_1 \mA_2)\mA_3$:} $\mB_1 = \mA_1 \mA_2 \in \sR^{10 \times
        5}$ costs 100 multiplications per element ($5,000$ in total). $\mB_1
      \mA_3 \in \sR^{10 \times 50}$ costs 5 multiplications per element ($2,500$
      in total).
    \item \textbf{$\mA_1 (\mA_2 \mA_3)$:} $\mB_2 = \mA_2 \mA_3 \in \sR^{100
        \times 50}$ costs 5 multiplications per element ($25,000$ in total).
      $\mA_1 \mB_2 \in \sR^{10 \times 50}$ costs 100 multiplications per element
      ($50,000$ in total).
    \end{itemize}
    The order $(\mA_1 \mA_2) \mA_3$ uses 10x fewer operations ($7,500$ versus
    $75,000$).
  \end{example}
}%
, mentioned in \Cref{sec:background::ADFoundations}.

In forward mode, the matrix chain \Cref{eq:background::ParameterGradient} would
be evaluated from left to right, starting with a $d^{(l)} \times h^{(l)}$
Jacobian that is transformed into $d^{(l)} \times h^{(l')}$ matrices where $l' >
l$. Since the parameter count $d^{(l)}$ in a layer is large in DNNs, these
intermediate matrices are costly to store.

In reverse mode, \Cref{eq:background::ParameterGradient} is evaluated from right
to left, starting with a $h^{(L)}$-dimensional vector that is transformed into
vectors of dimension $h^{(l')}$ with $L > l' \ge l$. These intermediate
accumulations require less memory than forward mode.

Each approach has drawbacks, however. Reverse mode uses a more efficient matrix
multiplication order, but the entire graph must have been evaluated and stored,
or re-computed, to construct the Jacobians. While this is not needed for forward
mode, forward accumulation starts with the Jacobian of an edge $(\vtheta^{(l)},
\vz^{(l)})$ that is not shared with paths for other parameters. Therefore,
intermediate accumulations can not be reused for other gradients. This is
another crucial property of reverse mode, which does allow reuse of intermediate
results: consider the accumulations to obtain the gradient $\grad{\vtheta}\ell$
of all layers,
\begin{align*}
  \setlength{\arraycolsep}{1.4pt} % auto-reverts outside the environment
  \begin{array}{cccrc}
    \grad{\vtheta^{(1)}}\ell
    &
      =
    &
      \left[
      \jac_{\vtheta^{(1)}} \vz^{(1)}
      \right]^{\top}
    &
      \left[
      \jac_{\vz^{(1)}} \vz^{(2)}
      \right]^{\top}
      \left[
      \jac_{\vz^{(2)}} \vz^{(3)}
      \right]^{\top}
      \cdots
      \left[
      \jac_{\vz^{(L-1)}} \vz^{(L)}
      \right]^{\top}
    &
      \grad{\vz^{(L)}} \ell
    \\
    \grad{\vtheta^{(2)}}\ell
    &
      =
    &
      \left[
      \jac_{\vtheta^{(2)}} \vz^{(2)}
      \right]^{\top}
    &
      \left[
      \jac_{\vz^{(2)}} \vz^{(3)}
      \right]^{\top}
      \cdots
      \left[
      \jac_{\vz^{(L-1)}} \vz^{(L)}
      \right]^{\top}
    &
      \grad{\vz^{(L)}} \ell
    \\
    \vdots
    &
      =
    &
      \vdots
    &
      \ddots
      \phantom{
      \left[
      \jac_{\vz^{(L-1)}} \vz^{(L)}
      \right]^{\top}
      }
    &
      \vdots
    \\
    \grad{\vtheta^{(L-1)}}\ell
    &
      =
    &
      \left[
      \jac_{\vtheta^{(L-1)}} \vz^{(L-1)}
      \right]^{\top}
    &
      \left[
      \jac_{\vz^{(L-1)}} \vz^{(L)}
      \right]^{\top}
    &
      \grad{\vz^{(L)}} \ell
    \\
    \grad{\vtheta^{(L)}}\ell
    &
      =
    &
      \left[
      \jac_{\vtheta^{(L)}} \vz^{(L)}
      \right]^{\top}
    &
    %
    &
      \grad{\vz^{(L)}} \ell
  \end{array}
\end{align*}
The paths for any two parameters $\vtheta^{(l_1)}, \vtheta^{(l_2)}$ with $l_2 >
l_1$ share edges
\begin{align*}
  \left( \vz^{(l_2)}, \vz^{(l_2+1)} \right),
  \dots,
  \left( \vz^{(L-1)}, \vz^{(L)} \right),
  \left( \vz^{(L)}, \ell \right)\,.
\end{align*}
Therefore, their matrix chains share the accumulated gradient
\begin{align*}
  \grad{\vz^{(l_2)}}\ell
  =
  \left[
  \jac_{\vz^{(l_2)}}\vz^{(l_2+1)}
  \right]^{\top}
  \left[
  \jac_{\vz^{(l_2+1)}}\vz^{(l_2+2)}
  \right]^{\top}
  \cdots
  \left[
  \jac_{vz^{(L-1)}}\vz^{(L)}
  \right]^{\top}
  \grad{\vz^{(L)}}\ell\,.
\end{align*}
This gradient \wrt hidden features is passed backwards through the graph and
used by a layer, before updating it and passing it to the next. The
interpretation of the described accumulation scheme is therefore known as
gradient backpropagation algorithm \cite{rumelhart1986learning}:%

\begin{definition}[\textbf{Gradient backpropagation for sequential feedforward
    neural networks}]
  \label{def:background::GradientBackpropagation}
  Given the computation graph of a loss $\ell(f_{\vtheta}(\vx), \vy)$ implied by
  a datum $(\vx, \vy)$, a loss function $\ell$, and a sequential feedforward
  neural network $f_{\vtheta}$ from \Cref{eq:background::neuralNetworkLoss},
  gradient backpropagation recovers the gradient vector $\grad{\vtheta}\ell =
  \begin{pmatrix}
    (\grad{\vtheta^{(1)}}\ell)^{\top},
    (\grad{\vtheta^{(2)}}\ell)^{\top},
    \dots,
    (\grad{\vtheta^{(L)}}\ell)^{\top}
  \end{pmatrix}^{\top}
  $ in stages by passing gradients backward through the graph (\Cref{subfig:background::gradientBackpropagation1}):
  \begin{itemize}
  \item Initialize the backpropagated vector with $\grad{\vz^{(L)}}\ell$ at $L$.
  \item For layer $l = L, \dots, 1$
    \begin{enumerate}
    \item Receive the output gradient $\grad{\vz^{(l)}}\ell$.
    \item Recover the parameter gradient $\grad{\vtheta^{(l)}}\ell =
      \left[ \jac_{\vtheta^{(l)}}\vz^{(l)}\right]^{\top}\grad{\vz^{(l)}}\ell$\,.
    \item Compute the input gradient $\grad{\vz^{(l-1)}}\ell = \left[
        \jac_{\vtheta^{(l)}}\vz^{(l)}\right]^{\top}\grad{\vz^{(l)}}\ell$\,.
    \item Free $\grad{\vz^{(l)}}\ell$ and send $\grad{\vz^{(l-1)}}\ell$ to layer
      $l-1$.
    \end{enumerate}
  \end{itemize}
\end{definition}
\Cref{def:background::GradientBackpropagation} is \emph{modular}, as mentioned
earlier in \Cref{sec:background::CommonOperations}: it only relies on local
derivatives. Supporting a new operation only requires specifying its forward
pass and the vector-Jacobian products (VJPs%
\sidenote[][-3.5\baselineskip]{During backpropagation, the transposed Jacobian is
  right-multiplied onto the backpropagated gradient vector
  (\Cref{eq:background::ParameterGradient} from right to left). One can see this
  as left-multiplying the Jacobian to a column vector
  (\Cref{eq:background::ParameterJacobian} from left to right), \ie computing a
  vector-Jacobian product. Forward accumulation instead left-multiplies the
  transposed Jacobian onto a matrix (\Cref{eq:background::ParameterGradient}
  from left to right). One can view this as right-multiplying the Jacobian onto
  the transposed matrix (\Cref{eq:background::ParameterJacobian} from right to
  left). This requires multiple Jacobian-vector products (JVPs), or a
  Jacobian-matrix product (JMP).}) %
with its input-output and parameter-output Jacobians. Given functionality to
create and traverse computation graphs, implementations of backpropagation are
very extensible due to its abstraction to the modular level.

Backpropagation itself performs a VJP with the network's parameter-output
Jacobian $\jac_{\vtheta} \vz^{(L)}$. Choosing the vector to be
$\grad{\vz^{(L)}}\ell$ yields the gradient $\grad{\vtheta} \ell =
[\jac_{\vtheta} \vz^{(L)}]^{\top} \grad{\vz^{(L)}}\ell$. But one can also
compute VJPs $[\jac_{\vtheta} \vz^{(L)}]^{\top} \vv$ with arbitrary vectors
$\vv$. The model's parameter-output Jacobian is crucial for computing
higher-order information (\Cref{chap:background::HigherOrder,part:papers}).

Although backpropagation only requires VJPs, the Jacobian matrix is an
interesting object for analytical studies into its structure, and for efficient
implementation of functionality that goes beyond computing gradients (\eg
matrix-Jacobian products (MJPs)). \Cref{tab:background::Jacobians} contains the
Jacobians of the common operations in neural networks from
\Cref{sec:background::CommonOperations}. These Jacobians are conveniently
obtained using matrix differential
calculus~\cite{magnus1999MatrixDifferentialCalculus}, presented in
\Cref{hbp::sec:matrixDifferentialCalculus}.

\begin{table*}[t]
  \centering
  \caption{\textbf{Jacobians (\Cref{hbp::def:generalizedJacobian}) for common
      modules in feedforward networks.} Input and output are denoted $\vx, \vz$
    rather than $\vz^{(l)}, \vz^{(l+1)}$ to avoid clutter. $\mI$ is the identity
    matrix. Matrices use bold upper-case symbols ($\mW, \mX, \mZ, \dots$),
    tensors use bold upper-case sans serif symbols ($\tW, \tX, \tZ, \dots$).
    Most Jacobians can be elegantly derived with matrix differential calculus,
    see \Cref{hbp::sec:matrixDifferentialCalculus} for
    details.}\label{tab:background::Jacobians}
  \begin{footnotesize}
    \begin{tabular}{llll}
      \toprule
      \textbf{OPERATION} & \textbf{FORWARD} & \textbf{JACOBIAN} (\Cref{hbp::def:generalizedJacobian}) & \textbf{DETAILS}
      \\
      \midrule
      % matrix-vector multiplication
      Matrix-vector multiplication & $\vz(\vx, \mW) = \mW\vx$ & $\jac_{\vx}\vz = \mW$\,,  & \Cref{hbp::subsec:linearLayerBackwardPass}
      \\
                         & & $\jac_{\mW} \vz = \vx^{\top} \otimes \mI$
      \\
      % matrix-matrix multiplication
      Matrix-matrix multiplication & $\mZ(\mX, \mW) = \mW\mX$ & $\jac_{\mX} \mZ = \mI \otimes
                                                                \mW$\,, & \Cref{hbp::subsec:linearLayerBackwardPass}
      \\
                         & & $\jac_{\mW} \mZ = \mX^\top \otimes \mI$
      \\
      % addition
      Addition & $\vz(\vx, \vb) = \vx + \vb$ & $\jac_{\vx}\vz = \jac_{\vb} \vz =\mI $ & \Cref{hbp::subsec:linearLayerBackwardPass}
      \\
      % elementwise activation
      Elementwise activation & $\vz(\vx) = \vphi(\vx)$\,,\ \text{s.t.} & $\jac_{\vx}\vz = \diag[\vphi'(\vx)]$ & \Cref{hbp::subsec:activationBackwardPass}
      \\
                         & $z_i(\vx) = \phi(x_i)$ &
      \\
      \midrule
      % residual unit/skip-connection
      Skip-connection & $\vz(\vx, \vtheta) = \vx + \vs(\vx, \vtheta)$ & $\jac_{\vx}\vz =
                                                                        \mI + \jac_{\vx}\vs$\,, & \Cref{hbp::subsec:skipconnectionBackwardPass}
      \\
                         & & $\jac_{\vtheta}\vz = \jac_{\vtheta} \vs $
      \\
      \midrule
      % reshape/view operation
      Reshape/view & $\tZ(\tX)=
                     \mathrm{reshape}(\tX)$ & $\jac_{\tX}\tZ = \mI$ & \Cref{hbp::subsec:HBPReshape}
      \\
      % extraction operator
      Index select/map $\pi$ & $\vz(\vx) = \mPi \vx\, ,$ $\emPi_{j,\pi(j)} =
                               1 $ & $\jac_{\vx}\vz = \mPi$ %(same as matrix-vector mul.\ with $W$)
                                                                                                      & \Cref{hbp::subsec:HBPIndexSelect}
      \\
      % convolution
      Convolution & $\tZ(\tX, \tW) = \tX
                    \star \tW$\,, & $\jac_{\llbracket \tX \rrbracket}\tZ =
                                    \mI \otimes \mW$\,,
                                                                                                      & \Cref{hbp::subsec:convolutions}
      \\
                         & $\mZ(\mW, \llbracket\tX\rrbracket) = \mW
                           \llbracket \tX \rrbracket$ & $\jac_{\mW} \tZ = \llbracket
                                                        \tX \rrbracket^\top \otimes \mI$
      \\
      \midrule
      % square loss
      Square loss & $\ell(\vf, \vy) = \nicefrac{1}{C}(\vy-\vf)^\top (\vy - \vf)$ & $\jac_{\vf}\ell = 2(\vf - \vy)^{\top}$ & \Cref{hbp::subsec:mselossBackwardPass}
      \\
      % cross-entropy loss
      Softmax cross-entropy & $\ell(\vf, y) = -\onehot(y)^{\top} \log[\vp(\vf)]$ & $\jac_{\vf}\ell = (\vy - \vp(\vf))^{\top}$ & \Cref{hbp::subsec:crossentropylossBackwardPass}
      \\
      \bottomrule
    \end{tabular}
  \end{footnotesize}
\end{table*}

\tikzexternalenable
\begin{figure*}[!t]
  \centering
  \begin{subfigure}[t]{1.0\linewidth}
    \caption{One datum}\label{subfig:background::gradientBackpropagation1}
    \centering\resizebox{\linewidth}{!}{
      {\footnotesize
        % basic setting of a fully-connected neural network with data flow for
% forward pass

\begin{tikzpicture}
  % first two layers
  \node (in1)
  [inner sep=0]
  {\tikz \drawMessagesWithArrows{$\vz^{(0)}$}{ }{ }{\hNodeDistance};};
  \node (layer1)
  [anchor=south west, inner sep=0]
  at (in1.south east)
  {\tikz \drawModuleWithParams{$f^{(1)}_{\vtheta^{(1)}}$}{16}{$\vtheta^{(1)}$}{$\grad{\vtheta^{(1)}}\ell$}{ };};
  \node (out1)
  [inner sep=0, anchor=south west]
  at (layer1.south east)
  {\tikz \drawMessagesWithArrows{$\vz^{(1)}$}{$\grad{\vz^{(1)}}\ell$}{ }{\hNodeDistance};};
  \node (layer2)
  [inner sep=0pt, anchor=south west]
  at (out1.south east)
  {\tikz \drawModuleWithParams{$f^{(2)}_{\vtheta^{(2)}}$}{16}{$\vtheta^{(2)}$}{$\grad{\vtheta^{(2)}}\ell$}{ };};

  % dots with messages
  \node (in2)
  [inner sep=0, anchor=south west]
  at (layer2.south east)
  {\tikz \drawMessagesWithArrows{$\vz^{(2)}$}{$\grad{\vz^{(2)}}\ell$}{ }{\hNodeDistance};};
  \node (dots)
  [xshift=0.75ex, inner sep=0pt, anchor=west]
  at (in2.east)
  {$\dots$};

  \node (inLast)
  [xshift=0.75ex, inner sep=0pt, anchor=west]
  at (dots.east)
  {\tikz \drawMessagesWithArrows{$\vz^{(L-1)}$}{$\grad{\vz^{(L-1)}}\ell$}{ }{\hNodeDistance};};

  \node (layerLast)
  [anchor=south west, inner sep=0]
  at (inLast.south east)
  {\tikz \drawModuleWithParams{$f^{(L)}_{\vtheta^{(L)}}$}{16}{$\vtheta^{(L)}$}{$\grad{\vtheta^{(L)}}\ell$}{ };};
  \node (outLast)
  [inner sep=0, anchor=south west]
  at (layerLast.south east)
  {\tikz \drawMessagesWithArrows{$\vz^{(L)}$}{$\grad{\vz^{(L)}}\ell$}{ }{\hNodeDistance};};

  % loss layer
  \node (lossLayer)
  [inner sep=0pt, anchor=south west]
  at (outLast.south east)
  {\tikz\drawModuleNoParams{$\ell$}{5};};
  \node (loss)
  [inner sep=0, anchor=south west]
  at (lossLayer.south east)
  {\tikz \drawMessagesWithArrows{$\ell$}{ }{ }{\hNodeDistance};};

  % label
  \node (label)
  [inner sep=0, anchor=south west, rotate=-90, xshift = -96, yshift = -50]
  at (lossLayer.south east)
  {\tikz \drawMessagesWithArrows{$\vy$}{ }{ }{\hNodeDistance};};
\end{tikzpicture}

%%% Local Variables:
%%% mode: latex
%%% TeX-master: "../../thesis"
%%% End:

      }}
  \end{subfigure}
  \begin{subfigure}[t]{1.0\linewidth}
    \caption{Batched data, arbitrary transformations}\label{subfig:background::gradientBackpropagation2}
    \centering\resizebox{\linewidth}{!}{
      {\footnotesize
        % basic setting of a fully-connected neural network with data flow for
% forward pass

\begin{tikzpicture}
  % first layer
  \node (out1)
  [inner sep=0, anchor=south west]
  {\tikz \drawMessagesWithArrows{$\mZ^{(0)}$}{ }{ }{\hNodeDistance};};
  \node (layer2)
  [inner sep=0pt, anchor=south west]
  at (out1.south east)
  {\tikz \drawModuleWithParams{$F^{(1)}_{\vtheta^{(1)}}$}{16}{$\vtheta^{(1)}$}{$\grad{\vtheta^{(1)}}\gL$}{ };};

  % dots with messages
  \node (in2)
  [inner sep=0, anchor=south west]
  at (layer2.south east)
  {\tikz \drawMessagesWithArrows{$\mZ^{(1)}$}{$\grad{\mZ^{(1)}}\gL$}{ }{\hNodeDistance};};
  \node (dots)
  [xshift=0.75ex, inner sep=0pt, anchor=west]
  at (in2.east)
  {$\dots$};

  \node (inLast)
  [xshift=0.75ex, inner sep=0pt, anchor=west]
  at (dots.east)
  {\tikz \drawMessagesWithArrows{$\mZ^{(L-1)}$}{$\grad{\mZ^{(L-1)}}\gL$}{ }{\hNodeDistance};};

  \node (layerLast)
  [anchor=south west, inner sep=0]
  at (inLast.south east)
  {\tikz \drawModuleWithParams{$F^{(L)}_{\vtheta^{(L)}}$}{16}{$\vtheta^{(L)}$}{$\grad{\vtheta^{(L)}}\gL$}{ };};
  \node (outLast)
  [inner sep=0, anchor=south west]
  at (layerLast.south east)
  {\tikz \drawMessagesWithArrows{$\mZ^{(L)}$}{$\grad{\mZ^{(L)}}\gL$}{ }{\hNodeDistance};};

  % loss layer
  \node (lossLayer)
  [inner sep=0pt, anchor=south west]
  at (outLast.south east)
  {\tikz\drawModuleNoParams{$\vell$}{5};};
  \node (losses)
  [inner sep=0, anchor=south west]
  at (lossLayer.south east)
  {\tikz \drawMessagesWithArrows{$\vell$}{$\grad{\vell}\gL$}{ }{\hNodeDistance};};

  % label
  \node (label)
  [inner sep=0, anchor=south west, rotate=-90, xshift = -96, yshift = -50]
  at (lossLayer.south east)
  {\tikz \drawMessagesWithArrows{$\mY$}{ }{ }{\hNodeDistance};};

  % Reduction layer
  \node (reductionLayer)
  [inner sep=0pt, anchor=south west]
  at (losses.south east)
  {\tikz\drawModuleNoParams{$\gL$}{5};};
  \node (reducedLoss)
  [inner sep=0, anchor=south west]
  at (reductionLayer.south east)
  {\tikz \drawMessagesWithArrows{$\gL$}{ }{ }{\hNodeDistance};};

\end{tikzpicture}

%%% Local Variables:
%%% mode: latex
%%% TeX-master: "../../thesis"
%%% End:

      }}
  \end{subfigure}
  \begin{subfigure}[t]{1.0\linewidth}
    \caption{Batched data, batched instructions}\label{subfig:background::gradientBackpropagation3}
    \centering\resizebox{\linewidth}{!}{
      {\footnotesize
        % basic setting of a fully-connected neural network with data flow for
% forward pass

\begin{tikzpicture}
  % first two layers
  % \node (in1)
  % [inner sep=0]
  % {\tikz \drawMessagesWithArrows{$\{\vz_{n}^{(0)}\}$}{ }{ }{\hNodeDistance};};
  % \node (layer1)
  % [anchor=south west, inner sep=0]
  % at (in1.south east)
  % {\tikz \drawModuleWithParams{$\vf^{(1)}_{\vtheta^{(1)}}$}{16}{$\vtheta^{(1)}$}{$\grad{\vtheta^{(1)}}\gL$}{ };};
  \node (out1)
  [inner sep=0, anchor=south west]
  % at (layer1.south east)
  {\tikz \drawMessagesWithArrows{\scalebox{0.8}{$\{\vz_{n}^{(0)}\}$}}{ }{ }{\hNodeDistance};};
  \node (layer2)
  [inner sep=0pt, anchor=south west]
  at (out1.south east)
  {\tikz \drawModuleWithParams{$\vf^{(1)}_{\vtheta^{(1)}}$}{16}{$\vtheta^{(1)}$}{$\grad{\vtheta^{(1)}}\gL$}{ };};

  % dots with messages
  \node (in2)
  [inner sep=0, anchor=south west]
  at (layer2.south east)
  {\tikz \drawMessagesWithArrows{\scalebox{0.8}{$\{\vz_{n}^{(2)}\}$}}{\scalebox{0.75}{$\{\grad{\vz_{n}^{(2)}}\gL\}$}}{ }{\hNodeDistance};};
  \node (dots)
  [xshift=0.75ex, inner sep=0pt, anchor=west]
  at (in2.east)
  {$\dots$};

  \node (inLast)
  [xshift=0.75ex, inner sep=0pt, anchor=west]
  at (dots.east)
  {\tikz \drawMessagesWithArrows{\,\scalebox{0.8}{$\{\vz_{n}^{(L-1)}\}$}\,}{\,\scalebox{0.75}{$\{\grad{\vz_{n}^{(L-1)}}\gL\}$}\,}{ }{\hNodeDistance};};

  \node (layerLast)
  [anchor=south west, inner sep=0]
  at (inLast.south east)
  {\tikz \drawModuleWithParams{$\vf^{(L)}_{\vtheta^{(L)}}$}{16}{$\vtheta^{(L)}$}{$\grad{\vtheta^{(L)}}\gL$}{ };};
  \node (outLast)
  [inner sep=0, anchor=south west]
  at (layerLast.south east)
  {\tikz \drawMessagesWithArrows{\scalebox{0.8}{$\{\vz_{n}^{(L)}\}$}}{\scalebox{0.75}{$\{\grad{\vz_{n}^{(L)}}\gL\}$}}{ }{\hNodeDistance};};

  % loss layer
  \node (lossLayer)
  [inner sep=0pt, anchor=south west]
  at (outLast.south east)
  {\tikz\drawModuleNoParams{$\vell$}{5};};
  \node (losses)
  [inner sep=0, anchor=south west]
  at (lossLayer.south east)
  {\tikz \drawMessagesWithArrows{\scalebox{1.0}{$\{\ell_n\}$}}{\scalebox{1.0}{$\{\grad{\ell_n}\gL\}$}}{ }{\hNodeDistance};};

  % label
  \node (label)
  [inner sep=0, anchor=south west, rotate=-90, xshift = -96, yshift = -50]
  at (lossLayer.south east)
  {\tikz \drawMessagesWithArrows{\scalebox{1.0}{$\{\vy_n\}$}}{ }{ }{\hNodeDistance};};

  % Reduction layer
  \node (reductionLayer)
  [inner sep=0pt, anchor=south west]
  at (losses.south east)
  {\tikz\drawModuleNoParams{$\gL$}{5};};
  \node (reducedLoss)
  [inner sep=0, anchor=south west]
  at (reductionLayer.south east)
  {\tikz \drawMessagesWithArrows{$\gL$}{ }{ }{\hNodeDistance};};

\end{tikzpicture}

%%% Local Variables:
%%% mode: latex
%%% TeX-master: "../../thesis"
%%% End:

      }}
  \end{subfigure}
  \caption{\textbf{Gradient backpropagation and (un)awareness of per-sample
      structure in many ML libraries.}
    \subfigref{subfig:background::gradientBackpropagation1} Computation graph of
    of a neural network's loss on a single datum $(\vx, \vy)$. Gradients are
    backpropagated through the graph as described by
    \Cref{def:background::GradientBackpropagation} to obtain
    $\grad{\vtheta}\ell$.
    \subfigref{subfig:background::gradientBackpropagation2} To exploit
    parallelism in the computations, multiple data are stacked into matrices
    $(\mX, \mY)$ which are then processed by a sequence of matrix-to-matrix
    functions $F_{\vtheta^{(l)}}^{(l)}$ into a batch of losses $\vell$, and
    reduced into a scalar $\gL$ via mean reduction. AD in popular ML libraries
    like \pytorch tracks variables on the level of batched tensors. Therefore,
    operations are allowed to build up dependencies between data---such that
    $\gL = \nicefrac{1}{|\sB|} \sum_{n} [\vell(\mX, \mY, \vtheta)]_n$ where each
    component of $\vell$ may depend on \emph{all} data (batch
    normalization~\cite{ioffe2015batch} is such a case)---without breaking
    gradient backpropagation. ML libraries implement VJPS for the
    matrix-to-matrix functions $F_{\vtheta^{(l)}}^{(l)}$. This loses structure
    for operations that treat inputs independently along the batch axis.
    \subfigref{subfig:background::gradientBackpropagation3} The empirical risk
    on a mini-batch (\Cref{eq:background::miniBatchRisk} is such a case: all
    operations in the graph process inputs independently and with the same
    instructions along the batch axis. The following connections to the single
    datum case \subfigref{subfig:background::gradientBackpropagation1} hold:
    $F_{\vtheta^{(l)}}^{(l)} \leftrightarrow \vf_{\vtheta^{(l)}}^{(l)} =
    \vmap(f_{\vtheta^{(l)}}^{(l)}), \vell = \vmap(\ell)$ with $\vmap$ from
    \Cref{def:background::vmap}. Due to the more general support of AD in ML
    libraries for graphs of the form
    \subfigref{subfig:background::gradientBackpropagation2}, their VJPs cannot
    be accessed per-sample.}\label{fig:background::gradientBackpropagation}
\end{figure*}

%%% Local Variables:
%%% mode: latex
%%% TeX-master: "../../thesis"
%%% End:

\tikzexternaldisable

\subsubsection{Backpropagation \& Batching}

To see the interplay between AD and batching
(\Cref{sec:background::MiniBatching}), consider differentiation of the
mini-batch loss (\Cref{eq:background::miniBatchRisk}). Popular ML libraries like
\PyTorch construct the computation graph on the level of tensor-valued variables
with an additional batch axis.

\paragraph{General case:} Starting from the previous differentiation of a single
datum loss, the mini-batch scenario follows by the substitutions
\begin{subequations}\label{eq:background::ADBatchedCase}
  \begin{align}\label{eq:background::ADBatchedCaseSubstitutions}
    \begin{split}
      \vx =: \vz^{(0)}
      \quad&\leftrightarrow\quad
             \mX =: \mZ^{(0)} \in \sR^{|\sB| \times h^{(0)}}\,,
      \\
      \vy
      \quad&\leftrightarrow\quad
             \mY \in \sR^{|\sB| \times C}\,,
      \\
      \vz^{(\ell)}
      \quad&\leftrightarrow\quad
             \mZ^{(l)} \in \sR^{|\sB| \times h^{(l)}}\,,
      \\
      \ell
      \quad&\leftrightarrow\quad
             \vell \in  \sR^{|\sB|}\,,
    \end{split}
  \end{align}
  with stacked data $\mX= (\vx_1\, \vx_2\, \dots\, \vx_{|\sB|})$ and $ \mY =
  (\vy_1\, \vy_2\, \dots\, \vy_{|\sB|})$, and assuming matrix-to-matrix layer
  functions. To account for the reduction of per-sample losses $\vell$, the
  graph is extended by
  \begin{align}
    \gL(\ell) = \mean(\ell) = \frac{1}{|\sB|} \sum_{n=1}^{|\sB|} [\vell]_n\,.
  \end{align}
  The computation graph, shown in
  \Cref{subfig:background::gradientBackpropagation2}, is $\gG = (\gE, \gV)$ with
  nodes
  \begin{align}
    \gV &=
          \left\{
          \vtheta^{(l)}
          \right\}_{l=1}^L
          \cup
          \left\{
          \mZ^{(l)}
          \right\}_{l=0}^L
          \cup
          \left\{
          \mY, \vell, \gL
          \right\}
          \shortintertext{and edges}
          \begin{split}
            \gE &=
                  \left\{
                  (\mZ^{(l-1)}, \mZ^{(l)}),
                  \right\}_{l=1}^L
                  \cup
                  \left\{
                  (\vtheta^{(l)}, \mZ^{(l)})
                  \right\}_{l=1}^{L}
            \\
                &\phantom{=}\,
                  \cup
                  \left\{
                  (\mZ^{(L)}, \vell),
                  (\mY, \vell),
                  (\vell, \gL)
                  \right\}\,.
          \end{split}
  \end{align}
\end{subequations}
Gradient backpropagation (\Cref{def:background::GradientBackpropagation})
carries over the mini-batch loss graph \Cref{eq:background::ADBatchedCase} and
efficiently recovers the gradient $\grad{\vtheta}\gL$.

Popular libraries like \pytorch implement the required functionality, VJPs, for
the matrix-to-matrix functions that process inputs with a batch axis (see
\Cref{hbp::def:generalizedJacobian} for the Jacobian's generaliation to matrix
functions). Hence, operations $\mZ^{(l)} \mapsto \mZ^{(l+1)}$ are allowed to
create dependencies across the batch axis without breaking the gradient
computation\sidenote[][-4\baselineskip]{%
  \Eg batch normalization \cite{ioffe2015batch} introduces
  dependencies along the batch dimension by centering and re-scaling the input
  with statistics computed across the batch axis.}%
.

\paragraph{Per-sample structure:} This work focuses on empirical risks that are
averages over \emph{per-sample} losses (recall
\Cref{eq:background::empiricalRisk}),
\begin{align*}
  \gL_{\sB}(\vtheta)
  =
  \frac{1}{|\sB|}
  \sum_{(\vx_n, \vy_n) \in \sB}
  \ell(f_{\vtheta}(\vx_n), \vy_n)\,.
\end{align*}
Hence, all operations act independently, and using the same instructions, along
the batch dimension. All variables in the graph inherit this independence along
their batch axis,
\begin{subequations}\label{eq:background::matrixNotationBatchedVariables}
  \begin{align}
    \mZ^{(l)}
    &=
      \begin{pmatrix}
        \vz_1^{(l)} & \vz_2^{(l)} & \dots & \vz_{|\sB|}^{(l)}
      \end{pmatrix}\,,\qquad l=0,\dots,L\,,
    \\
    \vell
    &=
      \begin{pmatrix}
        \ell(f_{\vtheta}(\vx_1), \vy_1)
        \\
        \ell(f_{\vtheta}(\vx_2), \vy_2)
        \\
        \vdots
        \\
        \ell(f_{\vtheta}(\vx_{|\sB|}), \vy_{|\sB|})
      \end{pmatrix}\,.
  \end{align}
  For all layers $l=1,\dots,L$, this independence across samples implies block-diagonal input-output Jacobians,
  \begin{align}
    \label{eq:background::perSampleStructureHiddenJacobian}
    \jac_{\mZ^{(l-1)}} \mZ^{(l)}
    =
    \begin{pmatrix}
      \jac_{\vz_1^{(l-1)}} \vz_1^{(l)} & 0 & \dots & 0
      \\
      0 & \jac_{\vz_2^{(l-1)}} \vz_2^{(l)} & \ddots & \vdots
      \\
      \vdots &  \ddots & \ddots & 0
      \\
      0 & \dots & 0 & \jac_{\vz_{|\sB|}^{(l-1)}} \vz_{|\sB|}^{(l)},
    \end{pmatrix}
  \end{align}
  and per-sample block structure in the output-parameter Jacobian
  \begin{align}
    \label{eq:background::perSampleStructureParameterJacobian}
    \jac_{\vtheta^{(l)}} \mZ^{(l)}
    =
    \begin{pmatrix}
      \jac_{\vtheta^{(l)}} \vz_1^{(l)}
      &
        \jac_{\vtheta^{(l)}} \vz_2^{(l)}
      &
        \dots
      &
        \jac_{\vtheta^{(l)}} \vz_{|\sB|}^{(l)},
    \end{pmatrix}\,.
  \end{align}
\end{subequations}
Many implementations of backpropagation make it difficult to access this
per-sample structure. They only expose VJPs $\vv \mapsto [\jac_{\mZ^{(l)}}
\mZ^{(l+1)}]^{\top} \vv$ and $\vv \mapsto [\jac_{\vtheta^{(l)}}
\mZ^{(l+1)}]^{\top} \vv$ for
\Cref{eq:background::perSampleStructureHiddenJacobian,eq:background::perSampleStructureParameterJacobian}.
While this allows supporting AD of more general graphs than those of an
empirical risk (\Cref{eq:background::empiricalRisk}, it limits access to only
the \emph{average} gradient
\begin{align}\label{eq:backround::miniBatchGradientNotation}
  \vg_{\sB}(\vtheta) :=
  \grad{\vtheta}\gL_{\sB}(\vtheta) =
  \frac{1}{|\sB|} \sum_{(\vx_n, \vy_n) \in \sB}
  \grad{\vtheta}\ell(f_{\vtheta}(\vx_n), \vy_n)
\end{align}
when differentiating an empirical risk. Computing per-sample gradients is not
more demanding than computing the average gradient---the only difference is
taking the average---but their computation is not supported. More flexible
access to the Jacobians
\Cref{eq:background::perSampleStructureHiddenJacobian,eq:background::perSampleStructureParameterJacobian}
through per-sample VJPs (or MJPs), enables the computation of various
higher-order quantities. Their efficient realization will be the main focus of
\Cref{chap:backpack,chap:hbp,chap:vivit}.

To highlight independence across the batch axis in a computation graph, a set
notation will be preferred over the matrix notation
(\Cref{eq:background::ADBatchedCaseSubstitutions,eq:background::matrixNotationBatchedVariables})
in the following. \Eg the text uses $\{ \vz_n^{(l)} \}_n$, or just $\{
\vz_n^{(l)} \}$, instead of $\mZ^{(l)}$.
\Cref{subfig:background::gradientBackpropagation3} illustrates this set
notation\sidenote[][-10.5\baselineskip]{%
  This notation is closer to recent developments in AD for ML
  libraries~\cite{bradbury2018jax,he2021functorch} that separate batching and AD
  more clearly through vectorization via a $\vmap$ interface
  (\Cref{def:background::vmap}). A different way to arrive at the batched
  computation graph in \Cref{subfig:background::gradientBackpropagation3} is to
  start from the computation graph of a single datum's loss
  $\ell(f_{\vtheta}(\vx), \vy)$ and vectorize it to obtain the set of graphs $
  \{\ell(f_{\vtheta}(\vx_n), \vy_n)\}$, which can be stacked into a single graph
  that produces $\vell$. Appending the reduction node to this graph, one obtains
  the computation graph for the average loss.}.

%%% Local Variables:
%%% mode: latex
%%% TeX-master: "../thesis"
%%% End:
