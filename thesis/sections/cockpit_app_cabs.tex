\subsection{\cabs Criterion: Coupling Adaptive Batch Sizes with Learning Rates
  (\robustInlinecode{CABS})}\label{cockpit::app:cabs}

The \cabs criterion, proposed by \citet{balles2017coupling}, can be used to adapt the
mini-batch size during training with \sgd. It relies on the gradient noise and
approximately optimizes the objective's expected gain per cost. The adaptation
rule is (with learning rate $\eta$)
\begin{equation}
  \label{cockpit::eq:cabs-theoretical}
  |\sB| \leftarrow \eta \frac{\Tr(\mSigma_{\pdata}(\vtheta))}{\gL_{{\pdata}}(\vtheta)}\,,
\end{equation}
and the practical implementation approximates $ \gL_{{\pdata}}(\vtheta) \approx
\gL_{\sB}(\vtheta)$and $\Tr(\mSigma_{\pdata}(\vtheta)) \approx
\nicefrac{(|\sB|-1)}{|\sB|} \Tr(\hat{\mSigma}_\sB(\vtheta))$ (compare equations
(10, 22) and first paragraph of Section 4 in \citep{balles2017coupling}). This
yields the quantity computed in \cockpit's \inlinecode{CABS} instrument,
\begin{align}
  \label{cockpit::eq:cabs-feature-table}
  |\sB| &\leftarrow \eta \frac{
          \frac{1}{|\sB|}
          \sum_{j=1}^D \sum_{n\in\sB}
          \left[
          \vg_n(\vtheta) - \vg_{\sB}(\vtheta)
          \right]_j^2
          }{
          \gL_{\sB}(\vtheta)
          }\,.
\end{align}

\subsubsection{Usage}

The \cabs criterion suggests a batch size which is optimal
under certain assumptions. This suggestion can support practitioners in the
batch size selection for their deep learning task.

%%% Local Variables:
%%% mode: latex
%%% TeX-master: "../thesis"
%%% End:
