\subsubsection{Checkpoints}

During training of the neural networks (see
\Cref{vivit::sec:training_of_nns}), we store a copy of the model (\ie the network's
current parameters) at specific checkpoints. This grid defines the temporal
resolution for all subsequent computations. Since training progresses much
faster in the early training stages, we use a log-grid with $100$ grid points
between $1$ and the number of training epochs and shift this grid by $-1$.

\subsubsection{Overlap}

Recall from \Cref{vivit::subsec:approx_quality}: for the set of orthonormal
eigenvectors $\{ \ve_c^\mU \}_{c=1}^C$ to the $C$ largest eigenvalues of some
symmetric matrix $\mU$, let $\mP^\mU = (\ve_1^\mU, \dots, \ve_C^\mU) (\ve_1^\mU,
\dots, \ve_C^\mU)^\top$. As in \cite{gurari2018gradient}, the overlap between two
subspaces $\mathcal{E}^\mU = \vecspan{}(\ve_1^\mU, \dots, \ve_C^\mU)$ and
$\mathcal{E}^\mV = \vecspan{}(\ve_1^\mV, \dots, \ve_C^\mV)$ of the matrices $\mU$
and $\mV$ is defined by
\begin{equation*}
  \text{overlap}(\mathcal{E}^\mU, \mathcal{E}^\mV)
  = \frac{\Tr{}(\mP^\mU \mP^\mV)}
  {\sqrt{\Tr{}(\mP^\mU) \Tr{}(\mP^\mV)}}
  \in [0, 1]\, .
\end{equation*}
%
The overlap can be computed efficiently by using the trace's cyclic property: it
holds $\Tr{}(\mP^\mU \mP^\mV) = \Tr{}(\mW^\top \mW)$ with $\mW = (\ve_1^\mU, \dots,
\ve_C^\mU)^\top (\ve_1^\mV, \dots, \ve_C^\mV) \in \mathbb{R}^{C \times C}$. Note
that this is a small $C \times C$ matrix, whereas $\mP^\mU, \mP^\mV \in
\mathbb{R}^{D \times D}$. Since
\begin{align*}
  \Tr{}(\mP^\mU)
  & = \Tr{}((\ve_1^\mU, \dots, \ve_C^\mU) (\ve_1^\mU, \dots, \ve_C^\mU)^\top) \\
  & = \Tr{}((\ve_1^\mU, \dots, \ve_C^\mU)^\top (\ve_1^\mU, \dots, \ve_C^\mU))
    \explainmath{(Cyclic property of trace)}                                 \\
  & = \Tr{}(\mI_C)
    \explainmath{(Orthonormality of the eigenvectors)}                       \\
  & = C
\end{align*}
(and analogous $\Tr{}(\mP^\mV) = C$), the denominator simplifies to $C$.

\subsubsection{Procedure}

For each checkpoint, we compute the top-$C$ eigenvalues and associated
eigenvectors of the full-batch \ggn and Hessian (\ie \ggn and Hessian are both
evaluated on the entire training set) using an iterative matrix-free approach.
We then compute the overlap between the top-$C$ eigenspaces as described above.
The eigspaces (\ie the top-$C$ eigenvalues and associated eigenvectors) are
stored on disk such that they can be used as a reference by subsequent
experiments.

\subsubsection{Results}

The results for all test problems are presented in
\Cref{vivit::fig:ggn_vs_hessian}. Except for a short phase at the beginning of
the optimization procedure (note the log scale for the epoch-axis), a strong
agreement (note the different limits for the overlap-axis) between the top-$C$
eigenspaces is observed. We make similar observations for all test problems, yet
to a slightly lesser extent for \cifarhun{}. A possible explanation for this
would be that the $100$-dimensional eigenspaces differ in the eigenvectors
associated with relatively small curvature. The corresponding eigenvalues
already transition into the bulk of the spectrum, where the "sharpness of
separation" decreases. However, since all directions are equally weighted in the
overlap, overall slightly lower values are obtained.
%
\begin{figure*}[p]
  \centering
  \begin{minipage}[t]{0.495\linewidth}
    \centering
    {\footnotesize\sgd}
  \end{minipage}\hfill
  \begin{minipage}[t]{0.495\linewidth}
    \centering
    {\footnotesize\adam}
  \end{minipage}

  \begin{subfigure}[t]{1.0\linewidth}
    \centering
    \caption{\fmnist \twoctwod}
    \begin{minipage}{0.50\linewidth}
      \centering
      % defines the pgfplots style "eigspacedefault"
\pgfkeys{/pgfplots/eigspacedefault/.style={
    width=1.0\linewidth,
    height=0.6\linewidth,
    every axis plot/.append style={line width = 1.5pt},
    tick pos = left,
    ylabel near ticks,
    xlabel near ticks,
    xtick align = inside,
    ytick align = inside,
    legend cell align = left,
    legend columns = 4,
    legend pos = south east,
    legend style = {
      fill opacity = 1,
      text opacity = 1,
      font = \footnotesize,
      at={(1, 1.025)},
      anchor=south east,
      column sep=0.25cm,
    },
    legend image post style={scale=2.5},
    xticklabel style = {font = \footnotesize},
    xlabel style = {font = \footnotesize},
    axis line style = {black},
    yticklabel style = {font = \footnotesize},
    ylabel style = {font = \footnotesize},
    title style = {font = \footnotesize},
    grid = major,
    grid style = {dashed}
  }
}

\pgfkeys{/pgfplots/eigspacedefaultapp/.style={
    eigspacedefault,
    height=0.6\linewidth,
    legend columns = 2,
  }
}

\pgfkeys{/pgfplots/eigspacenolegend/.style={
    legend image post style = {scale=0},
    legend style = {
      fill opacity = 0,
      draw opacity = 0,
      text opacity = 0,
      font = \footnotesize,
      at={(1, 1.025)},
      anchor=south east,
      column sep=0.25cm,
    },
  }
}
%%% Local Variables:
%%% mode: latex
%%% TeX-master: "../../thesis"
%%% End:

      \pgfkeys{/pgfplots/zmystyle/.style={
          eigspacedefault
        }}
      \tikzexternalenable
      % This file was created by tikzplotlib v0.9.7.
\begin{tikzpicture}

\definecolor{color0}{rgb}{0.145098039215686,0.490196078431373,0.349019607843137}

\begin{axis}[
axis line style={white!10!black},
log basis x={10},
tick pos=left,
xlabel={epoch (log scale)},
xmajorgrids,
xmin=0.794328234724281, xmax=125.892541179417,
xmode=log,
ylabel={overlap},
ymajorgrids,
ymin=0.718644052743912, ymax=1.01339790225029,
zmystyle
]
\addplot [, white!10!black, dashed]
table {%
0.794328234724281 1
125.892541179417 1
};
\addplot [, color0, mark=*, mark size=0.5, mark options={solid}, only marks]
table {%
1 0.873077690601349
1.04615384615385 0.732041954994202
1.0974358974359 0.842862904071808
1.14871794871795 0.956015467643738
1.2025641025641 0.86125260591507
1.26153846153846 0.776150465011597
1.32051282051282 0.94674426317215
1.38461538461538 0.759984731674194
1.44871794871795 0.81962251663208
1.51794871794872 0.94821959733963
1.58974358974359 0.871091246604919
1.66666666666667 0.959745109081268
1.74615384615385 0.960637748241425
1.82820512820513 0.963332533836365
1.91538461538462 0.911979377269745
2.00769230769231 0.889631152153015
2.1025641025641 0.990166962146759
2.20512820512821 0.856671035289764
2.30769230769231 0.950718760490417
2.41794871794872 0.915231823921204
2.53333333333333 0.980921626091003
2.65384615384615 0.890308201313019
2.78205128205128 0.938070893287659
2.91282051282051 0.985013604164124
3.05384615384615 0.92445707321167
3.1974358974359 0.909064292907715
3.35128205128205 0.87000036239624
3.51025641025641 0.900101482868195
3.67692307692308 0.95439350605011
3.85128205128205 0.907196640968323
4.03589743589744 0.963883280754089
4.22820512820513 0.957046687602997
4.42820512820513 0.91707170009613
4.64102564102564 0.985385715961456
4.86153846153846 0.908034920692444
5.09230769230769 0.967405438423157
5.33589743589744 0.959170997142792
5.58974358974359 0.913332760334015
5.85641025641026 0.968853175640106
6.13589743589744 0.954764246940613
6.42564102564103 0.943446040153503
6.73333333333333 0.969577014446259
7.05384615384615 0.935455620288849
7.38974358974359 0.974776446819305
7.74102564102564 0.947548508644104
8.11025641025641 0.956686794757843
8.4974358974359 0.870728135108948
8.9 0.978362858295441
9.32564102564103 0.97708797454834
9.76923076923077 0.985922038555145
10.2333333333333 0.994859397411346
10.7205128205128 0.946132063865662
11.2307692307692 0.939228713512421
11.7666666666667 0.960174083709717
12.3282051282051 0.975010395050049
12.9153846153846 0.966063797473907
13.5282051282051 0.927908599376678
14.174358974359 0.972076892852783
14.8487179487179 0.988854885101318
15.5564102564103 0.961053490638733
16.2974358974359 0.960028171539307
17.0717948717949 0.968356013298035
17.8846153846154 0.995735049247742
18.7358974358974 0.971260905265808
19.6282051282051 0.981684386730194
20.5641025641026 0.973080992698669
21.5435897435897 0.980144143104553
22.5692307692308 0.98397570848465
23.6435897435897 0.940322697162628
24.7692307692308 0.98731791973114
25.9487179487179 0.980996310710907
27.1846153846154 0.988605797290802
28.4794871794872 0.997254550457001
29.8358974358974 0.997567474842072
31.2564102564103 0.996417343616486
32.7435897435897 0.997253715991974
34.3025641025641 0.994132041931152
35.9358974358974 0.998004794120789
37.648717948718 0.997587859630585
39.4410256410256 0.998282134532928
41.3179487179487 0.998205363750458
43.2871794871795 0.995968163013458
45.3487179487179 0.998528361320496
47.5076923076923 0.9978888630867
49.7692307692308 0.981598019599915
52.1384615384615 0.998016059398651
54.6205128205128 0.998672127723694
57.2230769230769 0.998576462268829
59.9461538461538 0.998537719249725
62.8025641025641 0.99805611371994
65.7923076923077 0.998277962207794
68.925641025641 0.997998416423798
72.2076923076923 0.998399138450623
75.6461538461539 0.998551189899445
79.2461538461538 0.998721480369568
83.0205128205128 0.998946964740753
86.974358974359 0.998338520526886
91.1153846153846 0.998539626598358
95.4538461538462 0.997842907905579
100 0.998503386974335
};
\end{axis}

\end{tikzpicture}

      \tikzexternaldisable
    \end{minipage}\hfill
    \begin{minipage}{0.50\linewidth}
      \centering
      % defines the pgfplots style "eigspacedefault"
\pgfkeys{/pgfplots/eigspacedefault/.style={
    width=1.0\linewidth,
    height=0.6\linewidth,
    every axis plot/.append style={line width = 1.5pt},
    tick pos = left,
    ylabel near ticks,
    xlabel near ticks,
    xtick align = inside,
    ytick align = inside,
    legend cell align = left,
    legend columns = 4,
    legend pos = south east,
    legend style = {
      fill opacity = 1,
      text opacity = 1,
      font = \footnotesize,
      at={(1, 1.025)},
      anchor=south east,
      column sep=0.25cm,
    },
    legend image post style={scale=2.5},
    xticklabel style = {font = \footnotesize},
    xlabel style = {font = \footnotesize},
    axis line style = {black},
    yticklabel style = {font = \footnotesize},
    ylabel style = {font = \footnotesize},
    title style = {font = \footnotesize},
    grid = major,
    grid style = {dashed}
  }
}

\pgfkeys{/pgfplots/eigspacedefaultapp/.style={
    eigspacedefault,
    height=0.6\linewidth,
    legend columns = 2,
  }
}

\pgfkeys{/pgfplots/eigspacenolegend/.style={
    legend image post style = {scale=0},
    legend style = {
      fill opacity = 0,
      draw opacity = 0,
      text opacity = 0,
      font = \footnotesize,
      at={(1, 1.025)},
      anchor=south east,
      column sep=0.25cm,
    },
  }
}
%%% Local Variables:
%%% mode: latex
%%% TeX-master: "../../thesis"
%%% End:

      \pgfkeys{/pgfplots/zmystyle/.style={
          eigspacedefault
        }}
      \tikzexternalenable
      % This file was created by tikzplotlib v0.9.7.
\begin{tikzpicture}

\definecolor{color0}{rgb}{0.145098039215686,0.490196078431373,0.349019607843137}

\begin{axis}[
axis line style={white!10!black},
log basis x={10},
tick pos=left,
xlabel={epoch (log scale)},
xmajorgrids,
xmin=0.794328234724281, xmax=125.892541179417,
xmode=log,
ylabel={overlap},
ymajorgrids,
ymin=0.839661926031113, ymax=1.00763514637947,
zmystyle
]
\addplot [, white!10!black, dashed]
table {%
0.794328234724281 1
125.892541179417 1
};
\addplot [, color0, mark=*, mark size=0.5, mark options={solid}, only marks]
table {%
1 0.872092604637146
1.04615384615385 0.921695232391357
1.0974358974359 0.993611514568329
1.14871794871795 0.89234447479248
1.2025641025641 0.921713471412659
1.26153846153846 0.890910446643829
1.32051282051282 0.972811341285706
1.38461538461538 0.977183520793915
1.44871794871795 0.857341587543488
1.51794871794872 0.889195322990417
1.58974358974359 0.864228427410126
1.66666666666667 0.949607968330383
1.74615384615385 0.864488482475281
1.82820512820513 0.8835409283638
1.91538461538462 0.85186779499054
2.00769230769231 0.940789103507996
2.1025641025641 0.993580520153046
2.20512820512821 0.847297072410583
2.30769230769231 0.979656040668488
2.41794871794872 0.959355652332306
2.53333333333333 0.877789974212646
2.65384615384615 0.957183718681335
2.78205128205128 0.853241443634033
2.91282051282051 0.864687442779541
3.05384615384615 0.895600020885468
3.1974358974359 0.966269016265869
3.35128205128205 0.893738389015198
3.51025641025641 0.986423492431641
3.67692307692308 0.964503943920135
3.85128205128205 0.923362135887146
4.03589743589744 0.877373218536377
4.22820512820513 0.951965689659119
4.42820512820513 0.927387118339539
4.64102564102564 0.921734154224396
4.86153846153846 0.887033104896545
5.09230769230769 0.969671070575714
5.33589743589744 0.963532447814941
5.58974358974359 0.940945029258728
5.85641025641026 0.941979885101318
6.13589743589744 0.948404967784882
6.42564102564103 0.989803671836853
6.73333333333333 0.977945446968079
7.05384615384615 0.956406950950623
7.38974358974359 0.941308200359344
7.74102564102564 0.971573531627655
8.11025641025641 0.944472134113312
8.4974358974359 0.954324126243591
8.9 0.994396090507507
9.32564102564103 0.994179844856262
9.76923076923077 0.974213302135468
10.2333333333333 0.981061279773712
10.7205128205128 0.993523776531219
11.2307692307692 0.901409506797791
11.7666666666667 0.982514560222626
12.3282051282051 0.9706130027771
12.9153846153846 0.982797741889954
13.5282051282051 0.968455910682678
14.174358974359 0.986447036266327
14.8487179487179 0.995763957500458
15.5564102564103 0.996626079082489
16.2974358974359 0.996171772480011
17.0717948717949 0.997351944446564
17.8846153846154 0.980282604694366
18.7358974358974 0.963936805725098
19.6282051282051 0.986067950725555
20.5641025641026 0.995942234992981
21.5435897435897 0.95989465713501
22.5692307692308 0.943161845207214
23.6435897435897 0.949794113636017
24.7692307692308 0.992104411125183
25.9487179487179 0.960194885730743
27.1846153846154 0.989531397819519
28.4794871794872 0.988514721393585
29.8358974358974 0.973595142364502
31.2564102564103 0.976114094257355
32.7435897435897 0.942468345165253
34.3025641025641 0.957940936088562
35.9358974358974 0.990764617919922
37.648717948718 0.987808346748352
39.4410256410256 0.977329850196838
41.3179487179487 0.996530652046204
43.2871794871795 0.987970650196075
45.3487179487179 0.987531840801239
47.5076923076923 0.989563286304474
49.7692307692308 0.990059018135071
52.1384615384615 0.988878548145294
54.6205128205128 0.998126208782196
57.2230769230769 0.985849976539612
59.9461538461538 0.983811855316162
62.8025641025641 0.989851117134094
65.7923076923077 0.998559951782227
68.925641025641 0.993003964424133
72.2076923076923 0.995578110218048
75.6461538461539 0.997441947460175
79.2461538461538 0.995043158531189
83.0205128205128 0.993068873882294
86.974358974359 0.985715687274933
91.1153846153846 0.985586285591125
95.4538461538462 0.997935593128204
100 0.996270835399628
};
\end{axis}

\end{tikzpicture}

      \tikzexternaldisable
    \end{minipage}
  \end{subfigure}

  \begin{subfigure}[t]{1.0\linewidth}
    \centering
    \caption{\cifarten \threecthreed}
    \begin{minipage}{0.50\linewidth}
      \centering
      % defines the pgfplots style "eigspacedefault"
\pgfkeys{/pgfplots/eigspacedefault/.style={
    width=1.0\linewidth,
    height=0.6\linewidth,
    every axis plot/.append style={line width = 1.5pt},
    tick pos = left,
    ylabel near ticks,
    xlabel near ticks,
    xtick align = inside,
    ytick align = inside,
    legend cell align = left,
    legend columns = 4,
    legend pos = south east,
    legend style = {
      fill opacity = 1,
      text opacity = 1,
      font = \footnotesize,
      at={(1, 1.025)},
      anchor=south east,
      column sep=0.25cm,
    },
    legend image post style={scale=2.5},
    xticklabel style = {font = \footnotesize},
    xlabel style = {font = \footnotesize},
    axis line style = {black},
    yticklabel style = {font = \footnotesize},
    ylabel style = {font = \footnotesize},
    title style = {font = \footnotesize},
    grid = major,
    grid style = {dashed}
  }
}

\pgfkeys{/pgfplots/eigspacedefaultapp/.style={
    eigspacedefault,
    height=0.6\linewidth,
    legend columns = 2,
  }
}

\pgfkeys{/pgfplots/eigspacenolegend/.style={
    legend image post style = {scale=0},
    legend style = {
      fill opacity = 0,
      draw opacity = 0,
      text opacity = 0,
      font = \footnotesize,
      at={(1, 1.025)},
      anchor=south east,
      column sep=0.25cm,
    },
  }
}
%%% Local Variables:
%%% mode: latex
%%% TeX-master: "../../thesis"
%%% End:

      \pgfkeys{/pgfplots/zmystyle/.style={
          eigspacedefault
        }}
      \tikzexternalenable
      % This file was created by tikzplotlib v0.9.7.
\begin{tikzpicture}

\definecolor{color0}{rgb}{0.145098039215686,0.490196078431373,0.349019607843137}

\begin{axis}[
axis line style={white!10!black},
log basis x={10},
tick pos=left,
xlabel={epoch (log scale)},
xmajorgrids,
xmin=0.794328234724281, xmax=125.892541179417,
xmode=log,
ylabel={overlap},
ymajorgrids,
ymin=0.842878600955009, ymax=1.00748197138309,
zmystyle
]
\addplot [, white!10!black, dashed]
table {%
0.794328234724281 1
125.892541179417 1
};
\addplot [, color0, mark=*, mark size=0.5, mark options={solid}, only marks]
table {%
1 0.850360572338104
1.04487179487179 0.878133177757263
1.09615384615385 0.91087943315506
1.1474358974359 0.980224907398224
1.20192307692308 0.993440926074982
1.25961538461538 0.988232254981995
1.32051282051282 0.987772941589355
1.38461538461538 0.996818244457245
1.44871794871795 0.992499232292175
1.51923076923077 0.993051826953888
1.58974358974359 0.988804459571838
1.66666666666667 0.993102371692657
1.74679487179487 0.994531810283661
1.83012820512821 0.990985095500946
1.91666666666667 0.989441215991974
2.00641025641026 0.996042609214783
2.1025641025641 0.990116953849792
2.20512820512821 0.996223270893097
2.30769230769231 0.996023297309875
2.41987179487179 0.992021441459656
2.53525641025641 0.994040310382843
2.65384615384615 0.990503132343292
2.78205128205128 0.994559407234192
2.91346153846154 0.994334101676941
3.05128205128205 0.996053695678711
3.19871794871795 0.994520008563995
3.34935897435897 0.997268378734589
3.50961538461538 0.997389912605286
3.67628205128205 0.992188334465027
3.8525641025641 0.993352770805359
4.03525641025641 0.995678246021271
4.2275641025641 0.992665469646454
4.42948717948718 0.99213582277298
4.64102564102564 0.997660756111145
4.86217948717949 0.967168152332306
5.09294871794872 0.993785679340363
5.33653846153846 0.993943214416504
5.58974358974359 0.993542373180389
5.85576923076923 0.991654217243195
6.13461538461539 0.969390213489532
6.42628205128205 0.99034708738327
6.73397435897436 0.990933060646057
7.05448717948718 0.99559211730957
7.38782051282051 0.995241284370422
7.74038461538461 0.990916728973389
8.10897435897436 0.990530490875244
8.49679487179487 0.994798362255096
8.90064102564103 0.997474670410156
9.32371794871795 0.995593726634979
9.76923076923077 0.993783950805664
10.2339743589744 0.992836356163025
10.7211538461538 0.983698666095734
11.2307692307692 0.99790632724762
11.7660256410256 0.995324611663818
12.3269230769231 0.997409164905548
12.9134615384615 0.989908039569855
13.5288461538462 0.99628621339798
14.1730769230769 0.9957315325737
14.849358974359 0.996454894542694
15.5544871794872 0.997333228588104
16.2948717948718 0.994928538799286
17.0705128205128 0.989782631397247
17.8846153846154 0.997422814369202
18.7371794871795 0.997152030467987
19.6282051282051 0.99539452791214
20.5641025641026 0.995177149772644
21.5416666666667 0.993471741676331
22.5673076923077 0.994047284126282
23.6442307692308 0.997011065483093
24.7692307692308 0.99792355298996
25.9487179487179 0.995034217834473
27.1826923076923 0.994511604309082
28.4775641025641 0.996814131736755
29.8333333333333 0.99793940782547
31.2564102564103 0.997101902961731
32.7435897435897 0.995478749275208
34.3044871794872 0.99685537815094
35.9358974358974 0.997273564338684
37.6474358974359 0.994191527366638
39.4391025641026 0.99822723865509
41.3173076923077 0.995255470275879
43.2852564102564 0.996559143066406
45.3461538461538 0.992474675178528
47.5064102564103 0.994227528572083
49.7692307692308 0.998042583465576
52.1378205128205 0.996337890625
54.6217948717949 0.995224297046661
57.2211538461538 0.998927772045135
59.9455128205128 0.99811202287674
62.8012820512821 0.997424483299255
65.7916666666667 0.998414218425751
68.9230769230769 0.9983189702034
72.2051282051282 0.9963658452034
75.6442307692308 0.994187355041504
79.2467948717949 0.998039245605469
83.0192307692308 0.99797123670578
86.974358974359 0.996883273124695
91.1153846153846 0.997422218322754
95.4519230769231 0.995517551898956
100 0.998668372631073
};
\end{axis}

\end{tikzpicture}

      \tikzexternaldisable
    \end{minipage}\hfill
    \begin{minipage}{0.50\linewidth}
      \centering
      % defines the pgfplots style "eigspacedefault"
\pgfkeys{/pgfplots/eigspacedefault/.style={
    width=1.0\linewidth,
    height=0.6\linewidth,
    every axis plot/.append style={line width = 1.5pt},
    tick pos = left,
    ylabel near ticks,
    xlabel near ticks,
    xtick align = inside,
    ytick align = inside,
    legend cell align = left,
    legend columns = 4,
    legend pos = south east,
    legend style = {
      fill opacity = 1,
      text opacity = 1,
      font = \footnotesize,
      at={(1, 1.025)},
      anchor=south east,
      column sep=0.25cm,
    },
    legend image post style={scale=2.5},
    xticklabel style = {font = \footnotesize},
    xlabel style = {font = \footnotesize},
    axis line style = {black},
    yticklabel style = {font = \footnotesize},
    ylabel style = {font = \footnotesize},
    title style = {font = \footnotesize},
    grid = major,
    grid style = {dashed}
  }
}

\pgfkeys{/pgfplots/eigspacedefaultapp/.style={
    eigspacedefault,
    height=0.6\linewidth,
    legend columns = 2,
  }
}

\pgfkeys{/pgfplots/eigspacenolegend/.style={
    legend image post style = {scale=0},
    legend style = {
      fill opacity = 0,
      draw opacity = 0,
      text opacity = 0,
      font = \footnotesize,
      at={(1, 1.025)},
      anchor=south east,
      column sep=0.25cm,
    },
  }
}
%%% Local Variables:
%%% mode: latex
%%% TeX-master: "../../thesis"
%%% End:

      \pgfkeys{/pgfplots/zmystyle/.style={
          eigspacedefault
        }}
      \tikzexternalenable
      % This file was created by tikzplotlib v0.9.7.
\begin{tikzpicture}

\definecolor{color0}{rgb}{0.145098039215686,0.490196078431373,0.349019607843137}

\begin{axis}[
axis line style={white!10!black},
log basis x={10},
tick pos=left,
xlabel={epoch (log scale)},
xmajorgrids,
xmin=0.794328234724281, xmax=125.892541179417,
xmode=log,
ylabel={overlap},
ymajorgrids,
ymin=0.843578863143921, ymax=1.00744862556458,
zmystyle
]
\addplot [, white!10!black, dashed]
table {%
0.794328234724281 1
125.892541179417 1
};
\addplot [, color0, mark=*, mark size=0.5, mark options={solid}, only marks]
table {%
1 0.851027488708496
1.04487179487179 0.978357017040253
1.09615384615385 0.987865269184113
1.1474358974359 0.987098515033722
1.20192307692308 0.994354546070099
1.25961538461538 0.966632843017578
1.32051282051282 0.981844782829285
1.38461538461538 0.992616534233093
1.44871794871795 0.989068984985352
1.51923076923077 0.990721702575684
1.58974358974359 0.991576194763184
1.66666666666667 0.988918125629425
1.74679487179487 0.995641708374023
1.83012820512821 0.991575837135315
1.91666666666667 0.989520072937012
2.00641025641026 0.99610710144043
2.1025641025641 0.9941166639328
2.20512820512821 0.998058140277863
2.30769230769231 0.995820343494415
2.41987179487179 0.990787863731384
2.53525641025641 0.996366858482361
2.65384615384615 0.992045998573303
2.78205128205128 0.994105696678162
2.91346153846154 0.992032706737518
3.05128205128205 0.997964084148407
3.19871794871795 0.997196316719055
3.34935897435897 0.995748519897461
3.50961538461538 0.998512625694275
3.67628205128205 0.994609951972961
3.8525641025641 0.997748672962189
4.03525641025641 0.996739566326141
4.2275641025641 0.992980360984802
4.42948717948718 0.998505473136902
4.64102564102564 0.996007800102234
4.86217948717949 0.989669978618622
5.09294871794872 0.996848285198212
5.33653846153846 0.997749149799347
5.58974358974359 0.99856024980545
5.85576923076923 0.994812607765198
6.13461538461539 0.990816116333008
6.42628205128205 0.996141791343689
6.73397435897436 0.993492424488068
7.05448717948718 0.998186767101288
7.38782051282051 0.99757719039917
7.74038461538461 0.995778560638428
8.10897435897436 0.99692040681839
8.49679487179487 0.99719649553299
8.90064102564103 0.996037364006042
9.32371794871795 0.997768402099609
9.76923076923077 0.997484862804413
10.2339743589744 0.996788680553436
10.7211538461538 0.992404818534851
11.2307692307692 0.999168992042542
11.7660256410256 0.998628318309784
12.3269230769231 0.99838387966156
12.9134615384615 0.994900226593018
13.5288461538462 0.996962189674377
14.1730769230769 0.997451484203339
14.849358974359 0.967346847057343
15.5544871794872 0.998680472373962
16.2948717948718 0.998343288898468
17.0705128205128 0.998154997825623
17.8846153846154 0.998211741447449
18.7371794871795 0.998424351215363
19.6282051282051 0.99867308139801
20.5641025641026 0.997610867023468
21.5416666666667 0.998234272003174
22.5673076923077 0.998754024505615
23.6442307692308 0.999504864215851
24.7692307692308 0.999298751354218
25.9487179487179 0.998758316040039
27.1826923076923 0.998632788658142
28.4775641025641 0.998208701610565
29.8333333333333 0.998665988445282
31.2564102564103 0.997435390949249
32.7435897435897 0.999235808849335
34.3044871794872 0.998749434947968
35.9358974358974 0.99847537279129
37.6474358974359 0.99878466129303
39.4391025641026 0.998607993125916
41.3173076923077 0.997779965400696
43.2852564102564 0.998919129371643
45.3461538461538 0.999670386314392
47.5064102564103 0.998722076416016
49.7692307692308 0.998940110206604
52.1378205128205 0.999525666236877
54.6217948717949 0.998592674732208
57.2211538461538 0.977607846260071
59.9455128205128 0.999204277992249
62.8012820512821 0.998785674571991
65.7916666666667 0.999239921569824
68.9230769230769 0.996752738952637
72.2051282051282 0.998494505882263
75.6442307692308 0.998771071434021
79.2467948717949 0.999237537384033
83.0192307692308 0.998167157173157
86.974358974359 0.999243438243866
91.1153846153846 0.998964488506317
95.4519230769231 0.999459743499756
100 0.997690081596375
};
\end{axis}

\end{tikzpicture}

      \tikzexternaldisable
    \end{minipage}
  \end{subfigure}

  \begin{subfigure}[t]{1.0\linewidth}
    \centering
    \caption{\cifarten \resnetthirtytwo}
    \begin{minipage}{0.50\linewidth}
      \centering
      % defines the pgfplots style "eigspacedefault"
\pgfkeys{/pgfplots/eigspacedefault/.style={
    width=1.0\linewidth,
    height=0.6\linewidth,
    every axis plot/.append style={line width = 1.5pt},
    tick pos = left,
    ylabel near ticks,
    xlabel near ticks,
    xtick align = inside,
    ytick align = inside,
    legend cell align = left,
    legend columns = 4,
    legend pos = south east,
    legend style = {
      fill opacity = 1,
      text opacity = 1,
      font = \footnotesize,
      at={(1, 1.025)},
      anchor=south east,
      column sep=0.25cm,
    },
    legend image post style={scale=2.5},
    xticklabel style = {font = \footnotesize},
    xlabel style = {font = \footnotesize},
    axis line style = {black},
    yticklabel style = {font = \footnotesize},
    ylabel style = {font = \footnotesize},
    title style = {font = \footnotesize},
    grid = major,
    grid style = {dashed}
  }
}

\pgfkeys{/pgfplots/eigspacedefaultapp/.style={
    eigspacedefault,
    height=0.6\linewidth,
    legend columns = 2,
  }
}

\pgfkeys{/pgfplots/eigspacenolegend/.style={
    legend image post style = {scale=0},
    legend style = {
      fill opacity = 0,
      draw opacity = 0,
      text opacity = 0,
      font = \footnotesize,
      at={(1, 1.025)},
      anchor=south east,
      column sep=0.25cm,
    },
  }
}
%%% Local Variables:
%%% mode: latex
%%% TeX-master: "../../thesis"
%%% End:

      \pgfkeys{/pgfplots/zmystyle/.style={
          eigspacedefault
        }}
      \tikzexternalenable
      % This file was created by tikzplotlib v0.9.7.
\begin{tikzpicture}

\definecolor{color0}{rgb}{0.145098039215686,0.490196078431373,0.349019607843137}

\begin{axis}[
axis line style={white!10!black},
log basis x={10},
tick pos=left,
xlabel={epoch (log scale)},
xmajorgrids,
xmin=0.771323165184619, xmax=233.365219825747,
xmode=log,
ylabel={overlap},
ymajorgrids,
ymin=0.0296707909554243, ymax=1.04620615281165,
zmystyle
]
\addplot [, white!10!black, dashed]
table {%
0.771323165184619 1
233.365219825748 1
};
\addplot [, color0, mark=*, mark size=0.5, mark options={solid}, only marks]
table {%
1 0.0758769437670708
1.05128205128205 0.175527721643448
1.10897435897436 0.562887907028198
1.16987179487179 0.511466145515442
1.23076923076923 0.850968480110168
1.29807692307692 0.567541241645813
1.36858974358974 0.877959251403809
1.44230769230769 0.693752884864807
1.51923076923077 0.950367569923401
1.6025641025641 0.898041725158691
1.68910256410256 0.918483376502991
1.77884615384615 0.835304915904999
1.875 0.664470791816711
1.9775641025641 0.892116844654083
2.08333333333333 0.963935732841492
2.19551282051282 0.909787058830261
2.31410256410256 0.764615833759308
2.43910256410256 0.868504226207733
2.57051282051282 0.911486268043518
2.70833333333333 0.852096736431122
2.8525641025641 0.823445439338684
3.00641025641026 0.919346034526825
3.16987179487179 0.976591944694519
3.33974358974359 0.93577516078949
3.51923076923077 0.881943702697754
3.70833333333333 0.880431354045868
3.91025641025641 0.966978847980499
4.11858974358974 0.938574492931366
4.34294871794872 0.947316646575928
4.57692307692308 0.98882257938385
4.82371794871795 0.96997058391571
5.08333333333333 0.921084880828857
5.35576923076923 0.975445747375488
5.64423076923077 0.987588882446289
5.94871794871795 0.935854434967041
6.26923076923077 0.958848834037781
6.60576923076923 0.985344231128693
6.96153846153846 0.961620151996613
7.33653846153846 0.981544852256775
7.73397435897436 0.972874164581299
8.15064102564103 0.969390213489532
8.58974358974359 0.978043079376221
9.05128205128205 0.979115784168243
9.53846153846154 0.983906388282776
10.0512820512821 0.991364777088165
10.5929487179487 0.979225158691406
11.1634615384615 0.991371989250183
11.7660256410256 0.98114138841629
12.400641025641 0.989734351634979
13.0673076923077 0.989019572734833
13.7724358974359 0.987693428993225
14.5128205128205 0.99201762676239
15.2948717948718 0.975343883037567
16.1185897435897 0.980695247650146
16.9871794871795 0.978762447834015
17.900641025641 0.994581997394562
18.8653846153846 0.983127415180206
19.8814102564103 0.993127226829529
20.9519230769231 0.981705486774445
22.0801282051282 0.985893905162811
23.2692307692308 0.959465801715851
24.5224358974359 0.960145115852356
25.8429487179487 0.994009017944336
27.2371794871795 0.987131714820862
28.7019230769231 0.99342554807663
30.25 0.99464225769043
31.8782051282051 0.991647720336914
33.5961538461538 0.99458247423172
35.4038461538462 0.990595996379852
37.3108974358974 0.969861507415771
39.3205128205128 0.99425995349884
41.4391025641026 0.98828113079071
43.6698717948718 0.991572499275208
46.0224358974359 0.994888424873352
48.5 0.986674964427948
51.1121794871795 0.99597579240799
53.8653846153846 0.995907187461853
56.7660256410256 0.990193963050842
59.8237179487179 0.997666537761688
63.0448717948718 0.996249079704285
66.4391025641026 0.985042572021484
70.0192307692308 0.986966490745544
73.7884615384615 0.993395507335663
77.7628205128205 0.996701717376709
81.9519230769231 0.986942648887634
86.3653846153846 0.998145699501038
91.0160256410256 0.997241377830505
95.9166666666667 0.996314823627472
101.083333333333 0.992465972900391
106.525641025641 0.992295265197754
112.262820512821 0.997797608375549
118.310897435897 0.99855774641037
124.682692307692 0.99132376909256
131.397435897436 0.995581269264221
138.474358974359 0.995331764221191
145.929487179487 0.992706596851349
153.788461538462 0.997922539710999
162.070512820513 0.997540175914764
170.801282051282 0.998306453227997
180 0.996681332588196
};
\end{axis}

\end{tikzpicture}

      \tikzexternaldisable
    \end{minipage}\hfill
    \begin{minipage}{0.50\linewidth}
      \centering
      % defines the pgfplots style "eigspacedefault"
\pgfkeys{/pgfplots/eigspacedefault/.style={
    width=1.0\linewidth,
    height=0.6\linewidth,
    every axis plot/.append style={line width = 1.5pt},
    tick pos = left,
    ylabel near ticks,
    xlabel near ticks,
    xtick align = inside,
    ytick align = inside,
    legend cell align = left,
    legend columns = 4,
    legend pos = south east,
    legend style = {
      fill opacity = 1,
      text opacity = 1,
      font = \footnotesize,
      at={(1, 1.025)},
      anchor=south east,
      column sep=0.25cm,
    },
    legend image post style={scale=2.5},
    xticklabel style = {font = \footnotesize},
    xlabel style = {font = \footnotesize},
    axis line style = {black},
    yticklabel style = {font = \footnotesize},
    ylabel style = {font = \footnotesize},
    title style = {font = \footnotesize},
    grid = major,
    grid style = {dashed}
  }
}

\pgfkeys{/pgfplots/eigspacedefaultapp/.style={
    eigspacedefault,
    height=0.6\linewidth,
    legend columns = 2,
  }
}

\pgfkeys{/pgfplots/eigspacenolegend/.style={
    legend image post style = {scale=0},
    legend style = {
      fill opacity = 0,
      draw opacity = 0,
      text opacity = 0,
      font = \footnotesize,
      at={(1, 1.025)},
      anchor=south east,
      column sep=0.25cm,
    },
  }
}
%%% Local Variables:
%%% mode: latex
%%% TeX-master: "../../thesis"
%%% End:

      \pgfkeys{/pgfplots/zmystyle/.style={
          eigspacedefault
        }}
      \tikzexternalenable
      % This file was created by tikzplotlib v0.9.7.
\begin{tikzpicture}

\definecolor{color0}{rgb}{0.145098039215686,0.490196078431373,0.349019607843137}

\begin{axis}[
axis line style={white!10!black},
log basis x={10},
tick pos=left,
xlabel={epoch (log scale)},
xmajorgrids,
xmin=0.771323165184619, xmax=233.365219825747,
xmode=log,
ylabel={overlap},
ymajorgrids,
ymin=0.0241242062300444, ymax=1.04647027589381,
zmystyle
]
\addplot [, white!10!black, dashed]
table {%
0.771323165184619 1
233.365219825748 1
};
\addplot [, color0, mark=*, mark size=0.5, mark options={solid}, only marks]
table {%
1 0.0705944821238518
1.05128205128205 0.95068347454071
1.10897435897436 0.856163024902344
1.16987179487179 0.991746246814728
1.23076923076923 0.735004723072052
1.29807692307692 0.968465685844421
1.36858974358974 0.943033814430237
1.44230769230769 0.954832375049591
1.51923076923077 0.991986453533173
1.6025641025641 0.984900951385498
1.68910256410256 0.997577965259552
1.77884615384615 0.903916954994202
1.875 0.989762306213379
1.9775641025641 0.924025356769562
2.08333333333333 0.991471171379089
2.19551282051282 0.676475584506989
2.31410256410256 0.889772891998291
2.43910256410256 0.971712291240692
2.57051282051282 0.880101203918457
2.70833333333333 0.98236072063446
2.8525641025641 0.924176216125488
3.00641025641026 0.862906098365784
3.16987179487179 0.948005199432373
3.33974358974359 0.912280678749084
3.51923076923077 0.986277222633362
3.70833333333333 0.883581757545471
3.91025641025641 0.983112156391144
4.11858974358974 0.986141383647919
4.34294871794872 0.977289378643036
4.57692307692308 0.969769775867462
4.82371794871795 0.978937327861786
5.08333333333333 0.996120572090149
5.35576923076923 0.991199374198914
5.64423076923077 0.994466602802277
5.94871794871795 0.968710064888
6.26923076923077 0.975476920604706
6.60576923076923 0.993090748786926
6.96153846153846 0.987621963024139
7.33653846153846 0.994208157062531
7.73397435897436 0.96600615978241
8.15064102564103 0.957184672355652
8.58974358974359 0.986193299293518
9.05128205128205 0.993752837181091
9.53846153846154 0.982730031013489
10.0512820512821 0.98665463924408
10.5929487179487 0.993539214134216
11.1634615384615 0.976911425590515
11.7660256410256 0.990206837654114
12.400641025641 0.996215999126434
13.0673076923077 0.995976805686951
13.7724358974359 0.988674640655518
14.5128205128205 0.988697171211243
15.2948717948718 0.99525785446167
16.1185897435897 0.990448951721191
16.9871794871795 0.995240390300751
17.900641025641 0.994033515453339
18.8653846153846 0.983682453632355
19.8814102564103 0.992716491222382
20.9519230769231 0.99590003490448
22.0801282051282 0.983650386333466
23.2692307692308 0.992737591266632
24.5224358974359 0.995701432228088
25.8429487179487 0.995270729064941
27.2371794871795 0.993540585041046
28.7019230769231 0.995123267173767
30.25 0.996706008911133
31.8782051282051 0.995969295501709
33.5961538461538 0.9931640625
35.4038461538462 0.997571289539337
37.3108974358974 0.993003070354462
39.3205128205128 0.996488749980927
41.4391025641026 0.993738174438477
43.6698717948718 0.989822745323181
46.0224358974359 0.99801242351532
48.5 0.994913280010223
51.1121794871795 0.987682342529297
53.8653846153846 0.996215045452118
56.7660256410256 0.998407661914825
59.8237179487179 0.995986640453339
63.0448717948718 0.99789172410965
66.4391025641026 0.999024033546448
70.0192307692308 0.992345452308655
73.7884615384615 0.996231436729431
77.7628205128205 0.997561931610107
81.9519230769231 0.995594382286072
86.3653846153846 0.99841320514679
91.0160256410256 0.997795283794403
95.9166666666667 0.99911892414093
101.083333333333 0.996722877025604
106.525641025641 0.99618923664093
112.262820512821 0.997033476829529
118.310897435897 0.996328055858612
124.682692307692 0.998886704444885
131.397435897436 0.995704531669617
138.474358974359 0.998238742351532
145.929487179487 0.997261166572571
153.788461538462 0.999082684516907
162.070512820513 0.997128963470459
170.801282051282 0.9985671043396
180 0.998507678508759
};
\end{axis}

\end{tikzpicture}

      \tikzexternaldisable
    \end{minipage}
  \end{subfigure}

  \begin{subfigure}[t]{1.0\linewidth}
    \centering
    \caption{\cifarhun \allcnnc}
    \begin{minipage}{0.50\linewidth}
      \centering
      % defines the pgfplots style "eigspacedefault"
\pgfkeys{/pgfplots/eigspacedefault/.style={
    width=1.0\linewidth,
    height=0.6\linewidth,
    every axis plot/.append style={line width = 1.5pt},
    tick pos = left,
    ylabel near ticks,
    xlabel near ticks,
    xtick align = inside,
    ytick align = inside,
    legend cell align = left,
    legend columns = 4,
    legend pos = south east,
    legend style = {
      fill opacity = 1,
      text opacity = 1,
      font = \footnotesize,
      at={(1, 1.025)},
      anchor=south east,
      column sep=0.25cm,
    },
    legend image post style={scale=2.5},
    xticklabel style = {font = \footnotesize},
    xlabel style = {font = \footnotesize},
    axis line style = {black},
    yticklabel style = {font = \footnotesize},
    ylabel style = {font = \footnotesize},
    title style = {font = \footnotesize},
    grid = major,
    grid style = {dashed}
  }
}

\pgfkeys{/pgfplots/eigspacedefaultapp/.style={
    eigspacedefault,
    height=0.6\linewidth,
    legend columns = 2,
  }
}

\pgfkeys{/pgfplots/eigspacenolegend/.style={
    legend image post style = {scale=0},
    legend style = {
      fill opacity = 0,
      draw opacity = 0,
      text opacity = 0,
      font = \footnotesize,
      at={(1, 1.025)},
      anchor=south east,
      column sep=0.25cm,
    },
  }
}
%%% Local Variables:
%%% mode: latex
%%% TeX-master: "../../thesis"
%%% End:

      \pgfkeys{/pgfplots/zmystyle/.style={
          eigspacedefault
        }}
      \tikzexternalenable
      % This file was created by tikzplotlib v0.9.7.
\begin{tikzpicture}

\definecolor{color0}{rgb}{0.145098039215686,0.490196078431373,0.349019607843137}

\begin{axis}[
axis line style={white!10!black},
log basis x={10},
tick pos=left,
xlabel={epoch (log scale)},
xmajorgrids,
xmin=0.746099240306814, xmax=469.106495613199,
xmode=log,
ylabel={overlap},
ymajorgrids,
ymin=0.115845464915037, ymax=1.04210259690881,
zmystyle
]
\addplot [, white!10!black, dashed]
table {%
0.746099240306814 1
469.106495613199 1
};
\addplot [, color0, mark=*, mark size=0.5, mark options={solid}, only marks]
table {%
1 0.386417299509048
1.05769230769231 0.418932259082794
1.12179487179487 0.429068058729172
1.19230769230769 0.340360909700394
1.26282051282051 0.296373277902603
1.33974358974359 0.296169996261597
1.42307692307692 0.389026373624802
1.51282051282051 0.309975534677505
1.6025641025641 0.454967796802521
1.69871794871795 0.20670086145401
1.80128205128205 0.261316418647766
1.91666666666667 0.549367070198059
2.03205128205128 0.28644073009491
2.15384615384615 0.490236669778824
2.28846153846154 0.632620692253113
2.42307692307692 0.440548866987228
2.57692307692308 0.308936446905136
2.73076923076923 0.223712116479874
2.8974358974359 0.279196500778198
3.07692307692308 0.462846338748932
3.26282051282051 0.404943108558655
3.46153846153846 0.388403534889221
3.67307692307692 0.338918030261993
3.8974358974359 0.157948061823845
4.13461538461539 0.496893316507339
4.38461538461539 0.501740694046021
4.65384615384615 0.444027781486511
4.93589743589744 0.692538499832153
5.23717948717949 0.534952104091644
5.55769230769231 0.633862257003784
5.8974358974359 0.374679714441299
6.25641025641026 0.482461273670197
6.64102564102564 0.492965340614319
7.04487179487179 0.584171652793884
7.47435897435897 0.888950526714325
7.92948717948718 0.755341410636902
8.41025641025641 0.721909046173096
8.92307692307692 0.699930787086487
9.46794871794872 0.866701900959015
10.0448717948718 0.88215696811676
10.6602564102564 0.764813363552094
11.3076923076923 0.811613917350769
12 0.779640436172485
12.7307692307692 0.775027513504028
13.5064102564103 0.825958073139191
14.3333333333333 0.878829419612885
15.2051282051282 0.699858546257019
16.1346153846154 0.92185240983963
17.1153846153846 0.817578375339508
18.1602564102564 0.853261888027191
19.2692307692308 0.841851353645325
20.4423076923077 0.861587882041931
21.6858974358974 0.911116659641266
23.0128205128205 0.860004365444183
24.4102564102564 0.883511126041412
25.9038461538462 0.894382536411285
27.4807692307692 0.902079939842224
29.1538461538462 0.890643715858459
30.9358974358974 0.915258228778839
32.8205128205128 0.945243656635284
34.8205128205128 0.937311768531799
36.9423076923077 0.937949597835541
39.1923076923077 0.822693228721619
41.5833333333333 0.912825703620911
44.1153846153846 0.928557336330414
46.8076923076923 0.895105600357056
49.6602564102564 0.946525037288666
52.6858974358974 0.953090846538544
55.8974358974359 0.949208736419678
59.3076923076923 0.936187982559204
62.9230769230769 0.740674793720245
66.7564102564103 0.93524581193924
70.8269230769231 0.90137130022049
75.1474358974359 0.943356931209564
79.7307692307692 0.915778517723083
84.5897435897436 0.94258987903595
89.7435897435897 0.960110187530518
95.2179487179487 0.959887623786926
101.019230769231 0.959967136383057
107.179487179487 0.963168025016785
113.711538461538 0.957749605178833
120.641025641026 0.963780760765076
127.99358974359 0.934285581111908
135.801282051282 0.919079124927521
144.076923076923 0.915252923965454
152.858974358974 0.956082105636597
162.179487179487 0.954036951065063
172.064102564103 0.964087128639221
182.551282051282 0.951638579368591
193.679487179487 0.977284550666809
205.487179487179 0.963100731372833
218.012820512821 0.965840578079224
231.301282051282 0.976486444473267
245.403846153846 0.971389770507812
260.358974358974 0.955632388591766
276.230769230769 0.953461706638336
293.070512820513 0.963158845901489
310.935897435897 0.963678121566772
329.884615384615 0.943393766880035
350 0.931904077529907
};
\end{axis}

\end{tikzpicture}

      \tikzexternaldisable
    \end{minipage}\hfill
    \begin{minipage}{0.50\linewidth}
      \centering
      % defines the pgfplots style "eigspacedefault"
\pgfkeys{/pgfplots/eigspacedefault/.style={
    width=1.0\linewidth,
    height=0.6\linewidth,
    every axis plot/.append style={line width = 1.5pt},
    tick pos = left,
    ylabel near ticks,
    xlabel near ticks,
    xtick align = inside,
    ytick align = inside,
    legend cell align = left,
    legend columns = 4,
    legend pos = south east,
    legend style = {
      fill opacity = 1,
      text opacity = 1,
      font = \footnotesize,
      at={(1, 1.025)},
      anchor=south east,
      column sep=0.25cm,
    },
    legend image post style={scale=2.5},
    xticklabel style = {font = \footnotesize},
    xlabel style = {font = \footnotesize},
    axis line style = {black},
    yticklabel style = {font = \footnotesize},
    ylabel style = {font = \footnotesize},
    title style = {font = \footnotesize},
    grid = major,
    grid style = {dashed}
  }
}

\pgfkeys{/pgfplots/eigspacedefaultapp/.style={
    eigspacedefault,
    height=0.6\linewidth,
    legend columns = 2,
  }
}

\pgfkeys{/pgfplots/eigspacenolegend/.style={
    legend image post style = {scale=0},
    legend style = {
      fill opacity = 0,
      draw opacity = 0,
      text opacity = 0,
      font = \footnotesize,
      at={(1, 1.025)},
      anchor=south east,
      column sep=0.25cm,
    },
  }
}
%%% Local Variables:
%%% mode: latex
%%% TeX-master: "../../thesis"
%%% End:

      \pgfkeys{/pgfplots/zmystyle/.style={
          eigspacedefault
        }}
      \tikzexternalenable
      % This file was created by tikzplotlib v0.9.7.
\begin{tikzpicture}

\definecolor{color0}{rgb}{0.145098039215686,0.490196078431373,0.349019607843137}

\begin{axis}[
axis line style={white!10!black},
log basis x={10},
tick pos=left,
xlabel={epoch (log scale)},
xmajorgrids,
xmin=0.746099240306814, xmax=469.106495613199,
xmode=log,
ylabel={overlap},
ymajorgrids,
ymin=0.218809682130814, ymax=1.03719953894615,
zmystyle
]
\addplot [, white!10!black, dashed]
table {%
0.746099240306814 1
469.106495613199 1
};
\addplot [, color0, mark=*, mark size=0.5, mark options={solid}, only marks]
table {%
1 0.380998522043228
1.05769230769231 0.432481735944748
1.12179487179487 0.486287951469421
1.19230769230769 0.317287772893906
1.26282051282051 0.444183588027954
1.33974358974359 0.469926327466965
1.42307692307692 0.472592443227768
1.51282051282051 0.569239437580109
1.6025641025641 0.671595931053162
1.69871794871795 0.571328401565552
1.80128205128205 0.343975841999054
1.91666666666667 0.256009221076965
2.03205128205128 0.481311082839966
2.15384615384615 0.302192121744156
2.28846153846154 0.510162830352783
2.42307692307692 0.90375167131424
2.57692307692308 0.512550413608551
2.73076923076923 0.797624886035919
2.8974358974359 0.606736123561859
3.07692307692308 0.752561748027802
3.26282051282051 0.524358808994293
3.46153846153846 0.95850944519043
3.67307692307692 0.724418461322784
3.8974358974359 0.656598806381226
4.13461538461539 0.926303088665009
4.38461538461539 0.873789608478546
4.65384615384615 0.827855825424194
4.93589743589744 0.971795260906219
5.23717948717949 0.771609902381897
5.55769230769231 0.853511810302734
5.8974358974359 0.653037667274475
6.25641025641026 0.655647993087769
6.64102564102564 0.946461737155914
7.04487179487179 0.882922291755676
7.47435897435897 0.876116096973419
7.92948717948718 0.941702723503113
8.41025641025641 0.955277383327484
8.92307692307692 0.864720165729523
9.46794871794872 0.927317142486572
10.0448717948718 0.967952728271484
10.6602564102564 0.916394948959351
11.3076923076923 0.896960496902466
12 0.793735325336456
12.7307692307692 0.857550263404846
13.5064102564103 0.673811435699463
14.3333333333333 0.819997429847717
15.2051282051282 0.681863188743591
16.1346153846154 0.893854439258575
17.1153846153846 0.871999680995941
18.1602564102564 0.89844536781311
19.2692307692308 0.724242806434631
20.4423076923077 0.797550678253174
21.6858974358974 0.663462817668915
23.0128205128205 0.660562872886658
24.4102564102564 0.778646767139435
25.9038461538462 0.818094491958618
27.4807692307692 0.635459661483765
29.1538461538462 0.770263075828552
30.9358974358974 0.66588681936264
32.8205128205128 0.808058679103851
34.8205128205128 0.818628072738647
36.9423076923077 0.821372628211975
39.1923076923077 0.755211889743805
41.5833333333333 0.840404450893402
44.1153846153846 0.77805757522583
46.8076923076923 0.881657183170319
49.6602564102564 0.913554906845093
52.6858974358974 0.842320501804352
55.8974358974359 0.95589804649353
59.3076923076923 0.887811481952667
62.9230769230769 0.886723399162292
66.7564102564103 0.838664293289185
70.8269230769231 0.881826877593994
75.1474358974359 0.88467538356781
79.7307692307692 0.927326917648315
84.5897435897436 0.93161928653717
89.7435897435897 0.911451160907745
95.2179487179487 0.947983026504517
101.019230769231 0.857874929904938
107.179487179487 0.864149689674377
113.711538461538 0.919982731342316
120.641025641026 0.931507110595703
127.99358974359 0.930004358291626
135.801282051282 0.954390048980713
144.076923076923 0.951113939285278
152.858974358974 0.968225002288818
162.179487179487 0.940305411815643
172.064102564103 0.975947856903076
182.551282051282 0.983258724212646
193.679487179487 0.915634751319885
205.487179487179 0.845035791397095
218.012820512821 0.951557338237762
231.301282051282 0.861188352108002
245.403846153846 0.936766147613525
260.358974358974 0.943233013153076
276.230769230769 0.971764981746674
293.070512820513 0.835063457489014
310.935897435897 0.961303353309631
329.884615384615 0.879067480564117
350 0.949127733707428
};
\end{axis}

\end{tikzpicture}

      \tikzexternaldisable
    \end{minipage}
  \end{subfigure}
  \caption{ \textbf{Full-batch \ggn versus full-batch Hessian.} Overlap between
    the top-$C$ eigenspaces of the full-batch \ggn and full-batch Hessian during
    training for all test problems. }
  \label{vivit::fig:ggn_vs_hessian}
\end{figure*}

%%% Local Variables:
%%% mode: latex
%%% TeX-master: "../../thesis"
%%% End:


%%% Local Variables:
%%% mode: latex
%%% TeX-master: "../thesis"
%%% End:
