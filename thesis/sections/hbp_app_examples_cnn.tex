The recursive approaches in \cite{botev2017practical,wei2018bdapch} tackle the
computation of curvature blocks for FCNNs. To the best of our knowledge, an
extension to CNNs has not been achieved so far. One reason might be that
convolutions come with heavy notation that is difficult to deal with in index
notation.

\citet{martens2015optimizing} provide a procedure for computing a Kronecker-factored
approximation of the Fisher for convolutions (KFC). This scheme relies on the
property of the Fisher to describe the covariance of the log-likelihood's
gradients under the model's distribution. Curvature information is thus encoded
in the expectation value, and not by backpropagation of second-order
derivatives.

To derive the HBP for convolutions, we use that an efficient implementation of
the forward pass is decomposed into multiple operations (see
\Cref{hbp::fig:compositionConvolution}), which can be considered independently
by means of our modular approach (see \Cref{hbp::fig:convolutionSubmodules} for
details). Our analysis starts by considering the backpropagation of curvature
through operations that occur frequently in CNNs. This includes the reshape
(\Cref{hbp::subsec:HBPReshape}) and extraction operation
(\Cref{hbp::subsec:HBPIndexSelect}). In \Cref{hbp::subsec:convolutions} we
outline the modular decomposition and curvature backpropagation of convolution
in two dimensions. The approach carries over to other dimensions.

All operations under consideration in this section are linear. Hence the second
terms in \Cref{hbp::equ:hessianBackPropagation} vanish. Again, we use the
framework of matrix differential calculus
\citep{magnus1999MatrixDifferentialCalculus} to avoid index notation.

\subsection{Reshape/View}\label{hbp::subsec:HBPReshape}
The reshape operation reinterprets a tensor $\tX \in \mathbb{R}^{n_1 \times
  \dots \times n_x}$ as another tensor $\tZ\in \mathbb{Z}^{m_1 \times \dots
  \times m_z}$ (\Cref{def:background::TensorReshape})
\begin{align*}
  \tZ(\tX) = \mathrm{reshape}(\tX)\,,
\end{align*}
which possesses the same number of elements, \ie $\prod_i n_i = \prod_i m_i$.
One example is given by the $\vec$ operation from
\Cref{def:background::Flattening}. It corresponds to a reshape into a tensor of
order one. As the arrangement of elements remains unaffected, $\vec \tZ = \vec
\tX$, and reshaping corresponds to the identity map on the vectorized input.
Consequently, one finds (remember that $\gradsquared{\tX}\ell$ is a shorthand
notation for $\gradsquared{\vec \tX}\ell$)
\begin{align*}
  \gradsquared{\tX}\ell = \gradsquared{\tZ}\ell \,.
\end{align*}

\subsection{Index Select/Map}\label{hbp::subsec:HBPIndexSelect}
Selecting elements of a tensor can be phrased as matrix-vector multiplication of
a binary matrix $\mPi$ and the vectorized tensor. The mapping is described by an
index map $\pi$. Element $j$ of the output $\vz\in \mathbb{R}^m$ is selected as
element $\pi(j)$ from the input $\vx\in \mathbb{R}^n$. Only elements $[\mPi]_{j,
  \pi(j)}$ in the selection matrix $\mPi\in \mathbb{R}^{m\times n}$ are one,
while all other entries vanish. Consequently, index selection can be expressed
as
\begin{align*}
  [\vz]_j = [\vx]_{\pi(j)} \quad \Leftrightarrow\quad \vz(\vx) = \mPi \vx \quad \text{with} \quad [\mPi]_{j,\pi(j)} = 1\,.
\end{align*}
The HBP is equivalent to the linear layer discussed in
\Cref{hbp::subsec:linearLayerBackwardPass},
\begin{align*}
  \gradsquared{\vz}\ell = \mPi^\top (\gradsquared{\vx}\ell) \mPi\,.
\end{align*}
Applications include max-pooling and the \inlinecode{im2col}/\inlinecode{unfold}
operation (see \Cref{hbp::subsec:convolutions}). Average-pooling represents a
weighted sum of index selections and can be treated analogously.

\subsection{Convolution}\label{hbp::subsec:convolutions}
The convolution operation acts on local patches of a multi-channel input of
sequences, images, or volumes. In the following, we restrict the discussion to
two-dimensional convolution. \Cref{hbp::subfig:convolutionIllustration1}
illustrates the setting. A collection of filter maps, the \emph{kernel} $\tW$,
is slid over the spatial coordinates of the input tensor $\tX$. In each step,
the kernel is contracted with the current area of overlap (the \emph{patch}).

\tikzset{external/export next=false} % externalization does not correctly fill
% out the matrix background in subfigure (c)
\begin{figure}
  \centering
  \begin{subfigure}[t]{1.0\linewidth}
    \caption{}
    \label{hbp::subfig:convolutionIllustration1}
    \vspace{-4ex}
    \tikz{
      \node (sketch) {
        \resizebox{\textwidth}{!}{
            \begin{tikzpicture}
    \node [font=\Large] (inTensor) {\drawInputTensor};
    \node [yshift = -225, font=\Large] (inTensorExpanded) at (inTensor) {\drawInputTensorExpanded};

    \node [xshift = 100](convolutionSign) at (inTensor) {$\star$};

    \node [xshift = 80, yshift = 25, scale = 0.7, font=\Large] (inKernel1) at (convolutionSign) {\drawKernelOne};
    \node [xshift = 80, yshift = -25, scale = 0.7, font=\Large] (inKernel2) at (convolutionSign) {\drawKernelTwo};

    \coordinate [xshift=180] (kernelExpandedCenter) at (inTensorExpanded);
    \node [yshift = 75, scale = 0.7, font=\Large] (kernelOneExpanded) at (kernelExpandedCenter) {\drawKernelOneExpanded};
    \node [yshift = -75, scale = 0.7, font=\Large] (kernelTwoExpanded) at (kernelExpandedCenter) {\drawKernelTwoExpanded};

    \node [xshift = 170] (equalSign) at (convolutionSign) {$=$};

    \node [xshift = 80, font=\Large] (outTensor) at (equalSign) {\drawOutputTensor};
    \node [yshift = -225, font=\Large] (outTensorExpanded) at (outTensor) {\drawOutputTensorExpanded};

    \foreach \i/\j in {0.15/0.15, 0.5/0.5, 0.85/0.85}{
      \foreach \name in {kernelOneExpanded}{
        \draw [-latex, thick, out=0, in=180]
        ($($(inTensorExpanded.north east)!\i!(inTensorExpanded.south east)$)+(0.2,0)$)
        to
        ($(\name.north west)!\j!(\name.south west)$);
      }
    }

    \foreach \i/\j in {0.15/0.18, 0.5/0.24, 0.85/0.3}{
      \foreach \name in {outTensorExpanded}{
        \draw [-latex, thick, out=0, in=180]
        ($($(kernelOneExpanded.north east)!\i!(kernelOneExpanded.south east)$)+(0.2,0)$)
        to
        ($(\name.north west)!\j!(\name.south west)$);
      }
    }

    \node[yshift=-235] (inputMatrix) at (inTensorExpanded) {\drawInputMatrix};
    \node[yshift=-235, xshift=180] (kernelMatrix) at (inTensorExpanded) {\drawKernelMatrix};
    \node[yshift=-460] (equalSignMM) at (equalSign) {$=$};
    \node[xshift=80] (outMatrix) at (equalSignMM) {\drawOutputMatrix};
\end{tikzpicture}

        }
      };
    }
  \end{subfigure}

  \vspace{-0.43\textheight}

  \begin{subfigure}[t]{1.0\linewidth}
    \caption{}
    \label{hbp::subfig:convolutionIllustration2}
  \end{subfigure}

  \vspace{0.43\textheight}

  \vspace{-0.17\textheight}

  \begin{subfigure}[t]{1.0\linewidth}
    \caption{}
    \label{hbp::subfig:convolutionIllustration3}
  \end{subfigure}

  \vspace{0.13\textheight}

  \caption{\textbf{Two-dimensional convolution $\tY = \tX \star \tW$ without
      bias term.} \subfigref{hbp::subfig:convolutionIllustration1} The input
    $\tX$ consists of $C_\text{in} = 3$ channels (different colors) of $(3\times
    3)$ images. Filter maps of size $(2\times 2)$ are provided by the kernel
    $\tW$ for the generation of $C_\text{out} = 2$ output channels. Patch and
    kernel volumes that are contracted in the first step of the convolution are
    highlighted. Assuming no padding and a stride of one results in four
    patches. New features $\tY$ consist of $C_\text{out} = 2$ channels of
    $(2\times 2)$ images. \subfigref{hbp::subfig:convolutionIllustration2}
    Detailed view. All tensors are unrolled along the first axis.
    \subfigref{hbp::subfig:convolutionIllustration3} Convolution as matrix
    multiplication. From left to right, the matrices $\llbracket
    \tX\rrbracket^\top, \mW^\top,$ and $\mY^\top$ are shown.}
  \label{hbp::fig:convolutionIllustration}
\end{figure}

Both the sliding process as well as the structure of the patch area can be
controlled by hyperparameters of the operation (kernel size, stride, dilation).
Moreover, it is common practice to extend the input tensor, for instance by
zero-padding \citep[for an introduction to the arithmetics of convolutions,
see][]{dumoulin2016ConvolutionArithmeticGuide}. The approach presented here is
not limited to a certain choice of convolution hyperparameters.

\subsubsection{Forward Pass \& Notation}
We now introduce the quantities involved in the process along with their
dimensions. For a summary, see \Cref{hbp::table:convolution Quantities}. A
forward pass of convolution proceeds as follows (see
\Cref{hbp::subfig:convolutionIllustration2} for an example):

\begin{itemize}
\item The input $\tX$, a tensor of order three, stores a collection of
  two-dimensional data. Its components $\tX_{c_\text{in}, x_1, x_2}$ are
  referenced by indices for the channel $c_{\text{in}}$ and the spatial location
  $(x_1,x_2)$. $C_{\text{in}}, X_1, X_2$ denote input channels, width \& height
  of the image, respectively.

\item The kernel $\tW$ is a tensor of order four with shape $(C_{\text{out}},
  C_{\text{in}}, K_1, K_2)$. Kernel width $K_1$ and height $K_2$ determine the
  patch size $P= K_1 K_2$ for each channel. New features are obtained by
  contracting the patch and kernel. This is repeated for a collection of
  $C_{\text{out}}$ output channels stored along the first axis of $\tW$.

\item Each output channel $c_\text{out}$ is shifted by a bias
  $b_{c_\text{out}}$, stored in the $C_\text{out}$-dimensional vector $\vb$.

\item The output $\tY = \tX \star \tW$ with components $[\tY]_{c_\text{out},
    y_1, y_2}$ is of the same structure as the input. We denote the spatial
  dimensions of $\tY$ by $Y_1, Y_2$, respectively. Hence $\tY$ is of dimension
  ($C_\text{out}, Y_1, Y_2$).
\end{itemize}

\marginnote{
  \begin{center}
    \tikzexternalenable
    \resizebox{\linewidth}{!}{\tikzstyle{inputNode} = [ellipse, fill=foregroundChannel, rounded corners=1ex,
draw=black, drop shadow=black]
\tikzstyle{kernelNode} = [ellipse, fill=centerChannel, rounded corners=1ex, draw=black, drop shadow=black]
\tikzstyle{outputNode} = [ellipse, fill=foregroundOutput, rounded corners=1ex, draw=black, drop shadow=black]
\tikzstyle{opNode} = [rectangle, fill=gray!50!white, rounded corners=1ex, draw=black, drop shadow=black]
 \begin{tikzpicture}
  \node [inputNode] (input) {input};
  \node [opNode, right of=input, node distance=2cm, align=center] (im2col) {im2col /\\ unfold};
  \node [opNode, right of=im2col, node distance=2cm, align=center] (mm) {matmul};
  \node [opNode, below of=mm, node distance=1.2cm, align=center] (col2im) {col2im /\\ reshape};
  \node [outputNode, below of=col2im, node distance=1.2cm] (output) {output};
  \node [opNode, above of=mm, node distance=1cm] (kernelReshape) {reshape};
  \node [kernelNode, left of=kernelReshape, node distance=2cm] (kernel) {kernel};
  \node [opNode, left of=col2im, node distance=2cm] (expand) {copy};
  \node [kernelNode, left of=expand, node distance=2cm] (bias) {bias};
  \draw [-latex, thick] (input.east) to (im2col);
  \draw [-latex, thick] (im2col.east) to (mm.west);
  \draw [-latex, thick] (mm.south) to (col2im.north);
  \draw [-latex, thick] (col2im.south) to (output.north);
  \draw [-latex, thick] (kernel.east) to (kernelReshape.west);
  \draw [-latex, thick] (kernelReshape.south) to (mm.north);
  \draw [-latex, thick] (bias.east) to (expand.west);
  \draw [-latex, thick] (expand.north east) to (mm.south west);
\end{tikzpicture}
}
    \tikzexternaldisable
  \end{center}
  \captionof{figure}{\textbf{Decomposition of the convolution operation's
      forward pass.}}
  \label{hbp::fig:compositionConvolution}
}

\paragraph{Example (index notation):}
A special case where input and output have the same spatial
dimensions~\citep{grosse2016kronecker} uses a stride of one, kernel widths $K_1
= K_2 = 2 K + 1$, $(K\in \mathbb{N})$, and padding $K$. Elements of the filter
$[\tW]_{c_{\text{out}}, c_{\text{in}}, :, :}$ are addressed with the index set
$\{-K, \dots, 0, \dots, K\} \times \{-K, \dots, 0, \dots, K\}$:
\begin{align}
  [\tY]_{c_\text{out}, y_1, y_2} = \sum\limits_{k_1 = -K}^{K}
  \sum\limits_{k_2 = -K}^{K} [\tX]_{c_\text{in}, x_1 + k_1, x_2 + k_2}
  [\tW]_{c_\text{out}, c_{\text{in}}, k_1, k_2} + [b]_{c_\text{out}}\,.
  \label{hbp::equ:convolutionExample}
\end{align}
Elements of $\tX$ addressed out of bounds evaluate to zero. Arbitrary
convolutions come with even heavier notation.

\subsubsection{Convolution as Matrix Multiplication}

Evaluating convolutions by sums of the form \Cref{hbp::equ:convolutionExample}
leads to poor memory locality~\citep{grosse2016kronecker}. For improved
performance, the computation is mapped to a matrix
multiplication~\citep{chellapilla2006HighPerformanceCNN}. To do so, patches of
the input $\tX$ are extracted and flattened into columns of a matrix. The patch
extraction is indicated by the operator $\llbracket \cdot \rrbracket$ and the
resulting matrix $\llbracket \tX \rrbracket$ is of dimension $(C_\text{in} K_1
K_2\times P)$ (see left part of \Cref{hbp::subfig:convolutionIllustration3}
showing $\llbracket \tX\rrbracket^\top$). In other words, elements contracted
with the kernel are stored along the first axis of $\llbracket \tX \rrbracket$.
$\llbracket \cdot \rrbracket$ is also referred to as \inlinecode{im2col} or
\inlinecode{unfold} operation\sidenote{Our definition of the \inlinecode{unfold}
  operator slightly differs from~\cite{grosse2016kronecker}, where flattened
  patches are stacked rowwise. This lets us achieve an analogous form to a
  linear layer. Conversion is achieved by transposition.}, and accounts for
padding.

\begin{table}[t]
  \centering
  \caption{\textbf{Important quantities for the convolution operation.} The
    number of patches equals the product of the outputs' spatial dimensions,
    \ie~$P = Y_1 Y_2$.}
  \label{hbp::table:convolution Quantities}
  \begin{footnotesize}
    \begin{tabular}{rrr}
      \toprule
      Tensor     & Dimensionality     & Description \\
      \midrule
      $\tX$ & $(C_\text{in}, X_1, X_2)$ & Input \\
      $\tW$ & $(C_\text{out}, C_\text{in}, K_1, K_2)$ & Kernel\\
      $\tY$ & $(C_\text{out}, Y_1, Y_2)$ & Output \\
      $\llbracket \tX \rrbracket$ & $(C_\text{in} K_1 K_2, P)$ & Expanded input\\
      $\mW$ & $(C_\text{out}, C_\text{in} K_1 K_2)$ & Matricized kernel\\
      $\mY$ & $(C_\text{out}, P)$ & Matricized output\\
      $\vb$ & $C_\text{out}$ & Bias vector\\
      $\mB$ & $(C_\text{out}, P)$ & Bias matrix\\
      \bottomrule
    \end{tabular}
  \end{footnotesize}
\end{table}

The kernel tensor $\tW$ is reshaped into a $(C_\text{out} \times C_\text{in} K_1
K_2)$ matrix $\mW$, and elements of the bias vector $vb$ are copied columnwise
into a $(C_\text{out} \times P)$ matrix
% $B$. More formally,
\begin{math}
  \mB = \vb {\mathbf{1}}^\top_P\,,
\end{math}
where ${\mathbf{1}}_P$ is a $P$-dimensional vector of ones. Patchwise
contractions are carried out by matrix multiplication and yield a matrix $\mY$
of shape $(C_\text{out}, P)$ with $P = Y_1 Y_2$,
\begin{align}
  \label{hbp::eq:convolutionMatrixMultiplication}
  \mY = \mW \llbracket \tX\rrbracket + \mB
\end{align}
(\Cref{hbp::subfig:convolutionIllustration3} shows $\mW^\top, \llbracket \tX
\rrbracket^\top$, and $\mY$ from left to right). Reshaping $\mY$ into a
$(C_\text{out}, Y_1, Y_2)$ tensor, referred to as \inlinecode{col2im}, yields
$\tY$. \Cref{hbp::fig:compositionConvolution} summarizes the outlined
decomposition of the forward pass.

\subsection{HBP for Convolution}\label{hbp::subsec:hbpConvolutions}
We now compose the HBP for convolution, proceeding from right to left with the
operations depicted in \Cref{hbp::fig:convolutionSubmodules}, by analyzing the
backpropagation of curvature for each module, adopting the figure's notation.

\subsubsection{Reshape/\robustInlinecode{col2im}}

The \inlinecode{col2im} operation takes a matrix $\mY \in
\mathbb{R}^{C_\text{out} \times Y_1 Y_2}$ and reshapes it into the tensor $\tY
\in \mathbb{R}^{C_\text{out} \times Y_1 \times Y_2}$. According to
\Cref{hbp::subsec:HBPReshape},
\begin{math}
  \gradsquared{\mY}\ell = \gradsquared{\tY}\ell. %\,.
\end{math}

\subsubsection{Bias Hessian}

Forward pass $\mY = \mZ + \mB$ and \Cref{hbp::equ:biasHessianAddition} imply
\begin{math}
  \gradsquared{\mY}\ell = \gradsquared{\mB}\ell = \gradsquared{\mZ}\ell. %\,.
\end{math}
To obtain the Hessian \wrt $\vb$ from $\gradsquared{\mB}\ell$, consider the
columnwise copy operation $\mB(\vb) = \vb \, {\mathbf{1}}_P^\top$, whose matrix
differential is $\diff \mB(\vb) = (\diff \vb)\, {\mathbf{1}}_P^\top$.
Vectorization yields $\diff \vec \mB(\vb) = ({\mathbf{1}}_P \otimes
\mI_{C_\text{out}}) \diff \vb$. Hence, the Jacobian is $\jac_{\vb} \mB(\vb) =
{\mathbf{1}}_{P} \otimes \mI_{C_\text{out}}$, and
\Cref{hbp::equ:hessianBackPropagation} yields
\begin{align*}
  \gradsquared{\vb}\ell
  =
  \left( {\mathbf{1}}_P \otimes \mI_{C_\text{out}} \right)^\top
  \gradsquared{\mB}\ell
  \left({\mathbf{1}}_P \otimes \mI_{C_\text{out}} \right)\,.
\end{align*}
This performs a linewise and columnwise summation over $\gradsquared{\mB}\ell$,
summing entities that correspond to copies of the same entry of $\vb$ in the
matrix $\mB$. It can also be regarded as a reshape of $\gradsquared{\mB}\ell$
into a $(C_\text{out}, P, C_\text{out}, P)$ tensor, which is then contracted
over the second and fourth axis.

\begin{figure*}[t]
  \centering \resizebox{!}{3.25cm}{
    \tikzexternalenable
    \footnotesize
    % HBP for convolutional layer decomposed into submodules
\begin{tikzpicture}
    % first four layers
    \node (in1)
        [inner sep=0]
        {\tikz \drawMessagesWithArrows{$ \tensor{X} $}{ }{ $\HeCal \tensor{X} $}{\hNodeDistance};};
    \node (layer1)
        [anchor=south west, inner sep=0]
        at (in1.south east)
        {\tikz \drawModuleNoParams{unfold}{10};};
    \node (out1)
        [inner sep=0, anchor=south west]
        at (layer1.south east)
         {\tikz \drawMessagesWithArrows{$ \llbracket \tensor{X} \rrbracket$}{ }{ $\HeCal \llbracket \tensor{X}\rrbracket $}{\hNodeDistance};};
    \node (layer2)
        [inner sep=0pt, anchor=south west]
        at (out1.south east)
        {\tikz\drawModuleWithParams{matmul}{16}{$\tensor{W}$}{ }{ $\HeCal  \tensor{W} $};};
    \node (out2)
        [inner sep=0, anchor=south west]
        at (layer2.south east)
        {\tikz \drawMessagesWithArrows{$Z$}{ }{ $\HeCal Z $}{\hNodeDistance};};
    \node (layer3)
        [inner sep=0pt, anchor=south west]
        at (out2.south east)
        {\tikz\drawModuleWithParams{add}{16}{$b$}{ }{ $\HeCal b $};};
    \node (out3)
        [inner sep=0, anchor=south west]
        at (layer3.south east)
        {\tikz \drawMessagesWithArrows{$Y$}{ }{ $\HeCal Y $}{\hNodeDistance};};
    \node (layer4)
        [anchor=south west, inner sep=0]
        at (out3.south east)
        {\tikz \drawModuleNoParams{col2im}{10};};
    \node (out4)
        [inner sep=0, anchor=south west]
        at (layer4.south east)
        {\tikz \drawMessagesWithArrows{$\tensor{Y}$}{ }{$\HeCal \tensor{Y} $}{\hNodeDistance};};
\end{tikzpicture}

    \tikzexternaldisable
  }
  \caption{\textbf{Decomposition of convolution with notation for the study of
      curvature backpropagation.}}
  \label{hbp::fig:convolutionSubmodules}
\end{figure*}

\subsubsection{Weight Hessian}

HBP for the matrix-matrix multiplication $\mZ(\mW, \llbracket \tX \rrbracket) =
\mW \llbracket \tX \rrbracket$ was discussed in
\Cref{hbp::subsec:linearLayerBackwardPass}. The Jacobians are given by
$\jac_{\llbracket \tX \rrbracket} \mZ(\llbracket \tX \rrbracket) = \mI_P \otimes
\mW$ and $\jac_{\mW} \mZ(\mW) = \llbracket \tX \rrbracket^\top \otimes \mI_S$
with the patch size $ S = C_\text{in} K_1 K_2$. Hence, HBP for the weight matrix
\& unfolded input are
\begin{align*}
  \gradsquared{\llbracket \tX \rrbracket}\ell
  &=
    \left( \mI_P \otimes \mW \right)^\top
    \gradsquared{\mZ}\ell
    \left( \mI_P \otimes \mW \right)\,,
  \\
  \gradsquared{\mW}\ell
  &=
    \left( \llbracket \tX\rrbracket^\top \otimes \mI_S \right)^\top
    \gradsquared{\mZ}\ell
    \left( \llbracket \tX \rrbracket^\top \otimes \mI_S \right).
\end{align*}
From what has been said about the reshape operation in
\Cref{hbp::subsec:HBPReshape}, it follows that $\gradsquared{\tW}\ell =
\gradsquared{\mW}\ell$.

\subsubsection{Im2col/Unfold}

The patch extraction operation $\llbracket \cdot \rrbracket$ copies all patch
elements into the column of a matrix. It thus represents a selection of
elements by an index map which is hard to express in notation. Numerically, it
is obtained by calling \inlinecode{im2col} on a $(C_{\text{in}}, X_1, X_2)$
index tensor whose entries correspond to the indices. The resulting tensor
contains all information about the index map. HBP follows the relation of
\Cref{hbp::subsec:HBPIndexSelect}.

\subsubsection{Discussion}
Although convolution can be understood as a matrix multiplication, the parameter
Hessian is not identical to that of the linear layer discussed in
\Cref{hbp::subsec:linearLayerBackwardPass}. The difference is due to the
parameter sharing of the convolution. In the case of a linear layer $\vz = \mW
\vx + \vb$, the Hessian of the weight matrix for a single sample possesses
Kronecker structure
\citep{botev2017practical,wei2018bdapch,bakker2018OuterProductStructure},
\ie~$\gradsquared{\mW}\ell = \vx \otimes \vx^\top \otimes
\gradsquared{\vz}\ell$. For convolutional layers, however, it has been argued
by~\cite{bakker2018OuterProductStructure} that block diagonals of the Hessian do
not inherit the same structure. Rephrasing the forward
pass~\eqref{hbp::eq:convolutionMatrixMultiplication} in terms of vectorized
quantities, we find
\begin{align*}
  \label{hbp::eq:convolutionMatrixMultiplicationCircular}
  \vec \mY
  =
  \left( I_P \otimes \mW \right)
  \vec \llbracket \tX \rrbracket
  +
  \vec \mB\,.
\end{align*}
In this perspective, convolution corresponds to a fully-connected linear layer,
with the additional constraints that the weight and bias matrix be circular.
Defining $\mWhat := \mI_P \otimes \mW$, one then finds the Hessian
$\gradsquared{\mWhat}\ell$ to possess Kronecker structure. Parameterization with
a kernel tensor encodes the circularity constraint in weight sharing.

For the kernel Hessian $\gradsquared{\mW}\ell$ to possess Kronecker structure,
the output Hessian $\gradsquared{\mZ}\ell$ must be assumed to factorize into a
Kronecker product of $S\times S$ and $C_\text{out} \times C_\text{out}$
matrices. These assumptions are somewhat in parallel with the approximations
introduced in \cite{grosse2016kronecker} to obtain KFC.

%%% Local Variables:
%%% mode: latex
%%% TeX-master: "../thesis"
%%% End:
