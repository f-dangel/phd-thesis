Contemporary machine learning, in particular deep learning, remains a craft and
an art. High dimensionality, stochasticity, and non-convexity require constant
tracking and tuning, often resulting in a painful process of trial and error.
When things fail, popular performance measures, like the training loss, do not
provide enough information by themselves. These metrics only tell \emph{whether}
the model is learning, but not \emph{why}. Alternatively, traditional debugging
tools can provide access to individual weights and data. However, in models
whose power only arises from possessing myriad weights, this approach is
hopeless, like looking for the proverbial needle in a haystack.

To mitigate this, we proposed \cockpit, a practical visual debugging tool for
deep learning. It offers instruments to monitor the network's internal dynamics
during training, in real-time. In its presentation, we focused on two crucial
factors affecting user experience: Firstly, such a debugger must provide
meaningful insights. To demonstrate \cockpit's utility, we showed how it can
identify bugs where traditional tools fail. Secondly, it must come at a feasible
computational cost. Although \cockpit uses rich second-order information,
efficient computation keeps the necessary run time overhead cheap. The
open-source \pytorch package can be added to many existing training loops.

Obviously, such a tool is never complete. Just like there is no perfect
universal debugger, the list of current instruments is naturally incomplete.
Further practical experience with the tool, for example in the form of a future
larger user study, could provide additional evidence for its utility. However,
our analysis shows that \cockpit provides useful tools and extracts valuable
information presently not accessible to the user. We believe that this improves
algorithmic interpretability -- helping practitioners understand how to make
their models work -- but may also inspire new research. The code is designed
flexibly, deliberately separating the computation and visualization. New
instruments can be added easily and also be shown by the user's preferred
visualization tool, \eg \tensorboard. Of course, instead of just showing the
data, the same information can be used by novel algorithms directly,
side-stepping the human in the loop.

%%% Local Variables:
%%% mode: latex
%%% TeX-master: "../thesis"
%%% End:
