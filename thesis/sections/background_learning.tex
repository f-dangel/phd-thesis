``Learning'' is connected to optimization by the risk minimization paradigm. The
idea is to define a performance metric that assesses the quality of the model's
prediction. Then, learning happens by maximizing that performance metric.
Conversely, one can specify a metric for the prediction's error
(also referred to as \emph{risk}), and minimize the latter.


For example, an intuitive way to assess performance for a classification task is
\emph{accuracy}, the ratio of correct and total predictions (the error would be
the ratio of incorrect and total predictions). However, such direct performance
measures are hard to optimize with derivative-based methods\sidenote{\eg the
  accuracy on one datum is either 0 or 1, and hence its derivative vanishes
  everywhere it is defined.} and must be substituted by a surrogate function
that approximates the original performance measure, but is easier to optimize.
In the deep learning terminology, these surrogates are commonly referred to as
\emph{loss functions}. This section expands on risk minimization, its
characteristic properties in deep learning, and its probabilistic
interpretation.

\subsection{Notation \& Mathematical
  Details}\label{sec:background::empiricalRiskMinimization}

\subsubsection{Risk \& Empirical Risk}

Consider supervised learning with the goal to learn the functional relation
between inputs $\vx \in \sX$ and targets $\vy \in \sY$
(\Cref{fig:background::ModelLossFunctionSplit}). The mapping is described by a
model $f_{\vtheta}: \sX \to \sF$ with adjustable parameters $\vtheta \in
\sTheta$ that produces predictions $\vf := f_{\vtheta}(\vx) \in \sF$ for an
input $\vx$. A prediction's error is assessed through a convex loss function
$\ell: \sF \times \sY \to \sR$.

\begin{figure}[tb]
  \centering \tikzexternalenable
  \resizebox{!}{3.5cm}{% Model-loss function split with data flow
\input{figures/background/model_loss_split_command}

\begin{tikzpicture}
  \drawModelLossSplit{$\vx$}{$\vf$}{$\vy$}{$\ell(\vf, \vy)$}
\end{tikzpicture}

%%% Local Variables:
%%% mode: latex
%%% TeX-master: "../../thesis"
%%% End:
}
  \tikzexternaldisable
  \caption{\textbf{Components of supervised learning:} The goal is to infer
    parameters $\vtheta$ of a model $f_{\vtheta}$, that relates inputs $\vx$ to
    predictions $\vf$, by minimizing a loss function $\ell$ between the
    prediction $\vf$ and label $\vy$.}
  \label{fig:background::ModelLossFunctionSplit}
\end{figure}

\emph{Regression} (\Cref{ex:background::Regression}) and \emph{classification}
(\Cref{ex:background::Classification}) are two common tasks that will be used
frequently in later chapters. The remainder of this text sets $\sX = \sR^{M}$
and $\sF = \sR^{C}$, with input and prediction space dimensions $M, C$,
respectively ($C$ corresponds to the number of classes for classification).
Because the discussion focuses on neural networks as model, it sets $\sTheta =
\sR^{D}$ in what follows.

\begin{figure}[!b]
  \begin{example}[\textbf{Least squares regression \& square
      loss}]\label{ex:background::Regression}
    Regression associates features in $\sX = \sR^M$ with targets in $\sY =
    \sR^C$. A prediction in $\sF = \sR^{C}$ compares to its ground truth via the
    mean squared error\sidenote{%
      There exist different conventions for the normalization factor. This text
      adapts the implementation of
      \inlinecode{\href{https://pytorch.org/docs/1.11/generated/torch.nn.MSELoss.html\#torch.nn.MSELoss}{MSELoss}}
      (with \inlinecode{reduction="mean"} mode) in \pytorch for consistency with
      the code presented in later chapters. Normalizing by $\nicefrac{1}{C}$ is
      also close to what the name, mean squared error, suggests.}
    \begin{align}\label{eq:background::squareLoss}
      \ell(\vf, \vy)
      =
      \frac{1}{C} \sum_{c=1}^C (\evy_c - \evf_c)^2
      =
      \frac{1}{C} \lVert \vy - \vf  \rVert_{2}^{2}
    \end{align}
  \end{example}
  \begin{example}[\textbf{$C$-class classification \& softmax cross-entropy
      loss}]\label{ex:background::Classification}
    Classification assigns features in $\sX = \sR^{M}$ to classes in $\sY =
    \left\{ 1, \dots, C \right\}$ using a model to $\sF = \sR^C$. The softmax
    cross-entropy loss\sidenote{%
      Sometimes, the softmax is considered part of the model rather than the
      loss function. This text assigns it to the loss function, in line with the
      \pytorch implementation of
      \inlinecode{\href{https://pytorch.org/docs/1.11/generated/torch.nn.CrossEntropyLoss.html\#torch.nn.CrossEntropyLoss}{CrossEntropyLoss}},
      that combines softmax and cross-entropy, which is numerically more
      stable~\cite[][Chapter 4]{goodfellow2016deep}.}%
    maps the model's prediction to a probability distribution over classes, then
    uses cross-entropy to compare it with the ground truth,
    \begin{align}
      \label{eq:background::softmaxCrossEntropyLoss}
      \ell(\vf, y) = - \log([p(\vf)]_{y})
      =
      - {\log p(\vf)}^\top \onehot(y)
    \end{align}
    where $p(\vf) = \softmax(\vf)$ and $[\onehot(y)]_{c} = \delta_{c,y}$.
  \end{example}
\end{figure}

% \paragraph{Risk minimization:}
Assume that the population of input-target pairs,
\ie the \emph{data-generating process}, follows a distribution $\pdata(\vx,
\vy)$. A model's expected risk under this distribution is defined as
\begin{subequations}
  \begin{align}
    \label{eq:background::expectedRisk}
    \begin{split}
      \gL_{\pdata}(\vtheta)
      &:=
        \E_{\pdata(\vx, \vy)}
        \left[
        \ell(f_{\vtheta}(\vx), \vy)
        \right]
      \\
      &=
        \iint_{\sX,\sY}
        \ell(f_{\vtheta}(\vx), \vy)
        \pdata(\vx, \vy)
        \,\diff\vx\,\diff\vy\,.
    \end{split}
  \end{align}
  The incentive for a model to perform well is to achieve a small expected risk.
  Therefore, training a model is minimizing \Cref{eq:background::expectedRisk},
  \begin{equation}
    \label{eq:background::minimizeExpectedRisk}
    \minimize_{\vtheta} \gL_{\pdata}(\vtheta)\,.
  \end{equation}
\end{subequations}

% \paragraph{Empirical approximation:}
But in practice, \Cref{eq:background::minimizeExpectedRisk} is inaccessible
because the data-generating process $\pdata(\vx, \vy)$ is unknown. Instead, the
problem is approximated through an \iid dataset $\sD = \{ (\vx_n, \vy_n) \in \sX
\times \sY \}_n$ of labeled data collected from $\pdata(\vx,\vy)$. $\pdata(\vx,
\vy)$ is approximated by the empirical distribution
\begin{subequations}\label{eq:background::supervisedLearning}
  \begin{equation}
    \label{eq:background::empiricalDistribution}
    p_{\sD}(\vx, \vy)
    =
    \frac{1}{|\sD|}
    \sum_{(\vx_{n}, \vy_{n}) \in \sD}
    \delta(\vx - \vx_{n}) \delta(\vy - \vy_{n})
  \end{equation}
  implied by $\sD$. The model's empirical risk on $\sD$ follows from
  substituting $\pdata$ with $p_{\sD}$ in \Cref{eq:background::expectedRisk},
  which yields
  \begin{equation}
    \label{eq:background::empiricalRisk}
    \gL_{\sD}(\vtheta)
    :=
    \gL_{p_{\sD}}(\vtheta)
    =
    \frac{1}{|\sD|}
    \sum_{(\vx_{n}, \vy_{n}) \in \sD}
    \ell(f_{\vtheta}(\vx_{n}), \vy_{n})\,.
  \end{equation}
  In practice, learning happens by minimizing an empirical risk,
  \begin{equation}
    \label{eq:background::empiricalRiskMinimization}
    \minimize_{\vtheta} \gL_{\sD}(\vtheta)\,.
  \end{equation}
\end{subequations}

\subsubsection{Challenges for Optimization}

Training a model relies on optimization algorithms which are initialized at some
$\vtheta_{0}\in \sTheta$ and seek to iteratively improve the solution to
\Cref{eq:background::empiricalRiskMinimization}. At an iteration $t$, the
optimizer is given access to information about the objective at $\vtheta_t$,
like derivatives (\Cref{sec:background::GradientBackpropagation}), to deduct a
step $\vtheta_{t+1} \leftarrow \vtheta_t$, potentially using additional
information from past observations. In general, the more local information is
available, the more powerful a single update step can be. But richer information
is often more costly to compute. The large-scale nature of deep learning poses
challenges on the accessible information:
\begin{itemize}

\item \textbf{Big data:} deep learning datasets are often large because more
  data gives better approximations of the true distribution $\pdata$ via
  $p_{\sD}$, and thus better task performance. The simultaneous computation of
  all per-datum losses for $\gL_{\sD}(\vtheta)$ does not fit into memory (for
  many contemporary tasks, it is even infeasible to hold $\sD$ in
  memory\sidenote{%
    \Eg \imagenet~\cite{deng2009imagenet} consists of 1,281,167 training images.
    The Kaggle download requires roughly 166\,GB of memory.%
  }). Still, to obtain $\gL_{\sD}$, one could sequentially compute and reduce
  its summands on data chunks of manageable size. In practice, it is more common
  though to approximate $\gL_{\sD}$ on a randomly drawn small subset, a
  mini-batch, $\sB \subseteq \sD$ with $|\sB| \ll |\sD|$. While this avoids a
  computationally expensive full sweep over the data, subsampling introduces
  \emph{noise} in the loss (\Cref{sec:background::MiniBatching}). This noise is
  inherited by the computed information supplied to the optimizer. Algorithms
  must therefore take into account the stochastic nature of their observations.

\item \textbf{Very large models:} the parameter space dimension $D$ of DNNs
  usually exceeds the (already large) amount of data $|\sD|$, \ie $D \gg |\sD|$.
  It affects the complexity to store and compute information, such as
  derivatives of the loss \wrt $\vtheta$, and makes it more challenging to
  efficiently work with higher-order information\sidenote{%
    For example, interesting quantities for an optimizer are the loss
    landscape's local slope (gradient) and curvature (Hessian,
    \Cref{def:background::Hessian}). While the gradient is cheap to compute and
    store ($D$ elements), holding the $D\times D$ Hessian in memory is
    infeasible.%
  }. Therefore, working with higher-order information relies on implicit schemes
  (\eg matrix-free Hessian-vector products~\cite{pearlmutter1994fast}) or
  light-weight structured approximations (\eg through Kronecker
  products~\cite{martens2015optimizing}). Such approaches are usually technical
  and challenging to implement. They also add significant computational work
  that must be compensated for by an improved update step quality in optimizers.
\end{itemize}
In addition to these computational aspects, there exist other challenges:
\begin{itemize}
\item \textbf{Non-convexity:} although the loss function $\ell$ is convex \wrt
  the model's output $\vf$, convexity does not carry through to the model's
  parameters\sidenote[][-2.5\baselineskip]{Convexity of $\ell$ in $\vf$ is still
    useful to construct PSD approximations to the Hessian, see
    \Cref{sec:background::ggn}.}. This is because DNNs $f_{\vtheta}$ are highly
  non-linear, and therefore generally non-convex, in $\vtheta$. Local minima of
  convex functions are global minima, meaning that local improvement gets us
  closer to a global solution. But non-convex problems like
  \Cref{eq:background::empiricalRiskMinimization} have multiple local minima
  that need not be global. Through local improvements, an optimizer can arrive
  at one of these local minima, but with no path of improvement to a global
  solution. This depends on various aspects, such as the update rule,
  hyperparameters, initialization, \etc.

\item \textbf{Generalization:} learning is not pure optimization. While
  minimizing the empirical risk \Cref{eq:background::empiricalRiskMinimization}
  improves the model's performance on the collected data, the actual goal is to
  obtain good performance on new unseen data, \ie generalize well. To achieve
  generalization, it is crucial to prevent optimization to overfit specifics in
  the data.
  % train-test split
  It is common practice to split the data into three disjoint sets $\Dtrain$,
  $\Dvalid$, and $\Dtest$. The train set's empirical risk
  $\gL_{\Dtrain}(\vtheta)$ is minimized, and the validation loss
  $\gL_{\Dvalid}(\vtheta)$ serves to identify hyperparameters that lead to
  generalization on $\Dvalid$. The held-out examples in the test set $\Dtest$
  are used to assess generalization to new data.
  % data augmentation
  Another way to improve generalization is to use more data during training.
  Data augmentation~\cite{shorten2019survey} allows for cheap generation of new
  examples without collecting new data.
  % regularization
  Sometimes, it may be desirable to penalize model properties by adding a
  regularization term to the objective in
  \Cref{eq:background::empiricalRiskMinimization}.
\end{itemize}

\subsection{Batching \& Noise}\label{sec:background::MiniBatching}

Due to the large-scale nature of $\sD$ and $f_{\vtheta}$ in the empirical risk,
\Cref{eq:background::empiricalRisk} is usually stochastically approximated
through a mini-batch $\sB \subseteq \sD$, and assessed through a mini-batch loss
\begin{equation}
  \label{eq:background::miniBatchRisk}
  \gL_{\sB}(\vtheta)
  =
  \frac{1}{|\sB|}
  \sum_{(\vx_n, \vy_n) \in \sB}
  \ell(f_{\vtheta}(\vx_n), \vy_n)\,.
\end{equation}
In the following, per-sample predictions and losses will often be abbreviated as
$\vf_n := f_{\vtheta}(\vx_n)$ and $\ell_n := \ell(f_{\vtheta}(\vx_n), \vy_n)$.

\subsubsection{Batched Computations}

Evaluation of the mini-batch loss in \Cref{eq:background::miniBatchRisk} can be
parallelized (\Cref{subfig:background::BatchedComputation1}). All mini-batch
features $\{ \vx_n \}$ are mapped to predictions $\{ \vf_n \}$ by the \emph{same
  instructions $f_{\vtheta}$}, and compared in parallel with the labels $\{
\vy_n \}$, resulting in the per-sample losses $\{ \ell_n\}$. Hardware
accelerators can use this structure to achieve significant speed-up of
evaluating the loss, or its derivatives
(\Cref{sec:background::GradientBackpropagation}).

\begin{figure*}[t]
  \centering
  \begin{subfigure}[t]{0.495\linewidth}
    \centering
    \caption{Same instructions, multiple data}
    \label{subfig:background::BatchedComputation1}
    \tikzexternalenable
    \resizebox{!}{3.5cm}{\input{figures/background/model_loss_split_command}

\begin{tikzpicture}
  \foreach \x in {1,2,3} {
    \pgfmathsetmacro{\xshiftcm}{-0.2 * (\x - 1)}
    \pgfmathsetmacro{\yshiftcm}{-0.6 * (\x - 1)}
    \begin{scope}[xshift = \xshiftcm cm, yshift = \yshiftcm cm, opacity = 0.95]
      \drawModelLossSplit{$\vx_{\x}$}{$\vf_{\x}$}{$\vy_{\x}$}{$\ell_{\x}$}
    \end{scope}
   }
\end{tikzpicture}

%%% Local Variables:
%%% mode: latex
%%% TeX-master: "../../thesis"
%%% End:
}
    \tikzexternaldisable
  \end{subfigure}
  \hfill
  \begin{subfigure}[t]{0.495\linewidth}
    \centering
    \caption{Batched operations}
    \label{subfig:background::BatchedComputation2}
    \tikzexternalenable
    \resizebox{!}{3.5cm}{\input{figures/background/model_loss_split_command}

\begin{tikzpicture}
  \drawModelLossSplit{%
    $\left\{\vx_n\right\}$}{%
    $\left\{\vf_n\right\}$}{%
    $\left\{\vy_n\right\}$}{%
    $\left\{\ell_n\right\}$}
\end{tikzpicture}

%%% Local Variables:
%%% mode: latex
%%% TeX-master: "../../thesis"
%%% End:
}
    \tikzexternaldisable
  \end{subfigure}
  \caption{\textbf{Different visualizations of empirical risk computation
      graphs.} \subfigref{subfig:background::BatchedComputation1}, The empirical
    risk is a weighted sum of per-sample losses that are computed independently
    and with the same instructions, allowing for efficient parallelization.
    \subfigref{subfig:background::BatchedComputation2} Many ML libraries
    natively support batched operations for efficiency. In code, this batching
    is often assumed, but not explicitly expressed. The $\vmap$ concept allows
    to make batching explicit: $f_{\vtheta}$ and $\ell$ in
    \subfigref{subfig:background::BatchedComputation2} correspond to vectorized
    versions $\vmap(f)$ and $\vmap(\ell)$ from
    \subfigref{subfig:background::BatchedComputation1}.}
  \label{fig:background::BatchedComputation}
\end{figure*}

This single-instruction-multiple-data structure of the loss is often baked into
ML libraries. Many of their operations natively support batched behavior, \ie
accept stacked inputs and assume one, usually the first, axis to correspond to a
batch axis. The operation is then applied to all slices along the batch axis
(\Cref{subfig:background::BatchedComputation2}).

The concept of \inlinecode{map} in functional
programming~\cite{hughes1989functional} formalizes applying a function to a
collection of inputs, which can be seen as a function transformation. Batched
operations can be understood as transformations of the original operation.
Recently developed ML libraries~\cite{bradbury2018jax,he2021functorch} make
batching explicit by providing a $\inlinecode{vmap}$ interface to automatically
vectorize, \ie parallelize, the application of \inlinecode{map}.

\begin{definition}[\textbf{vmap}]\label{def:background::vmap} Let $f: \sX \to
  \sY, x \mapsto f(x)$ denote a function. The vectorized \inlinecode{map},
  $\vmap(f)$, of $f$ \wrt its argument $x$ is a function that accepts a
  collection of inputs and maps each item by $f$, resulting in a collection of
  outputs,
  \begin{align*}
    \vmap:\quad (\sX \to \sY)
    \to (\sX^{N} \to \sY^{N})
    \quad \forall\,N \in \sN
    \shortintertext{such that}
    \vmap(f)(\{ x_1, x_2, \dots, x_N\})
    =
    \{f(x_1), f(x_2), \dots, f(x_N)\}\,.
  \end{align*}
  Collections are often abbreviated as $\{x_n\} := \{ x_1, x_2, \dots, x_N\}$
  and $\{f(x_n)\} := \{f(x_1), f(x_2), \dots, f(x_N)\}$. Multi-variate functions
  can be mapped \wrt to a subset of arguments.
\end{definition}

Making batching explicit with \inlinecode{map}, the mini-batch loss computation
uses a vectorized model $\vmap(f_{\vtheta})$ \wrt the input features
$\vx$, such that
\begin{subequations}
  \begin{align}
    \vmap(f_{\vtheta})(\{ \vx_n\}) &= \{ \vf_{n}(\vtheta) \}\,.
  \end{align}
  Per-sample losses, which are reduced into the mini-batch loss, are
  \begin{align}
    \{ \ell_n(\vtheta) \}
    &=
      \vmap(\ell)(\{ \vf_{n}(\vtheta) \}, \{ \vy_n \})\,.
  \end{align}
\end{subequations}
Numerically, collections of vectors like $\{\vx_n\}$ \etc are represented as
matrices, and more generally, collections of $r$-dimensional arrays are stacked
into a $(r+1)$-dimensional arrays with an additional batch axis (\eg, $\{ \vx_n
\}$ is a $|\sB| \times M$ matrix). These arrays can then be efficiently
processed in hardware accelerators.

\subsubsection{Noise}

While
%
\marginnote[*-12]{
  \begin{center}
    \tikzexternalenable
    \resizebox{\linewidth}{!}{% #1 input
% #2 output
% #3 label
% #4 loss
\newcommand{\drawModelLossSplit}[4]{
  \node (in)
  [inner sep=0]
  {\tikz \drawMessagesWithArrows{#1}{ }{ }{\hNodeDistance};};
  \node (layer1)
  [anchor=south west, inner sep=0]
  at (in.south east)
  {\tikz \drawModuleWithParams{$f_{\vtheta}$}{16}{$\vtheta$}{ }{ };};
  \node (out)
  [inner sep=0, anchor=south west]
  at (layer1.south east)
  {\tikz \drawMessagesWithArrows{#2}{ }{ }{\hNodeDistance};};
  % loss layer
  \node (lossLayer)
  [inner sep=0pt, anchor=south west]
  at (out.south east)
  {\tikz\drawModuleNoParams{$\ell$}{5};};
  \node (loss)
  [inner sep=0, anchor=south west]
  at (lossLayer.south east)
  {\tikz \drawMessagesWithArrows{#4}{ }{ }{\hNodeDistance};};
  \node (label)
  [inner sep=0, anchor=south west, rotate=-90, xshift = -96, yshift = -50]
  at (lossLayer.south east)
  {\tikz \drawMessagesWithArrows{#3}{ }{ }{\hNodeDistance};};
}
%%% Local Variables:
%%% mode: latex
%%% TeX-master: "../../thesis"
%%% End:


\begin{tikzpicture}
  \pgfmathsetseed{42}
  \foreach \x in {1,2,3,4,5} {
    \pgfmathsetmacro{\xshiftcm}{-0.2 * (\x - 1)}
    \pgfmathsetmacro{\yshiftcm}{-0.6 * (\x - 1)}
    \pgfmathrandominteger{\opacitySwitch}{0}{10}
    \ifthenelse{\opacitySwitch > 6}{%
      \pgfmathsetmacro{\opacity}{0.15}}{%
      \pgfmathsetmacro{\opacity}{0.95}}
    % \node [opacity=\opacity] at (\x, 0) {\opacitySwitch};
    \begin{scope}[xshift = \xshiftcm cm, yshift = \yshiftcm cm, opacity = \opacity]
      \drawModelLossSplit{$\vx_{\x}$}{$\vf_{\x}$}{$\vy_{\x}$}{$\ell_{\x}$}
    \end{scope}
  }
\end{tikzpicture}

%%% Local Variables:
%%% mode: latex
%%% TeX-master: "../../thesis"
%%% End:
}
    \tikzexternaldisable
  \end{center}
  \captionof{figure}{\textbf{Illustration of stochastic sub-sampling:} The
    computation graph of the empirical risk (with five data in this example) is
    only evaluated on a subset of data (two in this example) to save
    computations. While this preserves parallelization, the transparent parts
    are not evaluated, which introduces noise.}\label{fig:background::MiniBatch}
}%
the sum structure in the loss \Cref{eq:background::empiricalRisk} can be
efficiently parallelized, it can also be used for stochastic approximation via
sub-sampling (\Cref{fig:background::MiniBatch}). The mini-batch loss $\gL_\sB$
in \Cref{eq:background::miniBatchRisk} is an estimator of the empirical risk
$\gL_\sD$ implied by the stochastic sampling procedure of $\sB$. Increasing the
batch size makes this estimator more precise but more costly to compute.


% Batch size = 1
To see this cost-accuracy trade-off, consider the loss of a single datum
$\ell_n$ where $n$ is uniformly drawn from $\{1, \dots, |\sD|\}$.
Then, $\ell_n$ is a random variable, implied by the sampling distribution $n \sim
p(n) = \gU(\{1, \dots, |\sD|\})$, and an unbiased estimator of the empirical risk,
\begin{subequations}
  \begin{align}\label{eq:background::unbiasedEstimator}
    \E_{p(n)}[\ell_n(\vtheta)]
    &=
      \sum_{n=1}^{|\sD|} p(n) \ell_n(\vtheta)
      =
      \gL_{\sD}(\vtheta)\,,
  \end{align}
  with variance
  \begin{align}
    \label{eq:background::unbiasedEstimatorVariance}
    \begin{split}
      \sigma^{2}
      :=
      \Var_{p(n)}[\ell_n(\vtheta)]
      &=
        \E_{p(n)}\left[(\ell_n(\vtheta) - \gL_{\sD}(\vtheta))^2\right]
      \\
      &=
        \E_{p(n)}\left[\ell_n(\vtheta)^2\right]
        -
        \gL_{\sD}(\vtheta)^2\,.
    \end{split}
  \end{align}
\end{subequations}
% Batch size B
Next, consider a batch with $\sB|$ randomly drawn samples $\vn = (n_1, \dots,
n_{|\sB|})$ from a joint distribution $p(\vn)$, \ie the mean of $\ell_{n_1},
\dots, \ell_{n_{|\sB|}}$,
\begin{subequations}
  \begin{align}
    \gL_{\vn}(\vtheta)
    &=
      \frac{1}{|\sB|}
      \sum_{i=1}^{|\sB|} \ell_{n_i}(\vtheta)\,.
  \end{align}
  If samples are drawn uniformly \iid ($p(\vn) = \prod_{i=1}^{|\sB|} p(n_i)$
  with $p(n_i) = \gU(\{1, \dots, |\sD|\})$), this estimator is also unbiased,
  \begin{align}
    \label{eq:background::unbiasedBatchEstimator}
    \E_{p(\vn)}[\gL_\vn(\vtheta)]
    =
    \frac{1}{|\sB|}
    \sum_{i=1}^{|\sB|} \E_{p(n_i)}[\ell_{n_i}(\vtheta)]
    =
    \gL_{\sD}(\vtheta)\,,
  \end{align}
  but has smaller variance than the single-sample estimator:
  \begin{align}
    \label{eq:background::varianceBatchEstimator}
    \Var_{p(\vn)}[\gL_{\vn}(\vtheta)]
    =
    \E_{p(\vn)}[\gL_\vn(\vtheta)^2] - \gL_{\sD}(\vtheta)^2
    =
    \frac{\sigma^{2}}{|\sB|}\,.
  \end{align}
  Using more samples, \ie a larger $|\sB|$, decreases the variance, and thereby
  reduces noise in the mini-batch loss.
\end{subequations}
The central limit theorem~\cite{fischer2010history} connects the mini-batch
estimator to a normal distribution. As the mini-batch size $|\sB|$ approaches
infinity, $\gL_{\vn}$ converges to a normal distribution with mean and variance
from
\Cref{eq:background::unbiasedBatchEstimator,eq:background::varianceBatchEstimator}
\begin{align}\label{eq:background::CentralLimitTheorem}
  \lim_{|\sB| \to \infty}:
  \quad
  \gL_{\vn}
  \sim
  \gN(
  \giventhat{
  \gL_{\vn}
  }
  {
  \gL_{\sD},
  \nicefrac{\sigma^{2}}{|\sB|}
  }
  )\,.
\end{align}
While applications rely on finite batch sizes, it is sometimes useful to assume
that this Gaussian distribution holds approximately.

Noise in the mini-batch loss propagates to other quantities, like the mini-batch
gradient $\grad{\vtheta}\gL_{\sB}(\vtheta)$, that are used for applications like
training. Therefore, it represents a fundamental challenge for deep learning
methods. It can be assessed through higher-order statistical moments such as the
variance (the centered second moment, see
\Cref{sec:background::gradientCovariance,cockpit::app:instruments} for
examples). Motivated by the central limit theorem
\Cref{eq:background::CentralLimitTheorem}---if the Gaussian approximation holds
sufficiently well---only second-order statistical moments are required.

Like higher-order derivatives of multi-variate functions, higher-order
statistical moments of multi-variate random variables such as the mini-batch
gradient $\grad{\vtheta}\gL_{\sB}(\vtheta) \in \sR^D$ (first moment) scale
exponentially with dimension $D$ (\eg the $D \times D$ gradient covariance
matrix, \Cref{sec:background::gradientCovariance}). Therefore, they are
challenging to work with using naive approaches.

\subsection{Probabilistic
  Interpretation}\label{sec:background::ProbabilisticInterpretation}

Risk-based supervised learning is often connected to learning the unknown true
distribution $\pdata(\vx, \vy)$ through a model distribution $p_{\vtheta}(\vx,
\vy)$ and data $\sD$, using estimation techniques to finding a good set of
parameters that minimizes a measure of dissimilarity between $\pdata$ and
$p_{\vtheta}$. The following section links the loss function $\ell$ and the
model $f_{\vtheta}$ to probabilistic objects. This will be helpful to identify
additional structure in the risk minimization problem and use it for structural
approximation of higher-order information (\eg the Fisher information matrix,
\Cref{sec:background::naturalGradientDescent}), and to motivate probabilistic
applications (\eg Laplace approximations,
\Cref{sec:background::LaplaceApproximation}).

\subsubsection{Connections to Maximum Likelihood Estimation (MLE)}
The KL-divergence between the true and the model distribution,
\begin{subequations}
  \begin{align}
    \label{eq:background::KLDivergence}
    \begin{split}
      &\KLdiv{\pdata(\vx, \vy)}{p_{\vtheta}(\vx, \vy)}
      \\
      &\quad =
        \E_{\pdata(\vx, \vy)}
        \left[
        \log\pdata(\vx, \vy)
        -
        \log p_{\vtheta}(\vx, \vy)
        \right]\,,
    \end{split}
  \end{align}
  can be used to measure their dissimilarity. Minimizing the above expression
  over $\vtheta$, and dropping parameter-independent terms leads to
  \begin{align}
    \label{eq:background::minimizeNegativeLogProbability}
    \begin{split}
      &\minimize_{\vtheta}
        \KLdiv{\pdata(\vx, \vy)}{p_{\vtheta}(\vx, \vy)}
      \\
      \Leftrightarrow
      &\minimize_{\vtheta}
        \E_{\pdata(\vx, \vy)}
        \left[
        - \log p_{\vtheta}(\vx, \vy)
        \right]\,.
    \end{split}
  \end{align}
  $\pdata(\vx, \vy)$ is inaccessible and therefore empirically approximated
  through data $\sD = \{ (\vx_n, \vy_n) \}_n$. Under the \iid assumption in the
  data, $\pdata$ is replaced with the empirical distribution $p_{\sD}$ from
  \Cref{eq:background::empiricalDistribution} and yields the accessible
  optimization task
  \begin{align}
    \label{eq:background::minimizeEmpiricalNegativeLogProbability}
    \minimize_{\vtheta}
    \frac{1}{|\sD|}
    \sum_{(\vx_{n}, \vy_{n}) \in \sD} - \log p_{\vtheta}(\vx_n, \vy_n)
  \end{align}
  This expression resembles the empirical risk minimization
  \Cref{eq:background::empiricalRiskMinimization} with a specific loss function
  $\ell$ that produces the negative log-probability of a datum $(\vx_n,\vy_n)$.
\end{subequations}

\begin{subequations}
  Supervised learning only processes features $\vx$ to predict labels $\vy$. The
  probabilistic model $p_{\vtheta}(\vx, \vy)$ thus has a more special form, in
  that it only parameterizes the likelihood $\giventhat{\vy}{\vx}$,
  \begin{align}
    \label{eq:background::modelOnlyLikelihood}
    p_{\vtheta}(\vx, \vy) = p_{\vtheta}(\giventhat{\vy}{\vx}) p(\vx)\,.
  \end{align}
  Since only the likelihood contains parameters,
  \Cref{eq:background::minimizeNegativeLogProbability} simplifies to minimizing
  the expected negative log-likelihood
  \begin{align*}
    &\minimize_{\vtheta}
      \E_{\pdata(\vx, \vy)}
      \left[
      - \log p_{\vtheta}(\giventhat{\vy}{\vx}) - \log p(\vx)
      \right]
    \\
    \Leftrightarrow
    &\minimize_{\vtheta}
      \E_{\pdata(\vx, \vy)}
      \left[
      - \log p_{\vtheta}(\giventhat{\vy}{\vx})
      \right]
  \end{align*}
  and with empirical approximation through \iid data as \Cref{eq:background::minimizeEmpiricalNegativeLogProbability},
  \begin{align}
    \label{eq:background::mnimizeEmpiricalNegativeLogLikelihood}
    \minimize_{\vtheta}
    \frac{1}{|\sD|}
    \sum_{(\vx_{n}, \vy_{n}) \in \sD}
    - \log p_{\vtheta}(\giventhat{\vy_n}{\vx_n})\,.
  \end{align}
\end{subequations}
This minimization problem corresponds to MLE\sidenote{%
  \label{side:background::MLE}
  Finding the negative log-likelihood's minimum,
  \Cref{eq:background::mnimizeEmpiricalNegativeLogLikelihood}, is equivalent to
  finding the maximum of the \iid data's likelihood
  \begin{align*}
    p(\giventhat{\sD}{\vtheta})
    =
    \prod_{(\vx_n, \vy_n)\in\sD}
    p_{\vtheta}(\giventhat{\vy_n}{\vx_n})
    p(\vx_n)\,.
  \end{align*}
  The MLE satisfies
  \begin{align*}
    \vtheta_{\text{MLE}}
    &= \argmax p(\giventhat{\sD}{\vtheta})
    \\
    &= \argmax \log p(\giventhat{\sD}{\vtheta})
    \\
    &= \argmin - \log p(\giventhat{\sD}{\vtheta})\,.
  \end{align*}
  Inserting $p(\giventhat{\vtheta}{\sD})$ and dropping parameter-independent
  terms recovers the problem
  \Cref{eq:background::mnimizeEmpiricalNegativeLogLikelihood}.} with a
statistical model $p_{\vtheta}(\giventhat{\vy}{\vx})$. This is a specific form
of empirical risk minimization where the model's prediction $f_{\vtheta}(\vx)$
parameterizes a likelihood $q$ for $\giventhat{\vy}{\vf}$, and a negative
log-likelihood loss function $\ell$, \ie
\begin{subequations}
  \label{eq:background::connectionsMLEandEmpiricalRiskMinimization}
  \begin{align}
    p_{\vtheta}(\giventhat{\vy}{\vx}) &= q(\giventhat{\vy}{f_{\vtheta}(\vx)})\,,
    \\
    \ell(\vf, \vy) &= - \log q(\giventhat{\vy}{\vf})\,.
  \end{align}
\end{subequations}
Both the square loss and softmax cross-entropy loss
(\Cref{ex:background::Regression,ex:background::Classification}) have such a
probabilistic interpretation, see
\Cref{ex:background::probabilisticInterpretationMSELoss,ex:background::probabilisticInterpretationCrossEntropyLoss}.

\begin{example}[\textbf{Probabilistic interpretation of square
    loss}]\label{ex:background::probabilisticInterpretationMSELoss} The square
  loss \Cref{eq:background::squareLoss} is the negative log-likelihood of a
  Gaussian centered around the model's prediction with diagonal constant
  covariance,
  \begin{align*}
    \ell(\vf, \vy) &= -\log q(\giventhat{\vy}{\vf})
    \\
    \text{with}\quad
    q(\giventhat{\vy}{f_{\vtheta}(\vx)}) &= \gN(\giventhat{\vy}{\vmu, \mSigma})
  \end{align*}
  where%
  \sidenote[][-4\baselineskip]{Inserting mean and covariance into the negative
    log-probability yields,
    \begin{align*}
      -& \log \gN(\vy; \vmu, \mSigma)
      \\
       &=
         \nicefrac{1}{2} (\vy - \vmu)^{\top} \mSigma^{-1} (\vy - \vmu)
      \\
       &\phantom{=}
         + \nicefrac{1}{2}
         \left[
         \log\det\mSigma
         + C \log 2\pi
         \right]\,,
    \end{align*}
    \ie the square loss \Cref{eq:background::squareLoss} up to a
    $\vtheta$-independent term which does not affect optimization.}
  %
  $\vmu = f_{\vtheta}(\vx)$ and $\mSigma = \nicefrac{C}{2} \mI$.
\end{example}

\begin{example}[\textbf{Probabilistic interpretation of softmax cross-entropy
    loss}]\label{ex:background::probabilisticInterpretationCrossEntropyLoss}
  The softmax cross-entropy loss \Cref{eq:background::softmaxCrossEntropyLoss}
  is the negative log-likelihood of a multinomial distribution parameterized by
  the softmax probabilities,
  \begin{align*}
    \ell(\vf, y) &= -\log q(\giventhat{y}{\vf})
    \\
    \text{with}\quad
    q(\giventhat{y}{f_{\vtheta}(\vx)}) &= \Cat(y; \vp)
  \end{align*}
  where%
  \sidenote[][-4\baselineskip]{ With $\evp_{c}$ denoting the probability to observe class $y=c$,
    \begin{align*}
      - \log \Cat(y; \vp) = -\log \evp_{y}\,,
    \end{align*}
    which is the softmax cross-entropy loss \Cref{eq:background::softmaxCrossEntropyLoss}.}
  %
  $\vp = \softmax(f_{\vtheta}(\vx))$.
\end{example}

Following the maximum likelihood principle results in the MLE parameter
$\vtheta_{\text{MLE}}$ which satisfies
\Cref{eq:background::mnimizeEmpiricalNegativeLogLikelihood} and gives rise to
the distribution $p_{\vtheta_{\text{MLE}}}(\giventhat{\vy}{\vx}) =
q(\giventhat{\vy}{f_{\vtheta_{\text{MLE}}}(\vx)})$ as an approximation to $\pdata(\giventhat{\vy}{\vx})$.

\subsubsection{Connections to Maximum A Posteriori (MAP) Estimation}
MLE maximizes the likelihood $ p(\giventhat{\sD}{\vtheta})$. In a probabilistic
formulation with a prior $p(\vtheta)$ over the parameters and evidence $p(\sD) =
\int_{\sTheta} p(\giventhat{\sD}{\vtheta}) p(\vtheta)\,\diff \vtheta$ for the
data, one can instead consider the posterior
\begin{align}
  \label{eq:background::BayesRule}
  p(\giventhat{\vtheta}{\sD})
  =
  \frac{
  p(\giventhat{\sD}{\vtheta}) p(\vtheta)
  }{
  p(\sD)
  }\,
\end{align}
which corrects the prior with data observations. Such a posterior is useful to
form probabilistic beliefs over predictions $\vy_{\star}$ for new inputs
$\vx_{\star}$,
\begin{align}
  \label{eq:background::BayesianPrediction}
  \begin{split}
    p(\giventhat{\vy_{\star}}{\vx_{\star}, \sD})
    &=
      \int_{\sTheta} p(\giventhat{\vy_{\star}}{\vx_{\star}, \sD, \vtheta})
      p(\giventhat{\vtheta}{\vx_{\star}, \sD})\,\diff \vtheta
    \\
    &=
      \int_{\sTheta} p(\vy_{\star} | \vx_{\star}, \vtheta)
      p(\giventhat{\vtheta}{\sD})\,\diff \vtheta\,.
  \end{split}
\end{align}
However, this requires integration over the posterior, which itself is almost
always intractable and thus needs to be approximated.

The MAP principle approximates the posterior with a delta distribution around
the posterior mode, \ie the point of maximum posterior density,
\begin{align}
  \label{eq:background::MAP}
  p(\giventhat{\vtheta}{\sD})
  \approx
  \delta(\vtheta - \vtheta_{\text{MAP}})
  \quad\text{where}\quad
  \vtheta_{\text{MAP}}
  =
  \argmax_{\vtheta} p(\giventhat{\vtheta}{\sD})\,.
\end{align}
It is connected to empirical risk minimization with a regularization term that
results from the prior $p(\vtheta)$. To see this, reformulate
\Cref{eq:background::MAP} to minimize the negative log-posterior, expand Baye's
rule (\Cref{eq:background::BayesRule}) and neglect the parameter-independent
evidence term. Then, apply the same assumptions as for MLE\sidenote{%
  Mathematically, they translate into
  \begin{align*}
    \begin{split}
      &p(\giventhat{\sD}{\vtheta})
      \\
      &\quad =
        \prod_{n}
        p(\giventhat{\vx_{n}, \vy_{n}}{\vtheta})
      \\
      &\quad =
        \prod_{n}
        p(\giventhat{\vy_{n}}{\vx_{n}, \vtheta}) p(\giventhat{\vx_{n}}{\vtheta})
      \\
      &\quad =
        \prod_{n}
        p(\giventhat{\vy_{n}}{\vx_n, \vtheta}) p(\vx_{n})\,
    \end{split}
  \end{align*}
  with slightly different notation for $\vtheta$ in comparison to the MLE
  discussion, as it is now treated probabilistically. }, \ie \iid data and
$\vtheta$ only parameterizing the likelihood $\giventhat{\vy}{\vx}$. This yields
\begin{align}
  \label{eq:background::MAPRelationToEmpiricalRiskMinimization}
  \begin{split}
    \vtheta_{\text{MAP}}
    &= \argmin_{\vtheta} -\log p(\giventhat{\vtheta}{\sD})
    \\
    &= \argmin_{\vtheta} -\log p(\giventhat{\sD}{\vtheta}) - \log p(\vtheta)
    \\
    &= \argmin_{\vtheta} \sum_{(\vx_n,\vy_n) \in \sD}
      -\log p(\giventhat{\vy_n}{\vx_n, \vtheta})
      -\log p(\vtheta)
    \\
    &= \argmin_{\vtheta}
      \frac{1}{|\sD|}
      \sum_{(\vx_n,\vy_n) \in \sD}
      -\log p(\giventhat{\vy_n}{\vx_n, \vtheta})
      -\frac{\log p(\vtheta)}{|\sD|}\,.
  \end{split}
\end{align}
In analogy to \Cref{eq:background::connectionsMLEandEmpiricalRiskMinimization},
the first term is an empirical risk
(\Cref{eq:background::mnimizeEmpiricalNegativeLogLikelihood}) with negative
log-likelihood loss of a distribution $q$ for targets given model predictions,
\begin{align}\label{eq:background::connectionsMAPandEmpiricalRiskMinimization}
  p(\giventhat{\vy}{\vx,\vtheta})
  = q(\giventhat{\vy}{f_{\vtheta}(\vx)})
  \quad\text{and}\quad
  \ell(\vf, \vy)
  = - \log q(\giventhat{\vy}{\vf})\,.
\end{align}
However, this risk is extended by a regularization term from the prior,
\begin{align}\label{eq:background::MAPLoss}
  \vtheta_{\text{MAP}}
  =
  \argmin_{\vtheta}
  \gL_\sD(\vtheta) + r(\vtheta)
  \quad
  \text{where}
  \quad
  r(\vtheta) = - \frac{\log p(\vtheta)}{|\sD|}\,.
\end{align}
\Cref{eq:background::MAPRelationToEmpiricalRiskMinimization,eq:background::MAPLoss,eq:background::BayesRule}
connect the posterior with the loss via $ \log p(\giventhat{\sD}{\vtheta}) = -
|\sD| \gL_{\sD}(\vtheta)$ and $\log p(\vtheta) = - |\sD| r(\vtheta)$,
\begin{align}\label{eq:background::RelationPosteriorLossRegularization}
  \begin{split}
    p(\giventhat{\vtheta}{\sD})
    &=
      \frac{
      \exp
      \left[
      \log p(\giventhat{\sD}{\vtheta})
      + \log p(\vtheta)
      \right]
      }{
      p(\sD)
      }
    \\
    &=
      \frac{
      \exp\left\{
      -|\sD| \left[
      \gL_{\sD}(\vtheta) + r(\vtheta)
      \right]
      \right\}
      }{p(\sD)}\,.
  \end{split}
\end{align}
It underlines the aforementioned challenges to track the posterior. The
exponent is non-linear in $\vtheta$, and $p(\sD)$ requires computing an
integral.

\Cref{eq:background::RelationPosteriorLossRegularization} gives rise to
posterior approximations that go beyond a delta distribution. The Laplace
approximation~\cite{laplace1774memoire}
(\Cref{sec:background::LaplaceApproximation}) also starts with the MAP estimate,
but uses a quadratic Taylor expansion of the log-posterior around
$\vtheta_{\text{MAP}}$ to approximate the posterior by a Gaussian. This
quadratic expansion requires higher-order information in form of second-order
derivatives, presented in \Cref{chap:background::HigherOrder}.

%%% Local Variables:
%%% mode: latex
%%% TeX-master: "../thesis"
%%% End:
