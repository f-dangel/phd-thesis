To illustrate the utility of \BackPACK, we implement preconditioned gradient
descent optimizers using diagonal and Kronecker approximations of the \GGN. To
our knowledge, and despite their apparent simplicity, results using diagonal
approximations or the naïve damping update rule we chose have not been reported
in publications so far. However, this section is not meant to introduce a
bona-fide new optimizer. Our goal is to show that \BackPACK can enable research
of this kind. The update rule we implement uses a curvature matrix
$\mC(\vtheta_t^{(l)})$, which could be a diagonal or Kronecker factorization of
the \GGN blocks, and a damping parameter $\lambda$ to precondition the gradient:
\begin{equation}
  \label{backpack::eq:update_rule}
  \vtheta_{t+1}^{(l)}
  =
  \vtheta_t^{(l)} - \eta \left(\mC(\vtheta_t^{(l)}) + \lambda \mI\right)^{-1}
  \grad{\vtheta_t^{(l)}}\Loss(\vtheta_t^{(l)})\equationPunctuation{,}
  \qquad l = 1,\dots, L\equationPunctuation{.}
\end{equation}
We run the update rule with the following approximations of the generalized
Gauss-Newton: the exact diagonal (\DiagGGN) and an \MC estimate (\DiagGGNMC),
and the Kronecker factorizations \KFAC \citep{martens2015optimizing}, \KFLR and
\KFRA\sidenote{ \KFRA was not originally designed for convolutions; we extend it
  using the Kronecker factorization of \citet{grosse2016kronecker}. While it can
  be computed for small networks on \MNIST, which we report in
  \Cref{backpack::app:additional-results}, the approximate backward pass of
  \KFRA does not seem to scale to large convolution layers. }
\citep{botev2017practical}.%
The inversion required by the update rule is straightforward for the diagonal
curvature. For the Kronecker-factored quantities, we use the approximation
introduced by \cite{martens2015optimizing} (see
\Cref{backpack::app:update_rule_details}).

\begin{figure*}[!t]
  \centering
  \pgfkeys{/pgfplots/BackPACKBenchBarplot/.style={
    width=1.02\linewidth,
    height=0.5\linewidth,
    grid=major,
    grid style = {dashed},
    every axis plot/.append style={line width = 1.5pt},
    tick pos = left,
    xtick align = outside,
    ytick align = inside,
    xmajorticks = true,
    ymajorticks = true,
    ylabel near ticks,
    xlabel near ticks,
    xticklabel style = {font = \footnotesize},
    xlabel style = {font = \footnotesize},
    axis line style = {black},
    yticklabel style = {font = \footnotesize},
    ylabel style = {font = \footnotesize},
    title style = {font = \footnotesize, inner ysep = 0ex},
    grid = major,
    grid style = {dashed},
    legend cell align = left,
    legend style = {
      fill opacity = 0.7,
      text opacity = 1,
      font = \footnotesize,
      legend columns = 4,
      legend cell align={left},
      legend image post style={opacity = 1, yscale = 1.1, line width=-1pt},
      column sep=0.25cm,
    },
  }
}

%%% Local Variables:
%%% mode: latex
%%% TeX-master: "../../thesis"
%%% End:

  \begin{subfigure}[t]{1.0\linewidth}
    \centering
    \caption{\threecthreed, \cifarten}\label{backpack::subfig:benchmark-all1}
    \vspace{-1ex}
    \pgfkeys{/pgfplots/zmystyle/.style={
        BackPACKBenchBarplot,
        height = 0.35\linewidth,
        title = \empty,
      }
    }
    \tikzexternalenable
    % This file was created by tikzplotlib v0.9.7.
\begin{tikzpicture}

\definecolor{color0}{rgb}{0.211764705882353,0.294117647058824,0.603921568627451}
\definecolor{color1}{rgb}{0.431372549019608,0.650980392156863,0.803921568627451}
\definecolor{color2}{rgb}{0.76078431372549,0.894117647058824,0.937254901960784}

\begin{axis}[
legend cell align={left},
legend columns=4,
legend style={fill opacity=0.8, draw opacity=1, text opacity=1, at={(0.03,0.97)}, anchor=north west, draw=white!80!black},
tick align=outside,
tick pos=left,
title={3C3D on CIFAR10},
x grid style={white!69.0196078431373!black},
xmin=-0.5, xmax=7.5,
xtick style={color=black},
xtick={0,1,2,3,4,5,6,7},
xticklabels={
Grad (ref.),2nd Moment,
Batch L2,DiagGGN-MC,
KFAC,Indiv. Grad,
KFLR,DiagGGN},
y grid style={white!69.0196078431373!black},
ylabel={Time [ms]},
ymin=0, ymax=50,
ytick style={color=black},
zmystyle
]
\draw[draw=none,fill=white!20!black] (axis cs:-0.375,0) rectangle (axis cs:-0.125,15.0664545000154);
\draw[draw=none,fill=color0] (axis cs:0.625,0) rectangle (axis cs:0.875,14.395405000073);
\addlegendimage{ybar,ybar legend,draw=none,fill=color0};
\addlegendentry{Batch = 32}

\draw[draw=none,fill=color0] (axis cs:1.625,0) rectangle (axis cs:1.875,15.2878979999969);
\draw[draw=none,fill=color0] (axis cs:2.625,0) rectangle (axis cs:2.875,19.3677970000863);
\draw[draw=none,fill=color0] (axis cs:3.625,0) rectangle (axis cs:3.875,16.8086539999877);
\draw[draw=none,fill=color0] (axis cs:4.625,0) rectangle (axis cs:4.875,15.2265290000742);
\draw[draw=none,fill=color0] (axis cs:5.625,0) rectangle (axis cs:5.875,21.8555310000283);
\draw[draw=none,fill=color0] (axis cs:6.625,0) rectangle (axis cs:6.875,22.4338249999505);
\draw[draw=none,fill=white!40!black] (axis cs:-0.125,0) rectangle (axis cs:0.125,20.5737690000092);
\draw[draw=none,fill=color1] (axis cs:0.875,0) rectangle (axis cs:1.125,20.6966069999908);
\addlegendimage{ybar,ybar legend,draw=none,fill=color1};
\addlegendentry{48}

\draw[draw=none,fill=color1] (axis cs:1.875,0) rectangle (axis cs:2.125,20.9739174999868);
\draw[draw=none,fill=color1] (axis cs:2.875,0) rectangle (axis cs:3.125,25.4147299999659);
\draw[draw=none,fill=color1] (axis cs:3.875,0) rectangle (axis cs:4.125,22.383585);
\draw[draw=none,fill=color1] (axis cs:4.875,0) rectangle (axis cs:5.125,20.8398080000052);
\draw[draw=none,fill=color1] (axis cs:5.875,0) rectangle (axis cs:6.125,30.4188420000173);
\draw[draw=none,fill=color1] (axis cs:6.875,0) rectangle (axis cs:7.125,33.1303219999199);
\draw[draw=none,fill=white!60!black] (axis cs:0.125,0) rectangle (axis cs:0.375,26.3089910000645);
\draw[draw=none,fill=color2] (axis cs:1.125,0) rectangle (axis cs:1.375,26.2560775000225);
\addlegendimage{ybar,ybar legend,draw=none,fill=color2};
\addlegendentry{64}

\draw[draw=none,fill=color2] (axis cs:2.125,0) rectangle (axis cs:2.375,28.7451689999898);
\draw[draw=none,fill=color2] (axis cs:3.125,0) rectangle (axis cs:3.375,32.0420350000177);
\draw[draw=none,fill=color2] (axis cs:4.125,0) rectangle (axis cs:4.375,28.0495179999889);
\draw[draw=none,fill=color2] (axis cs:5.125,0) rectangle (axis cs:5.375,30.8939720000581);
\draw[draw=none,fill=color2] (axis cs:6.125,0) rectangle (axis cs:6.375,37.0278835000022);
\draw[draw=none,fill=color2] (axis cs:7.125,0) rectangle (axis cs:7.375,47.5826415000711);
\addplot [semithick, white!20!black, opacity=0.4, forget plot]
table {%
-0.5 15.0664545000154
7.5 15.0664545000154
};
\addplot [semithick, white!40!black, opacity=0.4, forget plot]
table {%
-0.5 20.5737690000092
7.5 20.5737690000092
};
\addplot [semithick, white!60!black, opacity=0.4, forget plot]
table {%
-0.5 26.3089910000645
7.5 26.3089910000645
};
\end{axis}

\end{tikzpicture}

    \tikzexternaldisable
  \end{subfigure}
  \begin{subfigure}[t]{1.0\linewidth}
    \centering
    \caption{\allcnnc, \cifarhun}\label{backpack::subfig:benchmark-all2}
    \vspace{-1ex}
    \pgfkeys{/pgfplots/zmystyle/.style={
        BackPACKBenchBarplot,
        height = 0.35\linewidth,
        ylabel = {Time [ms]},
        title = \empty,
      }
    }
    \tikzexternalenable
    \hspace{-1ex}% This file was created by tikzplotlib v0.9.7.
\begin{tikzpicture}

\definecolor{color0}{rgb}{0.211764705882353,0.294117647058824,0.603921568627451}
\definecolor{color1}{rgb}{0.431372549019608,0.650980392156863,0.803921568627451}
\definecolor{color2}{rgb}{0.76078431372549,0.894117647058824,0.937254901960784}

\begin{axis}[
legend cell align={left},
legend columns=4,
legend style={fill opacity=0.8, draw opacity=1, text opacity=1, at={(0.03,0.97)}, anchor=north west, draw=white!80!black},
tick align=outside,
tick pos=left,
title={All-CNN-C on CIFAR100},
x grid style={white!69.0196078431373!black},
xmin=-0.5, xmax=5.5,
xtick style={color=black},
xtick={0,1,2,3,4,5},
xticklabels={
Grad (ref.),2nd Moment,
Batch L2,DiagGGN-MC,
KFAC,Indiv. Grad},
y grid style={white!69.0196078431373!black},
ymin=0, ymax=113.550937500011,
ytick style={color=black},
zmystyle
]
\draw[draw=none,fill=white!20!black] (axis cs:-0.375,0) rectangle (axis cs:-0.125,17.0513680000113);
\draw[draw=none,fill=color0] (axis cs:0.625,0) rectangle (axis cs:0.875,23.2273660000146);
\addlegendimage{ybar,ybar legend,draw=none,fill=color0};
\addlegendentry{Batch = 32}

\draw[draw=none,fill=color0] (axis cs:1.625,0) rectangle (axis cs:1.875,29.9791224999808);
\draw[draw=none,fill=color0] (axis cs:2.625,0) rectangle (axis cs:2.875,27.1620119999625);
\draw[draw=none,fill=color0] (axis cs:3.625,0) rectangle (axis cs:3.875,38.6299799999961);
\draw[draw=none,fill=color0] (axis cs:4.625,0) rectangle (axis cs:4.875,55.4445244999897);
\draw[draw=none,fill=white!40!black] (axis cs:-0.125,0) rectangle (axis cs:0.125,23.4135965000064);
\draw[draw=none,fill=color1] (axis cs:0.875,0) rectangle (axis cs:1.125,32.411140000022);
\addlegendimage{ybar,ybar legend,draw=none,fill=color1};
\addlegendentry{48}

\draw[draw=none,fill=color1] (axis cs:1.875,0) rectangle (axis cs:2.125,39.3030079999335);
\draw[draw=none,fill=color1] (axis cs:2.875,0) rectangle (axis cs:3.125,38.0351860000019);
\draw[draw=none,fill=color1] (axis cs:3.875,0) rectangle (axis cs:4.125,53.98943900002);
\draw[draw=none,fill=color1] (axis cs:4.875,0) rectangle (axis cs:5.125,81.7799224999476);
\draw[draw=none,fill=white!60!black] (axis cs:0.125,0) rectangle (axis cs:0.375,30.1599859999442);
\draw[draw=none,fill=color2] (axis cs:1.125,0) rectangle (axis cs:1.375,41.9460425000011);
\addlegendimage{ybar,ybar legend,draw=none,fill=color2};
\addlegendentry{64}

\draw[draw=none,fill=color2] (axis cs:2.125,0) rectangle (axis cs:2.375,49.1684415000009);
\draw[draw=none,fill=color2] (axis cs:3.125,0) rectangle (axis cs:3.375,48.6743639999077);
\draw[draw=none,fill=color2] (axis cs:4.125,0) rectangle (axis cs:4.375,69.9318230000472);
\draw[draw=none,fill=color2] (axis cs:5.125,0) rectangle (axis cs:5.375,108.143750000011);
\addplot [semithick, white!20!black, opacity=0.4, forget plot]
table {%
-0.5 17.0513680000113
5.5 17.0513680000113
};
\addplot [semithick, white!40!black, opacity=0.4, forget plot]
table {%
-0.5 23.4135965000064
5.5 23.4135965000064
};
\addplot [semithick, white!60!black, opacity=0.4, forget plot]
table {%
-0.5 30.1599859999442
5.5 30.1599859999442
};
\end{axis}

\end{tikzpicture}

    \tikzexternaldisable
  \end{subfigure}
  \caption{ \textbf{Overhead benchmark for computing the gradient \emph{and}
      first- or second-order extensions on real networks, compared to just the
      gradient.} Most quantities add little overhead. \KFLR and \DiagGGN
    propagate 100$\times$ more information than \KFAC and \DiagGGNMC on
    \CIFAR{100} and are two orders of magnitude slower. We report benchmarks on
    those, and the Hessian's diagonal, in \Cref{backpack::app:benchmark}. }
  \label{backpack::fig:benchmark-all}
\end{figure*}


\begin{figure*}[!t]
  \centering
  \begin{subfigure}[t]{1.0\linewidth}
    \centering
    \caption{}\label{backpack::subfig:cifar1}
    \vspace{-5.5ex}
    \begin{center}\textbf{\footnotesize \CIFARTEN : \CIFARTENNET}\end{center}
    \tikzexternalenable
    \plotDifferentCurvatures{cifar10}{3c3d}{const}{final}{valid}{accuracies}
    \tikzexternaldisable
  \end{subfigure}
  \vspace{-2ex}
  \begin{subfigure}[t]{1.0\linewidth}
    \centering
    \caption{}\label{backpack::subfig:cifar2}
    \vspace{-5.5ex}
    \begin{center}\textbf{\footnotesize \CIFARHUN : \ALLCNNC}\end{center}
    \tikzexternalenable
    \plotDifferentCurvatures{cifar100}{allcnnc}{const}{final}{valid}{accuracies}
    \tikzexternaldisable
  \end{subfigure}
  \caption{\textbf{Median performance with shaded quartiles of the \DeepOBS
      benchmark} for \subfigref{backpack::subfig:cifar1} \CIFARTENNET (895,210
    parameters) on \CIFARTEN and \subfigref{backpack::subfig:cifar1} \ALLCNNC
    (1,387,108 parameters) on \CIFARHUN. Solid lines show \DeepOBS' baselines of
    momentum \SGD and Adam.}
  \label{backpack::fig:cifar}
\end{figure*}

These curvature estimates are tested for the training of deep neural networks by
running the corresponding optimizers on the main test problems of the
benchmarking suite \DeepOBS\sidenote{%
  \href{https://deepobs.github.io/}{\texttt{deepobs.github.io}}. We cannot run
  \BackPACK on all test problems in this benchmark due to the limitations
  outlined in \Cref{backpack::sec:theory-and-implementation}. Despite this
  limitation, we still run on models spanning a representative range of image
  classification problems. } \citep{schneider2019deepobs}. We use the setup
(batch size, number of training epochs) of \DeepOBS' baselines, and tune the
learning rate $\eta$ and damping parameter $\lambda$ with a grid search for
each optimizer (details in \Cref{backpack::app:grid-search}). The best
hyperparameter settings is chosen according to the final accuracy on a
validation set. We report the median and quartiles of the performance for ten
random seeds.

\Cref{backpack::subfig:cifar1} shows the results for the \CIFARTENNET network
trained on \CIFAR{10}. The optimizers that leverage Kronecker-factored curvature
approximations beat the baseline performance in terms of per-iteration progress
on the training loss, training and test accuracy. Using the same
hyperparameters, there is little difference between \KFAC and \KFLR, or \DiagGGN
and \DiagGGNMC. Given that the quantities based on \MC-sampling are considerably
cheaper, this experiment suggests it being an important technique to reduce
the computational burden of curvature proxies.

\Cref{backpack::subfig:cifar2} shows benchmarks for the \ALLCNNC network trained
on \CIFAR{100}. Due to the high-dimensional output, the curvatures using a full
matrix propagation rather than an \MC sample cannot be run on this problem due
to memory issues. Both \DiagGGNMC and \KFAC can compete with the baselines in
terms of progress per iteration. As the update rule we implemented is simplistic
on purpose, this is promising for future applications of second-order methods
that can more efficiently use the additional information given by curvature
approximations.

%%% Local Variables:
%%% mode: latex
%%% TeX-master: "../thesis"
%%% End:
