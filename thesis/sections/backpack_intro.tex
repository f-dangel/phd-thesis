The success of deep learning and the applications it fuels can be traced to the
popularization of automatic differentiation frameworks. Packages like
\TensorFlow~\citep{abadi2016tensorflow}, \Chainer~\citep{tokui2015chainer},
\MXNet~\citep{chen2015mxnet}, and \PyTorch~\citep{paszke2019pytorch} provide
efficient implementations of parallel, GPU-based gradient computations to a wide
range of users, with elegant syntactic sugar.

However, this specialization also has its shortcomings: it assumes the user only
wants to compute gradients or, more precisely, the average of gradients across a
mini-batch of examples. Other quantities can also be computed with AD at a
comparable cost or minimal overhead to the gradient backpropagation pass; for
example, approximate second-order information or the variance of gradients
within the batch. These quantities are valuable to understand the geometry of
deep neural networks, for the identification of free parameters, and to push the
development of more efficient optimization algorithms. But researchers who want
to investigate their use face a chicken-and-egg problem: AD tools required to go
beyond standard gradient methods are not available, but there is no incentive
for their implementation in existing deep-learning software as long as no large
portion of the users need it.

Second-order methods for deep learning have been continuously investigated for
decades \citep[\eg][]{becker1988improving,amari1998natural,bordes2009sgdqn,
  martens2015optimizing}. But still, the standard optimizers used in deep
learning remain some variant of stochastic gradient descent (\SGD); more complex
methods have not found wide-spread, practical use. This is in stark contrast to
domains like convex optimization and generalized linear models, where
second-order methods are the default. There may of course be good scientific
reasons for this difference; maybe second-order methods do not work well in the
(non-convex, stochastic) setting of deep learning. And the computational cost
associated with the high dimensionality of deep models may offset their
benefits. Whether these are the case remains somewhat unclear though, because a
much more direct road-block is that these methods are so complex to implement
that few practitioners ever try them out.

Recent approximate second-order methods such as \KFAC
\citep{martens2015optimizing} show promising results, even on hard deep learning
problems \citep{tsuji2019performance}. Their approach, based on the earlier work
of \citet{schraudolph2002fast}, uses the structure of the network to compute
approximate second-order information in a way that is similar to gradient
backpropagation. This work sparked a new line of research to improve the
second-order approximation \citep{grosse2016kronecker, botev2017practical,
  martens2018kronecker, george2018fast}. However, all of these methods require
low-level applications of automatic differentiation to compute quantities other
than the averaged gradient. It is a daunting task to implement them from
scratch. Unless users spend significant time familiarizing themselves with the
internals of their software tools, the resulting implementation is often
inefficient, which also puts the original usability advantage of those packages
into question. Even motivated researchers trying to develop new methods, who
need not be expert software developers, face this problem. They often end up
with methods that cannot compete in run time, not necessarily because the method
is inherently bad, but because the implementation is not efficient. New methods
are also frequently not compared to their predecessors and competitors because
they are so hard to reproduce. Authors do not want to represent the competition
in an unfair light caused by a bad implementation.

Another example is offered by a recent string of research to adapt to the
\emph{stochasticity} induced by mini-batch sampling. An empirical estimate of
the (marginal) variance of the gradients within the batch has been found to be
theoretically and practically useful for adapting hyperparameters like learning
rates \citep{mahsereci2017probabilistic} and batch sizes
\citep{balles2017coupling}, or regularize first-order optimization
\citep{leroux2007topmoumoute, balles2018dissecting, katharopoulos2018samples}.
To get such a variance estimate, one simply has to square, then sum, the
individual gradients after the backpropagation, but before they are aggregated
to form the average gradient. Doing so should have negligible cost \emph{in
  principle}, but is programmatically challenging in the standard packages.

Members of the community have repeatedly asked for such features\sidenote{See
  \eg the Github issues
  \href{https://github.com/pytorch/pytorch/issues/1407}{\nolinkurl{github.com/pytorch/pytorch/issues/1407}},
  \href{https://github.com/pytorch/pytorch/issues/7786}{\nolinkurl{7786}},
  \href{https://github.com/pytorch/pytorch/issues/8897}{\nolinkurl{8897}} and
  forum discussions
  \href{https://discuss.pytorch.org/t/1433}{\nolinkurl{discuss.pytorch.org/t/1433}},
  \href{https://discuss.pytorch.org/t/8405}{\nolinkurl{8405}},
  \href{https://discuss.pytorch.org/t/15270}{\nolinkurl{15270}},
  \href{https://discuss.pytorch.org/t/17204}{\nolinkurl{17204}},
  \href{https://discuss.pytorch.org/t/19350}{\nolinkurl{19350}},
  \href{https://discuss.pytorch.org/t/24955}{\nolinkurl{24955}}. } %
but the established AD frameworks have yet to address such requests, as their
focus has been---rightly---on improving their technical backbone. Features like
those outlined above are not generally defined for arbitrary functions, but
rather emerge from the specific structure of machine learning applications.
General AD frameworks can not be expected to serve such specialist needs. This
does not mean, however, that it is impossible to efficiently realize such
features within these frameworks: in essence, backpropagation is a technique to
compute multiplications with Jacobians. Methods to extract second-order
information \citep{mizutani2008secondorder} or individual gradients from a
mini-batch \citep{goodfellow2015efficient} have been known to a small group of
specialists; they are just rarely discussed or implemented.

\subsection{Our Contribution}
To address this need for a specialized framework focused on machine learning, we
propose a framework for the implementation of generalized backpropagation to
compute additional quantities. The structure is based on the conceptual work of
\citet{dangel2020modular} for modular backpropagation. This framework can be
built on top of existing graph-based backpropagation modules; we provide an
implementation on top of \PyTorch, coined \BackPACK, available at
~\\[-1.75em]
\begin{center}
  \url{https://f-dangel.github.io/backpack/}.
\end{center}
~\\[-1.75em]
The initial release supports efficient computation of individual gradients from
a mini-batch, their $L_2$ norm, an estimate of the variance, as well as diagonal
and Kronecker factorizations of the generalized Gauss-Newton (\GGN) matrix
(see~\Cref{backpack::tab:features_backpack} for a feature overview). The library
was designed to be minimally verbose to the user, easy to use
(see~\Cref{fig:backpack-code-sample}), and to have low overhead (see
\Cref{backpack::sec:benchmark}). While other researchers are aiming to improve
the flexibility of AD systems \citep{innes2018flux, innes2018zygote,
  bradbury2018jax}, our goal with this package is to provide access to
quantities that are only byproducts of the backpropagation pass, rather than
gradients themselves.

\begin{figure*}[t]
  \centering
  \hfill
  \begin{minipage}[t]{.48\linewidth}
    \textbf{\!Computing the gradient with \pytorchtitle\ldots}\\[1em]
    \footnotesize
    \texttt{%
      X, y~~~~~= load\_mnist\_data()\\
      model~~~~= Linear(784, 10)\\
      lossfunc~= CrossEntropyLoss()\\
      ~\\
      loss~~~~~= lossfunc(model(X), y)\\
      ~\\
      loss.backward()\\
      ~\\
      for param in model.parameters():~\\
      \null\hspace{2.5em}print(param.grad)
    }
  \end{minipage}\hspace{-0.5em}\vline\hfill
  \begin{minipage}[t]{.48\linewidth}
    \textbf{\!\ldots and the variance with \backpacktitle}\\[1em]
    \footnotesize
    \texttt{%
      X, y~~~~~= load\_mnist\_data()\\
      model~~~~= \textbf{\color{maincolor} extend(}Linear(784, 10)\textbf{\color{maincolor})}\\
      lossfunc~= \textbf{\color{maincolor} extend(}CrossEntropyLoss()\textbf{\color{maincolor})}\\
      ~\\
      loss~~~~~= lossfunc(model(X), y)\\
      \textbf{\color{maincolor} with backpack(Variance()):}\\
      \null\hspace{2.5em}loss.backward()\\
      ~\\
      for param in model.parameters():~\\
      \null\hspace{2.5em}print(param.grad)\\
      \textbf{\color{maincolor} \null\hspace{2.5em}print(param.var)}
    }
  \end{minipage}
  \vspace{-2ex}
  \captionof{lstlisting}{ \textbf{\BackPACK integrates with \PyTorch to
      seamlessly extract more information from the backward pass.} Instead of
    the variance (or alongside it, in the same pass), \BackPACK can compute
    individual gradients in the mini-batch, their $L_2$ norm and
    2\textsuperscript{nd} moment. It can also compute curvature approximations
    like diagonal or Kronecker factorizations of the \GGN such as \KFAC, \KFLR
    \& \KFRA. }
  \label{fig:backpack-code-sample}
\end{figure*}

%%% Local Variables:
%%% mode: latex
%%% TeX-master: "../../thesis"
%%% End:


To illustrate the capabilities of \BackPACK, we use it to implement
preconditioned gradient descent optimizers with diagonal approximations of the
\GGN and recent Kronecker factorizations \KFAC \citep{martens2015optimizing},
\KFLR, and \KFRA \citep{botev2017practical}. Our results show that the curvature
approximations based on Monte Carlo (\MC) estimates of the \GGN, the approach
used by \KFAC, give similar progress per iteration to their more accurate
counterparts, but being much cheaper to compute. While the naïve update rule we
implement does not surpass first-order baselines such as \SGD with momentum and
Adam \citep{kingma2015adam}, its implementation with various curvature
approximations is made straightforward.

%%% Local Variables:
%%% mode: latex
%%% TeX-master: "../thesis"
%%% End:
