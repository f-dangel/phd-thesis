The modular decomposition of curvature backpropagation facilitates the analysis
of modules composed of multiple operations. Now, we analyze the composition of
two modules. This yields the recursive schemes presented by
\citet{botev2017practical} (KFRA) and \citet{wei2018bdapch} (BDA-PCH).

\begin{figure}[t]
  \centering
  \begin{subfigure}[t]{\linewidth}
    \centering
    \tikzexternalenable
    \resizebox{!}{3.25cm}{\footnotesize% KFRA composition before

\begin{tikzpicture}
    % first layers
    \node (in1)
        [inner sep=0]
        {\tikz \drawMessagesWithArrows{$x$}{$\delta x$}{$\HeCal  x$}{\hNodeDistance};};
    \node (activation)
        [anchor=south west, inner sep=0]
        at (in1.south east)
        {\tikz\drawModuleNoParams{$\phi$}{5};}; 
    \node (out)
        [inner sep=0, anchor=south west]
        at (activation.south east)
        {\tikz \drawMessagesWithArrows{$y$}{$\delta y$}{$\HeCal  y$}{\hNodeDistance};};      
    \node (layer1)
        [anchor=south west, inner sep=0]
         at (out.south east)
         {\tikz \drawModuleWithParams{$z = Wy$}{16}{\hspace{1.5ex}$ W$\hspace{1.5ex} }{\,$\delta  W$\,\,}{$\HeCal W$};};
    \node (layer2)
        [inner sep=0pt, anchor=south west]
        at (layer1.south east)
        {\tikz\drawModuleWithParams{$+ b$}{16}{$b$}{$\delta b$}{$\HeCal  b$};};
    \node (out2)
        [inner sep=0, anchor=south west]
        at (layer2.south east)
        {\tikz \drawMessagesWithArrows{$z$}{$\delta z$}{$\HeCal  z$}{\hNodeDistance};};       
\end{tikzpicture}	}
    \tikzexternaldisable
    \caption{Activation and linear layer as two separate modules}
    \label{hbp::subfig:composition1}
  \end{subfigure}

  \begin{subfigure}{\linewidth}
    \centering
    \tikzexternalenable
    \hspace{0.14\linewidth}\resizebox{!}{3.25cm}{\footnotesize% KFRA composition after

\begin{tikzpicture}
    % first layers
    \node (in1)
        [inner sep=0]
        {\tikz \drawMessagesWithArrows{$\vx$}{$\grad{\vx}\ell$}{$\gradsquared{\vx}\ell$}{\hNodeDistance};};
    \node (activation)
        [anchor=south west, inner sep=0]
        at (in1.south east)
        {\tikz\drawModuleNoParams{$\vphi$}{5};};
    \node (layer1)
        [anchor=south west, inner sep=0]
         at (activation.south east)
         {\tikz \drawModuleWithParams{$\vz = \mW\vy$}{16}{\hspace{1.5ex}$ \mW$\hspace{1.5ex} }{\,$\grad{\mW}\ell$\,\,}{$\gradsquared{\mW}\ell$};};
    \node (layer2)
        [inner sep=0pt, anchor=south west]
        at (layer1.south east)
        {\tikz\drawModuleWithParams{$+ \vb$}{16}{$\vb$}{$\grad{\vb}\ell$}{$\gradsquared{\vb}\ell$};};
    \node (out2)
        [inner sep=0, anchor=south west]
        at (layer2.south east)
        {\tikz \drawMessagesWithArrows{$\vz$}{$\grad{\vz}\ell$}{$\gradsquared{\vz}\ell$}{\hNodeDistance};};
\end{tikzpicture}
%%% Local Variables:
%%% mode: latex
%%% TeX-master: "../../thesis"
%%% End:
}
    \tikzexternaldisable
    \caption{Activation and linear layer fused into a single module}
    \label{hbp::subfig:composition2}
  \end{subfigure}
  \caption{\textbf{The sequence of elementwise activation $\vphi$ and linear
      layer can be interpreted as two modules, or a single one.}
    \subfigref{hbp::subfig:composition1} Both operations are analyzed separately
    to derive the HBP. \subfigref{hbp::subfig:composition2} Backpropagation of
    $\gradsquared{\vz}\ell$ is expressed in terms of $\gradsquared{\vx}\ell$
    without intermediate message.}
  \label{hbp::fig:composition}
\end{figure}

\subsubsection{Analytic Composition of Multiple Modules}
Consider the module $\vg = \vf \circ \vphi, \vx \mapsto \vy := \vphi(\vx)
\mapsto \vz = \vf(\vy(\vx))$. Assume $\vphi$ to act elementwise on the input,
followed by a linear layer $\vf: \vz(\vy) = \mW\vy + \vb$
(\Cref{hbp::subfig:composition1}). Analytic elimination of the intermediate
backward pass recovers the backward pass of the fused module that consists of
both operations (\Cref{hbp::subfig:composition2}). The first Hessian backward
pass through the linear module $\vf$
(\Cref{hbp::equ:linearHessianBackpropagation}) implies
\begin{subequations}\label{hbp::equ:compositionCurvatureBackpropagation}
  \begin{alignat}{2}
    \nonumber \gradsquared{\vy}\ell
    &=
      \mW^\top
      \left(\gradsquared{\vz}\ell\right)
      \mW\,,
    \\
    \gradsquared{\mW}\ell
    &=
      \vy \otimes \vy^\top \otimes \gradsquared{\vz}\ell
      =
      \vphi(\vx) \otimes \vphi(\vx)^\top \otimes \gradsquared{\vz}\ell\,,
      \label{hbp::equ:compositionWeightHessian}
    \\
    \gradsquared{\vb}\ell
    &=
      \gradsquared{\vz}\ell\,.
      \label{hbp::equ:compositionBiasHessian}
  \end{alignat}
  Further backpropagation through $\vphi$ with
  \Cref{hbp::equ:nonlinearHessianBackpropagation} results in
  \begin{align}
    \begin{split}
      \gradsquared{\vx}\ell
      &=
        \diag\left[\vphi'(\vx)\right]
        \left(\gradsquared{\vy}\ell\right)
        \diag\left[\vphi'(\vx)\right]
        +
        \diag\left[\vphi''(\vx) \odot (\grad{\vy}\ell) \right]
      \\
      &=
        \diag\left[\vphi'(\vx)\right]
        \left[\mW^\top \left(\gradsquared{\vz}\ell \right) \mW \right]
        \diag\left[\vphi'(\vx)\right]
      \\
      &\phantom{=}\,
        +
        \diag\left[\vphi''(\vx) \odot \mW^\top (\grad{\vz}\ell) \right]
      \\
      &=
        \left\{ \mW \diag\left[\vphi'(\vx)\right] \right\}^\top
        \left(\gradsquared{\vz}\ell\right)
        \left\{ \mW \diag\left[\vphi'(\vx)\right] \right\}
      \\
      &\phantom{=}\,
        +
        \diag\left[\vphi''(\vx) \odot \mW^\top (\grad{\vz}\ell) \right]\,.
    \end{split}
        \label{hbp::equ:compositionInputHessian}
  \end{align}
\end{subequations}
We use invariance of a diagonal matrix under transposition and $\grad{\vy}\ell =
\mW^\top (\grad{\vz}\ell)$ for the backpropagated gradient for the last
equality. The Jacobian $\jac_{\vx} \vg(\vx)$ of the module shown in
\Cref{hbp::subfig:composition2} is $\jac_{\vx}\vg(\vx) = \mW \diag [ \vphi(x) ]
= [ \mW^\top \odot \vphi'(\vx) ]^\top$ (broadcasting $\vphi'(\vx))$. In summary,
HBP for the composite layer $\vz(\vx) = \mW \vphi(\vx)+ \vb$ is given
by~\Cref{hbp::equ:compositionCurvatureBackpropagation}.

\subsubsection{Obtaining the Relations of KFRA and BDA-PCH}

The derivations for the composite module given above are closely related to the
recursive schemes of \citet{botev2017practical,wei2018bdapch}. Their relations
are obtained from a straightforward conversion of the HBP
rules~\eqref{hbp::equ:compositionCurvatureBackpropagation}. Consider a sequence
of a linear layer $f_{\vtheta^{(1)}}^{(1)}$ and multiple composite modules
$f_{\vtheta^{(2)}}^{(2)}, \dots, f_{\vtheta^{(L)}}^{(L)}$ as shown in
\Cref{hbp::fig:groupedScheme}.

According to \Cref{hbp::equ:compositionBiasHessian} both the linear layer and
the composite $f_{\vtheta^{(l)}}^{(l)}$ identify the gradient (Hessian) \wrt
their outputs, $\grad{\vz^{(l)}} \ell$ ($\gradsquared{\vz^{(l)}}\ell$), as the
gradient (Hessian) \wrt their bias term, $\grad{\vb^{(l)}}\ell$ ($\gradsquared{
  \vb^{(l)}}\ell$). Introducing layer indices for all quantities, one finds the
recursion
\begin{subequations}
  \label{hbp::equ:kfraAndBdapch}
  \begin{alignat}{2}
    \gradsquared{\vb^{(l)}}\ell
    &=
      \gradsquared{\vz^{(l)}}\ell \,,
    \\
    \gradsquared{\mW^{(l)}}\ell
    &=
      \vphi(\vz^{(l-1)}) \otimes \vphi(\vz^{(l-1)})^\top \otimes \gradsquared{\vb^{(l)}}\ell \,,
    \\
    \intertext{for $l = L-1, \ldots, 1$, and}
    \begin{split}
      \gradsquared{\vz^{(l-1)}}\ell
      &=
        \left\{ \mW^{(l)} \diag\left[\vphi'(\vz^{(l-1)})\right] \right\}^\top
        \gradsquared{\vb^{(l)}}\ell
        \left\{ \mW^{(l)} \diag\left[\vphi'(\vz^{(l - 1)})\right] \right\}
      \\
      &\phantom{=}\,
        +
        \diag\left[\vphi''(\vz^{(l - 1)}) \odot \mW^{(l)\top} \grad{\vb^{(l)}}\ell  \right]
      \\
      &=
        \left\{ \mW^{(l)\top} \odot \vphi'(\vz^{(l-1)}) \right\}
        \gradsquared{\vb^{(l)}}\ell
        \left\{ \mW^{(l)\top} \odot \vphi'(\vz^{(l-1)}) \right\}^\top
      \\
      &\phantom{=}\,
        +
        \diag\left[\vphi''(\vz^{(l-1)}) \odot \mW^{(l)\top} \grad{\vb^{(l)}}\ell  \right]
    \end{split}
  \end{alignat}
\end{subequations}
for $l = L - 1, \ldots, 2$. It is initialized with the loss function's gradient
(Hessian) $\grad{\vz^{(L)}}\ell$ ($\gradsquared{\vz^{(L)}}\ell$).

\begin{figure*}[t]
  \tikzexternalenable
  \centering \resizebox{!}{3.25cm}{% HBP scheme for KFRA and BDA-PCH

\begin{tikzpicture}
    % first layers
    \node (in1)
        [inner sep=0]
        {\tikz \drawMessagesWithArrows{$z^{(0)}$}{ }{ }{\hNodeDistance};};
    \node (layer1)
        [anchor=south west, inner sep=0]
         at (in1.south east)
         {\tikz \drawModuleWithParams{$f^{(1)}$}{16}{$\theta^{(1)}$}{$\delta \theta^{(1)}$}{$\HeCal \theta^{(1)}$};};
    \node (out1)
        [inner sep=0, anchor=south west]
        at (layer1.south east)
        {\tikz \drawMessagesWithArrows{$z^{(1)}$}{$\delta z^{(1)}$}{$\HeCal  z^{(1)}$}{\hNodeDistance};};
    \node (layer2)
        [inner sep=0pt, anchor=south west]
        at (out1.south east)
        {\tikz\drawModuleNoParams{$\phi$}{5};};
    \node (layer3)
        [anchor=south west, inner sep=0]
        at (layer2.south east)
        {\tikz \drawModuleWithParams{$f^{(2)}$}{16}{$\theta^{(2)}$}{$\delta \theta^{(2)}$}{$\HeCal \theta^{(2)}$};};
    \node (out3)
        [inner sep=0, anchor=south west]
        at (layer3.south east)
        {\tikz \drawMessagesWithArrows{$z^{(2)}$}{$\delta z^{(2)}$}{$\HeCal  z^{(2)}$}{\hNodeDistance};};
    \node (dots) [xshift=1.5ex, inner sep=0pt, anchor=west] at (out3.east) {$\dots$};
    
    % layers after dot
    \node (layer4)
        [xshift=7ex, inner sep=0pt, anchor=south west]
        at (out3.south east)
        {\tikz\drawModuleNoParams{$\phi$}{5};};	
    \node (layer5)
        [anchor=south west, inner sep=0]
        at (layer4.south east)
        {\tikz \drawModuleWithParams{$f^{(\ell)}$}{16}{$\theta^{(\ell)}$}{$\delta \theta^{(\ell)}$}{$\HeCal \theta^{(\ell)}$};};
    \node (out4)
        [inner sep=0, anchor=south west]
        at (layer5.south east)
         {\tikz \drawMessagesWithArrows{$z^{(\ell)}$}{$\delta z^{(\ell)}$}{$\HeCal  z^{(\ell)}$}{\hNodeDistance};};
    
    % loss layer
    \node (lossLayer)
        [inner sep=0pt, anchor=south west]
        at (out4.south east)
        {\tikz\drawModuleNoParams{$E$}{5};};
    \node (loss)
        [inner sep=0, anchor=south west]
        at (lossLayer.south east)
        {\tikz \drawMessagesWithArrows{$E$}{ }{ }{\hNodeDistance};};	
\end{tikzpicture}	
}
  \tikzexternaldisable
  \caption{\textbf{Grouping scheme for the recursive Hessian computation
      proposed by KFRA and BDA-PCH.} The backward messages between the linear
    layer and the preceding nonlinear activation are analytically
    fused.}\label{hbp::fig:groupedScheme}
\end{figure*}

\Cref{hbp::equ:kfraAndBdapch} are equivalent to the expressions provided in
\cite{botev2017practical, wei2018bdapch}. Their emergence from compositions of
HBP equations of simple operations represents one key insight of
\Cref{chap:hbp}. Both works use the batch average strategy presented in
\Cref{hbp::subsec:batchLearning} to obtain curvature estimates.

%%% Local Variables:
%%% mode: latex
%%% TeX-master: "../thesis"
%%% End:
