Second-order methods iteratively optimize an objective $\gL(\vtheta): \sR^{D}
\to \sR$ using a local quadratic approximation $\gL(\vtheta) \approx
m_{\vtheta_t}(\vtheta)$ around the current point $\vtheta_t$ after $t$
iterations,
\begin{subequations}
  \begin{align}\label{eq:background::localQuadraticApproximation}
    m_{\vtheta_t}(\vtheta)
    =
    a(\vtheta_t)
    +
    {\vb(\vtheta_t)}^{\top}(\vtheta - \vtheta_t)
    +
    \frac{1}{2}
    (\vtheta - \vtheta_t)^{\top}
    \mC(\vtheta_t)
    (\vtheta - \vtheta_t)\,,
  \end{align}
  with an offset $a(\vtheta_t) \in \sR$, a slope vector $\vb(\vtheta_t) \in
  \sR^{D}$, and a curvature matrix $\mC(\vtheta_t) \in \sR^{D \times D}$. As
  described in \Cref{sec:background::PopularOptimizers}, the next iterate
  $\vtheta_{t+1}$ is obtained by minimizing the local approximation. For this
  proxy to possess a minimum, the curvature matrix $\mC(\vtheta_t)$ must be PD.
  Then, \Cref{eq:background::localQuadraticApproximation} is minimized by
  \begin{align}\label{eq:background::SecondOrderUpdateInverseCurvature}
    \vtheta_{t+1}
    =
    \argmin_{\vtheta} m_{\vtheta_t}(\vtheta)
    =
    \vtheta_t - {\mC(\vtheta_t)}^{-1} \vb(\vtheta_t)\,.
  \end{align}
  This update is computationally challenging, because the size of the curvature
  matrix is quadratic in $D$ and generally infeasible to store. Additionally,
  the computational complexity of matrix inversion scales cubically in $D$.
  These problems can somewhat be addressed by approximately solving the linear
  system
  \begin{align}\label{eq:background::SecondOrderMethodsLinearSystem}
    \mC(\vtheta_t) \vtheta_{t+1} = - \vb(\vtheta_t)
  \end{align}
\end{subequations}
for $\vtheta_{t+1}$. This can be done with iterative solvers, such as CG, which
only require matrix-vector products with $\mC(\vtheta_t)$, which can often be
implemented without expanding the matrix representation in memory.

The first-order methods from \Cref{sec:background::PopularOptimizers} circumvent
these issues by using a diagonal---and often quite crude---curvature
approximation that is cheap to store and invert (see~\cite{choi2020on} for an
overview). \Eg, \sgd's local approximation,
\Cref{eq:background::SGDLocalApproximation}, uses $\nicefrac{1}{2\eta}\mI$ as
curvature matrix. The following sections introduce common curvature matrices for
local approximations of empirical risks.

\subsection{Newton's Method \& the Hessian
  Matrix}\label{sec:background::NewtonMethodAndHessian}
The Taylor series provides a meaningful local approximation of an analytic
function. Its expansion up to second-order is
\begin{align}
  \begin{split}
    \gL(\vtheta)
    &=
      \underbrace{
      \gL(\vtheta_t)
      }_{a(\vtheta_t)}
      +
      {
      \underbrace{
      \grad{\vtheta_{t}}\gL(\vtheta_t)
      }_{\vb_{\vtheta_t}}
      }^{\top}
      ( \vtheta - \vtheta_t)
    \\
    &\phantom{= }
      +
      \frac{1}{2}
      {( \vtheta - \vtheta_t)}^{\top}
      \underbrace{
      \gradsquared{\vtheta_t}\gL(\vtheta_t)
      }_{\mC(\vtheta_t)}
      ( \vtheta - \vtheta_t)
      +
      \gO\left( {(\vtheta - \vtheta_t)}^{3} \right)
  \end{split}
\end{align}
where $\gO((\vtheta - \vtheta_t)^{3})$ denotes polynomial terms in the
components of $\vtheta - \vtheta_t$ of cubic order and above. This reveals the
objective's Hessian $\gradsquared{\vtheta_t}\gL(\vtheta_t)$, which collects its
second-order partial derivatives in a matrix
${[\gradsquared{\vtheta_t}\gL(\vtheta_t)]}_{i,j} = \nicefrac{\partial^2
  \gL(\vtheta_t)}{\partial {[\vtheta_t]}_i \partial {[\vtheta_t]}_j}$, as
curvature matrix.

\begin{definition}[\textbf{Hessian}]\label{def:background::Hessian}
  Let $b: \sR^D \to \sR; \va \mapsto b(\va)$ be a differentiable
  vector-to-scalar function. The Hessian $\gradsquared{\va} b \in \sR^{D\times
    D}$ of $b$ \wrt $\va$ is a symmetric matrix containing the second-order
  partial derivatives
  \begin{align}
    \label{eq:background::HessianVectorToScalar}
    \gradsquared{\va}b = \frac{\partial^2 b}{\partial \va \partial \va^{\top}}
    \qquad
    \text{with}
    \qquad
    [\gradsquared{\va}b]_{i,j} = \frac{\partial^2 b}{\partial \eva_i \partial \eva_j}
  \end{align}
  \Cref{hbp::def:generalizedHessian} generalizes this to the matrix/tensor case.
  The Hessian will often be denoted by $\mH$. \Eg $\mH_{\pdata}(\vtheta) :=
  \gradsquared{\vtheta} \gL_{\pdata}(\vtheta)$ for the Hessian of the population
  risk \Cref{eq:background::expectedRisk}, and $\mH_{\sD}(\vtheta) :=
  \gradsquared{\vtheta} \gL_{\sD}(\vtheta)$ for the Hessian of the empirical
  risk \Cref{eq:background::empiricalRisk} on a dataset $\sD$ (with $\sD =
  \Dtrain,\sB$ for the train loss and mini-batch Hessian).
\end{definition}

\marginnote[*-5]{%
  \begin{definition}[\textbf{Generalized
      Hessian}]\label{hbp::def:generalizedHessian}
    Let $\mB: \mathbb{R}^{n\times q} \to \mathbb{R}^{m \times p}$ be a twice
    differentiable matrix function. The \emph{Hessian} $\gradsquared{\mA} \mB(\mA)$
    is an $(mnpq \times nq)$ matrix defined by
    \begin{align}
      \label{hbp::equ:generalizedHessian}
      \begin{split}
        &\gradsquared{\mA} \mB(\mA)
        \\
        &= \jac_{\mA} \left[\jac_{\mA} \mB(\mA)\right]^\top
        \\
        &= \frac{
          \partial
          }{
          \partial (\vec \mA)^\top
          }
          \vec \left\{
          \left[
          \frac{\partial \vec \mB(\mA)}{\partial (\vec
          \mA)^\top}
          \right]^\top
          \right\}
      \end{split}
    \end{align}
    \citep[][Chapter 10.2]{magnus1999MatrixDifferentialCalculus} with
    flattening defined by \Cref{def:background::Flattening}. In
    element-wise notation, this is
    the matrix-stack of all output component Hessians
    \begin{align*}
      \gradsquared{\mA}
      \mB(\mA)
      =
      \begin{pmatrix}
        \gradsquared{\vec \mA} [\vec \mB]_1
        \\
        \gradsquared{\vec \mA} [\vec \mB]_2
        \\
        \vdots
        \\
        \gradsquared{\vec \mA} [\vec \mB]_{np}
      \end{pmatrix}\,,
    \end{align*}
    with the Hessian from \Cref{def:background::Hessian}.
  \end{definition}
  The tensor case is analogous but requires cluttered notation and is therefore
  omitted. Common forms for neural networks include vector-to-vector functions
  $f: \sR^m \to \sR^n, \vx \mapsto f(\vx)$ with
  \begin{align*}
    \gradsquared{\vx} f(\vx)
    =
    \frac{\partial^2 f(\vx)}{\partial \vx^\top \partial \vx}\,,
  \end{align*}
  where $\vx$ can be considered the input or bias vector if a linear layer. In
  details,
  \begin{align}
    \label{hbp::equ:HessianVectorToVector}
    \gradsquared{\vx} f(\vx)
    =
    \begin{pmatrix}
      \gradsquared{\vx} f_1(\vx)
      \\
      \vdots
      \\
      \gradsquared{\vx} f_m(\vx)
    \end{pmatrix}\,.
  \end{align}
  Others are matrix-to-vector mappings $f:\sR^{n\times q}\to \sR^m, \mX
  \to f(\mX)$ with
  \begin{align*}
    \gradsquared{\mX} f(\mX)
    =
    \frac{\partial}{\partial (\vec \mX)^\top}\frac{\partial f(\mX)}{\partial \vec \mX}\,,
  \end{align*}
  \eg with $\mX$ the weight of a linear layer.%
}

Newton's method uses the Hessian as curvature matrix.

\begin{updaterule}[\textbf{Newton's method (simplified)}]\label{opt:background::Newton}
  A Newton step is
  \begin{align}
    \vtheta_{t+1} = \vtheta_t - {\mH_{\sD}(\vtheta_t)}^{-1} \vg_{\sD}(\vtheta_t)
  \end{align}
  with the gradient and Hessian from
  \Cref{def:background::JacobianVectorVector,def:background::Hessian} (practical
  implementations vary and often introduce additional hyperparameters such as a
  learning rate, damping term, mini-batch size, \etc).
\end{updaterule}

Matrix-free multiplication with the Hessian~\cite{pearlmutter1994fast} can been
combined with CG to compute Newton steps via solving the linear system
\Cref{eq:background::SecondOrderMethodsLinearSystem}. This idea is known as
Hessian-free optimization~\cite{martens2010deep}.

While the Taylor expansion motivates using the Hessian as curvature matrix, it
leads to problems for non-convex functions like the empirical risk, as it is in
general indefinite. Therefore
\Cref{eq:background::SecondOrderUpdateInverseCurvature} does not have a
solution. In practice, PSD curvature matrices that are approximations of the
Hessian are popular substitutes for the Hessian to avoid this issue
(\Cref{sec:background::gn,sec:background::ggn,sec:background::naturalGradientDescent}).

\subsection{The Gauss-Newton Method \& Matrix}\label{sec:background::gn}

The Gauss-Newton (GN) method~\cite[chapter 6.3]{bottou2016machine} tackles the
nonlinear least squares regression task (\Cref{ex:background::Regression}) of
fitting a function $f_{\vtheta}$ to data in the form of vector-valued inputs
$\vx_n \in \sR^M$ and scalar-valued outputs $y_n \in \sR$ by minimizing the mean
squared error,
\begin{align}\label{eq:background::nonlinearLeastSquares}
  \minimize_{\vtheta} \gL(\vtheta)\,,
  \quad
  \text{where}
  \quad
  \gL(\vtheta)
  =
  \frac{1}{|\sD|}
  \sum_{(\vx_n, y_n) \in \sD}
  {(
  \underbrace{f_{\vtheta}(\vx_n)}_{:=f_n} - y_n
  )}^{2}\,,
\end{align}
The objective's gradient is
\begin{align}
  \label{eq:background::GradientNonlinearLeastSquares}
  \grad{\vtheta}\gL(\vtheta)
  =
  \frac{2}{|\sD|}
  \sum_{(\vx_n, y_n) \in \sD}
  {(\jac_{\vtheta} f_n)}^{\top}
  \left(
  f_n - y_{n}
  \right)\,,
\end{align}
with the Jacobian $\jac_{\vtheta} f_n \in \sR^{1 \times D}$
(\Cref{def:background::JacobianVectorVector}). The Hessian is
\begin{subequations}\label{eq:background::HessianDecompositionGaussNewtonResidual}
  \begin{align}
    \label{eq:background::HessianNonlinearLeastSquares}
    % \begin{split}
    \gradsquared{\vtheta} \gL(\vtheta)
    &=
      \frac{2}{|\sD|}
      \sum_{(\vx_n, y_n) \in \sD}
      {(\jac_{\vtheta}f_n)}^{\top}
      \jac_{\vtheta}f_n
      % \\
      % &\phantom{= }
          +
          \frac{2}{|\sD|}
          \sum_{(\vx_n, y_n) \in \sD}
          \gradsquared{\vtheta} f_n
          \left(
          f_n - y_n
          \right)\,.
          % \end{split}
          \intertext{The first term is the PSD
          \emph{Gauss-Newton matrix} (up to scaling)}
          \label{eq:background::GaussNewton}
          \mG_{\sD}(\vtheta)
        &=
          \frac{2}{|\sD|}
          \sum_{(\vx_n, y_n) \in \sD}
          {(\jac_{\vtheta}f_n)}^{\top}
          \jac_{\vtheta}f_n\,,
          \intertext{and the second term is the residual matrix}
          \mR_{\sD}(\vtheta)
        &=
          \frac{2}{|\sD|}
          \sum_{(\vx_n, y_n) \in \sD}
          \gradsquared{\vtheta} f_n
          \left(
          f_n - y_n
          \right)\,.
  \end{align}
\end{subequations}
The \emph{Gauss-Newton matrix} approximates the Hessian through first-order
information of the model, ${(\jac_{\vtheta}f_n)}^{\top} = \grad{\vtheta}f_n$,
which is cheap to compute. For vanishing residual terms $\lim_{(f_n - y_n)\to 0}
$ (as the model predictions match the labels), or linear models
($\gradsquared{\vtheta} f_{\vtheta} = \vzero$) it corresponds to the Hessian.

\begin{updaterule}[\textbf{Gauss-Newton method (simplified)}]
  For the nonlinear least squares task \Cref{eq:background::nonlinearLeastSquares}, a Gauss-Newton step is%
  \sidenote{ Stacking Jacobians and residuals,
    \begin{align*}
      \mJ(\vtheta)
      &=
        \begin{pmatrix}
          \jac_{\vtheta}f_1
          \\
          \jac_{\vtheta}f_2
          \\
          \vdots
          \\
          \jac_{\vtheta}f_{|\sD|}
        \end{pmatrix} \in \sR^{|\sD| \times D}\,,
      \\
      \vr(\vtheta)
      &=
        \begin{pmatrix}
          f_1 - y_1
          \\
          f_2 - y_2
          \\
          \vdots
          \\
          f_{|\sD|} - y_{|\sD|}
        \end{pmatrix} \in \sR^{|\sD|}\,,
    \end{align*}
    absorbs the sums into matrix multiplies
    \begin{align*}
      \vtheta_{t+1}
      =
      \vtheta_t
      -
      {\left(
      {\mJ(\vtheta_t)}^{\top}
      \mJ(\vtheta_t)
      \right)}^{-1}
      {\mJ(\vtheta_t)}^{\top}
      \vr(\vtheta_t)
    \end{align*}
    which can be solved through
    \begin{align*}
      {\mJ(\vtheta_t)}^{\top}
      \mJ(\vtheta_t)
      \vx
      =
      -
      {\mJ(\vtheta_t)}^{\top}
      \vr(\vtheta_t)
    \end{align*}
    via JVPs \& VJPs in combination with CG. }
  %
  \begin{align}
    \vtheta_{t+1}
    =
    \vtheta_t
    -
    {\mG_{\sD}(\vtheta_t)}^{-1}
    \grad{\vtheta_t}\gL(\vtheta_t)\,,
  \end{align}
  with the gradient and Gauss-Newton matrix from
  \Cref{def:background::JacobianVectorVector,eq:background::GaussNewton}
  (practical implementations vary and often introduce additional hyperparameters
  such as a learning rate, damping term, mini-batch size, \etc).
\end{updaterule}

% Positive semi-definite Hessian approximations
\subsection{The Generalized Gauss-Newton Matrix}\label{sec:background::ggn}

The \emph{generalized Gauss-Newton (\ggn) matrix} is a PSD approximation to the
Hessian that generalizes the GN through abstraction via empirical risk
minimization (see \Cref{sec:background::empiricalRiskMinimization}). The \ggn
can be understood through different perspectives, and, using the probabilistic
interpretation of the model, is related to the natural gradient method through
its connections to the Fisher (\Cref{sec:background::naturalGradientDescent}) .

\subsubsection{From Gauss-Newton to Generalized Gauss-Newton}

The GN matrix from \Cref{sec:background::ggn} stems from a nonlinear least
squares problem that can be viewed as supervised regression
(\Cref{ex:background::Regression}), \ie mean squared error loss function, with
scalar-valued labels ($C=1$). For the general case of empirical risk
minimization (\Cref{eq:background::empiricalRisk}), the Hessian decomposes due to
the split between model and loss function,%
\begin{subequations}
  \begin{align}
    \ell(\vtheta)
    &=
      \ell(\cdot, \vy) \circ f_{\vtheta}(\vx)
      \intertext{as a result of the Hessian chain rule \Cref{hbp::the:chainRuleHessians}.
      For a single datum $(\vx, \vy)$ with prediction $\vf := f_{\vtheta}(\vx)$, this yields}
      \label{eq:background::chainRuleModelLossFunctionSplit}
      \gradsquared{\vtheta} \ell(\vtheta)
    &=
      (\jac_{\vtheta}\vf)^{\top}
      \left[
      \gradsquared{\vf} \ell(\vf, \vy)
      \right]
      \jac_{\vtheta}\vf
      +
      \sum_{c=1}^{C}
      \left(
      \gradsquared{\vtheta}[\vf]_{c}
      \right)
      \grad{\vf} \ell(\vf, \vy)
  \end{align}
\end{subequations}
\marginnote{%
  The arrangement of partial derivatives in the generalizations of Jacobian
  \Cref{hbp::def:generalizedJacobian} and Hessian
  \Cref{hbp::def:generalizedHessian} implies the following chain rule
  generalization for second-order derivatives:
  \begin{theorem}[\textbf{Chain rule for the generalized
      Hessian}]\label{hbp::the:chainRuleHessians}
    Let $\vb: \mathbb{R}^n \to \mathbb{R}^m$ and $\vc: \mathbb{R}^m \to
    \mathbb{R}^p$ be twice differentiable and $\vd = \vc \circ \vb: \mathbb{R}^n
    \to \mathbb{R}^p, \va \mapsto \vd(\va) = \vc(\vb(\va))$. The relation
    between the Hessian of $\vd$ and the Jacobians and Hessians of the
    constituents $\vc$ and $\vb$ is given by
    \begin{align}
      \label{hbp::equ:chainRuleHessians}
      \begin{split}
        &\gradsquared{\va}
          \vd(\va)
        \\
        &\,\,=
          \left[
          \mI_p \otimes \jac_{\va} \vb(\va)
          \right]^\top
          \left[
          \gradsquared{\vb} \vc(\vb)
          \right]
          \jac_{\va} \vb(\va)
        \\
        &\,\,\phantom{= }
          +
          \left[
          \jac_{\vb} \vc(\vb) \otimes \mI_n
          \right]
          \gradsquared{\va} \vb(\va)
      \end{split}
    \end{align}
    \citep[restricted from][Chapter 6.10]{magnus1999MatrixDifferentialCalculus}.
  \end{theorem}
  The matrix/vector case is analogous. Matrix differential
  calculus~\cite{magnus1999MatrixDifferentialCalculus} is a useful tool to
  easily read off the Hessian from specific expressions, see
  \Cref{hbp::sec:matrixDifferentialCalculus}. For applications of the Hessian
  chain rule, see \Cref{chap:hbp}. }%
The first term carries curvature information of the loss function, while the
second term contains curvature information of the model. Because $\ell(\vf,
\vy)$ is convex in $\vf$, the first term is PSD, whereas the second term is
indefinite in general.

The \ggn is the first term and neglects curvature information of the model. For
the empirical risk \Cref{eq:background::empiricalRisk}, and using the shorthand
$\vf_n := f_{\vtheta}(\vx_n)$, the Hessian is
\begin{subequations}\label{eq:background::HessianDecompositionGGNResidual}
  \begin{align}
    \mH_{\sD}(\vtheta)
    &=
      \frac{1}{|\sD|}
      \sum_{(\vx_n, \vy_n) \in \sD}
      \gradsquared{\vtheta} \ell(\vf_n, \vy_n)
      = \mG_{\sD}(\vtheta) + \sR_{\sD}(\vtheta)
      \shortintertext{with the \emph{generalized Gauss-Newton matrix}}
      \label{eq:background::generalizedGaussNewton}
      \mG_{\sD}(\vtheta)
    &:=
      \frac{1}{|\sD|}
      \sum_{(\vx_n, \vy_n) \in \sD}
      (\jac_{\vtheta}\vf_n)^{\top}
      \left[
      \gradsquared{\vf_n} \ell(\vf_n, \vy_n)
      \right]
      \jac_{\vtheta}\vf_n\,
      \shortintertext{and the \emph{residual matrix}}
      \mR_{\sD}(\vtheta)
    &:=
      \frac{1}{|\sD|}
      \sum_{(\vx_n, \vy_n) \in \sD}
      \sum_{c=1}^C
      \left(
      \gradsquared{\vtheta}[\vf_n]_c
      \right)
      \grad{\vf_n} \ell(\vf_n, \vy_n)\,.
  \end{align}
\end{subequations}
This decomposition reduces to
\Cref{eq:background::HessianDecompositionGaussNewtonResidual} for nonlinear
least squares, where $\gradsquared{\vf}(\vx) \ell(\vf, \vy) = \nicefrac{2}{C}
\mI$ (\Cref{hbp::table:backpropEquations}), $\grad{\vf}\ell(\vf, \vy) =
\nicefrac{2}{C} (\vf - \vy)$ (\Cref{tab:background::Jacobians}), and $C=1$.
Therefore, the \ggn is a generalization of the GN via the chain rule applied to
the model-loss function split.

\subsubsection{From Linearization to Generalized Gauss-Newton}

Alternatively, one can replace the model in $\ell(f_{\vtheta}(\vx), \vy)$ with a
linear Taylor expansion around $\vtheta$,
\begin{align*}
  f_{\vtheta}(\vx)
  \leftrightarrow
  \hat{f}_{\vtheta'}(\vx)
  =
  f_{\vtheta}(\vx)
  +
  \left[\jac_{\vtheta}f_{\vtheta}(\vx)\right] (\vtheta' - \vtheta)\,.
\end{align*}
This eliminates second-order terms in the model, \ie
$\gradsquared{\vtheta'}\hat{f}_{\vtheta'}(\vx) = \vzero$. Application of the
Hessian chain rule to the loss with a linearized model,
\begin{subequations}
  \begin{align}
    \hat{\ell}(\vtheta')
    &=
      \ell(\cdot, \vy) \circ \hat{f}_{\vtheta'}(\vx)\,,
      \shortintertext{
      and using the shorthand $\hat{f}_{\vtheta'}(\vx) :=\vfhat$, yields
      }
      \gradsquared{\vtheta'} \hat{\ell}(\vtheta')
    &=
      \left(\jac_{\vtheta'}\vfhat\right)^{\top}
      \left[
      \gradsquared{\vfhat} \ell(\vfhat, \vy)
      \right]
      \jac_{\vtheta'}\vfhat\,.
      \intertext{At the expansion point where model predictions and Jacobians match, this yields the first term of the decomposition \Cref{eq:background::chainRuleModelLossFunctionSplit}}
      \left.\left(\gradsquared{\vtheta'} \ell(\hat{f}_{\vtheta'}(\vx), \vy)\right)\right|_{\vtheta' = \vtheta}
    &=
      \left(\jac_{\vtheta}\vf\right)^{\top}
      \left[
      \gradsquared{\vf} \ell(\vf, \vy)
      \right]
      \jac_{\vtheta}\vf\,.
      \shortintertext{
      This carries over the empirical risk $\hat{\gL}(\vtheta') = \nicefrac{1}{|\sD|}
      \sum_{(\vx_n,\vy_n) \in \sD} \hat{\ell}(\vtheta')$ under a linearized model, whose Hessian is the \ggn from \Cref{eq:background::generalizedGaussNewton},}
      \left.\left(\gradsquared{\vtheta'} \hat{\gL}(\vtheta')\right)\right|_{\vtheta' = \vtheta}
    &=
      \mG_{\sD}(\vtheta)
  \end{align}
\end{subequations}
Hence, the \ggn is a Hessian approximation that neglects curvature from the
model, and becomes equivalent to the Hessian for linear models.

\subsection{Natural Gradient Descent \& the
  Fisher}\label{sec:background::naturalGradientDescent}

Natural gradient descent~\cite[][NGD]{amari1998natural} uses the Fisher
information matrix as curvature matrix. The natural gradient provides the
direction of steepest in the space of probability distributions described by a
model (\Cref{fig:background::sketchNGDSGDsteepestDescent}, recall the
probabilistic interpretation of certain empirical risks from
\Cref{sec:background::ProbabilisticInterpretation}). The Fisher is connected to
the \ggn from \Cref{sec:background::ggn} and for most models used in modern ML,
is equivalent but provides a probabilistic interpretation.

\subsubsection{Steepest Descent Direction \& Notion of Distance}

\marginnote{%
  \begin{center}
    \tikzexternalenable
    \begin{tikzpicture}
      \pgfmathsetmacro{\figwidth}{\linewidth}
      \pgfmathsetmacro{\figheight}{0.8\linewidth}
      \clip (0, 0) rectangle (\figwidth pt, \figheight pt);
      %
      \node [anchor = north, font=\footnotesize, minimum width=\figwidth pt] (spaceName) at
      (0.5*\figwidth pt, \figheight pt) {Parameter space $\sTheta$};
      %
      \fill (0.3*\figwidth pt, 0.3*\figheight pt) circle (0.4ex) coordinate (theta);
      \node [anchor=north east, font=\footnotesize, inner sep=2pt] at (theta) {$\vtheta$};
      %
      \begin{pgfonlayer}{background}
        \fill [thirdcolor!25] (0, 0) rectangle (spaceName.south east);
        %
        \node [circle, minimum width=7ex, fill=maincolor!25, draw=maincolor, dashed, ultra thick] (ball)
        at (theta) {};
        %
        \node [anchor=north, font=\footnotesize, maincolor]
        at (ball.south) {$\{\vtheta\mathbf{'}\,|\,d(\vtheta, \vtheta\mathbf{'}) \le \epsilon\}$};
        %
        \draw [->, >=stealth, secondcolor, ultra thick]
        (theta) to node [midway, below, sloped, font=\footnotesize] {$-\grad{\vtheta}\gL(\vtheta)$} ++(0.5*\figwidth pt, 0.3*\figheight pt);
      \end{pgfonlayer}
      %
      \node[anchor = north west, ellipse, minimum height=3.5ex, minimum width = 6ex, fill=maincolor!25, draw=maincolor, dashed, ultra thick, xshift=1.5ex, yshift = -1ex] (ellipse)
      at (spaceName.south west) {};
      \node [anchor = west, font=\footnotesize, maincolor] (distance) at (ellipse.east) {$d(\vtheta, \vtheta\mathbf{'}) = \lVert \vtheta - \vtheta\mathbf{'}\rVert_2$};
      %
      \fill ($(ellipse)!0.33!(ellipse.west)-(0,0.4ex)$) circle (0.4ex) coordinate (thetaEllipse);
      \fill ($(ellipse)!0.33!(ellipse.east)-(0,0.4ex)$) circle (0.4ex) coordinate (thetaPrimeEllipse);
      %
      \draw [thick] ($(thetaEllipse)+(0,1ex)$) to ++(0,0.4ex) to ++(0,-0.8ex) to ++(0,0.4ex) to ($(thetaPrimeEllipse)+(0,1ex)$) to ++(0,0.4ex) to ++(0,-0.8ex) to ++(0,0.4ex);
    \end{tikzpicture}%
    \tikzexternaldisable
    \newline
    \vspace{2ex}
    \tikzexternalenable
    \begin{tikzpicture}
      \pgfmathsetmacro{\figwidth}{\linewidth}
      \pgfmathsetmacro{\figheight}{0.8\linewidth}
      \clip (0, 0) rectangle (\figwidth pt, \figheight pt);
      %
      \node [anchor = north, font=\footnotesize, minimum width=\figwidth pt] (spaceName) at
      (0.5*\figwidth pt, \figheight pt) {Distribution space $\{p_{\vtheta}\,|\,\vtheta \in \sTheta\}$};
      %
      \fill (0.3*\figwidth pt, 0.3*\figheight pt) circle (0.4ex) coordinate (theta);
      \node [anchor=north east, font=\footnotesize, inner sep=2pt] at (theta) {$p_{\vtheta}$};
      %
      \begin{pgfonlayer}{background}
        \fill [thirdcolor!25] (0, 0) rectangle (spaceName.south east);
        %
        \node [circle, minimum width=7.5ex, fill=maincolor!25, draw=maincolor, dashed, ultra thick] (ball)
        at (theta) {};
        %
        \node [anchor=north, font=\footnotesize, maincolor]
        at (ball.south) {$\{p_{\vtheta\mathbf{'}}\,|\,d(p_{\vtheta}, p_{\vtheta\mathbf{'}}) \le \epsilon^2\}$};
        %
        \draw [->, >=stealth, secondcolor, ultra thick]
        (theta) to node [midway, above, sloped, font=\footnotesize] {$-\naturalgrad{\vtheta}\gL(p_{\vtheta})$} ++(0.45*\figwidth pt, -0.05*\figheight pt);
      \end{pgfonlayer}
      %
      \node[anchor = north west, ellipse, minimum height=3.5ex, minimum width = 6ex, fill=maincolor!25, draw=maincolor, dashed, ultra thick, xshift=1.5ex, yshift = -1ex] (ellipse)
      at (spaceName.south west) {};
      \node [anchor = west, font=\footnotesize, maincolor] (distance) at (ellipse.east) {$d(p_{\vtheta}, p_{\vtheta\mathbf{'}}) = \lVert \vtheta - \vtheta\mathbf{'}\rVert_{\mF(\vtheta)}$};
      \node [anchor = north west, font=\footnotesize, maincolor] at (distance.south west) {$\phantom{d(p_{\vtheta}, p_{\vtheta\mathbf{'}})} \approx \KLdiv{p_{\vtheta}}{p_{\vtheta\mathbf{'}}}$};
      %
      \fill ($(ellipse)!0.33!(ellipse.west)-(0,0.4ex)$) circle (0.4ex) coordinate (thetaEllipse);
      \fill ($(ellipse)!0.33!(ellipse.east)-(0,0.4ex)$) circle (0.4ex) coordinate (thetaPrimeEllipse);
      %
      \draw [thick] ($(thetaEllipse)+(0,1ex)$) to ++(0,0.4ex) to ++(0,-0.8ex) to ++(0,0.4ex) to ($(thetaPrimeEllipse)+(0,1ex)$) to ++(0,0.4ex) to ++(0,-0.8ex) to ++(0,0.4ex);
    \end{tikzpicture}%
    \tikzexternaldisable
  \end{center}
  \captionof{figure}{\textbf{Gradient descent and natural gradient descent via
      steepest descent.} Gradient descent (\emph{top}) follows the direction of
    steepest descent in Euclidean parameter space. NGD (\emph{bottom}) considers
    the space of distributions, where local distances are measured via a
    quadratic expansion of the KL divergence, which gives rise to the Fisher.
    Details in the text. Figure inspired by
    \cite{martens2014new}.}\label{fig:background::sketchNGDSGDsteepestDescent}
}

\paragraph{Concept (steepest descent direction):} Consider an arbitrary
objective function $\gL: \sTheta \to \sR$ with $\sTheta = \sR^D$. At a location
$\vtheta \in \sTheta$, the steepest descent direction is the direction in which
the objective increases at the fastest rate, \ie per infinitesimally small
distance moved. Formally,
\begin{align}
  \label{eq:background::steepestDirection}
  \Delta \vtheta^{(\text{steepest})}
  =
  \lim_{\epsilon \to 0} \frac{1}{\epsilon}
  \argmin_{\substack{\Delta \vtheta \\ d(\vtheta, \vtheta + \Delta \vtheta) \le \epsilon}}
  \gL(\vtheta + \Delta \vtheta)
\end{align}
and depends on the distance measure $d(\cdot, \cdot)$ between elements in a
small neighborhood around $\vtheta$.

\paragraph{Gradient descent as steepest descent:} Using the Euclidean 2-norm to
measure distances in $\sTheta$ via $d(\vtheta_1, \vtheta_2) = \lVert \vtheta_1 -
\vtheta_2 \rVert_2$, the steepest descent direction points along the negative
gradient $-\grad{\vtheta}\gL(\vtheta)$~\cite[][Chapter 6]{martens2014new}\,,
\begin{align}\label{eq:background::negativeGradientSteepestDescent}
  \lim_{\epsilon \to 0} \frac{1}{\epsilon}
  \argmin_{\Delta \vtheta: \lVert \Delta \vtheta \rVert_2 \le \epsilon}
  \gL(\vtheta + \Delta \vtheta)
  &=
    \frac{
    - \grad{\vtheta}\gL(\vtheta)
    }{
    \lVert \grad{\vtheta}\gL(\vtheta) \rVert_2
    }\,.
\end{align}

\paragraph{Natural gradient as steepest descent:}
\Cref{sec:background::ProbabilisticInterpretation} showed that in many tasks,
such as regression (\Cref{ex:background::Regression}) and softmax cross-entropy
classification (\Cref{ex:background::Classification}), the parameters $\vtheta$
model a probability distribution $p_{\vtheta}(\vz)$ over a random variable $\vz
\in \sOmega$. One could therefore establish a different notion of distance by
comparing probability distributions. The KL divergence is a similarity measure
between densities, but it is not a proper metric; \eg it is not symmetric in its
arguments. The steepest descent direction only requires measuring distances
within an infinitesimal ball though. For this purpose, a metric---described by
the Fisher---can be established through Taylor expansion of the KL divergence.

Starting from the KL divergence between two infinitesimally close
distributions $p_{\vtheta}(\vz), p_{\vtheta + \Delta\vtheta}(\vz)$,
\begin{align}
  \label{eq:background::KLDivergenceTaylor}
  \KLdiv{p_{\vtheta + \Delta \vtheta}}{p_{\vtheta}}
  =
  \int_{\sOmega} p_{\vtheta}(\vz)
  \left[
  \log p_{\vtheta}(\vz) - \log p_{\vtheta + \Delta \vtheta}(\vz)
  \right]
  \diff\vz
\end{align}
the first step is to Taylor-expand the logarithm around $\vtheta$,
\begin{align*}
  \log p_{\vtheta+\Delta \vtheta}(\vz)
  &=
    \log p_{\vtheta}(\vz)
    +
    ({\Delta \vtheta})^{\top} \grad{\vtheta} \log p_{\vtheta}(\vz)
  \\&\phantom{= }
  +
  \frac{1}{2} ({\Delta \vtheta})^{\top} \gradsquared{\vtheta} \log p_{\vtheta}(\vz) (\Delta \vtheta)
  + \gO\left((\Delta\vtheta)^3\right)\,.
\end{align*}
Inserting this into \Cref{eq:background::KLDivergenceTaylor} results in
%
\marginnote{
  \begin{remark}[\textbf{The log-probability's $\vtheta$-gradient vanishes in
      expectation}]\label{note:background::KLTaylorFirstOrderTermVanishes}
    \hspace{-\baselineskip}
    \begin{align*}
      -\int_{\sOmega}
      & p_{\vtheta}(\vz)
        \grad{\vtheta} \log p_{\vtheta}(\vz)
        \,\diff\vz
      \\
      &=
        -\int_{\sOmega} p_{\vtheta}(\vz)
        \frac{\grad{\vtheta}p_{\vtheta}(\vz)}{p_{\vtheta}(\vz)}
        \,\diff\vz
      \\
      &=
        - \grad{\vtheta}
        \left(
        \int_{\sOmega}
        p_{\vtheta}(\vz)
        \,\diff\vz
        \right)
      \\
      &=
        -\grad{\vtheta} 1 = 0
    \end{align*}
  \end{remark}
}
%
\begin{align*}
  \KLdiv{p_{\vtheta}(\vz)}{p_{\vtheta + \Delta \vtheta}(\vz)}
  &=
    - {\Delta \vtheta}^{\top}
    \underbrace{
    \int_{\sOmega} p_{\vtheta}(\vz)
    \grad{\vtheta} \log p_{\vtheta}(\vz)
    \,\diff\vz}_{\text{$=0$, see \Cref{note:background::KLTaylorFirstOrderTermVanishes}}}
  \\
  &\phantom{= }\
    -\frac{1}{2}{\Delta \vtheta}^{\top}
    \left(
    \int_{\sOmega} p_{\vtheta}(\vz)
    \gradsquared{\vtheta} \log p_{\vtheta}(\vz)
    \,\diff\vz
    \right)
    \Delta \vtheta
  \\
  &\phantom{= }\
    + \gO\left({(\Delta\vtheta)}^3\right)
\end{align*}
The Hessian in the second term is expressed as
(\Cref{note:background::KLTaylorHessianDecomposition})
%
\marginnote{
  \begin{remark}[\textbf{Decomposition of the log-probability's
      $\vtheta$-Hessian}]\label{note:background::KLTaylorHessianDecomposition}
    Consider element $(i,j)$ of the Hessian,
    \begin{align*}
      \frac{\partial^{2}\log p_{\vtheta}(\vz)}{\partial \evtheta_i \partial \evtheta_j}
      \hspace{-9ex}
      &
      \\
      &=
        \frac{\partial}{\partial \evtheta_i}
        \left(
        \frac{1}{p_{\vtheta}(\vz)}
        \frac{\partial p_{\vtheta}(\vz)}{\partial \evtheta_j}
        \right)
      \\
      &=
        - \frac{1}{p_{\vtheta}(\vz)^2}
        \frac{\partial p_{\vtheta}(\vz)}{\partial \evtheta_j}
        \frac{\partial p_{\vtheta}(\vz)}{\partial \evtheta_i}
      \\
      &\phantom{= }
        +
        \frac{1}{p_{\vtheta}(\vz)}
        \frac{\partial^2 p_{\vtheta}(\vz)}{\partial \evtheta_i \partial \evtheta_j}
      \\
      &=
        - \frac{\partial \log p_{\vtheta}(\vz)}{\partial \evtheta_j}
        \frac{\partial \log p_{\vtheta}(\vz)}{\partial \evtheta_i}
      \\
      &\phantom{= }
        +
        \frac{1}{p_{\vtheta}(\vz)}
        \frac{\partial^2 p_{\vtheta}(\vz)}{\partial \evtheta_i \partial \evtheta_j}
    \end{align*}
    which, in vector notation, translates into
    \Cref{eq:background::LogProbabilityThetaHessian}.
  \end{remark}
}
%
\begin{align}\label{eq:background::LogProbabilityThetaHessian}
  \gradsquared{\vtheta} \log p_{\vtheta}(\vz)
  =
  -
  \grad{\vtheta} \log p_{\vtheta}(\vz)
  \left(
  \grad{\vtheta} \log p_{\vtheta}(\vz)
  \right)^{\top}
  +
  \frac{1}{p_{\vtheta}(\vz)}
  \gradsquared{\vtheta}p_{\vtheta}(\vz)\,,
\end{align}
whose second term again vanishes in expectation (\Cref{note:background::KLTaylorSecondOrderTermVanishes}).
%
\marginnote{
  \begin{remark}[\textbf{Hessian of the model distribution vanishes in expectation}]\label{note:background::KLTaylorSecondOrderTermVanishes}
    \begin{align*}
      \int_{\sOmega}
      p_{\vtheta}(\vz)
      &
        \frac{1}{p_{\vtheta}(\vz)}
        \gradsquared{\vtheta}p_{\vtheta}(\vz)
        \,\diff\vz
      \\
      &=
        \gradsquared{\vtheta}
        \left(
        \int_{\sOmega}
        p_{\vtheta}(\vz)
        \,\diff\vz
        \right)
      \\&= \gradsquared{\vtheta}1 =  0
    \end{align*}
  \end{remark}
}%
Hence,
\begin{subequations}\label{eq:background::FisherOverview}
  \begin{align}
    \label{eq:background::KLDivergenceLocalFisher}
    \KLdiv{p_{\vtheta}(\vz)}{p_{\vtheta + \Delta \vtheta}(\vz)}
    =
    \frac{1}{2}
    {\Delta \vtheta}^{\top}
    \mF(\vtheta)
    \Delta \vtheta
    + \gO\left({(\Delta\vtheta)}^3\right)
  \end{align}
  with the two equivalent forms of the Fisher
  \begin{align}
    \label{eq:background::FisherHessianForm}
    \mF(\vtheta)
    &=
      - \int_{\sOmega} p_{\vtheta}(\vz)
      \gradsquared{\vtheta} \log p_{\vtheta}(\vz)
      \diff\vz
    \\
    \label{eq:background::FisherGradientForm}
    &= \int_{\sOmega} p_{\vtheta}(\vz)
      (\grad{\vtheta} \log p_{\vtheta}(\vz))
      (\grad{\vtheta} \log p_{\vtheta}(\vz))^{\top}
      \diff\vz
  \end{align}
\end{subequations}
\Cref{eq:background::FisherHessianForm} will be helpful to draw connections to
the Hessian, and \Cref{eq:background::FisherGradientForm} provides links to the
\ggn.

Locally, the KL divergence \Cref{eq:background::KLDivergenceLocalFisher} gives
rise to a metric induced by the Fisher-norm $\lVert \cdot \rVert_{\mF(\vtheta)}$,
\begin{align}
  d(p_{\vtheta}, p_{\vtheta + \Delta\vtheta})
  =
  \lVert \Delta\vtheta \rVert_{\mF(\vtheta)}
  :=
  \sqrt{{(\Delta\vtheta)}^{\top} \mF(\vtheta) \Delta\vtheta }
\end{align}
\citet{ollivier2011information} show that the steepest descent for a function $\gL(p_{\vtheta})$ points along the negative natural gradient $-
\naturalgrad{\vtheta}\gL(p_{\vtheta}) := - {\mF(\vtheta)}^{-1}
\grad{\vtheta}\gL(p_{\vtheta})$,
\begin{align}
  \lim_{\epsilon \to 0}
  \frac{1}{\epsilon}
  \argmin_{
  \substack{
  \Delta \vtheta
  \\
  \lVert \Delta \vtheta\rVert_{\mF(\vtheta)} \le \nicefrac{\epsilon^2}{2}}
  }
  \gL(p_{\vtheta + \Delta\vtheta})
  =
  - \frac{
  \naturalgrad{\vtheta}\gL(p_{\vtheta})
  }{
  \left\lVert
  \naturalgrad{\vtheta}\gL(p_{\vtheta})
  \right\rVert_{{\mF(\vtheta)}^{-1}}
  }\,.
\end{align}

\subsubsection{Natural Gradient Descent \& Fisher in Empirical Risk Minimization}

\Cref{sec:background::ProbabilisticInterpretation} presented connections of
empirical risk minimization to learning $\pdata(\vx, \vy)$ via $p_{\vtheta}(\vx,
\vy) = p_{\vtheta}(\giventhat{\vy}{\vx}) p(\vx) = q(\giventhat{\vy}{\vf}) p(\vx)
$ with a negative log-likelihood loss $-\log q(\giventhat{\vy}{\vf}) = \ell(\vf,
\vy)$ and $f_{\vtheta}(\vx):= \vf$. In these cases, the empirical risk
\Cref{eq:background::empiricalRisk} depends on a probability distribution, and
the distribution space's geometry gives rise to NGD with $\vz = (\vx, \vy)$ in
the Fisher \Cref{eq:background::FisherOverview}. After simplifying the
$\vtheta$-derivatives and grouping dependencies of $\vx$ and $\vy$, the Fisher
reads
\begin{align*}
  &\mF(\vtheta)
    \\
  &\quad =
    \int_{\sX} p(\vx)
    \left(
    - \int_{\sY} p_{\vtheta}(\giventhat{\vy}{\vx})
    \gradsquared{\vtheta} \log p_{\vtheta}(\giventhat{\vy}{\vx})
    \,\diff\vy
    \right)
    \diff\vx
  \\
  &\quad = \int_{\sX}
    p(\vx)
    \left(
    \int_{\sY}
    p_{\vtheta}(\giventhat{\vy}{\vx})
    (\grad{\vtheta} \log p_{\vtheta}(\giventhat{\vy}{\vx}))
    (\grad{\vtheta} \log p_{\vtheta}(\giventhat{\vy}{\vx}))^{\top}
    \,\diff\vy
    \right)
    \diff\vx\,,
\end{align*}
or in short form
\begin{align}
  \label{eq:background::FisherExpectationForms}
  \begin{split}
    \mF(\vtheta)
    &= \E_{\vx \sim p(\vx)}
      \E_{\vy \sim p_{\vtheta}(\giventhat{\vy}{\vx})}
      \left[
      - \gradsquared{\vtheta} \log p_{\vtheta}(\giventhat{\vy}{\vx})
      \right]
    \\
    &=
      \E_{\vx \sim p(\vx)}
      \E_{\vy \sim p_{\vtheta}(\giventhat{\vy}{\vx})}
      \left[
      \grad{\vtheta} \log p_{\vtheta}(\giventhat{\vy}{\vx})
      (\grad{\vtheta} \log p_{\vtheta}(\giventhat{\vy}{\vx}))^{\top}
      \right]
  \end{split}
\end{align}
Next, replace $p(\vx) \leftrightarrow p_{\sD}(\vx)$ by empirical approximation
through data,
\begin{align}\label{eq:background::FisherExpectationDifferentPerspectives}
  \begin{split}
    &\mF_{\sD}(\vtheta)
    \\
    &\quad=
      \frac{1}{|\sD|}
      \sum_{(\vx_n, \_) \in \sD}
      \E_{\vy \sim p_{\vtheta}(\giventhat{\vy}{\vx_n})}
      \left[
      - \gradsquared{\vtheta} \log p_{\vtheta}(\giventhat{\vy}{\vx_n})
      \right]
    \\
    &\quad=
      \frac{1}{|\sD|}
      \sum_{(\vx_n, \_) \in \sD}
      \E_{\vy \sim p_{\vtheta}(\giventhat{\vy}{\vx_n})}
      \left[
      \grad{\vtheta} \log p_{\vtheta}(\giventhat{\vy}{\vx_n})
      (\grad{\vtheta} \log p_{\vtheta}(\giventhat{\vy}{\vx_n}))^{\top}
      \right]
  \end{split}
\end{align}
(note that the Fisher is independent of the labels $\{\vy_n\}$!). Using the
relation between log-likelihood and loss function leads to (with $\vf_n :=
f_{\vtheta}(\vx_n)$)
\begin{align*}
  \mF_{\sD}(\vtheta)
  &=
    \frac{1}{|\sD|}
    \sum_{(\vx_n, \_) \in \sD}
    \E_{\vy \sim q(\giventhat{\vy}{\vf_n})}
    \left[
    \gradsquared{\vtheta} \ell(\vf_n, \vy)
    \right]
  \\
  &=
    \frac{1}{|\sD|}
    \sum_{(\vx_n, \_) \in \sD}
    \E_{\vy \sim q(\giventhat{\vy}{\vf_n})}
    \left[
    \grad{\vtheta} \ell(\vf_n, \vy)
    (\grad{\vtheta} \ell(\vf_n, \vy))^{\top}
    \right]\,.
\end{align*}
The second equality views the Fisher as expected gradient outer product. With
the Jacobian chain rule \Cref{hbp::the:chainRuleJacobians} applied to $\ell
\circ f_{\vtheta}$, one obtains
\begin{align}
  \label{eq:background::FisherSamplingPerspective}
  \begin{split}
    &\mF_{\sD}(\vtheta)
      \\
    &\quad=
      \frac{1}{|\sD|}
      \sum_{(\vx_n, \_) \in \sD}
      \E_{\vy \sim q(\giventhat{\vy}{\vf_n})}
      \left[
      {(\jac_{\vtheta}\vf_n)}^{\top}\grad{\vf_n} \ell(\vf_n, \vy)
      ({(\jac_{\vtheta}\vf_n)}^{\top}\grad{\vf_n} \ell(\vf_n, \vy))^{\top}
      \right]
    \\
    &\quad=
      \frac{1}{|\sD|}
      \sum_{(\vx_n, \_) \in \sD}
      {(\jac_{\vtheta}\vf_n)}^{\top}
      \E_{\vy \sim q(\giventhat{\vy}{\vf_n})}
      \left[
      \grad{\vf_n} \ell(\vf_n, \vy)
      (\grad{\vf_n} \ell(\vf_n, \vy))^{\top}
      \right]
      \jac_{\vtheta}\vf_n
  \end{split}
\end{align}
\Cref{eq:background::FisherSamplingPerspective} offers an interesting
perspective to approximate the Fisher via \mc sampling through computing
gradients of the loss on targets drawn from $q$ and will be used later, \eg
\Cref{chap:backpack,chap:vivit,backpack::app:backpack-extensions}.

The first equality in
\Cref{eq:background::FisherExpectationDifferentPerspectives} views the Fisher as
an expected Hessian under the model's likelihood\sidenote{%
  It is important to stress the expectation over the \emph{model's likelihood}
  $\giventhat{\vy}{\vx}$, and \emph{not} over the empirical data. It may be
  tempting to replace the expectation over $\vy$ with the empirical distribution
  $p_{\sD}(\vy)$. This leads to the empirical Fisher
  (\Cref{sec:background::gradientCovariance}), which is often used instead of
  the Fisher, but has limitations for applications like
  optimization~\cite{kunstner2019limitations}.}%
. However, in general it does not coincide with the Hessian, $\mF_{\sD}(\vtheta)
\neq \mH_{\sD}(\vtheta)$, as $ \E_{\vy \sim q(\giventhat{\vy}{\vf_n})}
[\gradsquared{\vtheta} \ell(\vf_n, \vy) ] \neq \gradsquared{\vtheta} \ell(\vf_n,
\vy_n)$. After applying the Hessian chain rule \Cref{hbp::the:chainRuleHessians}
onto $\ell \circ f_{\vtheta}$,
\begin{align}
  \label{eq:background::FisherGGNPerspective}
  \begin{split}
    \mF_{\sD}(\vtheta)
    &=
      \frac{1}{|\sD|}
      \sum_{(\vx_n, \_) \in \sD}
      {(\jac_{\vtheta}\vf_n)}^{\top}
      \E_{\vy \sim q(\giventhat{\vy}{\vf_n})}
      \left[
      \gradsquared{\vf_n}\ell(\vf_n, \vy)
      \right]
      {\jac_{\vtheta}\vf_n}
    \\
    &\phantom{=} \
      +
      \frac{1}{|\sD|}
      \sum_{(\vx_n, \_) \in \sD}
      \sum_{c=1}^{C}
      \gradsquared{\vtheta}{([\vf_n]_{c})}
      \underbrace{
      \E_{\vy \sim q(\giventhat{\vy}{\vf_n})}
      \left[
      \grad{f_{\vtheta}(\vx_{n})} \ell(\vf_n, \vy)
      \right]
      }_{
      = \vzero\text{, same argument as \Cref{note:background::KLTaylorFirstOrderTermVanishes}}
      }
    \\
    &=
      \frac{1}{|\sD|}
      \sum_{(\vx_n, \_) \in \sD}
      {(\jac_{\vtheta}\vf_n)}^{\top}
      \E_{\vy \sim q(\giventhat{\vy}{\vf_n})}
      \left[
      \gradsquared{\vf_n}\ell(\vf_n, \vy)
      \right]
      {\jac_{\vtheta}\vf_n}\,,
  \end{split}
\end{align}
one can see connections to the \ggn
(\Cref{eq:background::generalizedGaussNewton}).

\subsubsection{Connections Between Fisher \& \ggn}

\marginnote{%
  \begin{updaterule}[\textbf{Natural gradient descent
      (simplified}]\label{opt:background:NGD}%
    For empirical risks with loss functions that have a probabilistic
    interpretation, an NGD step is
    \begin{align*}
      \vtheta_{t+1} = \vtheta_t - {\mF_{\sD}(\vtheta_t)}^{-1} \vg_{\sD}(\vtheta_t)
    \end{align*}
    with the gradient from \Cref{def:background::JacobianVectorVector} and the
    Fisher from \Cref{eq:background::FisherSamplingPerspective} or
    \Cref{eq:background::FisherGGNPerspective} (practical implementations vary and
    often introduce additional hyperparameters such as a learning rate, damping
    term, mini-batch size, \etc).
  \end{updaterule}%
}

The Fisher (\Cref{eq:background::FisherGGNPerspective}) and the \ggn
(\Cref{eq:background::generalizedGaussNewton}) are structurally similar. Both
are identical if the expected Hessian of the loss \wrt the model's prediction
under the model is identical to the empirical Hessian,
\begin{align}
  \gradsquared{\vf_n}\ell(\vf_n, \vy_n)
  =
  \E_{\vy \sim q(\giventhat{\vy}{\vf_n})}
  \left[
  \gradsquared{\vf_n}\ell(\vf_n, \vy)
  \right]
  \,\,
  \implies
  \,\,
  \mG_{\sD} = \mF_{\sD}\,.
\end{align}
For both softmax cross-entropy and square loss, this $\vf$-Hessian is
independent of $\vy$, see \Cref{hbp::table:backpropEquations}. Therefore, the
expectation has no effect, and the above equality is satisfied: for least
squares regression (\Cref{ex:background::Regression}) and softmax cross-entropy
classification (\Cref{ex:background::Regression}), the Fisher equals the
\ggn{}\sidenote{%
  There are more scenarios in which Fisher and \ggn coincide, for instance if
  $q(\giventhat{\vy}{\vf})$ is an exponential family distribution with natural
  parameters $\vf$, see~\cite[][Chapter 9]{martens2014new} for a detailed
  presentation.%
}. %
NGD (\Cref{opt:background:NGD}) for those models is thus equivalent to the \ggn
and can be seen as an approximation of Newton's method
(\Cref{opt:background::Newton}) with the \ggn instead of the Hessian.

\subsection{The Gradient Covariance Matrix}\label{sec:background::gradientCovariance}

The Fisher's form in \Cref{eq:background::FisherSamplingPerspective} reminds of
an uncentered second moment of ``would-be'' gradients sampled from the
likelihood implied by the model~\cite{papyan2020traces}. The uncentered gradient
covariance matrix
\begin{align}\label{eq:background::uncenteredGradientCovariance}
  \begin{split}
    \mK_{\sD}(\vtheta)
    &=
      \E_{(\vx, \vy) \sim p_\sD(\vx, \vy)}
      \left[
      \grad{\vtheta} \ell(f_{\vtheta}(\vx), \vy)
      {(\grad{\vtheta} \ell(f_{\vtheta}(\vx), \vy))}^{\top}
      \right]
    \\
    &=
      \frac{1}{|\sD|}
      \sum_{(\vx_n, \vy_n) \in \sD}
      \grad{\vtheta} \ell(\vf_{n}, \vy_n)
      {(\grad{\vtheta} \ell(\vf_{n}, \vy_n))}^{\top}
  \end{split}
\end{align}
if often referred to as empirical Fisher in some of the ML literature, because
it follows from replacing the expectation over the model's distribution
$q(\giventhat{\vy}{\vf_{n}})$ by the expectation over the empirical data
distribution $p_{\sD}(\vy)$ in the Fisher. However, empirical Fisher and Fisher
are, almost always, \emph{not} identical. The importance to distinguish them has
been expressed in works like~\cite{kunstner2019limitations}
and~\cite{thomas2020interplay}, and it is questionable whether the uncentered
gradient covariance \Cref{eq:background::uncenteredGradientCovariance}
approximates curvature.

However, an alternative perspective to use $\mK_{\sD}$ for optimization is to
make stochastic gradient-based optimization aware of the
noise~\cite{balles2022noise} (sketch in
\Cref{fig:background::gradientVariance}). With the centered gradient covariance
on data $\sD$
\begin{align}
  \label{eq:background::centeredGradientCovariance}
  \begin{split}
    \mSigma_{\sD}(\vtheta)
    &= \Var_{(\vx, \vy) \sim p_\sD(\vx, \vy)}
      \left[
      \grad{\vtheta} \ell(f_{\vtheta}(\vx), \vy)
      \right]
    \\
    &=
      \frac{1}{|\sD|}
      \sum_{(\vx_n, \vy_n) \in \sD}
      \left(
      \grad{\vtheta} \ell(\vf_{n}, \vy_n) - \vg_{\sD}(\vtheta)
      % \grad{\vtheta}\gL_{\sD}(\vtheta)
      \right)
      {\left(
      \grad{\vtheta} \ell(\vf_{n}, \vy_n) - \vg_{\sD}(\vtheta)
      % \grad{\vtheta}\gL_{\sD}(\vtheta)
      \right)}^{\top}
    \\
    &=
      \mK_{\sD}(\vtheta)
      -
      \vg_{\sD}(\vtheta)
      {\vg_{\sD}(\vtheta)}^{\top}\,,
      % \grad{\vtheta}\gL_{\sD}(\vtheta)
      % ({\grad{\vtheta}\gL_{\sD}(\vtheta))}^{\top}\,,
  \end{split}
\end{align}
an update step of the form (usually with $\sD = \sB_t$)%
\begin{align*}
  \vtheta_{t+1}
  =
  \vtheta_{t}
  -
  {\mSigma_{\sD}(\vtheta_t)}^{-1}
  \vg_{\sD}(\vtheta_t)
  % \grad{\vtheta}\gL_{\sD}(\vtheta)
\end{align*}
can be regarded as re-scaling the gradient according to fluctuations: directions
that are subject to stronger noise will be shortened more strongly than
directions of small noise.

\marginnote[*-15]{
  \begin{center}
    \pgfkeys{/pgfplots/BackPACKVarianceAdaptationMNIST/.style={
    width = 1.15\linewidth,
    height = 0.9\linewidth,
    every axis plot/.append style={line width = 1.5pt},
    axis line style = black,
    tick pos = left,
    xtick align = inside,
    ytick align = inside,
    xmajorticks = true,
    ymajorticks = true,
    ylabel near ticks,
    xlabel near ticks,
    xtick style = {black},
    xticklabel style = {font = \footnotesize},
    xlabel style = {font = \footnotesize},
    xlabel = epoch,
    ytick style = {black},
    yticklabel style = {font = \footnotesize},
    ylabel style = {font = \footnotesize},
    title style = {font = \footnotesize},
    grid = major,
    grid style = {dashed},
    legend cell align = left,
    legend style = {
      fill opacity = 0,
      text opacity = 0,
      draw opacity = 0,
      font = \footnotesize,
    },
  }
}


%%% Local Variables:
%%% mode: latex
%%% TeX-master: "../../thesis"
%%% End:

    \captionsetup[sub]{labelformat=parens}
    \begingroup
    \captionsetup{type=figure}
    \begin{subfigure}{\linewidth}
      \caption{Train loss}\label{subfig:background::gradientVariance1}
      \pgfkeys{/pgfplots/zmystyle/.style={
          BackPACKVarianceAdaptationMNIST,
          xlabel = \empty,
          xticklabel style = {opacity = 0},
        }
      }
      \tikzexternalenable
      % This file was created by tikzplotlib v0.9.0.
\begin{tikzpicture}

\definecolor{color0}{rgb}{0.196078431372549,0.254901960784314,0.294117647058824}
\definecolor{color1}{rgb}{0.298039215686275,0.447058823529412,0.690196078431373}

\begin{axis}[
axis line style={white!80!black},
legend cell align={left},
legend style={fill opacity=0.8, draw opacity=1, text opacity=1, draw=white!80!black},
tick pos=left,
x grid style={white!80!black},
xmin=-2.45, xmax=51.45,
xtick style={color=gray},
y grid style={white!80!black},
ymin=0.249514521979369, ymax=2.40035014200669,
ytick style={color=gray},
zmystyle
]
\addplot [color0]
table {%
0 2.3025848865509
1 1.20642179739781
2 0.889125862029883
3 0.747893991378637
4 0.667487844289877
5 0.614401750763257
6 0.576804807094427
7 0.548120393126439
8 0.525748381553553
9 0.507354838955097
10 0.492171342556293
11 0.478937127574897
12 0.467922623035235
13 0.458079994030487
14 0.449501695541235
15 0.44190462797116
16 0.434934877126645
17 0.428464715373821
18 0.423088945639439
19 0.417867049651268
20 0.413054488102595
21 0.408689185976982
22 0.404457137867426
23 0.4006716202849
24 0.397119051218033
25 0.393545066928252
26 0.390257908251041
27 0.387462731049611
28 0.384480674603046
29 0.382126632065345
30 0.379549784729114
31 0.377064033081899
32 0.374713193988188
33 0.372608376389895
34 0.370394869454396
35 0.368313059325401
36 0.366417792172004
37 0.364494079007552
38 0.362671494522156
39 0.361339311645581
40 0.359552466831146
41 0.358262416414726
42 0.356722271060333
43 0.35504524490008
44 0.353712660303482
45 0.351949155216033
46 0.350935108501177
47 0.349531286878464
48 0.348238931099574
49 0.347279777435156
};
\addlegendentry{\ Epoch: 49, loss: 0.347}
\addplot [color1, opacity=0.0, mark=*, mark options={solid}, only marks, forget plot]
table {%
0 2.3025848865509
1 1.20642179739781
2 0.889125862029883
3 0.747893991378637
4 0.667487844289877
5 0.614401750763257
6 0.576804807094427
7 0.548120393126439
8 0.525748381553553
9 0.507354838955097
10 0.492171342556293
11 0.478937127574897
12 0.467922623035235
13 0.458079994030487
14 0.449501695541235
15 0.44190462797116
16 0.434934877126645
17 0.428464715373821
18 0.423088945639439
19 0.417867049651268
20 0.413054488102595
21 0.408689185976982
22 0.404457137867426
23 0.4006716202849
24 0.397119051218033
25 0.393545066928252
26 0.390257908251041
27 0.387462731049611
28 0.384480674603046
29 0.382126632065345
30 0.379549784729114
31 0.377064033081899
32 0.374713193988188
33 0.372608376389895
34 0.370394869454396
35 0.368313059325401
36 0.366417792172004
37 0.364494079007552
38 0.362671494522156
39 0.361339311645581
40 0.359552466831146
41 0.358262416414726
42 0.356722271060333
43 0.35504524490008
44 0.353712660303482
45 0.351949155216033
46 0.350935108501177
47 0.349531286878464
48 0.348238931099574
49 0.347279777435156
};
\end{axis}

\end{tikzpicture}

      \tikzexternaldisable
      \vspace{-4ex}
    \end{subfigure}
    \begin{subfigure}{\linewidth}
      \caption{Gradient signal-to-noise
        ratio}\label{subfig:background::gradientVariance2}
      \pgfkeys{/pgfplots/zmystyle/.style={
          BackPACKVarianceAdaptationMNIST,
        }
      }
      \tikzexternalenable
      \hspace{0.7ex}% This file was created by tikzplotlib v0.9.0.
\begin{tikzpicture}

\definecolor{color0}{rgb}{0.941176470588235,0.341176470588235,0.196078431372549}
\definecolor{color1}{rgb}{0.298039215686275,0.447058823529412,0.690196078431373}

\begin{axis}[
axis line style={white!80!black},
legend cell align={left},
legend style={fill opacity=0.8, draw opacity=1, text opacity=1, draw=white!80!black},
tick pos=left,
x grid style={white!80!black},
xmin=-2.45, xmax=51.45,
xtick style={color=gray},
y grid style={white!80!black},
ymin=0.527287566661835, ymax=12.9704472184181,
ytick style={color=gray},
zmystyle
]
\addplot [color0]
table {%
0 12.4048490524292
1 8.24521160125732
2 5.90012836456299
3 4.67441272735596
4 3.94444513320923
5 3.46416854858398
6 3.12312793731689
7 2.8520622253418
8 2.6461398601532
9 2.47222185134888
10 2.3329963684082
11 2.21363735198975
12 2.11418485641479
13 2.02364683151245
14 1.94681537151337
15 1.87832367420197
16 1.81491911411285
17 1.75786185264587
18 1.71885323524475
19 1.66266596317291
20 1.62263524532318
21 1.58327269554138
22 1.55058157444
23 1.51438057422638
24 1.48520469665527
25 1.45760858058929
26 1.43537366390228
27 1.4016969203949
28 1.38217902183533
29 1.36160790920258
30 1.33999931812286
31 1.32310557365417
32 1.30530798435211
33 1.28276085853577
34 1.27259576320648
35 1.25272440910339
36 1.23673510551453
37 1.2228444814682
38 1.20626890659332
39 1.19294035434723
40 1.18065583705902
41 1.16576743125916
42 1.15505576133728
43 1.14469528198242
44 1.13865959644318
45 1.12845051288605
46 1.11092102527618
47 1.10080456733704
48 1.09288573265076
49 1.09951364994049
};
\addlegendentry{\ Epoch 49, SNR: 01.10}
\addplot [color1, opacity=0.0, mark=*, mark options={solid}, only marks, forget plot]
table {%
0 12.4048490524292
1 8.24521160125732
2 5.90012836456299
3 4.67441272735596
4 3.94444513320923
5 3.46416854858398
6 3.12312793731689
7 2.8520622253418
8 2.6461398601532
9 2.47222185134888
10 2.3329963684082
11 2.21363735198975
12 2.11418485641479
13 2.02364683151245
14 1.94681537151337
15 1.87832367420197
16 1.81491911411285
17 1.75786185264587
18 1.71885323524475
19 1.66266596317291
20 1.62263524532318
21 1.58327269554138
22 1.55058157444
23 1.51438057422638
24 1.48520469665527
25 1.45760858058929
26 1.43537366390228
27 1.4016969203949
28 1.38217902183533
29 1.36160790920258
30 1.33999931812286
31 1.32310557365417
32 1.30530798435211
33 1.28276085853577
34 1.27259576320648
35 1.25272440910339
36 1.23673510551453
37 1.2228444814682
38 1.20626890659332
39 1.19294035434723
40 1.18065583705902
41 1.16576743125916
42 1.15505576133728
43 1.14469528198242
44 1.13865959644318
45 1.12845051288605
46 1.11092102527618
47 1.10080456733704
48 1.09288573265076
49 1.09951364994049
};
\end{axis}

\end{tikzpicture}

      \tikzexternaldisable
    \end{subfigure}
    \endgroup
    \captionof{figure}{\textbf{Illustration of gradient noise during training.}
      Evolution of \subfigref{subfig:background::gradientVariance1} training
      loss and \subfigref{subfig:background::gradientVariance2} gradient
      signal-to-noise ratio (computed with \backpack, \Cref{chap:backpack})
      during training (logistic regression on \mnist). As the loss decreases,
      the gradient noise increases.}\label{fig:background::gradientVariance}
  \end{center}
}

The gradient covariance has also been proposed to adapt the batch size during
training~\cite{balles2017coupling,byrd2012sample,bollapragada2017adaptive,bahamou2019dynamic}
and to stop training before overfitting sets in~\cite{mahsereci2017early}.
Similar ideas of variance-adaptation can be found in currently popular deep
learning optimizers such as Adam (\Cref{ex:background::Adam}), that keep an
exponential average of the squared mini-batch gradient. This shows up in the
diagonal of the gradient covariance (second term)
\begin{align}\label{eq:background::GradientCovarianceDiag}
  \diag(\mSigma_{\sD}(\vtheta))
  =
  \frac{1}{|\sD|}
  \left(
  \sum_{(\vx_n, \vy_n) \in \sD}
  {(\grad{\vtheta} \ell(\vf_n, \vy_n))}^{\odot 2}
  \right)
  -
  {\vg_{\sD}(\vtheta)}^{\odot 2}\,.
  % {(\grad{\vtheta}\gL_{\sD}(\vtheta))}^{\odot 2}\,.
\end{align}
See~\cite{balles2018dissecting} for a precise analysis of the connections.

The empirical Fisher is formed by the gradients that make up the mini-batch
average gradient $\grad{\vtheta}\gL(\vtheta)$, and therefore basically free to
compute on top of the gradient. But the required per-sample gradients are
difficult to access in popular ML libraries because they are not agnostic to
this per-sample structure (see
\Cref{sec:background:GradientBackpropagationInMLLibraries}). This complicates
efficient usage of the gradient covariance matrix for applications like
optimization, where performance is key.

%%% Local Variables:
%%% mode: latex
%%% TeX-master: "../thesis"
%%% End:
