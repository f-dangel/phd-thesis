\subsection{Gradient Tests (\robustInlinecode{NormTest},
  \robustInlinecode{InnerTest},
  \robustInlinecode{OrthoTest})}\label{cockpit::app:gradient_tests}

\citet{bollapragada2017adaptive} and \citet{byrd2012sample} propose batch size adaptation
schemes based on the gradient noise. They formulate geometric constraints
between population and mini-batch gradient and accessible approximations that
can be probed to decide whether the mini-batch size should be increased. Because
mini-batches are \iid from $\pdata$, it holds that
\begin{subequations}
  \label{cockpit::eq:iid-sampling-expectation}
  \begin{align}
    \E\left[\vg_{\sB}(\vtheta) \right]
    &=
      \vg_{\pdata}(\vtheta),
    \\
    \E\left[ \vg_{\sB}(\vtheta)^\top \vg_{\pdata}(\vtheta)  \right]
    &=
      \lVert \vg_{\pdata}(\vtheta)  \rVert^2.
  \end{align}
\end{subequations}

The above works propose enforcing other weaker similarity in expectation during
optimization. These geometric constraints reduce to basic vector geometry (see
\Cref{cockpit::subfig:gradient-test-sketch1} for an overview of the relevant
vectors). We recall their formulation here for consistency and derive the
practical versions, which can be computed from training observables and are used
in \cockpit (consult \Cref{cockpit::subfig:gradient-test-sketch2} for the
visualization).

\begin{figure*}[ht]
  \centering
  \begin{subfigure}[t]{0.59\linewidth}
    \centering
    \tikzexternalenable
    \begin{tikzpicture}[rotate=5, >=latex, very thick, xscale = 1., yscale =
      1.25]

      % node style
      \tikzset{my label style/.style={fill=white, fill opacity=0.5, text
          opacity=1, font=\footnotesize}}


      % expected risk gradient
      \draw[->, ultra thick] (0,0) to node [midway, anchor=north west, my label
      style] {$\vg_{\pdata}$} (7,0);

      % mini-batch gradient
      \draw[->] (0,0) to node [midway, anchor=south east, my label style]
      {$\vg_{\sB}$} (6,3);
      \draw[->, >=latex] (0,0) -- (6,3);

      % right angle
      % \draw (6,0) rectangle ++(-0.15,0.15);

      % residual
      \draw[->, sns_blue] (7,0) to node [midway, right, anchor=south west, my label
      style] {$\vg_{\sB} - \vg_{\pdata}$} (6,3);

      % projection
      \draw[->, sns_orange] (0,0) to node [midway, above, anchor=south east, my label
      style] {$\mathrm{proj}_{\vg_{\pdata}}\left(\vg_{\sB}\right)$} (6,0);

      % orthogonal residual
      \draw[->, sns_green] (6,0) to node [midway, left, anchor=north east, my label
      style] {$\vg_{\sB} - \mathrm{proj}_{\vg_{\pdata}}\left(\vg_{\sB}\right)$} (6,3);
    \end{tikzpicture}
    \tikzexternaldisable
    \caption{Relevant vectors}
    \label{cockpit::subfig:gradient-test-sketch1}
  \end{subfigure}
  \begin{subfigure}[t]{0.39\linewidth}
    \centering
    \tikzexternalenable
    \begin{tikzpicture}[>=latex, very thick, xscale = 2, yscale = 2]
      \clip (-1,0) rectangle (1,2);

      % expected risk gradient
      % \draw[->, ultra thick] (0,0) to (0,1);
      % target indicator
      \draw[ultra thick] (0,0.9) to (0,1.1);
      \draw[ultra thick] (-0.1,1) to (0.1,1);

      % norm test
      \pgfmathsetmacro{\normTestRadius}{0.5}
      \filldraw [fill=sns_blue, opacity=0.4] (0,1) circle (\normTestRadius);

      % inner product test
      \pgfmathsetmacro{\innerProductWidth}{0.2}
      \filldraw [fill=sns_orange, opacity=0.4] (-2,1 - \innerProductWidth) rectangle
      (2,1 + \innerProductWidth);

      % orthogonality test
      \pgfmathsetmacro{\orthogonalityTestWidth}{0.3}
      \filldraw [fill=sns_green, opacity=0.4] (-\orthogonalityTestWidth, 0) rectangle
      (\orthogonalityTestWidth, 2);

      % norm test label
      \draw[ultra thick, <->, sns_blue] (0,1) to ++(45:\normTestRadius) node [above, right] {\footnotesize $\theta_{\text{norm}}$};

      % inner product test label
      \draw[ultra thick, <->, sns_orange] (-0.75, 1 - \innerProductWidth) to ++(0, 2 *\innerProductWidth) node [above] {\footnotesize $2 \theta_{\text{inner}}$};

      % orthogonality test label
      \draw[ultra thick, <->, sns_green] (-\orthogonalityTestWidth0, 0.2) to ++(2 *\orthogonalityTestWidth, 0.) node [right] {\footnotesize $2 \nu_{\text{ortho}}$};

      % acute angle test
      % \pgfmathsetmacro{\acuteAngle}{30}
      % \pgfmathsetmacro{\initAngle}{90 - \acuteAngle}
      % \pgfmathsetmacro{\endAngle}{90 + \acuteAngle}
      % \pgfmathsetmacro{\acuteAngleRadius}{3}

      % \filldraw [fill=black, opacity=0.2] (0,0) --
      % (\initAngle:\acuteAngleRadius) arc
      % (\initAngle:\endAngle:\acuteAngleRadius) -- cycle;
    \end{tikzpicture}
    \tikzexternaldisable
    \caption{\cockpit's gradient test visualization.}
    \label{cockpit::subfig:gradient-test-sketch2}
  \end{subfigure}
  \caption{\textbf{Conceptual sketch for gradient tests.}
    \subfigref{cockpit::subfig:gradient-test-sketch1} Relevant vectors to
    formulate the geometric constraints between population and mini-batch
    gradient probed by the gradient tests.
    \subfigref{cockpit::subfig:gradient-test-sketch2} Gradient test
    visualization in \cockpit.}
  \label{cockpit::fig:gradient-tests-sketch}
\end{figure*}

\subsubsection{Usage}

All three gradient tests describe the noise level of the gradients.
\citet{bollapragada2017adaptive} and \citet{byrd2012sample} adapt the batch size
so that the proposed geometric constraints are fulfilled. Practitioners can use
the combined gradient test plot, \ie top center plot in
\Cref{cockpit::fig:showcase}, to monitor gradient noise during training and
adjust hyperparameters such as the batch size.


\subsubsection{Norm Test (\robustInlinecode{NormTest})}\label{cockpit::app:norm-test}
The norm test \citep{byrd2012sample} constrains the residual norm $\lVert
\vg_{\sB}(\vtheta) - \vg_{\pdata}(\vtheta) \rVert$, rescaled by $\lVert
\vg_{\pdata}(\vtheta) \rVert$. This gives rise to a standardized ball of radius
$\theta_{\text{norm}} \in (0, \infty)$ around the population gradient, where the
mini-batch gradient should reside. \citet{byrd2012sample} set
$\theta_{\text{norm}} = 0.9$ in their experiments and increase the batch size if
(in the practical version, see below) the following constraint is not fulfilled
\begin{subequations}
  \begin{align}
    \label{cockpit::eq:norm-test-constraint}
    \E\left[ \frac{ \left\lVert \vg_{\sB}(\vtheta) - \vg_{\pdata}(\vtheta)
    \right\rVert^2 }{\left\lVert \vg_{\pdata}(\vtheta) \right\rVert^2} \right]
    \le
    \theta_{\text{norm}}^2\,.
  \end{align}
  Instead of taking the expectation over mini-batches, \citet{byrd2012sample} note
  that the above will be satisfied if
  \begin{equation}
    \label{cockpit::eq:norm-test-individual}
    \frac{1}{|\sB|} \E\left[ \frac{ \left\lVert \vg_n(\vtheta) - \vg_{\pdata}(\vtheta)
        \right\rVert^2 }{\left\lVert \vg_{\pdata}(\vtheta) \right\rVert^2} \right]
    \le \theta_{\text{norm}}^2\,.
  \end{equation}
\end{subequations}
They propose a practical form of this test,
\begin{subequations}
  \begin{equation}
    \label{cockpit::eq:norm-test-practical-proposed}
    \frac{1}{|\sB| (|\sB| - 1)} \frac{\sum_{n \in \sB} \left\lVert \vg_n(\vtheta) -
        \vg_{\sB}(\vtheta) \right\rVert^2}{\left\lVert
        \vg_{\sB}(\vtheta)\right\rVert^2} \le \theta_{\text{norm}}^2\,,
  \end{equation}
  which can be computed from mini-batch statistics. Rearranging
  \begin{align}
    \sum_{n \in \sB} \left\lVert \vg_n(\vtheta) - \vg_{\sB}(\vtheta) \right\rVert^2
    &= \left( \sum_{n \in \sB} \left\lVert \vg_n(\vtheta) \right\rVert^2 \right) - |\sB|
      \left\lVert \vg_{\sB}(\vtheta) \right\rVert^2\,,
  \end{align}
  we arrive at
  \begin{align}
    \label{cockpit::eq:norm-test-feature-table}
    \frac{1}{|\sB| (|\sB| - 1)} \left[ \frac{ \sum_{n \in \sB} \left\lVert
    \vg_n(\vtheta) \right\rVert^2 }{\left\lVert
    \vg_{\sB}(\vtheta)\right\rVert^2} - |\sB| \right] &\le
                                                        \theta_{\text{norm}}^2
  \end{align}
\end{subequations}
that leverages the norm of both the mini-batch and the individual gradients,
which can be aggregated over parameters during a backward pass. \cockpit's
\inlinecode{NormTest} corresponds to the maximum radius $\theta_{\text{norm}}$
for which the above inequality holds.

\subsubsection{Inner Product Test
  (\robustInlinecode{InnerTest})}\label{cockpit::app:inner-product-test}
The inner product test \citep{bollapragada2017adaptive} constrains the
projection of $\vg_{\sB}(\vtheta)$ onto $\vg_{\pdata}(\vtheta)$ (compare
\Cref{cockpit::subfig:gradient-test-sketch1}),
\begin{align}
  \label{cockpit::eq:inner-product-projection}
  \mathrm{proj}_{\vg_{\pdata}(\vtheta)}\left(\vg_{\sB}(\vtheta)\right)
  :=
  \frac{\vg_{\sB}(\vtheta)^\top \vg_{\pdata}(\vtheta)}{\left\lVert
  \vg_{\pdata}(\vtheta) \right\rVert^2} \vg_{\pdata}(\vtheta)\,,
\end{align}
rescaled by $\lVert \vg_{\pdata}(\vtheta) \rVert$. This restricts the mini-batch
gradient to reside in a standardized band of relative width
$\theta_{\text{inner}}\in (0, \infty)$ around the population risk gradient.
\citet{bollapragada2017adaptive} use $\theta_{\text{inner}} = 0.9$ (in the practical
version, see below) to adapt the batch size if the parallel component's variance
does not satisfy the condition
\begin{subequations}
  \begin{align}
    \label{cockpit::eq:inner-product-test}
    \Var\left( \frac{\vg_{\sB}(\vtheta)^\top \vg_{\pdata}(\vtheta)}{\left\lVert
    \vg_{\pdata}(\vtheta) \right\rVert^2} \right)
    &= \E\left[ \left( \frac{\vg_{\sB}(\vtheta)^\top \vg_{\pdata}(\vtheta)}{\left\lVert
      \vg_{\pdata}(\vtheta) \right\rVert^2}  -1 \right)^2 \right]
      \le \theta_{\text{inner}}^2
  \end{align}
  (note that by \Cref{cockpit::eq:iid-sampling-expectation} we have $\E[
    \nicefrac{\vg_{\sB}(\vtheta)^\top \vg_{\pdata}(\vtheta)}{\left\lVert \vg_{\pdata}(\vtheta)
      \right\rVert^2} ] = 1 $). \citet{bollapragada2017adaptive} bound
  \Cref{cockpit::eq:inner-product-test} by the individual gradient variance,
  \begin{align}
    \label{cockpit::eq:inner-product-test-individual}
    \begin{split}
      &\frac{1}{|\sB|}\Var\left( \frac{\vg_n(\vtheta)^\top \vg_{\pdata}(\vtheta)}{\left\lVert
      \vg_{\pdata}(\vtheta) \right\rVert^2}\right)
      \\
      &\qquad =
      \frac{1}{|\sB|} \E \left[ \left( \frac{\vg_n(\vtheta)^\top \vg_{\pdata}(\vtheta) }{\left\lVert
      \vg_{\pdata}(\vtheta) \right\rVert^2} - 1    \right)^2  \right] \le \theta_{\text{inner}}^2\,.
    \end{split}
  \end{align}
\end{subequations}
They then propose a practical form of \Cref{cockpit::eq:inner-product-test-individual},
which uses the mini-batch sample variance,
\begin{subequations}
  \begin{align}
    \label{cockpit::eq:inner-product-test-practical}
    \begin{split}
    &\frac{1}{|\sB|} \Var\left( \frac{\vg_n(\vtheta)^\top \vg_{\sB}(\vtheta)}{\left\lVert
    \vg_{\sB}(\vtheta)\right\rVert^2}\right)
      \\
      &\qquad
    = \frac{1}{|\sB| (|\sB| - 1)}\left[  \sum_{n\in \sB}  \left( \frac{\vg_n(\vtheta)^\top \vg_{\sB}(\vtheta)}{\left\lVert
    \vg_{\sB}(\vtheta)\right\rVert^2} - 1    \right)^2  \right]
    \le \theta_{\text{inner}}^2\,.
    \end{split}
  \end{align}
  Expanding
  \begin{align}
    \label{cockpit::eq:inner-product-test-practical-rewrite}
    \sum_{n\in \sB}  \left( \frac{\vg_n(\vtheta)^\top \vg_{\sB}(\vtheta)}{\left\lVert
    \vg_{\sB}(\vtheta)\right\rVert^2} - 1    \right)^2
    &=
      \frac{\sum_{n\in \sB}  \left( \vg_n(\vtheta)^\top \vg_{\sB}(\vtheta)\right)^2}{\left\lVert
      \vg_{\sB}(\vtheta)\right\rVert^4} - |\sB|
  \end{align}
  and inserting \Cref{cockpit::eq:inner-product-test-practical-rewrite} into
  \Cref{cockpit::eq:inner-product-test-practical} yields
  \begin{align}
    \label{cockpit::eq:inner-product-test-feature-table}
    \frac{1}{|\sB| (|\sB| - 1)}
    \left[   \frac{\sum_{n\in \sB}  \left( \vg_n(\vtheta)^\top \vg_{\sB}(\vtheta)\right)^2}{\left\lVert
    \vg_{\sB}(\vtheta)\right\rVert^4} - |\sB|\right]
    &\le \theta_{\text{inner}}^2\,.
  \end{align}
\end{subequations}
It relies on pairwise scalar products between individual gradients, which can be
aggregated over layers during backpropagation. \cockpit's \inlinecode{InnerTest}
quantity computes the maximum band width $\theta_{\text{inner}}$ that satisfies
\Cref{cockpit::eq:inner-product-test-feature-table}.

\subsubsection{Orthogonality Test (\robustInlinecode{OrthoTest})}\label{cockpit::app:orthogonality-test}
In contrast to the inner product test (\Cref{cockpit::app:inner-product-test})
which constrains the projection (\Cref{cockpit::eq:inner-product-projection}),
the orthogonality test \citep{bollapragada2017adaptive} constrains the
orthogonal part (see \Cref{cockpit::fig:gradient-tests-sketch} (a))
\begin{align}
  \label{cockpit::eq:orthogonality-projection}
  \vg_{\sB}(\vtheta)
  -
  \mathrm{proj}_{\vg_{\pdata}(\vtheta)}\left(\vg_{\sB}(\vtheta)\right)\,,
\end{align}
rescaled by $\lVert \vg_{\pdata}(\vtheta) \rVert$. This restricts the mini-batch
gradient to a standardized band of relative width $\nu_{\text{ortho}} \in (0,
\infty)$ parallel to the population gradient. \citet{bollapragada2017adaptive} use $\nu
= \tan(80^{\circ}) \approx 5.84$ (in the practical version, see below) to adapt
the batch size if the following condition is violated,
\begin{subequations}
  \begin{align}
    \label{cockpit::eq:orthogonality-test-constraint}
    \E\left[
    \left\lVert
    \frac{
    \vg_{\sB}(\vtheta)
    -
    \mathrm{proj}_{\vg_{\pdata}(\vtheta)}\left(\vg_{\sB}(\vtheta)\right)
    }{
    \lVert \vg_{\pdata}(\vtheta) \rVert
    }
    \right\rVert^2
    \right]
    \le \nu^2_{\text{ortho}}\,.
  \end{align}
  Expanding the norm, and inserting \Cref{cockpit::eq:inner-product-projection}, this
  simplifies to
  \begin{align}
    \begin{split}
      \E \left[ \left\lVert \frac{ \vg_{\sB}(\vtheta) }{ \lVert \vg_{\pdata}(\vtheta)
      \rVert } - \frac{ \vg_{\sB}(\vtheta)^\top \vg_{\pdata}(\vtheta) }{ \lVert
      \vg_{\pdata}(\vtheta) \rVert^2 } \frac{ \vg_{\pdata}(\vtheta) }{ \lVert
      \vg_{\pdata}(\vtheta) \rVert} \right\rVert^2 \right] &\le
                                                        \nu^2_{\text{ortho}}\,,
      \\
      \E\left[ \frac{ \lVert \vg_{\sB}(\vtheta) \rVert^2 }{ \lVert
      \vg_{\pdata}(\vtheta) \rVert^2 } - \frac{ \left( \vg_{\sB}(\vtheta)^\top
      \vg_{\pdata}(\vtheta) \right)^2 }{ \lVert \vg_{\pdata}(\vtheta) \rVert^4 }
      \right] &\le \nu^2_{\text{ortho}}\,.
    \end{split}
  \end{align}
  \citet{bollapragada2017adaptive} bound this inequality with individual gradients,
  \begin{align}
    \frac{1}{|\sB|}
    \E \left[
    \left\lVert
    \frac{
    \vg_n(\vtheta)
    }{
    \lVert
    \vg_{\pdata}(\vtheta)
    \rVert^2
    }
    -
    \frac{
    \vg_n(\vtheta)^\top
    \vg_{\pdata}(\vtheta)
    }{
    \lVert
    \vg_{\pdata}(\vtheta)
    \rVert^2
    }
    \frac{
    \vg_{\pdata}(\vtheta)
    }{
    \left\lVert
    \vg_{\pdata}(\vtheta)
    \right\rVert}
    \right\rVert^2
    \right]
    &\le \nu^2_{\text{ortho}}\,.
  \end{align}
\end{subequations}
They propose the practical form
\begin{subequations}
  \begin{align}
    \frac{1}{|\sB|(|\sB|-1)}
    \E\left[
    \left\lVert
    \frac{
    \vg_n(\vtheta)
    }{
    \lVert
    \vg_{\sB}(\vtheta)
    \rVert
    }
    -
    \frac{
    \vg_n(\vtheta)^\top
    \vg_{\sB}(\vtheta)
    }{
    \lVert
    \vg_{\sB}(\vtheta)
    \rVert^2
    }
    \frac{
    \vg_{\sB}(\vtheta)
    }{
    \left\lVert
    \vg_{\sB}(\vtheta)
    \right\rVert}
    \right\rVert^2
    \right]
    &\le \nu^2_{\text{ortho}}\,,
  \end{align}
  which simplifies to
  \begin{align}
    \label{cockpit::eq:orthogonality-test-feature-table}
    \frac{1}{|\sB| (|\sB| - 1)}
    \sum_{n \in \sB }
    \left(
    \frac{
    \lVert
    \vg_n(\vtheta)
    \rVert^2
    }{
    \lVert
    \vg_{\sB}(\vtheta)
    \rVert^2
    }
    -
    \frac{
    \left(
    \vg_n(\vtheta)^\top
    \vg_{\sB}(\vtheta)
    \right)^2
    }{
    \lVert
    \vg_{\sB}(\vtheta)
    \rVert^4
    }
    \right)
    &\le \nu^2_{\text{ortho}}\,.
  \end{align}
\end{subequations}
It relies on pairwise scalar products between individual gradients which can be
aggregated over layers during a backward pass. \cockpit's \inlinecode{OrthoTest}
quantity computes the maximum band width $\nu_{\text{ortho}}$ which satisfies
\Cref{cockpit::eq:orthogonality-test-feature-table}.

\subsubsection{Relation to Acute Angle Test}

Recently, a novel ``acute angle test'' was proposed by
\citet{bahamou2019dynamic}. While the theoretical constraint between
$\vg_{\sB}(\vtheta)$ and $\vg_{\pdata}(\vtheta)$ differs from the orthogonality
test, the practical versions coincide. Hence, we do not incorporate the acute
angle here.

%%% Local Variables:
%%% mode: latex
%%% TeX-master: "../thesis"
%%% End:
