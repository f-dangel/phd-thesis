%========== mb and further approximations versus full-batch GGN
\subsubsection{Procedure (1)}

We use the checkpoints and the definition of overlaps between eigenspaces from
\Cref{vivit::sec:ggn_vs_hessian}. For the approximation of the \ggn{}, we consider the
cases listed in \Cref{vivit::tab:cases_full_batch}.

\begin{table*}[ht]
  \centering
  \caption{ \textbf{Considered cases for approximation of the eigenspace:} We
    use a different set of cases for the approximation of the \ggn{}'s
    full-batch eigenspace depending on the test problem. For the test problems
    with $C=10$, we use $M=1$ \mc-sample, for the \cifarhun \allcnnc test
    problem ($C=100$), we use $M=10$ \mc-samples in order to reduce the
    computational costs by the same factor. }
  \label{vivit::tab:cases_full_batch}
  \vspace{1ex}
  \begin{footnotesize}
    \begin{tabular}{ll}
      \toprule
      \textbf{Problem}
      & \textbf{Cases} \\
      \midrule
      \makecell[tl]{
      \fmnist \twoctwod \\
      \cifarten \threecthreed and \\
      \cifarten \resnetthirtytwo}
      & \makecell[tl]{
        \textbf{mb, exact} with mini-batch sizes $N \in \{2, 8, 32, 128\}$\\
      \textbf{mb, mc} with $N=128$ and $M=1$ \mc{}-sample\\
      \textbf{sub, exact} using $16$ samples from the mini-batch\\
      \textbf{sub, mc} using $16$ samples from the mini-batch and $M=1$ \mc{}-sample
      }
      \\
      \midrule
      \cifarhun \allcnnc
      & \makecell[tl]{
        \textbf{mb, exact} with mini-batch sizes $N \in \{2, 8, 32, 128\}$\\
      \textbf{mb, mc} with $N=128$ and $M=10$ \mc{}-samples\\
      \textbf{sub, exact} using $16$ samples from the mini-batch\\
      \textbf{sub, mc} using $16$ samples from the mini-batch and $M=10$ \mc{}-samples
      }
      \\
      \bottomrule
    \end{tabular}
  \end{footnotesize}
\end{table*}

For every checkpoint and case, we compute the top-$C$ eigenvectors of the
respective approximation to the \ggn{}. The eigenvectors are either computed
directly using \vivit (by transforming eigenvectors of the Gram matrix into
parameter space, see \Cref{vivit::sec:computing-full-ggn-eigenspectrum}) or, if not
applicable (because memory requirements exceed $N_\text{crit}$, see
\Cref{vivit::subsec:scalability}), using an iterative matrix-free approach. The overlap
is computed in reference to the \ggn{}'s full-batch top-$C$ eigenspace (see
\Cref{vivit::sec:ggn_vs_hessian}). We extract $5$ mini-batches from the training data
and repeat the above procedure for each mini-batch (\ie we obtain $5$ overlap
measurements for every checkpoint and case). The same $5$ mini-batches are used
over all checkpoints and cases.

\subsubsection{Results (1)}

The results can be found in
\Cref{vivit::fig:vivit_vs_full_batch_ggn_1} and \ref{vivit::fig:vivit_vs_full_batch_ggn_2}.
All test problems show the same characteristics: with decreasing computational
effort, the approximation carries less and less structure of its full-batch
counterpart, as indicated by dropping overlaps. In addition, for a fixed
approximation method, a decrease in approximation quality can be observed over
the course of training.


\newcommand{\plotEigspaceVivitvsFB}[4]{
  % defines the pgfplots style "eigspacedefault"
\pgfkeys{/pgfplots/eigspacedefault/.style={
    width=1.03\linewidth,
    height=\goldenRatioInv*1.03*\linewidth,
    every axis plot/.append style={line width = 1pt},
    tick pos = left,
    ylabel near ticks,
    xlabel near ticks,
    xtick align = inside,
    ytick align = inside,
    legend cell align = left,
    legend columns = 1,
    legend pos = north east,
    legend style = {
      fill opacity = 0.9,
      text opacity = 1,
      font = \tiny,
      % column sep=0.1cm,
    },
    legend image post style={scale=2},
    xticklabel style = {font = \small},
    xlabel style = {font = \small},
    axis line style = {black},
    yticklabel style = {font = \small},
    ylabel style = {font = \small},
    title style = {font = \small},
    grid = major,
    grid style = {dashed}
  }
}

\pgfkeys{/pgfplots/eigspacedefaultapp/.style={
    eigspacedefault,
    height=0.6\linewidth,
    legend columns = 2,
  }
}

\pgfkeys{/pgfplots/eigspacenolegend/.style={
    legend image post style = {scale=0},
    legend style = {
      fill opacity = 0,
      draw opacity = 0,
      text opacity = 0,
      font = \small,
      at={(1, 1.025)},
      anchor=south east,
      column sep=0.25cm,
    },
  }
}
%%% Local Variables:
%%% mode: latex
%%% TeX-master: "../main"
%%% End:

  \pgfkeys{/pgfplots/zmystyle/.style={
      eigspacedefault
    }}
  \input{../../fig/exp13_full_batch_monitoring/results/plots/eigspace_vivit_vs_fb/#1_#2_#3_plot_#4}
}

\begin{figure}[p]
\centering
\textbf{\fmnist \twoctwod \sgd}\\[1mm]
\begin{minipage}{0.50\textwidth}
\centering
\plotEigspaceVivitvsFB{fmnist}{2c2d}{sgd}{bs}
% \includegraphics[scale=1.0]{fig/exp13_plots/eigspace_vivit_vs_fb/fmnist_2c2d_sgd_plot_bs.pdf}
\end{minipage}\hfill
\begin{minipage}{0.50\textwidth}
\centering
\plotEigspaceVivitvsFB{fmnist}{2c2d}{sgd}{mc_sub}
% \includegraphics[scale=1.0]{fig/exp13_plots/eigspace_vivit_vs_fb/fmnist_2c2d_sgd_plot_mc_sub.pdf}
\end{minipage}

\textbf{\fmnist \twoctwod \adam}\\[1mm]
\begin{minipage}{0.50\textwidth}
\centering
\plotEigspaceVivitvsFB{fmnist}{2c2d}{adam}{bs}
% \includegraphics[scale=1.0]{fig/exp13_plots/eigspace_vivit_vs_fb/fmnist_2c2d_adam_plot_bs.pdf}
\end{minipage}\hfill
\begin{minipage}{0.50\textwidth}
\centering
\plotEigspaceVivitvsFB{fmnist}{2c2d}{adam}{mc_sub}
% \includegraphics[scale=1.0]{fig/exp13_plots/eigspace_vivit_vs_fb/fmnist_2c2d_adam_plot_mc_sub.pdf}
\end{minipage}

\textbf{\cifarten \threecthreed \sgd}\\[1mm]
\begin{minipage}{0.50\textwidth}
\centering
\plotEigspaceVivitvsFB{cifar10}{3c3d}{sgd}{bs}
% \includegraphics[scale=1.0]{fig/exp13_plots/eigspace_vivit_vs_fb/cifar10_3c3d_sgd_plot_bs.pdf}
\end{minipage}\hfill
\begin{minipage}{0.50\textwidth}
\centering
\plotEigspaceVivitvsFB{cifar10}{3c3d}{sgd}{mc_sub}
% \includegraphics[scale=1.0]{fig/exp13_plots/eigspace_vivit_vs_fb/cifar10_3c3d_sgd_plot_mc_sub.pdf}
\end{minipage}

\textbf{\cifarten \threecthreed \adam}\\[1mm]
\begin{minipage}{0.50\textwidth}
\centering
\plotEigspaceVivitvsFB{cifar10}{3c3d}{adam}{bs}
% \includegraphics[scale=1.0]{fig/exp13_plots/eigspace_vivit_vs_fb/cifar10_3c3d_adam_plot_bs.pdf}
\end{minipage}\hfill
\begin{minipage}{0.50\textwidth}
\centering
\plotEigspaceVivitvsFB{cifar10}{3c3d}{adam}{mc_sub}
% \includegraphics[scale=1.0]{fig/exp13_plots/eigspace_vivit_vs_fb/cifar10_3c3d_adam_plot_mc_sub.pdf}
\end{minipage}

\vspace{-2mm}

\caption{\textbf{\bfvivit{} vs. full-batch \ggn (1):} Overlap between the top-$C$ eigenspaces of different \ggn approximations and the full-batch \ggn during training for all test problems.
Each approximation is evaluated on $5$ different mini-batches.}
\label{fig:vivit_vs_full_batch_ggn_1}
\end{figure}




%=================================================================



\begin{figure}[p]
\centering
\textbf{\cifarten \resnetthirtytwo \sgd}\\[1mm]
\begin{minipage}{0.50\textwidth}
\centering
\plotEigspaceVivitvsFB{cifar10}{resnet32}{sgd}{bs}
% \includegraphics[scale=1.0]{fig/exp13_plots/eigspace_vivit_vs_fb/cifar10_resnet32_sgd_plot_bs.pdf}
\end{minipage}\hfill
\begin{minipage}{0.50\textwidth}
\centering
\plotEigspaceVivitvsFB{cifar10}{resnet32}{sgd}{mc_sub}
% \includegraphics[scale=1.0]{fig/exp13_plots/eigspace_vivit_vs_fb/cifar10_resnet32_sgd_plot_mc_sub.pdf}
\end{minipage}

\vspace{3mm}

\textbf{\cifarten \resnetthirtytwo \adam}\\[1mm]
\begin{minipage}{0.50\textwidth}
\centering
\plotEigspaceVivitvsFB{cifar10}{resnet32}{adam}{bs}
% \includegraphics[scale=1.0]{fig/exp13_plots/eigspace_vivit_vs_fb/cifar10_resnet32_adam_plot_bs.pdf}
\end{minipage}\hfill
\begin{minipage}{0.50\textwidth}
\centering
\plotEigspaceVivitvsFB{cifar10}{resnet32}{adam}{mc_sub}
% \includegraphics[scale=1.0]{fig/exp13_plots/eigspace_vivit_vs_fb/cifar10_resnet32_adam_plot_mc_sub.pdf}
\end{minipage}

\vspace{3mm}

\textbf{\cifarhun \allcnnc \sgd}\\[1mm]
\begin{minipage}{0.50\textwidth}
\centering
\plotEigspaceVivitvsFB{cifar100}{allcnnc}{sgd}{bs}
% \includegraphics[scale=1.0]{fig/exp13_plots/eigspace_vivit_vs_fb/cifar100_allcnnc_sgd_plot_bs.pdf}
\end{minipage}\hfill
\begin{minipage}{0.50\textwidth}
\centering
\plotEigspaceVivitvsFB{cifar100}{allcnnc}{sgd}{mc_sub}
% \includegraphics[scale=1.0]{fig/exp13_plots/eigspace_vivit_vs_fb/cifar100_allcnnc_sgd_plot_mc_sub.pdf}
\end{minipage}

\vspace{3mm}

\textbf{\cifarhun \allcnnc \adam}\\[1mm]
\begin{minipage}{0.50\textwidth}
\centering
\plotEigspaceVivitvsFB{cifar100}{allcnnc}{adam}{bs}
% \includegraphics[scale=1.0]{fig/exp13_plots/eigspace_vivit_vs_fb/cifar100_allcnnc_adam_plot_bs.pdf}
\end{minipage}\hfill
\begin{minipage}{0.50\textwidth}
\centering
\plotEigspaceVivitvsFB{cifar100}{allcnnc}{adam}{mc_sub}
% \includegraphics[scale=1.0]{fig/exp13_plots/eigspace_vivit_vs_fb/cifar100_allcnnc_adam_plot_mc_sub.pdf}
\end{minipage}

\caption{\textbf{\bfvivit{} vs. full-batch \ggn (2):} Overlap between the top-$C$ eigenspaces of different \ggn approximations and the full-batch \ggn during training for all test problems.
Each approximation is evaluated on $5$ different mini-batches.}
\label{fig:vivit_vs_full_batch_ggn_2}
\end{figure}

%%% Local Variables:
%%% mode: latex
%%% TeX-master: "../main"
%%% End:


% ViViT versus mini-batch GGN
%
%========== approximations versus mini-batch GGN
\subsubsection{Procedure (2)}
Since \vivit{}'s \ggn approximations using curvature sub-sampling and/or the MC
approximation (the cases \textbf{mb, mc} as well as \textbf{sub, exact} and
\textbf{sub, mc} in \Cref{vivit::tab:cases_full_batch}) are based on the
\textit{mini}-batch \ggn{}, we cannot expect them to perform better than this
baseline. We thus repeat the analysis from above but use the mini-batch \ggn
with batch-size $N=128$ as ground truth instead of the full-batch \ggn. Of
course, the mini-batch reference top-$C$ eigenspace is always evaluated on the
same mini-batch as the approximation.

\subsubsection{Results (2)}

\Cref{vivit::fig:vivit_vs_mini_batch_ggn} shows the results. Over large parts of
the optimization (note the log scale for the epoch-axis), the \mc approximation
seems to be better suited than curvature sub-sampling (which has comparable
computational cost). For the \cifarhun \allcnnc problem, the \mc approximation
stands out particularly early from the other approximations and consistently
yields higher overlaps with the mini-batch \ggn.

\begin{figure*}[p]
\centering
\begin{minipage}[t]{0.495\linewidth}
  \centering
  {\footnotesize \sgd}
\end{minipage}\hfill
\begin{minipage}[t]{0.495\linewidth}
  \centering
  {\footnotesize \adam}
\end{minipage}

\begin{subfigure}[t]{\linewidth}
  \centering
  \caption{\fmnist \twoctwod}
  \begin{minipage}{0.50\linewidth}
    \centering
    % defines the pgfplots style "eigspacedefault"
\pgfkeys{/pgfplots/eigspacedefault/.style={
    width=1.03\linewidth,
    height=\goldenRatioInv*1.03*\linewidth,
    every axis plot/.append style={line width = 1pt},
    tick pos = left,
    ylabel near ticks,
    xlabel near ticks,
    xtick align = inside,
    ytick align = inside,
    legend cell align = left,
    legend columns = 1,
    legend pos = north east,
    legend style = {
      fill opacity = 0.9,
      text opacity = 1,
      font = \tiny,
      % column sep=0.1cm,
    },
    legend image post style={scale=2},
    xticklabel style = {font = \small},
    xlabel style = {font = \small},
    axis line style = {black},
    yticklabel style = {font = \small},
    ylabel style = {font = \small},
    title style = {font = \small},
    grid = major,
    grid style = {dashed}
  }
}

\pgfkeys{/pgfplots/eigspacedefaultapp/.style={
    eigspacedefault,
    height=0.6\linewidth,
    legend columns = 2,
  }
}

\pgfkeys{/pgfplots/eigspacenolegend/.style={
    legend image post style = {scale=0},
    legend style = {
      fill opacity = 0,
      draw opacity = 0,
      text opacity = 0,
      font = \small,
      at={(1, 1.025)},
      anchor=south east,
      column sep=0.25cm,
    },
  }
}
%%% Local Variables:
%%% mode: latex
%%% TeX-master: "../main"
%%% End:

    \pgfkeys{/pgfplots/zmystyle/.style={
        eigspacedefaultapp,
        legend columns = 3,
      }}
    \tikzexternalenable
    % This file was created by tikzplotlib v0.9.7.
\begin{tikzpicture}

\definecolor{color0}{rgb}{0.274509803921569,0.6,0.564705882352941}
\definecolor{color1}{rgb}{0.870588235294118,0.623529411764706,0.0862745098039216}
\definecolor{color2}{rgb}{0.501960784313725,0.184313725490196,0.6}

\begin{axis}[
axis line style={white!10!black},
legend style={fill opacity=0.8, draw opacity=1, text opacity=1, at={(0.91,0.5)}, anchor=east, draw=white!80!black},
log basis x={10},
tick pos=left,
xlabel={epoch (log scale)},
xmajorgrids,
xmin=0.794328234724281, xmax=125.892541179417,
xmode=log,
ylabel={overlap},
ymajorgrids,
ymin=-0.05, ymax=1.05,
zmystyle
]
\addplot [, white!10!black, dashed, forget plot]
table {%
0.794328234724281 1
125.892541179417 1
};
\addplot [, white!10!black, dashed, forget plot]
table {%
0.794328234724281 0
125.892541179417 0
};
\addplot [, color0, opacity=0.6, mark=diamond*, mark size=0.5, mark options={solid}, only marks]
table {%
1 0.936004817485809
1.04615384615385 0.868970990180969
1.0974358974359 0.804600715637207
1.14871794871795 0.709396302700043
1.2025641025641 0.6698197722435
1.26153846153846 0.732007443904877
1.32051282051282 0.604784429073334
1.38461538461538 0.692672908306122
1.44871794871795 0.598861515522003
1.51794871794872 0.622860729694366
1.58974358974359 0.575215518474579
1.66666666666667 0.538802087306976
1.74615384615385 0.570880055427551
1.82820512820513 0.538296282291412
1.91538461538462 0.515714943408966
2.00769230769231 0.514336049556732
2.1025641025641 0.484077841043472
2.20512820512821 0.460080832242966
2.30769230769231 0.52014833688736
2.41794871794872 0.461889714002609
2.53333333333333 0.460127264261246
2.65384615384615 0.520245492458344
2.78205128205128 0.475404798984528
2.91282051282051 0.425367444753647
3.05384615384615 0.479890823364258
3.1974358974359 0.444502800703049
3.35128205128205 0.418782144784927
3.51025641025641 0.391462922096252
3.67692307692308 0.390369087457657
3.85128205128205 0.345760554075241
4.03589743589744 0.384414702653885
4.22820512820513 0.309506952762604
4.42820512820513 0.311140060424805
4.64102564102564 0.334086149930954
4.86153846153846 0.353588908910751
5.09230769230769 0.272242039442062
5.33589743589744 0.283742070198059
5.58974358974359 0.283878058195114
5.85641025641026 0.263398021459579
6.13589743589744 0.311459094285965
6.42564102564103 0.365124851465225
6.73333333333333 0.295078933238983
7.05384615384615 0.283853620290756
7.38974358974359 0.282484024763107
7.74102564102564 0.242592558264732
8.11025641025641 0.234644964337349
8.4974358974359 0.229494601488113
8.9 0.275288373231888
9.32564102564103 0.216282367706299
9.76923076923077 0.198605611920357
10.2333333333333 0.233089923858643
10.7205128205128 0.147863700985909
11.2307692307692 0.256242543458939
11.7666666666667 0.307410687208176
12.3282051282051 0.199975058436394
12.9153846153846 0.243313655257225
13.5282051282051 0.2171870470047
14.174358974359 0.177268967032433
14.8487179487179 0.171582102775574
15.5564102564103 0.150610268115997
16.2974358974359 0.0790159627795219
17.0717948717949 0.225891157984734
17.8846153846154 0.0557145848870277
18.7358974358974 0.333670496940613
19.6282051282051 0.33229324221611
20.5641025641026 0.21984301507473
21.5435897435897 0.150105938315392
22.5692307692308 0.148283421993256
23.6435897435897 0.224274069070816
24.7692307692308 0.132933720946312
25.9487179487179 0.226027011871338
27.1846153846154 0.225310474634171
28.4794871794872 0.240366533398628
29.8358974358974 0.141018897294998
31.2564102564103 0.142142608761787
32.7435897435897 0.203517839312553
34.3025641025641 0.181047663092613
35.9358974358974 0.172072932124138
37.648717948718 0.147612363100052
39.4410256410256 0.149318769574165
41.3179487179487 0.147554293274879
43.2871794871795 0.144426614046097
45.3487179487179 0.227442190051079
47.5076923076923 0.144580408930779
49.7692307692308 0.229933723807335
52.1384615384615 0.229476019740105
54.6205128205128 0.229191944003105
57.2230769230769 0.230118349194527
59.9461538461538 0.230636104941368
62.8025641025641 0.230438977479935
65.7923076923077 0.230797871947289
68.925641025641 0.231035932898521
72.2076923076923 0.230983763933182
75.6461538461539 0.231911331415176
79.2461538461538 0.231428578495979
83.0205128205128 0.232441902160645
86.974358974359 0.232638835906982
91.1153846153846 0.232392340898514
95.4538461538462 0.232618913054466
100 0.233565762639046
};
\addlegendentry{sub 16, exact}
\addplot [, color0, opacity=0.6, mark=diamond*, mark size=0.5, mark options={solid}, only marks, forget plot]
table {%
1 0.962017953395844
1.04615384615385 0.861895561218262
1.0974358974359 0.746412694454193
1.14871794871795 0.72264575958252
1.2025641025641 0.615411996841431
1.26153846153846 0.627482831478119
1.32051282051282 0.651982247829437
1.38461538461538 0.599012196063995
1.44871794871795 0.552116811275482
1.51794871794872 0.548406302928925
1.58974358974359 0.527815759181976
1.66666666666667 0.504942893981934
1.74615384615385 0.557456851005554
1.82820512820513 0.543117940425873
1.91538461538462 0.502581894397736
2.00769230769231 0.522199809551239
2.1025641025641 0.527817070484161
2.20512820512821 0.453012853860855
2.30769230769231 0.479049414396286
2.41794871794872 0.518686711788177
2.53333333333333 0.460291296243668
2.65384615384615 0.489824295043945
2.78205128205128 0.450023144483566
2.91282051282051 0.359014958143234
3.05384615384615 0.535807907581329
3.1974358974359 0.389375925064087
3.35128205128205 0.377820461988449
3.51025641025641 0.417566031217575
3.67692307692308 0.361541539430618
3.85128205128205 0.328938543796539
4.03589743589744 0.422144323587418
4.22820512820513 0.272004365921021
4.42820512820513 0.310802191495895
4.64102564102564 0.353472322225571
4.86153846153846 0.346219837665558
5.09230769230769 0.384192913770676
5.33589743589744 0.23760949075222
5.58974358974359 0.330328226089478
5.85641025641026 0.306428492069244
6.13589743589744 0.344117909669876
6.42564102564103 0.293919235467911
6.73333333333333 0.268591463565826
7.05384615384615 0.302452147006989
7.38974358974359 0.175466224551201
7.74102564102564 0.303216308355331
8.11025641025641 0.202379494905472
8.4974358974359 0.292166084051132
8.9 0.36177995800972
9.32564102564103 0.243516951799393
9.76923076923077 0.215846702456474
10.2333333333333 0.292372912168503
10.7205128205128 0.202913090586662
11.2307692307692 0.194614320993423
11.7666666666667 0.283453196287155
12.3282051282051 0.370611757040024
12.9153846153846 0.150556311011314
13.5282051282051 0.104391314089298
14.174358974359 0.215660020709038
14.8487179487179 0.211577519774437
15.5564102564103 0.142870336771011
16.2974358974359 0.130266770720482
17.0717948717949 0.130049899220467
17.8846153846154 0.0459813550114632
18.7358974358974 0.238838538527489
19.6282051282051 0.143758997321129
20.5641025641026 0.138734012842178
21.5435897435897 0.309867382049561
22.5692307692308 0.036349143832922
23.6435897435897 0.119288399815559
24.7692307692308 0.213739305734634
25.9487179487179 0.0430782027542591
27.1846153846154 0.228551626205444
28.4794871794872 0.0580994375050068
29.8358974358974 0.135497972369194
31.2564102564103 0.132669895887375
32.7435897435897 0.0772591754794121
34.3025641025641 0.0667497292160988
35.9358974358974 0.0596593096852303
37.648717948718 0.0595200918614864
39.4410256410256 0.0615243390202522
41.3179487179487 0.0570103488862514
43.2871794871795 0.0650031492114067
45.3487179487179 0.0655249804258347
47.5076923076923 0.0596979446709156
49.7692307692308 0.0587072037160397
52.1384615384615 0.0633323192596436
54.6205128205128 0.0567691922187805
57.2230769230769 0.0573495030403137
59.9461538461538 0.0603199191391468
62.8025641025641 0.0568912141025066
65.7923076923077 0.0578992180526257
68.925641025641 0.0556127727031708
72.2076923076923 0.0575870536267757
75.6461538461539 0.0582781806588173
79.2461538461538 0.0597911886870861
83.0205128205128 0.055868711322546
86.974358974359 0.0569387078285217
91.1153846153846 0.0558736734092236
95.4538461538462 0.0593321099877357
100 0.0587193034589291
};
\addplot [, color0, opacity=0.6, mark=diamond*, mark size=0.5, mark options={solid}, only marks, forget plot]
table {%
1 0.892605245113373
1.04615384615385 0.725819885730743
1.0974358974359 0.575800001621246
1.14871794871795 0.649764120578766
1.2025641025641 0.559903681278229
1.26153846153846 0.482453256845474
1.32051282051282 0.524944007396698
1.38461538461538 0.524809777736664
1.44871794871795 0.467938154935837
1.51794871794872 0.393592923879623
1.58974358974359 0.416958570480347
1.66666666666667 0.434503942728043
1.74615384615385 0.443664282560349
1.82820512820513 0.410621970891953
1.91538461538462 0.399627059698105
2.00769230769231 0.414451211690903
2.1025641025641 0.402742594480515
2.20512820512821 0.308602422475815
2.30769230769231 0.380892276763916
2.41794871794872 0.398750931024551
2.53333333333333 0.390778392553329
2.65384615384615 0.334907412528992
2.78205128205128 0.389802068471909
2.91282051282051 0.371166616678238
3.05384615384615 0.408107757568359
3.1974358974359 0.370369881391525
3.35128205128205 0.314373165369034
3.51025641025641 0.419851392507553
3.67692307692308 0.299794435501099
3.85128205128205 0.345716059207916
4.03589743589744 0.343336164951324
4.22820512820513 0.274246394634247
4.42820512820513 0.334462016820908
4.64102564102564 0.268615245819092
4.86153846153846 0.302003592252731
5.09230769230769 0.353498041629791
5.33589743589744 0.282950729131699
5.58974358974359 0.291389614343643
5.85641025641026 0.325099468231201
6.13589743589744 0.278197377920151
6.42564102564103 0.2484450340271
6.73333333333333 0.304479449987411
7.05384615384615 0.278808504343033
7.38974358974359 0.249676421284676
7.74102564102564 0.225431278347969
8.11025641025641 0.207061007618904
8.4974358974359 0.232374474406242
8.9 0.250244706869125
9.32564102564103 0.207041695713997
9.76923076923077 0.179723426699638
10.2333333333333 0.125246629118919
10.7205128205128 0.1474978774786
11.2307692307692 0.161285787820816
11.7666666666667 0.25934162735939
12.3282051282051 0.16595247387886
12.9153846153846 0.165838733315468
13.5282051282051 0.145770356059074
14.174358974359 0.2396449893713
14.8487179487179 0.144441574811935
15.5564102564103 0.238449618220329
16.2974358974359 0.150196492671967
17.0717948717949 0.0759950205683708
17.8846153846154 0.137908130884171
18.7358974358974 0.150885730981827
19.6282051282051 0.129608422517776
20.5641025641026 0.223726585507393
21.5435897435897 0.0349251255393028
22.5692307692308 0.124682582914829
23.6435897435897 0.134370893239975
24.7692307692308 0.118422262370586
25.9487179487179 0.127803012728691
27.1846153846154 0.122646749019623
28.4794871794872 0.132970958948135
29.8358974358974 0.121761441230774
31.2564102564103 0.12557515501976
32.7435897435897 0.207542568445206
34.3025641025641 0.199676379561424
35.9358974358974 0.222655221819878
37.648717948718 0.223028570413589
39.4410256410256 0.222633197903633
41.3179487179487 0.222850486636162
43.2871794871795 0.22296242415905
45.3487179487179 0.22319483757019
47.5076923076923 0.223420485854149
49.7692307692308 0.22414219379425
52.1384615384615 0.223647981882095
54.6205128205128 0.224450752139091
57.2230769230769 0.224371820688248
59.9461538461538 0.223313808441162
62.8025641025641 0.224815055727959
65.7923076923077 0.225141033530235
68.925641025641 0.225380763411522
72.2076923076923 0.225516557693481
75.6461538461539 0.225761532783508
79.2461538461538 0.225356966257095
83.0205128205128 0.225607201457024
86.974358974359 0.225896313786507
91.1153846153846 0.225946053862572
95.4538461538462 0.225222751498222
100 0.226427271962166
};
\addplot [, color0, opacity=0.6, mark=diamond*, mark size=0.5, mark options={solid}, only marks, forget plot]
table {%
1 0.967906177043915
1.04615384615385 0.82199639081955
1.0974358974359 0.708317339420319
1.14871794871795 0.636027753353119
1.2025641025641 0.649511635303497
1.26153846153846 0.565900921821594
1.32051282051282 0.489386945962906
1.38461538461538 0.567384779453278
1.44871794871795 0.538549363613129
1.51794871794872 0.442706346511841
1.58974358974359 0.517245590686798
1.66666666666667 0.444920361042023
1.74615384615385 0.448114693164825
1.82820512820513 0.358034163713455
1.91538461538462 0.398070275783539
2.00769230769231 0.467765241861343
2.1025641025641 0.364387512207031
2.20512820512821 0.370374113321304
2.30769230769231 0.432212024927139
2.41794871794872 0.39304780960083
2.53333333333333 0.341598004102707
2.65384615384615 0.43699699640274
2.78205128205128 0.442636549472809
2.91282051282051 0.350589334964752
3.05384615384615 0.340226113796234
3.1974358974359 0.378767251968384
3.35128205128205 0.359765559434891
3.51025641025641 0.302192360162735
3.67692307692308 0.258797973394394
3.85128205128205 0.372339874505997
4.03589743589744 0.41323509812355
4.22820512820513 0.420709222555161
4.42820512820513 0.441124439239502
4.64102564102564 0.433457463979721
4.86153846153846 0.410541623830795
5.09230769230769 0.398242205381393
5.33589743589744 0.309214323759079
5.58974358974359 0.304281145334244
5.85641025641026 0.360666126012802
6.13589743589744 0.33919683098793
6.42564102564103 0.445295423269272
6.73333333333333 0.417826384305954
7.05384615384615 0.327795118093491
7.38974358974359 0.414164841175079
7.74102564102564 0.364914119243622
8.11025641025641 0.265596061944962
8.4974358974359 0.275900900363922
8.9 0.37898388504982
9.32564102564103 0.344468176364899
9.76923076923077 0.272732883691788
10.2333333333333 0.345374405384064
10.7205128205128 0.257235288619995
11.2307692307692 0.250312954187393
11.7666666666667 0.299260437488556
12.3282051282051 0.309075593948364
12.9153846153846 0.231927141547203
13.5282051282051 0.150212422013283
14.174358974359 0.192829832434654
14.8487179487179 0.273397147655487
15.5564102564103 0.312559425830841
16.2974358974359 0.214126944541931
17.0717948717949 0.31502291560173
17.8846153846154 0.127558633685112
18.7358974358974 0.347147524356842
19.6282051282051 0.154177233576775
20.5641025641026 0.151090428233147
21.5435897435897 0.209787771105766
22.5692307692308 0.122091725468636
23.6435897435897 0.129362657666206
24.7692307692308 0.210338160395622
25.9487179487179 0.309892743825912
27.1846153846154 0.225799009203911
28.4794871794872 0.122734241187572
29.8358974358974 0.122139908373356
31.2564102564103 0.111193589866161
32.7435897435897 0.135528609156609
34.3025641025641 0.135739728808403
35.9358974358974 0.133106932044029
37.648717948718 0.132152065634727
39.4410256410256 0.134474724531174
41.3179487179487 0.138256654143333
43.2871794871795 0.135838359594345
45.3487179487179 0.136555060744286
47.5076923076923 0.13682709634304
49.7692307692308 0.136128649115562
52.1384615384615 0.136475011706352
54.6205128205128 0.135009735822678
57.2230769230769 0.13650082051754
59.9461538461538 0.136914655566216
62.8025641025641 0.138357624411583
65.7923076923077 0.137260749936104
68.925641025641 0.138695329427719
72.2076923076923 0.139356315135956
75.6461538461539 0.136977180838585
79.2461538461538 0.140830099582672
83.0205128205128 0.13982842862606
86.974358974359 0.14177118241787
91.1153846153846 0.140810832381248
95.4538461538462 0.141851544380188
100 0.143214598298073
};
\addplot [, color0, opacity=0.6, mark=diamond*, mark size=0.5, mark options={solid}, only marks, forget plot]
table {%
1 0.939375400543213
1.04615384615385 0.738456249237061
1.0974358974359 0.74108362197876
1.14871794871795 0.624462127685547
1.2025641025641 0.643597662448883
1.26153846153846 0.575832307338715
1.32051282051282 0.592270255088806
1.38461538461538 0.517629325389862
1.44871794871795 0.507388770580292
1.51794871794872 0.483558475971222
1.58974358974359 0.561569154262543
1.66666666666667 0.600242018699646
1.74615384615385 0.514516115188599
1.82820512820513 0.602156579494476
1.91538461538462 0.553084433078766
2.00769230769231 0.506339192390442
2.1025641025641 0.476807326078415
2.20512820512821 0.469651073217392
2.30769230769231 0.509989738464355
2.41794871794872 0.518802464008331
2.53333333333333 0.458188623189926
2.65384615384615 0.501897990703583
2.78205128205128 0.451869934797287
2.91282051282051 0.401311784982681
3.05384615384615 0.424437373876572
3.1974358974359 0.397109121084213
3.35128205128205 0.397421330213547
3.51025641025641 0.393283694982529
3.67692307692308 0.400670528411865
3.85128205128205 0.392106860876083
4.03589743589744 0.396413058042526
4.22820512820513 0.327330589294434
4.42820512820513 0.402595847845078
4.64102564102564 0.340076684951782
4.86153846153846 0.353765726089478
5.09230769230769 0.332072556018829
5.33589743589744 0.356610804796219
5.58974358974359 0.331212282180786
5.85641025641026 0.296869993209839
6.13589743589744 0.359526395797729
6.42564102564103 0.227211147546768
6.73333333333333 0.415561497211456
7.05384615384615 0.207393407821655
7.38974358974359 0.262146085500717
7.74102564102564 0.241294607520103
8.11025641025641 0.241339549422264
8.4974358974359 0.250610888004303
8.9 0.193404987454414
9.32564102564103 0.21280899643898
9.76923076923077 0.248014256358147
10.2333333333333 0.303582400083542
10.7205128205128 0.224086090922356
11.2307692307692 0.234642177820206
11.7666666666667 0.181355506181717
12.3282051282051 0.263634622097015
12.9153846153846 0.18697027862072
13.5282051282051 0.185125827789307
14.174358974359 0.232351213693619
14.8487179487179 0.34440690279007
15.5564102564103 0.17741671204567
16.2974358974359 0.138911291956902
17.0717948717949 0.162436425685883
17.8846153846154 0.17393696308136
18.7358974358974 0.224364146590233
19.6282051282051 0.165562853217125
20.5641025641026 0.326208800077438
21.5435897435897 0.235986858606339
22.5692307692308 0.234258994460106
23.6435897435897 0.257224142551422
24.7692307692308 0.100705087184906
25.9487179487179 0.332544773817062
27.1846153846154 0.14900229871273
28.4794871794872 0.233560785651207
29.8358974358974 0.253976583480835
31.2564102564103 0.165092155337334
32.7435897435897 0.321374267339706
34.3025641025641 0.262130558490753
35.9358974358974 0.317630618810654
37.648717948718 0.291932195425034
39.4410256410256 0.244164898991585
41.3179487179487 0.251635879278183
43.2871794871795 0.256999015808105
45.3487179487179 0.244925901293755
47.5076923076923 0.24824070930481
49.7692307692308 0.244257673621178
52.1384615384615 0.247642159461975
54.6205128205128 0.248381048440933
57.2230769230769 0.246532708406448
59.9461538461538 0.246660187840462
62.8025641025641 0.249960020184517
65.7923076923077 0.248122408986092
68.925641025641 0.248224377632141
72.2076923076923 0.245962098240852
75.6461538461539 0.247468337416649
79.2461538461538 0.244350537657738
83.0205128205128 0.246103599667549
86.974358974359 0.247069746255875
91.1153846153846 0.247350722551346
95.4538461538462 0.247433856129646
100 0.245320662856102
};
\addplot [, color1, opacity=0.6, mark=square*, mark size=0.5, mark options={solid}, only marks]
table {%
1 0.957848727703094
1.04615384615385 0.833355903625488
1.0974358974359 0.760295808315277
1.14871794871795 0.730186879634857
1.2025641025641 0.679652988910675
1.26153846153846 0.697504758834839
1.32051282051282 0.621926963329315
1.38461538461538 0.689764678478241
1.44871794871795 0.620876729488373
1.51794871794872 0.6256143450737
1.58974358974359 0.639800369739532
1.66666666666667 0.565501928329468
1.74615384615385 0.702140152454376
1.82820512820513 0.655050337314606
1.91538461538462 0.675862431526184
2.00769230769231 0.554590821266174
2.1025641025641 0.497472614049911
2.20512820512821 0.525117993354797
2.30769230769231 0.576285183429718
2.41794871794872 0.574722588062286
2.53333333333333 0.457760632038116
2.65384615384615 0.465418428182602
2.78205128205128 0.516821801662445
2.91282051282051 0.500231742858887
3.05384615384615 0.449769407510757
3.1974358974359 0.574230194091797
3.35128205128205 0.416923493146896
3.51025641025641 0.457148373126984
3.67692307692308 0.417429834604263
3.85128205128205 0.478844553232193
4.03589743589744 0.473095029592514
4.22820512820513 0.435540109872818
4.42820512820513 0.510932564735413
4.64102564102564 0.47735133767128
4.86153846153846 0.49323359131813
5.09230769230769 0.397619217634201
5.33589743589744 0.587922871112823
5.58974358974359 0.439208328723907
5.85641025641026 0.488596826791763
6.13589743589744 0.495432913303375
6.42564102564103 0.509944558143616
6.73333333333333 0.418376505374908
7.05384615384615 0.418474286794662
7.38974358974359 0.38752469420433
7.74102564102564 0.513098061084747
8.11025641025641 0.421157449483871
8.4974358974359 0.354152023792267
8.9 0.436614960432053
9.32564102564103 0.404541224241257
9.76923076923077 0.427877962589264
10.2333333333333 0.436946481466293
10.7205128205128 0.555807948112488
11.2307692307692 0.456604719161987
11.7666666666667 0.608454883098602
12.3282051282051 0.459780216217041
12.9153846153846 0.441834449768066
13.5282051282051 0.569943726062775
14.174358974359 0.769886314868927
14.8487179487179 0.68795520067215
15.5564102564103 0.832554638385773
16.2974358974359 0.819323539733887
17.0717948717949 0.700359642505646
17.8846153846154 0.983083844184875
18.7358974358974 0.869165062904358
19.6282051282051 0.893020331859589
20.5641025641026 0.951839625835419
21.5435897435897 0.985411584377289
22.5692307692308 0.863348960876465
23.6435897435897 0.797388851642609
24.7692307692308 0.99148017168045
25.9487179487179 0.903759956359863
27.1846153846154 0.899074077606201
28.4794871794872 0.895025432109833
29.8358974358974 0.988793194293976
31.2564102564103 0.898646950721741
32.7435897435897 0.92162150144577
34.3025641025641 0.954361736774445
35.9358974358974 0.950784504413605
37.648717948718 0.896674335002899
39.4410256410256 0.893844127655029
41.3179487179487 0.892224252223969
43.2871794871795 0.895514488220215
45.3487179487179 0.89911937713623
47.5076923076923 0.967299461364746
49.7692307692308 0.900401771068573
52.1384615384615 0.899879097938538
54.6205128205128 0.899955272674561
57.2230769230769 0.899509608745575
59.9461538461538 0.899903118610382
62.8025641025641 0.899751305580139
65.7923076923077 0.899486720561981
68.925641025641 0.899560153484344
72.2076923076923 0.899524807929993
75.6461538461539 0.899471700191498
79.2461538461538 0.899791836738586
83.0205128205128 0.895518720149994
86.974358974359 0.898404598236084
91.1153846153846 0.89948970079422
95.4538461538462 0.899650394916534
100 0.89646589756012
};
\addlegendentry{mb 128, mc 1}
\addplot [, color1, opacity=0.6, mark=square*, mark size=0.5, mark options={solid}, only marks, forget plot]
table {%
1 0.942690074443817
1.04615384615385 0.86713832616806
1.0974358974359 0.858007907867432
1.14871794871795 0.873467922210693
1.2025641025641 0.773766160011292
1.26153846153846 0.737349689006805
1.32051282051282 0.788738369941711
1.38461538461538 0.714215219020844
1.44871794871795 0.736101567745209
1.51794871794872 0.602200925350189
1.58974358974359 0.724372327327728
1.66666666666667 0.585667073726654
1.74615384615385 0.61141300201416
1.82820512820513 0.620705604553223
1.91538461538462 0.597606837749481
2.00769230769231 0.596745669841766
2.1025641025641 0.642804205417633
2.20512820512821 0.673573434352875
2.30769230769231 0.713185966014862
2.41794871794872 0.689747333526611
2.53333333333333 0.624708771705627
2.65384615384615 0.494757652282715
2.78205128205128 0.686464250087738
2.91282051282051 0.533526718616486
3.05384615384615 0.629171192646027
3.1974358974359 0.614723384380341
3.35128205128205 0.60503077507019
3.51025641025641 0.556003212928772
3.67692307692308 0.582121849060059
3.85128205128205 0.662426054477692
4.03589743589744 0.641505837440491
4.22820512820513 0.655398666858673
4.42820512820513 0.628180146217346
4.64102564102564 0.610266983509064
4.86153846153846 0.49938440322876
5.09230769230769 0.608484983444214
5.33589743589744 0.662531852722168
5.58974358974359 0.572003185749054
5.85641025641026 0.599265694618225
6.13589743589744 0.602979958057404
6.42564102564103 0.681568324565887
6.73333333333333 0.515038073062897
7.05384615384615 0.783084571361542
7.38974358974359 0.711827874183655
7.74102564102564 0.675182461738586
8.11025641025641 0.694654405117035
8.4974358974359 0.657098650932312
8.9 0.67378443479538
9.32564102564103 0.724365949630737
9.76923076923077 0.618667602539062
10.2333333333333 0.829421520233154
10.7205128205128 0.70025509595871
11.2307692307692 0.811740875244141
11.7666666666667 0.749396622180939
12.3282051282051 0.733491241931915
12.9153846153846 0.893311321735382
13.5282051282051 0.789274871349335
14.174358974359 0.855167806148529
14.8487179487179 0.930248856544495
15.5564102564103 0.917655766010284
16.2974358974359 0.889698922634125
17.0717948717949 0.791980445384979
17.8846153846154 0.89545738697052
18.7358974358974 0.85824579000473
19.6282051282051 0.893822491168976
20.5641025641026 0.924585163593292
21.5435897435897 0.900547206401825
22.5692307692308 0.799648821353912
23.6435897435897 0.892943680286407
24.7692307692308 0.845711827278137
25.9487179487179 0.895661950111389
27.1846153846154 0.888187050819397
28.4794871794872 0.892907917499542
29.8358974358974 0.909106910228729
31.2564102564103 0.882825195789337
32.7435897435897 0.986594021320343
34.3025641025641 0.955657780170441
35.9358974358974 0.897328197956085
37.648717948718 0.898166120052338
39.4410256410256 0.964179217815399
41.3179487179487 0.989481568336487
43.2871794871795 0.900054931640625
45.3487179487179 0.896478295326233
47.5076923076923 0.897653698921204
49.7692307692308 0.99067348241806
52.1384615384615 0.896791100502014
54.6205128205128 0.993881046772003
57.2230769230769 0.994608879089355
59.9461538461538 0.930653035640717
62.8025641025641 0.992711842060089
65.7923076923077 0.921556890010834
68.925641025641 0.994740903377533
72.2076923076923 0.977632999420166
75.6461538461539 0.978404641151428
79.2461538461538 0.94403612613678
83.0205128205128 0.899148881435394
86.974358974359 0.988734662532806
91.1153846153846 0.899707436561584
95.4538461538462 0.89967155456543
100 0.896391034126282
};
\addplot [, color1, opacity=0.6, mark=square*, mark size=0.5, mark options={solid}, only marks, forget plot]
table {%
1 0.921656429767609
1.04615384615385 0.898259580135345
1.0974358974359 0.818662941455841
1.14871794871795 0.770749866962433
1.2025641025641 0.667155861854553
1.26153846153846 0.674977779388428
1.32051282051282 0.668884754180908
1.38461538461538 0.681874454021454
1.44871794871795 0.751790821552277
1.51794871794872 0.662052094936371
1.58974358974359 0.669375240802765
1.66666666666667 0.708254277706146
1.74615384615385 0.684301853179932
1.82820512820513 0.541012227535248
1.91538461538462 0.61588591337204
2.00769230769231 0.622236967086792
2.1025641025641 0.661319196224213
2.20512820512821 0.513049244880676
2.30769230769231 0.500097513198853
2.41794871794872 0.506514310836792
2.53333333333333 0.477663815021515
2.65384615384615 0.552952110767365
2.78205128205128 0.540445744991302
2.91282051282051 0.633853375911713
3.05384615384615 0.536807179450989
3.1974358974359 0.659977078437805
3.35128205128205 0.615314483642578
3.51025641025641 0.540070652961731
3.67692307692308 0.553362309932709
3.85128205128205 0.617566227912903
4.03589743589744 0.506883203983307
4.22820512820513 0.515620291233063
4.42820512820513 0.45104119181633
4.64102564102564 0.542449593544006
4.86153846153846 0.364789068698883
5.09230769230769 0.455491155385971
5.33589743589744 0.441257953643799
5.58974358974359 0.560394704341888
5.85641025641026 0.578310012817383
6.13589743589744 0.411669075489044
6.42564102564103 0.558330535888672
6.73333333333333 0.510091245174408
7.05384615384615 0.631362020969391
7.38974358974359 0.503063142299652
7.74102564102564 0.593015253543854
8.11025641025641 0.670337319374084
8.4974358974359 0.569488167762756
8.9 0.613337874412537
9.32564102564103 0.742851257324219
9.76923076923077 0.612782120704651
10.2333333333333 0.600774765014648
10.7205128205128 0.721608817577362
11.2307692307692 0.794548690319061
11.7666666666667 0.75697273015976
12.3282051282051 0.771108210086823
12.9153846153846 0.866081833839417
13.5282051282051 0.836605727672577
14.174358974359 0.88742333650589
14.8487179487179 0.862485229969025
15.5564102564103 0.723941922187805
16.2974358974359 0.788476467132568
17.0717948717949 0.893614768981934
17.8846153846154 0.896127641201019
18.7358974358974 0.941044270992279
19.6282051282051 0.897974908351898
20.5641025641026 0.982138097286224
21.5435897435897 0.826799511909485
22.5692307692308 0.891406238079071
23.6435897435897 0.942820489406586
24.7692307692308 0.996870636940002
25.9487179487179 0.996213734149933
27.1846153846154 0.885719478130341
28.4794871794872 0.810873985290527
29.8358974358974 0.940930962562561
31.2564102564103 0.994974553585052
32.7435897435897 0.987436413764954
34.3025641025641 0.987698554992676
35.9358974358974 0.9975865483284
37.648717948718 0.99799519777298
39.4410256410256 0.998528659343719
41.3179487179487 0.998277306556702
43.2871794871795 0.998535454273224
45.3487179487179 0.998576581478119
47.5076923076923 0.998590290546417
49.7692307692308 0.998758733272552
52.1384615384615 0.998707115650177
54.6205128205128 0.998626053333282
57.2230769230769 0.998427212238312
59.9461538461538 0.998748421669006
62.8025641025641 0.998623311519623
65.7923076923077 0.998376846313477
68.925641025641 0.998425602912903
72.2076923076923 0.998095333576202
75.6461538461539 0.998618245124817
79.2461538461538 0.998682677745819
83.0205128205128 0.997532486915588
86.974358974359 0.998243510723114
91.1153846153846 0.997492253780365
95.4538461538462 0.99787312746048
100 0.997606217861176
};
\addplot [, color1, opacity=0.6, mark=square*, mark size=0.5, mark options={solid}, only marks, forget plot]
table {%
1 0.927200734615326
1.04615384615385 0.835246384143829
1.0974358974359 0.82979553937912
1.14871794871795 0.843486905097961
1.2025641025641 0.794834673404694
1.26153846153846 0.750288784503937
1.32051282051282 0.711942732334137
1.38461538461538 0.783633232116699
1.44871794871795 0.705378592014313
1.51794871794872 0.593595504760742
1.58974358974359 0.625828742980957
1.66666666666667 0.668091952800751
1.74615384615385 0.719674706459045
1.82820512820513 0.713862895965576
1.91538461538462 0.564746201038361
2.00769230769231 0.5983567237854
2.1025641025641 0.697489738464355
2.20512820512821 0.655353128910065
2.30769230769231 0.563371002674103
2.41794871794872 0.629700005054474
2.53333333333333 0.588695645332336
2.65384615384615 0.506794154644012
2.78205128205128 0.569029033184052
2.91282051282051 0.552099049091339
3.05384615384615 0.582968890666962
3.1974358974359 0.504111230373383
3.35128205128205 0.600934267044067
3.51025641025641 0.536122918128967
3.67692307692308 0.622009456157684
3.85128205128205 0.520639896392822
4.03589743589744 0.524611413478851
4.22820512820513 0.642555058002472
4.42820512820513 0.551307618618011
4.64102564102564 0.463294506072998
4.86153846153846 0.545727550983429
5.09230769230769 0.63814502954483
5.33589743589744 0.448156893253326
5.58974358974359 0.487399101257324
5.85641025641026 0.547890543937683
6.13589743589744 0.516440451145172
6.42564102564103 0.601024508476257
6.73333333333333 0.52196854352951
7.05384615384615 0.713043987751007
7.38974358974359 0.662899196147919
7.74102564102564 0.561909854412079
8.11025641025641 0.573903620243073
8.4974358974359 0.59346330165863
8.9 0.555081367492676
9.32564102564103 0.503796994686127
9.76923076923077 0.519101142883301
10.2333333333333 0.799540936946869
10.7205128205128 0.631284654140472
11.2307692307692 0.611667394638062
11.7666666666667 0.715315103530884
12.3282051282051 0.870968997478485
12.9153846153846 0.785666704177856
13.5282051282051 0.569979429244995
14.174358974359 0.867416203022003
14.8487179487179 0.703119397163391
15.5564102564103 0.768798172473907
16.2974358974359 0.876714885234833
17.0717948717949 0.720046818256378
17.8846153846154 0.832312703132629
18.7358974358974 0.794714391231537
19.6282051282051 0.836330354213715
20.5641025641026 0.988329231739044
21.5435897435897 0.885040938854218
22.5692307692308 0.896116852760315
23.6435897435897 0.909578144550323
24.7692307692308 0.896529316902161
25.9487179487179 0.802074611186981
27.1846153846154 0.979644775390625
28.4794871794872 0.888455390930176
29.8358974358974 0.805100083351135
31.2564102564103 0.838900566101074
32.7435897435897 0.892108380794525
34.3025641025641 0.898391902446747
35.9358974358974 0.897099912166595
37.648717948718 0.899209201335907
39.4410256410256 0.898131966590881
41.3179487179487 0.897706985473633
43.2871794871795 0.898090660572052
45.3487179487179 0.89848804473877
47.5076923076923 0.898798406124115
49.7692307692308 0.89871734380722
52.1384615384615 0.898118436336517
54.6205128205128 0.898693382740021
57.2230769230769 0.898447632789612
59.9461538461538 0.89803546667099
62.8025641025641 0.89819872379303
65.7923076923077 0.898736178874969
68.925641025641 0.897642433643341
72.2076923076923 0.899059236049652
75.6461538461539 0.89784300327301
79.2461538461538 0.897017657756805
83.0205128205128 0.898508012294769
86.974358974359 0.897048771381378
91.1153846153846 0.897159516811371
95.4538461538462 0.897286832332611
100 0.897006034851074
};
\addplot [, color1, opacity=0.6, mark=square*, mark size=0.5, mark options={solid}, only marks, forget plot]
table {%
1 0.959721982479095
1.04615384615385 0.830729305744171
1.0974358974359 0.742717444896698
1.14871794871795 0.828927218914032
1.2025641025641 0.690834939479828
1.26153846153846 0.62695050239563
1.32051282051282 0.79490727186203
1.38461538461538 0.616857826709747
1.44871794871795 0.657617688179016
1.51794871794872 0.640562474727631
1.58974358974359 0.636058449745178
1.66666666666667 0.573876619338989
1.74615384615385 0.600208401679993
1.82820512820513 0.564954876899719
1.91538461538462 0.604279339313507
2.00769230769231 0.51511162519455
2.1025641025641 0.541607201099396
2.20512820512821 0.617485105991364
2.30769230769231 0.551774501800537
2.41794871794872 0.540675163269043
2.53333333333333 0.482316583395004
2.65384615384615 0.535847663879395
2.78205128205128 0.607663214206696
2.91282051282051 0.562097012996674
3.05384615384615 0.545476138591766
3.1974358974359 0.579287886619568
3.35128205128205 0.627390086650848
3.51025641025641 0.48978129029274
3.67692307692308 0.529894351959229
3.85128205128205 0.671911895275116
4.03589743589744 0.531701683998108
4.22820512820513 0.573252975940704
4.42820512820513 0.664062261581421
4.64102564102564 0.412338465452194
4.86153846153846 0.497174948453903
5.09230769230769 0.548701405525208
5.33589743589744 0.50704038143158
5.58974358974359 0.516927361488342
5.85641025641026 0.509875178337097
6.13589743589744 0.48820161819458
6.42564102564103 0.55949479341507
6.73333333333333 0.432226002216339
7.05384615384615 0.644303262233734
7.38974358974359 0.513285100460052
7.74102564102564 0.430866450071335
8.11025641025641 0.588914811611176
8.4974358974359 0.592985451221466
8.9 0.503445744514465
9.32564102564103 0.632157146930695
9.76923076923077 0.466337502002716
10.2333333333333 0.412775248289108
10.7205128205128 0.700217843055725
11.2307692307692 0.640450179576874
11.7666666666667 0.480789482593536
12.3282051282051 0.771128952503204
12.9153846153846 0.545400142669678
13.5282051282051 0.663133978843689
14.174358974359 0.785820305347443
14.8487179487179 0.858202159404755
15.5564102564103 0.945003151893616
16.2974358974359 0.909439027309418
17.0717948717949 0.846515119075775
17.8846153846154 0.844776332378387
18.7358974358974 0.988903939723969
19.6282051282051 0.954987645149231
20.5641025641026 0.85643857717514
21.5435897435897 0.899477303028107
22.5692307692308 0.889916121959686
23.6435897435897 0.955689907073975
24.7692307692308 0.943973362445831
25.9487179487179 0.898830056190491
27.1846153846154 0.893009781837463
28.4794871794872 0.964155316352844
29.8358974358974 0.894880294799805
31.2564102564103 0.788550078868866
32.7435897435897 0.786848485469818
34.3025641025641 0.794748246669769
35.9358974358974 0.777309417724609
37.648717948718 0.795575082302094
39.4410256410256 0.799899101257324
41.3179487179487 0.802555978298187
43.2871794871795 0.794930696487427
45.3487179487179 0.804093182086945
47.5076923076923 0.795473515987396
49.7692307692308 0.805805325508118
52.1384615384615 0.803564190864563
54.6205128205128 0.801897168159485
57.2230769230769 0.803995072841644
59.9461538461538 0.804649949073792
62.8025641025641 0.801822602748871
65.7923076923077 0.804203033447266
68.925641025641 0.802935898303986
72.2076923076923 0.806071102619171
75.6461538461539 0.804591596126556
79.2461538461538 0.805501103401184
83.0205128205128 0.805583596229553
86.974358974359 0.805423080921173
91.1153846153846 0.806487679481506
95.4538461538462 0.805665791034698
100 0.80473655462265
};
\addplot [, color2, opacity=0.6, mark=triangle*, mark size=0.5, mark options={solid,rotate=180}, only marks]
table {%
1 0.74284303188324
1.04615384615385 0.594265580177307
1.0974358974359 0.582200944423676
1.14871794871795 0.535216331481934
1.2025641025641 0.550718545913696
1.26153846153846 0.523808181285858
1.32051282051282 0.489897310733795
1.38461538461538 0.403536528348923
1.44871794871795 0.469120025634766
1.51794871794872 0.545713663101196
1.58974358974359 0.447541862726212
1.66666666666667 0.426879703998566
1.74615384615385 0.438715845346451
1.82820512820513 0.426827818155289
1.91538461538462 0.392357647418976
2.00769230769231 0.367561250925064
2.1025641025641 0.366983622312546
2.20512820512821 0.355259388685226
2.30769230769231 0.412537574768066
2.41794871794872 0.354205518960953
2.53333333333333 0.356822073459625
2.65384615384615 0.440038204193115
2.78205128205128 0.417433738708496
2.91282051282051 0.345641702413559
3.05384615384615 0.410464376211166
3.1974358974359 0.37027570605278
3.35128205128205 0.291887372732162
3.51025641025641 0.344459235668182
3.67692307692308 0.34151753783226
3.85128205128205 0.305305063724518
4.03589743589744 0.315390855073929
4.22820512820513 0.278640657663345
4.42820512820513 0.261379957199097
4.64102564102564 0.286091476678848
4.86153846153846 0.309860944747925
5.09230769230769 0.228047519922256
5.33589743589744 0.239862680435181
5.58974358974359 0.240155413746834
5.85641025641026 0.210618242621422
6.13589743589744 0.272115707397461
6.42564102564103 0.33628985285759
6.73333333333333 0.241422653198242
7.05384615384615 0.239400193095207
7.38974358974359 0.258078306913376
7.74102564102564 0.2094886302948
8.11025641025641 0.215820342302322
8.4974358974359 0.220022484660149
8.9 0.225999981164932
9.32564102564103 0.205221489071846
9.76923076923077 0.188242077827454
10.2333333333333 0.210363194346428
10.7205128205128 0.145949944853783
11.2307692307692 0.226569920778275
11.7666666666667 0.253981381654739
12.3282051282051 0.164124175906181
12.9153846153846 0.233429297804832
13.5282051282051 0.209901735186577
14.174358974359 0.160326391458511
14.8487179487179 0.16858683526516
15.5564102564103 0.139621928334236
16.2974358974359 0.0773060843348503
17.0717948717949 0.213088855147362
17.8846153846154 0.0524809435009956
18.7358974358974 0.237321779131889
19.6282051282051 0.329720348119736
20.5641025641026 0.214002177119255
21.5435897435897 0.139905467629433
22.5692307692308 0.131376191973686
23.6435897435897 0.134114429354668
24.7692307692308 0.113936819136143
25.9487179487179 0.135955482721329
27.1846153846154 0.226227685809135
28.4794871794872 0.13983927667141
29.8358974358974 0.140501692891121
31.2564102564103 0.12704886496067
32.7435897435897 0.132055222988129
34.3025641025641 0.138734966516495
35.9358974358974 0.137806639075279
37.648717948718 0.133760407567024
39.4410256410256 0.130749568343163
41.3179487179487 0.202853426337242
43.2871794871795 0.135560631752014
45.3487179487179 0.146380946040154
47.5076923076923 0.131638512015343
49.7692307692308 0.145544931292534
52.1384615384615 0.144693225622177
54.6205128205128 0.143227770924568
57.2230769230769 0.23012638092041
59.9461538461538 0.196058496832848
62.8025641025641 0.147666320204735
65.7923076923077 0.235942795872688
68.925641025641 0.140061125159264
72.2076923076923 0.19306543469429
75.6461538461539 0.141008034348488
79.2461538461538 0.175163432955742
83.0205128205128 0.140365228056908
86.974358974359 0.221244767308235
91.1153846153846 0.153332620859146
95.4538461538462 0.13706211745739
100 0.18977153301239
};
\addlegendentry{sub 16, mc 1}
\addplot [, color2, opacity=0.6, mark=triangle*, mark size=0.5, mark options={solid,rotate=180}, only marks, forget plot]
table {%
1 0.621782720088959
1.04615384615385 0.633713185787201
1.0974358974359 0.567273020744324
1.14871794871795 0.647039532661438
1.2025641025641 0.510732769966125
1.26153846153846 0.513084888458252
1.32051282051282 0.607173144817352
1.38461538461538 0.513554990291595
1.44871794871795 0.489306449890137
1.51794871794872 0.509667098522186
1.58974358974359 0.434726923704147
1.66666666666667 0.469236761331558
1.74615384615385 0.468300253152847
1.82820512820513 0.470266819000244
1.91538461538462 0.444177448749542
2.00769230769231 0.456110954284668
2.1025641025641 0.47337332367897
2.20512820512821 0.406676054000854
2.30769230769231 0.423035144805908
2.41794871794872 0.478607326745987
2.53333333333333 0.402779310941696
2.65384615384615 0.468663901090622
2.78205128205128 0.42091828584671
2.91282051282051 0.342396110296249
3.05384615384615 0.477976620197296
3.1974358974359 0.287411689758301
3.35128205128205 0.368641436100006
3.51025641025641 0.430614918470383
3.67692307692308 0.343341112136841
3.85128205128205 0.31524384021759
4.03589743589744 0.399893164634705
4.22820512820513 0.28250440955162
4.42820512820513 0.337821930646896
4.64102564102564 0.326971888542175
4.86153846153846 0.316693514585495
5.09230769230769 0.410803645849228
5.33589743589744 0.262609243392944
5.58974358974359 0.342314392328262
5.85641025641026 0.305377811193466
6.13589743589744 0.348776757717133
6.42564102564103 0.356109797954559
6.73333333333333 0.283529013395309
7.05384615384615 0.304987847805023
7.38974358974359 0.199842497706413
7.74102564102564 0.342151492834091
8.11025641025641 0.220849618315697
8.4974358974359 0.299403965473175
8.9 0.391372472047806
9.32564102564103 0.281635761260986
9.76923076923077 0.256845772266388
10.2333333333333 0.30902037024498
10.7205128205128 0.268728703260422
11.2307692307692 0.193685501813889
11.7666666666667 0.284249782562256
12.3282051282051 0.406966120004654
12.9153846153846 0.176910251379013
13.5282051282051 0.116016246378422
14.174358974359 0.231103286147118
14.8487179487179 0.228584364056587
15.5564102564103 0.164062276482582
16.2974358974359 0.144531056284904
17.0717948717949 0.157027319073677
17.8846153846154 0.0465348549187183
18.7358974358974 0.244962126016617
19.6282051282051 0.163117036223412
20.5641025641026 0.153436571359634
21.5435897435897 0.324886560440063
22.5692307692308 0.0582617409527302
23.6435897435897 0.129971414804459
24.7692307692308 0.2214724868536
25.9487179487179 0.0502018742263317
27.1846153846154 0.240282967686653
28.4794871794872 0.0759215578436852
29.8358974358974 0.137685284018517
31.2564102564103 0.138853088021278
32.7435897435897 0.0995348319411278
34.3025641025641 0.0692401081323624
35.9358974358974 0.0633756816387177
37.648717948718 0.0604003071784973
39.4410256410256 0.057694997638464
41.3179487179487 0.0463193207979202
43.2871794871795 0.0679224282503128
45.3487179487179 0.0683257281780243
47.5076923076923 0.061274316161871
49.7692307692308 0.0580927617847919
52.1384615384615 0.0731623247265816
54.6205128205128 0.0567981600761414
57.2230769230769 0.0618058256804943
59.9461538461538 0.0675641074776649
62.8025641025641 0.0550076626241207
65.7923076923077 0.0676625892519951
68.925641025641 0.0693793073296547
72.2076923076923 0.0610913708806038
75.6461538461539 0.0616944395005703
79.2461538461538 0.0633292570710182
83.0205128205128 nan
86.974358974359 nan
91.1153846153846 0.0588746778666973
95.4538461538462 0.0566262863576412
100 0.054128885269165
};
\addplot [, color2, opacity=0.6, mark=triangle*, mark size=0.5, mark options={solid,rotate=180}, only marks, forget plot]
table {%
1 0.68679141998291
1.04615384615385 0.511579930782318
1.0974358974359 0.478756427764893
1.14871794871795 0.539990901947021
1.2025641025641 0.464941948652267
1.26153846153846 0.388928681612015
1.32051282051282 0.471978902816772
1.38461538461538 0.422588318586349
1.44871794871795 0.456825703382492
1.51794871794872 0.356220453977585
1.58974358974359 0.372715681791306
1.66666666666667 0.403001934289932
1.74615384615385 0.394618093967438
1.82820512820513 0.356650918722153
1.91538461538462 0.368763029575348
2.00769230769231 0.380229771137238
2.1025641025641 0.387009769678116
2.20512820512821 0.290951013565063
2.30769230769231 0.321035444736481
2.41794871794872 0.383918553590775
2.53333333333333 0.316956460475922
2.65384615384615 0.298249334096909
2.78205128205128 0.351589560508728
2.91282051282051 0.346791684627533
3.05384615384615 0.372452020645142
3.1974358974359 0.328254997730255
3.35128205128205 0.319634675979614
3.51025641025641 0.380030125379562
3.67692307692308 0.278124332427979
3.85128205128205 0.313339918851852
4.03589743589744 0.285797655582428
4.22820512820513 0.253774404525757
4.42820512820513 0.33165630698204
4.64102564102564 0.232301279902458
4.86153846153846 0.278479427099228
5.09230769230769 0.342129290103912
5.33589743589744 0.231906577944756
5.58974358974359 0.295335739850998
5.85641025641026 0.323696881532669
6.13589743589744 0.207163333892822
6.42564102564103 0.264175683259964
6.73333333333333 0.289712429046631
7.05384615384615 0.254226177930832
7.38974358974359 0.238140925765038
7.74102564102564 0.227580308914185
8.11025641025641 0.23001816868782
8.4974358974359 0.223182320594788
8.9 0.239782854914665
9.32564102564103 0.193378448486328
9.76923076923077 0.175044104456902
10.2333333333333 0.109779201447964
10.7205128205128 0.146693512797356
11.2307692307692 0.147559508681297
11.7666666666667 0.251749753952026
12.3282051282051 0.154674634337425
12.9153846153846 0.160333827137947
13.5282051282051 0.145509570837021
14.174358974359 0.230059057474136
14.8487179487179 0.150194451212883
15.5564102564103 0.254905968904495
16.2974358974359 0.148294702172279
17.0717948717949 0.0737981870770454
17.8846153846154 0.139655560255051
18.7358974358974 0.135342463850975
19.6282051282051 0.136136278510094
20.5641025641026 0.220452308654785
21.5435897435897 0.0357583239674568
22.5692307692308 0.124774575233459
23.6435897435897 0.13560950756073
24.7692307692308 0.137832924723625
25.9487179487179 0.120718769729137
27.1846153846154 0.123197630047798
28.4794871794872 0.136786639690399
29.8358974358974 0.1281828135252
31.2564102564103 0.140085130929947
32.7435897435897 0.219805344939232
34.3025641025641 0.198580726981163
35.9358974358974 0.3084976375103
37.648717948718 0.221294194459915
39.4410256410256 0.221197232604027
41.3179487179487 0.304427355527878
43.2871794871795 0.220856785774231
45.3487179487179 0.224959447979927
47.5076923076923 0.403458118438721
49.7692307692308 0.405332148075104
52.1384615384615 0.221586629748344
54.6205128205128 0.287837356328964
57.2230769230769 0.307997077703476
59.9461538461538 0.21554659307003
62.8025641025641 0.224887683987617
65.7923076923077 0.218262076377869
68.925641025641 0.220648869872093
72.2076923076923 0.219732567667961
75.6461538461539 0.292805820703506
79.2461538461538 0.245918139815331
83.0205128205128 0.282368868589401
86.974358974359 0.317468345165253
91.1153846153846 0.316363424062729
95.4538461538462 0.309882044792175
100 0.273269027471542
};
\addplot [, color2, opacity=0.6, mark=triangle*, mark size=0.5, mark options={solid,rotate=180}, only marks, forget plot]
table {%
1 0.616450667381287
1.04615384615385 0.57198041677475
1.0974358974359 0.532268762588501
1.14871794871795 0.548265874385834
1.2025641025641 0.456155121326447
1.26153846153846 0.44397822022438
1.32051282051282 0.447184145450592
1.38461538461538 0.471030056476593
1.44871794871795 0.451751321554184
1.51794871794872 0.437752217054367
1.58974358974359 0.507778584957123
1.66666666666667 0.47242346405983
1.74615384615385 0.464070618152618
1.82820512820513 0.42892798781395
1.91538461538462 0.423027336597443
2.00769230769231 0.422724068164825
2.1025641025641 0.414006799459457
2.20512820512821 0.383384615182877
2.30769230769231 0.428860902786255
2.41794871794872 0.36404949426651
2.53333333333333 0.387068957090378
2.65384615384615 0.400747835636139
2.78205128205128 0.469235181808472
2.91282051282051 0.363458961248398
3.05384615384615 0.395039945840836
3.1974358974359 0.37841796875
3.35128205128205 0.384096652269363
3.51025641025641 0.321300089359283
3.67692307692308 0.305724501609802
3.85128205128205 0.328292846679688
4.03589743589744 0.369243949651718
4.22820512820513 0.407843321561813
4.42820512820513 0.407473772764206
4.64102564102564 0.337266057729721
4.86153846153846 0.390900164842606
5.09230769230769 0.357556849718094
5.33589743589744 0.307588785886765
5.58974358974359 0.307081192731857
5.85641025641026 0.352511078119278
6.13589743589744 0.307879030704498
6.42564102564103 0.394352644681931
6.73333333333333 0.377731770277023
7.05384615384615 0.310075104236603
7.38974358974359 0.393482357263565
7.74102564102564 0.307421684265137
8.11025641025641 0.260790824890137
8.4974358974359 0.242032870650291
8.9 0.32552170753479
9.32564102564103 0.329445749521255
9.76923076923077 0.301464706659317
10.2333333333333 0.336090534925461
10.7205128205128 0.221394822001457
11.2307692307692 0.244100958108902
11.7666666666667 0.308478385210037
12.3282051282051 0.280924528837204
12.9153846153846 0.226736381649971
13.5282051282051 0.137592777609825
14.174358974359 0.184461265802383
14.8487179487179 0.218868732452393
15.5564102564103 0.315623193979263
16.2974358974359 0.211334899067879
17.0717948717949 0.309948891401291
17.8846153846154 0.131523072719574
18.7358974358974 0.318255752325058
19.6282051282051 0.0981003642082214
20.5641025641026 0.147278770804405
21.5435897435897 0.234940528869629
22.5692307692308 0.119395278394222
23.6435897435897 0.129205614328384
24.7692307692308 0.208319619297981
25.9487179487179 0.210576340556145
27.1846153846154 0.230031877756119
28.4794871794872 0.114473305642605
29.8358974358974 0.125488460063934
31.2564102564103 0.111400365829468
32.7435897435897 0.129786536097527
34.3025641025641 0.133529767394066
35.9358974358974 0.131855502724648
37.648717948718 0.131231427192688
39.4410256410256 0.134585723280907
41.3179487179487 0.133143976330757
43.2871794871795 0.132312551140785
45.3487179487179 0.198470458388329
47.5076923076923 0.132774144411087
49.7692307692308 0.187002643942833
52.1384615384615 0.167074546217918
54.6205128205128 0.133658528327942
57.2230769230769 0.13813242316246
59.9461538461538 0.232295155525208
62.8025641025641 0.214778617024422
65.7923076923077 0.17849163711071
68.925641025641 0.230980917811394
72.2076923076923 0.142784148454666
75.6461538461539 0.206542015075684
79.2461538461538 0.144371077418327
83.0205128205128 0.216123625636101
86.974358974359 0.140805304050446
91.1153846153846 0.217264324426651
95.4538461538462 0.189136326313019
100 0.23533533513546
};
\addplot [, color2, opacity=0.6, mark=triangle*, mark size=0.5, mark options={solid,rotate=180}, only marks, forget plot]
table {%
1 0.621948480606079
1.04615384615385 0.509655058383942
1.0974358974359 0.455432713031769
1.14871794871795 0.50035947561264
1.2025641025641 0.474085241556168
1.26153846153846 0.365668147802353
1.32051282051282 0.462633579969406
1.38461538461538 0.426335066556931
1.44871794871795 0.421554386615753
1.51794871794872 0.405564785003662
1.58974358974359 0.473861902952194
1.66666666666667 0.464175432920456
1.74615384615385 0.440974444150925
1.82820512820513 0.447666168212891
1.91538461538462 0.45708566904068
2.00769230769231 0.424381643533707
2.1025641025641 0.415000766515732
2.20512820512821 0.371209472417831
2.30769230769231 0.4297776222229
2.41794871794872 0.376116126775742
2.53333333333333 0.355036377906799
2.65384615384615 0.414596945047379
2.78205128205128 0.381172001361847
2.91282051282051 0.353638023138046
3.05384615384615 0.336888760328293
3.1974358974359 0.345975965261459
3.35128205128205 0.334907948970795
3.51025641025641 0.323156446218491
3.67692307692308 0.342840224504471
3.85128205128205 0.35687991976738
4.03589743589744 0.349708795547485
4.22820512820513 0.301390737295151
4.42820512820513 0.348885595798492
4.64102564102564 0.287563979625702
4.86153846153846 0.299157470464706
5.09230769230769 0.315125554800034
5.33589743589744 0.297979116439819
5.58974358974359 0.293383359909058
5.85641025641026 0.266661405563354
6.13589743589744 0.272027730941772
6.42564102564103 0.217613607645035
6.73333333333333 0.289548367261887
7.05384615384615 0.202727660536766
7.38974358974359 0.210116177797318
7.74102564102564 0.19472873210907
8.11025641025641 0.201257735490799
8.4974358974359 0.225104287266731
8.9 0.171181604266167
9.32564102564103 0.172673925757408
9.76923076923077 0.194942459464073
10.2333333333333 0.192020401358604
10.7205128205128 0.159116148948669
11.2307692307692 0.220798596739769
11.7666666666667 0.163042649626732
12.3282051282051 0.20427393913269
12.9153846153846 0.127197876572609
13.5282051282051 0.173477575182915
14.174358974359 0.173957988619804
14.8487179487179 0.263855367898941
15.5564102564103 0.161006763577461
16.2974358974359 0.146021142601967
17.0717948717949 0.150520473718643
17.8846153846154 0.162144914269447
18.7358974358974 0.2149628251791
19.6282051282051 0.184642001986504
20.5641025641026 0.219014689326286
21.5435897435897 0.258532106876373
22.5692307692308 0.257124155759811
23.6435897435897 0.215046882629395
24.7692307692308 0.0956316217780113
25.9487179487179 0.24002905189991
27.1846153846154 0.144243642687798
28.4794871794872 0.232993319630623
29.8358974358974 0.163536831736565
31.2564102564103 0.148799508810043
32.7435897435897 0.226669296622276
34.3025641025641 0.262429088354111
35.9358974358974 0.32195708155632
37.648717948718 0.232509762048721
39.4410256410256 0.154468044638634
41.3179487179487 0.188809290528297
43.2871794871795 0.166729435324669
45.3487179487179 0.151349052786827
47.5076923076923 0.155028611421585
49.7692307692308 0.150149777531624
52.1384615384615 0.150920256972313
54.6205128205128 0.250243723392487
57.2230769230769 0.154768511652946
59.9461538461538 0.240688920021057
62.8025641025641 0.170719489455223
65.7923076923077 0.153680682182312
68.925641025641 0.254215568304062
72.2076923076923 0.162367030978203
75.6461538461539 0.149551376700401
79.2461538461538 0.158602118492126
83.0205128205128 0.200760990381241
86.974358974359 0.147988706827164
91.1153846153846 0.14683435857296
95.4538461538462 0.151927590370178
100 0.156697437167168
};
\end{axis}

\end{tikzpicture}

    \tikzexternaldisable
  \end{minipage}\hfill
  \begin{minipage}{0.50\linewidth}
    \centering
    % defines the pgfplots style "eigspacedefault"
\pgfkeys{/pgfplots/eigspacedefault/.style={
    width=1.03\linewidth,
    height=\goldenRatioInv*1.03*\linewidth,
    every axis plot/.append style={line width = 1pt},
    tick pos = left,
    ylabel near ticks,
    xlabel near ticks,
    xtick align = inside,
    ytick align = inside,
    legend cell align = left,
    legend columns = 1,
    legend pos = north east,
    legend style = {
      fill opacity = 0.9,
      text opacity = 1,
      font = \tiny,
      % column sep=0.1cm,
    },
    legend image post style={scale=2},
    xticklabel style = {font = \small},
    xlabel style = {font = \small},
    axis line style = {black},
    yticklabel style = {font = \small},
    ylabel style = {font = \small},
    title style = {font = \small},
    grid = major,
    grid style = {dashed}
  }
}

\pgfkeys{/pgfplots/eigspacedefaultapp/.style={
    eigspacedefault,
    height=0.6\linewidth,
    legend columns = 2,
  }
}

\pgfkeys{/pgfplots/eigspacenolegend/.style={
    legend image post style = {scale=0},
    legend style = {
      fill opacity = 0,
      draw opacity = 0,
      text opacity = 0,
      font = \small,
      at={(1, 1.025)},
      anchor=south east,
      column sep=0.25cm,
    },
  }
}
%%% Local Variables:
%%% mode: latex
%%% TeX-master: "../main"
%%% End:

    \pgfkeys{/pgfplots/zmystyle/.style={
        eigspacedefaultapp,
        legend columns = 3,
      }}
    \tikzexternalenable
    % This file was created by tikzplotlib v0.9.7.
\begin{tikzpicture}

\definecolor{color0}{rgb}{0.274509803921569,0.6,0.564705882352941}
\definecolor{color1}{rgb}{0.870588235294118,0.623529411764706,0.0862745098039216}
\definecolor{color2}{rgb}{0.501960784313725,0.184313725490196,0.6}

\begin{axis}[
axis line style={white!10!black},
legend style={fill opacity=0.8, draw opacity=1, text opacity=1, at={(0.03,0.03)}, anchor=south west, draw=white!80!black},
log basis x={10},
tick pos=left,
xlabel={epoch (log scale)},
xmajorgrids,
xmin=0.794328234724281, xmax=125.892541179417,
xmode=log,
ylabel={overlap},
ymajorgrids,
ymin=-0.05, ymax=1.05,
zmystyle
]
\addplot [, white!10!black, dashed, forget plot]
table {%
0.794328234724281 1
125.892541179417 1
};
\addplot [, white!10!black, dashed, forget plot]
table {%
0.794328234724281 0
125.892541179417 0
};
\addplot [, color0, opacity=0.6, mark=diamond*, mark size=0.5, mark options={solid}, only marks]
table {%
1 0.931371986865997
1.04615384615385 0.96322900056839
1.0974358974359 0.938497722148895
1.14871794871795 0.848906457424164
1.2025641025641 0.834847569465637
1.26153846153846 0.822537422180176
1.32051282051282 0.77154940366745
1.38461538461538 0.787878215312958
1.44871794871795 0.768930733203888
1.51794871794872 0.743101239204407
1.58974358974359 0.742821156978607
1.66666666666667 0.746570408344269
1.74615384615385 0.685557842254639
1.82820512820513 0.688570737838745
1.91538461538462 0.693896174430847
2.00769230769231 0.632382392883301
2.1025641025641 0.640466034412384
2.20512820512821 0.664279282093048
2.30769230769231 0.653564870357513
2.41794871794872 0.668948829174042
2.53333333333333 0.607766091823578
2.65384615384615 0.623151302337646
2.78205128205128 0.598952054977417
2.91282051282051 0.591911494731903
3.05384615384615 0.588782787322998
3.1974358974359 0.569038331508636
3.35128205128205 0.593773305416107
3.51025641025641 0.575068473815918
3.67692307692308 0.574278652667999
3.85128205128205 0.552568852901459
4.03589743589744 0.559215068817139
4.22820512820513 0.548115074634552
4.42820512820513 0.530776500701904
4.64102564102564 0.539300858974457
4.86153846153846 0.55285257101059
5.09230769230769 0.53750604391098
5.33589743589744 0.6219162940979
5.58974358974359 0.525110602378845
5.85641025641026 0.570009052753448
6.13589743589744 0.539032459259033
6.42564102564103 0.490142166614532
6.73333333333333 0.538312911987305
7.05384615384615 0.49654945731163
7.38974358974359 0.523170351982117
7.74102564102564 0.525020778179169
8.11025641025641 0.510048031806946
8.4974358974359 0.51035863161087
8.9 0.495437532663345
9.32564102564103 0.540705621242523
9.76923076923077 0.494072407484055
10.2333333333333 0.513156056404114
10.7205128205128 0.544552505016327
11.2307692307692 0.509057819843292
11.7666666666667 0.516443192958832
12.3282051282051 0.533310055732727
12.9153846153846 0.492357701063156
13.5282051282051 0.504920721054077
14.174358974359 0.473214685916901
14.8487179487179 0.487729400396347
15.5564102564103 0.494375437498093
16.2974358974359 0.470282763242722
17.0717948717949 0.461355358362198
17.8846153846154 0.461103767156601
18.7358974358974 0.463863134384155
19.6282051282051 0.438121140003204
20.5641025641026 0.461266607046127
21.5435897435897 0.4592305123806
22.5692307692308 0.474714666604996
23.6435897435897 0.402314484119415
24.7692307692308 0.419796526432037
25.9487179487179 0.374137163162231
27.1846153846154 0.393620014190674
28.4794871794872 0.370377421379089
29.8358974358974 0.4055016040802
31.2564102564103 0.338426887989044
32.7435897435897 0.331007122993469
34.3025641025641 0.346495479345322
35.9358974358974 0.379899471998215
37.648717948718 0.327445179224014
39.4410256410256 0.347462981939316
41.3179487179487 0.390766382217407
43.2871794871795 0.397229343652725
45.3487179487179 0.289783477783203
47.5076923076923 0.388422101736069
49.7692307692308 0.314549833536148
52.1384615384615 0.28733816742897
54.6205128205128 0.333138734102249
57.2230769230769 0.375070095062256
59.9461538461538 0.301859706640244
62.8025641025641 0.208200097084045
65.7923076923077 0.292004317045212
68.925641025641 0.282974034547806
72.2076923076923 0.35561215877533
75.6461538461539 0.312119007110596
79.2461538461538 0.296703338623047
83.0205128205128 0.210383400321007
86.974358974359 0.359309434890747
91.1153846153846 0.375915110111237
95.4538461538462 0.313500642776489
100 0.27208599448204
};
\addlegendentry{sub 16, exact}
\addplot [, color0, opacity=0.6, mark=diamond*, mark size=0.5, mark options={solid}, only marks, forget plot]
table {%
1 0.904443919658661
1.04615384615385 0.977323949337006
1.0974358974359 0.964760959148407
1.14871794871795 0.949945628643036
1.2025641025641 0.930304527282715
1.26153846153846 0.93744957447052
1.32051282051282 0.897895634174347
1.38461538461538 0.924534976482391
1.44871794871795 0.860171318054199
1.51794871794872 0.873505115509033
1.58974358974359 0.849352061748505
1.66666666666667 0.822315037250519
1.74615384615385 0.775716781616211
1.82820512820513 0.761541485786438
1.91538461538462 0.785734832286835
2.00769230769231 0.795824348926544
2.1025641025641 0.784194946289062
2.20512820512821 0.777622759342194
2.30769230769231 0.766446709632874
2.41794871794872 0.785049855709076
2.53333333333333 0.791922569274902
2.65384615384615 0.758206188678741
2.78205128205128 0.765681564807892
2.91282051282051 0.727280080318451
3.05384615384615 0.750959694385529
3.1974358974359 0.711860954761505
3.35128205128205 0.701527833938599
3.51025641025641 0.739156007766724
3.67692307692308 0.690954983234406
3.85128205128205 0.681222081184387
4.03589743589744 0.749838590621948
4.22820512820513 0.707533180713654
4.42820512820513 0.660037279129028
4.64102564102564 0.666621565818787
4.86153846153846 0.654085457324982
5.09230769230769 0.626910984516144
5.33589743589744 0.629892766475677
5.58974358974359 0.66740757226944
5.85641025641026 0.655122756958008
6.13589743589744 0.647192180156708
6.42564102564103 0.619207322597504
6.73333333333333 0.56365305185318
7.05384615384615 0.552297055721283
7.38974358974359 0.620782494544983
7.74102564102564 0.592422604560852
8.11025641025641 0.565629422664642
8.4974358974359 0.593790471553802
8.9 0.628581345081329
9.32564102564103 0.550622403621674
9.76923076923077 0.559242427349091
10.2333333333333 0.57891172170639
10.7205128205128 0.55890291929245
11.2307692307692 0.584519982337952
11.7666666666667 0.582470417022705
12.3282051282051 0.562712728977203
12.9153846153846 0.518826544284821
13.5282051282051 0.550443828105927
14.174358974359 0.571772634983063
14.8487179487179 0.53442257642746
15.5564102564103 0.510375320911407
16.2974358974359 0.528646171092987
17.0717948717949 0.525865852832794
17.8846153846154 0.536481738090515
18.7358974358974 0.50893759727478
19.6282051282051 0.453103065490723
20.5641025641026 0.51286518573761
21.5435897435897 0.468848317861557
22.5692307692308 0.441456943750381
23.6435897435897 0.45984360575676
24.7692307692308 0.455177038908005
25.9487179487179 0.401520073413849
27.1846153846154 0.401866525411606
28.4794871794872 0.382660001516342
29.8358974358974 0.414797782897949
31.2564102564103 0.378682196140289
32.7435897435897 0.406859695911407
34.3025641025641 0.375180780887604
35.9358974358974 0.356091290712357
37.648717948718 0.426462024450302
39.4410256410256 0.341139376163483
41.3179487179487 0.37810605764389
43.2871794871795 0.409331232309341
45.3487179487179 0.444772005081177
47.5076923076923 0.41332283616066
49.7692307692308 0.356132507324219
52.1384615384615 0.371595472097397
54.6205128205128 0.35612553358078
57.2230769230769 0.409404337406158
59.9461538461538 0.333006113767624
62.8025641025641 0.403908640146255
65.7923076923077 0.350346326828003
68.925641025641 0.361907094717026
72.2076923076923 0.303952723741531
75.6461538461539 0.34015092253685
79.2461538461538 0.2955182492733
83.0205128205128 0.47076627612114
86.974358974359 0.362420678138733
91.1153846153846 0.359449833631516
95.4538461538462 0.338150084018707
100 0.338216066360474
};
\addplot [, color0, opacity=0.6, mark=diamond*, mark size=0.5, mark options={solid}, only marks, forget plot]
table {%
1 0.948158860206604
1.04615384615385 0.982429325580597
1.0974358974359 0.969897449016571
1.14871794871795 0.944676876068115
1.2025641025641 0.921186447143555
1.26153846153846 0.938880741596222
1.32051282051282 0.919399857521057
1.38461538461538 0.87498539686203
1.44871794871795 0.832820236682892
1.51794871794872 0.853379666805267
1.58974358974359 0.850881397724152
1.66666666666667 0.851366221904755
1.74615384615385 0.838709652423859
1.82820512820513 0.834705948829651
1.91538461538462 0.774690687656403
2.00769230769231 0.804994583129883
2.1025641025641 0.782826125621796
2.20512820512821 0.75502997636795
2.30769230769231 0.751113891601562
2.41794871794872 0.698835372924805
2.53333333333333 0.715139865875244
2.65384615384615 0.728498637676239
2.78205128205128 0.718666970729828
2.91282051282051 0.70594334602356
3.05384615384615 0.709532380104065
3.1974358974359 0.714299440383911
3.35128205128205 0.635534942150116
3.51025641025641 0.69182938337326
3.67692307692308 0.655323624610901
3.85128205128205 0.696106135845184
4.03589743589744 0.672778785228729
4.22820512820513 0.644194722175598
4.42820512820513 0.625506699085236
4.64102564102564 0.688338696956635
4.86153846153846 0.678610563278198
5.09230769230769 0.604192197322845
5.33589743589744 0.732703685760498
5.58974358974359 0.574887990951538
5.85641025641026 0.597924590110779
6.13589743589744 0.64366352558136
6.42564102564103 0.647051632404327
6.73333333333333 0.634600281715393
7.05384615384615 0.56816577911377
7.38974358974359 0.619665443897247
7.74102564102564 0.598133862018585
8.11025641025641 0.633527398109436
8.4974358974359 0.572057783603668
8.9 0.606604278087616
9.32564102564103 0.579414069652557
9.76923076923077 0.585327804088593
10.2333333333333 0.613525569438934
10.7205128205128 0.55722188949585
11.2307692307692 0.588339507579803
11.7666666666667 0.578786373138428
12.3282051282051 0.560280025005341
12.9153846153846 0.529961109161377
13.5282051282051 0.545022785663605
14.174358974359 0.515532612800598
14.8487179487179 0.540223062038422
15.5564102564103 0.514771342277527
16.2974358974359 0.513055503368378
17.0717948717949 0.463017702102661
17.8846153846154 0.511470556259155
18.7358974358974 0.552581906318665
19.6282051282051 0.464524835348129
20.5641025641026 0.504813373088837
21.5435897435897 0.46931990981102
22.5692307692308 0.481405168771744
23.6435897435897 0.472581624984741
24.7692307692308 0.475986003875732
25.9487179487179 0.439350217580795
27.1846153846154 0.505414247512817
28.4794871794872 0.50215208530426
29.8358974358974 0.484551519155502
31.2564102564103 0.409794956445694
32.7435897435897 0.426196664571762
34.3025641025641 0.439276993274689
35.9358974358974 0.464258342981339
37.648717948718 0.551153957843781
39.4410256410256 0.525932550430298
41.3179487179487 0.48059019446373
43.2871794871795 0.471961885690689
45.3487179487179 0.483078300952911
47.5076923076923 0.425368636846542
49.7692307692308 0.551836907863617
52.1384615384615 0.466845989227295
54.6205128205128 0.546646535396576
57.2230769230769 0.48881521821022
59.9461538461538 0.408119261264801
62.8025641025641 0.457699358463287
65.7923076923077 0.545246481895447
68.925641025641 0.471869677305222
72.2076923076923 0.375852346420288
75.6461538461539 0.380674779415131
79.2461538461538 0.392376273870468
83.0205128205128 0.488833427429199
86.974358974359 0.465224742889404
91.1153846153846 0.436120241880417
95.4538461538462 0.45813399553299
100 0.546186745166779
};
\addplot [, color0, opacity=0.6, mark=diamond*, mark size=0.5, mark options={solid}, only marks, forget plot]
table {%
1 0.959887027740479
1.04615384615385 0.972443699836731
1.0974358974359 0.958850800991058
1.14871794871795 0.936413109302521
1.2025641025641 0.900718510150909
1.26153846153846 0.906850278377533
1.32051282051282 0.846279621124268
1.38461538461538 0.84942227602005
1.44871794871795 0.786740243434906
1.51794871794872 0.802100777626038
1.58974358974359 0.775953829288483
1.66666666666667 0.767481505870819
1.74615384615385 0.756786644458771
1.82820512820513 0.749194085597992
1.91538461538462 0.711860477924347
2.00769230769231 0.756849706172943
2.1025641025641 0.737291634082794
2.20512820512821 0.739580571651459
2.30769230769231 0.731592774391174
2.41794871794872 0.766760349273682
2.53333333333333 0.76762193441391
2.65384615384615 0.702036678791046
2.78205128205128 0.642773985862732
2.91282051282051 0.68508642911911
3.05384615384615 0.743305563926697
3.1974358974359 0.669253468513489
3.35128205128205 0.701866745948792
3.51025641025641 0.700826942920685
3.67692307692308 0.703789830207825
3.85128205128205 0.638071417808533
4.03589743589744 0.720981895923615
4.22820512820513 0.66344803571701
4.42820512820513 0.684986412525177
4.64102564102564 0.640621721744537
4.86153846153846 0.619284689426422
5.09230769230769 0.682232797145844
5.33589743589744 0.651261508464813
5.58974358974359 0.684238970279694
5.85641025641026 0.694397568702698
6.13589743589744 0.676140487194061
6.42564102564103 0.62019556760788
6.73333333333333 0.619551122188568
7.05384615384615 0.624957680702209
7.38974358974359 0.611274480819702
7.74102564102564 0.575191676616669
8.11025641025641 0.545213401317596
8.4974358974359 0.603205859661102
8.9 0.625392973423004
9.32564102564103 0.585570752620697
9.76923076923077 0.54549115896225
10.2333333333333 0.55239075422287
10.7205128205128 0.564889073371887
11.2307692307692 0.553384006023407
11.7666666666667 0.518928706645966
12.3282051282051 0.525848925113678
12.9153846153846 0.500984013080597
13.5282051282051 0.525131583213806
14.174358974359 0.513846755027771
14.8487179487179 0.498111635446548
15.5564102564103 0.505757808685303
16.2974358974359 0.475689560174942
17.0717948717949 0.507245123386383
17.8846153846154 0.484034270048141
18.7358974358974 0.465162485837936
19.6282051282051 0.453984081745148
20.5641025641026 0.466948807239532
21.5435897435897 0.410718977451324
22.5692307692308 0.406611055135727
23.6435897435897 0.415300279855728
24.7692307692308 0.370809257030487
25.9487179487179 0.377887547016144
27.1846153846154 0.339982628822327
28.4794871794872 0.383543372154236
29.8358974358974 0.422264784574509
31.2564102564103 0.373331636190414
32.7435897435897 0.338352113962173
34.3025641025641 0.295729666948318
35.9358974358974 0.256378382444382
37.648717948718 0.332439810037613
39.4410256410256 0.243611976504326
41.3179487179487 0.339317589998245
43.2871794871795 0.277715653181076
45.3487179487179 0.271051943302155
47.5076923076923 0.201074600219727
49.7692307692308 0.255845874547958
52.1384615384615 0.259632378816605
54.6205128205128 0.252970188856125
57.2230769230769 0.266061782836914
59.9461538461538 0.195412069559097
62.8025641025641 0.24686972796917
65.7923076923077 0.278451055288315
68.925641025641 0.26027050614357
72.2076923076923 0.223380714654922
75.6461538461539 0.251470774412155
79.2461538461538 0.242872908711433
83.0205128205128 0.262097537517548
86.974358974359 0.259025096893311
91.1153846153846 0.256770193576813
95.4538461538462 0.265696734189987
100 0.241440296173096
};
\addplot [, color0, opacity=0.6, mark=diamond*, mark size=0.5, mark options={solid}, only marks, forget plot]
table {%
1 0.949813485145569
1.04615384615385 0.970609664916992
1.0974358974359 0.953747570514679
1.14871794871795 0.870120048522949
1.2025641025641 0.836910247802734
1.26153846153846 0.833300054073334
1.32051282051282 0.850605964660645
1.38461538461538 0.838150024414062
1.44871794871795 0.772736549377441
1.51794871794872 0.829325318336487
1.58974358974359 0.832783818244934
1.66666666666667 0.8333340883255
1.74615384615385 0.799189209938049
1.82820512820513 0.810373485088348
1.91538461538462 0.778378009796143
2.00769230769231 0.79144412279129
2.1025641025641 0.795003056526184
2.20512820512821 0.834752023220062
2.30769230769231 0.829061925411224
2.41794871794872 0.766360938549042
2.53333333333333 0.721598565578461
2.65384615384615 0.814976990222931
2.78205128205128 0.682437896728516
2.91282051282051 0.801694214344025
3.05384615384615 0.726515233516693
3.1974358974359 0.813776791095734
3.35128205128205 0.743092358112335
3.51025641025641 0.743223130702972
3.67692307692308 0.755147576332092
3.85128205128205 0.764708638191223
4.03589743589744 0.728664875030518
4.22820512820513 0.756740629673004
4.42820512820513 0.754509270191193
4.64102564102564 0.828991413116455
4.86153846153846 0.762236297130585
5.09230769230769 0.724137485027313
5.33589743589744 0.727580964565277
5.58974358974359 0.748494446277618
5.85641025641026 0.754816472530365
6.13589743589744 0.756534397602081
6.42564102564103 0.714700639247894
6.73333333333333 0.70684015750885
7.05384615384615 0.667402923107147
7.38974358974359 0.702855885028839
7.74102564102564 0.708306849002838
8.11025641025641 0.702651143074036
8.4974358974359 0.682196795940399
8.9 0.683450222015381
9.32564102564103 0.668584048748016
9.76923076923077 0.689306199550629
10.2333333333333 0.712567687034607
10.7205128205128 0.606508910655975
11.2307692307692 0.591009259223938
11.7666666666667 0.587374866008759
12.3282051282051 0.570475399494171
12.9153846153846 0.61616313457489
13.5282051282051 0.576504826545715
14.174358974359 0.581608057022095
14.8487179487179 0.592369556427002
15.5564102564103 0.598106324672699
16.2974358974359 0.559328615665436
17.0717948717949 0.607000827789307
17.8846153846154 0.547135710716248
18.7358974358974 0.478776663541794
19.6282051282051 0.559852421283722
20.5641025641026 0.524091720581055
21.5435897435897 0.534081280231476
22.5692307692308 0.465452522039413
23.6435897435897 0.522611558437347
24.7692307692308 0.518734276294708
25.9487179487179 0.576691389083862
27.1846153846154 0.475851207971573
28.4794871794872 0.560927093029022
29.8358974358974 0.410897701978683
31.2564102564103 0.487320959568024
32.7435897435897 0.470346182584763
34.3025641025641 0.441788047552109
35.9358974358974 0.50613272190094
37.648717948718 0.449213415384293
39.4410256410256 0.421553134918213
41.3179487179487 0.393825381994247
43.2871794871795 0.426820367574692
45.3487179487179 0.417636007070541
47.5076923076923 0.440937727689743
49.7692307692308 0.451377779245377
52.1384615384615 0.426462382078171
54.6205128205128 0.45313748717308
57.2230769230769 0.406198412179947
59.9461538461538 0.413283079862595
62.8025641025641 0.445576101541519
65.7923076923077 0.438165187835693
68.925641025641 0.351419508457184
72.2076923076923 0.354734480381012
75.6461538461539 0.421379238367081
79.2461538461538 0.370885759592056
83.0205128205128 0.353044807910919
86.974358974359 0.340442389249802
91.1153846153846 0.331017464399338
95.4538461538462 0.354842573404312
100 0.32182440161705
};
\addplot [, color1, opacity=0.6, mark=square*, mark size=0.5, mark options={solid}, only marks]
table {%
1 0.94192373752594
1.04615384615385 0.976818740367889
1.0974358974359 0.953231453895569
1.14871794871795 0.938212990760803
1.2025641025641 0.891559243202209
1.26153846153846 0.91098940372467
1.32051282051282 0.87635463476181
1.38461538461538 0.90366405248642
1.44871794871795 0.881346046924591
1.51794871794872 0.841041266918182
1.58974358974359 0.849465012550354
1.66666666666667 0.750778794288635
1.74615384615385 0.777495384216309
1.82820512820513 0.694984138011932
1.91538461538462 0.705000877380371
2.00769230769231 0.768650352954865
2.1025641025641 0.728638112545013
2.20512820512821 0.686745822429657
2.30769230769231 0.674915194511414
2.41794871794872 0.717428505420685
2.53333333333333 0.755853652954102
2.65384615384615 0.653877258300781
2.78205128205128 0.682962119579315
2.91282051282051 0.714322149753571
3.05384615384615 0.748128831386566
3.1974358974359 0.713957965373993
3.35128205128205 0.694500386714935
3.51025641025641 0.671588838100433
3.67692307692308 0.688564002513885
3.85128205128205 0.659489750862122
4.03589743589744 0.71548455953598
4.22820512820513 0.747915089130402
4.42820512820513 0.70844841003418
4.64102564102564 0.748433530330658
4.86153846153846 0.611654579639435
5.09230769230769 0.702064514160156
5.33589743589744 0.701511263847351
5.58974358974359 0.779936611652374
5.85641025641026 0.723538637161255
6.13589743589744 0.672076523303986
6.42564102564103 0.705211222171783
6.73333333333333 0.647769868373871
7.05384615384615 0.650123298168182
7.38974358974359 0.762286305427551
7.74102564102564 0.70545357465744
8.11025641025641 0.688543796539307
8.4974358974359 0.622262597084045
8.9 0.671277642250061
9.32564102564103 0.57720273733139
9.76923076923077 0.67518150806427
10.2333333333333 0.621470868587494
10.7205128205128 0.66202700138092
11.2307692307692 0.622226178646088
11.7666666666667 0.622698783874512
12.3282051282051 0.642379760742188
12.9153846153846 0.690475344657898
13.5282051282051 0.720897018909454
14.174358974359 0.697210490703583
14.8487179487179 0.697289228439331
15.5564102564103 0.661920726299286
16.2974358974359 0.699290692806244
17.0717948717949 0.698863089084625
17.8846153846154 0.760176777839661
18.7358974358974 0.741424977779388
19.6282051282051 0.62260776758194
20.5641025641026 0.648022353649139
21.5435897435897 0.678304493427277
22.5692307692308 0.631327331066132
23.6435897435897 0.720421195030212
24.7692307692308 0.570111751556396
25.9487179487179 0.776286542415619
27.1846153846154 0.713627874851227
28.4794871794872 0.793471932411194
29.8358974358974 0.740260541439056
31.2564102564103 0.862897872924805
32.7435897435897 0.811603546142578
34.3025641025641 0.775090157985687
35.9358974358974 0.842748641967773
37.648717948718 0.766509473323822
39.4410256410256 0.80524080991745
41.3179487179487 0.873386025428772
43.2871794871795 0.886651039123535
45.3487179487179 0.894459187984467
47.5076923076923 0.923799812793732
49.7692307692308 0.871179223060608
52.1384615384615 0.830314815044403
54.6205128205128 0.886920094490051
57.2230769230769 0.871956348419189
59.9461538461538 0.858484089374542
62.8025641025641 0.993415653705597
65.7923076923077 0.900068581104279
68.925641025641 0.994720160961151
72.2076923076923 0.968303680419922
75.6461538461539 0.952653110027313
79.2461538461538 0.995873153209686
83.0205128205128 0.983590722084045
86.974358974359 0.874656319618225
91.1153846153846 0.89493465423584
95.4538461538462 0.987270176410675
100 0.950343728065491
};
\addlegendentry{mb 128, mc 1}
\addplot [, color1, opacity=0.6, mark=square*, mark size=0.5, mark options={solid}, only marks, forget plot]
table {%
1 0.958248913288116
1.04615384615385 0.976048588752747
1.0974358974359 0.94904100894928
1.14871794871795 0.936571598052979
1.2025641025641 0.901867508888245
1.26153846153846 0.800782978534698
1.32051282051282 0.905483186244965
1.38461538461538 0.891826748847961
1.44871794871795 0.883632659912109
1.51794871794872 0.8232421875
1.58974358974359 0.86372321844101
1.66666666666667 0.795344471931458
1.74615384615385 0.85113525390625
1.82820512820513 0.795354664325714
1.91538461538462 0.803300678730011
2.00769230769231 0.838076770305634
2.1025641025641 0.755711138248444
2.20512820512821 0.854673504829407
2.30769230769231 0.809967041015625
2.41794871794872 0.76309996843338
2.53333333333333 0.78917533159256
2.65384615384615 0.77189826965332
2.78205128205128 0.756643772125244
2.91282051282051 0.782468914985657
3.05384615384615 0.80991268157959
3.1974358974359 0.768899261951447
3.35128205128205 0.804912686347961
3.51025641025641 0.799109160900116
3.67692307692308 0.840062916278839
3.85128205128205 0.755114078521729
4.03589743589744 0.782238185405731
4.22820512820513 0.743315517902374
4.42820512820513 0.845744132995605
4.64102564102564 0.719880521297455
4.86153846153846 0.73018479347229
5.09230769230769 0.748167157173157
5.33589743589744 0.752479195594788
5.58974358974359 0.747208297252655
5.85641025641026 0.736875653266907
6.13589743589744 0.722541034221649
6.42564102564103 0.766248464584351
6.73333333333333 0.723269581794739
7.05384615384615 0.761767566204071
7.38974358974359 0.637299358844757
7.74102564102564 0.705607652664185
8.11025641025641 0.757139146327972
8.4974358974359 0.725491523742676
8.9 0.724743247032166
9.32564102564103 0.697837769985199
9.76923076923077 0.680380702018738
10.2333333333333 0.701986014842987
10.7205128205128 0.715434551239014
11.2307692307692 0.710214972496033
11.7666666666667 0.688733339309692
12.3282051282051 0.738079905509949
12.9153846153846 0.698518514633179
13.5282051282051 0.695686757564545
14.174358974359 0.71127450466156
14.8487179487179 0.675187230110168
15.5564102564103 0.666110455989838
16.2974358974359 0.760617196559906
17.0717948717949 0.619593799114227
17.8846153846154 0.682009935379028
18.7358974358974 0.697333633899689
19.6282051282051 0.705341994762421
20.5641025641026 0.678953289985657
21.5435897435897 0.73777163028717
22.5692307692308 0.740595936775208
23.6435897435897 0.652162492275238
24.7692307692308 0.732277393341064
25.9487179487179 0.629048526287079
27.1846153846154 0.654356896877289
28.4794871794872 0.719364762306213
29.8358974358974 0.698231816291809
31.2564102564103 0.738152742385864
32.7435897435897 0.685322821140289
34.3025641025641 0.690349578857422
35.9358974358974 0.754822373390198
37.648717948718 0.755016505718231
39.4410256410256 0.694171130657196
41.3179487179487 0.832574069499969
43.2871794871795 0.832428395748138
45.3487179487179 0.770084500312805
47.5076923076923 0.777730882167816
49.7692307692308 0.809498012065887
52.1384615384615 0.865198731422424
54.6205128205128 0.896412551403046
57.2230769230769 0.937066853046417
59.9461538461538 0.897550046443939
62.8025641025641 0.916484475135803
65.7923076923077 0.862279534339905
68.925641025641 0.92023366689682
72.2076923076923 0.899656116962433
75.6461538461539 0.923564076423645
79.2461538461538 0.976962208747864
83.0205128205128 0.916792511940002
86.974358974359 0.99090176820755
91.1153846153846 0.989422142505646
95.4538461538462 0.912403047084808
100 0.988819897174835
};
\addplot [, color1, opacity=0.6, mark=square*, mark size=0.5, mark options={solid}, only marks, forget plot]
table {%
1 0.945786654949188
1.04615384615385 0.974947154521942
1.0974358974359 0.945127665996552
1.14871794871795 0.885274887084961
1.2025641025641 0.853883385658264
1.26153846153846 0.870778739452362
1.32051282051282 0.823977112770081
1.38461538461538 0.865713775157928
1.44871794871795 0.864140689373016
1.51794871794872 0.770281255245209
1.58974358974359 0.761290490627289
1.66666666666667 0.759315609931946
1.74615384615385 0.772099435329437
1.82820512820513 0.767327308654785
1.91538461538462 0.782072067260742
2.00769230769231 0.781100928783417
2.1025641025641 0.721639811992645
2.20512820512821 0.711820781230927
2.30769230769231 0.693787217140198
2.41794871794872 0.709099769592285
2.53333333333333 0.765683770179749
2.65384615384615 0.750941812992096
2.78205128205128 0.784226596355438
2.91282051282051 0.714505016803741
3.05384615384615 0.689058125019073
3.1974358974359 0.732104897499084
3.35128205128205 0.702791690826416
3.51025641025641 0.695631444454193
3.67692307692308 0.728388965129852
3.85128205128205 0.659397780895233
4.03589743589744 0.666098654270172
4.22820512820513 0.68618768453598
4.42820512820513 0.723300099372864
4.64102564102564 0.633171021938324
4.86153846153846 0.760955274105072
5.09230769230769 0.701535165309906
5.33589743589744 0.721564054489136
5.58974358974359 0.699035108089447
5.85641025641026 0.65361213684082
6.13589743589744 0.733973443508148
6.42564102564103 0.73095577955246
6.73333333333333 0.699957191944122
7.05384615384615 0.752398371696472
7.38974358974359 0.695437073707581
7.74102564102564 0.709762871265411
8.11025641025641 0.739453494548798
8.4974358974359 0.714772284030914
8.9 0.757281720638275
9.32564102564103 0.684405148029327
9.76923076923077 0.727223575115204
10.2333333333333 0.688075184822083
10.7205128205128 0.795240819454193
11.2307692307692 0.697132229804993
11.7666666666667 0.741702735424042
12.3282051282051 0.742010414600372
12.9153846153846 0.696933686733246
13.5282051282051 0.711653351783752
14.174358974359 0.667291104793549
14.8487179487179 0.677173793315887
15.5564102564103 0.704285323619843
16.2974358974359 0.72856330871582
17.0717948717949 0.68200409412384
17.8846153846154 0.715749204158783
18.7358974358974 0.680002152919769
19.6282051282051 0.678201019763947
20.5641025641026 0.729542195796967
21.5435897435897 0.686078250408173
22.5692307692308 0.712745487689972
23.6435897435897 0.641085028648376
24.7692307692308 0.645069599151611
25.9487179487179 0.676373958587646
27.1846153846154 0.63044947385788
28.4794871794872 0.791323184967041
29.8358974358974 0.749311208724976
31.2564102564103 0.787514865398407
32.7435897435897 0.747543215751648
34.3025641025641 0.701508402824402
35.9358974358974 0.701525747776031
37.648717948718 0.788980960845947
39.4410256410256 0.848997533321381
41.3179487179487 0.858397603034973
43.2871794871795 0.829757511615753
45.3487179487179 0.816436290740967
47.5076923076923 0.777370870113373
49.7692307692308 0.818832039833069
52.1384615384615 0.816319108009338
54.6205128205128 0.921358287334442
57.2230769230769 0.973700821399689
59.9461538461538 0.892673671245575
62.8025641025641 0.874566078186035
65.7923076923077 0.887063980102539
68.925641025641 0.879256248474121
72.2076923076923 0.921890437602997
75.6461538461539 0.957567870616913
79.2461538461538 0.983278870582581
83.0205128205128 0.814664661884308
86.974358974359 0.885950207710266
91.1153846153846 0.892029106616974
95.4538461538462 0.86728972196579
100 0.882810056209564
};
\addplot [, color1, opacity=0.6, mark=square*, mark size=0.5, mark options={solid}, only marks, forget plot]
table {%
1 0.931701302528381
1.04615384615385 0.973757266998291
1.0974358974359 0.945250153541565
1.14871794871795 0.89120352268219
1.2025641025641 0.832647323608398
1.26153846153846 0.831245541572571
1.32051282051282 0.859472692012787
1.38461538461538 0.885139584541321
1.44871794871795 0.780946373939514
1.51794871794872 0.850793182849884
1.58974358974359 0.843400180339813
1.66666666666667 0.86012190580368
1.74615384615385 0.840017318725586
1.82820512820513 0.801431775093079
1.91538461538462 0.820572018623352
2.00769230769231 0.798683762550354
2.1025641025641 0.797450959682465
2.20512820512821 0.810002624988556
2.30769230769231 0.791927754878998
2.41794871794872 0.789192795753479
2.53333333333333 0.859344899654388
2.65384615384615 0.798830211162567
2.78205128205128 0.796762764453888
2.91282051282051 0.765724360942841
3.05384615384615 0.8428675532341
3.1974358974359 0.745102107524872
3.35128205128205 0.836925804615021
3.51025641025641 0.781729161739349
3.67692307692308 0.825998902320862
3.85128205128205 0.842827796936035
4.03589743589744 0.83186262845993
4.22820512820513 0.816758334636688
4.42820512820513 0.848060786724091
4.64102564102564 0.836192429065704
4.86153846153846 0.844686806201935
5.09230769230769 0.788341462612152
5.33589743589744 0.834633529186249
5.58974358974359 0.823691308498383
5.85641025641026 0.748726963996887
6.13589743589744 0.739810943603516
6.42564102564103 0.745484292507172
6.73333333333333 0.771238744258881
7.05384615384615 0.707592487335205
7.38974358974359 0.744190692901611
7.74102564102564 0.709074318408966
8.11025641025641 0.725247025489807
8.4974358974359 0.739657998085022
8.9 0.716888964176178
9.32564102564103 0.763029813766479
9.76923076923077 0.693202495574951
10.2333333333333 0.672634840011597
10.7205128205128 0.713196754455566
11.2307692307692 0.660144746303558
11.7666666666667 0.732316315174103
12.3282051282051 0.731077134609222
12.9153846153846 0.676992833614349
13.5282051282051 0.675350904464722
14.174358974359 0.634487330913544
14.8487179487179 0.663292825222015
15.5564102564103 0.691311180591583
16.2974358974359 0.675833880901337
17.0717948717949 0.667942702770233
17.8846153846154 0.626939356327057
18.7358974358974 0.803158700466156
19.6282051282051 0.708568811416626
20.5641025641026 0.746818721294403
21.5435897435897 0.770217597484589
22.5692307692308 0.767114281654358
23.6435897435897 0.704236447811127
24.7692307692308 0.755072295665741
25.9487179487179 0.612408339977264
27.1846153846154 0.725867867469788
28.4794871794872 0.690034985542297
29.8358974358974 0.817761123180389
31.2564102564103 0.718297600746155
32.7435897435897 0.652769684791565
34.3025641025641 0.63062995672226
35.9358974358974 0.801308155059814
37.648717948718 0.797775506973267
39.4410256410256 0.690148770809174
41.3179487179487 0.852025926113129
43.2871794871795 0.864590287208557
45.3487179487179 0.87977522611618
47.5076923076923 0.839311599731445
49.7692307692308 0.847140491008759
52.1384615384615 0.874131381511688
54.6205128205128 0.844642281532288
57.2230769230769 0.941859424114227
59.9461538461538 0.884585678577423
62.8025641025641 0.903121411800385
65.7923076923077 0.891826272010803
68.925641025641 0.975147426128387
72.2076923076923 0.832457542419434
75.6461538461539 0.893660664558411
79.2461538461538 0.951414704322815
83.0205128205128 0.977946281433105
86.974358974359 0.94107860326767
91.1153846153846 0.976512908935547
95.4538461538462 0.951927781105042
100 0.966737449169159
};
\addplot [, color1, opacity=0.6, mark=square*, mark size=0.5, mark options={solid}, only marks, forget plot]
table {%
1 0.950482070446014
1.04615384615385 0.961005687713623
1.0974358974359 0.946151196956635
1.14871794871795 0.852972030639648
1.2025641025641 0.844632744789124
1.26153846153846 0.863714993000031
1.32051282051282 0.861481308937073
1.38461538461538 0.814154446125031
1.44871794871795 0.817956924438477
1.51794871794872 0.883370995521545
1.58974358974359 0.847991287708282
1.66666666666667 0.860145032405853
1.74615384615385 0.847654640674591
1.82820512820513 0.883530139923096
1.91538461538462 0.765230000019073
2.00769230769231 0.816060543060303
2.1025641025641 0.857405483722687
2.20512820512821 0.821012496948242
2.30769230769231 0.755450248718262
2.41794871794872 0.809442639350891
2.53333333333333 0.869926929473877
2.65384615384615 0.753012955188751
2.78205128205128 0.815297722816467
2.91282051282051 0.826720416545868
3.05384615384615 0.866314589977264
3.1974358974359 0.724790751934052
3.35128205128205 0.807469964027405
3.51025641025641 0.798451900482178
3.67692307692308 0.803541600704193
3.85128205128205 0.774239242076874
4.03589743589744 0.807102143764496
4.22820512820513 0.771797597408295
4.42820512820513 0.722094535827637
4.64102564102564 0.719067394733429
4.86153846153846 0.741371095180511
5.09230769230769 0.791683971881866
5.33589743589744 0.749395132064819
5.58974358974359 0.771623432636261
5.85641025641026 0.72326797246933
6.13589743589744 0.707643151283264
6.42564102564103 0.770362079143524
6.73333333333333 0.747718334197998
7.05384615384615 0.781316757202148
7.38974358974359 0.7628014087677
7.74102564102564 0.774133861064911
8.11025641025641 0.701797485351562
8.4974358974359 0.82624876499176
8.9 0.702834188938141
9.32564102564103 0.673160552978516
9.76923076923077 0.738293349742889
10.2333333333333 0.718926727771759
10.7205128205128 0.735359668731689
11.2307692307692 0.715785205364227
11.7666666666667 0.65590912103653
12.3282051282051 0.747045695781708
12.9153846153846 0.783111929893494
13.5282051282051 0.755444645881653
14.174358974359 0.667137742042542
14.8487179487179 0.740897536277771
15.5564102564103 0.704724013805389
16.2974358974359 0.67224133014679
17.0717948717949 0.692768394947052
17.8846153846154 0.718317151069641
18.7358974358974 0.75210577249527
19.6282051282051 0.685774028301239
20.5641025641026 0.560592293739319
21.5435897435897 0.698275208473206
22.5692307692308 0.620763778686523
23.6435897435897 0.656507313251495
24.7692307692308 0.584664165973663
25.9487179487179 0.720365226268768
27.1846153846154 0.648519337177277
28.4794871794872 0.733914017677307
29.8358974358974 0.593075394630432
31.2564102564103 0.65371960401535
32.7435897435897 0.702346622943878
34.3025641025641 0.684443652629852
35.9358974358974 0.696815848350525
37.648717948718 0.776048839092255
39.4410256410256 0.844969093799591
41.3179487179487 0.743224263191223
43.2871794871795 0.909925937652588
45.3487179487179 0.729101598262787
47.5076923076923 0.887892663478851
49.7692307692308 0.95882910490036
52.1384615384615 0.864857196807861
54.6205128205128 0.878896176815033
57.2230769230769 0.851144134998322
59.9461538461538 0.954115808010101
62.8025641025641 0.9702068567276
65.7923076923077 0.833140969276428
68.925641025641 0.954611718654633
72.2076923076923 0.890580475330353
75.6461538461539 0.991189420223236
79.2461538461538 0.997001826763153
83.0205128205128 0.945874035358429
86.974358974359 0.924999415874481
91.1153846153846 0.928979873657227
95.4538461538462 0.907484471797943
100 0.976702153682709
};
\addplot [, color2, opacity=0.6, mark=triangle*, mark size=0.5, mark options={solid,rotate=180}, only marks]
table {%
1 0.679985761642456
1.04615384615385 0.7115678191185
1.0974358974359 0.657540440559387
1.14871794871795 0.620733618736267
1.2025641025641 0.614829540252686
1.26153846153846 0.644012570381165
1.32051282051282 0.64054799079895
1.38461538461538 0.63720428943634
1.44871794871795 0.598570466041565
1.51794871794872 0.634649395942688
1.58974358974359 0.591947853565216
1.66666666666667 0.623505294322968
1.74615384615385 0.555689036846161
1.82820512820513 0.574970126152039
1.91538461538462 0.562912106513977
2.00769230769231 0.535982728004456
2.1025641025641 0.54005640745163
2.20512820512821 0.55210554599762
2.30769230769231 0.505798995494843
2.41794871794872 0.554286956787109
2.53333333333333 0.539993703365326
2.65384615384615 0.563437223434448
2.78205128205128 0.543099641799927
2.91282051282051 0.548007905483246
3.05384615384615 0.532557308673859
3.1974358974359 0.514586925506592
3.35128205128205 0.550830006599426
3.51025641025641 0.544722378253937
3.67692307692308 0.536729514598846
3.85128205128205 0.52365255355835
4.03589743589744 0.518151879310608
4.22820512820513 0.514798939228058
4.42820512820513 0.493630707263947
4.64102564102564 0.505635440349579
4.86153846153846 0.588132500648499
5.09230769230769 0.594875931739807
5.33589743589744 0.5506511926651
5.58974358974359 0.526655614376068
5.85641025641026 0.509879767894745
6.13589743589744 0.51197224855423
6.42564102564103 0.50181645154953
6.73333333333333 0.525832951068878
7.05384615384615 0.454970359802246
7.38974358974359 0.527557551860809
7.74102564102564 0.504554808139801
8.11025641025641 0.527572453022003
8.4974358974359 0.442246437072754
8.9 0.497840851545334
9.32564102564103 0.490144073963165
9.76923076923077 0.500274121761322
10.2333333333333 0.495826065540314
10.7205128205128 0.474566221237183
11.2307692307692 0.47711381316185
11.7666666666667 0.460855007171631
12.3282051282051 0.499438494443893
12.9153846153846 0.487183183431625
13.5282051282051 0.460460394620895
14.174358974359 0.461189836263657
14.8487179487179 0.43670254945755
15.5564102564103 0.468182533979416
16.2974358974359 0.421885341405869
17.0717948717949 0.449288457632065
17.8846153846154 0.427962154150009
18.7358974358974 0.449324607849121
19.6282051282051 0.462636709213257
20.5641025641026 0.42394495010376
21.5435897435897 0.467438071966171
22.5692307692308 0.434462547302246
23.6435897435897 0.385987311601639
24.7692307692308 0.424402236938477
25.9487179487179 0.407541990280151
27.1846153846154 0.374782800674438
28.4794871794872 0.418734759092331
29.8358974358974 0.394929707050323
31.2564102564103 0.309350252151489
32.7435897435897 0.334923088550568
34.3025641025641 0.403249830007553
35.9358974358974 0.368883222341537
37.648717948718 0.335628509521484
39.4410256410256 0.347359269857407
41.3179487179487 0.428533166646957
43.2871794871795 0.38937446475029
45.3487179487179 0.269297450780869
47.5076923076923 0.393448621034622
49.7692307692308 0.330950498580933
52.1384615384615 0.305447787046432
54.6205128205128 0.330398201942444
57.2230769230769 0.377449691295624
59.9461538461538 0.347426205873489
62.8025641025641 0.188897535204887
65.7923076923077 0.300286054611206
68.925641025641 0.325470626354218
72.2076923076923 0.31389507651329
75.6461538461539 0.325666755437851
79.2461538461538 0.306229919195175
83.0205128205128 0.236430674791336
86.974358974359 0.343841552734375
91.1153846153846 0.362791031599045
95.4538461538462 0.31374579668045
100 0.254591286182404
};
\addlegendentry{sub 16, mc 1}
\addplot [, color2, opacity=0.6, mark=triangle*, mark size=0.5, mark options={solid,rotate=180}, only marks, forget plot]
table {%
1 0.609531939029694
1.04615384615385 0.749182820320129
1.0974358974359 0.69086229801178
1.14871794871795 0.693500459194183
1.2025641025641 0.676925659179688
1.26153846153846 0.709791362285614
1.32051282051282 0.650371491909027
1.38461538461538 0.671522736549377
1.44871794871795 0.642296254634857
1.51794871794872 0.661283135414124
1.58974358974359 0.650317788124084
1.66666666666667 0.661212205886841
1.74615384615385 0.680784165859222
1.82820512820513 0.642096698284149
1.91538461538462 0.655320167541504
2.00769230769231 0.637710869312286
2.1025641025641 0.644869267940521
2.20512820512821 0.605227947235107
2.30769230769231 0.635520577430725
2.41794871794872 0.627608239650726
2.53333333333333 0.575490415096283
2.65384615384615 0.626273334026337
2.78205128205128 0.674213767051697
2.91282051282051 0.623071134090424
3.05384615384615 0.590344607830048
3.1974358974359 0.626595795154572
3.35128205128205 0.58951324224472
3.51025641025641 0.578020930290222
3.67692307692308 0.636671960353851
3.85128205128205 0.60143369436264
4.03589743589744 0.580205917358398
4.22820512820513 0.597512423992157
4.42820512820513 0.60347455739975
4.64102564102564 0.56816029548645
4.86153846153846 0.583955764770508
5.09230769230769 0.599515438079834
5.33589743589744 0.586233139038086
5.58974358974359 0.552801311016083
5.85641025641026 0.530638694763184
6.13589743589744 0.527270972728729
6.42564102564103 0.548207342624664
6.73333333333333 0.558467030525208
7.05384615384615 0.559272110462189
7.38974358974359 0.553839683532715
7.74102564102564 0.542761385440826
8.11025641025641 0.552704453468323
8.4974358974359 0.513544738292694
8.9 0.542305886745453
9.32564102564103 0.502696573734283
9.76923076923077 0.522841334342957
10.2333333333333 0.524320840835571
10.7205128205128 0.527204632759094
11.2307692307692 0.501441061496735
11.7666666666667 0.508847892284393
12.3282051282051 0.513864934444427
12.9153846153846 0.470519065856934
13.5282051282051 0.509582221508026
14.174358974359 0.527908682823181
14.8487179487179 0.496969372034073
15.5564102564103 0.495376169681549
16.2974358974359 0.46031054854393
17.0717948717949 0.453687012195587
17.8846153846154 0.462118238210678
18.7358974358974 0.479263782501221
19.6282051282051 0.40896812081337
20.5641025641026 0.440728336572647
21.5435897435897 0.412588566541672
22.5692307692308 0.41799408197403
23.6435897435897 0.411017626523972
24.7692307692308 0.418606579303741
25.9487179487179 0.420555114746094
27.1846153846154 0.413618415594101
28.4794871794872 0.367841303348541
29.8358974358974 0.432470619678497
31.2564102564103 0.417124480009079
32.7435897435897 0.404908269643784
34.3025641025641 0.369007915258408
35.9358974358974 0.366209477186203
37.648717948718 0.40035754442215
39.4410256410256 0.383717179298401
41.3179487179487 0.391122102737427
43.2871794871795 0.452154606580734
45.3487179487179 0.49501371383667
47.5076923076923 0.391038715839386
49.7692307692308 0.349455684423447
52.1384615384615 0.438138782978058
54.6205128205128 0.385201930999756
57.2230769230769 0.423942565917969
59.9461538461538 0.404599905014038
62.8025641025641 0.434614390134811
65.7923076923077 0.403169840574265
68.925641025641 0.417980581521988
72.2076923076923 0.341756582260132
75.6461538461539 0.435684680938721
79.2461538461538 0.350570887327194
83.0205128205128 0.556321084499359
86.974358974359 0.373792976140976
91.1153846153846 0.398933619260788
95.4538461538462 0.373206824064255
100 0.398498058319092
};
\addplot [, color2, opacity=0.6, mark=triangle*, mark size=0.5, mark options={solid,rotate=180}, only marks, forget plot]
table {%
1 0.664133131504059
1.04615384615385 0.742021501064301
1.0974358974359 0.669355094432831
1.14871794871795 0.624003350734711
1.2025641025641 0.604327142238617
1.26153846153846 0.705462455749512
1.32051282051282 0.602650225162506
1.38461538461538 0.651018619537354
1.44871794871795 0.581097304821014
1.51794871794872 0.568470001220703
1.58974358974359 0.639023900032043
1.66666666666667 0.551868975162506
1.74615384615385 0.501139342784882
1.82820512820513 0.60296505689621
1.91538461538462 0.551508843898773
2.00769230769231 0.569350838661194
2.1025641025641 0.58341509103775
2.20512820512821 0.595696926116943
2.30769230769231 0.542159795761108
2.41794871794872 0.591327488422394
2.53333333333333 0.568028926849365
2.65384615384615 0.614602565765381
2.78205128205128 0.605683922767639
2.91282051282051 0.595213055610657
3.05384615384615 0.603728175163269
3.1974358974359 0.596028625965118
3.35128205128205 0.596465647220612
3.51025641025641 0.55497145652771
3.67692307692308 0.628120720386505
3.85128205128205 0.552897572517395
4.03589743589744 0.583718597888947
4.22820512820513 0.593847572803497
4.42820512820513 0.611643612384796
4.64102564102564 0.594580352306366
4.86153846153846 0.605619966983795
5.09230769230769 0.589016377925873
5.33589743589744 0.60254031419754
5.58974358974359 0.580131709575653
5.85641025641026 0.570542097091675
6.13589743589744 0.571931540966034
6.42564102564103 0.607606887817383
6.73333333333333 0.586728513240814
7.05384615384615 0.551428020000458
7.38974358974359 0.589957058429718
7.74102564102564 0.581372439861298
8.11025641025641 0.552854537963867
8.4974358974359 0.548898279666901
8.9 0.572866201400757
9.32564102564103 0.556757092475891
9.76923076923077 0.534628868103027
10.2333333333333 0.562076032161713
10.7205128205128 0.544695019721985
11.2307692307692 0.53574275970459
11.7666666666667 0.543239772319794
12.3282051282051 0.548626899719238
12.9153846153846 0.504410266876221
13.5282051282051 0.527966201305389
14.174358974359 0.490926712751389
14.8487179487179 0.52117794752121
15.5564102564103 0.483173340559006
16.2974358974359 0.485036194324493
17.0717948717949 0.501787126064301
17.8846153846154 0.490939527750015
18.7358974358974 0.543093025684357
19.6282051282051 0.49184250831604
20.5641025641026 0.487862318754196
21.5435897435897 0.467890471220016
22.5692307692308 0.471752554178238
23.6435897435897 0.438517570495605
24.7692307692308 0.443475067615509
25.9487179487179 0.405333340167999
27.1846153846154 0.491725414991379
28.4794871794872 0.461670875549316
29.8358974358974 0.451597660779953
31.2564102564103 0.427940189838409
32.7435897435897 0.398850947618484
34.3025641025641 0.384532153606415
35.9358974358974 0.446811884641647
37.648717948718 0.499656111001968
39.4410256410256 0.495842754840851
41.3179487179487 0.421808063983917
43.2871794871795 0.421496719121933
45.3487179487179 0.393514692783356
47.5076923076923 0.406216204166412
49.7692307692308 0.480474680662155
52.1384615384615 0.422617256641388
54.6205128205128 0.471716701984406
57.2230769230769 0.477424442768097
59.9461538461538 0.35184833407402
62.8025641025641 0.43095788359642
65.7923076923077 0.388055741786957
68.925641025641 0.420408576726913
72.2076923076923 0.335992366075516
75.6461538461539 0.357446759939194
79.2461538461538 0.352704852819443
83.0205128205128 0.380814522504807
86.974358974359 0.456069946289062
91.1153846153846 0.400624424219131
95.4538461538462 0.377329915761948
100 0.501639008522034
};
\addplot [, color2, opacity=0.6, mark=triangle*, mark size=0.5, mark options={solid,rotate=180}, only marks, forget plot]
table {%
1 0.661202728748322
1.04615384615385 0.696987569332123
1.0974358974359 0.640337646007538
1.14871794871795 0.699087560176849
1.2025641025641 0.655234277248383
1.26153846153846 0.616808891296387
1.32051282051282 0.672836422920227
1.38461538461538 0.65803998708725
1.44871794871795 0.602547109127045
1.51794871794872 0.571843147277832
1.58974358974359 0.567402303218842
1.66666666666667 0.560483276844025
1.74615384615385 0.563518166542053
1.82820512820513 0.495382130146027
1.91538461538462 0.556142508983612
2.00769230769231 0.535636723041534
2.1025641025641 0.566289365291595
2.20512820512821 0.533777952194214
2.30769230769231 0.557669937610626
2.41794871794872 0.57751601934433
2.53333333333333 0.574429929256439
2.65384615384615 0.531158149242401
2.78205128205128 0.541740238666534
2.91282051282051 0.520841300487518
3.05384615384615 0.557223618030548
3.1974358974359 0.497204303741455
3.35128205128205 0.560685098171234
3.51025641025641 0.554442405700684
3.67692307692308 0.557515621185303
3.85128205128205 0.510431885719299
4.03589743589744 0.551983118057251
4.22820512820513 0.51150780916214
4.42820512820513 0.556466221809387
4.64102564102564 0.48903414607048
4.86153846153846 0.523623406887054
5.09230769230769 0.541903436183929
5.33589743589744 0.519254863262177
5.58974358974359 0.510743081569672
5.85641025641026 0.52313107252121
6.13589743589744 0.51476001739502
6.42564102564103 0.504145622253418
6.73333333333333 0.510161995887756
7.05384615384615 0.480819225311279
7.38974358974359 0.509220361709595
7.74102564102564 0.482606947422028
8.11025641025641 0.494603723287582
8.4974358974359 0.449146419763565
8.9 0.476795494556427
9.32564102564103 0.464368343353271
9.76923076923077 0.42785182595253
10.2333333333333 0.43813881278038
10.7205128205128 0.456308752298355
11.2307692307692 0.457170963287354
11.7666666666667 0.429328739643097
12.3282051282051 0.43947896361351
12.9153846153846 0.415423214435577
13.5282051282051 0.416642516851425
14.174358974359 0.434665769338608
14.8487179487179 0.438004493713379
15.5564102564103 0.409752756357193
16.2974358974359 0.400711536407471
17.0717948717949 0.420467585325241
17.8846153846154 0.392722338438034
18.7358974358974 0.424811452627182
19.6282051282051 0.387176722288132
20.5641025641026 0.408434599637985
21.5435897435897 0.365570694208145
22.5692307692308 0.362525910139084
23.6435897435897 0.387166857719421
24.7692307692308 0.33959624171257
25.9487179487179 0.349498063325882
27.1846153846154 0.322229951620102
28.4794871794872 0.313500821590424
29.8358974358974 0.400332510471344
31.2564102564103 0.385442167520523
32.7435897435897 0.335928589105606
34.3025641025641 0.277344703674316
35.9358974358974 0.26038932800293
37.648717948718 0.346099883317947
39.4410256410256 0.237144708633423
41.3179487179487 0.374276608228683
43.2871794871795 0.304925054311752
45.3487179487179 0.285973221063614
47.5076923076923 0.176246270537376
49.7692307692308 0.314584732055664
52.1384615384615 0.26629912853241
54.6205128205128 0.217731118202209
57.2230769230769 0.289781898260117
59.9461538461538 0.216622114181519
62.8025641025641 0.271614640951157
65.7923076923077 0.272000283002853
68.925641025641 0.299354135990143
72.2076923076923 0.265561699867249
75.6461538461539 0.229075580835342
79.2461538461538 0.215735912322998
83.0205128205128 0.311091154813766
86.974358974359 0.219729468226433
91.1153846153846 0.247721150517464
95.4538461538462 0.311950117349625
100 0.731382131576538
};
\addplot [, color2, opacity=0.6, mark=triangle*, mark size=0.5, mark options={solid,rotate=180}, only marks, forget plot]
table {%
1 0.695692479610443
1.04615384615385 0.767917394638062
1.0974358974359 0.719641447067261
1.14871794871795 0.654242515563965
1.2025641025641 0.648763000965118
1.26153846153846 0.6117884516716
1.32051282051282 0.58800745010376
1.38461538461538 0.597212076187134
1.44871794871795 0.58151763677597
1.51794871794872 0.5639688372612
1.58974358974359 0.562051892280579
1.66666666666667 0.578132152557373
1.74615384615385 0.577941596508026
1.82820512820513 0.561784446239471
1.91538461538462 0.490266710519791
2.00769230769231 0.515415370464325
2.1025641025641 0.508491039276123
2.20512820512821 0.649473488330841
2.30769230769231 0.58608478307724
2.41794871794872 0.651898205280304
2.53333333333333 0.538510203361511
2.65384615384615 0.573584675788879
2.78205128205128 0.535235583782196
2.91282051282051 0.599426209926605
3.05384615384615 0.551235973834991
3.1974358974359 0.601561307907104
3.35128205128205 0.584451913833618
3.51025641025641 0.576986968517303
3.67692307692308 0.620223343372345
3.85128205128205 0.603001713752747
4.03589743589744 0.559737622737885
4.22820512820513 0.610504567623138
4.42820512820513 0.61690753698349
4.64102564102564 0.576382100582123
4.86153846153846 0.585867345333099
5.09230769230769 0.59400486946106
5.33589743589744 0.623044192790985
5.58974358974359 0.636821746826172
5.85641025641026 0.601960599422455
6.13589743589744 0.583829581737518
6.42564102564103 0.585218608379364
6.73333333333333 0.551797032356262
7.05384615384615 0.536976635456085
7.38974358974359 0.557251155376434
7.74102564102564 0.606894135475159
8.11025641025641 0.581734299659729
8.4974358974359 0.600983142852783
8.9 0.566285908222198
9.32564102564103 0.586548507213593
9.76923076923077 0.600540339946747
10.2333333333333 0.591115057468414
10.7205128205128 0.560014843940735
11.2307692307692 0.548631608486176
11.7666666666667 0.553769648075104
12.3282051282051 0.549574315547943
12.9153846153846 0.571690618991852
13.5282051282051 0.526150465011597
14.174358974359 0.501663863658905
14.8487179487179 0.521821975708008
15.5564102564103 0.517585933208466
16.2974358974359 0.496373385190964
17.0717948717949 0.504104554653168
17.8846153846154 0.503425121307373
18.7358974358974 0.436725705862045
19.6282051282051 0.524384677410126
20.5641025641026 0.496122270822525
21.5435897435897 0.485489904880524
22.5692307692308 0.49681344628334
23.6435897435897 0.465708941221237
24.7692307692308 0.457208603620529
25.9487179487179 0.496414870023727
27.1846153846154 0.456822454929352
28.4794871794872 0.518645465373993
29.8358974358974 0.388004928827286
31.2564102564103 0.425083845853806
32.7435897435897 0.435062617063522
34.3025641025641 0.428002119064331
35.9358974358974 0.456764847040176
37.648717948718 0.416017442941666
39.4410256410256 0.381022781133652
41.3179487179487 0.39781990647316
43.2871794871795 0.419008105993271
45.3487179487179 0.375866740942001
47.5076923076923 0.402279376983643
49.7692307692308 0.486214727163315
52.1384615384615 0.393014490604401
54.6205128205128 0.490770727396011
57.2230769230769 0.395261615514755
59.9461538461538 0.386183947324753
62.8025641025641 0.470850676298141
65.7923076923077 0.437193632125854
68.925641025641 0.339673399925232
72.2076923076923 0.369417637586594
75.6461538461539 0.416514158248901
79.2461538461538 0.344016939401627
83.0205128205128 0.39773690700531
86.974358974359 0.325302511453629
91.1153846153846 0.387012124061584
95.4538461538462 0.358590811491013
100 0.299362510442734
};
\end{axis}

\end{tikzpicture}

    \tikzexternaldisable
  \end{minipage}
\end{subfigure}

\begin{subfigure}[t]{\linewidth}
  \centering
  \caption{\cifarten \threecthreed}
  \begin{minipage}{0.50\linewidth}
    \centering
    % defines the pgfplots style "eigspacedefault"
\pgfkeys{/pgfplots/eigspacedefault/.style={
    width=1.03\linewidth,
    height=\goldenRatioInv*1.03*\linewidth,
    every axis plot/.append style={line width = 1pt},
    tick pos = left,
    ylabel near ticks,
    xlabel near ticks,
    xtick align = inside,
    ytick align = inside,
    legend cell align = left,
    legend columns = 1,
    legend pos = north east,
    legend style = {
      fill opacity = 0.9,
      text opacity = 1,
      font = \tiny,
      % column sep=0.1cm,
    },
    legend image post style={scale=2},
    xticklabel style = {font = \small},
    xlabel style = {font = \small},
    axis line style = {black},
    yticklabel style = {font = \small},
    ylabel style = {font = \small},
    title style = {font = \small},
    grid = major,
    grid style = {dashed}
  }
}

\pgfkeys{/pgfplots/eigspacedefaultapp/.style={
    eigspacedefault,
    height=0.6\linewidth,
    legend columns = 2,
  }
}

\pgfkeys{/pgfplots/eigspacenolegend/.style={
    legend image post style = {scale=0},
    legend style = {
      fill opacity = 0,
      draw opacity = 0,
      text opacity = 0,
      font = \small,
      at={(1, 1.025)},
      anchor=south east,
      column sep=0.25cm,
    },
  }
}
%%% Local Variables:
%%% mode: latex
%%% TeX-master: "../main"
%%% End:

    \pgfkeys{/pgfplots/zmystyle/.style={
        eigspacedefaultapp,
        legend columns = 3,
        eigspacenolegend,
      }}
    \tikzexternalenable
    \vspace{-3ex}
    % This file was created by tikzplotlib v0.9.7.
\begin{tikzpicture}

\definecolor{color0}{rgb}{0.274509803921569,0.6,0.564705882352941}
\definecolor{color1}{rgb}{0.870588235294118,0.623529411764706,0.0862745098039216}
\definecolor{color2}{rgb}{0.501960784313725,0.184313725490196,0.6}

\begin{axis}[
axis line style={white!10!black},
legend style={fill opacity=0.8, draw opacity=1, text opacity=1, at={(0.03,0.03)}, anchor=south west, draw=white!80!black},
log basis x={10},
tick pos=left,
xlabel={epoch (log scale)},
xmajorgrids,
xmin=0.794328234724281, xmax=125.892541179417,
xmode=log,
ylabel={overlap},
ymajorgrids,
ymin=-0.05, ymax=1.05,
zmystyle
]
\addplot [, white!10!black, dashed, forget plot]
table {%
0.794328234724281 1
125.892541179417 1
};
\addplot [, white!10!black, dashed, forget plot]
table {%
0.794328234724281 0
125.892541179417 0
};
\addplot [, color0, opacity=0.6, mark=diamond*, mark size=0.5, mark options={solid}, only marks]
table {%
1 0.900430381298065
1.04487179487179 0.886731564998627
1.09615384615385 0.91850084066391
1.1474358974359 0.935516953468323
1.20192307692308 0.950066089630127
1.25961538461538 0.945077836513519
1.32051282051282 0.929778516292572
1.38461538461538 0.928590953350067
1.44871794871795 0.946700513362885
1.51923076923077 0.919388771057129
1.58974358974359 0.921060085296631
1.66666666666667 0.873826444149017
1.74679487179487 0.90721982717514
1.83012820512821 0.870177865028381
1.91666666666667 0.85889595746994
2.00641025641026 0.878538131713867
2.1025641025641 0.802766442298889
2.20512820512821 0.83222770690918
2.30769230769231 0.840608298778534
2.41987179487179 0.858874917030334
2.53525641025641 0.798944890499115
2.65384615384615 0.757811963558197
2.78205128205128 0.755878031253815
2.91346153846154 0.697557866573334
3.05128205128205 0.693157136440277
3.19871794871795 0.700852334499359
3.34935897435897 0.699650347232819
3.50961538461538 0.692317306995392
3.67628205128205 0.680674850940704
3.8525641025641 0.67500364780426
4.03525641025641 0.667843520641327
4.2275641025641 0.629268109798431
4.42948717948718 0.66686224937439
4.64102564102564 0.638322830200195
4.86217948717949 0.652617156505585
5.09294871794872 0.621285855770111
5.33653846153846 0.596221446990967
5.58974358974359 0.603265404701233
5.85576923076923 0.532636642456055
6.13461538461539 0.514267385005951
6.42628205128205 0.533519208431244
6.73397435897436 0.514649987220764
7.05448717948718 0.526903927326202
7.38782051282051 0.537785768508911
7.74038461538461 0.496264934539795
8.10897435897436 0.454264879226685
8.49679487179487 0.501817166805267
8.90064102564103 0.489875227212906
9.32371794871795 0.494772672653198
9.76923076923077 0.427493572235107
10.2339743589744 0.468816578388214
10.7211538461538 0.472870826721191
11.2307692307692 0.423752546310425
11.7660256410256 0.399371832609177
12.3269230769231 0.397924274206161
12.9134615384615 0.406538933515549
13.5288461538462 0.424441397190094
14.1730769230769 0.413355827331543
14.849358974359 0.383061975240707
15.5544871794872 0.36189004778862
16.2948717948718 0.382096141576767
17.0705128205128 0.336932182312012
17.8846153846154 0.305559396743774
18.7371794871795 0.312087684869766
19.6282051282051 0.311925560235977
20.5641025641026 0.295475780963898
21.5416666666667 0.301006287336349
22.5673076923077 0.327838510274887
23.6442307692308 0.260527700185776
24.7692307692308 0.271343231201172
25.9487179487179 0.304213017225266
27.1826923076923 0.290576279163361
28.4775641025641 0.26312717795372
29.8333333333333 0.203029200434685
31.2564102564103 0.236186310648918
32.7435897435897 0.257236033678055
34.3044871794872 0.265147745609283
35.9358974358974 0.210466727614403
37.6474358974359 0.195367887616158
39.4391025641026 0.208237171173096
41.3173076923077 0.247518643736839
43.2852564102564 0.295193880796432
45.3461538461538 0.270332276821136
47.5064102564103 0.236348867416382
49.7692307692308 0.235848501324654
52.1378205128205 0.149200066924095
54.6217948717949 0.248634323477745
57.2211538461538 0.203858524560928
59.9455128205128 0.274047672748566
62.8012820512821 0.1414754986763
65.7916666666667 0.26089283823967
68.9230769230769 0.230002030730247
72.2051282051282 0.161107808351517
75.6442307692308 0.353127151727676
79.2467948717949 0.0584991537034512
83.0192307692308 0.0996852889657021
86.974358974359 0.14545601606369
91.1153846153846 0.239404782652855
95.4519230769231 0.245538786053658
100 0.132678911089897
};
\addlegendentry{sub 16, exact}
\addplot [, color0, opacity=0.6, mark=diamond*, mark size=0.5, mark options={solid}, only marks, forget plot]
table {%
1 0.92079371213913
1.04487179487179 0.927169620990753
1.09615384615385 0.948464810848236
1.1474358974359 0.957361698150635
1.20192307692308 0.961495399475098
1.25961538461538 0.958748459815979
1.32051282051282 0.943971812725067
1.38461538461538 0.935104489326477
1.44871794871795 0.920330941677094
1.51923076923077 0.890888810157776
1.58974358974359 0.873867452144623
1.66666666666667 0.846007943153381
1.74679487179487 0.877481758594513
1.83012820512821 0.802021205425262
1.91666666666667 0.858036696910858
2.00641025641026 0.810580670833588
2.1025641025641 0.764240086078644
2.20512820512821 0.788184642791748
2.30769230769231 0.777264297008514
2.41987179487179 0.785275936126709
2.53525641025641 0.724171996116638
2.65384615384615 0.694069087505341
2.78205128205128 0.719691336154938
2.91346153846154 0.655286610126495
3.05128205128205 0.667177379131317
3.19871794871795 0.680621087551117
3.34935897435897 0.644111931324005
3.50961538461538 0.622357785701752
3.67628205128205 0.611611366271973
3.8525641025641 0.636419117450714
4.03525641025641 0.604954659938812
4.2275641025641 0.576358675956726
4.42948717948718 0.633906364440918
4.64102564102564 0.568512737751007
4.86217948717949 0.649107873439789
5.09294871794872 0.543856620788574
5.33653846153846 0.573671340942383
5.58974358974359 0.546430587768555
5.85576923076923 0.554791450500488
6.13461538461539 0.534180819988251
6.42628205128205 0.499080181121826
6.73397435897436 0.46892175078392
7.05448717948718 0.534217476844788
7.38782051282051 0.525696933269501
7.74038461538461 0.458494156599045
8.10897435897436 0.478506565093994
8.49679487179487 0.45926496386528
8.90064102564103 0.45716580748558
9.32371794871795 0.462975591421127
9.76923076923077 0.433444023132324
10.2339743589744 0.452879250049591
10.7211538461538 0.444012373685837
11.2307692307692 0.421893924474716
11.7660256410256 0.452226161956787
12.3269230769231 0.432783514261246
12.9134615384615 0.439874172210693
13.5288461538462 0.3600854575634
14.1730769230769 0.366194099187851
14.849358974359 0.374024897813797
15.5544871794872 0.371864646673203
16.2948717948718 0.423205763101578
17.0705128205128 0.370662480592728
17.8846153846154 0.347842216491699
18.7371794871795 0.335666745901108
19.6282051282051 0.327924937009811
20.5641025641026 0.370717525482178
21.5416666666667 0.339724987745285
22.5673076923077 0.293109714984894
23.6442307692308 0.259064197540283
24.7692307692308 0.283286243677139
25.9487179487179 0.2574263215065
27.1826923076923 0.295086741447449
28.4775641025641 0.325431436300278
29.8333333333333 0.342428982257843
31.2564102564103 0.299903839826584
32.7435897435897 0.258514136075974
34.3044871794872 0.260723263025284
35.9358974358974 0.212197691202164
37.6474358974359 0.361058264970779
39.4391025641026 0.202253252267838
41.3173076923077 0.172704860568047
43.2852564102564 0.169606193900108
45.3461538461538 0.135091379284859
47.5064102564103 0.210497125983238
49.7692307692308 0.211582258343697
52.1378205128205 0.211585864424706
54.6217948717949 0.171237468719482
57.2211538461538 0.199658513069153
59.9455128205128 0.101619318127632
62.8012820512821 0.105201937258244
65.7916666666667 0.074571318924427
68.9230769230769 0.263298571109772
72.2051282051282 0.0837535038590431
75.6442307692308 0.0623635314404964
79.2467948717949 0.109027482569218
83.0192307692308 0.198338627815247
86.974358974359 0.283816397190094
91.1153846153846 0.0339203141629696
95.4519230769231 0.19444015622139
100 0.030828446149826
};
\addplot [, color0, opacity=0.6, mark=diamond*, mark size=0.5, mark options={solid}, only marks, forget plot]
table {%
1 0.918657124042511
1.04487179487179 0.918396890163422
1.09615384615385 0.942203998565674
1.1474358974359 0.944535255432129
1.20192307692308 0.957195699214935
1.25961538461538 0.956563413143158
1.32051282051282 0.947965264320374
1.38461538461538 0.947304546833038
1.44871794871795 0.952404022216797
1.51923076923077 0.925373017787933
1.58974358974359 0.929836452007294
1.66666666666667 0.873408734798431
1.74679487179487 0.922114193439484
1.83012820512821 0.823908030986786
1.91666666666667 0.828122735023499
2.00641025641026 0.842956960201263
2.1025641025641 0.804186284542084
2.20512820512821 0.794424712657928
2.30769230769231 0.857735931873322
2.41987179487179 0.85961389541626
2.53525641025641 0.71062433719635
2.65384615384615 0.812987804412842
2.78205128205128 0.834612190723419
2.91346153846154 0.738170444965363
3.05128205128205 0.683432519435883
3.19871794871795 0.738227367401123
3.34935897435897 0.69095116853714
3.50961538461538 0.695846021175385
3.67628205128205 0.686433970928192
3.8525641025641 0.691582679748535
4.03525641025641 0.653391182422638
4.2275641025641 0.678915917873383
4.42948717948718 0.688633739948273
4.64102564102564 0.644662618637085
4.86217948717949 0.650230705738068
5.09294871794872 0.669832110404968
5.33653846153846 0.620664775371552
5.58974358974359 0.632856369018555
5.85576923076923 0.593298614025116
6.13461538461539 0.576827585697174
6.42628205128205 0.576250612735748
6.73397435897436 0.554677724838257
7.05448717948718 0.586200535297394
7.38782051282051 0.478739082813263
7.74038461538461 0.510753691196442
8.10897435897436 0.462569534778595
8.49679487179487 0.533488571643829
8.90064102564103 0.511804819107056
9.32371794871795 0.482722848653793
9.76923076923077 0.474782556295395
10.2339743589744 0.508318364620209
10.7211538461538 0.448005020618439
11.2307692307692 0.435300350189209
11.7660256410256 0.430087953805923
12.3269230769231 0.390214532613754
12.9134615384615 0.403236001729965
13.5288461538462 0.360037088394165
14.1730769230769 0.377526253461838
14.849358974359 0.388696491718292
15.5544871794872 0.39914733171463
16.2948717948718 0.41264209151268
17.0705128205128 0.404261976480484
17.8846153846154 0.368996769189835
18.7371794871795 0.373172700405121
19.6282051282051 0.416049808263779
20.5641025641026 0.369826585054398
21.5416666666667 0.29476472735405
22.5673076923077 0.42455992102623
23.6442307692308 0.258323520421982
24.7692307692308 0.28448161482811
25.9487179487179 0.262832969427109
27.1826923076923 0.257038414478302
28.4775641025641 0.247354373335838
29.8333333333333 0.369669497013092
31.2564102564103 0.273457646369934
32.7435897435897 0.221724137663841
34.3044871794872 0.290928900241852
35.9358974358974 0.259882807731628
37.6474358974359 0.240457013249397
39.4391025641026 0.184648752212524
41.3173076923077 0.298074156045914
43.2852564102564 0.191974967718124
45.3461538461538 0.163400456309319
47.5064102564103 0.187700316309929
49.7692307692308 0.236141830682755
52.1378205128205 0.173238202929497
54.6217948717949 0.16104955971241
57.2211538461538 0.20845864713192
59.9455128205128 0.142012163996696
62.8012820512821 0.0946959480643272
65.7916666666667 0.0710245594382286
68.9230769230769 0.199165105819702
72.2051282051282 0.0749408975243568
75.6442307692308 0.236189171671867
79.2467948717949 0.233917817473412
83.0192307692308 0.128950819373131
86.974358974359 0.234861716628075
91.1153846153846 0.0946845561265945
95.4519230769231 0.0613366030156612
100 0.0828211084008217
};
\addplot [, color0, opacity=0.6, mark=diamond*, mark size=0.5, mark options={solid}, only marks, forget plot]
table {%
1 0.898417413234711
1.04487179487179 0.851814270019531
1.09615384615385 0.928842723369598
1.1474358974359 0.931977391242981
1.20192307692308 0.939489185810089
1.25961538461538 0.93865966796875
1.32051282051282 0.924961745738983
1.38461538461538 0.91200190782547
1.44871794871795 0.843735158443451
1.51923076923077 0.833331704139709
1.58974358974359 0.824720203876495
1.66666666666667 0.794365048408508
1.74679487179487 0.866570472717285
1.83012820512821 0.749715745449066
1.91666666666667 0.849232494831085
2.00641025641026 0.79074639081955
2.1025641025641 0.780781447887421
2.20512820512821 0.779139280319214
2.30769230769231 0.759755432605743
2.41987179487179 0.77288693189621
2.53525641025641 0.690141022205353
2.65384615384615 0.720438003540039
2.78205128205128 0.750119745731354
2.91346153846154 0.712960362434387
3.05128205128205 0.705355167388916
3.19871794871795 0.716013193130493
3.34935897435897 0.69077855348587
3.50961538461538 0.637857913970947
3.67628205128205 0.679377675056458
3.8525641025641 0.652207314968109
4.03525641025641 0.619421899318695
4.2275641025641 0.612200736999512
4.42948717948718 0.614670932292938
4.64102564102564 0.584074676036835
4.86217948717949 0.633702397346497
5.09294871794872 0.544771075248718
5.33653846153846 0.558192729949951
5.58974358974359 0.531719744205475
5.85576923076923 0.548494517803192
6.13461538461539 0.489826291799545
6.42628205128205 0.493423610925674
6.73397435897436 0.513332307338715
7.05448717948718 0.476200878620148
7.38782051282051 0.510102272033691
7.74038461538461 0.472250998020172
8.10897435897436 0.463788330554962
8.49679487179487 0.471033543348312
8.90064102564103 0.453337401151657
9.32371794871795 0.437231630086899
9.76923076923077 0.401941627264023
10.2339743589744 0.389676630496979
10.7211538461538 0.396192997694016
11.2307692307692 0.369161427021027
11.7660256410256 0.386666417121887
12.3269230769231 0.421507358551025
12.9134615384615 0.423249691724777
13.5288461538462 0.367647737264633
14.1730769230769 0.396536141633987
14.849358974359 0.33983963727951
15.5544871794872 0.331656634807587
16.2948717948718 0.361022710800171
17.0705128205128 0.391286253929138
17.8846153846154 0.352502673864365
18.7371794871795 0.333785533905029
19.6282051282051 0.360387414693832
20.5641025641026 0.307435482740402
21.5416666666667 0.303281515836716
22.5673076923077 0.357786864042282
23.6442307692308 0.342148691415787
24.7692307692308 0.350759506225586
25.9487179487179 0.383415669202805
27.1826923076923 0.365994065999985
28.4775641025641 0.314508259296417
29.8333333333333 0.295742034912109
31.2564102564103 0.254866600036621
32.7435897435897 0.282463699579239
34.3044871794872 0.286301702260971
35.9358974358974 0.305498450994492
37.6474358974359 0.352061837911606
39.4391025641026 0.252502679824829
41.3173076923077 0.271017700433731
43.2852564102564 0.296779841184616
45.3461538461538 0.263206213712692
47.5064102564103 0.282372713088989
49.7692307692308 0.262777656316757
52.1378205128205 0.230278491973877
54.6217948717949 0.232639953494072
57.2211538461538 0.208274275064468
59.9455128205128 0.286186188459396
62.8012820512821 0.0690152794122696
65.7916666666667 0.215859651565552
68.9230769230769 0.188717633485794
72.2051282051282 0.160466015338898
75.6442307692308 0.236104592680931
79.2467948717949 0.121762298047543
83.0192307692308 0.0549960322678089
86.974358974359 0.139825090765953
91.1153846153846 0.145651921629906
95.4519230769231 0.147025808691978
100 0.117409586906433
};
\addplot [, color0, opacity=0.6, mark=diamond*, mark size=0.5, mark options={solid}, only marks, forget plot]
table {%
1 0.899999260902405
1.04487179487179 0.882456421852112
1.09615384615385 0.927780747413635
1.1474358974359 0.943979442119598
1.20192307692308 0.953908443450928
1.25961538461538 0.945058763027191
1.32051282051282 0.938016831874847
1.38461538461538 0.937312722206116
1.44871794871795 0.947974503040314
1.51923076923077 0.934428811073303
1.58974358974359 0.912378132343292
1.66666666666667 0.908989906311035
1.74679487179487 0.893058300018311
1.83012820512821 0.901787579059601
1.91666666666667 0.896422028541565
2.00641025641026 0.888144671916962
2.1025641025641 0.807176411151886
2.20512820512821 0.873046100139618
2.30769230769231 0.778032720088959
2.41987179487179 0.79463803768158
2.53525641025641 0.761512994766235
2.65384615384615 0.720474660396576
2.78205128205128 0.7614426612854
2.91346153846154 0.757330358028412
3.05128205128205 0.74747771024704
3.19871794871795 0.710917472839355
3.34935897435897 0.679825723171234
3.50961538461538 0.701754868030548
3.67628205128205 0.693893611431122
3.8525641025641 0.665106534957886
4.03525641025641 0.681425631046295
4.2275641025641 0.663657128810883
4.42948717948718 0.674525558948517
4.64102564102564 0.55382776260376
4.86217948717949 0.620808243751526
5.09294871794872 0.559154331684113
5.33653846153846 0.586264550685883
5.58974358974359 0.621337831020355
5.85576923076923 0.603823959827423
6.13461538461539 0.571065068244934
6.42628205128205 0.588352024555206
6.73397435897436 0.531150996685028
7.05448717948718 0.492500990629196
7.38782051282051 0.502230584621429
7.74038461538461 0.512735247612
8.10897435897436 0.544516026973724
8.49679487179487 0.496012508869171
8.90064102564103 0.504679799079895
9.32371794871795 0.457757472991943
9.76923076923077 0.455803006887436
10.2339743589744 0.433708280324936
10.7211538461538 0.491513460874557
11.2307692307692 0.431506931781769
11.7660256410256 0.420132726430893
12.3269230769231 0.42909836769104
12.9134615384615 0.410400211811066
13.5288461538462 0.353893339633942
14.1730769230769 0.464257001876831
14.849358974359 0.39624285697937
15.5544871794872 0.340957581996918
16.2948717948718 0.375788420438766
17.0705128205128 0.414536386728287
17.8846153846154 0.37130019068718
18.7371794871795 0.418677866458893
19.6282051282051 0.312666565179825
20.5641025641026 0.394855737686157
21.5416666666667 0.384037405252457
22.5673076923077 0.354711502790451
23.6442307692308 0.457789152860641
24.7692307692308 0.336047530174255
25.9487179487179 0.415532916784286
27.1826923076923 0.402599573135376
28.4775641025641 0.333068937063217
29.8333333333333 0.310533404350281
31.2564102564103 0.432157427072525
32.7435897435897 0.359858125448227
34.3044871794872 0.292284190654755
35.9358974358974 0.357258051633835
37.6474358974359 0.353576332330704
39.4391025641026 0.430161148309708
41.3173076923077 0.281323999166489
43.2852564102564 0.462880045175552
45.3461538461538 0.302064478397369
47.5064102564103 0.204286336898804
49.7692307692308 0.429232835769653
52.1378205128205 0.194210603833199
54.6217948717949 0.395629078149796
57.2211538461538 0.292696237564087
59.9455128205128 0.29472091794014
62.8012820512821 0.24073289334774
65.7916666666667 0.426403522491455
68.9230769230769 0.180630162358284
72.2051282051282 0.275787174701691
75.6442307692308 0.245878964662552
79.2467948717949 0.302150577306747
83.0192307692308 0.133457228541374
86.974358974359 0.118880271911621
91.1153846153846 0.244807407259941
95.4519230769231 0.256155252456665
100 0.30969712138176
};
\addplot [, color1, opacity=0.6, mark=square*, mark size=0.5, mark options={solid}, only marks]
table {%
1 0.821536481380463
1.04487179487179 0.836988627910614
1.09615384615385 0.900912880897522
1.1474358974359 0.918984889984131
1.20192307692308 0.943018853664398
1.25961538461538 0.948175072669983
1.32051282051282 0.939832806587219
1.38461538461538 0.937181949615479
1.44871794871795 0.939040958881378
1.51923076923077 0.939977586269379
1.58974358974359 0.924660861492157
1.66666666666667 0.922764420509338
1.74679487179487 0.925840079784393
1.83012820512821 0.881533443927765
1.91666666666667 0.907904326915741
2.00641025641026 0.918061196804047
2.1025641025641 0.896691799163818
2.20512820512821 0.889156818389893
2.30769230769231 0.904992282390594
2.41987179487179 0.909627735614777
2.53525641025641 0.826855957508087
2.65384615384615 0.885266602039337
2.78205128205128 0.881583869457245
2.91346153846154 0.808540165424347
3.05128205128205 0.801463305950165
3.19871794871795 0.834697246551514
3.34935897435897 0.746722996234894
3.50961538461538 0.761654257774353
3.67628205128205 0.715825080871582
3.8525641025641 0.784887254238129
4.03525641025641 0.684176087379456
4.2275641025641 0.690336406230927
4.42948717948718 0.712054550647736
4.64102564102564 0.735484719276428
4.86217948717949 0.74969357252121
5.09294871794872 0.813213527202606
5.33653846153846 0.671560764312744
5.58974358974359 0.675161480903625
5.85576923076923 0.682844161987305
6.13461538461539 0.698985278606415
6.42628205128205 0.665590286254883
6.73397435897436 0.659144937992096
7.05448717948718 0.605575501918793
7.38782051282051 0.608599483966827
7.74038461538461 0.605737626552582
8.10897435897436 0.63532680273056
8.49679487179487 0.639264464378357
8.90064102564103 0.599627196788788
9.32371794871795 0.630376756191254
9.76923076923077 0.493766516447067
10.2339743589744 0.586206138134003
10.7211538461538 0.554126679897308
11.2307692307692 0.568827509880066
11.7660256410256 0.580055177211761
12.3269230769231 0.51358026266098
12.9134615384615 0.506496369838715
13.5288461538462 0.484460651874542
14.1730769230769 0.527990758419037
14.849358974359 0.500033974647522
15.5544871794872 0.513541638851166
16.2948717948718 0.604045569896698
17.0705128205128 0.499924272298813
17.8846153846154 0.502290546894073
18.7371794871795 0.425588846206665
19.6282051282051 0.472161144018173
20.5641025641026 0.50146210193634
21.5416666666667 0.520479798316956
22.5673076923077 0.44746071100235
23.6442307692308 0.445253938436508
24.7692307692308 0.42743706703186
25.9487179487179 0.370585352182388
27.1826923076923 0.476613521575928
28.4775641025641 0.474534034729004
29.8333333333333 0.472054809331894
31.2564102564103 0.407167255878448
32.7435897435897 0.454943001270294
34.3044871794872 0.411929666996002
35.9358974358974 0.372235864400864
37.6474358974359 0.404720693826675
39.4391025641026 0.376040905714035
41.3173076923077 0.394573301076889
43.2852564102564 0.396208763122559
45.3461538461538 0.502143085002899
47.5064102564103 0.382223695516586
49.7692307692308 0.48518705368042
52.1378205128205 0.252993285655975
54.6217948717949 0.392103165388107
57.2211538461538 0.28157976269722
59.9455128205128 0.317719876766205
62.8012820512821 0.592977166175842
65.7916666666667 0.364730596542358
68.9230769230769 0.520031154155731
72.2051282051282 0.535465478897095
75.6442307692308 0.379101991653442
79.2467948717949 0.691120564937592
83.0192307692308 0.412434101104736
86.974358974359 0.622314929962158
91.1153846153846 0.767853617668152
95.4519230769231 0.685227036476135
100 0.683497726917267
};
\addlegendentry{mb 128, mc 1}
\addplot [, color1, opacity=0.6, mark=square*, mark size=0.5, mark options={solid}, only marks, forget plot]
table {%
1 0.855037689208984
1.04487179487179 0.829615890979767
1.09615384615385 0.90369188785553
1.1474358974359 0.896299779415131
1.20192307692308 0.95094758272171
1.25961538461538 0.95475959777832
1.32051282051282 0.85222589969635
1.38461538461538 0.937429070472717
1.44871794871795 0.943259418010712
1.51923076923077 0.861702561378479
1.58974358974359 0.93083667755127
1.66666666666667 0.872320592403412
1.74679487179487 0.926131367683411
1.83012820512821 0.832128345966339
1.91666666666667 0.823247909545898
2.00641025641026 0.841150879859924
2.1025641025641 0.85786497592926
2.20512820512821 0.844186961650848
2.30769230769231 0.885405242443085
2.41987179487179 0.834536075592041
2.53525641025641 0.820364594459534
2.65384615384615 0.864756047725677
2.78205128205128 0.855366408824921
2.91346153846154 0.799132883548737
3.05128205128205 0.753038287162781
3.19871794871795 0.751687943935394
3.34935897435897 0.80396693944931
3.50961538461538 0.696290135383606
3.67628205128205 0.751822471618652
3.8525641025641 0.773364365100861
4.03525641025641 0.694341540336609
4.2275641025641 0.723410308361053
4.42948717948718 0.666123926639557
4.64102564102564 0.675699174404144
4.86217948717949 0.7330282330513
5.09294871794872 0.639433264732361
5.33653846153846 0.629316747188568
5.58974358974359 0.586908161640167
5.85576923076923 0.635476529598236
6.13461538461539 0.642790794372559
6.42628205128205 0.697350919246674
6.73397435897436 0.573751628398895
7.05448717948718 0.568109214305878
7.38782051282051 0.548824846744537
7.74038461538461 0.549994647502899
8.10897435897436 0.630341947078705
8.49679487179487 0.538522601127625
8.90064102564103 0.590901553630829
9.32371794871795 0.541363656520844
9.76923076923077 0.63256710767746
10.2339743589744 0.626124024391174
10.7211538461538 0.597727954387665
11.2307692307692 0.619581520557404
11.7660256410256 0.531105935573578
12.3269230769231 0.500059545040131
12.9134615384615 0.552487552165985
13.5288461538462 0.570802927017212
14.1730769230769 0.530659854412079
14.849358974359 0.505206406116486
15.5544871794872 0.444728851318359
16.2948717948718 0.549237191677094
17.0705128205128 0.494614809751511
17.8846153846154 0.495690196752548
18.7371794871795 0.489203363656998
19.6282051282051 0.476017475128174
20.5641025641026 0.419704347848892
21.5416666666667 0.516767203807831
22.5673076923077 0.47017976641655
23.6442307692308 0.384391754865646
24.7692307692308 0.497098058462143
25.9487179487179 0.491053313016891
27.1826923076923 0.412255734205246
28.4775641025641 0.479010879993439
29.8333333333333 0.305399239063263
31.2564102564103 0.548373401165009
32.7435897435897 0.47672364115715
34.3044871794872 0.478894144296646
35.9358974358974 0.476817458868027
37.6474358974359 0.444651365280151
39.4391025641026 0.467564880847931
41.3173076923077 0.528242886066437
43.2852564102564 0.412080138921738
45.3461538461538 0.495342552661896
47.5064102564103 0.385786026716232
49.7692307692308 0.448743641376495
52.1378205128205 0.385011792182922
54.6217948717949 0.546417653560638
57.2211538461538 0.430354982614517
59.9455128205128 0.455570697784424
62.8012820512821 0.495752066373825
65.7916666666667 0.564198195934296
68.9230769230769 0.492966502904892
72.2051282051282 0.61952942609787
75.6442307692308 0.492105454206467
79.2467948717949 0.682148635387421
83.0192307692308 0.502669632434845
86.974358974359 0.637248814105988
91.1153846153846 0.570687234401703
95.4519230769231 0.52857917547226
100 0.715697705745697
};
\addplot [, color1, opacity=0.6, mark=square*, mark size=0.5, mark options={solid}, only marks, forget plot]
table {%
1 0.870231330394745
1.04487179487179 0.864265084266663
1.09615384615385 0.913632810115814
1.1474358974359 0.895847976207733
1.20192307692308 0.935676276683807
1.25961538461538 0.928650856018066
1.32051282051282 0.850744664669037
1.38461538461538 0.858252346515656
1.44871794871795 0.931420624256134
1.51923076923077 0.837611854076385
1.58974358974359 0.935239791870117
1.66666666666667 0.868233680725098
1.74679487179487 0.906620502471924
1.83012820512821 0.803380906581879
1.91666666666667 0.827847123146057
2.00641025641026 0.889524102210999
2.1025641025641 0.874110043048859
2.20512820512821 0.866022706031799
2.30769230769231 0.906705319881439
2.41987179487179 0.847125828266144
2.53525641025641 0.794109523296356
2.65384615384615 0.832841515541077
2.78205128205128 0.878423392772675
2.91346153846154 0.810570240020752
3.05128205128205 0.822215259075165
3.19871794871795 0.808277606964111
3.34935897435897 0.763199508190155
3.50961538461538 0.783763408660889
3.67628205128205 0.797513425350189
3.8525641025641 0.758368909358978
4.03525641025641 0.752797663211823
4.2275641025641 0.751946985721588
4.42948717948718 0.756550192832947
4.64102564102564 0.733992874622345
4.86217948717949 0.782665550708771
5.09294871794872 0.699683487415314
5.33653846153846 0.695080637931824
5.58974358974359 0.685530483722687
5.85576923076923 0.622712075710297
6.13461538461539 0.727340340614319
6.42628205128205 0.631759166717529
6.73397435897436 0.567541062831879
7.05448717948718 0.654831886291504
7.38782051282051 0.567745506763458
7.74038461538461 0.551700115203857
8.10897435897436 0.578812420368195
8.49679487179487 0.569134831428528
8.90064102564103 0.579435110092163
9.32371794871795 0.546377003192902
9.76923076923077 0.564019978046417
10.2339743589744 0.560285210609436
10.7211538461538 0.54930704832077
11.2307692307692 0.483713686466217
11.7660256410256 0.544830679893494
12.3269230769231 0.42657932639122
12.9134615384615 0.536512136459351
13.5288461538462 0.459861665964127
14.1730769230769 0.419750690460205
14.849358974359 0.452296078205109
15.5544871794872 0.413903534412384
16.2948717948718 0.514133632183075
17.0705128205128 0.409888744354248
17.8846153846154 0.452392488718033
18.7371794871795 0.401988655328751
19.6282051282051 0.436873733997345
20.5641025641026 0.426780134439468
21.5416666666667 0.47317361831665
22.5673076923077 0.430841773748398
23.6442307692308 0.366391390562057
24.7692307692308 0.418620824813843
25.9487179487179 0.344549238681793
27.1826923076923 0.430529117584229
28.4775641025641 0.340495109558105
29.8333333333333 0.331642359495163
31.2564102564103 0.38271901011467
32.7435897435897 0.39160019159317
34.3044871794872 0.362328052520752
35.9358974358974 0.322470009326935
37.6474358974359 0.305771738290787
39.4391025641026 0.446733236312866
41.3173076923077 0.42604598402977
43.2852564102564 0.426809698343277
45.3461538461538 0.438206493854523
47.5064102564103 0.377445787191391
49.7692307692308 0.45376244187355
52.1378205128205 0.36466246843338
54.6217948717949 0.506075203418732
57.2211538461538 0.520283401012421
59.9455128205128 0.335807949304581
62.8012820512821 0.359207898378372
65.7916666666667 0.518457353115082
68.9230769230769 0.318611562252045
72.2051282051282 0.67891937494278
75.6442307692308 0.461025923490524
79.2467948717949 0.732236683368683
83.0192307692308 0.677261590957642
86.974358974359 0.423845052719116
91.1153846153846 0.640069305896759
95.4519230769231 0.735985577106476
100 0.773350179195404
};
\addplot [, color1, opacity=0.6, mark=square*, mark size=0.5, mark options={solid}, only marks, forget plot]
table {%
1 0.824598014354706
1.04487179487179 0.870675504207611
1.09615384615385 0.909427762031555
1.1474358974359 0.898670792579651
1.20192307692308 0.950956344604492
1.25961538461538 0.943220734596252
1.32051282051282 0.940022170543671
1.38461538461538 0.937577903270721
1.44871794871795 0.865824162960052
1.51923076923077 0.895094692707062
1.58974358974359 0.914236545562744
1.66666666666667 0.915562331676483
1.74679487179487 0.933828949928284
1.83012820512821 0.859357953071594
1.91666666666667 0.916995048522949
2.00641025641026 0.926380336284637
2.1025641025641 0.901451051235199
2.20512820512821 0.918374538421631
2.30769230769231 0.900839984416962
2.41987179487179 0.85970801115036
2.53525641025641 0.838645935058594
2.65384615384615 0.861213684082031
2.78205128205128 0.880528628826141
2.91346153846154 0.814410209655762
3.05128205128205 0.765917778015137
3.19871794871795 0.791609406471252
3.34935897435897 0.744612693786621
3.50961538461538 0.7879478931427
3.67628205128205 0.718950748443604
3.8525641025641 0.815241634845734
4.03525641025641 0.737640857696533
4.2275641025641 0.752559781074524
4.42948717948718 0.73018616437912
4.64102564102564 0.670349717140198
4.86217948717949 0.673598229885101
5.09294871794872 0.736574470996857
5.33653846153846 0.673747479915619
5.58974358974359 0.714199960231781
5.85576923076923 0.663001120090485
6.13461538461539 0.661270081996918
6.42628205128205 0.661600768566132
6.73397435897436 0.599381744861603
7.05448717948718 0.594195783138275
7.38782051282051 0.624175548553467
7.74038461538461 0.603714108467102
8.10897435897436 0.605467081069946
8.49679487179487 0.585493743419647
8.90064102564103 0.572417438030243
9.32371794871795 0.526614010334015
9.76923076923077 0.556668400764465
10.2339743589744 0.537765204906464
10.7211538461538 0.55686628818512
11.2307692307692 0.513161480426788
11.7660256410256 0.548825204372406
12.3269230769231 0.528359830379486
12.9134615384615 0.522439181804657
13.5288461538462 0.524537742137909
14.1730769230769 0.497906506061554
14.849358974359 0.452903568744659
15.5544871794872 0.466869175434113
16.2948717948718 0.455758482217789
17.0705128205128 0.462883561849594
17.8846153846154 0.502047955989838
18.7371794871795 0.394104987382889
19.6282051282051 0.425890445709229
20.5641025641026 0.486678510904312
21.5416666666667 0.520444571971893
22.5673076923077 0.475084275007248
23.6442307692308 0.478956431150436
24.7692307692308 0.465332567691803
25.9487179487179 0.385926097631454
27.1826923076923 0.293023407459259
28.4775641025641 0.376670360565186
29.8333333333333 0.524049162864685
31.2564102564103 0.446782201528549
32.7435897435897 0.428802162408829
34.3044871794872 0.387484282255173
35.9358974358974 0.396704912185669
37.6474358974359 0.362026304006577
39.4391025641026 0.277877420186996
41.3173076923077 0.36803987622261
43.2852564102564 0.378222912549973
45.3461538461538 0.590081870555878
47.5064102564103 0.29718005657196
49.7692307692308 0.374723762273788
52.1378205128205 0.400397598743439
54.6217948717949 0.528149783611298
57.2211538461538 0.443580150604248
59.9455128205128 0.446119517087936
62.8012820512821 0.37411031126976
65.7916666666667 0.504120349884033
68.9230769230769 0.457499951124191
72.2051282051282 0.286293476819992
75.6442307692308 0.434602946043015
79.2467948717949 0.433737516403198
83.0192307692308 0.76132744550705
86.974358974359 0.547272264957428
91.1153846153846 0.676672637462616
95.4519230769231 0.70393180847168
100 0.890640437602997
};
\addplot [, color1, opacity=0.6, mark=square*, mark size=0.5, mark options={solid}, only marks, forget plot]
table {%
1 0.836081922054291
1.04487179487179 0.878108322620392
1.09615384615385 0.89419287443161
1.1474358974359 0.91962605714798
1.20192307692308 0.946121633052826
1.25961538461538 0.941932857036591
1.32051282051282 0.922869145870209
1.38461538461538 0.945886254310608
1.44871794871795 0.939644932746887
1.51923076923077 0.924158275127411
1.58974358974359 0.932953000068665
1.66666666666667 0.946211040019989
1.74679487179487 0.927273690700531
1.83012820512821 0.931307256221771
1.91666666666667 0.909711480140686
2.00641025641026 0.92072594165802
2.1025641025641 0.833494484424591
2.20512820512821 0.888323605060577
2.30769230769231 0.878596723079681
2.41987179487179 0.844358623027802
2.53525641025641 0.803017795085907
2.65384615384615 0.815050780773163
2.78205128205128 0.752181947231293
2.91346153846154 0.839597702026367
3.05128205128205 0.792268812656403
3.19871794871795 0.783621251583099
3.34935897435897 0.846132874488831
3.50961538461538 0.773506343364716
3.67628205128205 0.738732159137726
3.8525641025641 0.789678454399109
4.03525641025641 0.764963030815125
4.2275641025641 0.787097752094269
4.42948717948718 0.776150047779083
4.64102564102564 0.760660111904144
4.86217948717949 0.759917914867401
5.09294871794872 0.738684356212616
5.33653846153846 0.784147024154663
5.58974358974359 0.711253762245178
5.85576923076923 0.703017234802246
6.13461538461539 0.701045989990234
6.42628205128205 0.616722881793976
6.73397435897436 0.6111940741539
7.05448717948718 0.59862744808197
7.38782051282051 0.611221849918365
7.74038461538461 0.643155753612518
8.10897435897436 0.583229839801788
8.49679487179487 0.608367919921875
8.90064102564103 0.583752155303955
9.32371794871795 0.625399708747864
9.76923076923077 0.53704446554184
10.2339743589744 0.562770366668701
10.7211538461538 0.541897296905518
11.2307692307692 0.592826962471008
11.7660256410256 0.486340522766113
12.3269230769231 0.613977909088135
12.9134615384615 0.4786636531353
13.5288461538462 0.558842539787292
14.1730769230769 0.478711098432541
14.849358974359 0.449088394641876
15.5544871794872 0.533777356147766
16.2948717948718 0.492726653814316
17.0705128205128 0.554800987243652
17.8846153846154 0.563252866268158
18.7371794871795 0.45338162779808
19.6282051282051 0.478230148553848
20.5641025641026 0.491569817066193
21.5416666666667 0.482208102941513
22.5673076923077 0.527066826820374
23.6442307692308 0.445668131113052
24.7692307692308 0.386629104614258
25.9487179487179 0.582802653312683
27.1826923076923 0.374874114990234
28.4775641025641 0.437712967395782
29.8333333333333 0.5157830119133
31.2564102564103 0.456702679395676
32.7435897435897 0.34437894821167
34.3044871794872 0.442425459623337
35.9358974358974 0.569804966449738
37.6474358974359 0.480719953775406
39.4391025641026 0.424489408731461
41.3173076923077 0.455940455198288
43.2852564102564 0.407915115356445
45.3461538461538 0.460985004901886
47.5064102564103 0.418169885873795
49.7692307692308 0.533890247344971
52.1378205128205 0.398973137140274
54.6217948717949 0.410561770200729
57.2211538461538 0.554765403270721
59.9455128205128 0.580201089382172
62.8012820512821 0.455897718667984
65.7916666666667 0.740860402584076
68.9230769230769 0.667912662029266
72.2051282051282 0.559895694255829
75.6442307692308 0.609615921974182
79.2467948717949 0.83687162399292
83.0192307692308 0.805305421352386
86.974358974359 0.861376404762268
91.1153846153846 0.785144805908203
95.4519230769231 0.79277241230011
100 0.834657669067383
};
\addplot [, color2, opacity=0.6, mark=triangle*, mark size=0.5, mark options={solid,rotate=180}, only marks]
table {%
1 0.443183243274689
1.04487179487179 0.461680233478546
1.09615384615385 0.498393684625626
1.1474358974359 0.52729195356369
1.20192307692308 0.553349316120148
1.25961538461538 0.569040715694427
1.32051282051282 0.541220664978027
1.38461538461538 0.561562180519104
1.44871794871795 0.587412774562836
1.51923076923077 0.553801715373993
1.58974358974359 0.56314754486084
1.66666666666667 0.528475701808929
1.74679487179487 0.54828017950058
1.83012820512821 0.549692094326019
1.91666666666667 0.482747882604599
2.00641025641026 0.557420015335083
2.1025641025641 0.527479827404022
2.20512820512821 0.493456929922104
2.30769230769231 0.516636788845062
2.41987179487179 0.536119937896729
2.53525641025641 0.487029790878296
2.65384615384615 0.508204638957977
2.78205128205128 0.486515045166016
2.91346153846154 0.453017204999924
3.05128205128205 0.421819418668747
3.19871794871795 0.424238413572311
3.34935897435897 0.454678624868393
3.50961538461538 0.433881759643555
3.67628205128205 0.481412798166275
3.8525641025641 0.42001873254776
4.03525641025641 0.43138200044632
4.2275641025641 0.420171171426773
4.42948717948718 0.425742626190186
4.64102564102564 0.414144247770309
4.86217948717949 0.418724775314331
5.09294871794872 0.43002986907959
5.33653846153846 0.344026982784271
5.58974358974359 0.381075620651245
5.85576923076923 0.407561302185059
6.13461538461539 0.344098001718521
6.42628205128205 0.353757977485657
6.73397435897436 0.314651817083359
7.05448717948718 0.340841501951218
7.38782051282051 0.315876215696335
7.74038461538461 0.334727823734283
8.10897435897436 0.344644069671631
8.49679487179487 0.281014204025269
8.90064102564103 0.293249160051346
9.32371794871795 0.312870800495148
9.76923076923077 0.293457478284836
10.2339743589744 0.310952991247177
10.7211538461538 0.279437750577927
11.2307692307692 0.313344091176987
11.7660256410256 0.310264319181442
12.3269230769231 0.275815099477768
12.9134615384615 0.267825961112976
13.5288461538462 0.306696712970734
14.1730769230769 0.28173166513443
14.849358974359 0.276656478643417
15.5544871794872 0.272863924503326
16.2948717948718 0.293922156095505
17.0705128205128 0.261333853006363
17.8846153846154 0.234603866934776
18.7371794871795 0.24436990916729
19.6282051282051 0.273804187774658
20.5641025641026 0.245383411645889
21.5416666666667 0.233716487884521
22.5673076923077 0.270071446895599
23.6442307692308 0.215493202209473
24.7692307692308 0.214627459645271
25.9487179487179 0.213408857584
27.1826923076923 0.245386987924576
28.4775641025641 0.211775168776512
29.8333333333333 0.162962526082993
31.2564102564103 0.215501740574837
32.7435897435897 0.225812658667564
34.3044871794872 0.201081871986389
35.9358974358974 0.168067172169685
37.6474358974359 0.142940431833267
39.4391025641026 0.184664472937584
41.3173076923077 0.215630769729614
43.2852564102564 0.252500623464584
45.3461538461538 0.234359785914421
47.5064102564103 0.229869589209557
49.7692307692308 0.20531153678894
52.1378205128205 0.173069760203362
54.6217948717949 0.234892204403877
57.2211538461538 0.187612161040306
59.9455128205128 0.237855941057205
62.8012820512821 0.129249215126038
65.7916666666667 0.257640689611435
68.9230769230769 0.22495137155056
72.2051282051282 0.146102502942085
75.6442307692308 0.290668785572052
79.2467948717949 0.0598441250622272
83.0192307692308 0.0896671563386917
86.974358974359 0.141880854964256
91.1153846153846 0.248976349830627
95.4519230769231 0.242623284459114
100 0.134660050272942
};
\addlegendentry{sub 16, mc 1}
\addplot [, color2, opacity=0.6, mark=triangle*, mark size=0.5, mark options={solid,rotate=180}, only marks, forget plot]
table {%
1 0.459099680185318
1.04487179487179 0.421785652637482
1.09615384615385 0.522989451885223
1.1474358974359 0.565946757793427
1.20192307692308 0.613647401332855
1.25961538461538 0.66900771856308
1.32051282051282 0.547292709350586
1.38461538461538 0.590806126594543
1.44871794871795 0.629748463630676
1.51923076923077 0.596082329750061
1.58974358974359 0.558642864227295
1.66666666666667 0.542568564414978
1.74679487179487 0.553018867969513
1.83012820512821 0.538771271705627
1.91666666666667 0.535280644893646
2.00641025641026 0.504095494747162
2.1025641025641 0.518736779689789
2.20512820512821 0.493586629629135
2.30769230769231 0.461242407560349
2.41987179487179 0.488189786672592
2.53525641025641 0.493773996829987
2.65384615384615 0.487510293722153
2.78205128205128 0.456386178731918
2.91346153846154 0.432655185461044
3.05128205128205 0.4466852247715
3.19871794871795 0.420371353626251
3.34935897435897 0.414314091205597
3.50961538461538 0.401930421590805
3.67628205128205 0.41614505648613
3.8525641025641 0.427086800336838
4.03525641025641 0.404346197843552
4.2275641025641 0.393365144729614
4.42948717948718 0.3445725440979
4.64102564102564 0.344242244958878
4.86217948717949 0.357523918151855
5.09294871794872 0.373908251523972
5.33653846153846 0.369693160057068
5.58974358974359 0.31505098938942
5.85576923076923 0.369931697845459
6.13461538461539 0.358280509710312
6.42628205128205 0.365333735942841
6.73397435897436 0.324304491281509
7.05448717948718 0.326930791139603
7.38782051282051 0.320257514715195
7.74038461538461 0.317346185445786
8.10897435897436 0.341115444898605
8.49679487179487 0.284621685743332
8.90064102564103 0.318036496639252
9.32371794871795 0.327747315168381
9.76923076923077 0.310488522052765
10.2339743589744 0.326534807682037
10.7211538461538 0.303269296884537
11.2307692307692 0.315123647451401
11.7660256410256 0.27140411734581
12.3269230769231 0.295341670513153
12.9134615384615 0.293097406625748
13.5288461538462 0.286772638559341
14.1730769230769 0.270005434751511
14.849358974359 0.27742725610733
15.5544871794872 0.283681809902191
16.2948717948718 0.316442787647247
17.0705128205128 0.279329538345337
17.8846153846154 0.235471919178963
18.7371794871795 0.26641520857811
19.6282051282051 0.243001207709312
20.5641025641026 0.267534255981445
21.5416666666667 0.234942674636841
22.5673076923077 0.198397278785706
23.6442307692308 0.216548755764961
24.7692307692308 0.233836084604263
25.9487179487179 0.200447797775269
27.1826923076923 0.262571066617966
28.4775641025641 0.23483319580555
29.8333333333333 0.231047630310059
31.2564102564103 0.252297252416611
32.7435897435897 0.207586720585823
34.3044871794872 0.211200222373009
35.9358974358974 0.1795734167099
37.6474358974359 0.316597282886505
39.4391025641026 0.183264061808586
41.3173076923077 0.154708743095398
43.2852564102564 0.141129612922668
45.3461538461538 0.139020368456841
47.5064102564103 0.202856540679932
49.7692307692308 0.157290443778038
52.1378205128205 0.20845590531826
54.6217948717949 0.148975908756256
57.2211538461538 0.192067533731461
59.9455128205128 0.0964966416358948
62.8012820512821 0.116804197430611
65.7916666666667 0.0704308673739433
68.9230769230769 0.259082138538361
72.2051282051282 0.0705569311976433
75.6442307692308 0.0630605295300484
79.2467948717949 0.112299375236034
83.0192307692308 0.193100854754448
86.974358974359 0.288638919591904
91.1153846153846 0.0290741007775068
95.4519230769231 0.140903815627098
100 0.030736593529582
};
\addplot [, color2, opacity=0.6, mark=triangle*, mark size=0.5, mark options={solid,rotate=180}, only marks, forget plot]
table {%
1 0.46060848236084
1.04487179487179 0.444808214902878
1.09615384615385 0.507137715816498
1.1474358974359 0.535048425197601
1.20192307692308 0.607606112957001
1.25961538461538 0.625393748283386
1.32051282051282 0.594526052474976
1.38461538461538 0.595980286598206
1.44871794871795 0.605781495571136
1.51923076923077 0.550068616867065
1.58974358974359 0.568249046802521
1.66666666666667 0.582857668399811
1.74679487179487 0.532801270484924
1.83012820512821 0.546487510204315
1.91666666666667 0.494395583868027
2.00641025641026 0.558755338191986
2.1025641025641 0.496748030185699
2.20512820512821 0.473741441965103
2.30769230769231 0.458877772092819
2.41987179487179 0.508251309394836
2.53525641025641 0.433827489614487
2.65384615384615 0.483291000127792
2.78205128205128 0.420655250549316
2.91346153846154 0.482466220855713
3.05128205128205 0.418230295181274
3.19871794871795 0.409984111785889
3.34935897435897 0.450831413269043
3.50961538461538 0.420507490634918
3.67628205128205 0.421756744384766
3.8525641025641 0.403111279010773
4.03525641025641 0.412190824747086
4.2275641025641 0.439417332410812
4.42948717948718 0.413871973752975
4.64102564102564 0.38001748919487
4.86217948717949 0.37824273109436
5.09294871794872 0.405451387166977
5.33653846153846 0.365028500556946
5.58974358974359 0.392019897699356
5.85576923076923 0.404700428247452
6.13461538461539 0.351345807313919
6.42628205128205 0.351863384246826
6.73397435897436 0.346027225255966
7.05448717948718 0.343984931707382
7.38782051282051 0.33364936709404
7.74038461538461 0.307180345058441
8.10897435897436 0.336517035961151
8.49679487179487 0.317655086517334
8.90064102564103 0.352890729904175
9.32371794871795 0.321988254785538
9.76923076923077 0.320173352956772
10.2339743589744 0.288186103105545
10.7211538461538 0.331517487764359
11.2307692307692 0.313499361276627
11.7660256410256 0.290019989013672
12.3269230769231 0.3125339448452
12.9134615384615 0.298749178647995
13.5288461538462 0.272432059049606
14.1730769230769 0.286094427108765
14.849358974359 0.329167038202286
15.5544871794872 0.303139477968216
16.2948717948718 0.355805993080139
17.0705128205128 0.302943676710129
17.8846153846154 0.265824317932129
18.7371794871795 0.280080437660217
19.6282051282051 0.337733447551727
20.5641025641026 0.279076784849167
21.5416666666667 0.214953258633614
22.5673076923077 0.333964645862579
23.6442307692308 0.204595446586609
24.7692307692308 0.200422093272209
25.9487179487179 0.197743654251099
27.1826923076923 0.207462906837463
28.4775641025641 0.189585879445076
29.8333333333333 0.28414449095726
31.2564102564103 0.217070683836937
32.7435897435897 0.162597641348839
34.3044871794872 0.24205981194973
35.9358974358974 0.227291494607925
37.6474358974359 0.181064680218697
39.4391025641026 0.144889324903488
41.3173076923077 0.247471436858177
43.2852564102564 0.14010851085186
45.3461538461538 0.115776062011719
47.5064102564103 0.128872618079185
49.7692307692308 0.18106672167778
52.1378205128205 0.11106264591217
54.6217948717949 0.13419733941555
57.2211538461538 0.140472337603569
59.9455128205128 0.124132312834263
62.8012820512821 0.082518182694912
65.7916666666667 0.0564144365489483
68.9230769230769 0.152045249938965
72.2051282051282 0.0704030692577362
75.6442307692308 0.227775320410728
79.2467948717949 0.238144442439079
83.0192307692308 0.121952965855598
86.974358974359 0.22827835381031
91.1153846153846 0.085664801299572
95.4519230769231 0.0642145797610283
100 0.0812308415770531
};
\addplot [, color2, opacity=0.6, mark=triangle*, mark size=0.5, mark options={solid,rotate=180}, only marks, forget plot]
table {%
1 0.468360036611557
1.04487179487179 0.459622442722321
1.09615384615385 0.515558063983917
1.1474358974359 0.497603505849838
1.20192307692308 0.547643661499023
1.25961538461538 0.58798760175705
1.32051282051282 0.539084613323212
1.38461538461538 0.551168382167816
1.44871794871795 0.566796481609344
1.51923076923077 0.557830810546875
1.58974358974359 0.497636884450912
1.66666666666667 0.547896325588226
1.74679487179487 0.558273017406464
1.83012820512821 0.548366189002991
1.91666666666667 0.510311007499695
2.00641025641026 0.501809656620026
2.1025641025641 0.577862858772278
2.20512820512821 0.493613809347153
2.30769230769231 0.50023490190506
2.41987179487179 0.405930608510971
2.53525641025641 0.464221924543381
2.65384615384615 0.456874936819077
2.78205128205128 0.448304086923599
2.91346153846154 0.429339647293091
3.05128205128205 0.486745834350586
3.19871794871795 0.479446321725845
3.34935897435897 0.42621785402298
3.50961538461538 0.412496656179428
3.67628205128205 0.407979309558868
3.8525641025641 0.412470072507858
4.03525641025641 0.365743696689606
4.2275641025641 0.378519743680954
4.42948717948718 0.40040197968483
4.64102564102564 0.367256969213486
4.86217948717949 0.339912325143814
5.09294871794872 0.367052882909775
5.33653846153846 0.358173996210098
5.58974358974359 0.367408901453018
5.85576923076923 0.341219186782837
6.13461538461539 0.358297079801559
6.42628205128205 0.312251091003418
6.73397435897436 0.333145350217819
7.05448717948718 0.331956833600998
7.38782051282051 0.324038028717041
7.74038461538461 0.314362019300461
8.10897435897436 0.330395430326462
8.49679487179487 0.320378810167313
8.90064102564103 0.349906504154205
9.32371794871795 0.296192169189453
9.76923076923077 0.296834021806717
10.2339743589744 0.315752625465393
10.7211538461538 0.310085028409958
11.2307692307692 0.282522529363632
11.7660256410256 0.304887115955353
12.3269230769231 0.302248477935791
12.9134615384615 0.291464477777481
13.5288461538462 0.271158784627914
14.1730769230769 0.279025733470917
14.849358974359 0.260247200727463
15.5544871794872 0.25503346323967
16.2948717948718 0.310952454805374
17.0705128205128 0.342807769775391
17.8846153846154 0.260327816009521
18.7371794871795 0.265037506818771
19.6282051282051 0.293451458215714
20.5641025641026 0.276291459798813
21.5416666666667 0.284191846847534
22.5673076923077 0.285402357578278
23.6442307692308 0.263569802045822
24.7692307692308 0.323480278253555
25.9487179487179 0.296347707509995
27.1826923076923 0.271309971809387
28.4775641025641 0.245105370879173
29.8333333333333 0.276887029409409
31.2564102564103 0.229489430785179
32.7435897435897 0.268259912729263
34.3044871794872 0.266650915145874
35.9358974358974 0.244438454508781
37.6474358974359 0.326812595129013
39.4391025641026 0.244253799319267
41.3173076923077 0.263484328985214
43.2852564102564 0.29088231921196
45.3461538461538 0.258020490407944
47.5064102564103 0.268028020858765
49.7692307692308 0.261376291513443
52.1378205128205 0.232510849833488
54.6217948717949 0.223984345793724
57.2211538461538 0.207580640912056
59.9455128205128 0.273846238851547
62.8012820512821 0.0800492912530899
65.7916666666667 0.157258555293083
68.9230769230769 0.138146728277206
72.2051282051282 0.0791272297501564
75.6442307692308 0.230548575520515
79.2467948717949 0.124857783317566
83.0192307692308 0.058684166520834
86.974358974359 0.143108740448952
91.1153846153846 0.141209483146667
95.4519230769231 0.133550718426704
100 0.115915238857269
};
\addplot [, color2, opacity=0.6, mark=triangle*, mark size=0.5, mark options={solid,rotate=180}, only marks, forget plot]
table {%
1 0.487849056720734
1.04487179487179 0.503144562244415
1.09615384615385 0.507828652858734
1.1474358974359 0.582532465457916
1.20192307692308 0.652088820934296
1.25961538461538 0.654988288879395
1.32051282051282 0.639990031719208
1.38461538461538 0.626319944858551
1.44871794871795 0.617661654949188
1.51923076923077 0.666435837745667
1.58974358974359 0.567228376865387
1.66666666666667 0.6569744348526
1.74679487179487 0.564230263233185
1.83012820512821 0.598322808742523
1.91666666666667 0.550613880157471
2.00641025641026 0.55488920211792
2.1025641025641 0.522726535797119
2.20512820512821 0.538582623004913
2.30769230769231 0.507644712924957
2.41987179487179 0.529459893703461
2.53525641025641 0.433620274066925
2.65384615384615 0.444872617721558
2.78205128205128 0.462971299886703
2.91346153846154 0.46326956152916
3.05128205128205 0.45486307144165
3.19871794871795 0.475704580545425
3.34935897435897 0.429263323545456
3.50961538461538 0.413610845804214
3.67628205128205 0.450439184904099
3.8525641025641 0.354828089475632
4.03525641025641 0.393180459737778
4.2275641025641 0.333753049373627
4.42948717948718 0.40708801150322
4.64102564102564 0.382900089025497
4.86217948717949 0.38586750626564
5.09294871794872 0.395539522171021
5.33653846153846 0.330690383911133
5.58974358974359 0.400960832834244
5.85576923076923 0.367634236812592
6.13461538461539 0.321968853473663
6.42628205128205 0.337781518697739
6.73397435897436 0.322888731956482
7.05448717948718 0.332407683134079
7.38782051282051 0.339911073446274
7.74038461538461 0.338326275348663
8.10897435897436 0.35493466258049
8.49679487179487 0.284298151731491
8.90064102564103 0.281417816877365
9.32371794871795 0.27386799454689
9.76923076923077 0.294411033391953
10.2339743589744 0.276785790920258
10.7211538461538 0.29885670542717
11.2307692307692 0.296620339155197
11.7660256410256 0.266939371824265
12.3269230769231 0.300469040870667
12.9134615384615 0.277458757162094
13.5288461538462 0.225193783640862
14.1730769230769 0.331798225641251
14.849358974359 0.300455152988434
15.5544871794872 0.227322340011597
16.2948717948718 0.264253258705139
17.0705128205128 0.290343552827835
17.8846153846154 0.261224806308746
18.7371794871795 0.307387590408325
19.6282051282051 0.267526209354401
20.5641025641026 0.281106323003769
21.5416666666667 0.319106429815292
22.5673076923077 0.235638901591301
23.6442307692308 0.331540256738663
24.7692307692308 0.299479246139526
25.9487179487179 0.311059147119522
27.1826923076923 0.342276811599731
28.4775641025641 0.298169821500778
29.8333333333333 0.228612571954727
31.2564102564103 0.345184206962585
32.7435897435897 0.237868785858154
34.3044871794872 0.238088175654411
35.9358974358974 0.303478389978409
37.6474358974359 0.283593654632568
39.4391025641026 0.322584360837936
41.3173076923077 0.252563774585724
43.2852564102564 0.301423460245132
45.3461538461538 0.211159065365791
47.5064102564103 0.186297878623009
49.7692307692308 0.340202510356903
52.1378205128205 0.180685952305794
54.6217948717949 0.340726971626282
57.2211538461538 0.264386802911758
59.9455128205128 0.284801453351974
62.8012820512821 0.0957750603556633
65.7916666666667 0.415447622537613
68.9230769230769 0.128111734986305
72.2051282051282 0.262952238321304
75.6442307692308 0.21674445271492
79.2467948717949 0.293905168771744
83.0192307692308 0.130949959158897
86.974358974359 0.111615851521492
91.1153846153846 0.23984132707119
95.4519230769231 0.250548094511032
100 0.305393278598785
};
\end{axis}

\end{tikzpicture}

    \tikzexternaldisable
  \end{minipage}\hfill
  \begin{minipage}{0.50\linewidth}
    \centering
    % defines the pgfplots style "eigspacedefault"
\pgfkeys{/pgfplots/eigspacedefault/.style={
    width=1.03\linewidth,
    height=\goldenRatioInv*1.03*\linewidth,
    every axis plot/.append style={line width = 1pt},
    tick pos = left,
    ylabel near ticks,
    xlabel near ticks,
    xtick align = inside,
    ytick align = inside,
    legend cell align = left,
    legend columns = 1,
    legend pos = north east,
    legend style = {
      fill opacity = 0.9,
      text opacity = 1,
      font = \tiny,
      % column sep=0.1cm,
    },
    legend image post style={scale=2},
    xticklabel style = {font = \small},
    xlabel style = {font = \small},
    axis line style = {black},
    yticklabel style = {font = \small},
    ylabel style = {font = \small},
    title style = {font = \small},
    grid = major,
    grid style = {dashed}
  }
}

\pgfkeys{/pgfplots/eigspacedefaultapp/.style={
    eigspacedefault,
    height=0.6\linewidth,
    legend columns = 2,
  }
}

\pgfkeys{/pgfplots/eigspacenolegend/.style={
    legend image post style = {scale=0},
    legend style = {
      fill opacity = 0,
      draw opacity = 0,
      text opacity = 0,
      font = \small,
      at={(1, 1.025)},
      anchor=south east,
      column sep=0.25cm,
    },
  }
}
%%% Local Variables:
%%% mode: latex
%%% TeX-master: "../main"
%%% End:

    \pgfkeys{/pgfplots/zmystyle/.style={
        eigspacedefaultapp,
        legend columns = 3,
        eigspacenolegend,
      }}
    \tikzexternalenable
    \vspace{-3ex}
    % This file was created by tikzplotlib v0.9.7.
\begin{tikzpicture}

\definecolor{color0}{rgb}{0.274509803921569,0.6,0.564705882352941}
\definecolor{color1}{rgb}{0.870588235294118,0.623529411764706,0.0862745098039216}
\definecolor{color2}{rgb}{0.501960784313725,0.184313725490196,0.6}

\begin{axis}[
axis line style={white!10!black},
legend style={fill opacity=0.8, draw opacity=1, text opacity=1, at={(0.03,0.03)}, anchor=south west, draw=white!80!black},
log basis x={10},
tick pos=left,
xlabel={epoch (log scale)},
xmajorgrids,
xmin=0.794328234724281, xmax=125.892541179417,
xmode=log,
ylabel={overlap},
ymajorgrids,
ymin=-0.05, ymax=1.05,
zmystyle
]
\addplot [, white!10!black, dashed, forget plot]
table {%
0.794328234724281 1
125.892541179417 1
};
\addplot [, white!10!black, dashed, forget plot]
table {%
0.794328234724281 0
125.892541179417 0
};
\addplot [, color0, opacity=0.6, mark=diamond*, mark size=0.5, mark options={solid}, only marks]
table {%
1 0.909232914447784
1.04487179487179 0.923885762691498
1.09615384615385 0.909879624843597
1.1474358974359 0.755468964576721
1.20192307692308 0.752680599689484
1.25961538461538 0.766166865825653
1.32051282051282 0.73922061920166
1.38461538461538 0.738886773586273
1.44871794871795 0.733388125896454
1.51923076923077 0.689278602600098
1.58974358974359 0.656192719936371
1.66666666666667 0.736126363277435
1.74679487179487 0.668253719806671
1.83012820512821 0.700960218906403
1.91666666666667 0.703250050544739
2.00641025641026 0.673937320709229
2.1025641025641 0.678528785705566
2.20512820512821 0.658896625041962
2.30769230769231 0.620537281036377
2.41987179487179 0.652545869350433
2.53525641025641 0.555296957492828
2.65384615384615 0.633886516094208
2.78205128205128 0.619269132614136
2.91346153846154 0.622246205806732
3.05128205128205 0.594921112060547
3.19871794871795 0.598114788532257
3.34935897435897 0.570846498012543
3.50961538461538 0.537323117256165
3.67628205128205 0.5431187748909
3.8525641025641 0.527508676052094
4.03525641025641 0.550376117229462
4.2275641025641 0.511216282844543
4.42948717948718 0.517074406147003
4.64102564102564 0.512771248817444
4.86217948717949 0.556178212165833
5.09294871794872 0.551505565643311
5.33653846153846 0.484433978796005
5.58974358974359 0.494940012693405
5.85576923076923 0.427653461694717
6.13461538461539 0.500254511833191
6.42628205128205 0.423456400632858
6.73397435897436 0.436500608921051
7.05448717948718 0.48198327422142
7.38782051282051 0.502404808998108
7.74038461538461 0.385764688253403
8.10897435897436 0.45712685585022
8.49679487179487 0.424519628286362
8.90064102564103 0.456117242574692
9.32371794871795 0.418356478214264
9.76923076923077 0.376654177904129
10.2339743589744 0.403656095266342
10.7211538461538 0.395650953054428
11.2307692307692 0.372651904821396
11.7660256410256 0.357868045568466
12.3269230769231 0.427935421466827
12.9134615384615 0.410019874572754
13.5288461538462 0.345589846372604
14.1730769230769 0.353064686059952
14.849358974359 0.327596247196198
15.5544871794872 0.375583946704865
16.2948717948718 0.335086792707443
17.0705128205128 0.318727821111679
17.8846153846154 0.289265841245651
18.7371794871795 0.292518764734268
19.6282051282051 0.325301557779312
20.5641025641026 0.333605945110321
21.5416666666667 0.3563012778759
22.5673076923077 0.254142194986343
23.6442307692308 0.323976844549179
24.7692307692308 0.264278024435043
25.9487179487179 0.32735002040863
27.1826923076923 0.267452210187912
28.4775641025641 0.325715512037277
29.8333333333333 0.327790021896362
31.2564102564103 0.29289785027504
32.7435897435897 0.295085668563843
34.3044871794872 0.383501499891281
35.9358974358974 0.237196832895279
37.6474358974359 0.278297632932663
39.4391025641026 0.358555942773819
41.3173076923077 0.323584049940109
43.2852564102564 0.258672952651978
45.3461538461538 0.308058708906174
47.5064102564103 0.165110543370247
49.7692307692308 0.259628146886826
52.1378205128205 0.202272579073906
54.6217948717949 0.186924427747726
57.2211538461538 0.202682837843895
59.9455128205128 0.183598846197128
62.8012820512821 0.224639534950256
65.7916666666667 0.163621738553047
68.9230769230769 0.17722499370575
72.2051282051282 0.179859682917595
75.6442307692308 0.121666193008423
79.2467948717949 0.209512025117874
83.0192307692308 0.200167655944824
86.974358974359 0.185516774654388
91.1153846153846 0.216727644205093
95.4519230769231 0.220978781580925
100 0.138221636414528
};
\addlegendentry{sub 16, exact}
\addplot [, color0, opacity=0.6, mark=diamond*, mark size=0.5, mark options={solid}, only marks, forget plot]
table {%
1 0.876168251037598
1.04487179487179 0.91895979642868
1.09615384615385 0.934461712837219
1.1474358974359 0.916575610637665
1.20192307692308 0.931377708911896
1.25961538461538 0.806423127651215
1.32051282051282 0.731221139431
1.38461538461538 0.818828403949738
1.44871794871795 0.82328999042511
1.51923076923077 0.811212718486786
1.58974358974359 0.820380806922913
1.66666666666667 0.739875555038452
1.74679487179487 0.776599884033203
1.83012820512821 0.762060523033142
1.91666666666667 0.696026980876923
2.00641025641026 0.758289694786072
2.1025641025641 0.733550548553467
2.20512820512821 0.707034707069397
2.30769230769231 0.706031024456024
2.41987179487179 0.67691558599472
2.53525641025641 0.625911712646484
2.65384615384615 0.631565988063812
2.78205128205128 0.636468291282654
2.91346153846154 0.634806752204895
3.05128205128205 0.601399600505829
3.19871794871795 0.628953635692596
3.34935897435897 0.628905117511749
3.50961538461538 0.624261856079102
3.67628205128205 0.670965254306793
3.8525641025641 0.616125285625458
4.03525641025641 0.598204731941223
4.2275641025641 0.573052525520325
4.42948717948718 0.589631021022797
4.64102564102564 0.541348874568939
4.86217948717949 0.548676908016205
5.09294871794872 0.55664187669754
5.33653846153846 0.512182712554932
5.58974358974359 0.58887904882431
5.85576923076923 0.525392353534698
6.13461538461539 0.564314603805542
6.42628205128205 0.535809516906738
6.73397435897436 0.558244824409485
7.05448717948718 0.518901765346527
7.38782051282051 0.496192932128906
7.74038461538461 0.516369760036469
8.10897435897436 0.482204914093018
8.49679487179487 0.567737698554993
8.90064102564103 0.537470936775208
9.32371794871795 0.516681790351868
9.76923076923077 0.478830397129059
10.2339743589744 0.468300342559814
10.7211538461538 0.491769403219223
11.2307692307692 0.464001029729843
11.7660256410256 0.481923192739487
12.3269230769231 0.461114317178726
12.9134615384615 0.455042660236359
13.5288461538462 0.422095596790314
14.1730769230769 0.428727835416794
14.849358974359 0.472225666046143
15.5544871794872 0.405297189950943
16.2948717948718 0.36622342467308
17.0705128205128 0.515228092670441
17.8846153846154 0.431068897247314
18.7371794871795 0.373973578214645
19.6282051282051 0.411285012960434
20.5641025641026 0.454144567251205
21.5416666666667 0.491637527942657
22.5673076923077 0.432954519987106
23.6442307692308 0.409274011850357
24.7692307692308 0.470253199338913
25.9487179487179 0.449192434549332
27.1826923076923 0.445342928171158
28.4775641025641 0.454434245824814
29.8333333333333 0.498307377099991
31.2564102564103 0.422457456588745
32.7435897435897 0.356711834669113
34.3044871794872 0.37958636879921
35.9358974358974 0.395704180002213
37.6474358974359 0.300480544567108
39.4391025641026 0.379869133234024
41.3173076923077 0.421638697385788
43.2852564102564 0.283525556325912
45.3461538461538 0.391620129346848
47.5064102564103 0.335511058568954
49.7692307692308 0.230618789792061
52.1378205128205 0.185840830206871
54.6217948717949 0.162458166480064
57.2211538461538 0.289542466402054
59.9455128205128 0.187127754092216
62.8012820512821 0.169646054506302
65.7916666666667 0.285038709640503
68.9230769230769 0.317520380020142
72.2051282051282 0.227931410074234
75.6442307692308 0.263989418745041
79.2467948717949 0.27031746506691
83.0192307692308 0.194132000207901
86.974358974359 0.232591584324837
91.1153846153846 0.188429549336433
95.4519230769231 0.171600699424744
100 0.324702829122543
};
\addplot [, color0, opacity=0.6, mark=diamond*, mark size=0.5, mark options={solid}, only marks, forget plot]
table {%
1 0.893970668315887
1.04487179487179 0.922465145587921
1.09615384615385 0.938755810260773
1.1474358974359 0.930597007274628
1.20192307692308 0.833596229553223
1.25961538461538 0.811134159564972
1.32051282051282 0.769637584686279
1.38461538461538 0.785365879535675
1.44871794871795 0.739860236644745
1.51923076923077 0.78192013502121
1.58974358974359 0.742506325244904
1.66666666666667 0.772150456905365
1.74679487179487 0.742501258850098
1.83012820512821 0.771187007427216
1.91666666666667 0.74017608165741
2.00641025641026 0.687607228755951
2.1025641025641 0.730086982250214
2.20512820512821 0.646952748298645
2.30769230769231 0.65813797712326
2.41987179487179 0.642821490764618
2.53525641025641 0.640369057655334
2.65384615384615 0.634284019470215
2.78205128205128 0.599266827106476
2.91346153846154 0.662761569023132
3.05128205128205 0.655627071857452
3.19871794871795 0.645905435085297
3.34935897435897 0.632459580898285
3.50961538461538 0.622966110706329
3.67628205128205 0.605554759502411
3.8525641025641 0.630968809127808
4.03525641025641 0.608911752700806
4.2275641025641 0.62769889831543
4.42948717948718 0.591249883174896
4.64102564102564 0.56837671995163
4.86217948717949 0.581910610198975
5.09294871794872 0.552392482757568
5.33653846153846 0.540545701980591
5.58974358974359 0.546854615211487
5.85576923076923 0.548738479614258
6.13461538461539 0.565165817737579
6.42628205128205 0.564119935035706
6.73397435897436 0.476524591445923
7.05448717948718 0.555889427661896
7.38782051282051 0.533176720142365
7.74038461538461 0.520006716251373
8.10897435897436 0.482365518808365
8.49679487179487 0.544084966182709
8.90064102564103 0.4895900785923
9.32371794871795 0.509484469890594
9.76923076923077 0.485654920339584
10.2339743589744 0.499332427978516
10.7211538461538 0.455929964780807
11.2307692307692 0.42078772187233
11.7660256410256 0.473042696714401
12.3269230769231 0.472374200820923
12.9134615384615 0.499917507171631
13.5288461538462 0.412555754184723
14.1730769230769 0.406570881605148
14.849358974359 0.407441705465317
15.5544871794872 0.424358189105988
16.2948717948718 0.449702352285385
17.0705128205128 0.441735118627548
17.8846153846154 0.448992818593979
18.7371794871795 0.49288883805275
19.6282051282051 0.4867242872715
20.5641025641026 0.416908115148544
21.5416666666667 0.363399058580399
22.5673076923077 0.422906488180161
23.6442307692308 0.414324432611465
24.7692307692308 0.454640597105026
25.9487179487179 0.454439789056778
27.1826923076923 0.485217779874802
28.4775641025641 0.457606047391891
29.8333333333333 0.392002046108246
31.2564102564103 0.409357309341431
32.7435897435897 0.358967989683151
34.3044871794872 0.313063710927963
35.9358974358974 0.365739792585373
37.6474358974359 0.393070071935654
39.4391025641026 0.292976438999176
41.3173076923077 0.326159328222275
43.2852564102564 0.261739015579224
45.3461538461538 0.201719403266907
47.5064102564103 0.273561716079712
49.7692307692308 0.228086143732071
52.1378205128205 0.193606421351433
54.6217948717949 0.262405782938004
57.2211538461538 0.199555024504662
59.9455128205128 0.225373461842537
62.8012820512821 0.290073901414871
65.7916666666667 0.333071321249008
68.9230769230769 0.191870599985123
72.2051282051282 0.21045808494091
75.6442307692308 0.152266815304756
79.2467948717949 0.251813620328903
83.0192307692308 0.322749853134155
86.974358974359 0.1986293643713
91.1153846153846 0.117084562778473
95.4519230769231 0.242122337222099
100 0.114601634442806
};
\addplot [, color0, opacity=0.6, mark=diamond*, mark size=0.5, mark options={solid}, only marks, forget plot]
table {%
1 0.910792350769043
1.04487179487179 0.946472942829132
1.09615384615385 0.952463090419769
1.1474358974359 0.951622188091278
1.20192307692308 0.93939208984375
1.25961538461538 0.836513519287109
1.32051282051282 0.835577011108398
1.38461538461538 0.835086643695831
1.44871794871795 0.848892152309418
1.51923076923077 0.807782173156738
1.58974358974359 0.791922628879547
1.66666666666667 0.771913349628448
1.74679487179487 0.838438451290131
1.83012820512821 0.746255338191986
1.91666666666667 0.749547064304352
2.00641025641026 0.769061863422394
2.1025641025641 0.725242018699646
2.20512820512821 0.727359890937805
2.30769230769231 0.72566282749176
2.41987179487179 0.718861103057861
2.53525641025641 0.719398856163025
2.65384615384615 0.740060746669769
2.78205128205128 0.698009252548218
2.91346153846154 0.651191830635071
3.05128205128205 0.710459530353546
3.19871794871795 0.707655370235443
3.34935897435897 0.683420896530151
3.50961538461538 0.698959112167358
3.67628205128205 0.663612484931946
3.8525641025641 0.668070316314697
4.03525641025641 0.646026074886322
4.2275641025641 0.586627244949341
4.42948717948718 0.620739758014679
4.64102564102564 0.611144542694092
4.86217948717949 0.632784366607666
5.09294871794872 0.61199676990509
5.33653846153846 0.629556477069855
5.58974358974359 0.606973826885223
5.85576923076923 0.580711483955383
6.13461538461539 0.541855275630951
6.42628205128205 0.608972549438477
6.73397435897436 0.565792083740234
7.05448717948718 0.612600445747375
7.38782051282051 0.542777299880981
7.74038461538461 0.538200795650482
8.10897435897436 0.564289748668671
8.49679487179487 0.560433685779572
8.90064102564103 0.528335154056549
9.32371794871795 0.548975348472595
9.76923076923077 0.528235614299774
10.2339743589744 0.506702899932861
10.7211538461538 0.552429616451263
11.2307692307692 0.541772544384003
11.7660256410256 0.520183026790619
12.3269230769231 0.48952841758728
12.9134615384615 0.5571368932724
13.5288461538462 0.470293134450912
14.1730769230769 0.475500583648682
14.849358974359 0.491050004959106
15.5544871794872 0.440042495727539
16.2948717948718 0.448256403207779
17.0705128205128 0.466557949781418
17.8846153846154 0.516028046607971
18.7371794871795 0.458947658538818
19.6282051282051 0.39753395318985
20.5641025641026 0.447610378265381
21.5416666666667 0.452709525823593
22.5673076923077 0.440747946500778
23.6442307692308 0.441219091415405
24.7692307692308 0.34676057100296
25.9487179487179 0.439662665128708
27.1826923076923 0.370091885328293
28.4775641025641 0.427419394254684
29.8333333333333 0.350712269544601
31.2564102564103 0.410242944955826
32.7435897435897 0.332238644361496
34.3044871794872 0.359737157821655
35.9358974358974 0.363213390111923
37.6474358974359 0.411854267120361
39.4391025641026 0.373847484588623
41.3173076923077 0.335351765155792
43.2852564102564 0.329174607992172
45.3461538461538 0.298569202423096
47.5064102564103 0.302596151828766
49.7692307692308 0.242060706019402
52.1378205128205 0.33931165933609
54.6217948717949 0.257418155670166
57.2211538461538 0.276286423206329
59.9455128205128 0.237611562013626
62.8012820512821 0.190804809331894
65.7916666666667 0.279486745595932
68.9230769230769 0.237813666462898
72.2051282051282 0.218671351671219
75.6442307692308 0.292458981275558
79.2467948717949 0.303318917751312
83.0192307692308 0.297749757766724
86.974358974359 0.329286575317383
91.1153846153846 0.214231207966805
95.4519230769231 0.226495787501335
100 0.278015613555908
};
\addplot [, color0, opacity=0.6, mark=diamond*, mark size=0.5, mark options={solid}, only marks, forget plot]
table {%
1 0.886360585689545
1.04487179487179 0.930614948272705
1.09615384615385 0.944102466106415
1.1474358974359 0.928659915924072
1.20192307692308 0.923804938793182
1.25961538461538 0.773468971252441
1.32051282051282 0.810349106788635
1.38461538461538 0.77033144235611
1.44871794871795 0.787077605724335
1.51923076923077 0.773834109306335
1.58974358974359 0.758726894855499
1.66666666666667 0.768801808357239
1.74679487179487 0.778357207775116
1.83012820512821 0.736198365688324
1.91666666666667 0.773763000965118
2.00641025641026 0.724755525588989
2.1025641025641 0.688784778118134
2.20512820512821 0.713503479957581
2.30769230769231 0.640099823474884
2.41987179487179 0.672130763530731
2.53525641025641 0.66060346364975
2.65384615384615 0.727181375026703
2.78205128205128 0.690460443496704
2.91346153846154 0.65181428194046
3.05128205128205 0.682477056980133
3.19871794871795 0.675009906291962
3.34935897435897 0.636113107204437
3.50961538461538 0.657343089580536
3.67628205128205 0.621995747089386
3.8525641025641 0.64112776517868
4.03525641025641 0.628964066505432
4.2275641025641 0.622923195362091
4.42948717948718 0.600685238838196
4.64102564102564 0.599049687385559
4.86217948717949 0.628704488277435
5.09294871794872 0.558133006095886
5.33653846153846 0.587741553783417
5.58974358974359 0.616630673408508
5.85576923076923 0.589990437030792
6.13461538461539 0.619790554046631
6.42628205128205 0.583471179008484
6.73397435897436 0.586551487445831
7.05448717948718 0.587748050689697
7.38782051282051 0.584265828132629
7.74038461538461 0.571362435817719
8.10897435897436 0.55088597536087
8.49679487179487 0.565701961517334
8.90064102564103 0.593072891235352
9.32371794871795 0.51137113571167
9.76923076923077 0.538002014160156
10.2339743589744 0.496368378400803
10.7211538461538 0.527734935283661
11.2307692307692 0.514013826847076
11.7660256410256 0.489770948886871
12.3269230769231 0.456888288259506
12.9134615384615 0.472417563199997
13.5288461538462 0.553471565246582
14.1730769230769 0.435482710599899
14.849358974359 0.488814413547516
15.5544871794872 0.414948552846909
16.2948717948718 0.417721569538116
17.0705128205128 0.461767584085464
17.8846153846154 0.463971942663193
18.7371794871795 0.418496280908585
19.6282051282051 0.465823650360107
20.5641025641026 0.44148063659668
21.5416666666667 0.442857950925827
22.5673076923077 0.397028028964996
23.6442307692308 0.365816503763199
24.7692307692308 0.31719383597374
25.9487179487179 0.344769358634949
27.1826923076923 0.438289165496826
28.4775641025641 0.361378103494644
29.8333333333333 0.368985116481781
31.2564102564103 0.40246906876564
32.7435897435897 0.289604514837265
34.3044871794872 0.309597343206406
35.9358974358974 0.390245914459229
37.6474358974359 0.334154814481735
39.4391025641026 0.32131040096283
41.3173076923077 0.361025184392929
43.2852564102564 0.280573815107346
45.3461538461538 0.322834104299545
47.5064102564103 0.337794095277786
49.7692307692308 0.27132722735405
52.1378205128205 0.32023498415947
54.6217948717949 0.238322883844376
57.2211538461538 0.307823002338409
59.9455128205128 0.206722185015678
62.8012820512821 0.334238290786743
65.7916666666667 0.21309557557106
68.9230769230769 0.248896792531013
72.2051282051282 0.209451362490654
75.6442307692308 0.283519268035889
79.2467948717949 0.295351415872574
83.0192307692308 0.332443386316299
86.974358974359 0.203100830316544
91.1153846153846 0.345566719770432
95.4519230769231 0.305236130952835
100 0.404710978269577
};
\addplot [, color1, opacity=0.6, mark=square*, mark size=0.5, mark options={solid}, only marks]
table {%
1 0.880387127399445
1.04487179487179 0.851132214069366
1.09615384615385 0.945746898651123
1.1474358974359 0.911067605018616
1.20192307692308 0.89253443479538
1.25961538461538 0.797854542732239
1.32051282051282 0.852469444274902
1.38461538461538 0.834906697273254
1.44871794871795 0.846734642982483
1.51923076923077 0.812425136566162
1.58974358974359 0.826243698596954
1.66666666666667 0.838518261909485
1.74679487179487 0.842685341835022
1.83012820512821 0.886096954345703
1.91666666666667 0.86854076385498
2.00641025641026 0.910673320293427
2.1025641025641 0.789516270160675
2.20512820512821 0.771781146526337
2.30769230769231 0.810390889644623
2.41987179487179 0.842981159687042
2.53525641025641 0.877731323242188
2.65384615384615 0.758861660957336
2.78205128205128 0.792684733867645
2.91346153846154 0.817279458045959
3.05128205128205 0.78932398557663
3.19871794871795 0.800912022590637
3.34935897435897 0.73395174741745
3.50961538461538 0.759668469429016
3.67628205128205 0.812620103359222
3.8525641025641 0.755738079547882
4.03525641025641 0.742218434810638
4.2275641025641 0.748018443584442
4.42948717948718 0.713016450405121
4.64102564102564 0.692176938056946
4.86217948717949 0.790730237960815
5.09294871794872 0.750874042510986
5.33653846153846 0.727641999721527
5.58974358974359 0.734012067317963
5.85576923076923 0.723309457302094
6.13461538461539 0.741547763347626
6.42628205128205 0.731770813465118
6.73397435897436 0.70714944601059
7.05448717948718 0.722360074520111
7.38782051282051 0.712389945983887
7.74038461538461 0.69825679063797
8.10897435897436 0.678225159645081
8.49679487179487 0.634600937366486
8.90064102564103 0.687267422676086
9.32371794871795 0.714861929416656
9.76923076923077 0.653734505176544
10.2339743589744 0.763818442821503
10.7211538461538 0.664517521858215
11.2307692307692 0.587937414646149
11.7660256410256 0.567774713039398
12.3269230769231 0.594283759593964
12.9134615384615 0.636477530002594
13.5288461538462 0.574892938137054
14.1730769230769 0.642801105976105
14.849358974359 0.620545864105225
15.5544871794872 0.541880905628204
16.2948717948718 0.596259534358978
17.0705128205128 0.629538953304291
17.8846153846154 0.629744052886963
18.7371794871795 0.600418746471405
19.6282051282051 0.549253165721893
20.5641025641026 0.580366730690002
21.5416666666667 0.523881554603577
22.5673076923077 0.534558296203613
23.6442307692308 0.528196811676025
24.7692307692308 0.651303112506866
25.9487179487179 0.506357312202454
27.1826923076923 0.553501546382904
28.4775641025641 0.514539062976837
29.8333333333333 0.447673082351685
31.2564102564103 0.61303186416626
32.7435897435897 0.502297639846802
34.3044871794872 0.506312966346741
35.9358974358974 0.447196960449219
37.6474358974359 0.550845563411713
39.4391025641026 0.497750103473663
41.3173076923077 0.643068969249725
43.2852564102564 0.492440611124039
45.3461538461538 0.590391099452972
47.5064102564103 0.634836137294769
49.7692307692308 0.653424799442291
52.1378205128205 0.627911031246185
54.6217948717949 0.64363044500351
57.2211538461538 0.574983537197113
59.9455128205128 0.774812698364258
62.8012820512821 0.685828924179077
65.7916666666667 0.793271362781525
68.9230769230769 0.613888561725616
72.2051282051282 0.65961617231369
75.6442307692308 0.810623466968536
79.2467948717949 0.686509072780609
83.0192307692308 0.608023464679718
86.974358974359 0.881974637508392
91.1153846153846 0.688892066478729
95.4519230769231 0.855817794799805
100 0.760974586009979
};
\addlegendentry{mb 128, mc 1}
\addplot [, color1, opacity=0.6, mark=square*, mark size=0.5, mark options={solid}, only marks, forget plot]
table {%
1 0.841879367828369
1.04487179487179 0.927127063274384
1.09615384615385 0.937754154205322
1.1474358974359 0.874579071998596
1.20192307692308 0.930513978004456
1.25961538461538 0.866646766662598
1.32051282051282 0.831312000751495
1.38461538461538 0.849208533763885
1.44871794871795 0.84640234708786
1.51923076923077 0.864909827709198
1.58974358974359 0.867293775081635
1.66666666666667 0.848813652992249
1.74679487179487 0.822556614875793
1.83012820512821 0.855952084064484
1.91666666666667 0.831332623958588
2.00641025641026 0.822817146778107
2.1025641025641 0.854352951049805
2.20512820512821 0.785868465900421
2.30769230769231 0.806279599666595
2.41987179487179 0.803731739521027
2.53525641025641 0.830787479877472
2.65384615384615 0.82921040058136
2.78205128205128 0.822867214679718
2.91346153846154 0.826583564281464
3.05128205128205 0.7868532538414
3.19871794871795 0.739204525947571
3.34935897435897 0.760863900184631
3.50961538461538 0.756380021572113
3.67628205128205 0.731726884841919
3.8525641025641 0.796285569667816
4.03525641025641 0.801879107952118
4.2275641025641 0.730103433132172
4.42948717948718 0.709723949432373
4.64102564102564 0.722369968891144
4.86217948717949 0.717845737934113
5.09294871794872 0.691772758960724
5.33653846153846 0.765407979488373
5.58974358974359 0.670718848705292
5.85576923076923 0.691193521022797
6.13461538461539 0.679515361785889
6.42628205128205 0.674813866615295
6.73397435897436 0.637457311153412
7.05448717948718 0.64546525478363
7.38782051282051 0.677089214324951
7.74038461538461 0.586326718330383
8.10897435897436 0.682168900966644
8.49679487179487 0.663423240184784
8.90064102564103 0.644830346107483
9.32371794871795 0.62557053565979
9.76923076923077 0.650278270244598
10.2339743589744 0.565510511398315
10.7211538461538 0.648004710674286
11.2307692307692 0.647497117519379
11.7660256410256 0.642820000648499
12.3269230769231 0.644833564758301
12.9134615384615 0.653501689434052
13.5288461538462 0.654853701591492
14.1730769230769 0.593917787075043
14.849358974359 0.625298261642456
15.5544871794872 0.604493260383606
16.2948717948718 0.606814324855804
17.0705128205128 0.577826917171478
17.8846153846154 0.563007235527039
18.7371794871795 0.584407925605774
19.6282051282051 0.554803311824799
20.5641025641026 0.445368975400925
21.5416666666667 0.541028797626495
22.5673076923077 0.589394807815552
23.6442307692308 0.530459821224213
24.7692307692308 0.567716777324677
25.9487179487179 0.497638314962387
27.1826923076923 0.535468518733978
28.4775641025641 0.483051300048828
29.8333333333333 0.515324294567108
31.2564102564103 0.445199310779572
32.7435897435897 0.539105415344238
34.3044871794872 0.533484160900116
35.9358974358974 0.561452209949493
37.6474358974359 0.626488626003265
39.4391025641026 0.385825783014297
41.3173076923077 0.539093196392059
43.2852564102564 0.589161813259125
45.3461538461538 0.572323501110077
47.5064102564103 0.575833916664124
49.7692307692308 0.531717240810394
52.1378205128205 0.631384551525116
54.6217948717949 0.593941390514374
57.2211538461538 0.591166794300079
59.9455128205128 0.639524102210999
62.8012820512821 0.64361983537674
65.7916666666667 0.664969444274902
68.9230769230769 0.559683740139008
72.2051282051282 0.691854655742645
75.6442307692308 0.733657658100128
79.2467948717949 0.791212677955627
83.0192307692308 0.889229297637939
86.974358974359 0.817389667034149
91.1153846153846 0.94678795337677
95.4519230769231 0.69160783290863
100 0.90664541721344
};
\addplot [, color1, opacity=0.6, mark=square*, mark size=0.5, mark options={solid}, only marks, forget plot]
table {%
1 0.878505051136017
1.04487179487179 0.911553025245667
1.09615384615385 0.935479164123535
1.1474358974359 0.940314412117004
1.20192307692308 0.923645436763763
1.25961538461538 0.851247489452362
1.32051282051282 0.839400291442871
1.38461538461538 0.817616403102875
1.44871794871795 0.841689229011536
1.51923076923077 0.845323741436005
1.58974358974359 0.846799075603485
1.66666666666667 0.829333305358887
1.74679487179487 0.854536652565002
1.83012820512821 0.828670918941498
1.91666666666667 0.886336624622345
2.00641025641026 0.84687215089798
2.1025641025641 0.824959754943848
2.20512820512821 0.822316765785217
2.30769230769231 0.812655568122864
2.41987179487179 0.823741912841797
2.53525641025641 0.802851617336273
2.65384615384615 0.834020912647247
2.78205128205128 0.825894355773926
2.91346153846154 0.78577446937561
3.05128205128205 0.791261255741119
3.19871794871795 0.78691953420639
3.34935897435897 0.806600511074066
3.50961538461538 0.785962045192719
3.67628205128205 0.793426692485809
3.8525641025641 0.712728440761566
4.03525641025641 0.736608982086182
4.2275641025641 0.783704578876495
4.42948717948718 0.789443135261536
4.64102564102564 0.73138040304184
4.86217948717949 0.760695815086365
5.09294871794872 0.72733199596405
5.33653846153846 0.77374142408371
5.58974358974359 0.823859035968781
5.85576923076923 0.7359219789505
6.13461538461539 0.757125556468964
6.42628205128205 0.790037095546722
6.73397435897436 0.792262852191925
7.05448717948718 0.794249713420868
7.38782051282051 0.749410629272461
7.74038461538461 0.71256560087204
8.10897435897436 0.705088496208191
8.49679487179487 0.688521444797516
8.90064102564103 0.74530965089798
9.32371794871795 0.619983971118927
9.76923076923077 0.668689846992493
10.2339743589744 0.706608414649963
10.7211538461538 0.686945259571075
11.2307692307692 0.677223443984985
11.7660256410256 0.732347905635834
12.3269230769231 0.692470371723175
12.9134615384615 0.610925316810608
13.5288461538462 0.647472739219666
14.1730769230769 0.655998289585114
14.849358974359 0.658728122711182
15.5544871794872 0.599507033824921
16.2948717948718 0.657906711101532
17.0705128205128 0.628273606300354
17.8846153846154 0.631183445453644
18.7371794871795 0.610303819179535
19.6282051282051 0.516991794109344
20.5641025641026 0.495219051837921
21.5416666666667 0.583321690559387
22.5673076923077 0.580880641937256
23.6442307692308 0.484619438648224
24.7692307692308 0.5278639793396
25.9487179487179 0.581698894500732
27.1826923076923 0.529969692230225
28.4775641025641 0.55069774389267
29.8333333333333 0.402721256017685
31.2564102564103 0.528745293617249
32.7435897435897 0.43184620141983
34.3044871794872 0.536777913570404
35.9358974358974 0.503852307796478
37.6474358974359 0.468379706144333
39.4391025641026 0.556218981742859
41.3173076923077 0.560861229896545
43.2852564102564 0.524866104125977
45.3461538461538 0.499293297529221
47.5064102564103 0.578482449054718
49.7692307692308 0.705080449581146
52.1378205128205 0.537027657032013
54.6217948717949 0.47570464015007
57.2211538461538 0.574219405651093
59.9455128205128 0.522797524929047
62.8012820512821 0.609938323497772
65.7916666666667 0.750860869884491
68.9230769230769 0.601351082324982
72.2051282051282 0.700526356697083
75.6442307692308 0.735872387886047
79.2467948717949 0.823148727416992
83.0192307692308 0.793748199939728
86.974358974359 0.803036034107208
91.1153846153846 0.883446037769318
95.4519230769231 0.724227249622345
100 0.903059005737305
};
\addplot [, color1, opacity=0.6, mark=square*, mark size=0.5, mark options={solid}, only marks, forget plot]
table {%
1 0.867020606994629
1.04487179487179 0.905487477779388
1.09615384615385 0.943493187427521
1.1474358974359 0.86050283908844
1.20192307692308 0.898320019245148
1.25961538461538 0.925362706184387
1.32051282051282 0.863592088222504
1.38461538461538 0.857154846191406
1.44871794871795 0.840887486934662
1.51923076923077 0.827979862689972
1.58974358974359 0.869199693202972
1.66666666666667 0.865558624267578
1.74679487179487 0.789409279823303
1.83012820512821 0.804426193237305
1.91666666666667 0.887813031673431
2.00641025641026 0.84085750579834
2.1025641025641 0.852443873882294
2.20512820512821 0.833270072937012
2.30769230769231 0.802961051464081
2.41987179487179 0.811400592327118
2.53525641025641 0.781136512756348
2.65384615384615 0.848664879798889
2.78205128205128 0.799598395824432
2.91346153846154 0.845438182353973
3.05128205128205 0.818756878376007
3.19871794871795 0.781807720661163
3.34935897435897 0.743425905704498
3.50961538461538 0.814546883106232
3.67628205128205 0.74198454618454
3.8525641025641 0.780217707157135
4.03525641025641 0.774914920330048
4.2275641025641 0.735803067684174
4.42948717948718 0.74385803937912
4.64102564102564 0.72255939245224
4.86217948717949 0.773299336433411
5.09294871794872 0.737456440925598
5.33653846153846 0.769524216651917
5.58974358974359 0.708892405033112
5.85576923076923 0.733729422092438
6.13461538461539 0.753636240959167
6.42628205128205 0.714174568653107
6.73397435897436 0.690610408782959
7.05448717948718 0.672723889350891
7.38782051282051 0.764937698841095
7.74038461538461 0.658387124538422
8.10897435897436 0.746525943279266
8.49679487179487 0.737653255462646
8.90064102564103 0.704396188259125
9.32371794871795 0.622162997722626
9.76923076923077 0.689056575298309
10.2339743589744 0.634257674217224
10.7211538461538 0.65362161397934
11.2307692307692 0.601907908916473
11.7660256410256 0.609993398189545
12.3269230769231 0.669391870498657
12.9134615384615 0.56008905172348
13.5288461538462 0.624847173690796
14.1730769230769 0.642331779003143
14.849358974359 0.611996471881866
15.5544871794872 0.59939044713974
16.2948717948718 0.696066439151764
17.0705128205128 0.628548741340637
17.8846153846154 0.602256715297699
18.7371794871795 0.558450639247894
19.6282051282051 0.548576354980469
20.5641025641026 0.488404959440231
21.5416666666667 0.523762881755829
22.5673076923077 0.598495900630951
23.6442307692308 0.540113925933838
24.7692307692308 0.596057951450348
25.9487179487179 0.599088966846466
27.1826923076923 0.61542409658432
28.4775641025641 0.540221095085144
29.8333333333333 0.42018261551857
31.2564102564103 0.472074568271637
32.7435897435897 0.604727327823639
34.3044871794872 0.553451001644135
35.9358974358974 0.425581067800522
37.6474358974359 0.589859426021576
39.4391025641026 0.510362982749939
41.3173076923077 0.593199193477631
43.2852564102564 0.655445754528046
45.3461538461538 0.51564085483551
47.5064102564103 0.547426700592041
49.7692307692308 0.674781024456024
52.1378205128205 0.554384350776672
54.6217948717949 0.487564384937286
57.2211538461538 0.715740919113159
59.9455128205128 0.66802304983139
62.8012820512821 0.667199611663818
65.7916666666667 0.832424581050873
68.9230769230769 0.721846640110016
72.2051282051282 0.699018478393555
75.6442307692308 0.653936386108398
79.2467948717949 0.827097415924072
83.0192307692308 0.632597506046295
86.974358974359 0.74693775177002
91.1153846153846 0.78449159860611
95.4519230769231 0.873057067394257
100 0.915701329708099
};
\addplot [, color1, opacity=0.6, mark=square*, mark size=0.5, mark options={solid}, only marks, forget plot]
table {%
1 0.863184750080109
1.04487179487179 0.885529458522797
1.09615384615385 0.930442452430725
1.1474358974359 0.859042942523956
1.20192307692308 0.9259272813797
1.25961538461538 0.90226936340332
1.32051282051282 0.823753356933594
1.38461538461538 0.825037658214569
1.44871794871795 0.829842567443848
1.51923076923077 0.888482511043549
1.58974358974359 0.910128116607666
1.66666666666667 0.905763149261475
1.74679487179487 0.795850098133087
1.83012820512821 0.845826804637909
1.91666666666667 0.871029496192932
2.00641025641026 0.830865502357483
2.1025641025641 0.844721436500549
2.20512820512821 0.85896360874176
2.30769230769231 0.849537312984467
2.41987179487179 0.85366690158844
2.53525641025641 0.864025294780731
2.65384615384615 0.800161361694336
2.78205128205128 0.744588017463684
2.91346153846154 0.861502468585968
3.05128205128205 0.818996608257294
3.19871794871795 0.800134658813477
3.34935897435897 0.832971572875977
3.50961538461538 0.785261154174805
3.67628205128205 0.762819051742554
3.8525641025641 0.825892627239227
4.03525641025641 0.811792969703674
4.2275641025641 0.79277640581131
4.42948717948718 0.767267227172852
4.64102564102564 0.730199038982391
4.86217948717949 0.783983886241913
5.09294871794872 0.719151675701141
5.33653846153846 0.718031585216522
5.58974358974359 0.740192115306854
5.85576923076923 0.777187883853912
6.13461538461539 0.770930707454681
6.42628205128205 0.709399819374084
6.73397435897436 0.687454879283905
7.05448717948718 0.755893468856812
7.38782051282051 0.6915442943573
7.74038461538461 0.720788776874542
8.10897435897436 0.698723614215851
8.49679487179487 0.710529327392578
8.90064102564103 0.732994794845581
9.32371794871795 0.629063725471497
9.76923076923077 0.634564220905304
10.2339743589744 0.671368062496185
10.7211538461538 0.610738039016724
11.2307692307692 0.613214731216431
11.7660256410256 0.660027205944061
12.3269230769231 0.584761738777161
12.9134615384615 0.620519399642944
13.5288461538462 0.611364364624023
14.1730769230769 0.610468149185181
14.849358974359 0.613355994224548
15.5544871794872 0.612738072872162
16.2948717948718 0.635246753692627
17.0705128205128 0.506513357162476
17.8846153846154 0.585278928279877
18.7371794871795 0.565641582012177
19.6282051282051 0.594022214412689
20.5641025641026 0.63103061914444
21.5416666666667 0.59104597568512
22.5673076923077 0.517587542533875
23.6442307692308 0.616756916046143
24.7692307692308 0.593147039413452
25.9487179487179 0.542998969554901
27.1826923076923 0.655147612094879
28.4775641025641 0.560044229030609
29.8333333333333 0.618832111358643
31.2564102564103 0.466788679361343
32.7435897435897 0.609824359416962
34.3044871794872 0.478716343641281
35.9358974358974 0.565468311309814
37.6474358974359 0.63240259885788
39.4391025641026 0.522269904613495
41.3173076923077 0.532448947429657
43.2852564102564 0.498300164937973
45.3461538461538 0.514391243457794
47.5064102564103 0.606869399547577
49.7692307692308 0.554995954036713
52.1378205128205 0.536082565784454
54.6217948717949 0.437994807958603
57.2211538461538 0.640512943267822
59.9455128205128 0.670480370521545
62.8012820512821 0.667349517345428
65.7916666666667 0.589012622833252
68.9230769230769 0.53997278213501
72.2051282051282 0.660556077957153
75.6442307692308 0.665683567523956
79.2467948717949 0.659403383731842
83.0192307692308 0.60189962387085
86.974358974359 0.74744588136673
91.1153846153846 0.740732550621033
95.4519230769231 0.849364280700684
100 0.855970859527588
};
\addplot [, color2, opacity=0.6, mark=triangle*, mark size=0.5, mark options={solid,rotate=180}, only marks]
table {%
1 0.492817968130112
1.04487179487179 0.533709764480591
1.09615384615385 0.586886346340179
1.1474358974359 0.520990669727325
1.20192307692308 0.547443807125092
1.25961538461538 0.51116019487381
1.32051282051282 0.508819341659546
1.38461538461538 0.482669413089752
1.44871794871795 0.552517831325531
1.51923076923077 0.435748904943466
1.58974358974359 0.506152331829071
1.66666666666667 0.505156695842743
1.74679487179487 0.497194200754166
1.83012820512821 0.457457929849625
1.91666666666667 0.46180859208107
2.00641025641026 0.467998802661896
2.1025641025641 0.470081895589828
2.20512820512821 0.458179473876953
2.30769230769231 0.444158136844635
2.41987179487179 0.485709577798843
2.53525641025641 0.375255405902863
2.65384615384615 0.453163832426071
2.78205128205128 0.427120923995972
2.91346153846154 0.358714193105698
3.05128205128205 0.402286857366562
3.19871794871795 0.38916739821434
3.34935897435897 0.425388723611832
3.50961538461538 0.363735049962997
3.67628205128205 0.394412100315094
3.8525641025641 0.388162910938263
4.03525641025641 0.397320091724396
4.2275641025641 0.386041134595871
4.42948717948718 0.37510284781456
4.64102564102564 0.395732849836349
4.86217948717949 0.395678043365479
5.09294871794872 0.415099143981934
5.33653846153846 0.333293408155441
5.58974358974359 0.389273881912231
5.85576923076923 0.31717723608017
6.13461538461539 0.364820927381516
6.42628205128205 0.357950836420059
6.73397435897436 0.344491690397263
7.05448717948718 0.396526902914047
7.38782051282051 0.367167085409164
7.74038461538461 0.342705488204956
8.10897435897436 0.388826936483383
8.49679487179487 0.316042900085449
8.90064102564103 0.357025593519211
9.32371794871795 0.339292734861374
9.76923076923077 0.355223029851913
10.2339743589744 0.334224462509155
10.7211538461538 0.34380167722702
11.2307692307692 0.357537984848022
11.7660256410256 0.328603744506836
12.3269230769231 0.333012312650681
12.9134615384615 0.368622869253159
13.5288461538462 0.339934676885605
14.1730769230769 0.306144565343857
14.849358974359 0.312410175800323
15.5544871794872 0.326269090175629
16.2948717948718 0.324330866336823
17.0705128205128 0.294767916202545
17.8846153846154 0.263942629098892
18.7371794871795 0.269186109304428
19.6282051282051 0.290833294391632
20.5641025641026 0.29497891664505
21.5416666666667 0.358269721269608
22.5673076923077 0.23120990395546
23.6442307692308 0.326652377843857
24.7692307692308 0.23841105401516
25.9487179487179 0.297570794820786
27.1826923076923 0.261428534984589
28.4775641025641 0.315438896417618
29.8333333333333 0.300430566072464
31.2564102564103 0.262817472219467
32.7435897435897 0.286222457885742
34.3044871794872 0.359499901533127
35.9358974358974 0.210883781313896
37.6474358974359 0.279422491788864
39.4391025641026 0.325270026922226
41.3173076923077 0.288196623325348
43.2852564102564 0.274236291646957
45.3461538461538 0.293267458677292
47.5064102564103 0.159225359559059
49.7692307692308 0.223384693264961
52.1378205128205 0.184992864727974
54.6217948717949 0.162841245532036
57.2211538461538 0.150259628891945
59.9455128205128 0.181300804018974
62.8012820512821 0.201264649629593
65.7916666666667 0.147336915135384
68.9230769230769 0.181185439229012
72.2051282051282 0.150410637259483
75.6442307692308 0.12765283882618
79.2467948717949 0.248671963810921
83.0192307692308 0.208666637539864
86.974358974359 0.153627499938011
91.1153846153846 0.193192377686501
95.4519230769231 0.241883039474487
100 0.13374400138855
};
\addlegendentry{sub 16, mc 1}
\addplot [, color2, opacity=0.6, mark=triangle*, mark size=0.5, mark options={solid,rotate=180}, only marks, forget plot]
table {%
1 0.474336296319962
1.04487179487179 0.531555652618408
1.09615384615385 0.551536083221436
1.1474358974359 0.552497744560242
1.20192307692308 0.608869850635529
1.25961538461538 0.619980275630951
1.32051282051282 0.508068859577179
1.38461538461538 0.577738761901855
1.44871794871795 0.546042263507843
1.51923076923077 0.544329464435577
1.58974358974359 0.519405364990234
1.66666666666667 0.500891923904419
1.74679487179487 0.519209206104279
1.83012820512821 0.480200260877609
1.91666666666667 0.496424674987793
2.00641025641026 0.467425912618637
2.1025641025641 0.464299529790878
2.20512820512821 0.46529895067215
2.30769230769231 0.452146202325821
2.41987179487179 0.45824733376503
2.53525641025641 0.42003870010376
2.65384615384615 0.392435342073441
2.78205128205128 0.443376749753952
2.91346153846154 0.426578760147095
3.05128205128205 0.392664968967438
3.19871794871795 0.415402859449387
3.34935897435897 0.392608493566513
3.50961538461538 0.373419970273972
3.67628205128205 0.428230375051498
3.8525641025641 0.411505311727524
4.03525641025641 0.400649219751358
4.2275641025641 0.381123065948486
4.42948717948718 0.452527910470963
4.64102564102564 0.350554525852203
4.86217948717949 0.312230437994003
5.09294871794872 0.353469163179398
5.33653846153846 0.327762514352798
5.58974358974359 0.345420122146606
5.85576923076923 0.348952621221542
6.13461538461539 0.389540404081345
6.42628205128205 0.381421983242035
6.73397435897436 0.376848071813583
7.05448717948718 0.356772601604462
7.38782051282051 0.324276268482208
7.74038461538461 0.382111936807632
8.10897435897436 0.325270861387253
8.49679487179487 0.352264374494553
8.90064102564103 0.323938816785812
9.32371794871795 0.396765559911728
9.76923076923077 0.294809579849243
10.2339743589744 0.344403982162476
10.7211538461538 0.38504296541214
11.2307692307692 0.350764632225037
11.7660256410256 0.344622254371643
12.3269230769231 0.351129740476608
12.9134615384615 0.358426749706268
13.5288461538462 0.388691157102585
14.1730769230769 0.34651979804039
14.849358974359 0.380395501852036
15.5544871794872 0.277374356985092
16.2948717948718 0.244578287005424
17.0705128205128 0.385542571544647
17.8846153846154 0.298082768917084
18.7371794871795 0.319842308759689
19.6282051282051 0.335941880941391
20.5641025641026 0.377877086400986
21.5416666666667 0.361874610185623
22.5673076923077 0.370424002408981
23.6442307692308 0.390497595071793
24.7692307692308 0.427864283323288
25.9487179487179 0.34140482544899
27.1826923076923 0.390966862440109
28.4775641025641 0.374364286661148
29.8333333333333 0.45689782500267
31.2564102564103 0.310476094484329
32.7435897435897 0.2886021733284
34.3044871794872 0.340748965740204
35.9358974358974 0.350868314504623
37.6474358974359 0.310143560171127
39.4391025641026 0.385319083929062
41.3173076923077 0.298711031675339
43.2852564102564 0.233414888381958
45.3461538461538 0.355008989572525
47.5064102564103 0.277779102325439
49.7692307692308 0.230847984552383
52.1378205128205 0.200595140457153
54.6217948717949 0.195012450218201
57.2211538461538 0.265316963195801
59.9455128205128 0.199583500623703
62.8012820512821 0.171904191374779
65.7916666666667 0.222986459732056
68.9230769230769 0.22821281850338
72.2051282051282 0.161470606923103
75.6442307692308 0.289015263319016
79.2467948717949 0.299738645553589
83.0192307692308 0.222233489155769
86.974358974359 0.230799987912178
91.1153846153846 0.177903637290001
95.4519230769231 0.0941188707947731
100 0.340422302484512
};
\addplot [, color2, opacity=0.6, mark=triangle*, mark size=0.5, mark options={solid,rotate=180}, only marks, forget plot]
table {%
1 0.434135347604752
1.04487179487179 0.438068300485611
1.09615384615385 0.570744335651398
1.1474358974359 0.626971006393433
1.20192307692308 0.636229038238525
1.25961538461538 0.570429444313049
1.32051282051282 0.494844257831573
1.38461538461538 0.530172049999237
1.44871794871795 0.499217510223389
1.51923076923077 0.507899284362793
1.58974358974359 0.475315660238266
1.66666666666667 0.471138000488281
1.74679487179487 0.459126770496368
1.83012820512821 0.450203716754913
1.91666666666667 0.451359748840332
2.00641025641026 0.471253216266632
2.1025641025641 0.447458356618881
2.20512820512821 0.40611743927002
2.30769230769231 0.426639944314957
2.41987179487179 0.432644188404083
2.53525641025641 0.436910480260849
2.65384615384615 0.456370264291763
2.78205128205128 0.403673559427261
2.91346153846154 0.429541677236557
3.05128205128205 0.428914129734039
3.19871794871795 0.415391743183136
3.34935897435897 0.385743379592896
3.50961538461538 0.375365823507309
3.67628205128205 0.425610512495041
3.8525641025641 0.372903823852539
4.03525641025641 0.398030370473862
4.2275641025641 0.391590118408203
4.42948717948718 0.376275956630707
4.64102564102564 0.373080343008041
4.86217948717949 0.355205208063126
5.09294871794872 0.388328075408936
5.33653846153846 0.371477246284485
5.58974358974359 0.376310080289841
5.85576923076923 0.385658025741577
6.13461538461539 0.373646020889282
6.42628205128205 0.43796244263649
6.73397435897436 0.372688144445419
7.05448717948718 0.361011505126953
7.38782051282051 0.396529525518417
7.74038461538461 0.386414021253586
8.10897435897436 0.363293886184692
8.49679487179487 0.372029840946198
8.90064102564103 0.347440749406815
9.32371794871795 0.389452636241913
9.76923076923077 0.359799832105637
10.2339743589744 0.351677924394608
10.7211538461538 0.376603364944458
11.2307692307692 0.352645725011826
11.7660256410256 0.417850971221924
12.3269230769231 0.409605234861374
12.9134615384615 0.387071937322617
13.5288461538462 0.358011305332184
14.1730769230769 0.333217829465866
14.849358974359 0.349235057830811
15.5544871794872 0.331735819578171
16.2948717948718 0.361501932144165
17.0705128205128 0.378136783838272
17.8846153846154 0.350675642490387
18.7371794871795 0.422671794891357
19.6282051282051 0.435416430234909
20.5641025641026 0.34951788187027
21.5416666666667 0.310374081134796
22.5673076923077 0.370467185974121
23.6442307692308 0.353870004415512
24.7692307692308 0.395684391260147
25.9487179487179 0.351314663887024
27.1826923076923 0.386640280485153
28.4775641025641 0.417212963104248
29.8333333333333 0.321405380964279
31.2564102564103 0.349586009979248
32.7435897435897 0.296596199274063
34.3044871794872 0.283212721347809
35.9358974358974 0.324445724487305
37.6474358974359 0.313589155673981
39.4391025641026 0.237673684954643
41.3173076923077 0.286814361810684
43.2852564102564 0.210438445210457
45.3461538461538 0.180011615157127
47.5064102564103 0.262737989425659
49.7692307692308 0.219652682542801
52.1378205128205 0.188538461923599
54.6217948717949 0.244103953242302
57.2211538461538 0.206806808710098
59.9455128205128 0.211995229125023
62.8012820512821 0.249120458960533
65.7916666666667 0.302036464214325
68.9230769230769 0.160360142588615
72.2051282051282 0.209850072860718
75.6442307692308 0.152976900339127
79.2467948717949 0.253846734762192
83.0192307692308 0.28327876329422
86.974358974359 0.202574476599693
91.1153846153846 0.120467536151409
95.4519230769231 0.248918399214745
100 0.116668395698071
};
\addplot [, color2, opacity=0.6, mark=triangle*, mark size=0.5, mark options={solid,rotate=180}, only marks, forget plot]
table {%
1 0.423607736825943
1.04487179487179 0.462370693683624
1.09615384615385 0.593815505504608
1.1474358974359 0.647504270076752
1.20192307692308 0.682707905769348
1.25961538461538 0.552941024303436
1.32051282051282 0.556097328662872
1.38461538461538 0.558621644973755
1.44871794871795 0.582578718662262
1.51923076923077 0.529379069805145
1.58974358974359 0.519278049468994
1.66666666666667 0.526629447937012
1.74679487179487 0.505978286266327
1.83012820512821 0.539227247238159
1.91666666666667 0.527361512184143
2.00641025641026 0.526754319667816
2.1025641025641 0.518968760967255
2.20512820512821 0.476523697376251
2.30769230769231 0.487732142210007
2.41987179487179 0.479890823364258
2.53525641025641 0.500012576580048
2.65384615384615 0.502467155456543
2.78205128205128 0.511992394924164
2.91346153846154 0.494823694229126
3.05128205128205 0.490530550479889
3.19871794871795 0.441974878311157
3.34935897435897 0.474692016839981
3.50961538461538 0.434839010238647
3.67628205128205 0.425247579813004
3.8525641025641 0.43286520242691
4.03525641025641 0.438587188720703
4.2275641025641 0.376356810331345
4.42948717948718 0.398063123226166
4.64102564102564 0.404373645782471
4.86217948717949 0.388197511434555
5.09294871794872 0.438395231962204
5.33653846153846 0.349118649959564
5.58974358974359 0.39485427737236
5.85576923076923 0.349782109260559
6.13461538461539 0.325167924165726
6.42628205128205 0.35055536031723
6.73397435897436 0.458180338144302
7.05448717948718 0.380905240774155
7.38782051282051 0.328544497489929
7.74038461538461 0.344989567995071
8.10897435897436 0.299337446689606
8.49679487179487 0.375485450029373
8.90064102564103 0.41198655962944
9.32371794871795 0.377371907234192
9.76923076923077 0.351524978876114
10.2339743589744 0.414626985788345
10.7211538461538 0.345684379339218
11.2307692307692 0.375708609819412
11.7660256410256 0.379842430353165
12.3269230769231 0.378875881433487
12.9134615384615 0.379856258630753
13.5288461538462 0.319378286600113
14.1730769230769 0.32644909620285
14.849358974359 0.302377551794052
15.5544871794872 0.356998324394226
16.2948717948718 0.380316942930222
17.0705128205128 0.32493668794632
17.8846153846154 0.316669851541519
18.7371794871795 0.307313919067383
19.6282051282051 0.275677889585495
20.5641025641026 0.332906693220139
21.5416666666667 0.319981396198273
22.5673076923077 0.367391973733902
23.6442307692308 0.365154564380646
24.7692307692308 0.24313397705555
25.9487179487179 0.343766093254089
27.1826923076923 0.294465929269791
28.4775641025641 0.355682343244553
29.8333333333333 0.245614215731621
31.2564102564103 0.314234048128128
32.7435897435897 0.249794155359268
34.3044871794872 0.256534427404404
35.9358974358974 0.262380033731461
37.6474358974359 0.291126161813736
39.4391025641026 0.30424776673317
41.3173076923077 0.308029264211655
43.2852564102564 0.27175897359848
45.3461538461538 0.23081174492836
47.5064102564103 0.250052273273468
49.7692307692308 0.199483662843704
52.1378205128205 0.259782284498215
54.6217948717949 0.205923825502396
57.2211538461538 0.242417767643929
59.9455128205128 0.21500201523304
62.8012820512821 0.172721728682518
65.7916666666667 0.201203688979149
68.9230769230769 0.224298074841499
72.2051282051282 0.195315226912498
75.6442307692308 0.276840329170227
79.2467948717949 0.265990078449249
83.0192307692308 0.256078630685806
86.974358974359 0.292821645736694
91.1153846153846 0.181604757905006
95.4519230769231 0.196093007922173
100 0.262879103422165
};
\addplot [, color2, opacity=0.6, mark=triangle*, mark size=0.5, mark options={solid,rotate=180}, only marks, forget plot]
table {%
1 0.479237645864487
1.04487179487179 0.53229421377182
1.09615384615385 0.528994917869568
1.1474358974359 0.506504476070404
1.20192307692308 0.563952147960663
1.25961538461538 0.533517062664032
1.32051282051282 0.474269956350327
1.38461538461538 0.522387862205505
1.44871794871795 0.521273791790009
1.51923076923077 0.492540597915649
1.58974358974359 0.474770754575729
1.66666666666667 0.473383724689484
1.74679487179487 0.501995980739594
1.83012820512821 0.486024230718613
1.91666666666667 0.475933849811554
2.00641025641026 0.443197339773178
2.1025641025641 0.492752373218536
2.20512820512821 0.48458606004715
2.30769230769231 0.428913176059723
2.41987179487179 0.448388010263443
2.53525641025641 0.464550316333771
2.65384615384615 0.471744298934937
2.78205128205128 0.446346193552017
2.91346153846154 0.475303888320923
3.05128205128205 0.459503561258316
3.19871794871795 0.439862489700317
3.34935897435897 0.41966700553894
3.50961538461538 0.428878515958786
3.67628205128205 0.416507691144943
3.8525641025641 0.46666431427002
4.03525641025641 0.456033140420914
4.2275641025641 0.422855615615845
4.42948717948718 0.423860758543015
4.64102564102564 0.405951589345932
4.86217948717949 0.394399851560593
5.09294871794872 0.404429405927658
5.33653846153846 0.392322182655334
5.58974358974359 0.393885344266891
5.85576923076923 0.416658610105515
6.13461538461539 0.449294239282608
6.42628205128205 0.370849668979645
6.73397435897436 0.41604471206665
7.05448717948718 0.396148651838303
7.38782051282051 0.45703849196434
7.74038461538461 0.39750263094902
8.10897435897436 0.391539543867111
8.49679487179487 0.42792272567749
8.90064102564103 0.372155994176865
9.32371794871795 0.36951407790184
9.76923076923077 0.403700262308121
10.2339743589744 0.388724446296692
10.7211538461538 0.369096845388412
11.2307692307692 0.406954675912857
11.7660256410256 0.37892010807991
12.3269230769231 0.360497534275055
12.9134615384615 0.356354087591171
13.5288461538462 0.426518052816391
14.1730769230769 0.338335365056992
14.849358974359 0.357535690069199
15.5544871794872 0.350909262895584
16.2948717948718 0.348464816808701
17.0705128205128 0.376914590597153
17.8846153846154 0.353987187147141
18.7371794871795 0.343839138746262
19.6282051282051 0.389762252569199
20.5641025641026 0.373763144016266
21.5416666666667 0.350327968597412
22.5673076923077 0.326457053422928
23.6442307692308 0.289544343948364
24.7692307692308 0.275632590055466
25.9487179487179 0.291604280471802
27.1826923076923 0.351727455854416
28.4775641025641 0.309516549110413
29.8333333333333 0.318853288888931
31.2564102564103 0.354357570409775
32.7435897435897 0.237384721636772
34.3044871794872 0.255122393369675
35.9358974358974 0.360491901636124
37.6474358974359 0.297423422336578
39.4391025641026 0.277607589960098
41.3173076923077 0.32546266913414
43.2852564102564 0.274338334798813
45.3461538461538 0.307046085596085
47.5064102564103 0.336239457130432
49.7692307692308 0.228427082300186
52.1378205128205 0.301088571548462
54.6217948717949 0.234789773821831
57.2211538461538 0.295726239681244
59.9455128205128 0.187393814325333
62.8012820512821 0.314486116170883
65.7916666666667 0.207019358873367
68.9230769230769 0.190162807703018
72.2051282051282 0.202738791704178
75.6442307692308 0.249486520886421
79.2467948717949 0.306935459375381
83.0192307692308 0.311731785535812
86.974358974359 0.214546233415604
91.1153846153846 0.271374672651291
95.4519230769231 0.298299193382263
100 0.396949082612991
};
\end{axis}

\end{tikzpicture}

    \tikzexternaldisable
  \end{minipage}
\end{subfigure}

\begin{subfigure}[t]{\linewidth}
  \centering
  \caption{\cifarten \resnetthirtytwo}
  \begin{minipage}{0.50\linewidth}
    \centering
    % defines the pgfplots style "eigspacedefault"
\pgfkeys{/pgfplots/eigspacedefault/.style={
    width=1.03\linewidth,
    height=\goldenRatioInv*1.03*\linewidth,
    every axis plot/.append style={line width = 1pt},
    tick pos = left,
    ylabel near ticks,
    xlabel near ticks,
    xtick align = inside,
    ytick align = inside,
    legend cell align = left,
    legend columns = 1,
    legend pos = north east,
    legend style = {
      fill opacity = 0.9,
      text opacity = 1,
      font = \tiny,
      % column sep=0.1cm,
    },
    legend image post style={scale=2},
    xticklabel style = {font = \small},
    xlabel style = {font = \small},
    axis line style = {black},
    yticklabel style = {font = \small},
    ylabel style = {font = \small},
    title style = {font = \small},
    grid = major,
    grid style = {dashed}
  }
}

\pgfkeys{/pgfplots/eigspacedefaultapp/.style={
    eigspacedefault,
    height=0.6\linewidth,
    legend columns = 2,
  }
}

\pgfkeys{/pgfplots/eigspacenolegend/.style={
    legend image post style = {scale=0},
    legend style = {
      fill opacity = 0,
      draw opacity = 0,
      text opacity = 0,
      font = \small,
      at={(1, 1.025)},
      anchor=south east,
      column sep=0.25cm,
    },
  }
}
%%% Local Variables:
%%% mode: latex
%%% TeX-master: "../main"
%%% End:

    \pgfkeys{/pgfplots/zmystyle/.style={
        eigspacedefaultapp,
        legend columns = 3,
        eigspacenolegend,
      }}
    \tikzexternalenable
    \vspace{-3ex}
    % This file was created by tikzplotlib v0.9.7.
\begin{tikzpicture}

\definecolor{color0}{rgb}{0.274509803921569,0.6,0.564705882352941}
\definecolor{color1}{rgb}{0.870588235294118,0.623529411764706,0.0862745098039216}
\definecolor{color2}{rgb}{0.501960784313725,0.184313725490196,0.6}

\begin{axis}[
axis line style={white!10!black},
legend style={fill opacity=0.8, draw opacity=1, text opacity=1, at={(0.03,0.03)}, anchor=south west, draw=white!80!black},
log basis x={10},
tick pos=left,
xlabel={epoch (log scale)},
xmajorgrids,
xmin=0.812908354450748, xmax=232.782414148795,
xmode=log,
ylabel={overlap},
ymajorgrids,
ymin=-0.05, ymax=1.05,
zmystyle
]
\addplot [, white!10!black, dashed, forget plot]
table {%
0.812908354450747 1
232.782414148795 1
};
\addplot [, white!10!black, dashed, forget plot]
table {%
0.812908354450747 0
232.782414148795 0
};
\addplot [, color0, opacity=0.6, mark=diamond*, mark size=0.5, mark options={solid}, only marks]
table {%
1 nan
1.05128205128205 0.786806702613831
1.10897435897436 0.700974583625793
1.16987179487179 0.880908012390137
1.23076923076923 0.828021049499512
1.29807692307692 0.865706086158752
1.36858974358974 0.840286672115326
1.44230769230769 0.839321255683899
1.51923076923077 0.84922868013382
1.6025641025641 0.825080513954163
1.68910256410256 0.83488941192627
1.77884615384615 0.818330585956573
1.875 0.857167899608612
1.9775641025641 0.847613453865051
2.08333333333333 0.798644244670868
2.19551282051282 0.81528502702713
2.31410256410256 0.805564045906067
2.43910256410256 0.761670053005219
2.57051282051282 0.780188858509064
2.70833333333333 0.611747920513153
2.8525641025641 0.675550162792206
3.00641025641026 0.698321163654327
3.16987179487179 0.662490665912628
3.33974358974359 0.732862889766693
3.51923076923077 0.652845561504364
3.70833333333333 0.661060333251953
3.91025641025641 0.736571311950684
4.11858974358974 0.665007829666138
4.34294871794872 0.616171777248383
4.57692307692308 0.65239429473877
4.82371794871795 0.668202877044678
5.08333333333333 0.611065745353699
5.35576923076923 0.7247234582901
5.64423076923077 0.605319917201996
5.94871794871795 0.624130189418793
6.26923076923077 0.627655982971191
6.60576923076923 0.553628444671631
6.96153846153846 0.586081206798553
7.33653846153846 0.574766457080841
7.73397435897436 0.629219353199005
8.15064102564103 0.570189893245697
8.58974358974359 0.550879657268524
9.05128205128205 0.587928295135498
9.53846153846154 0.504767656326294
10.0512820512821 0.563534677028656
10.5929487179487 0.564829349517822
11.1634615384615 0.532373368740082
11.7660256410256 0.517597198486328
12.400641025641 0.476776033639908
13.0673076923077 0.599408805370331
13.7724358974359 0.464790552854538
14.5128205128205 0.512352228164673
15.2948717948718 0.484028428792953
16.1185897435897 0.479222625494003
16.9871794871795 0.517431437969208
17.900641025641 0.510924279689789
18.8653846153846 0.453106611967087
19.8814102564103 0.448055267333984
20.9519230769231 0.42202290892601
22.0801282051282 0.476025015115738
23.2692307692308 0.38262739777565
24.5224358974359 0.474443197250366
25.8429487179487 0.422714620828629
27.2371794871795 0.393973857164383
28.7019230769231 0.398205012083054
30.25 0.521466493606567
31.8782051282051 0.379068464040756
33.5961538461538 0.426438868045807
35.4038461538462 0.438194006681442
37.3108974358974 0.403373003005981
39.3205128205128 0.415454000234604
41.4391025641026 0.376313358545303
43.6698717948718 0.333949655294418
46.0224358974359 0.341855138540268
48.5 0.39868700504303
51.1121794871795 0.280263155698776
53.8653846153846 0.339340150356293
56.7660256410256 0.356146901845932
59.8237179487179 0.39173835515976
63.0448717948718 0.348429292440414
66.4391025641026 0.268025547266006
70.0192307692308 0.405853182077408
73.7884615384615 0.361953556537628
77.7628205128205 0.282596737146378
81.9519230769231 0.496977001428604
86.3653846153846 0.299012809991837
91.0160256410256 0.353487700223923
95.9166666666667 0.272731214761734
101.083333333333 0.48134708404541
106.525641025641 0.508853733539581
112.262820512821 0.21975389122963
118.310897435897 0.410986036062241
124.682692307692 0.261111795902252
131.397435897436 0.253098607063293
138.474358974359 0.211384639143944
145.929487179487 0.374904155731201
153.788461538462 0.362356245517731
162.070512820513 0.280943959951401
170.801282051282 0.454894602298737
180 0.388313770294189
};
\addlegendentry{sub 16, exact}
\addplot [, color0, opacity=0.6, mark=diamond*, mark size=0.5, mark options={solid}, only marks, forget plot]
table {%
1 nan
1.05128205128205 0.775372803211212
1.10897435897436 0.774005353450775
1.16987179487179 0.801693916320801
1.23076923076923 0.789825618267059
1.29807692307692 0.849802494049072
1.36858974358974 0.897225320339203
1.44230769230769 0.83552211523056
1.51923076923077 0.854391396045685
1.6025641025641 0.864746570587158
1.68910256410256 0.75647109746933
1.77884615384615 0.759904205799103
1.875 0.7872314453125
1.9775641025641 0.732275903224945
2.08333333333333 0.755435585975647
2.19551282051282 0.791303336620331
2.31410256410256 0.760285675525665
2.43910256410256 0.723615109920502
2.57051282051282 0.691653072834015
2.70833333333333 0.701993882656097
2.8525641025641 0.730019986629486
3.00641025641026 0.715771019458771
3.16987179487179 0.726961433887482
3.33974358974359 0.753287136554718
3.51923076923077 0.729753911495209
3.70833333333333 0.677748084068298
3.91025641025641 0.810456871986389
4.11858974358974 0.70648330450058
4.34294871794872 0.699753284454346
4.57692307692308 0.723564743995667
4.82371794871795 0.750298023223877
5.08333333333333 0.709575593471527
5.35576923076923 0.749051988124847
5.64423076923077 0.724649906158447
5.94871794871795 0.674677073955536
6.26923076923077 0.668367624282837
6.60576923076923 0.671224117279053
6.96153846153846 0.650358378887177
7.33653846153846 0.643141746520996
7.73397435897436 0.607226669788361
8.15064102564103 0.646840691566467
8.58974358974359 0.600280821323395
9.05128205128205 0.594422519207001
9.53846153846154 0.662885963916779
10.0512820512821 0.581959307193756
10.5929487179487 0.630197465419769
11.1634615384615 0.631017923355103
11.7660256410256 0.558700561523438
12.400641025641 0.577786386013031
13.0673076923077 0.554883420467377
13.7724358974359 0.536742925643921
14.5128205128205 0.564975917339325
15.2948717948718 0.520958602428436
16.1185897435897 0.543540477752686
16.9871794871795 0.577799022197723
17.900641025641 0.571280300617218
18.8653846153846 0.534522473812103
19.8814102564103 0.459122717380524
20.9519230769231 0.491148382425308
22.0801282051282 0.46566715836525
23.2692307692308 0.462036579847336
24.5224358974359 0.492426604032516
25.8429487179487 0.519119083881378
27.2371794871795 0.545294046401978
28.7019230769231 0.516684949398041
30.25 0.542794287204742
31.8782051282051 0.421546667814255
33.5961538461538 0.410704612731934
35.4038461538462 0.449551969766617
37.3108974358974 0.420071333646774
39.3205128205128 0.37591677904129
41.4391025641026 0.472232729196548
43.6698717948718 0.413329601287842
46.0224358974359 0.347829103469849
48.5 0.341673851013184
51.1121794871795 0.385172963142395
53.8653846153846 0.293686002492905
56.7660256410256 0.303456008434296
59.8237179487179 0.397027462720871
63.0448717948718 0.347458451986313
66.4391025641026 0.34926700592041
70.0192307692308 0.421336859464645
73.7884615384615 0.293829679489136
77.7628205128205 0.240166783332825
81.9519230769231 0.285500377416611
86.3653846153846 0.486273437738419
91.0160256410256 0.334366083145142
95.9166666666667 0.251307845115662
101.083333333333 0.25835919380188
106.525641025641 0.247062802314758
112.262820512821 0.432202070951462
118.310897435897 0.323935985565186
124.682692307692 0.220913335680962
131.397435897436 0.293208330869675
138.474358974359 0.18596750497818
145.929487179487 0.28526583313942
153.788461538462 0.302357167005539
162.070512820513 0.205192610621452
170.801282051282 0.139693900942802
180 0.445288747549057
};
\addplot [, color0, opacity=0.6, mark=diamond*, mark size=0.5, mark options={solid}, only marks, forget plot]
table {%
1 nan
1.05128205128205 0.829406261444092
1.10897435897436 0.573716998100281
1.16987179487179 0.565604031085968
1.23076923076923 0.659210681915283
1.29807692307692 0.677785098552704
1.36858974358974 0.621673822402954
1.44230769230769 0.696548700332642
1.51923076923077 0.723314106464386
1.6025641025641 0.65492308139801
1.68910256410256 0.609032154083252
1.77884615384615 0.698820888996124
1.875 0.793082237243652
1.9775641025641 0.70573216676712
2.08333333333333 0.643017411231995
2.19551282051282 0.741144359111786
2.31410256410256 0.71509176492691
2.43910256410256 0.695260345935822
2.57051282051282 0.710782945156097
2.70833333333333 0.665231227874756
2.8525641025641 0.695142924785614
3.00641025641026 0.645451664924622
3.16987179487179 0.698727130889893
3.33974358974359 0.640647232532501
3.51923076923077 0.699214398860931
3.70833333333333 0.718547999858856
3.91025641025641 0.744778394699097
4.11858974358974 0.724320828914642
4.34294871794872 0.686227679252625
4.57692307692308 0.720291197299957
4.82371794871795 0.711628437042236
5.08333333333333 0.677840888500214
5.35576923076923 0.671223938465118
5.64423076923077 0.65051281452179
5.94871794871795 0.619163632392883
6.26923076923077 0.57956725358963
6.60576923076923 0.600234508514404
6.96153846153846 0.621982991695404
7.33653846153846 0.580846011638641
7.73397435897436 0.588401019573212
8.15064102564103 0.547604262828827
8.58974358974359 0.570439040660858
9.05128205128205 0.558831393718719
9.53846153846154 0.597730576992035
10.0512820512821 0.498805046081543
10.5929487179487 0.565400540828705
11.1634615384615 0.498547375202179
11.7660256410256 0.576973736286163
12.400641025641 0.557663381099701
13.0673076923077 0.493927001953125
13.7724358974359 0.458154886960983
14.5128205128205 0.500651299953461
15.2948717948718 0.475159794092178
16.1185897435897 0.452559471130371
16.9871794871795 0.502930819988251
17.900641025641 0.460674256086349
18.8653846153846 0.477005571126938
19.8814102564103 0.467777341604233
20.9519230769231 0.380398958921432
22.0801282051282 0.391797691583633
23.2692307692308 0.387128561735153
24.5224358974359 0.355058640241623
25.8429487179487 0.385707527399063
27.2371794871795 0.34741884469986
28.7019230769231 0.325432777404785
30.25 0.336216509342194
31.8782051282051 0.278728187084198
33.5961538461538 0.304745525121689
35.4038461538462 0.277730286121368
37.3108974358974 0.288374811410904
39.3205128205128 0.295837730169296
41.4391025641026 0.372715324163437
43.6698717948718 0.232433423399925
46.0224358974359 0.253463834524155
48.5 0.330917954444885
51.1121794871795 0.276942729949951
53.8653846153846 0.253028839826584
56.7660256410256 0.220781564712524
59.8237179487179 0.229789778590202
63.0448717948718 0.221022948622704
66.4391025641026 0.170679911971092
70.0192307692308 0.175333619117737
73.7884615384615 0.222132727503777
77.7628205128205 0.236815497279167
81.9519230769231 0.1810312718153
86.3653846153846 0.234818503260612
91.0160256410256 0.261010080575943
95.9166666666667 0.166516557335854
101.083333333333 0.247047588229179
106.525641025641 0.14454410970211
112.262820512821 0.214605495333672
118.310897435897 0.127466678619385
124.682692307692 0.163673281669617
131.397435897436 0.195012718439102
138.474358974359 0.146842554211617
145.929487179487 0.200461745262146
153.788461538462 0.127883121371269
162.070512820513 0.227205082774162
170.801282051282 0.210162505507469
180 0.210531905293465
};
\addplot [, color0, opacity=0.6, mark=diamond*, mark size=0.5, mark options={solid}, only marks, forget plot]
table {%
1 nan
1.05128205128205 0.829119205474854
1.10897435897436 0.657194554805756
1.16987179487179 0.770666062831879
1.23076923076923 0.818994164466858
1.29807692307692 0.726982295513153
1.36858974358974 0.782657265663147
1.44230769230769 0.756235957145691
1.51923076923077 0.734261512756348
1.6025641025641 0.797254681587219
1.68910256410256 0.710923612117767
1.77884615384615 0.745464384555817
1.875 0.807281911373138
1.9775641025641 0.759232223033905
2.08333333333333 0.665154933929443
2.19551282051282 0.755708158016205
2.31410256410256 0.694888949394226
2.43910256410256 0.731908142566681
2.57051282051282 0.709702968597412
2.70833333333333 0.797450244426727
2.8525641025641 0.672591388225555
3.00641025641026 0.722034156322479
3.16987179487179 0.736404895782471
3.33974358974359 0.697915494441986
3.51923076923077 0.742741584777832
3.70833333333333 0.746174275875092
3.91025641025641 0.786363422870636
4.11858974358974 0.765238046646118
4.34294871794872 0.733445107936859
4.57692307692308 0.75815361738205
4.82371794871795 0.750365197658539
5.08333333333333 0.643351674079895
5.35576923076923 0.795629501342773
5.64423076923077 0.768994271755219
5.94871794871795 0.646649479866028
6.26923076923077 0.629236817359924
6.60576923076923 0.684715032577515
6.96153846153846 0.642951846122742
7.33653846153846 0.720947444438934
7.73397435897436 0.590232193470001
8.15064102564103 0.593414604663849
8.58974358974359 0.67114645242691
9.05128205128205 0.600878775119781
9.53846153846154 0.570074558258057
10.0512820512821 0.596939504146576
10.5929487179487 0.659844398498535
11.1634615384615 0.611972033977509
11.7660256410256 0.604771792888641
12.400641025641 0.622288644313812
13.0673076923077 0.628147423267365
13.7724358974359 0.57127183675766
14.5128205128205 0.532866299152374
15.2948717948718 0.549735248088837
16.1185897435897 0.584459006786346
16.9871794871795 0.541984081268311
17.900641025641 0.471041023731232
18.8653846153846 0.451338678598404
19.8814102564103 0.527599930763245
20.9519230769231 0.472607433795929
22.0801282051282 0.437922865152359
23.2692307692308 0.459335178136826
24.5224358974359 0.484477013349533
25.8429487179487 0.399110615253448
27.2371794871795 0.40915510058403
28.7019230769231 0.417366981506348
30.25 0.362803339958191
31.8782051282051 0.446218580007553
33.5961538461538 0.467807918787003
35.4038461538462 0.339508950710297
37.3108974358974 0.340012311935425
39.3205128205128 0.376047939062119
41.4391025641026 0.409245103597641
43.6698717948718 0.476523399353027
46.0224358974359 0.31098935008049
48.5 0.437600284814835
51.1121794871795 0.332188367843628
53.8653846153846 0.310879766941071
56.7660256410256 0.282694876194
59.8237179487179 0.203688502311707
63.0448717948718 0.24633252620697
66.4391025641026 0.283303558826447
70.0192307692308 0.342228502035141
73.7884615384615 0.287705183029175
77.7628205128205 0.369214296340942
81.9519230769231 0.20264644920826
86.3653846153846 0.267844587564468
91.0160256410256 0.16618375480175
95.9166666666667 0.269058376550674
101.083333333333 0.308255434036255
106.525641025641 0.302396774291992
112.262820512821 0.295313119888306
118.310897435897 0.223154738545418
124.682692307692 0.285332053899765
131.397435897436 0.200529009103775
138.474358974359 0.249139741063118
145.929487179487 0.340974181890488
153.788461538462 0.455497652292252
162.070512820513 0.224236443638802
170.801282051282 0.271543115377426
180 0.139676660299301
};
\addplot [, color0, opacity=0.6, mark=diamond*, mark size=0.5, mark options={solid}, only marks, forget plot]
table {%
1 nan
1.05128205128205 0.824510037899017
1.10897435897436 0.716818392276764
1.16987179487179 0.793766438961029
1.23076923076923 0.885192692279816
1.29807692307692 0.766349911689758
1.36858974358974 0.916901588439941
1.44230769230769 0.798832595348358
1.51923076923077 0.811773598194122
1.6025641025641 0.75998318195343
1.68910256410256 0.775116264820099
1.77884615384615 0.822644889354706
1.875 0.824404358863831
1.9775641025641 0.740648150444031
2.08333333333333 0.849513232707977
2.19551282051282 0.856392502784729
2.31410256410256 0.776458144187927
2.43910256410256 0.737258434295654
2.57051282051282 0.750855624675751
2.70833333333333 0.670948445796967
2.8525641025641 0.801405906677246
3.00641025641026 0.80079174041748
3.16987179487179 0.767162382602692
3.33974358974359 0.79926198720932
3.51923076923077 0.728187263011932
3.70833333333333 0.741733133792877
3.91025641025641 0.815581798553467
4.11858974358974 0.753889203071594
4.34294871794872 0.759807109832764
4.57692307692308 0.730531692504883
4.82371794871795 0.732256412506104
5.08333333333333 0.681616485118866
5.35576923076923 0.720499515533447
5.64423076923077 0.753719210624695
5.94871794871795 0.640903413295746
6.26923076923077 0.638386011123657
6.60576923076923 0.667445838451385
6.96153846153846 0.713362336158752
7.33653846153846 0.672810971736908
7.73397435897436 0.646781384944916
8.15064102564103 0.583291947841644
8.58974358974359 0.680403769016266
9.05128205128205 0.578904986381531
9.53846153846154 0.644657373428345
10.0512820512821 0.614099979400635
10.5929487179487 0.656712651252747
11.1634615384615 0.603906452655792
11.7660256410256 0.532192230224609
12.400641025641 0.515822649002075
13.0673076923077 0.537614285945892
13.7724358974359 0.53514301776886
14.5128205128205 0.542097210884094
15.2948717948718 0.555603206157684
16.1185897435897 0.527148067951202
16.9871794871795 0.53460568189621
17.900641025641 0.5381920337677
18.8653846153846 0.505687832832336
19.8814102564103 0.471593707799911
20.9519230769231 0.433699607849121
22.0801282051282 0.473871797323227
23.2692307692308 0.429493427276611
24.5224358974359 0.405445545911789
25.8429487179487 0.432658672332764
27.2371794871795 0.477295070886612
28.7019230769231 0.379495620727539
30.25 0.423509448766708
31.8782051282051 0.445847123861313
33.5961538461538 0.34607145190239
35.4038461538462 0.497172445058823
37.3108974358974 0.308651298284531
39.3205128205128 0.332051873207092
41.4391025641026 0.277253538370132
43.6698717948718 0.343852609395981
46.0224358974359 0.296028703451157
48.5 0.272984325885773
51.1121794871795 0.305849134922028
53.8653846153846 0.301155418157578
56.7660256410256 0.336846798658371
59.8237179487179 0.198344811797142
63.0448717948718 0.219960734248161
66.4391025641026 0.248977974057198
70.0192307692308 0.295293807983398
73.7884615384615 0.173766732215881
77.7628205128205 0.260215699672699
81.9519230769231 0.340647757053375
86.3653846153846 0.315193653106689
91.0160256410256 0.182530969381332
95.9166666666667 0.329264611005783
101.083333333333 0.212228283286095
106.525641025641 0.222560405731201
112.262820512821 0.299534559249878
118.310897435897 0.141319558024406
124.682692307692 0.153547197580338
131.397435897436 0.210421830415726
138.474358974359 0.334126174449921
145.929487179487 0.371595859527588
153.788461538462 0.263360232114792
162.070512820513 0.199259385466576
170.801282051282 0.508557438850403
180 0.285242229700089
};
\addplot [, color1, opacity=0.6, mark=square*, mark size=0.5, mark options={solid}, only marks]
table {%
1 nan
1.05128205128205 0.716174423694611
1.10897435897436 0.786820590496063
1.16987179487179 0.815349400043488
1.23076923076923 0.909816920757294
1.29807692307692 0.84481543302536
1.36858974358974 0.904722630977631
1.44230769230769 0.827989220619202
1.51923076923077 0.842465102672577
1.6025641025641 0.841242015361786
1.68910256410256 0.857356071472168
1.77884615384615 0.825534045696259
1.875 0.89112377166748
1.9775641025641 0.896426796913147
2.08333333333333 0.901608288288116
2.19551282051282 0.864069938659668
2.31410256410256 0.795582234859467
2.43910256410256 0.837038815021515
2.57051282051282 0.807192623615265
2.70833333333333 0.826963603496552
2.8525641025641 0.804006040096283
3.00641025641026 0.781618595123291
3.16987179487179 0.790025651454926
3.33974358974359 0.852796196937561
3.51923076923077 0.791161000728607
3.70833333333333 0.813343465328217
3.91025641025641 0.7784783244133
4.11858974358974 0.735700190067291
4.34294871794872 0.811472058296204
4.57692307692308 0.841889560222626
4.82371794871795 0.82429963350296
5.08333333333333 0.897823929786682
5.35576923076923 0.858557343482971
5.64423076923077 0.857897400856018
5.94871794871795 0.797312796115875
6.26923076923077 0.822096288204193
6.60576923076923 0.753056824207306
6.96153846153846 0.858740627765656
7.33653846153846 0.724076509475708
7.73397435897436 0.851885437965393
8.15064102564103 0.776394605636597
8.58974358974359 0.762688517570496
9.05128205128205 0.736349046230316
9.53846153846154 0.723780453205109
10.0512820512821 0.748187065124512
10.5929487179487 0.691082179546356
11.1634615384615 0.721250534057617
11.7660256410256 0.654278516769409
12.400641025641 0.665500581264496
13.0673076923077 0.735042214393616
13.7724358974359 0.691174447536469
14.5128205128205 0.671539425849915
15.2948717948718 0.624631822109222
16.1185897435897 0.689874112606049
16.9871794871795 0.643178641796112
17.900641025641 0.630512654781342
18.8653846153846 0.682055652141571
19.8814102564103 0.68333512544632
20.9519230769231 0.626796364784241
22.0801282051282 0.590017795562744
23.2692307692308 0.663619697093964
24.5224358974359 0.565859913825989
25.8429487179487 0.575459659099579
27.2371794871795 0.698630928993225
28.7019230769231 0.641173541545868
30.25 0.637918651103973
31.8782051282051 0.609664559364319
33.5961538461538 0.620641648769379
35.4038461538462 0.582833111286163
37.3108974358974 0.594120025634766
39.3205128205128 0.681575536727905
41.4391025641026 0.648967146873474
43.6698717948718 0.640548884868622
46.0224358974359 0.634347498416901
48.5 0.54023951292038
51.1121794871795 0.577287316322327
53.8653846153846 0.563381910324097
56.7660256410256 0.647076547145844
59.8237179487179 0.68886137008667
63.0448717948718 0.64919513463974
66.4391025641026 0.589577257633209
70.0192307692308 0.645102202892303
73.7884615384615 0.63496321439743
77.7628205128205 0.825106799602509
81.9519230769231 0.736413955688477
86.3653846153846 0.848214268684387
91.0160256410256 0.795377731323242
95.9166666666667 0.759507834911346
101.083333333333 0.7520911693573
106.525641025641 0.882093906402588
112.262820512821 0.885043263435364
118.310897435897 0.767517030239105
124.682692307692 0.821166813373566
131.397435897436 0.795381724834442
138.474358974359 0.873148739337921
145.929487179487 0.853738725185394
153.788461538462 0.978366076946259
162.070512820513 0.915838897228241
170.801282051282 0.696771562099457
180 0.887769877910614
};
\addlegendentry{mb 128, mc 1}
\addplot [, color1, opacity=0.6, mark=square*, mark size=0.5, mark options={solid}, only marks, forget plot]
table {%
1 nan
1.05128205128205 0.687217891216278
1.10897435897436 0.769706428050995
1.16987179487179 0.889419734477997
1.23076923076923 0.84445583820343
1.29807692307692 0.775531351566315
1.36858974358974 0.927840232849121
1.44230769230769 0.89152193069458
1.51923076923077 0.836866557598114
1.6025641025641 0.830600082874298
1.68910256410256 0.830674588680267
1.77884615384615 0.821666538715363
1.875 0.882746160030365
1.9775641025641 0.810724675655365
2.08333333333333 0.87158739566803
2.19551282051282 0.789737462997437
2.31410256410256 0.817181587219238
2.43910256410256 0.790809988975525
2.57051282051282 0.825059711933136
2.70833333333333 0.841601967811584
2.8525641025641 0.753888368606567
3.00641025641026 0.774105548858643
3.16987179487179 0.743530333042145
3.33974358974359 0.857951104640961
3.51923076923077 0.810245990753174
3.70833333333333 0.765592873096466
3.91025641025641 0.821620166301727
4.11858974358974 0.804874062538147
4.34294871794872 0.774400174617767
4.57692307692308 0.850450992584229
4.82371794871795 0.809539616107941
5.08333333333333 0.791907489299774
5.35576923076923 0.847932517528534
5.64423076923077 0.900417745113373
5.94871794871795 0.728397965431213
6.26923076923077 0.781345188617706
6.60576923076923 0.803387939929962
6.96153846153846 0.864462077617645
7.33653846153846 0.868878662586212
7.73397435897436 0.757800698280334
8.15064102564103 0.71281236410141
8.58974358974359 0.774016559123993
9.05128205128205 0.691958904266357
9.53846153846154 0.761378347873688
10.0512820512821 0.756441533565521
10.5929487179487 0.758110284805298
11.1634615384615 0.749862611293793
11.7660256410256 0.646670520305634
12.400641025641 0.799836754798889
13.0673076923077 0.740152835845947
13.7724358974359 0.698032855987549
14.5128205128205 0.689008235931396
15.2948717948718 0.618502259254456
16.1185897435897 0.677314937114716
16.9871794871795 0.638412952423096
17.900641025641 0.652559220790863
18.8653846153846 0.659998714923859
19.8814102564103 0.669591307640076
20.9519230769231 0.615668296813965
22.0801282051282 0.68681138753891
23.2692307692308 0.642100095748901
24.5224358974359 0.692153453826904
25.8429487179487 0.605390548706055
27.2371794871795 0.660663068294525
28.7019230769231 0.519055843353271
30.25 0.616597890853882
31.8782051282051 0.576384961605072
33.5961538461538 0.667305946350098
35.4038461538462 0.603630006313324
37.3108974358974 0.647975027561188
39.3205128205128 0.462220281362534
41.4391025641026 0.624800801277161
43.6698717948718 0.613933145999908
46.0224358974359 0.52377837896347
48.5 0.686535835266113
51.1121794871795 0.566215217113495
53.8653846153846 0.657075226306915
56.7660256410256 0.601461589336395
59.8237179487179 0.678262233734131
63.0448717948718 0.610158503055573
66.4391025641026 0.600715458393097
70.0192307692308 0.656067371368408
73.7884615384615 0.631201684474945
77.7628205128205 0.773281872272491
81.9519230769231 0.647524178028107
86.3653846153846 0.788686871528625
91.0160256410256 0.740809679031372
95.9166666666667 0.869321167469025
101.083333333333 0.747201263904572
106.525641025641 0.95617550611496
112.262820512821 0.83413553237915
118.310897435897 0.870876491069794
124.682692307692 0.977365612983704
131.397435897436 0.892624497413635
138.474358974359 0.855044782161713
145.929487179487 0.806616425514221
153.788461538462 0.888513743877411
162.070512820513 0.766789615154266
170.801282051282 0.982142746448517
180 0.895659446716309
};
\addplot [, color1, opacity=0.6, mark=square*, mark size=0.5, mark options={solid}, only marks, forget plot]
table {%
1 nan
1.05128205128205 0.761629402637482
1.10897435897436 0.710578858852386
1.16987179487179 0.861655831336975
1.23076923076923 0.912299573421478
1.29807692307692 0.888229191303253
1.36858974358974 0.896243274211884
1.44230769230769 0.801083743572235
1.51923076923077 0.881246984004974
1.6025641025641 0.798024356365204
1.68910256410256 0.783951461315155
1.77884615384615 0.799905598163605
1.875 0.828026592731476
1.9775641025641 0.774505734443665
2.08333333333333 0.805661022663116
2.19551282051282 0.81094092130661
2.31410256410256 0.866979420185089
2.43910256410256 0.82789534330368
2.57051282051282 0.824867069721222
2.70833333333333 0.709739685058594
2.8525641025641 0.794375598430634
3.00641025641026 0.789557337760925
3.16987179487179 0.738595008850098
3.33974358974359 0.830671489238739
3.51923076923077 0.80332612991333
3.70833333333333 0.824302315711975
3.91025641025641 0.82524836063385
4.11858974358974 0.881271064281464
4.34294871794872 0.81638777256012
4.57692307692308 0.795919418334961
4.82371794871795 0.889580547809601
5.08333333333333 0.803668200969696
5.35576923076923 0.789197623729706
5.64423076923077 0.886210441589355
5.94871794871795 0.860214054584503
6.26923076923077 0.707514762878418
6.60576923076923 0.78748220205307
6.96153846153846 0.774817645549774
7.33653846153846 0.82949298620224
7.73397435897436 0.783886849880219
8.15064102564103 0.781236231327057
8.58974358974359 0.775857329368591
9.05128205128205 0.81686270236969
9.53846153846154 0.731668531894684
10.0512820512821 0.730591475963593
10.5929487179487 0.642018437385559
11.1634615384615 0.683449745178223
11.7660256410256 0.706539154052734
12.400641025641 0.641910493373871
13.0673076923077 0.719828069210052
13.7724358974359 0.662658154964447
14.5128205128205 0.638359904289246
15.2948717948718 0.665939331054688
16.1185897435897 0.691766738891602
16.9871794871795 0.611902177333832
17.900641025641 0.655708491802216
18.8653846153846 0.609560430049896
19.8814102564103 0.697319626808167
20.9519230769231 0.66458535194397
22.0801282051282 0.653103828430176
23.2692307692308 0.577856957912445
24.5224358974359 0.646214783191681
25.8429487179487 0.606563746929169
27.2371794871795 0.604827225208282
28.7019230769231 0.612672686576843
30.25 0.684676229953766
31.8782051282051 0.593729615211487
33.5961538461538 0.507325351238251
35.4038461538462 0.581675291061401
37.3108974358974 0.649649202823639
39.3205128205128 0.587513387203217
41.4391025641026 0.628822147846222
43.6698717948718 0.702714443206787
46.0224358974359 0.497266203165054
48.5 0.535844147205353
51.1121794871795 0.532553493976593
53.8653846153846 0.526303708553314
56.7660256410256 0.470534235239029
59.8237179487179 0.62492698431015
63.0448717948718 0.716020524501801
66.4391025641026 0.704863011837006
70.0192307692308 0.561402976512909
73.7884615384615 0.680457770824432
77.7628205128205 0.805723786354065
81.9519230769231 0.842522621154785
86.3653846153846 0.853309452533722
91.0160256410256 0.821591794490814
95.9166666666667 0.714007496833801
101.083333333333 0.834287285804749
106.525641025641 0.891369163990021
112.262820512821 0.885679841041565
118.310897435897 0.890623390674591
124.682692307692 0.711465954780579
131.397435897436 0.801866829395294
138.474358974359 0.818677842617035
145.929487179487 0.727832853794098
153.788461538462 0.85657799243927
162.070512820513 0.924128353595734
170.801282051282 0.800963044166565
180 0.843315601348877
};
\addplot [, color1, opacity=0.6, mark=square*, mark size=0.5, mark options={solid}, only marks, forget plot]
table {%
1 nan
1.05128205128205 0.714819610118866
1.10897435897436 0.772639274597168
1.16987179487179 0.899999439716339
1.23076923076923 0.867042481899261
1.29807692307692 0.82215291261673
1.36858974358974 0.944978177547455
1.44230769230769 0.889332413673401
1.51923076923077 0.857829093933105
1.6025641025641 0.875756680965424
1.68910256410256 0.884883880615234
1.77884615384615 0.778695046901703
1.875 0.867547631263733
1.9775641025641 0.82174414396286
2.08333333333333 0.845974564552307
2.19551282051282 0.902170836925507
2.31410256410256 0.769980430603027
2.43910256410256 0.780185222625732
2.57051282051282 0.849614918231964
2.70833333333333 0.860463738441467
2.8525641025641 0.734524846076965
3.00641025641026 0.783248662948608
3.16987179487179 0.830875873565674
3.33974358974359 0.817472279071808
3.51923076923077 0.846594631671906
3.70833333333333 0.82217925786972
3.91025641025641 0.852016270160675
4.11858974358974 0.797398507595062
4.34294871794872 0.846418797969818
4.57692307692308 0.823869705200195
4.82371794871795 0.866110146045685
5.08333333333333 0.780635833740234
5.35576923076923 0.862114727497101
5.64423076923077 0.80875301361084
5.94871794871795 0.82875269651413
6.26923076923077 0.781891465187073
6.60576923076923 0.814208984375
6.96153846153846 0.813802897930145
7.33653846153846 0.886691272258759
7.73397435897436 0.733578860759735
8.15064102564103 0.793689668178558
8.58974358974359 0.754051804542542
9.05128205128205 0.760437965393066
9.53846153846154 0.766721963882446
10.0512820512821 0.708816647529602
10.5929487179487 0.771201848983765
11.1634615384615 0.755366742610931
11.7660256410256 0.759950578212738
12.400641025641 0.800786972045898
13.0673076923077 0.790883660316467
13.7724358974359 0.633029341697693
14.5128205128205 0.6054967045784
15.2948717948718 0.637842655181885
16.1185897435897 0.710385024547577
16.9871794871795 0.742466151714325
17.900641025641 0.629166901111603
18.8653846153846 0.64002388715744
19.8814102564103 0.603003203868866
20.9519230769231 0.643853664398193
22.0801282051282 0.659371674060822
23.2692307692308 0.681363999843597
24.5224358974359 0.580614984035492
25.8429487179487 0.602573335170746
27.2371794871795 0.597896575927734
28.7019230769231 0.689150452613831
30.25 0.617081165313721
31.8782051282051 0.544281125068665
33.5961538461538 0.630832374095917
35.4038461538462 0.568193078041077
37.3108974358974 0.675634562969208
39.3205128205128 0.559497594833374
41.4391025641026 0.582826554775238
43.6698717948718 0.666724920272827
46.0224358974359 0.482093662023544
48.5 0.639004170894623
51.1121794871795 0.593892395496368
53.8653846153846 0.754264533519745
56.7660256410256 0.819014847278595
59.8237179487179 0.783010482788086
63.0448717948718 0.855789124965668
66.4391025641026 0.800900101661682
70.0192307692308 0.658201634883881
73.7884615384615 0.802941143512726
77.7628205128205 0.855058670043945
81.9519230769231 0.79275906085968
86.3653846153846 0.888341844081879
91.0160256410256 0.851317584514618
95.9166666666667 0.890779137611389
101.083333333333 0.827335000038147
106.525641025641 0.719598174095154
112.262820512821 0.889818787574768
118.310897435897 0.858124911785126
124.682692307692 0.913787364959717
131.397435897436 0.87234228849411
138.474358974359 0.872475266456604
145.929487179487 0.896082401275635
153.788461538462 0.762622654438019
162.070512820513 0.915688514709473
170.801282051282 0.974562108516693
180 0.880572021007538
};
\addplot [, color1, opacity=0.6, mark=square*, mark size=0.5, mark options={solid}, only marks, forget plot]
table {%
1 nan
1.05128205128205 0.73395299911499
1.10897435897436 0.784039556980133
1.16987179487179 0.82407134771347
1.23076923076923 0.72992604970932
1.29807692307692 0.828305423259735
1.36858974358974 0.879517018795013
1.44230769230769 0.83626651763916
1.51923076923077 0.846219956874847
1.6025641025641 0.72918713092804
1.68910256410256 0.807668149471283
1.77884615384615 0.80799275636673
1.875 0.916709065437317
1.9775641025641 0.89715987443924
2.08333333333333 0.876571953296661
2.19551282051282 0.894019305706024
2.31410256410256 0.829007804393768
2.43910256410256 0.869685649871826
2.57051282051282 0.835956037044525
2.70833333333333 0.847668945789337
2.8525641025641 0.828348815441132
3.00641025641026 0.871370613574982
3.16987179487179 0.813427925109863
3.33974358974359 0.757614314556122
3.51923076923077 0.821447849273682
3.70833333333333 0.804847359657288
3.91025641025641 0.782588183879852
4.11858974358974 0.80888044834137
4.34294871794872 0.81537789106369
4.57692307692308 0.859490036964417
4.82371794871795 0.854433059692383
5.08333333333333 0.788437306880951
5.35576923076923 0.789473354816437
5.64423076923077 0.860528647899628
5.94871794871795 0.82369202375412
6.26923076923077 0.781094491481781
6.60576923076923 0.765853404998779
6.96153846153846 0.822102963924408
7.33653846153846 0.709409654140472
7.73397435897436 0.809126377105713
8.15064102564103 0.678810000419617
8.58974358974359 0.70979642868042
9.05128205128205 0.6944659948349
9.53846153846154 0.772719502449036
10.0512820512821 0.702443242073059
10.5929487179487 0.733681499958038
11.1634615384615 0.68797767162323
11.7660256410256 0.730062961578369
12.400641025641 0.675607204437256
13.0673076923077 0.673485457897186
13.7724358974359 0.665484607219696
14.5128205128205 0.676423370838165
15.2948717948718 0.594284474849701
16.1185897435897 0.750814616680145
16.9871794871795 0.66286563873291
17.900641025641 0.707372963428497
18.8653846153846 0.691792488098145
19.8814102564103 0.639574825763702
20.9519230769231 0.625425159931183
22.0801282051282 0.602512776851654
23.2692307692308 0.672272503376007
24.5224358974359 0.53896701335907
25.8429487179487 0.606096088886261
27.2371794871795 0.637996971607208
28.7019230769231 0.608951568603516
30.25 0.653835117816925
31.8782051282051 0.602201163768768
33.5961538461538 0.73060268163681
35.4038461538462 0.701290130615234
37.3108974358974 0.550567269325256
39.3205128205128 0.615728497505188
41.4391025641026 0.700930237770081
43.6698717948718 0.677856266498566
46.0224358974359 0.5782430768013
48.5 0.631415188312531
51.1121794871795 0.707898080348969
53.8653846153846 0.651752412319183
56.7660256410256 0.608509719371796
59.8237179487179 0.791920363903046
63.0448717948718 0.965276181697845
66.4391025641026 0.864027619361877
70.0192307692308 0.664921402931213
73.7884615384615 0.78334379196167
77.7628205128205 0.882566928863525
81.9519230769231 0.85017204284668
86.3653846153846 0.826574921607971
91.0160256410256 0.749454975128174
95.9166666666667 0.775220394134521
101.083333333333 0.862888514995575
106.525641025641 0.891425430774689
112.262820512821 0.894512951374054
118.310897435897 0.869726002216339
124.682692307692 0.892788052558899
131.397435897436 0.860824882984161
138.474358974359 0.796919345855713
145.929487179487 0.945326805114746
153.788461538462 0.982609212398529
162.070512820513 0.787012040615082
170.801282051282 0.924612164497375
180 0.74078506231308
};
\addplot [, color2, opacity=0.6, mark=triangle*, mark size=0.5, mark options={solid,rotate=180}, only marks]
table {%
1 nan
1.05128205128205 0.539330899715424
1.10897435897436 0.468410581350327
1.16987179487179 0.689375102519989
1.23076923076923 0.719642102718353
1.29807692307692 0.524804055690765
1.36858974358974 0.590523660182953
1.44230769230769 0.593758881092072
1.51923076923077 0.631567478179932
1.6025641025641 0.547334671020508
1.68910256410256 0.555680751800537
1.77884615384615 0.526724457740784
1.875 0.576648354530334
1.9775641025641 0.659916341304779
2.08333333333333 0.629819214344025
2.19551282051282 0.643963992595673
2.31410256410256 0.631373703479767
2.43910256410256 0.58045619726181
2.57051282051282 0.527381479740143
2.70833333333333 0.556996762752533
2.8525641025641 0.538106143474579
3.00641025641026 0.495051115751266
3.16987179487179 0.490615278482437
3.33974358974359 0.52300614118576
3.51923076923077 0.486466705799103
3.70833333333333 0.502110183238983
3.91025641025641 0.468085289001465
4.11858974358974 0.452196598052979
4.34294871794872 0.51061874628067
4.57692307692308 0.554195582866669
4.82371794871795 0.545387208461761
5.08333333333333 0.430738747119904
5.35576923076923 0.527588188648224
5.64423076923077 0.463773638010025
5.94871794871795 0.483343422412872
6.26923076923077 0.444835156202316
6.60576923076923 0.449297338724136
6.96153846153846 0.445643156766891
7.33653846153846 0.464698135852814
7.73397435897436 0.521259963512421
8.15064102564103 0.448985069990158
8.58974358974359 0.464424461126328
9.05128205128205 0.452080219984055
9.53846153846154 0.417078226804733
10.0512820512821 0.477493762969971
10.5929487179487 0.436828523874283
11.1634615384615 0.43031245470047
11.7660256410256 0.364794880151749
12.400641025641 0.419221967458725
13.0673076923077 0.505739390850067
13.7724358974359 0.423139661550522
14.5128205128205 0.445603758096695
15.2948717948718 0.420539289712906
16.1185897435897 0.418769836425781
16.9871794871795 0.443344205617905
17.900641025641 0.431123644113541
18.8653846153846 0.389937549829483
19.8814102564103 0.334980696439743
20.9519230769231 0.348159730434418
22.0801282051282 0.404939323663712
23.2692307692308 0.324714958667755
24.5224358974359 0.409874737262726
25.8429487179487 0.369244128465652
27.2371794871795 0.315978765487671
28.7019230769231 0.34478160738945
30.25 0.434856325387955
31.8782051282051 0.298427373170853
33.5961538461538 0.325981676578522
35.4038461538462 0.358459234237671
37.3108974358974 0.381308704614639
39.3205128205128 0.359420329332352
41.4391025641026 0.319950371980667
43.6698717948718 0.270100742578506
46.0224358974359 0.304963558912277
48.5 0.347441881895065
51.1121794871795 0.229757383465767
53.8653846153846 0.273446559906006
56.7660256410256 0.286829620599747
59.8237179487179 0.336326837539673
63.0448717948718 0.311071246862411
66.4391025641026 0.230483680963516
70.0192307692308 0.312299072742462
73.7884615384615 0.278573483228683
77.7628205128205 0.240278169512749
81.9519230769231 0.391953706741333
86.3653846153846 0.277208685874939
91.0160256410256 0.269931823015213
95.9166666666667 0.249478846788406
101.083333333333 0.378850996494293
106.525641025641 0.467933803796768
112.262820512821 0.193774849176407
118.310897435897 0.37462916970253
124.682692307692 0.199663206934929
131.397435897436 0.23869900405407
138.474358974359 0.186354905366898
145.929487179487 0.328847497701645
153.788461538462 0.333797454833984
162.070512820513 0.244717761874199
170.801282051282 0.348982572555542
180 0.349520653486252
};
\addlegendentry{sub 16, mc 1}
\addplot [, color2, opacity=0.6, mark=triangle*, mark size=0.5, mark options={solid,rotate=180}, only marks, forget plot]
table {%
1 nan
1.05128205128205 0.515507221221924
1.10897435897436 0.445523351430893
1.16987179487179 0.506191909313202
1.23076923076923 0.655781447887421
1.29807692307692 0.605731308460236
1.36858974358974 0.594135582447052
1.44230769230769 0.632994413375854
1.51923076923077 0.575390934944153
1.6025641025641 0.561360657215118
1.68910256410256 0.486041277647018
1.77884615384615 0.556425213813782
1.875 0.587523698806763
1.9775641025641 0.584166884422302
2.08333333333333 0.524686336517334
2.19551282051282 0.548040628433228
2.31410256410256 0.560217022895813
2.43910256410256 0.490497887134552
2.57051282051282 0.563433289527893
2.70833333333333 0.613287270069122
2.8525641025641 0.484328657388687
3.00641025641026 0.557941555976868
3.16987179487179 0.570147633552551
3.33974358974359 0.599031388759613
3.51923076923077 0.593993186950684
3.70833333333333 0.548873662948608
3.91025641025641 0.58147406578064
4.11858974358974 0.51309871673584
4.34294871794872 0.548560082912445
4.57692307692308 0.560696363449097
4.82371794871795 0.546298623085022
5.08333333333333 0.562561213970184
5.35576923076923 0.500801980495453
5.64423076923077 0.488202482461929
5.94871794871795 0.560946941375732
6.26923076923077 0.527419924736023
6.60576923076923 0.49864473938942
6.96153846153846 0.549328327178955
7.33653846153846 0.501450002193451
7.73397435897436 0.473686695098877
8.15064102564103 0.484909355640411
8.58974358974359 0.48957958817482
9.05128205128205 0.487733036279678
9.53846153846154 0.529993951320648
10.0512820512821 0.459017008543015
10.5929487179487 0.513163864612579
11.1634615384615 0.460362017154694
11.7660256410256 0.405087321996689
12.400641025641 0.442386448383331
13.0673076923077 0.395791739225388
13.7724358974359 0.424308687448502
14.5128205128205 0.398734360933304
15.2948717948718 0.387627005577087
16.1185897435897 0.403666406869888
16.9871794871795 0.448805779218674
17.900641025641 0.414749830961227
18.8653846153846 0.419475376605988
19.8814102564103 0.37654373049736
20.9519230769231 0.384151428937912
22.0801282051282 0.394796252250671
23.2692307692308 0.421703726053238
24.5224358974359 0.352777808904648
25.8429487179487 0.429805904626846
27.2371794871795 0.461886614561081
28.7019230769231 0.402035057544708
30.25 0.37567988038063
31.8782051282051 0.342663764953613
33.5961538461538 0.345583021640778
35.4038461538462 0.324988603591919
37.3108974358974 0.363877773284912
39.3205128205128 0.322251230478287
41.4391025641026 0.417150944471359
43.6698717948718 0.353749424219131
46.0224358974359 0.325803726911545
48.5 0.291279852390289
51.1121794871795 0.339533239603043
53.8653846153846 0.23796072602272
56.7660256410256 0.234263643622398
59.8237179487179 0.33842945098877
63.0448717948718 0.303530424833298
66.4391025641026 0.322621375322342
70.0192307692308 0.318154275417328
73.7884615384615 0.242416530847549
77.7628205128205 0.214300543069839
81.9519230769231 0.27634784579277
86.3653846153846 0.359801799058914
91.0160256410256 0.202877476811409
95.9166666666667 0.178188189864159
101.083333333333 0.24758143723011
106.525641025641 0.243481427431107
112.262820512821 0.300176024436951
118.310897435897 0.373156756162643
124.682692307692 0.216052442789078
131.397435897436 0.279103130102158
138.474358974359 0.158069685101509
145.929487179487 0.278473138809204
153.788461538462 0.298441588878632
162.070512820513 0.187081977725029
170.801282051282 0.124977104365826
180 0.478155821561813
};
\addplot [, color2, opacity=0.6, mark=triangle*, mark size=0.5, mark options={solid,rotate=180}, only marks, forget plot]
table {%
1 nan
1.05128205128205 0.528380453586578
1.10897435897436 0.328974634408951
1.16987179487179 0.430580347776413
1.23076923076923 0.579952657222748
1.29807692307692 0.512941777706146
1.36858974358974 0.504012107849121
1.44230769230769 0.520276010036469
1.51923076923077 0.510021209716797
1.6025641025641 0.465551197528839
1.68910256410256 0.46180984377861
1.77884615384615 0.521097779273987
1.875 0.520822048187256
1.9775641025641 0.553811013698578
2.08333333333333 0.454511076211929
2.19551282051282 0.569289147853851
2.31410256410256 0.540990948677063
2.43910256410256 0.589566051959991
2.57051282051282 0.553077161312103
2.70833333333333 0.484630435705185
2.8525641025641 0.535945117473602
3.00641025641026 0.519459843635559
3.16987179487179 0.528891265392303
3.33974358974359 0.484507709741592
3.51923076923077 0.581809222698212
3.70833333333333 0.498824328184128
3.91025641025641 0.553664207458496
4.11858974358974 0.489226907491684
4.34294871794872 0.491338729858398
4.57692307692308 0.565465569496155
4.82371794871795 0.551780760288239
5.08333333333333 0.508043587207794
5.35576923076923 0.513013184070587
5.64423076923077 0.51517641544342
5.94871794871795 0.471658855676651
6.26923076923077 0.445197105407715
6.60576923076923 0.514951527118683
6.96153846153846 0.510987877845764
7.33653846153846 0.516723453998566
7.73397435897436 0.515934109687805
8.15064102564103 0.50456041097641
8.58974358974359 0.488319545984268
9.05128205128205 0.455340147018433
9.53846153846154 0.447036951780319
10.0512820512821 0.433711737394333
10.5929487179487 0.477499306201935
11.1634615384615 0.405327528715134
11.7660256410256 0.47814878821373
12.400641025641 0.482257843017578
13.0673076923077 0.410509407520294
13.7724358974359 0.375354379415512
14.5128205128205 0.442274391651154
15.2948717948718 0.438366740942001
16.1185897435897 0.391088873147964
16.9871794871795 0.436356842517853
17.900641025641 0.385856121778488
18.8653846153846 0.396426528692245
19.8814102564103 0.390687137842178
20.9519230769231 0.337801992893219
22.0801282051282 0.353954941034317
23.2692307692308 0.383781015872955
24.5224358974359 0.309177309274673
25.8429487179487 0.345576882362366
27.2371794871795 0.278716564178467
28.7019230769231 0.303710669279099
30.25 0.295536011457443
31.8782051282051 0.245652630925179
33.5961538461538 0.276236683130264
35.4038461538462 0.252439647912979
37.3108974358974 0.278237909078598
39.3205128205128 0.255476713180542
41.4391025641026 0.346780598163605
43.6698717948718 0.22123284637928
46.0224358974359 0.248906478285789
48.5 0.310072690248489
51.1121794871795 0.243165969848633
53.8653846153846 0.216846466064453
56.7660256410256 0.195437863469124
59.8237179487179 0.201472043991089
63.0448717948718 0.199234172701836
66.4391025641026 0.173857569694519
70.0192307692308 0.164687231183052
73.7884615384615 0.206432729959488
77.7628205128205 0.213749319314957
81.9519230769231 0.170161992311478
86.3653846153846 0.240199953317642
91.0160256410256 0.250251203775406
95.9166666666667 0.152613118290901
101.083333333333 0.224612787365913
106.525641025641 0.126213893294334
112.262820512821 0.204849526286125
118.310897435897 0.128752693533897
124.682692307692 0.150735557079315
131.397435897436 0.191736400127411
138.474358974359 0.132074281573296
145.929487179487 0.207050323486328
153.788461538462 0.129341870546341
162.070512820513 0.204171016812325
170.801282051282 0.199051022529602
180 0.218877181410789
};
\addplot [, color2, opacity=0.6, mark=triangle*, mark size=0.5, mark options={solid,rotate=180}, only marks, forget plot]
table {%
1 nan
1.05128205128205 0.553271472454071
1.10897435897436 0.370441436767578
1.16987179487179 0.587096810340881
1.23076923076923 0.682796895503998
1.29807692307692 0.555138289928436
1.36858974358974 0.603710174560547
1.44230769230769 0.563899517059326
1.51923076923077 0.598824977874756
1.6025641025641 0.587352097034454
1.68910256410256 0.547771871089935
1.77884615384615 0.604685723781586
1.875 0.596102178096771
1.9775641025641 0.624635696411133
2.08333333333333 0.465083509683609
2.19551282051282 0.493848145008087
2.31410256410256 0.490494817495346
2.43910256410256 0.550651490688324
2.57051282051282 0.510605275630951
2.70833333333333 0.519402265548706
2.8525641025641 0.42816162109375
3.00641025641026 0.524270951747894
3.16987179487179 0.534136116504669
3.33974358974359 0.478748142719269
3.51923076923077 0.48905873298645
3.70833333333333 0.582311570644379
3.91025641025641 0.531498074531555
4.11858974358974 0.54113358259201
4.34294871794872 0.521425068378448
4.57692307692308 0.510938584804535
4.82371794871795 0.58825296163559
5.08333333333333 0.50835657119751
5.35576923076923 0.554719865322113
5.64423076923077 0.532970368862152
5.94871794871795 0.472462147474289
6.26923076923077 0.47275048494339
6.60576923076923 0.448355406522751
6.96153846153846 0.482304483652115
7.33653846153846 0.4611596763134
7.73397435897436 0.463710218667984
8.15064102564103 0.498847216367722
8.58974358974359 0.483158648014069
9.05128205128205 0.441886574029922
9.53846153846154 0.426224440336227
10.0512820512821 0.44011402130127
10.5929487179487 0.532805621623993
11.1634615384615 0.469774633646011
11.7660256410256 0.411602646112442
12.400641025641 0.441097974777222
13.0673076923077 0.46853706240654
13.7724358974359 0.415052086114883
14.5128205128205 0.404113501310349
15.2948717948718 0.384685486555099
16.1185897435897 0.417344659566879
16.9871794871795 0.443424701690674
17.900641025641 0.403536707162857
18.8653846153846 0.409640938043594
19.8814102564103 0.420042425394058
20.9519230769231 0.366423100233078
22.0801282051282 0.362053155899048
23.2692307692308 0.390585422515869
24.5224358974359 0.381334900856018
25.8429487179487 0.340485870838165
27.2371794871795 0.365633010864258
28.7019230769231 0.336378425359726
30.25 0.308592528104782
31.8782051282051 0.375430911779404
33.5961538461538 0.356663793325424
35.4038461538462 0.290408670902252
37.3108974358974 0.338251501321793
39.3205128205128 0.289291828870773
41.4391025641026 0.397341877222061
43.6698717948718 0.426767200231552
46.0224358974359 0.238901004195213
48.5 0.385723203420639
51.1121794871795 0.306573301553726
53.8653846153846 0.298731297254562
56.7660256410256 0.267045915126801
59.8237179487179 0.210807844996452
63.0448717948718 0.231297165155411
66.4391025641026 0.292383879423141
70.0192307692308 0.354147344827652
73.7884615384615 0.284822821617126
77.7628205128205 0.307111918926239
81.9519230769231 0.204937890172005
86.3653846153846 0.237056717276573
91.0160256410256 0.180168122053146
95.9166666666667 0.265181750059128
101.083333333333 0.281030654907227
106.525641025641 0.288844436407089
112.262820512821 0.293064832687378
118.310897435897 0.201423525810242
124.682692307692 0.265595287084579
131.397435897436 0.200583025813103
138.474358974359 0.227221250534058
145.929487179487 0.362549692392349
153.788461538462 0.380500495433807
162.070512820513 0.233359202742577
170.801282051282 0.246173694729805
180 0.1288211196661
};
\addplot [, color2, opacity=0.6, mark=triangle*, mark size=0.5, mark options={solid,rotate=180}, only marks, forget plot]
table {%
1 nan
1.05128205128205 0.562918126583099
1.10897435897436 0.466103225946426
1.16987179487179 0.486678421497345
1.23076923076923 0.580841362476349
1.29807692307692 0.574810385704041
1.36858974358974 0.576413869857788
1.44230769230769 0.584504723548889
1.51923076923077 0.585157752037048
1.6025641025641 0.556840717792511
1.68910256410256 0.51783686876297
1.77884615384615 0.579100668430328
1.875 0.638068377971649
1.9775641025641 0.536256790161133
2.08333333333333 0.604916393756866
2.19551282051282 0.589182019233704
2.31410256410256 0.568793475627899
2.43910256410256 0.530314445495605
2.57051282051282 0.589999973773956
2.70833333333333 0.55078536272049
2.8525641025641 0.581265091896057
3.00641025641026 0.607067406177521
3.16987179487179 0.611179172992706
3.33974358974359 0.561167061328888
3.51923076923077 0.572151362895966
3.70833333333333 0.556359887123108
3.91025641025641 0.562366425991058
4.11858974358974 0.528896272182465
4.34294871794872 0.544181406497955
4.57692307692308 0.58333432674408
4.82371794871795 0.56512326002121
5.08333333333333 0.541321873664856
5.35576923076923 0.547169864177704
5.64423076923077 0.546573102474213
5.94871794871795 0.492004603147507
6.26923076923077 0.486454397439957
6.60576923076923 0.510255932807922
6.96153846153846 0.52107959985733
7.33653846153846 0.495823293924332
7.73397435897436 0.5125932097435
8.15064102564103 0.489418357610703
8.58974358974359 0.52296370267868
9.05128205128205 0.45198854804039
9.53846153846154 0.468686789274216
10.0512820512821 0.438629150390625
10.5929487179487 0.505351483821869
11.1634615384615 0.47247526049614
11.7660256410256 0.474580675363541
12.400641025641 0.421859949827194
13.0673076923077 0.432313501834869
13.7724358974359 0.401516020298004
14.5128205128205 0.458027929067612
15.2948717948718 0.466754347085953
16.1185897435897 0.468060880899429
16.9871794871795 0.432911157608032
17.900641025641 0.467410564422607
18.8653846153846 0.405686676502228
19.8814102564103 0.444575309753418
20.9519230769231 0.381813526153564
22.0801282051282 0.401071459054947
23.2692307692308 0.370043098926544
24.5224358974359 0.36849245429039
25.8429487179487 0.393609404563904
27.2371794871795 0.419673293828964
28.7019230769231 0.331862062215805
30.25 0.367884308099747
31.8782051282051 0.380271583795547
33.5961538461538 0.338007926940918
35.4038461538462 0.442267745733261
37.3108974358974 0.252572953701019
39.3205128205128 0.279582113027573
41.4391025641026 0.288341522216797
43.6698717948718 0.328007370233536
46.0224358974359 0.261647701263428
48.5 0.251455962657928
51.1121794871795 0.291109412908554
53.8653846153846 0.279187977313995
56.7660256410256 0.285501509904861
59.8237179487179 0.227306604385376
63.0448717948718 0.209921672940254
66.4391025641026 0.261210143566132
70.0192307692308 0.276171535253525
73.7884615384615 0.18740913271904
77.7628205128205 0.25382524728775
81.9519230769231 0.345142334699631
86.3653846153846 0.301636755466461
91.0160256410256 0.15669809281826
95.9166666666667 0.313667923212051
101.083333333333 0.219275623559952
106.525641025641 0.219585135579109
112.262820512821 0.275606900453568
118.310897435897 0.147346675395966
124.682692307692 0.147940427064896
131.397435897436 0.217151075601578
138.474358974359 0.350811690092087
145.929487179487 0.352132141590118
153.788461538462 0.275113582611084
162.070512820513 0.196395382285118
170.801282051282 0.446588516235352
180 0.276539474725723
};
\end{axis}

\end{tikzpicture}

    \tikzexternaldisable
  \end{minipage}\hfill
  \begin{minipage}{0.50\linewidth}
    \centering
    % defines the pgfplots style "eigspacedefault"
\pgfkeys{/pgfplots/eigspacedefault/.style={
    width=1.03\linewidth,
    height=\goldenRatioInv*1.03*\linewidth,
    every axis plot/.append style={line width = 1pt},
    tick pos = left,
    ylabel near ticks,
    xlabel near ticks,
    xtick align = inside,
    ytick align = inside,
    legend cell align = left,
    legend columns = 1,
    legend pos = north east,
    legend style = {
      fill opacity = 0.9,
      text opacity = 1,
      font = \tiny,
      % column sep=0.1cm,
    },
    legend image post style={scale=2},
    xticklabel style = {font = \small},
    xlabel style = {font = \small},
    axis line style = {black},
    yticklabel style = {font = \small},
    ylabel style = {font = \small},
    title style = {font = \small},
    grid = major,
    grid style = {dashed}
  }
}

\pgfkeys{/pgfplots/eigspacedefaultapp/.style={
    eigspacedefault,
    height=0.6\linewidth,
    legend columns = 2,
  }
}

\pgfkeys{/pgfplots/eigspacenolegend/.style={
    legend image post style = {scale=0},
    legend style = {
      fill opacity = 0,
      draw opacity = 0,
      text opacity = 0,
      font = \small,
      at={(1, 1.025)},
      anchor=south east,
      column sep=0.25cm,
    },
  }
}
%%% Local Variables:
%%% mode: latex
%%% TeX-master: "../main"
%%% End:

    \pgfkeys{/pgfplots/zmystyle/.style={
        eigspacedefaultapp,
        legend columns = 3,
        eigspacenolegend,
      }}
    \tikzexternalenable
    \vspace{-3ex}
    % This file was created by tikzplotlib v0.9.7.
\begin{tikzpicture}

\definecolor{color0}{rgb}{0.274509803921569,0.6,0.564705882352941}
\definecolor{color1}{rgb}{0.870588235294118,0.623529411764706,0.0862745098039216}
\definecolor{color2}{rgb}{0.501960784313725,0.184313725490196,0.6}

\begin{axis}[
axis line style={white!10!black},
legend style={fill opacity=0.8, draw opacity=1, text opacity=1, at={(0.03,0.03)}, anchor=south west, draw=white!80!black},
log basis x={10},
tick pos=left,
xlabel={epoch (log scale)},
xmajorgrids,
xmin=0.812908354450748, xmax=232.782414148795,
xmode=log,
ylabel={overlap},
ymajorgrids,
ymin=-0.05, ymax=1.05,
zmystyle
]
\addplot [, white!10!black, dashed, forget plot]
table {%
0.812908354450747 1
232.782414148795 1
};
\addplot [, white!10!black, dashed, forget plot]
table {%
0.812908354450747 0
232.782414148795 0
};
\addplot [, color0, opacity=0.6, mark=diamond*, mark size=0.5, mark options={solid}, only marks]
table {%
1 nan
1.05128205128205 0.845952451229095
1.10897435897436 0.887079358100891
1.16987179487179 0.92416524887085
1.23076923076923 0.769045770168304
1.29807692307692 0.77489823102951
1.36858974358974 0.910669922828674
1.44230769230769 0.745540797710419
1.51923076923077 0.889901638031006
1.6025641025641 0.770019710063934
1.68910256410256 0.751518726348877
1.77884615384615 0.774445235729218
1.875 0.780597507953644
1.9775641025641 0.804741501808167
2.08333333333333 0.754346072673798
2.19551282051282 0.779011845588684
2.31410256410256 0.766887843608856
2.43910256410256 0.818009495735168
2.57051282051282 0.68803882598877
2.70833333333333 0.840835511684418
2.8525641025641 0.658719837665558
3.00641025641026 0.75188797712326
3.16987179487179 0.684867322444916
3.33974358974359 0.7175452709198
3.51923076923077 0.72003573179245
3.70833333333333 0.665715575218201
3.91025641025641 0.665278255939484
4.11858974358974 0.66851532459259
4.34294871794872 0.667374610900879
4.57692307692308 0.664417922496796
4.82371794871795 0.676997065544128
5.08333333333333 0.645565986633301
5.35576923076923 0.649685978889465
5.64423076923077 0.676091849803925
5.94871794871795 0.631436705589294
6.26923076923077 0.659340798854828
6.60576923076923 0.604578495025635
6.96153846153846 0.646810054779053
7.33653846153846 0.597398936748505
7.73397435897436 0.638350903987885
8.15064102564103 0.630074918270111
8.58974358974359 0.626124382019043
9.05128205128205 0.599595487117767
9.53846153846154 0.583326935768127
10.0512820512821 0.539160311222076
10.5929487179487 0.579619586467743
11.1634615384615 0.557400166988373
11.7660256410256 0.592771410942078
12.400641025641 0.603026509284973
13.0673076923077 0.530454337596893
13.7724358974359 0.513915359973907
14.5128205128205 0.498725980520248
15.2948717948718 0.526476919651031
16.1185897435897 0.548177659511566
16.9871794871795 0.515902459621429
17.900641025641 0.415068626403809
18.8653846153846 0.495117157697678
19.8814102564103 0.454751014709473
20.9519230769231 0.490729331970215
22.0801282051282 0.491680920124054
23.2692307692308 0.404469072818756
24.5224358974359 0.418725967407227
25.8429487179487 0.448134034872055
27.2371794871795 0.384837299585342
28.7019230769231 0.390335410833359
30.25 0.344281375408173
31.8782051282051 0.446138948202133
33.5961538461538 0.374794453382492
35.4038461538462 0.278156846761703
37.3108974358974 0.31094291806221
39.3205128205128 0.332379311323166
41.4391025641026 0.303095072507858
43.6698717948718 0.35417565703392
46.0224358974359 0.260710716247559
48.5 0.243296816945076
51.1121794871795 0.268090307712555
53.8653846153846 0.334496647119522
56.7660256410256 0.238969832658768
59.8237179487179 0.294326394796371
63.0448717948718 0.292796343564987
66.4391025641026 0.217868164181709
70.0192307692308 0.289180964231491
73.7884615384615 0.287610799074173
77.7628205128205 0.185929089784622
81.9519230769231 0.29485747218132
86.3653846153846 0.223556950688362
91.0160256410256 0.20306222140789
95.9166666666667 0.28301739692688
101.083333333333 0.225006684660912
106.525641025641 0.20213408768177
112.262820512821 0.287670940160751
118.310897435897 0.237058326601982
124.682692307692 0.243705987930298
131.397435897436 0.150340765714645
138.474358974359 0.192152962088585
145.929487179487 0.192765384912491
153.788461538462 0.179507032036781
162.070512820513 0.213819980621338
170.801282051282 0.23477029800415
180 0.15720109641552
};
\addlegendentry{sub 16, exact}
\addplot [, color0, opacity=0.6, mark=diamond*, mark size=0.5, mark options={solid}, only marks, forget plot]
table {%
1 nan
1.05128205128205 0.880568206310272
1.10897435897436 0.894485652446747
1.16987179487179 0.932443916797638
1.23076923076923 0.811491012573242
1.29807692307692 0.812104225158691
1.36858974358974 0.890359878540039
1.44230769230769 0.841425120830536
1.51923076923077 0.865024983882904
1.6025641025641 0.796218872070312
1.68910256410256 0.814590454101562
1.77884615384615 0.776846826076508
1.875 0.807340621948242
1.9775641025641 0.816898345947266
2.08333333333333 0.792773902416229
2.19551282051282 0.778872847557068
2.31410256410256 0.837847352027893
2.43910256410256 0.846647381782532
2.57051282051282 0.797481060028076
2.70833333333333 0.780971348285675
2.8525641025641 0.663655340671539
3.00641025641026 0.849768102169037
3.16987179487179 0.696243584156036
3.33974358974359 0.759462773799896
3.51923076923077 0.744417488574982
3.70833333333333 0.774789273738861
3.91025641025641 0.679393410682678
4.11858974358974 0.73499596118927
4.34294871794872 0.753440141677856
4.57692307692308 0.684836387634277
4.82371794871795 0.748540163040161
5.08333333333333 0.746154427528381
5.35576923076923 0.639201581478119
5.64423076923077 0.720592260360718
5.94871794871795 0.69750851392746
6.26923076923077 0.657629489898682
6.60576923076923 0.64645653963089
6.96153846153846 0.64509117603302
7.33653846153846 0.610652983188629
7.73397435897436 0.611158549785614
8.15064102564103 0.680737793445587
8.58974358974359 0.585833489894867
9.05128205128205 0.622548878192902
9.53846153846154 0.523724257946014
10.0512820512821 0.616989195346832
10.5929487179487 0.556315004825592
11.1634615384615 0.582811415195465
11.7660256410256 0.523393452167511
12.400641025641 0.559852719306946
13.0673076923077 0.533217191696167
13.7724358974359 0.523768424987793
14.5128205128205 0.487180233001709
15.2948717948718 0.503941357135773
16.1185897435897 0.48960867524147
16.9871794871795 0.510646820068359
17.900641025641 0.478980839252472
18.8653846153846 0.434113472700119
19.8814102564103 0.456162840127945
20.9519230769231 0.548520922660828
22.0801282051282 0.51278680562973
23.2692307692308 0.446924448013306
24.5224358974359 0.4520063996315
25.8429487179487 0.374791204929352
27.2371794871795 0.396520346403122
28.7019230769231 0.438885688781738
30.25 0.390827536582947
31.8782051282051 0.417647570371628
33.5961538461538 0.417807012796402
35.4038461538462 0.355967849493027
37.3108974358974 0.338132679462433
39.3205128205128 0.393198579549789
41.4391025641026 0.255392044782639
43.6698717948718 0.359964311122894
46.0224358974359 0.294658273458481
48.5 0.288297742605209
51.1121794871795 0.416025400161743
53.8653846153846 0.3217953145504
56.7660256410256 0.291262239217758
59.8237179487179 0.416190385818481
63.0448717948718 0.277580887079239
66.4391025641026 0.509202003479004
70.0192307692308 0.297512292861938
73.7884615384615 0.302528709173203
77.7628205128205 0.222624972462654
81.9519230769231 0.297486871480942
86.3653846153846 0.37017497420311
91.0160256410256 0.280172675848007
95.9166666666667 0.196332365274429
101.083333333333 0.421773672103882
106.525641025641 0.204311475157738
112.262820512821 0.304975628852844
118.310897435897 0.318666040897369
124.682692307692 0.350552976131439
131.397435897436 0.497522503137589
138.474358974359 0.297110587358475
145.929487179487 0.339456558227539
153.788461538462 0.362876027822495
162.070512820513 0.247056126594543
170.801282051282 0.235056474804878
180 0.223538383841515
};
\addplot [, color0, opacity=0.6, mark=diamond*, mark size=0.5, mark options={solid}, only marks, forget plot]
table {%
1 nan
1.05128205128205 0.933079659938812
1.10897435897436 0.876352250576019
1.16987179487179 0.856239140033722
1.23076923076923 0.784868776798248
1.29807692307692 0.88547956943512
1.36858974358974 0.826424419879913
1.44230769230769 0.828078389167786
1.51923076923077 0.891618371009827
1.6025641025641 0.824002265930176
1.68910256410256 0.742048859596252
1.77884615384615 0.74464225769043
1.875 0.815667331218719
1.9775641025641 0.721863031387329
2.08333333333333 0.772409856319427
2.19551282051282 0.698415398597717
2.31410256410256 0.729591012001038
2.43910256410256 0.764035642147064
2.57051282051282 0.698951661586761
2.70833333333333 0.727365911006927
2.8525641025641 0.732313871383667
3.00641025641026 0.656535267829895
3.16987179487179 0.68596476316452
3.33974358974359 0.701538264751434
3.51923076923077 0.645912170410156
3.70833333333333 0.594078540802002
3.91025641025641 0.682231783866882
4.11858974358974 0.731773316860199
4.34294871794872 0.669080555438995
4.57692307692308 0.635768294334412
4.82371794871795 0.615546226501465
5.08333333333333 0.585683226585388
5.35576923076923 0.616478383541107
5.64423076923077 0.715201020240784
5.94871794871795 0.734517276287079
6.26923076923077 0.678455293178558
6.60576923076923 0.632390677928925
6.96153846153846 0.730983734130859
7.33653846153846 0.66774046421051
7.73397435897436 0.51072096824646
8.15064102564103 0.668868482112885
8.58974358974359 0.572174847126007
9.05128205128205 0.556101560592651
9.53846153846154 0.512398481369019
10.0512820512821 0.521651685237885
10.5929487179487 0.507261991500854
11.1634615384615 0.488807111978531
11.7660256410256 0.611039340496063
12.400641025641 0.509229779243469
13.0673076923077 0.50752180814743
13.7724358974359 0.529403507709503
14.5128205128205 0.522581994533539
15.2948717948718 0.5164994597435
16.1185897435897 0.481152296066284
16.9871794871795 0.407091945409775
17.900641025641 0.461017698049545
18.8653846153846 0.427736967802048
19.8814102564103 0.477311611175537
20.9519230769231 0.423234075307846
22.0801282051282 0.465024560689926
23.2692307692308 0.411818027496338
24.5224358974359 0.410395920276642
25.8429487179487 0.364303439855576
27.2371794871795 0.420776456594467
28.7019230769231 0.496346056461334
30.25 0.328654706478119
31.8782051282051 0.32921439409256
33.5961538461538 0.297664135694504
35.4038461538462 0.370750367641449
37.3108974358974 0.323755979537964
39.3205128205128 0.348172843456268
41.4391025641026 0.324376732110977
43.6698717948718 0.303319752216339
46.0224358974359 0.414534538984299
48.5 0.280931293964386
51.1121794871795 0.222117617726326
53.8653846153846 0.272096633911133
56.7660256410256 0.345082730054855
59.8237179487179 0.287190049886703
63.0448717948718 0.281724900007248
66.4391025641026 0.292959302663803
70.0192307692308 0.256471365690231
73.7884615384615 0.284204572439194
77.7628205128205 0.221262693405151
81.9519230769231 0.314358621835709
86.3653846153846 0.178730040788651
91.0160256410256 0.149797603487968
95.9166666666667 0.277761429548264
101.083333333333 0.243222638964653
106.525641025641 0.198811739683151
112.262820512821 0.297645717859268
118.310897435897 0.188401505351067
124.682692307692 0.272477477788925
131.397435897436 0.236811742186546
138.474358974359 0.249645099043846
145.929487179487 0.292615234851837
153.788461538462 0.36573526263237
162.070512820513 0.177167728543282
170.801282051282 0.301712036132812
180 0.411178320646286
};
\addplot [, color0, opacity=0.6, mark=diamond*, mark size=0.5, mark options={solid}, only marks, forget plot]
table {%
1 nan
1.05128205128205 0.916965782642365
1.10897435897436 0.838736176490784
1.16987179487179 0.887458622455597
1.23076923076923 0.734510481357574
1.29807692307692 0.781142175197601
1.36858974358974 0.773971736431122
1.44230769230769 0.835543274879456
1.51923076923077 0.831505000591278
1.6025641025641 0.760798871517181
1.68910256410256 0.818786799907684
1.77884615384615 0.763281524181366
1.875 0.865028202533722
1.9775641025641 0.798579216003418
2.08333333333333 0.785067021846771
2.19551282051282 0.650784015655518
2.31410256410256 0.717027127742767
2.43910256410256 0.803787231445312
2.57051282051282 0.672489285469055
2.70833333333333 0.762112021446228
2.8525641025641 0.71685117483139
3.00641025641026 0.751989603042603
3.16987179487179 0.693975567817688
3.33974358974359 0.639611661434174
3.51923076923077 0.621710121631622
3.70833333333333 0.721821010112762
3.91025641025641 0.786433935165405
4.11858974358974 0.70819491147995
4.34294871794872 0.684893429279327
4.57692307692308 0.663334727287292
4.82371794871795 0.697541058063507
5.08333333333333 0.665653884410858
5.35576923076923 0.726099133491516
5.64423076923077 0.683059990406036
5.94871794871795 0.707021653652191
6.26923076923077 0.682747960090637
6.60576923076923 0.695081651210785
6.96153846153846 0.650654911994934
7.33653846153846 0.706998229026794
7.73397435897436 0.544842898845673
8.15064102564103 0.702304542064667
8.58974358974359 0.616825222969055
9.05128205128205 0.596209228038788
9.53846153846154 0.537591159343719
10.0512820512821 0.568658292293549
10.5929487179487 0.568979620933533
11.1634615384615 0.666436016559601
11.7660256410256 0.596031844615936
12.400641025641 0.561998903751373
13.0673076923077 0.55564934015274
13.7724358974359 0.541031539440155
14.5128205128205 0.551623344421387
15.2948717948718 0.398031443357468
16.1185897435897 0.535155475139618
16.9871794871795 0.477098137140274
17.900641025641 0.461286395788193
18.8653846153846 0.47277045249939
19.8814102564103 0.501725018024445
20.9519230769231 0.412847727537155
22.0801282051282 0.438026040792465
23.2692307692308 0.448325157165527
24.5224358974359 0.401025146245956
25.8429487179487 0.351346164941788
27.2371794871795 0.400681883096695
28.7019230769231 0.454736471176147
30.25 0.386743932962418
31.8782051282051 0.37306272983551
33.5961538461538 0.382440149784088
35.4038461538462 0.318506717681885
37.3108974358974 0.371991455554962
39.3205128205128 0.363455265760422
41.4391025641026 0.408448129892349
43.6698717948718 0.373431533575058
46.0224358974359 0.364008754491806
48.5 0.33003494143486
51.1121794871795 0.378536194562912
53.8653846153846 0.32795575261116
56.7660256410256 0.340066164731979
59.8237179487179 0.304344177246094
63.0448717948718 0.208644971251488
66.4391025641026 0.231290057301521
70.0192307692308 0.367448061704636
73.7884615384615 0.28504142165184
77.7628205128205 0.293099254369736
81.9519230769231 0.304087519645691
86.3653846153846 0.344183743000031
91.0160256410256 0.217766910791397
95.9166666666667 0.227719351649284
101.083333333333 0.2202477902174
106.525641025641 0.25678613781929
112.262820512821 0.266540914773941
118.310897435897 0.20132540166378
124.682692307692 0.245131775736809
131.397435897436 0.177492260932922
138.474358974359 0.189563080668449
145.929487179487 0.175923258066177
153.788461538462 0.244058877229691
162.070512820513 0.254153519868851
170.801282051282 0.204037711024284
180 0.154524698853493
};
\addplot [, color0, opacity=0.6, mark=diamond*, mark size=0.5, mark options={solid}, only marks, forget plot]
table {%
1 nan
1.05128205128205 0.901219010353088
1.10897435897436 0.861569225788116
1.16987179487179 0.849109470844269
1.23076923076923 0.73636120557785
1.29807692307692 0.837625682353973
1.36858974358974 0.789353489875793
1.44230769230769 0.873866200447083
1.51923076923077 0.814528167247772
1.6025641025641 0.881116688251495
1.68910256410256 0.816385865211487
1.77884615384615 0.680386662483215
1.875 0.874637305736542
1.9775641025641 0.853235065937042
2.08333333333333 0.774335980415344
2.19551282051282 0.737047493457794
2.31410256410256 0.768744945526123
2.43910256410256 0.786899864673615
2.57051282051282 0.719787001609802
2.70833333333333 0.785077154636383
2.8525641025641 0.722902238368988
3.00641025641026 0.747106313705444
3.16987179487179 0.765040338039398
3.33974358974359 0.762858390808105
3.51923076923077 0.761013746261597
3.70833333333333 0.744606971740723
3.91025641025641 0.732713520526886
4.11858974358974 0.708206176757812
4.34294871794872 0.733864724636078
4.57692307692308 0.720754563808441
4.82371794871795 0.700273752212524
5.08333333333333 0.683549523353577
5.35576923076923 0.712268531322479
5.64423076923077 0.711101412773132
5.94871794871795 0.650569379329681
6.26923076923077 0.715632379055023
6.60576923076923 0.705678284168243
6.96153846153846 0.621224522590637
7.33653846153846 0.677820026874542
7.73397435897436 0.684507489204407
8.15064102564103 0.670924305915833
8.58974358974359 0.664994418621063
9.05128205128205 0.614045143127441
9.53846153846154 0.615472376346588
10.0512820512821 0.673744380474091
10.5929487179487 0.647878110408783
11.1634615384615 0.56255716085434
11.7660256410256 0.654252707958221
12.400641025641 0.565503358840942
13.0673076923077 0.562654912471771
13.7724358974359 0.647227346897125
14.5128205128205 0.597023665904999
15.2948717948718 0.490539222955704
16.1185897435897 0.524348556995392
16.9871794871795 0.529428958892822
17.900641025641 0.520271360874176
18.8653846153846 0.506369233131409
19.8814102564103 0.526465594768524
20.9519230769231 0.553492963314056
22.0801282051282 0.521792888641357
23.2692307692308 0.454572200775146
24.5224358974359 0.472187429666519
25.8429487179487 0.525436341762543
27.2371794871795 0.495310217142105
28.7019230769231 0.47020360827446
30.25 0.421179682016373
31.8782051282051 0.513336479663849
33.5961538461538 0.422047913074493
35.4038461538462 0.396633744239807
37.3108974358974 0.440071672201157
39.3205128205128 0.33419480919838
41.4391025641026 0.512174785137177
43.6698717948718 0.436430066823959
46.0224358974359 0.355961382389069
48.5 0.299216032028198
51.1121794871795 0.413456171751022
53.8653846153846 0.31479811668396
56.7660256410256 0.294973433017731
59.8237179487179 0.489247411489487
63.0448717948718 0.394639104604721
66.4391025641026 0.392863482236862
70.0192307692308 0.321657627820969
73.7884615384615 0.239932700991631
77.7628205128205 0.31121638417244
81.9519230769231 0.285105466842651
86.3653846153846 0.302321583032608
91.0160256410256 0.262007355690002
95.9166666666667 0.32746759057045
101.083333333333 0.342595100402832
106.525641025641 0.245605856180191
112.262820512821 0.218721628189087
118.310897435897 0.265070378780365
124.682692307692 0.306853324174881
131.397435897436 0.213632255792618
138.474358974359 0.279212445020676
145.929487179487 0.351907223463058
153.788461538462 0.266928374767303
162.070512820513 0.198077484965324
170.801282051282 0.335840463638306
180 0.204841017723083
};
\addplot [, color1, opacity=0.6, mark=square*, mark size=0.5, mark options={solid}, only marks]
table {%
1 nan
1.05128205128205 0.766435563564301
1.10897435897436 0.815506458282471
1.16987179487179 0.960033059120178
1.23076923076923 0.794786989688873
1.29807692307692 0.854095160961151
1.36858974358974 0.860375583171844
1.44230769230769 0.882435500621796
1.51923076923077 0.943279445171356
1.6025641025641 0.837862968444824
1.68910256410256 0.876173138618469
1.77884615384615 0.75059711933136
1.875 0.916287064552307
1.9775641025641 0.926161587238312
2.08333333333333 0.841211795806885
2.19551282051282 0.748540103435516
2.31410256410256 0.874509811401367
2.43910256410256 0.868827342987061
2.57051282051282 0.811054408550262
2.70833333333333 0.824824333190918
2.8525641025641 0.873573124408722
3.00641025641026 0.807521343231201
3.16987179487179 0.776278376579285
3.33974358974359 0.855055749416351
3.51923076923077 0.805657207965851
3.70833333333333 0.843013405799866
3.91025641025641 0.856534659862518
4.11858974358974 0.788829147815704
4.34294871794872 0.768439590930939
4.57692307692308 0.770748138427734
4.82371794871795 0.76244729757309
5.08333333333333 0.808247029781342
5.35576923076923 0.736177146434784
5.64423076923077 0.784630417823792
5.94871794871795 0.791908025741577
6.26923076923077 0.768626093864441
6.60576923076923 0.775022506713867
6.96153846153846 0.789546191692352
7.33653846153846 0.747547507286072
7.73397435897436 0.741038262844086
8.15064102564103 0.82216739654541
8.58974358974359 0.727279663085938
9.05128205128205 0.694727122783661
9.53846153846154 0.72478324174881
10.0512820512821 0.684782862663269
10.5929487179487 0.682145833969116
11.1634615384615 0.673709571361542
11.7660256410256 0.669126451015472
12.400641025641 0.700037896633148
13.0673076923077 0.696626126766205
13.7724358974359 0.655668377876282
14.5128205128205 0.69490921497345
15.2948717948718 0.668035387992859
16.1185897435897 0.604824542999268
16.9871794871795 0.556138336658478
17.900641025641 0.636211693286896
18.8653846153846 0.576333165168762
19.8814102564103 0.685958504676819
20.9519230769231 0.663835525512695
22.0801282051282 0.702950954437256
23.2692307692308 0.570980608463287
24.5224358974359 0.588911533355713
25.8429487179487 0.664305984973907
27.2371794871795 0.591735303401947
28.7019230769231 0.620515286922455
30.25 0.584264636039734
31.8782051282051 0.756350457668304
33.5961538461538 0.683595538139343
35.4038461538462 0.61667811870575
37.3108974358974 0.59260767698288
39.3205128205128 0.61994069814682
41.4391025641026 0.65684050321579
43.6698717948718 0.697279751300812
46.0224358974359 0.729406297206879
48.5 0.710465848445892
51.1121794871795 0.685556948184967
53.8653846153846 0.773328423500061
56.7660256410256 0.806142628192902
59.8237179487179 0.776897370815277
63.0448717948718 0.801891922950745
66.4391025641026 0.826189815998077
70.0192307692308 0.882430076599121
73.7884615384615 0.792209684848785
77.7628205128205 0.921805799007416
81.9519230769231 0.843303203582764
86.3653846153846 0.873958051204681
91.0160256410256 0.88057154417038
95.9166666666667 0.866033554077148
101.083333333333 0.774160504341125
106.525641025641 0.823786079883575
112.262820512821 0.884197354316711
118.310897435897 0.887198388576508
124.682692307692 0.853850066661835
131.397435897436 0.710595726966858
138.474358974359 0.896115481853485
145.929487179487 0.846881091594696
153.788461538462 0.818390488624573
162.070512820513 0.978847205638885
170.801282051282 0.900226056575775
180 0.966721951961517
};
\addlegendentry{mb 128, mc 1}
\addplot [, color1, opacity=0.6, mark=square*, mark size=0.5, mark options={solid}, only marks, forget plot]
table {%
1 nan
1.05128205128205 0.924019336700439
1.10897435897436 0.892184376716614
1.16987179487179 0.937880635261536
1.23076923076923 0.826497077941895
1.29807692307692 0.871836185455322
1.36858974358974 0.891627311706543
1.44230769230769 0.863466084003448
1.51923076923077 0.92532742023468
1.6025641025641 0.896387279033661
1.68910256410256 0.843770623207092
1.77884615384615 0.832200467586517
1.875 0.902577579021454
1.9775641025641 0.849785625934601
2.08333333333333 0.881843984127045
2.19551282051282 0.823380291461945
2.31410256410256 0.850949227809906
2.43910256410256 0.891238033771515
2.57051282051282 0.804505169391632
2.70833333333333 0.838325321674347
2.8525641025641 0.846987843513489
3.00641025641026 0.786271274089813
3.16987179487179 0.828044533729553
3.33974358974359 0.772278249263763
3.51923076923077 0.778990387916565
3.70833333333333 0.833392918109894
3.91025641025641 0.859408318996429
4.11858974358974 0.777815699577332
4.34294871794872 0.764941394329071
4.57692307692308 0.791637897491455
4.82371794871795 0.833204090595245
5.08333333333333 0.811407089233398
5.35576923076923 0.83097106218338
5.64423076923077 0.802867352962494
5.94871794871795 0.788989841938019
6.26923076923077 0.76969188451767
6.60576923076923 0.837310433387756
6.96153846153846 0.826200485229492
7.33653846153846 0.752197742462158
7.73397435897436 0.741659462451935
8.15064102564103 0.746158480644226
8.58974358974359 0.775152981281281
9.05128205128205 0.685867011547089
9.53846153846154 0.678233325481415
10.0512820512821 0.725788474082947
10.5929487179487 0.694320201873779
11.1634615384615 0.724492371082306
11.7660256410256 0.711697280406952
12.400641025641 0.761731445789337
13.0673076923077 0.709842801094055
13.7724358974359 0.704722821712494
14.5128205128205 0.661676466464996
15.2948717948718 0.684293925762177
16.1185897435897 0.612794578075409
16.9871794871795 0.584759652614594
17.900641025641 0.542458117008209
18.8653846153846 0.595830142498016
19.8814102564103 0.673247277736664
20.9519230769231 0.617895841598511
22.0801282051282 0.607468068599701
23.2692307692308 0.620822906494141
24.5224358974359 0.569821059703827
25.8429487179487 0.577969610691071
27.2371794871795 0.615253746509552
28.7019230769231 0.604135692119598
30.25 0.639641940593719
31.8782051282051 0.64636105298996
33.5961538461538 0.538424372673035
35.4038461538462 0.584251582622528
37.3108974358974 0.629091620445251
39.3205128205128 0.623126327991486
41.4391025641026 0.620820999145508
43.6698717948718 0.57728499174118
46.0224358974359 0.713576138019562
48.5 0.737167179584503
51.1121794871795 0.627103984355927
53.8653846153846 0.665493011474609
56.7660256410256 0.783943235874176
59.8237179487179 0.641126811504364
63.0448717948718 0.627098262310028
66.4391025641026 0.650802314281464
70.0192307692308 0.754552066326141
73.7884615384615 0.765094697475433
77.7628205128205 0.727054297924042
81.9519230769231 0.815317749977112
86.3653846153846 0.927884578704834
91.0160256410256 0.800260484218597
95.9166666666667 0.809879243373871
101.083333333333 0.727022588253021
106.525641025641 0.86650162935257
112.262820512821 0.729208827018738
118.310897435897 0.933937847614288
124.682692307692 0.868005752563477
131.397435897436 0.797441422939301
138.474358974359 0.876581609249115
145.929487179487 0.767559707164764
153.788461538462 0.808745682239532
162.070512820513 0.98346883058548
170.801282051282 0.908898532390594
180 0.845731675624847
};
\addplot [, color1, opacity=0.6, mark=square*, mark size=0.5, mark options={solid}, only marks, forget plot]
table {%
1 nan
1.05128205128205 0.893488109111786
1.10897435897436 0.858025014400482
1.16987179487179 0.956170082092285
1.23076923076923 0.820052921772003
1.29807692307692 0.904899060726166
1.36858974358974 0.88742333650589
1.44230769230769 0.915346264839172
1.51923076923077 0.940819919109344
1.6025641025641 0.865977704524994
1.68910256410256 0.873261153697968
1.77884615384615 0.922671973705292
1.875 0.908008754253387
1.9775641025641 0.852436184883118
2.08333333333333 0.863660633563995
2.19551282051282 0.829620957374573
2.31410256410256 0.83256071805954
2.43910256410256 0.857814967632294
2.57051282051282 0.789027035236359
2.70833333333333 0.866725146770477
2.8525641025641 0.81316339969635
3.00641025641026 0.823071122169495
3.16987179487179 0.868672013282776
3.33974358974359 0.797222495079041
3.51923076923077 0.824357688426971
3.70833333333333 0.827807247638702
3.91025641025641 0.80381965637207
4.11858974358974 0.801276803016663
4.34294871794872 0.758119702339172
4.57692307692308 0.776486158370972
4.82371794871795 0.805474936962128
5.08333333333333 0.804347932338715
5.35576923076923 0.846638679504395
5.64423076923077 0.776859700679779
5.94871794871795 0.746785402297974
6.26923076923077 0.846759796142578
6.60576923076923 0.814108371734619
6.96153846153846 0.809746444225311
7.33653846153846 0.786851465702057
7.73397435897436 0.718504071235657
8.15064102564103 0.776994168758392
8.58974358974359 0.718100607395172
9.05128205128205 0.693438529968262
9.53846153846154 0.722697854042053
10.0512820512821 0.808285713195801
10.5929487179487 0.741431891918182
11.1634615384615 0.704609394073486
11.7660256410256 0.721525967121124
12.400641025641 0.687634408473969
13.0673076923077 0.775043785572052
13.7724358974359 0.641537606716156
14.5128205128205 0.604003071784973
15.2948717948718 0.692860662937164
16.1185897435897 0.713739693164825
16.9871794871795 0.596064746379852
17.900641025641 0.528752982616425
18.8653846153846 0.622618138790131
19.8814102564103 0.685720562934875
20.9519230769231 0.664200723171234
22.0801282051282 0.690519273281097
23.2692307692308 0.650872111320496
24.5224358974359 0.567818820476532
25.8429487179487 0.545547187328339
27.2371794871795 0.573729813098907
28.7019230769231 0.570515930652618
30.25 0.657076179981232
31.8782051282051 0.563553512096405
33.5961538461538 0.56172388792038
35.4038461538462 0.647534072399139
37.3108974358974 0.519223988056183
39.3205128205128 0.637771606445312
41.4391025641026 0.631868481636047
43.6698717948718 0.70712411403656
46.0224358974359 0.622723698616028
48.5 0.713294208049774
51.1121794871795 0.61147004365921
53.8653846153846 0.682836949825287
56.7660256410256 0.608153164386749
59.8237179487179 0.821491241455078
63.0448717948718 0.780193328857422
66.4391025641026 0.783545672893524
70.0192307692308 0.722307801246643
73.7884615384615 0.781289994716644
77.7628205128205 0.879352688789368
81.9519230769231 0.884796321392059
86.3653846153846 0.820357918739319
91.0160256410256 0.785521626472473
95.9166666666667 0.965612888336182
101.083333333333 0.987405240535736
106.525641025641 0.891698002815247
112.262820512821 0.900712430477142
118.310897435897 0.814418971538544
124.682692307692 0.914018154144287
131.397435897436 0.899149239063263
138.474358974359 0.806586086750031
145.929487179487 0.809126853942871
153.788461538462 0.806531071662903
162.070512820513 0.854063451290131
170.801282051282 0.893840432167053
180 0.715341329574585
};
\addplot [, color1, opacity=0.6, mark=square*, mark size=0.5, mark options={solid}, only marks, forget plot]
table {%
1 nan
1.05128205128205 0.939779877662659
1.10897435897436 0.839693009853363
1.16987179487179 0.930764973163605
1.23076923076923 0.80866014957428
1.29807692307692 0.893122851848602
1.36858974358974 0.862128078937531
1.44230769230769 0.909266293048859
1.51923076923077 0.918307960033417
1.6025641025641 0.904208779335022
1.68910256410256 0.932582318782806
1.77884615384615 0.853374779224396
1.875 0.907706439495087
1.9775641025641 0.820680618286133
2.08333333333333 0.879228055477142
2.19551282051282 0.864263832569122
2.31410256410256 0.830939888954163
2.43910256410256 0.897354304790497
2.57051282051282 0.872081100940704
2.70833333333333 0.829363763332367
2.8525641025641 0.822146534919739
3.00641025641026 0.785108983516693
3.16987179487179 0.894850075244904
3.33974358974359 0.860649287700653
3.51923076923077 0.817633450031281
3.70833333333333 0.878290772438049
3.91025641025641 0.81133109331131
4.11858974358974 0.803566575050354
4.34294871794872 0.859989583492279
4.57692307692308 0.849872052669525
4.82371794871795 0.766207039356232
5.08333333333333 0.75656121969223
5.35576923076923 0.763168573379517
5.64423076923077 0.757915675640106
5.94871794871795 0.757714748382568
6.26923076923077 0.824929654598236
6.60576923076923 0.815079391002655
6.96153846153846 0.80527937412262
7.33653846153846 0.783267974853516
7.73397435897436 0.787010490894318
8.15064102564103 0.745582044124603
8.58974358974359 0.714505314826965
9.05128205128205 0.683622419834137
9.53846153846154 0.706868648529053
10.0512820512821 0.749195694923401
10.5929487179487 0.716701686382294
11.1634615384615 0.707069873809814
11.7660256410256 0.667273163795471
12.400641025641 0.738715171813965
13.0673076923077 0.72091019153595
13.7724358974359 0.632293105125427
14.5128205128205 0.681657373905182
15.2948717948718 0.670220375061035
16.1185897435897 0.58705735206604
16.9871794871795 0.684945404529572
17.900641025641 0.668396592140198
18.8653846153846 0.565572202205658
19.8814102564103 0.623241543769836
20.9519230769231 0.727784991264343
22.0801282051282 0.612297713756561
23.2692307692308 0.661574065685272
24.5224358974359 0.61776727437973
25.8429487179487 0.70098340511322
27.2371794871795 0.596043705940247
28.7019230769231 0.647915303707123
30.25 0.617186009883881
31.8782051282051 0.564400374889374
33.5961538461538 0.552673757076263
35.4038461538462 0.621520698070526
37.3108974358974 0.610977590084076
39.3205128205128 0.749129235744476
41.4391025641026 0.703686833381653
43.6698717948718 0.568328738212585
46.0224358974359 0.701961934566498
48.5 0.745943009853363
51.1121794871795 0.751238822937012
53.8653846153846 0.723228216171265
56.7660256410256 0.781070649623871
59.8237179487179 0.741040289402008
63.0448717948718 0.794548869132996
66.4391025641026 0.77395099401474
70.0192307692308 0.783065915107727
73.7884615384615 0.869692623615265
77.7628205128205 0.770353555679321
81.9519230769231 0.682642996311188
86.3653846153846 0.771658360958099
91.0160256410256 0.89211767911911
95.9166666666667 0.876659572124481
101.083333333333 0.835080564022064
106.525641025641 0.837879300117493
112.262820512821 0.858087480068207
118.310897435897 0.885283768177032
124.682692307692 0.959060668945312
131.397435897436 0.887883603572845
138.474358974359 0.869802892208099
145.929487179487 0.903243839740753
153.788461538462 0.950216114521027
162.070512820513 0.975314736366272
170.801282051282 0.820117592811584
180 0.887540996074677
};
\addplot [, color1, opacity=0.6, mark=square*, mark size=0.5, mark options={solid}, only marks, forget plot]
table {%
1 nan
1.05128205128205 0.90132611989975
1.10897435897436 0.884086787700653
1.16987179487179 0.948686242103577
1.23076923076923 0.840571224689484
1.29807692307692 0.851829528808594
1.36858974358974 0.797310292720795
1.44230769230769 0.918612480163574
1.51923076923077 0.945499062538147
1.6025641025641 0.858861744403839
1.68910256410256 0.872751533985138
1.77884615384615 0.840168416500092
1.875 0.893720805644989
1.9775641025641 0.911955296993256
2.08333333333333 0.873834252357483
2.19551282051282 0.864694237709045
2.31410256410256 0.825484097003937
2.43910256410256 0.828824639320374
2.57051282051282 0.892161548137665
2.70833333333333 0.800636291503906
2.8525641025641 0.808099567890167
3.00641025641026 0.827821731567383
3.16987179487179 0.838038563728333
3.33974358974359 0.807968318462372
3.51923076923077 0.832380473613739
3.70833333333333 0.866932690143585
3.91025641025641 0.81560343503952
4.11858974358974 0.765149652957916
4.34294871794872 0.750845193862915
4.57692307692308 0.857683777809143
4.82371794871795 0.837546348571777
5.08333333333333 0.774793744087219
5.35576923076923 0.859717786312103
5.64423076923077 0.79128223657608
5.94871794871795 0.74357545375824
6.26923076923077 0.681595146656036
6.60576923076923 0.739316761493683
6.96153846153846 0.775231778621674
7.33653846153846 0.754409492015839
7.73397435897436 0.8016317486763
8.15064102564103 0.766548633575439
8.58974358974359 0.713542938232422
9.05128205128205 0.710825562477112
9.53846153846154 0.685811519622803
10.0512820512821 0.645962238311768
10.5929487179487 0.712086379528046
11.1634615384615 0.732593953609467
11.7660256410256 0.735398888587952
12.400641025641 0.730346858501434
13.0673076923077 0.627837479114532
13.7724358974359 0.61719286441803
14.5128205128205 0.609879791736603
15.2948717948718 0.706385493278503
16.1185897435897 0.656141042709351
16.9871794871795 0.65481835603714
17.900641025641 0.663930654525757
18.8653846153846 0.621218860149384
19.8814102564103 0.583469390869141
20.9519230769231 0.582064807415009
22.0801282051282 0.632500112056732
23.2692307692308 0.586586475372314
24.5224358974359 0.532524585723877
25.8429487179487 0.505493760108948
27.2371794871795 0.54572993516922
28.7019230769231 0.633138239383698
30.25 0.666289687156677
31.8782051282051 0.640438318252563
33.5961538461538 0.58492249250412
35.4038461538462 0.70843231678009
37.3108974358974 0.639712274074554
39.3205128205128 0.689892172813416
41.4391025641026 0.672608971595764
43.6698717948718 0.605610549449921
46.0224358974359 0.634587049484253
48.5 0.706622242927551
51.1121794871795 0.651824295520782
53.8653846153846 0.715122938156128
56.7660256410256 0.793537616729736
59.8237179487179 0.729048907756805
63.0448717948718 0.820633053779602
66.4391025641026 0.810501039028168
70.0192307692308 0.88181209564209
73.7884615384615 0.826221644878387
77.7628205128205 0.795208156108856
81.9519230769231 0.870182514190674
86.3653846153846 0.809381306171417
91.0160256410256 0.985062599182129
95.9166666666667 0.880972027778625
101.083333333333 0.810260951519012
106.525641025641 0.85271817445755
112.262820512821 0.905699372291565
118.310897435897 0.913008391857147
124.682692307692 0.876755177974701
131.397435897436 0.872237801551819
138.474358974359 0.807197988033295
145.929487179487 0.879739761352539
153.788461538462 0.961223900318146
162.070512820513 0.83326929807663
170.801282051282 0.899756252765656
180 0.72117805480957
};
\addplot [, color2, opacity=0.6, mark=triangle*, mark size=0.5, mark options={solid,rotate=180}, only marks]
table {%
1 nan
1.05128205128205 0.507046520709991
1.10897435897436 0.54054468870163
1.16987179487179 0.608591735363007
1.23076923076923 0.584151208400726
1.29807692307692 0.531746983528137
1.36858974358974 0.61540561914444
1.44230769230769 0.567481935024261
1.51923076923077 0.634341359138489
1.6025641025641 0.618940830230713
1.68910256410256 0.576001584529877
1.77884615384615 0.511391758918762
1.875 0.570805132389069
1.9775641025641 0.607018649578094
2.08333333333333 0.544921517372131
2.19551282051282 0.50193190574646
2.31410256410256 0.650727391242981
2.43910256410256 0.612380802631378
2.57051282051282 0.53531140089035
2.70833333333333 0.511054456233978
2.8525641025641 0.499906152486801
3.00641025641026 0.556451082229614
3.16987179487179 0.48745322227478
3.33974358974359 0.56784999370575
3.51923076923077 0.53886216878891
3.70833333333333 0.523551404476166
3.91025641025641 0.516849517822266
4.11858974358974 0.525819003582001
4.34294871794872 0.525904476642609
4.57692307692308 0.498640835285187
4.82371794871795 0.570542752742767
5.08333333333333 0.469080924987793
5.35576923076923 0.447592109441757
5.64423076923077 0.479277282953262
5.94871794871795 0.496955364942551
6.26923076923077 0.495715916156769
6.60576923076923 0.471866190433502
6.96153846153846 0.534239172935486
7.33653846153846 0.502407193183899
7.73397435897436 0.47911873459816
8.15064102564103 0.492436796426773
8.58974358974359 0.478968620300293
9.05128205128205 0.476867765188217
9.53846153846154 0.478391617536545
10.0512820512821 0.46235391497612
10.5929487179487 0.440313577651978
11.1634615384615 0.475204199552536
11.7660256410256 0.455414980649948
12.400641025641 0.478106498718262
13.0673076923077 0.429880142211914
13.7724358974359 0.452160179615021
14.5128205128205 0.437697559595108
15.2948717948718 0.455114185810089
16.1185897435897 0.427718549966812
16.9871794871795 0.439372539520264
17.900641025641 0.37231832742691
18.8653846153846 0.415956497192383
19.8814102564103 0.400119751691818
20.9519230769231 0.383794873952866
22.0801282051282 0.40045091509819
23.2692307692308 0.347894310951233
24.5224358974359 0.379991978406906
25.8429487179487 0.363328903913498
27.2371794871795 0.347733467817307
28.7019230769231 0.296391814947128
30.25 0.318330526351929
31.8782051282051 0.417533487081528
33.5961538461538 0.25270876288414
35.4038461538462 0.279991209506989
37.3108974358974 0.27165886759758
39.3205128205128 0.307940483093262
41.4391025641026 0.256883025169373
43.6698717948718 0.326746761798859
46.0224358974359 0.23363009095192
48.5 0.246456027030945
51.1121794871795 0.242552667856216
53.8653846153846 0.321840316057205
56.7660256410256 0.218963667750359
59.8237179487179 0.245758250355721
63.0448717948718 0.258526653051376
66.4391025641026 0.224015101790428
70.0192307692308 0.25306960940361
73.7884615384615 0.226066663861275
77.7628205128205 0.187203392386436
81.9519230769231 0.264148980379105
86.3653846153846 0.169503271579742
91.0160256410256 0.164510235190392
95.9166666666667 0.251642197370529
101.083333333333 0.192207038402557
106.525641025641 0.166183739900589
112.262820512821 0.272538721561432
118.310897435897 0.253246933221817
124.682692307692 0.219338700175285
131.397435897436 0.128909334540367
138.474358974359 0.174060210585594
145.929487179487 0.166843101382256
153.788461538462 0.170174166560173
162.070512820513 0.162705942988396
170.801282051282 0.222893789410591
180 0.160499736666679
};
\addlegendentry{sub 16, mc 1}
\addplot [, color2, opacity=0.6, mark=triangle*, mark size=0.5, mark options={solid,rotate=180}, only marks, forget plot]
table {%
1 nan
1.05128205128205 0.527320027351379
1.10897435897436 0.667989909648895
1.16987179487179 0.58683443069458
1.23076923076923 0.539161682128906
1.29807692307692 0.627083957195282
1.36858974358974 0.59124231338501
1.44230769230769 0.580912709236145
1.51923076923077 0.667618870735168
1.6025641025641 0.61346822977066
1.68910256410256 0.605119705200195
1.77884615384615 0.51312267780304
1.875 0.574469745159149
1.9775641025641 0.610222101211548
2.08333333333333 0.56263792514801
2.19551282051282 0.619407474994659
2.31410256410256 0.589800000190735
2.43910256410256 0.636058270931244
2.57051282051282 0.550893843173981
2.70833333333333 0.514419198036194
2.8525641025641 0.527119278907776
3.00641025641026 0.620721518993378
3.16987179487179 0.479934602975845
3.33974358974359 0.53504079580307
3.51923076923077 0.510137736797333
3.70833333333333 0.563288331031799
3.91025641025641 0.483208477497101
4.11858974358974 0.586216688156128
4.34294871794872 0.483460992574692
4.57692307692308 0.53276139497757
4.82371794871795 0.489336162805557
5.08333333333333 0.564039707183838
5.35576923076923 0.466929018497467
5.64423076923077 0.552241623401642
5.94871794871795 0.540566742420197
6.26923076923077 0.572246551513672
6.60576923076923 0.453236192464828
6.96153846153846 0.484917730093002
7.33653846153846 0.445417702198029
7.73397435897436 0.490322202444077
8.15064102564103 0.521009624004364
8.58974358974359 0.493858724832535
9.05128205128205 0.466604441404343
9.53846153846154 0.454942315816879
10.0512820512821 0.458650082349777
10.5929487179487 0.444343328475952
11.1634615384615 0.471235990524292
11.7660256410256 0.434050858020782
12.400641025641 0.47841939330101
13.0673076923077 0.4374078810215
13.7724358974359 0.384983479976654
14.5128205128205 0.407872498035431
15.2948717948718 0.402504444122314
16.1185897435897 0.369808048009872
16.9871794871795 0.445909082889557
17.900641025641 0.388820052146912
18.8653846153846 0.350965887308121
19.8814102564103 0.405746281147003
20.9519230769231 0.406908571720123
22.0801282051282 0.39311271905899
23.2692307692308 0.398885250091553
24.5224358974359 0.318972408771515
25.8429487179487 0.322703808546066
27.2371794871795 0.332486629486084
28.7019230769231 0.375641345977783
30.25 0.314483880996704
31.8782051282051 0.387021541595459
33.5961538461538 0.335109412670135
35.4038461538462 0.307519197463989
37.3108974358974 0.310905456542969
39.3205128205128 0.38490292429924
41.4391025641026 0.265262454748154
43.6698717948718 0.361866623163223
46.0224358974359 0.290642499923706
48.5 0.297199815511703
51.1121794871795 0.34317547082901
53.8653846153846 0.328789204359055
56.7660256410256 0.290742963552475
59.8237179487179 0.397571444511414
63.0448717948718 0.285016745328903
66.4391025641026 0.442217350006104
70.0192307692308 0.278439730405807
73.7884615384615 0.288698613643646
77.7628205128205 0.224870204925537
81.9519230769231 0.275289267301559
86.3653846153846 0.376938104629517
91.0160256410256 0.272658973932266
95.9166666666667 0.176231935620308
101.083333333333 0.308448165655136
106.525641025641 0.201515540480614
112.262820512821 0.269076853990555
118.310897435897 0.320882558822632
124.682692307692 0.323220640420914
131.397435897436 0.322025150060654
138.474358974359 0.287905067205429
145.929487179487 0.340006679296494
153.788461538462 0.267192721366882
162.070512820513 0.226333379745483
170.801282051282 0.267208158969879
180 0.224706277251244
};
\addplot [, color2, opacity=0.6, mark=triangle*, mark size=0.5, mark options={solid,rotate=180}, only marks, forget plot]
table {%
1 nan
1.05128205128205 0.566257536411285
1.10897435897436 0.652865529060364
1.16987179487179 0.606684684753418
1.23076923076923 0.610048711299896
1.29807692307692 0.521948277950287
1.36858974358974 0.589834034442902
1.44230769230769 0.495348453521729
1.51923076923077 0.64075380563736
1.6025641025641 0.530162334442139
1.68910256410256 0.527036368846893
1.77884615384615 0.56304794549942
1.875 0.531660497188568
1.9775641025641 0.51905769109726
2.08333333333333 0.5339435338974
2.19551282051282 0.498544216156006
2.31410256410256 0.550394654273987
2.43910256410256 0.553880751132965
2.57051282051282 0.553487241268158
2.70833333333333 0.520529866218567
2.8525641025641 0.601728081703186
3.00641025641026 0.542404592037201
3.16987179487179 0.572788536548615
3.33974358974359 0.536239743232727
3.51923076923077 0.544754326343536
3.70833333333333 0.483976751565933
3.91025641025641 0.500301539897919
4.11858974358974 0.520681262016296
4.34294871794872 0.507518291473389
4.57692307692308 0.491395682096481
4.82371794871795 0.473139673471451
5.08333333333333 0.509205758571625
5.35576923076923 0.499370902776718
5.64423076923077 0.491623878479004
5.94871794871795 0.529604077339172
6.26923076923077 0.505208969116211
6.60576923076923 0.48202657699585
6.96153846153846 0.501221477985382
7.33653846153846 0.530895888805389
7.73397435897436 0.443004667758942
8.15064102564103 0.533445000648499
8.58974358974359 0.533367991447449
9.05128205128205 0.504547297954559
9.53846153846154 0.449671596288681
10.0512820512821 0.423226445913315
10.5929487179487 0.440436571836472
11.1634615384615 0.450791418552399
11.7660256410256 0.481748580932617
12.400641025641 0.444496840238571
13.0673076923077 0.459767431020737
13.7724358974359 0.452846437692642
14.5128205128205 0.451346009969711
15.2948717948718 0.417321115732193
16.1185897435897 0.348387509584427
16.9871794871795 0.343316256999969
17.900641025641 0.388623684644699
18.8653846153846 0.356731802225113
19.8814102564103 0.368738830089569
20.9519230769231 0.32674977183342
22.0801282051282 0.39145103096962
23.2692307692308 0.385136604309082
24.5224358974359 0.325129717588425
25.8429487179487 0.339328974485397
27.2371794871795 0.347035229206085
28.7019230769231 0.413341850042343
30.25 0.314684122800827
31.8782051282051 0.292393952608109
33.5961538461538 0.230498462915421
35.4038461538462 0.295782089233398
37.3108974358974 0.238429218530655
39.3205128205128 0.29514953494072
41.4391025641026 0.295249551534653
43.6698717948718 0.257116287946701
46.0224358974359 0.409882754087448
48.5 0.258787840604782
51.1121794871795 0.212667509913445
53.8653846153846 0.234781131148338
56.7660256410256 0.298619210720062
59.8237179487179 0.169483587145805
63.0448717948718 0.254916250705719
66.4391025641026 0.257059544324875
70.0192307692308 0.233914718031883
73.7884615384615 0.2281703799963
77.7628205128205 0.178579613566399
81.9519230769231 0.29112908244133
86.3653846153846 0.16239121556282
91.0160256410256 0.149047330021858
95.9166666666667 0.237938553094864
101.083333333333 0.262854814529419
106.525641025641 0.162141874432564
112.262820512821 0.277194708585739
118.310897435897 0.123299241065979
124.682692307692 0.272239834070206
131.397435897436 0.189425855875015
138.474358974359 0.214262172579765
145.929487179487 0.200643301010132
153.788461538462 0.279613137245178
162.070512820513 0.143712073564529
170.801282051282 0.285155534744263
180 0.205578610301018
};
\addplot [, color2, opacity=0.6, mark=triangle*, mark size=0.5, mark options={solid,rotate=180}, only marks, forget plot]
table {%
1 nan
1.05128205128205 0.590201795101166
1.10897435897436 0.579698741436005
1.16987179487179 0.65179455280304
1.23076923076923 0.599152863025665
1.29807692307692 0.635994732379913
1.36858974358974 0.652515232563019
1.44230769230769 0.660710513591766
1.51923076923077 0.655574202537537
1.6025641025641 0.637209355831146
1.68910256410256 0.636815965175629
1.77884615384615 0.62910521030426
1.875 0.561979651451111
1.9775641025641 0.590338945388794
2.08333333333333 0.52570766210556
2.19551282051282 0.504833400249481
2.31410256410256 0.552627444267273
2.43910256410256 0.635986983776093
2.57051282051282 0.474628984928131
2.70833333333333 0.543527901172638
2.8525641025641 0.547011435031891
3.00641025641026 0.461565315723419
3.16987179487179 0.612588346004486
3.33974358974359 0.546268463134766
3.51923076923077 0.466416656970978
3.70833333333333 0.56539660692215
3.91025641025641 0.58255797624588
4.11858974358974 0.490988880395889
4.34294871794872 0.53264856338501
4.57692307692308 0.531173467636108
4.82371794871795 0.522679507732391
5.08333333333333 0.478180557489395
5.35576923076923 0.539711952209473
5.64423076923077 0.490232765674591
5.94871794871795 0.512117385864258
6.26923076923077 0.487030029296875
6.60576923076923 0.530021071434021
6.96153846153846 0.469004780054092
7.33653846153846 0.50938618183136
7.73397435897436 0.464754670858383
8.15064102564103 0.475765854120255
8.58974358974359 0.492333233356476
9.05128205128205 0.460608214139938
9.53846153846154 0.500013291835785
10.0512820512821 0.456856399774551
10.5929487179487 0.506441116333008
11.1634615384615 0.53175550699234
11.7660256410256 0.459443897008896
12.400641025641 0.430714994668961
13.0673076923077 0.463404893875122
13.7724358974359 0.387983173131943
14.5128205128205 0.419643014669418
15.2948717948718 0.341560900211334
16.1185897435897 0.410501301288605
16.9871794871795 0.374721378087997
17.900641025641 0.39534518122673
18.8653846153846 0.361176699399948
19.8814102564103 0.37381237745285
20.9519230769231 0.331669718027115
22.0801282051282 0.379326224327087
23.2692307692308 0.365155696868896
24.5224358974359 0.344053834676743
25.8429487179487 0.297096341848373
27.2371794871795 0.365696102380753
28.7019230769231 0.410417556762695
30.25 0.333098560571671
31.8782051282051 0.323310017585754
33.5961538461538 0.343496143817902
35.4038461538462 0.284566849470139
37.3108974358974 0.318835318088531
39.3205128205128 0.349818140268326
41.4391025641026 0.377251833677292
43.6698717948718 0.332910090684891
46.0224358974359 0.315569669008255
48.5 0.308617502450943
51.1121794871795 0.356053084135056
53.8653846153846 0.275846153497696
56.7660256410256 0.296733886003494
59.8237179487179 0.311437696218491
63.0448717948718 0.189668342471123
66.4391025641026 0.218996077775955
70.0192307692308 0.358066558837891
73.7884615384615 0.246287107467651
77.7628205128205 0.242349699139595
81.9519230769231 0.268446654081345
86.3653846153846 0.322257727384567
91.0160256410256 0.205143362283707
95.9166666666667 0.214524298906326
101.083333333333 0.188510447740555
106.525641025641 0.260299116373062
112.262820512821 0.230442032217979
118.310897435897 0.184379860758781
124.682692307692 0.22948245704174
131.397435897436 0.176720589399338
138.474358974359 0.15906785428524
145.929487179487 0.150570511817932
153.788461538462 0.242770910263062
162.070512820513 0.2250637114048
170.801282051282 0.173022195696831
180 0.131123542785645
};
\addplot [, color2, opacity=0.6, mark=triangle*, mark size=0.5, mark options={solid,rotate=180}, only marks, forget plot]
table {%
1 nan
1.05128205128205 0.578181147575378
1.10897435897436 0.655675113201141
1.16987179487179 0.603068947792053
1.23076923076923 0.538072764873505
1.29807692307692 0.674018800258636
1.36858974358974 0.592463433742523
1.44230769230769 0.642550766468048
1.51923076923077 0.599445998668671
1.6025641025641 0.600294768810272
1.68910256410256 0.58671110868454
1.77884615384615 0.538512408733368
1.875 0.639296650886536
1.9775641025641 0.638975560665131
2.08333333333333 0.573452472686768
2.19551282051282 0.527677476406097
2.31410256410256 0.56496399641037
2.43910256410256 0.627095222473145
2.57051282051282 0.530440926551819
2.70833333333333 0.584184348583221
2.8525641025641 0.537234008312225
3.00641025641026 0.535864651203156
3.16987179487179 0.56990259885788
3.33974358974359 0.559064209461212
3.51923076923077 0.550202667713165
3.70833333333333 0.582254707813263
3.91025641025641 0.565349996089935
4.11858974358974 0.592988193035126
4.34294871794872 0.49678561091423
4.57692307692308 0.534341633319855
4.82371794871795 0.538838088512421
5.08333333333333 0.50652813911438
5.35576923076923 0.589715778827667
5.64423076923077 0.511061012744904
5.94871794871795 0.525760352611542
6.26923076923077 0.483526676893234
6.60576923076923 0.539352893829346
6.96153846153846 0.532944858074188
7.33653846153846 0.525031983852386
7.73397435897436 0.524608552455902
8.15064102564103 0.49442932009697
8.58974358974359 0.516455888748169
9.05128205128205 0.516543865203857
9.53846153846154 0.467437416315079
10.0512820512821 0.494113981723785
10.5929487179487 0.485403388738632
11.1634615384615 0.461144894361496
11.7660256410256 0.555292248725891
12.400641025641 0.490715026855469
13.0673076923077 0.44089064002037
13.7724358974359 0.534387052059174
14.5128205128205 0.465297192335129
15.2948717948718 0.414224624633789
16.1185897435897 0.465317010879517
16.9871794871795 0.444039553403854
17.900641025641 0.432559788227081
18.8653846153846 0.467419594526291
19.8814102564103 0.447340548038483
20.9519230769231 0.458647549152374
22.0801282051282 0.476179927587509
23.2692307692308 0.424674332141876
24.5224358974359 0.390030592679977
25.8429487179487 0.436635255813599
27.2371794871795 0.417910486459732
28.7019230769231 0.422269970178604
30.25 0.405841827392578
31.8782051282051 0.446821212768555
33.5961538461538 0.362594097852707
35.4038461538462 0.371973127126694
37.3108974358974 0.41907986998558
39.3205128205128 0.308425307273865
41.4391025641026 0.482859045267105
43.6698717948718 0.403907597064972
46.0224358974359 0.290010303258896
48.5 0.299258172512054
51.1121794871795 0.382398039102554
53.8653846153846 0.284842491149902
56.7660256410256 0.291155904531479
59.8237179487179 0.406510323286057
63.0448717948718 0.366695463657379
66.4391025641026 0.334899753332138
70.0192307692308 0.289236903190613
73.7884615384615 0.245808437466621
77.7628205128205 0.2941033244133
81.9519230769231 0.273036926984787
86.3653846153846 0.265001624822617
91.0160256410256 0.280488222837448
95.9166666666667 0.331530570983887
101.083333333333 0.262062966823578
106.525641025641 0.25320166349411
112.262820512821 0.20296673476696
118.310897435897 0.239752367138863
124.682692307692 0.269314885139465
131.397435897436 0.203287839889526
138.474358974359 0.272347271442413
145.929487179487 0.348243325948715
153.788461538462 0.255559414625168
162.070512820513 0.188024789094925
170.801282051282 0.320671737194061
180 0.179580301046371
};
\end{axis}

\end{tikzpicture}

    \tikzexternaldisable
  \end{minipage}
\end{subfigure}

\begin{subfigure}[t]{\linewidth}
  \centering
  \caption{\cifarhun \allcnnc}
  \begin{minipage}{0.50\linewidth}
    \centering
    % defines the pgfplots style "eigspacedefault"
\pgfkeys{/pgfplots/eigspacedefault/.style={
    width=1.03\linewidth,
    height=\goldenRatioInv*1.03*\linewidth,
    every axis plot/.append style={line width = 1pt},
    tick pos = left,
    ylabel near ticks,
    xlabel near ticks,
    xtick align = inside,
    ytick align = inside,
    legend cell align = left,
    legend columns = 1,
    legend pos = north east,
    legend style = {
      fill opacity = 0.9,
      text opacity = 1,
      font = \tiny,
      % column sep=0.1cm,
    },
    legend image post style={scale=2},
    xticklabel style = {font = \small},
    xlabel style = {font = \small},
    axis line style = {black},
    yticklabel style = {font = \small},
    ylabel style = {font = \small},
    title style = {font = \small},
    grid = major,
    grid style = {dashed}
  }
}

\pgfkeys{/pgfplots/eigspacedefaultapp/.style={
    eigspacedefault,
    height=0.6\linewidth,
    legend columns = 2,
  }
}

\pgfkeys{/pgfplots/eigspacenolegend/.style={
    legend image post style = {scale=0},
    legend style = {
      fill opacity = 0,
      draw opacity = 0,
      text opacity = 0,
      font = \small,
      at={(1, 1.025)},
      anchor=south east,
      column sep=0.25cm,
    },
  }
}
%%% Local Variables:
%%% mode: latex
%%% TeX-master: "../main"
%%% End:

    \pgfkeys{/pgfplots/zmystyle/.style={
        eigspacedefaultapp,
        legend columns = 3,
        eigspacenolegend,
      }}
    \tikzexternalenable
    \vspace{-3ex}
    % This file was created by tikzplotlib v0.9.7.
\begin{tikzpicture}

\definecolor{color0}{rgb}{0.274509803921569,0.6,0.564705882352941}
\definecolor{color1}{rgb}{0.870588235294118,0.623529411764706,0.0862745098039216}
\definecolor{color2}{rgb}{0.501960784313725,0.184313725490196,0.6}

\begin{axis}[
axis line style={white!10!black},
legend style={fill opacity=0.8, draw opacity=1, text opacity=1, at={(0.91,0.5)}, anchor=east, draw=white!80!black},
log basis x={10},
tick pos=left,
xlabel={epoch (log scale)},
xmajorgrids,
xmin=0.746099240306814, xmax=469.106495613199,
xmode=log,
ylabel={overlap},
ymajorgrids,
ymin=-0.05, ymax=1.05,
zmystyle
]
\addplot [, white!10!black, dashed, forget plot]
table {%
0.746099240306814 1
469.106495613199 1
};
\addplot [, white!10!black, dashed, forget plot]
table {%
0.746099240306814 0
469.106495613199 0
};
\addplot [, color0, opacity=0.6, mark=diamond*, mark size=0.5, mark options={solid}, only marks]
table {%
1 0.937581181526184
1.05769230769231 0.922635316848755
1.12179487179487 0.942588925361633
1.19230769230769 0.944088101387024
1.26282051282051 0.952483892440796
1.33974358974359 0.94643372297287
1.42307692307692 0.954710185527802
1.51282051282051 0.87758994102478
1.6025641025641 0.872245848178864
1.69871794871795 0.842793881893158
1.80128205128205 0.806931614875793
1.91666666666667 0.787581443786621
2.03205128205128 0.809600532054901
2.15384615384615 0.822068750858307
2.28846153846154 0.789439082145691
2.42307692307692 0.776165068149567
2.57692307692308 0.771472275257111
2.73076923076923 0.747488975524902
2.8974358974359 0.767336547374725
3.07692307692308 0.746502339839935
3.26282051282051 0.717796206474304
3.46153846153846 0.75628262758255
3.67307692307692 0.718355238437653
3.8974358974359 0.710424482822418
4.13461538461539 0.70780223608017
4.38461538461539 0.709408402442932
4.65384615384615 0.688105463981628
4.93589743589744 0.719551205635071
5.23717948717949 0.695317983627319
5.55769230769231 0.67035984992981
5.8974358974359 0.663535594940186
6.25641025641026 0.661308169364929
6.64102564102564 0.623116493225098
7.04487179487179 0.61381858587265
7.47435897435897 0.608611345291138
7.92948717948718 0.580015420913696
8.41025641025641 0.557450652122498
8.92307692307692 0.570096790790558
9.46794871794872 0.554607510566711
10.0448717948718 0.516325771808624
10.6602564102564 0.508365631103516
11.3076923076923 0.477993547916412
12 0.466207563877106
12.7307692307692 0.465197861194611
13.5064102564103 0.452254921197891
14.3333333333333 0.460005700588226
15.2051282051282 0.428471446037292
16.1346153846154 0.423788517713547
17.1153846153846 0.453862279653549
18.1602564102564 0.409023493528366
19.2692307692308 0.373001396656036
20.4423076923077 0.398407816886902
21.6858974358974 0.374241828918457
23.0128205128205 0.365072846412659
24.4102564102564 0.359515994787216
25.9038461538462 0.34672212600708
27.4807692307692 0.318506926298141
29.1538461538462 0.328457593917847
30.9358974358974 0.315061688423157
32.8205128205128 0.286424160003662
34.8205128205128 0.301192760467529
36.9423076923077 0.269445687532425
39.1923076923077 0.261177659034729
41.5833333333333 0.250625401735306
44.1153846153846 0.235251799225807
46.8076923076923 0.27566334605217
49.6602564102564 0.242490112781525
52.6858974358974 0.259562343358994
55.8974358974359 0.247348353266716
59.3076923076923 0.196718022227287
62.9230769230769 0.241729959845543
66.7564102564103 0.211357265710831
70.8269230769231 0.218697130680084
75.1474358974359 0.219647288322449
79.7307692307692 0.225905373692513
84.5897435897436 0.227966606616974
89.7435897435897 0.204578772187233
95.2179487179487 0.198797911405563
101.019230769231 0.202846705913544
107.179487179487 0.202811390161514
113.711538461538 0.201824560761452
120.641025641026 0.199758410453796
127.99358974359 0.198581531643867
135.801282051282 0.203909710049629
144.076923076923 0.191054567694664
152.858974358974 0.189190402626991
162.179487179487 0.18217109143734
172.064102564103 0.168118640780449
182.551282051282 0.170081272721291
193.679487179487 0.193025425076485
205.487179487179 0.197849035263062
218.012820512821 0.167410090565681
231.301282051282 0.149216577410698
245.403846153846 0.181258216500282
260.358974358974 0.157669752836227
276.230769230769 0.162172392010689
293.070512820513 0.180160880088806
310.935897435897 0.194721102714539
329.884615384615 0.195818096399307
350 0.189270436763763
};
\addlegendentry{sub 16, exact}
\addplot [, color0, opacity=0.6, mark=diamond*, mark size=0.5, mark options={solid}, only marks, forget plot]
table {%
1 0.92511397600174
1.05769230769231 0.933486700057983
1.12179487179487 0.938036143779755
1.19230769230769 0.936247706413269
1.26282051282051 0.941067636013031
1.33974358974359 0.945724785327911
1.42307692307692 0.922895789146423
1.51282051282051 0.931287348270416
1.6025641025641 0.892382025718689
1.69871794871795 0.917123973369598
1.80128205128205 0.893858015537262
1.91666666666667 0.861504793167114
2.03205128205128 0.848914682865143
2.15384615384615 0.845843613147736
2.28846153846154 0.840817868709564
2.42307692307692 0.84293133020401
2.57692307692308 0.861801028251648
2.73076923076923 0.848062753677368
2.8974358974359 0.83060359954834
3.07692307692308 0.849447906017303
3.26282051282051 0.77325314283371
3.46153846153846 0.815473318099976
3.67307692307692 0.788126647472382
3.8974358974359 0.77327972650528
4.13461538461539 0.77638578414917
4.38461538461539 0.728356897830963
4.65384615384615 0.723148047924042
4.93589743589744 0.753884196281433
5.23717948717949 0.738874971866608
5.55769230769231 0.711867988109589
5.8974358974359 0.682506263256073
6.25641025641026 0.654131889343262
6.64102564102564 0.639886319637299
7.04487179487179 0.652170240879059
7.47435897435897 0.637721061706543
7.92948717948718 0.598964095115662
8.41025641025641 0.577364206314087
8.92307692307692 0.593272864818573
9.46794871794872 0.586264312267303
10.0448717948718 0.537537336349487
10.6602564102564 0.530237138271332
11.3076923076923 0.517817914485931
12 0.496434777975082
12.7307692307692 0.493979394435883
13.5064102564103 0.466847211122513
14.3333333333333 0.45997554063797
15.2051282051282 0.450997084379196
16.1346153846154 0.424528479576111
17.1153846153846 0.429766297340393
18.1602564102564 0.400194078683853
19.2692307692308 0.37841060757637
20.4423076923077 0.375755161046982
21.6858974358974 0.366879045963287
23.0128205128205 0.358576357364655
24.4102564102564 0.360970377922058
25.9038461538462 0.333934307098389
27.4807692307692 0.338396400213242
29.1538461538462 0.334904789924622
30.9358974358974 0.309105396270752
32.8205128205128 0.297036826610565
34.8205128205128 0.298461496829987
36.9423076923077 0.288945734500885
39.1923076923077 0.28368079662323
41.5833333333333 0.275374680757523
44.1153846153846 0.270525306463242
46.8076923076923 0.267005860805511
49.6602564102564 0.25913992524147
52.6858974358974 0.238478735089302
55.8974358974359 0.253686517477036
59.3076923076923 0.24018195271492
62.9230769230769 0.231788650155067
66.7564102564103 0.231311872601509
70.8269230769231 0.201214730739594
75.1474358974359 0.21222011744976
79.7307692307692 0.206512987613678
84.5897435897436 0.214142188429832
89.7435897435897 0.200858071446419
95.2179487179487 0.230902776122093
101.019230769231 0.224933370947838
107.179487179487 0.209989637136459
113.711538461538 0.198990985751152
120.641025641026 0.203542768955231
127.99358974359 0.213083550333977
135.801282051282 0.222225680947304
144.076923076923 0.195189744234085
152.858974358974 0.18438883125782
162.179487179487 0.216622963547707
172.064102564103 0.200191602110863
182.551282051282 0.168388903141022
193.679487179487 0.188531205058098
205.487179487179 0.182962015271187
218.012820512821 0.174137130379677
231.301282051282 0.167254641652107
245.403846153846 0.208708643913269
260.358974358974 0.210213124752045
276.230769230769 0.182638540863991
293.070512820513 0.189035728573799
310.935897435897 0.18598872423172
329.884615384615 0.203566700220108
350 0.197859600186348
};
\addplot [, color0, opacity=0.6, mark=diamond*, mark size=0.5, mark options={solid}, only marks, forget plot]
table {%
1 0.898708760738373
1.05769230769231 0.891044139862061
1.12179487179487 0.912497520446777
1.19230769230769 0.913806438446045
1.26282051282051 0.936162710189819
1.33974358974359 0.947584331035614
1.42307692307692 0.924908101558685
1.51282051282051 0.832841455936432
1.6025641025641 0.747997343540192
1.69871794871795 0.742354094982147
1.80128205128205 0.744499504566193
1.91666666666667 0.752137422561646
2.03205128205128 0.774000823497772
2.15384615384615 0.823955059051514
2.28846153846154 0.738425135612488
2.42307692307692 0.748494029045105
2.57692307692308 0.709626019001007
2.73076923076923 0.704068064689636
2.8974358974359 0.754766523838043
3.07692307692308 0.699448525905609
3.26282051282051 0.677858173847198
3.46153846153846 0.717577219009399
3.67307692307692 0.675565600395203
3.8974358974359 0.716910541057587
4.13461538461539 0.675445556640625
4.38461538461539 0.704296231269836
4.65384615384615 0.696390151977539
4.93589743589744 0.72110641002655
5.23717948717949 0.72314465045929
5.55769230769231 0.670754134654999
5.8974358974359 0.64761608839035
6.25641025641026 0.672710418701172
6.64102564102564 0.630694091320038
7.04487179487179 0.620206236839294
7.47435897435897 0.612429082393646
7.92948717948718 0.631450772285461
8.41025641025641 0.594704866409302
8.92307692307692 0.587580502033234
9.46794871794872 0.565456211566925
10.0448717948718 0.550863862037659
10.6602564102564 0.540319442749023
11.3076923076923 0.521593987941742
12 0.514307856559753
12.7307692307692 0.476623594760895
13.5064102564103 0.482502728700638
14.3333333333333 0.469973564147949
15.2051282051282 0.455308675765991
16.1346153846154 0.442448735237122
17.1153846153846 0.433015048503876
18.1602564102564 0.400821417570114
19.2692307692308 0.399378627538681
20.4423076923077 0.389773160219193
21.6858974358974 0.394900351762772
23.0128205128205 0.396190792322159
24.4102564102564 0.380995392799377
25.9038461538462 0.373160094022751
27.4807692307692 0.35030409693718
29.1538461538462 0.351141184568405
30.9358974358974 0.323260962963104
32.8205128205128 0.317476272583008
34.8205128205128 0.316512018442154
36.9423076923077 0.315866738557816
39.1923076923077 0.30485001206398
41.5833333333333 0.287849843502045
44.1153846153846 0.277397632598877
46.8076923076923 0.29241818189621
49.6602564102564 0.286146014928818
52.6858974358974 0.265644520521164
55.8974358974359 0.273725777864456
59.3076923076923 0.259488582611084
62.9230769230769 0.263461202383041
66.7564102564103 0.23895400762558
70.8269230769231 0.237614095211029
75.1474358974359 0.222768664360046
79.7307692307692 0.253966808319092
84.5897435897436 0.231595993041992
89.7435897435897 0.205799892544746
95.2179487179487 0.229564875364304
101.019230769231 0.208560705184937
107.179487179487 0.239937201142311
113.711538461538 0.219333276152611
120.641025641026 0.186112940311432
127.99358974359 0.181172177195549
135.801282051282 0.211698681116104
144.076923076923 0.197432890534401
152.858974358974 0.223790779709816
162.179487179487 0.21940016746521
172.064102564103 0.191751554608345
182.551282051282 0.177100464701653
193.679487179487 0.199389889836311
205.487179487179 0.222641617059708
218.012820512821 0.228913888335228
231.301282051282 0.204385682940483
245.403846153846 0.199808657169342
260.358974358974 0.185935378074646
276.230769230769 0.145766228437424
293.070512820513 0.215920075774193
310.935897435897 0.238378077745438
329.884615384615 0.203266069293022
350 0.205557137727737
};
\addplot [, color0, opacity=0.6, mark=diamond*, mark size=0.5, mark options={solid}, only marks, forget plot]
table {%
1 0.893867015838623
1.05769230769231 0.8915074467659
1.12179487179487 0.909411430358887
1.19230769230769 0.91003543138504
1.26282051282051 0.906083047389984
1.33974358974359 0.91674131155014
1.42307692307692 0.886600017547607
1.51282051282051 0.712612807750702
1.6025641025641 0.676389813423157
1.69871794871795 0.674472332000732
1.80128205128205 0.650995314121246
1.91666666666667 0.65347808599472
2.03205128205128 0.662145018577576
2.15384615384615 0.69178432226181
2.28846153846154 0.670621752738953
2.42307692307692 0.705210864543915
2.57692307692308 0.650365114212036
2.73076923076923 0.637496829032898
2.8974358974359 0.672862529754639
3.07692307692308 0.65608024597168
3.26282051282051 0.645018577575684
3.46153846153846 0.657853305339813
3.67307692307692 0.655462265014648
3.8974358974359 0.674566686153412
4.13461538461539 0.670873641967773
4.38461538461539 0.651531636714935
4.65384615384615 0.652149856090546
4.93589743589744 0.695730030536652
5.23717948717949 0.698941469192505
5.55769230769231 0.658465206623077
5.8974358974359 0.630078852176666
6.25641025641026 0.630689382553101
6.64102564102564 0.586956262588501
7.04487179487179 0.569914281368256
7.47435897435897 0.60086852312088
7.92948717948718 0.578093409538269
8.41025641025641 0.547429978847504
8.92307692307692 0.560156106948853
9.46794871794872 0.553458034992218
10.0448717948718 0.529184699058533
10.6602564102564 0.513037323951721
11.3076923076923 0.482349157333374
12 0.465003430843353
12.7307692307692 0.444134712219238
13.5064102564103 0.448561549186707
14.3333333333333 0.444610387086868
15.2051282051282 0.437339276075363
16.1346153846154 0.423719674348831
17.1153846153846 0.438586086034775
18.1602564102564 0.401035755872726
19.2692307692308 0.413941740989685
20.4423076923077 0.400263726711273
21.6858974358974 0.380850672721863
23.0128205128205 0.377182900905609
24.4102564102564 0.36927542090416
25.9038461538462 0.354103982448578
27.4807692307692 0.35357192158699
29.1538461538462 0.331187188625336
30.9358974358974 0.32531151175499
32.8205128205128 0.304935187101364
34.8205128205128 0.295028924942017
36.9423076923077 0.295753747224808
39.1923076923077 0.304417371749878
41.5833333333333 0.295810043811798
44.1153846153846 0.285668820142746
46.8076923076923 0.28866520524025
49.6602564102564 0.279908448457718
52.6858974358974 0.263795524835587
55.8974358974359 0.271252512931824
59.3076923076923 0.240949586033821
62.9230769230769 0.247471824288368
66.7564102564103 0.240124240517616
70.8269230769231 0.244805693626404
75.1474358974359 0.22792361676693
79.7307692307692 0.248690828680992
84.5897435897436 0.223964393138885
89.7435897435897 0.227678596973419
95.2179487179487 0.190514385700226
101.019230769231 0.196138113737106
107.179487179487 0.222683355212212
113.711538461538 0.239126950502396
120.641025641026 0.20205745100975
127.99358974359 0.20535783469677
135.801282051282 0.227114945650101
144.076923076923 0.201525703072548
152.858974358974 0.191954284906387
162.179487179487 0.219057306647301
172.064102564103 0.245830610394478
182.551282051282 0.209165260195732
193.679487179487 0.208785697817802
205.487179487179 0.187848031520844
218.012820512821 0.159688919782639
231.301282051282 0.202117279171944
245.403846153846 0.200474262237549
260.358974358974 0.242929816246033
276.230769230769 0.212994873523712
293.070512820513 0.22601792216301
310.935897435897 0.232245743274689
329.884615384615 0.210587903857231
350 0.197501763701439
};
\addplot [, color0, opacity=0.6, mark=diamond*, mark size=0.5, mark options={solid}, only marks, forget plot]
table {%
1 0.932449758052826
1.05769230769231 0.925216257572174
1.12179487179487 0.938235938549042
1.19230769230769 0.942610442638397
1.26282051282051 0.949080348014832
1.33974358974359 0.950080156326294
1.42307692307692 0.935472190380096
1.51282051282051 0.883467555046082
1.6025641025641 0.860576152801514
1.69871794871795 0.874143838882446
1.80128205128205 0.832949042320251
1.91666666666667 0.804596245288849
2.03205128205128 0.832087635993958
2.15384615384615 0.748561322689056
2.28846153846154 0.743344008922577
2.42307692307692 0.76098644733429
2.57692307692308 0.783465087413788
2.73076923076923 0.648492097854614
2.8974358974359 0.778575658798218
3.07692307692308 0.778875708580017
3.26282051282051 0.740498244762421
3.46153846153846 0.747991919517517
3.67307692307692 0.755491316318512
3.8974358974359 0.762648284435272
4.13461538461539 0.762139141559601
4.38461538461539 0.723016619682312
4.65384615384615 0.691159784793854
4.93589743589744 0.72696590423584
5.23717948717949 0.718894481658936
5.55769230769231 0.698548138141632
5.8974358974359 0.688091933727264
6.25641025641026 0.682381808757782
6.64102564102564 0.658156394958496
7.04487179487179 0.643949568271637
7.47435897435897 0.624071002006531
7.92948717948718 0.603318214416504
8.41025641025641 0.588537335395813
8.92307692307692 0.597224473953247
9.46794871794872 0.577855050563812
10.0448717948718 0.578990161418915
10.6602564102564 0.559691607952118
11.3076923076923 0.52429187297821
12 0.528333604335785
12.7307692307692 0.520380258560181
13.5064102564103 0.470735818147659
14.3333333333333 0.500509798526764
15.2051282051282 0.464901328086853
16.1346153846154 0.43258810043335
17.1153846153846 0.413143992424011
18.1602564102564 0.412753283977509
19.2692307692308 0.396216690540314
20.4423076923077 0.369296967983246
21.6858974358974 0.355114817619324
23.0128205128205 0.354922920465469
24.4102564102564 0.31905597448349
25.9038461538462 0.326167583465576
27.4807692307692 0.302039474248886
29.1538461538462 0.312900841236115
30.9358974358974 0.29700168967247
32.8205128205128 0.291092902421951
34.8205128205128 0.289694786071777
36.9423076923077 0.288082271814346
39.1923076923077 0.256058841943741
41.5833333333333 0.266443222761154
44.1153846153846 0.247959896922112
46.8076923076923 0.241273641586304
49.6602564102564 0.243878602981567
52.6858974358974 0.200604736804962
55.8974358974359 0.228994786739349
59.3076923076923 0.196228697896004
62.9230769230769 0.216332778334618
66.7564102564103 0.221352159976959
70.8269230769231 0.190912961959839
75.1474358974359 0.183481782674789
79.7307692307692 0.210055097937584
84.5897435897436 0.147915974259377
89.7435897435897 0.168282583355904
95.2179487179487 0.157421946525574
101.019230769231 0.204131945967674
107.179487179487 0.173139467835426
113.711538461538 0.168658703565598
120.641025641026 0.166934654116631
127.99358974359 0.177257642149925
135.801282051282 0.177927076816559
144.076923076923 0.186973065137863
152.858974358974 0.167839616537094
162.179487179487 0.153153941035271
172.064102564103 0.178358644247055
182.551282051282 0.19580489397049
193.679487179487 0.145132228732109
205.487179487179 0.147957190871239
218.012820512821 0.155343875288963
231.301282051282 0.150348722934723
245.403846153846 0.170141115784645
260.358974358974 0.170723676681519
276.230769230769 0.121445082128048
293.070512820513 0.157836601138115
310.935897435897 0.148740097880363
329.884615384615 0.146143347024918
350 0.121548056602478
};
\addplot [, color1, opacity=0.6, mark=square*, mark size=0.5, mark options={solid}, only marks]
table {%
1 0.917830586433411
1.05769230769231 0.916284143924713
1.12179487179487 0.920102596282959
1.19230769230769 0.916280627250671
1.26282051282051 0.925693869590759
1.33974358974359 0.909586787223816
1.42307692307692 0.917571604251862
1.51282051282051 0.87862765789032
1.6025641025641 0.882966995239258
1.69871794871795 0.830889463424683
1.80128205128205 0.842455327510834
1.91666666666667 0.838577032089233
2.03205128205128 0.856413602828979
2.15384615384615 0.848704755306244
2.28846153846154 0.845400989055634
2.42307692307692 0.842867255210876
2.57692307692308 0.832375824451447
2.73076923076923 0.843515455722809
2.8974358974359 0.851314842700958
3.07692307692308 0.839495062828064
3.26282051282051 0.815182387828827
3.46153846153846 0.847006797790527
3.67307692307692 0.840971350669861
3.8974358974359 0.830373525619507
4.13461538461539 0.8444784283638
4.38461538461539 0.828305959701538
4.65384615384615 0.838536024093628
4.93589743589744 0.844193339347839
5.23717948717949 0.834755837917328
5.55769230769231 0.806811988353729
5.8974358974359 0.816743910312653
6.25641025641026 0.820064425468445
6.64102564102564 0.817893803119659
7.04487179487179 0.819248199462891
7.47435897435897 0.838805913925171
7.92948717948718 0.831580102443695
8.41025641025641 0.818409264087677
8.92307692307692 0.808122992515564
9.46794871794872 0.822980999946594
10.0448717948718 0.810391843318939
10.6602564102564 0.836999356746674
11.3076923076923 0.809498131275177
12 0.81167858839035
12.7307692307692 0.793366849422455
13.5064102564103 0.782086789608002
14.3333333333333 0.810436964035034
15.2051282051282 0.807610929012299
16.1346153846154 0.776302456855774
17.1153846153846 0.799048602581024
18.1602564102564 0.784743785858154
19.2692307692308 0.767131805419922
20.4423076923077 0.788646399974823
21.6858974358974 0.786943018436432
23.0128205128205 0.755285799503326
24.4102564102564 0.770455598831177
25.9038461538462 0.763260960578918
27.4807692307692 0.747686505317688
29.1538461538462 0.752114236354828
30.9358974358974 0.77151495218277
32.8205128205128 0.747929990291595
34.8205128205128 0.744613945484161
36.9423076923077 0.726805627346039
39.1923076923077 0.727860987186432
41.5833333333333 0.721495926380157
44.1153846153846 0.748321294784546
46.8076923076923 0.715723812580109
49.6602564102564 0.755286037921906
52.6858974358974 0.701936602592468
55.8974358974359 0.735390901565552
59.3076923076923 0.71305125951767
62.9230769230769 0.736764967441559
66.7564102564103 0.692656219005585
70.8269230769231 0.706414401531219
75.1474358974359 0.740538597106934
79.7307692307692 0.72257924079895
84.5897435897436 0.737451672554016
89.7435897435897 0.715464293956757
95.2179487179487 0.709693968296051
101.019230769231 0.705316841602325
107.179487179487 0.723280549049377
113.711538461538 0.699775516986847
120.641025641026 0.70047390460968
127.99358974359 0.69715678691864
135.801282051282 0.707223474979401
144.076923076923 0.684165477752686
152.858974358974 0.713411748409271
162.179487179487 0.696109592914581
172.064102564103 0.71299797296524
182.551282051282 0.712961256504059
193.679487179487 0.714417994022369
205.487179487179 0.697755038738251
218.012820512821 0.679644405841827
231.301282051282 0.687822341918945
245.403846153846 0.691141486167908
260.358974358974 0.701807916164398
276.230769230769 0.69388073682785
293.070512820513 0.706382870674133
310.935897435897 0.713246762752533
329.884615384615 0.699564516544342
350 0.68597823381424
};
\addlegendentry{mb 128, mc 10}
\addplot [, color1, opacity=0.6, mark=square*, mark size=0.5, mark options={solid}, only marks, forget plot]
table {%
1 0.922261774539948
1.05769230769231 0.915669143199921
1.12179487179487 0.930274426937103
1.19230769230769 0.92735767364502
1.26282051282051 0.918950021266937
1.33974358974359 0.922505497932434
1.42307692307692 0.921913206577301
1.51282051282051 0.906273186206818
1.6025641025641 0.854153037071228
1.69871794871795 0.841176152229309
1.80128205128205 0.826763689517975
1.91666666666667 0.823616445064545
2.03205128205128 0.833372175693512
2.15384615384615 0.83092725276947
2.28846153846154 0.847436726093292
2.42307692307692 0.827658295631409
2.57692307692308 0.824987292289734
2.73076923076923 0.846198260784149
2.8974358974359 0.838687896728516
3.07692307692308 0.845102429389954
3.26282051282051 0.83860057592392
3.46153846153846 0.848734796047211
3.67307692307692 0.822784304618835
3.8974358974359 0.84404981136322
4.13461538461539 0.850513577461243
4.38461538461539 0.835622370243073
4.65384615384615 0.827687978744507
4.93589743589744 0.849286496639252
5.23717948717949 0.835106611251831
5.55769230769231 0.828618168830872
5.8974358974359 0.845140993595123
6.25641025641026 0.831215798854828
6.64102564102564 0.832469642162323
7.04487179487179 0.813011765480042
7.47435897435897 0.835904955863953
7.92948717948718 0.834197521209717
8.41025641025641 0.820302546024323
8.92307692307692 0.83158141374588
9.46794871794872 0.81036901473999
10.0448717948718 0.819624364376068
10.6602564102564 0.820828437805176
11.3076923076923 0.802255392074585
12 0.821942448616028
12.7307692307692 0.801949739456177
13.5064102564103 0.802090346813202
14.3333333333333 0.778367817401886
15.2051282051282 0.7927325963974
16.1346153846154 0.770501971244812
17.1153846153846 0.796294391155243
18.1602564102564 0.768989562988281
19.2692307692308 0.781743764877319
20.4423076923077 0.778632044792175
21.6858974358974 0.764838516712189
23.0128205128205 0.772388279438019
24.4102564102564 0.746501266956329
25.9038461538462 0.746868550777435
27.4807692307692 0.748911321163177
29.1538461538462 0.750665545463562
30.9358974358974 0.74033796787262
32.8205128205128 0.745612323284149
34.8205128205128 0.74505740404129
36.9423076923077 0.743563950061798
39.1923076923077 0.727583765983582
41.5833333333333 0.742683708667755
44.1153846153846 0.718596458435059
46.8076923076923 0.742733597755432
49.6602564102564 0.709457159042358
52.6858974358974 0.719332993030548
55.8974358974359 0.702495336532593
59.3076923076923 0.717082560062408
62.9230769230769 0.730219423770905
66.7564102564103 0.698431015014648
70.8269230769231 0.732516586780548
75.1474358974359 0.728855133056641
79.7307692307692 0.726793050765991
84.5897435897436 0.705692410469055
89.7435897435897 0.696116328239441
95.2179487179487 0.737302243709564
101.019230769231 0.709917366504669
107.179487179487 0.689447641372681
113.711538461538 0.714770436286926
120.641025641026 0.764359593391418
127.99358974359 0.725080549716949
135.801282051282 0.73695296049118
144.076923076923 0.722137153148651
152.858974358974 0.718161165714264
162.179487179487 0.720337212085724
172.064102564103 0.702482521533966
182.551282051282 0.697492182254791
193.679487179487 0.709788620471954
205.487179487179 0.696412205696106
218.012820512821 0.688946723937988
231.301282051282 0.715047895908356
245.403846153846 0.742136657238007
260.358974358974 0.73305881023407
276.230769230769 0.72188138961792
293.070512820513 0.706073582172394
310.935897435897 0.726127862930298
329.884615384615 0.710666000843048
350 0.714021384716034
};
\addplot [, color1, opacity=0.6, mark=square*, mark size=0.5, mark options={solid}, only marks, forget plot]
table {%
1 0.921218931674957
1.05769230769231 0.907539069652557
1.12179487179487 0.926560938358307
1.19230769230769 0.911590099334717
1.26282051282051 0.908619821071625
1.33974358974359 0.911460638046265
1.42307692307692 0.926040947437286
1.51282051282051 0.894256711006165
1.6025641025641 0.860710442066193
1.69871794871795 0.865254044532776
1.80128205128205 0.874039590358734
1.91666666666667 0.843354880809784
2.03205128205128 0.841761350631714
2.15384615384615 0.868118524551392
2.28846153846154 0.853881001472473
2.42307692307692 0.833199262619019
2.57692307692308 0.807535529136658
2.73076923076923 0.8390092253685
2.8974358974359 0.836057364940643
3.07692307692308 0.82785701751709
3.26282051282051 0.823021173477173
3.46153846153846 0.845201253890991
3.67307692307692 0.848641037940979
3.8974358974359 0.845335841178894
4.13461538461539 0.830774545669556
4.38461538461539 0.839201331138611
4.65384615384615 0.839012861251831
4.93589743589744 0.845549285411835
5.23717948717949 0.815747261047363
5.55769230769231 0.820959746837616
5.8974358974359 0.822837889194489
6.25641025641026 0.827582657337189
6.64102564102564 0.8221156001091
7.04487179487179 0.809728384017944
7.47435897435897 0.837195038795471
7.92948717948718 0.820951700210571
8.41025641025641 0.818071722984314
8.92307692307692 0.815687239170074
9.46794871794872 0.813625335693359
10.0448717948718 0.809137582778931
10.6602564102564 0.823672711849213
11.3076923076923 0.802860498428345
12 0.807098269462585
12.7307692307692 0.809897124767303
13.5064102564103 0.793535590171814
14.3333333333333 0.810489118099213
15.2051282051282 0.795965731143951
16.1346153846154 0.792035341262817
17.1153846153846 0.787177920341492
18.1602564102564 0.770757436752319
19.2692307692308 0.779329180717468
20.4423076923077 0.784445941448212
21.6858974358974 0.779057919979095
23.0128205128205 0.764607548713684
24.4102564102564 0.773333847522736
25.9038461538462 0.758629620075226
27.4807692307692 0.771482348442078
29.1538461538462 0.753666579723358
30.9358974358974 0.756474733352661
32.8205128205128 0.734402894973755
34.8205128205128 0.757445871829987
36.9423076923077 0.731889069080353
39.1923076923077 0.754440903663635
41.5833333333333 0.75377082824707
44.1153846153846 0.720771253108978
46.8076923076923 0.739096820354462
49.6602564102564 0.758776247501373
52.6858974358974 0.734833955764771
55.8974358974359 0.71284294128418
59.3076923076923 0.756323635578156
62.9230769230769 0.739389300346375
66.7564102564103 0.740710198879242
70.8269230769231 0.696313321590424
75.1474358974359 0.736386120319366
79.7307692307692 0.729203224182129
84.5897435897436 0.695161700248718
89.7435897435897 0.748430490493774
95.2179487179487 0.738171458244324
101.019230769231 0.719768643379211
107.179487179487 0.72112625837326
113.711538461538 0.706914663314819
120.641025641026 0.719811260700226
127.99358974359 0.70773184299469
135.801282051282 0.717873692512512
144.076923076923 0.720285952091217
152.858974358974 0.730775535106659
162.179487179487 0.708013594150543
172.064102564103 0.702820420265198
182.551282051282 0.702290654182434
193.679487179487 0.724627077579498
205.487179487179 0.738416194915771
218.012820512821 0.715100526809692
231.301282051282 0.728269279003143
245.403846153846 0.737527430057526
260.358974358974 0.716477334499359
276.230769230769 0.740606188774109
293.070512820513 0.721500277519226
310.935897435897 0.729542553424835
329.884615384615 0.723211407661438
350 0.704092979431152
};
\addplot [, color1, opacity=0.6, mark=square*, mark size=0.5, mark options={solid}, only marks, forget plot]
table {%
1 0.916057050228119
1.05769230769231 0.918076932430267
1.12179487179487 0.930907189846039
1.19230769230769 0.912520945072174
1.26282051282051 0.930588841438293
1.33974358974359 0.925143122673035
1.42307692307692 0.926280558109283
1.51282051282051 0.899732291698456
1.6025641025641 0.873821377754211
1.69871794871795 0.879424393177032
1.80128205128205 0.835189163684845
1.91666666666667 0.866747975349426
2.03205128205128 0.851498246192932
2.15384615384615 0.832865417003632
2.28846153846154 0.859002947807312
2.42307692307692 0.833979070186615
2.57692307692308 0.853875279426575
2.73076923076923 0.820719718933105
2.8974358974359 0.843808948993683
3.07692307692308 0.851402044296265
3.26282051282051 0.841019809246063
3.46153846153846 0.856058776378632
3.67307692307692 0.834814786911011
3.8974358974359 0.809754490852356
4.13461538461539 0.848446786403656
4.38461538461539 0.835762441158295
4.65384615384615 0.811248540878296
4.93589743589744 0.841135323047638
5.23717948717949 0.824494123458862
5.55769230769231 0.830772519111633
5.8974358974359 0.811329424381256
6.25641025641026 0.821707129478455
6.64102564102564 0.809893488883972
7.04487179487179 0.830104649066925
7.47435897435897 0.842915177345276
7.92948717948718 0.827158331871033
8.41025641025641 0.807721257209778
8.92307692307692 0.817796468734741
9.46794871794872 0.819351851940155
10.0448717948718 0.824605226516724
10.6602564102564 0.825037360191345
11.3076923076923 0.794584572315216
12 0.80288428068161
12.7307692307692 0.789073646068573
13.5064102564103 0.782297492027283
14.3333333333333 0.803570747375488
15.2051282051282 0.795892775058746
16.1346153846154 0.755213618278503
17.1153846153846 0.762868762016296
18.1602564102564 0.774142563343048
19.2692307692308 0.768839657306671
20.4423076923077 0.77180951833725
21.6858974358974 0.766599774360657
23.0128205128205 0.74063515663147
24.4102564102564 0.729494333267212
25.9038461538462 0.766710042953491
27.4807692307692 0.745421707630157
29.1538461538462 0.729682147502899
30.9358974358974 0.759487926959991
32.8205128205128 0.741780817508698
34.8205128205128 0.791276276111603
36.9423076923077 0.74979841709137
39.1923076923077 0.755863964557648
41.5833333333333 0.73871123790741
44.1153846153846 0.724460422992706
46.8076923076923 0.725739896297455
49.6602564102564 0.763168931007385
52.6858974358974 0.710173547267914
55.8974358974359 0.741186022758484
59.3076923076923 0.754783451557159
62.9230769230769 0.724052965641022
66.7564102564103 0.731796562671661
70.8269230769231 0.721516907215118
75.1474358974359 0.714088499546051
79.7307692307692 0.7331503033638
84.5897435897436 0.741695761680603
89.7435897435897 0.715101063251495
95.2179487179487 0.728611290454865
101.019230769231 0.729109346866608
107.179487179487 0.708878934383392
113.711538461538 0.712848782539368
120.641025641026 0.741324603557587
127.99358974359 0.727310597896576
135.801282051282 0.737441718578339
144.076923076923 0.723592519760132
152.858974358974 0.761046707630157
162.179487179487 0.742489397525787
172.064102564103 0.710858464241028
182.551282051282 0.71365761756897
193.679487179487 0.721174299716949
205.487179487179 0.732453286647797
218.012820512821 0.733183145523071
231.301282051282 0.701266825199127
245.403846153846 0.743224442005157
260.358974358974 0.663586795330048
276.230769230769 0.718144953250885
293.070512820513 0.732698321342468
310.935897435897 0.704092264175415
329.884615384615 0.7379429936409
350 0.723699331283569
};
\addplot [, color1, opacity=0.6, mark=square*, mark size=0.5, mark options={solid}, only marks, forget plot]
table {%
1 0.924145817756653
1.05769230769231 0.927922189235687
1.12179487179487 0.929546415805817
1.19230769230769 0.916543126106262
1.26282051282051 0.923245429992676
1.33974358974359 0.926817893981934
1.42307692307692 0.913119316101074
1.51282051282051 0.892318844795227
1.6025641025641 0.877053022384644
1.69871794871795 0.863240361213684
1.80128205128205 0.851514875888824
1.91666666666667 0.84632420539856
2.03205128205128 0.842624127864838
2.15384615384615 0.848289608955383
2.28846153846154 0.84556370973587
2.42307692307692 0.833992719650269
2.57692307692308 0.852254152297974
2.73076923076923 0.827507615089417
2.8974358974359 0.859226524829865
3.07692307692308 0.850180387496948
3.26282051282051 0.846479475498199
3.46153846153846 0.850580871105194
3.67307692307692 0.840810060501099
3.8974358974359 0.833078682422638
4.13461538461539 0.857012152671814
4.38461538461539 0.851222813129425
4.65384615384615 0.84191107749939
4.93589743589744 0.855356574058533
5.23717948717949 0.836727499961853
5.55769230769231 0.836978137493134
5.8974358974359 0.825623750686646
6.25641025641026 0.838660538196564
6.64102564102564 0.824098646640778
7.04487179487179 0.838161766529083
7.47435897435897 0.823459446430206
7.92948717948718 0.835310220718384
8.41025641025641 0.819376647472382
8.92307692307692 0.822742223739624
9.46794871794872 0.824240028858185
10.0448717948718 0.817242741584778
10.6602564102564 0.808787643909454
11.3076923076923 0.813059747219086
12 0.81121426820755
12.7307692307692 0.797608911991119
13.5064102564103 0.790081441402435
14.3333333333333 0.786209344863892
15.2051282051282 0.791637539863586
16.1346153846154 0.762528002262115
17.1153846153846 0.784332096576691
18.1602564102564 0.799434423446655
19.2692307692308 0.782906651496887
20.4423076923077 0.759523928165436
21.6858974358974 0.754217982292175
23.0128205128205 0.753812670707703
24.4102564102564 0.756678819656372
25.9038461538462 0.76713764667511
27.4807692307692 0.742891669273376
29.1538461538462 0.757013201713562
30.9358974358974 0.746866464614868
32.8205128205128 0.73893278837204
34.8205128205128 0.739836871623993
36.9423076923077 0.750670373439789
39.1923076923077 0.745740592479706
41.5833333333333 0.724116802215576
44.1153846153846 0.710143744945526
46.8076923076923 0.712833404541016
49.6602564102564 0.714381217956543
52.6858974358974 0.70675140619278
55.8974358974359 0.709993720054626
59.3076923076923 0.725559890270233
62.9230769230769 0.74949049949646
66.7564102564103 0.727689981460571
70.8269230769231 0.746931314468384
75.1474358974359 0.727557718753815
79.7307692307692 0.735905766487122
84.5897435897436 0.714363992214203
89.7435897435897 0.721586585044861
95.2179487179487 0.725729763507843
101.019230769231 0.725337326526642
107.179487179487 0.726125597953796
113.711538461538 0.74640828371048
120.641025641026 0.695107042789459
127.99358974359 0.717093169689178
135.801282051282 0.702060401439667
144.076923076923 0.711073815822601
152.858974358974 0.713372647762299
162.179487179487 0.758619129657745
172.064102564103 0.749653458595276
182.551282051282 0.722936987876892
193.679487179487 0.718969643115997
205.487179487179 0.721748173236847
218.012820512821 0.680116891860962
231.301282051282 0.695131957530975
245.403846153846 0.698901295661926
260.358974358974 0.743169248104095
276.230769230769 0.75443309545517
293.070512820513 0.748823702335358
310.935897435897 0.752195954322815
329.884615384615 0.76972758769989
350 0.715176820755005
};
\addplot [, color2, opacity=0.6, mark=triangle*, mark size=0.5, mark options={solid,rotate=180}, only marks]
table {%
1 0.68508106470108
1.05769230769231 0.66087418794632
1.12179487179487 0.661652565002441
1.19230769230769 0.610527992248535
1.26282051282051 0.619019508361816
1.33974358974359 0.600366115570068
1.42307692307692 0.626243412494659
1.51282051282051 0.513346254825592
1.6025641025641 0.544231355190277
1.69871794871795 0.534927785396576
1.80128205128205 0.558764338493347
1.91666666666667 0.528045237064362
2.03205128205128 0.571089386940002
2.15384615384615 0.566987216472626
2.28846153846154 0.533642172813416
2.42307692307692 0.543747186660767
2.57692307692308 0.549441635608673
2.73076923076923 0.449151605367661
2.8974358974359 0.559983611106873
3.07692307692308 0.513101696968079
3.26282051282051 0.473906397819519
3.46153846153846 0.486296385526657
3.67307692307692 0.511876821517944
3.8974358974359 0.466906815767288
4.13461538461539 0.487080216407776
4.38461538461539 0.500582814216614
4.65384615384615 0.465287774801254
4.93589743589744 0.512511432170868
5.23717948717949 0.493021160364151
5.55769230769231 0.478457033634186
5.8974358974359 0.459841459989548
6.25641025641026 0.450883477926254
6.64102564102564 0.437606304883957
7.04487179487179 0.439576327800751
7.47435897435897 0.431869715452194
7.92948717948718 0.422254025936127
8.41025641025641 0.432370811700821
8.92307692307692 0.418539643287659
9.46794871794872 0.425332099199295
10.0448717948718 0.384041666984558
10.6602564102564 0.385997653007507
11.3076923076923 0.369466006755829
12 0.344410091638565
12.7307692307692 0.365298807621002
13.5064102564103 0.340966671705246
14.3333333333333 0.392189860343933
15.2051282051282 0.348364442586899
16.1346153846154 0.329135894775391
17.1153846153846 0.386427789926529
18.1602564102564 0.351288586854935
19.2692307692308 0.34680837392807
20.4423076923077 0.371069490909576
21.6858974358974 0.368578404188156
23.0128205128205 0.334677815437317
24.4102564102564 0.355350077152252
25.9038461538462 0.37458860874176
27.4807692307692 0.326476246118546
29.1538461538462 0.346418231725693
30.9358974358974 0.283417582511902
32.8205128205128 0.2998386323452
34.8205128205128 0.283362954854965
36.9423076923077 0.278610080480576
39.1923076923077 0.220878258347511
41.5833333333333 0.251024961471558
44.1153846153846 0.249916568398476
46.8076923076923 0.233859494328499
49.6602564102564 0.25516203045845
52.6858974358974 0.250489443540573
55.8974358974359 0.22822605073452
59.3076923076923 0.207060232758522
62.9230769230769 0.234426036477089
66.7564102564103 0.211392670869827
70.8269230769231 0.203728899359703
75.1474358974359 0.219986125826836
79.7307692307692 0.207800596952438
84.5897435897436 0.221369415521622
89.7435897435897 0.207707017660141
95.2179487179487 0.178387120366096
101.019230769231 0.189045831561089
107.179487179487 0.212529018521309
113.711538461538 0.202345803380013
120.641025641026 0.202510714530945
127.99358974359 0.189659655094147
135.801282051282 0.199289351701736
144.076923076923 0.202908962965012
152.858974358974 0.19020102918148
162.179487179487 0.225462675094604
172.064102564103 0.191262617707253
182.551282051282 0.2060257345438
193.679487179487 0.19772769510746
205.487179487179 0.193183839321136
218.012820512821 0.191968113183975
231.301282051282 0.158046692609787
245.403846153846 0.192499533295631
260.358974358974 0.211127758026123
276.230769230769 0.200536906719208
293.070512820513 0.195338934659958
310.935897435897 0.197893708944321
329.884615384615 0.191829293966293
350 0.191689789295197
};
\addlegendentry{sub 16, mc 10}
\addplot [, color2, opacity=0.6, mark=triangle*, mark size=0.5, mark options={solid,rotate=180}, only marks, forget plot]
table {%
1 0.692606091499329
1.05769230769231 0.650279819965363
1.12179487179487 0.637342274188995
1.19230769230769 0.628862977027893
1.26282051282051 0.619979679584503
1.33974358974359 0.637672305107117
1.42307692307692 0.6446772813797
1.51282051282051 0.606335639953613
1.6025641025641 0.545744895935059
1.69871794871795 0.551894009113312
1.80128205128205 0.570977628231049
1.91666666666667 0.572828471660614
2.03205128205128 0.544022262096405
2.15384615384615 0.576786994934082
2.28846153846154 0.595273852348328
2.42307692307692 0.579102098941803
2.57692307692308 0.539643466472626
2.73076923076923 0.565832257270813
2.8974358974359 0.541889548301697
3.07692307692308 0.528311610221863
3.26282051282051 0.518778443336487
3.46153846153846 0.531238377094269
3.67307692307692 0.511827170848846
3.8974358974359 0.47942054271698
4.13461538461539 0.519092977046967
4.38461538461539 0.501289010047913
4.65384615384615 0.503649115562439
4.93589743589744 0.523295938968658
5.23717948717949 0.515287399291992
5.55769230769231 0.496485739946365
5.8974358974359 0.500594854354858
6.25641025641026 0.474339336156845
6.64102564102564 0.454821079969406
7.04487179487179 0.456184387207031
7.47435897435897 0.466998368501663
7.92948717948718 0.446526110172272
8.41025641025641 0.422366142272949
8.92307692307692 0.452616274356842
9.46794871794872 0.40812960267067
10.0448717948718 0.424041599035263
10.6602564102564 0.431695163249969
11.3076923076923 0.420599967241287
12 0.418765068054199
12.7307692307692 0.401979088783264
13.5064102564103 0.395127475261688
14.3333333333333 0.381410598754883
15.2051282051282 0.383760750293732
16.1346153846154 0.395243674516678
17.1153846153846 0.346808761358261
18.1602564102564 0.377295672893524
19.2692307692308 0.361148685216904
20.4423076923077 0.396028190851212
21.6858974358974 0.399606019258499
23.0128205128205 0.317532241344452
24.4102564102564 0.368979245424271
25.9038461538462 0.338321954011917
27.4807692307692 0.397363871335983
29.1538461538462 0.368570387363434
30.9358974358974 0.349543452262878
32.8205128205128 0.315456837415695
34.8205128205128 0.340048909187317
36.9423076923077 0.32190665602684
39.1923076923077 0.318432241678238
41.5833333333333 0.305241852998734
44.1153846153846 0.248879611492157
46.8076923076923 0.271497547626495
49.6602564102564 0.307065963745117
52.6858974358974 0.276867836713791
55.8974358974359 0.265297949314117
59.3076923076923 0.296791017055511
62.9230769230769 0.229445070028305
66.7564102564103 0.255035400390625
70.8269230769231 0.25339737534523
75.1474358974359 0.24343204498291
79.7307692307692 0.245492681860924
84.5897435897436 0.234198838472366
89.7435897435897 0.225865393877029
95.2179487179487 0.231799468398094
101.019230769231 0.264721661806107
107.179487179487 0.210601150989532
113.711538461538 0.213272660970688
120.641025641026 0.204000741243362
127.99358974359 0.247636675834656
135.801282051282 0.258658319711685
144.076923076923 0.198829710483551
152.858974358974 0.211017072200775
162.179487179487 0.254148334264755
172.064102564103 0.235590249300003
182.551282051282 0.215362578630447
193.679487179487 0.214536145329475
205.487179487179 0.232288464903831
218.012820512821 0.183880567550659
231.301282051282 0.17010360956192
245.403846153846 0.244587972760201
260.358974358974 0.205288484692574
276.230769230769 0.18864917755127
293.070512820513 0.255223870277405
310.935897435897 0.209774285554886
329.884615384615 0.21392185986042
350 0.188685223460197
};
\addplot [, color2, opacity=0.6, mark=triangle*, mark size=0.5, mark options={solid,rotate=180}, only marks, forget plot]
table {%
1 0.695305705070496
1.05769230769231 0.655203282833099
1.12179487179487 0.640068054199219
1.19230769230769 0.622300624847412
1.26282051282051 0.626141488552094
1.33974358974359 0.619978070259094
1.42307692307692 0.603269994258881
1.51282051282051 0.60606986284256
1.6025641025641 0.555340588092804
1.69871794871795 0.534297704696655
1.80128205128205 0.551537156105042
1.91666666666667 0.542018830776215
2.03205128205128 0.549703121185303
2.15384615384615 0.604572355747223
2.28846153846154 0.554520189762115
2.42307692307692 0.531350910663605
2.57692307692308 0.507184863090515
2.73076923076923 0.508043348789215
2.8974358974359 0.521057069301605
3.07692307692308 0.518930971622467
3.26282051282051 0.476943969726562
3.46153846153846 0.512081444263458
3.67307692307692 0.491487443447113
3.8974358974359 0.495128005743027
4.13461538461539 0.504884302616119
4.38461538461539 0.476865082979202
4.65384615384615 0.481262803077698
4.93589743589744 0.503276407718658
5.23717948717949 0.496780008077621
5.55769230769231 0.487568229436874
5.8974358974359 0.474168926477432
6.25641025641026 0.483692497014999
6.64102564102564 0.441180944442749
7.04487179487179 0.431006550788879
7.47435897435897 0.480631470680237
7.92948717948718 0.452740013599396
8.41025641025641 0.436651408672333
8.92307692307692 0.435332357883453
9.46794871794872 0.414337366819382
10.0448717948718 0.405592828989029
10.6602564102564 0.423019886016846
11.3076923076923 0.391844630241394
12 0.397182375192642
12.7307692307692 0.376242816448212
13.5064102564103 0.369332432746887
14.3333333333333 0.360490709543228
15.2051282051282 0.340938687324524
16.1346153846154 0.338099539279938
17.1153846153846 0.353080540895462
18.1602564102564 0.317406445741653
19.2692307692308 0.354744553565979
20.4423076923077 0.344069629907608
21.6858974358974 0.337185055017471
23.0128205128205 0.331666946411133
24.4102564102564 0.333627820014954
25.9038461538462 0.346547603607178
27.4807692307692 0.328998565673828
29.1538461538462 0.31771844625473
30.9358974358974 0.315305560827255
32.8205128205128 0.3155597448349
34.8205128205128 0.335287511348724
36.9423076923077 0.284095674753189
39.1923076923077 0.308559328317642
41.5833333333333 0.270900905132294
44.1153846153846 0.301446348428726
46.8076923076923 0.305576682090759
49.6602564102564 0.267441719770432
52.6858974358974 0.240946114063263
55.8974358974359 0.278340041637421
59.3076923076923 0.331883311271667
62.9230769230769 0.242965042591095
66.7564102564103 0.21841536462307
70.8269230769231 0.229594051837921
75.1474358974359 0.218725100159645
79.7307692307692 0.27863684296608
84.5897435897436 0.224088162183762
89.7435897435897 0.197167411446571
95.2179487179487 0.222420111298561
101.019230769231 0.205639779567719
107.179487179487 0.219042003154755
113.711538461538 0.216682225465775
120.641025641026 0.162360161542892
127.99358974359 0.1835847645998
135.801282051282 0.202766373753548
144.076923076923 0.199882924556732
152.858974358974 0.228356018662453
162.179487179487 0.221716895699501
172.064102564103 0.174896374344826
182.551282051282 0.175827026367188
193.679487179487 0.188880503177643
205.487179487179 0.204511985182762
218.012820512821 0.216891080141068
231.301282051282 0.199518546462059
245.403846153846 0.211396962404251
260.358974358974 0.197102680802345
276.230769230769 0.164916381239891
293.070512820513 0.195772513747215
310.935897435897 0.233328506350517
329.884615384615 0.188434615731239
350 0.217441663146019
};
\addplot [, color2, opacity=0.6, mark=triangle*, mark size=0.5, mark options={solid,rotate=180}, only marks, forget plot]
table {%
1 0.671213507652283
1.05769230769231 0.647183656692505
1.12179487179487 0.639871954917908
1.19230769230769 0.612515687942505
1.26282051282051 0.620424687862396
1.33974358974359 0.586354792118073
1.42307692307692 0.580841481685638
1.51282051282051 0.46063169836998
1.6025641025641 0.441307961940765
1.69871794871795 0.437108844518661
1.80128205128205 0.475277155637741
1.91666666666667 0.475372940301895
2.03205128205128 0.489798426628113
2.15384615384615 0.5139000415802
2.28846153846154 0.458277881145477
2.42307692307692 0.528634428977966
2.57692307692308 0.46423727273941
2.73076923076923 0.492889970541
2.8974358974359 0.483532965183258
3.07692307692308 0.490634679794312
3.26282051282051 0.445988982915878
3.46153846153846 0.453546732664108
3.67307692307692 0.454425901174545
3.8974358974359 0.478317856788635
4.13461538461539 0.458399415016174
4.38461538461539 0.437852531671524
4.65384615384615 0.447223305702209
4.93589743589744 0.501933693885803
5.23717948717949 0.452463716268539
5.55769230769231 0.448493123054504
5.8974358974359 0.458173394203186
6.25641025641026 0.467470854520798
6.64102564102564 0.43584144115448
7.04487179487179 0.427344858646393
7.47435897435897 0.438781499862671
7.92948717948718 0.427267134189606
8.41025641025641 0.415301591157913
8.92307692307692 0.432559967041016
9.46794871794872 0.415840834379196
10.0448717948718 0.40339395403862
10.6602564102564 0.398860543966293
11.3076923076923 0.379258006811142
12 0.363461226224899
12.7307692307692 0.365635067224503
13.5064102564103 0.346490114927292
14.3333333333333 0.360361158847809
15.2051282051282 0.343751817941666
16.1346153846154 0.335422217845917
17.1153846153846 0.370491713285446
18.1602564102564 0.358397662639618
19.2692307692308 0.358752816915512
20.4423076923077 0.400478839874268
21.6858974358974 0.347515374422073
23.0128205128205 0.382539957761765
24.4102564102564 0.324851453304291
25.9038461538462 0.374162673950195
27.4807692307692 0.34186714887619
29.1538461538462 0.34895196557045
30.9358974358974 0.316679447889328
32.8205128205128 0.327801287174225
34.8205128205128 0.309395909309387
36.9423076923077 0.330883711576462
39.1923076923077 0.318531185388565
41.5833333333333 0.313966453075409
44.1153846153846 0.293682247400284
46.8076923076923 0.363837271928787
49.6602564102564 0.295662760734558
52.6858974358974 0.28606241941452
55.8974358974359 0.280127316713333
59.3076923076923 0.261923283338547
62.9230769230769 0.240508288145065
66.7564102564103 0.27230316400528
70.8269230769231 0.246747627854347
75.1474358974359 0.21163871884346
79.7307692307692 0.239247739315033
84.5897435897436 0.23553729057312
89.7435897435897 0.240788742899895
95.2179487179487 0.20213408768177
101.019230769231 0.199037358164787
107.179487179487 0.301194041967392
113.711538461538 0.268076330423355
120.641025641026 0.200561881065369
127.99358974359 0.218199491500854
135.801282051282 0.214993000030518
144.076923076923 0.209330216050148
152.858974358974 0.206874027848244
162.179487179487 0.258429378271103
172.064102564103 0.233392938971519
182.551282051282 0.234018206596375
193.679487179487 0.224439203739166
205.487179487179 0.197729215025902
218.012820512821 0.168540507555008
231.301282051282 0.202254369854927
245.403846153846 0.190700814127922
260.358974358974 0.24010893702507
276.230769230769 0.220711663365364
293.070512820513 0.234306156635284
310.935897435897 0.245648741722107
329.884615384615 0.261036485433578
350 0.225633278489113
};
\addplot [, color2, opacity=0.6, mark=triangle*, mark size=0.5, mark options={solid,rotate=180}, only marks, forget plot]
table {%
1 0.710713565349579
1.05769230769231 0.67939567565918
1.12179487179487 0.654001295566559
1.19230769230769 0.647691309452057
1.26282051282051 0.649094820022583
1.33974358974359 0.669391632080078
1.42307692307692 0.652797222137451
1.51282051282051 0.599964380264282
1.6025641025641 0.590945780277252
1.69871794871795 0.571211934089661
1.80128205128205 0.561746180057526
1.91666666666667 0.562033534049988
2.03205128205128 0.566722095012665
2.15384615384615 0.569714367389679
2.28846153846154 0.559767425060272
2.42307692307692 0.552459120750427
2.57692307692308 0.549789369106293
2.73076923076923 0.463083863258362
2.8974358974359 0.584400773048401
3.07692307692308 0.547395288944244
3.26282051282051 0.517797231674194
3.46153846153846 0.525991499423981
3.67307692307692 0.528686165809631
3.8974358974359 0.513725638389587
4.13461538461539 0.521929204463959
4.38461538461539 0.51346480846405
4.65384615384615 0.471559137105942
4.93589743589744 0.514398097991943
5.23717948717949 0.507175803184509
5.55769230769231 0.471127688884735
5.8974358974359 0.466412574052811
6.25641025641026 0.501344680786133
6.64102564102564 0.460088610649109
7.04487179487179 0.439247161149979
7.47435897435897 0.444207370281219
7.92948717948718 0.441365867853165
8.41025641025641 0.418585896492004
8.92307692307692 0.434144079685211
9.46794871794872 0.411167174577713
10.0448717948718 0.433527201414108
10.6602564102564 0.397725522518158
11.3076923076923 0.402263522148132
12 0.407755434513092
12.7307692307692 0.398100465536118
13.5064102564103 0.39450877904892
14.3333333333333 0.397728949785233
15.2051282051282 0.386945396661758
16.1346153846154 0.407445013523102
17.1153846153846 0.38619464635849
18.1602564102564 0.38150480389595
19.2692307692308 0.369074076414108
20.4423076923077 0.35505598783493
21.6858974358974 0.352026611566544
23.0128205128205 0.345464736223221
24.4102564102564 0.29752704501152
25.9038461538462 0.36726576089859
27.4807692307692 0.279569894075394
29.1538461538462 0.292290896177292
30.9358974358974 0.317397952079773
32.8205128205128 0.280919879674911
34.8205128205128 0.274915963411331
36.9423076923077 0.283323496580124
39.1923076923077 0.264188528060913
41.5833333333333 0.274872541427612
44.1153846153846 0.245837971568108
46.8076923076923 0.266014873981476
49.6602564102564 0.258613049983978
52.6858974358974 0.208517029881477
55.8974358974359 0.241394385695457
59.3076923076923 0.200696006417274
62.9230769230769 0.223444327712059
66.7564102564103 0.251863479614258
70.8269230769231 0.219272196292877
75.1474358974359 0.198321223258972
79.7307692307692 0.224110662937164
84.5897435897436 0.161760538816452
89.7435897435897 0.185537472367287
95.2179487179487 0.171417862176895
101.019230769231 0.231606349349022
107.179487179487 0.1800587028265
113.711538461538 0.173021033406258
120.641025641026 0.167860701680183
127.99358974359 0.18875789642334
135.801282051282 0.164582207798958
144.076923076923 0.215291947126389
152.858974358974 0.183410197496414
162.179487179487 0.183155372738838
172.064102564103 0.164063215255737
182.551282051282 0.198841288685799
193.679487179487 0.177082359790802
205.487179487179 0.20552434027195
218.012820512821 0.160842761397362
231.301282051282 0.195437774062157
245.403846153846 0.204711526632309
260.358974358974 0.20160411298275
276.230769230769 0.170120045542717
293.070512820513 0.202779054641724
310.935897435897 0.147987321019173
329.884615384615 0.194676369428635
350 0.136876836419106
};
\end{axis}

\end{tikzpicture}

    \tikzexternaldisable
  \end{minipage}\hfill
  \begin{minipage}{0.50\linewidth}
    \centering
    % defines the pgfplots style "eigspacedefault"
\pgfkeys{/pgfplots/eigspacedefault/.style={
    width=1.03\linewidth,
    height=\goldenRatioInv*1.03*\linewidth,
    every axis plot/.append style={line width = 1pt},
    tick pos = left,
    ylabel near ticks,
    xlabel near ticks,
    xtick align = inside,
    ytick align = inside,
    legend cell align = left,
    legend columns = 1,
    legend pos = north east,
    legend style = {
      fill opacity = 0.9,
      text opacity = 1,
      font = \tiny,
      % column sep=0.1cm,
    },
    legend image post style={scale=2},
    xticklabel style = {font = \small},
    xlabel style = {font = \small},
    axis line style = {black},
    yticklabel style = {font = \small},
    ylabel style = {font = \small},
    title style = {font = \small},
    grid = major,
    grid style = {dashed}
  }
}

\pgfkeys{/pgfplots/eigspacedefaultapp/.style={
    eigspacedefault,
    height=0.6\linewidth,
    legend columns = 2,
  }
}

\pgfkeys{/pgfplots/eigspacenolegend/.style={
    legend image post style = {scale=0},
    legend style = {
      fill opacity = 0,
      draw opacity = 0,
      text opacity = 0,
      font = \small,
      at={(1, 1.025)},
      anchor=south east,
      column sep=0.25cm,
    },
  }
}
%%% Local Variables:
%%% mode: latex
%%% TeX-master: "../main"
%%% End:

    \pgfkeys{/pgfplots/zmystyle/.style={
        eigspacedefaultapp,
        legend columns = 3,
        eigspacenolegend,
      }}
    \tikzexternalenable
    \vspace{-3ex}
    % This file was created by tikzplotlib v0.9.7.
\begin{tikzpicture}

\definecolor{color0}{rgb}{0.274509803921569,0.6,0.564705882352941}
\definecolor{color1}{rgb}{0.870588235294118,0.623529411764706,0.0862745098039216}
\definecolor{color2}{rgb}{0.501960784313725,0.184313725490196,0.6}

\begin{axis}[
axis line style={white!10!black},
legend style={fill opacity=0.8, draw opacity=1, text opacity=1, at={(0.03,0.03)}, anchor=south west, draw=white!80!black},
log basis x={10},
tick pos=left,
xlabel={epoch (log scale)},
xmajorgrids,
xmin=0.746099240306814, xmax=469.106495613199,
xmode=log,
ylabel={overlap},
ymajorgrids,
ymin=-0.05, ymax=1.05,
zmystyle
]
\addplot [, white!10!black, dashed, forget plot]
table {%
0.746099240306814 1
469.106495613199 1
};
\addplot [, white!10!black, dashed, forget plot]
table {%
0.746099240306814 0
469.106495613199 0
};
\addplot [, color0, opacity=0.6, mark=diamond*, mark size=0.5, mark options={solid}, only marks]
table {%
1 0.908253312110901
1.05769230769231 0.922341883182526
1.12179487179487 0.930884540081024
1.19230769230769 0.942146718502045
1.26282051282051 0.919140934944153
1.33974358974359 0.910205364227295
1.42307692307692 0.912995457649231
1.51282051282051 0.902219474315643
1.6025641025641 0.900016844272614
1.69871794871795 0.868671119213104
1.80128205128205 0.851147294044495
1.91666666666667 0.81996750831604
2.03205128205128 0.826496720314026
2.15384615384615 0.856815457344055
2.28846153846154 0.83719277381897
2.42307692307692 0.81653243303299
2.57692307692308 0.796987533569336
2.73076923076923 0.819041550159454
2.8974358974359 0.802467465400696
3.07692307692308 0.812875986099243
3.26282051282051 0.776677906513214
3.46153846153846 0.804302215576172
3.67307692307692 0.76967054605484
3.8974358974359 0.73759663105011
4.13461538461539 0.783335089683533
4.38461538461539 0.743171036243439
4.65384615384615 0.76623922586441
4.93589743589744 0.754954993724823
5.23717948717949 0.742784798145294
5.55769230769231 0.750972211360931
5.8974358974359 0.668011009693146
6.25641025641026 0.654925465583801
6.64102564102564 0.700164616107941
7.04487179487179 0.678082406520844
7.47435897435897 0.676306307315826
7.92948717948718 0.646399915218353
8.41025641025641 0.675457417964935
8.92307692307692 0.641982436180115
9.46794871794872 0.641972124576569
10.0448717948718 0.65328460931778
10.6602564102564 0.63255387544632
11.3076923076923 0.610514461994171
12 0.57517945766449
12.7307692307692 0.596145808696747
13.5064102564103 0.575676202774048
14.3333333333333 0.551617801189423
15.2051282051282 0.565923929214478
16.1346153846154 0.554842352867126
17.1153846153846 0.535389542579651
18.1602564102564 0.53792941570282
19.2692307692308 0.5347039103508
20.4423076923077 0.505327045917511
21.6858974358974 0.507860898971558
23.0128205128205 0.518848478794098
24.4102564102564 0.496431022882462
25.9038461538462 0.486910909414291
27.4807692307692 0.501863241195679
29.1538461538462 0.503057420253754
30.9358974358974 0.494059205055237
32.8205128205128 0.507228076457977
34.8205128205128 0.492959767580032
36.9423076923077 0.504646837711334
39.1923076923077 0.465039193630219
41.5833333333333 0.452243268489838
44.1153846153846 0.467781066894531
46.8076923076923 0.45063591003418
49.6602564102564 0.426152020692825
52.6858974358974 0.445585787296295
55.8974358974359 0.43711906671524
59.3076923076923 0.42331662774086
62.9230769230769 0.415903449058533
66.7564102564103 0.41527071595192
70.8269230769231 0.422233611345291
75.1474358974359 0.414836764335632
79.7307692307692 0.404572755098343
84.5897435897436 0.396961599588394
89.7435897435897 0.397258818149567
95.2179487179487 0.387589573860168
101.019230769231 0.384171813726425
107.179487179487 0.363438814878464
113.711538461538 0.379161477088928
120.641025641026 0.38197261095047
127.99358974359 0.387970268726349
135.801282051282 0.391326755285263
144.076923076923 0.382451236248016
152.858974358974 0.360764294862747
162.179487179487 0.373680651187897
172.064102564103 0.360010534524918
182.551282051282 0.365965187549591
193.679487179487 0.356352597475052
205.487179487179 0.356350004673004
218.012820512821 0.381245076656342
231.301282051282 0.33203786611557
245.403846153846 0.357529670000076
260.358974358974 0.363813608884811
276.230769230769 0.352246284484863
293.070512820513 0.345279633998871
310.935897435897 0.361688554286957
329.884615384615 0.380432575941086
350 0.344984531402588
};
\addlegendentry{sub 16, exact}
\addplot [, color0, opacity=0.6, mark=diamond*, mark size=0.5, mark options={solid}, only marks, forget plot]
table {%
1 0.850025832653046
1.05769230769231 0.888059198856354
1.12179487179487 0.893580734729767
1.19230769230769 0.923091232776642
1.26282051282051 0.875908315181732
1.33974358974359 0.816000044345856
1.42307692307692 0.80831116437912
1.51282051282051 0.79279613494873
1.6025641025641 0.778436899185181
1.69871794871795 0.753804206848145
1.80128205128205 0.699002504348755
1.91666666666667 0.753850936889648
2.03205128205128 0.765199542045593
2.15384615384615 0.72775799036026
2.28846153846154 0.787870466709137
2.42307692307692 0.772697508335114
2.57692307692308 0.787986874580383
2.73076923076923 0.732500433921814
2.8974358974359 0.781770706176758
3.07692307692308 0.701006531715393
3.26282051282051 0.733540177345276
3.46153846153846 0.766958594322205
3.67307692307692 0.729451417922974
3.8974358974359 0.748975992202759
4.13461538461539 0.751317262649536
4.38461538461539 0.733930349349976
4.65384615384615 0.727367401123047
4.93589743589744 0.729844629764557
5.23717948717949 0.721593022346497
5.55769230769231 0.71239310503006
5.8974358974359 0.653897821903229
6.25641025641026 0.663958370685577
6.64102564102564 0.65057235956192
7.04487179487179 0.641782164573669
7.47435897435897 0.619992613792419
7.92948717948718 0.627587199211121
8.41025641025641 0.627400815486908
8.92307692307692 0.572039246559143
9.46794871794872 0.582727193832397
10.0448717948718 0.586842119693756
10.6602564102564 0.566601634025574
11.3076923076923 0.55889892578125
12 0.52729719877243
12.7307692307692 0.537216126918793
13.5064102564103 0.513735353946686
14.3333333333333 0.52333664894104
15.2051282051282 0.499785900115967
16.1346153846154 0.524249970912933
17.1153846153846 0.503286361694336
18.1602564102564 0.505581140518188
19.2692307692308 0.492814868688583
20.4423076923077 0.476853728294373
21.6858974358974 0.452842384576797
23.0128205128205 0.470907479524612
24.4102564102564 0.474052041769028
25.9038461538462 0.454030632972717
27.4807692307692 0.462952524423599
29.1538461538462 0.45620658993721
30.9358974358974 0.45130717754364
32.8205128205128 0.448566287755966
34.8205128205128 0.447163164615631
36.9423076923077 0.445830464363098
39.1923076923077 0.420125871896744
41.5833333333333 0.438203871250153
44.1153846153846 0.42360708117485
46.8076923076923 0.416566073894501
49.6602564102564 0.413346230983734
52.6858974358974 0.426117092370987
55.8974358974359 0.414993494749069
59.3076923076923 0.405598849058151
62.9230769230769 0.391586869955063
66.7564102564103 0.388898313045502
70.8269230769231 0.365368783473969
75.1474358974359 0.393276065587997
79.7307692307692 0.379032820463181
84.5897435897436 0.373196095228195
89.7435897435897 0.363757938146591
95.2179487179487 0.369010001420975
101.019230769231 0.354555755853653
107.179487179487 0.351917862892151
113.711538461538 0.34237015247345
120.641025641026 0.348555088043213
127.99358974359 0.343830496072769
135.801282051282 0.313350945711136
144.076923076923 0.327949672937393
152.858974358974 0.348775714635849
162.179487179487 0.330742716789246
172.064102564103 0.320634067058563
182.551282051282 0.333329737186432
193.679487179487 0.35053214430809
205.487179487179 0.338732451200485
218.012820512821 0.328408032655716
231.301282051282 0.312008023262024
245.403846153846 0.314271956682205
260.358974358974 0.315717816352844
276.230769230769 0.321474730968475
293.070512820513 0.328138411045074
310.935897435897 0.336233973503113
329.884615384615 0.35118305683136
350 0.328409761190414
};
\addplot [, color0, opacity=0.6, mark=diamond*, mark size=0.5, mark options={solid}, only marks, forget plot]
table {%
1 0.928679466247559
1.05769230769231 0.921768009662628
1.12179487179487 0.929368436336517
1.19230769230769 0.911377370357513
1.26282051282051 0.921983182430267
1.33974358974359 0.89397007226944
1.42307692307692 0.887739300727844
1.51282051282051 0.895821213722229
1.6025641025641 0.8806232213974
1.69871794871795 0.88171124458313
1.80128205128205 0.843221664428711
1.91666666666667 0.830707967281342
2.03205128205128 0.834349274635315
2.15384615384615 0.861291468143463
2.28846153846154 0.834343612194061
2.42307692307692 0.796954989433289
2.57692307692308 0.780791461467743
2.73076923076923 0.775543987751007
2.8974358974359 0.730113506317139
3.07692307692308 0.744879305362701
3.26282051282051 0.76437771320343
3.46153846153846 0.783760190010071
3.67307692307692 0.749980449676514
3.8974358974359 0.738587856292725
4.13461538461539 0.758326709270477
4.38461538461539 0.748125493526459
4.65384615384615 0.750613689422607
4.93589743589744 0.738876461982727
5.23717948717949 0.735000967979431
5.55769230769231 0.722884476184845
5.8974358974359 0.676536679267883
6.25641025641026 0.67848265171051
6.64102564102564 0.700705707073212
7.04487179487179 0.668428182601929
7.47435897435897 0.64497709274292
7.92948717948718 0.632584512233734
8.41025641025641 0.629197120666504
8.92307692307692 0.621508359909058
9.46794871794872 0.607914745807648
10.0448717948718 0.588935971260071
10.6602564102564 0.573802709579468
11.3076923076923 0.56688529253006
12 0.522561013698578
12.7307692307692 0.527263641357422
13.5064102564103 0.515554130077362
14.3333333333333 0.500315308570862
15.2051282051282 0.48992446064949
16.1346153846154 0.510973036289215
17.1153846153846 0.503003358840942
18.1602564102564 0.497848182916641
19.2692307692308 0.498298168182373
20.4423076923077 0.475064903497696
21.6858974358974 0.476269215345383
23.0128205128205 0.477911204099655
24.4102564102564 0.450560450553894
25.9038461538462 0.450692117214203
27.4807692307692 0.475887656211853
29.1538461538462 0.451755553483963
30.9358974358974 0.451362133026123
32.8205128205128 0.465687096118927
34.8205128205128 0.462005466222763
36.9423076923077 0.418752163648605
39.1923076923077 0.418640404939651
41.5833333333333 0.417586594820023
44.1153846153846 0.428917527198792
46.8076923076923 0.419801324605942
49.6602564102564 0.417209029197693
52.6858974358974 0.408679157495499
55.8974358974359 0.401029944419861
59.3076923076923 0.390081793069839
62.9230769230769 0.395399391651154
66.7564102564103 0.394378662109375
70.8269230769231 0.383631855249405
75.1474358974359 0.387792646884918
79.7307692307692 0.376828670501709
84.5897435897436 0.4033023416996
89.7435897435897 0.379792004823685
95.2179487179487 0.3877914249897
101.019230769231 0.374140232801437
107.179487179487 0.373996943235397
113.711538461538 0.384448677301407
120.641025641026 0.347656011581421
127.99358974359 0.383551865816116
135.801282051282 0.363391667604446
144.076923076923 0.370554089546204
152.858974358974 0.368544310331345
162.179487179487 0.363972514867783
172.064102564103 0.346124351024628
182.551282051282 0.36381521821022
193.679487179487 0.353965073823929
205.487179487179 0.382845610380173
218.012820512821 0.37168288230896
231.301282051282 0.387990891933441
245.403846153846 0.381729274988174
260.358974358974 0.360295742750168
276.230769230769 0.363706231117249
293.070512820513 0.379066526889801
310.935897435897 0.402354806661606
329.884615384615 0.374571233987808
350 0.35294172167778
};
\addplot [, color0, opacity=0.6, mark=diamond*, mark size=0.5, mark options={solid}, only marks, forget plot]
table {%
1 0.945022344589233
1.05769230769231 0.943203151226044
1.12179487179487 0.917214035987854
1.19230769230769 0.886052310466766
1.26282051282051 0.902897417545319
1.33974358974359 0.893886268138885
1.42307692307692 0.865401268005371
1.51282051282051 0.854660332202911
1.6025641025641 0.831500887870789
1.69871794871795 0.785849213600159
1.80128205128205 0.804155647754669
1.91666666666667 0.781831622123718
2.03205128205128 0.789932250976562
2.15384615384615 0.775953650474548
2.28846153846154 0.767624855041504
2.42307692307692 0.791837155818939
2.57692307692308 0.732231140136719
2.73076923076923 0.771580159664154
2.8974358974359 0.745230555534363
3.07692307692308 0.763864099979401
3.26282051282051 0.749298453330994
3.46153846153846 0.775178968906403
3.67307692307692 0.731133699417114
3.8974358974359 0.735632479190826
4.13461538461539 0.754192352294922
4.38461538461539 0.725134134292603
4.65384615384615 0.728803098201752
4.93589743589744 0.745698928833008
5.23717948717949 0.728285670280457
5.55769230769231 0.717329144477844
5.8974358974359 0.682875216007233
6.25641025641026 0.672109663486481
6.64102564102564 0.695178806781769
7.04487179487179 0.637244582176208
7.47435897435897 0.631074726581573
7.92948717948718 0.613128423690796
8.41025641025641 0.618059396743774
8.92307692307692 0.593435168266296
9.46794871794872 0.577675998210907
10.0448717948718 0.60308837890625
10.6602564102564 0.568313956260681
11.3076923076923 0.549082636833191
12 0.540174245834351
12.7307692307692 0.54734742641449
13.5064102564103 0.539418160915375
14.3333333333333 0.524245083332062
15.2051282051282 0.519679248332977
16.1346153846154 0.506890952587128
17.1153846153846 0.50753253698349
18.1602564102564 0.501852691173553
19.2692307692308 0.504765331745148
20.4423076923077 0.483956903219223
21.6858974358974 0.467447549104691
23.0128205128205 0.47777533531189
24.4102564102564 0.455589890480042
25.9038461538462 0.444614470005035
27.4807692307692 0.438031375408173
29.1538461538462 0.443132251501083
30.9358974358974 0.428252846002579
32.8205128205128 0.43818998336792
34.8205128205128 0.423566877841949
36.9423076923077 0.41444805264473
39.1923076923077 0.407517910003662
41.5833333333333 0.38951712846756
44.1153846153846 0.417507648468018
46.8076923076923 0.406643778085709
49.6602564102564 0.3892502784729
52.6858974358974 0.386926501989365
55.8974358974359 0.402001440525055
59.3076923076923 0.395663976669312
62.9230769230769 0.411407232284546
66.7564102564103 0.407477408647537
70.8269230769231 0.378914326429367
75.1474358974359 0.359110325574875
79.7307692307692 0.364143282175064
84.5897435897436 0.362417936325073
89.7435897435897 0.373323738574982
95.2179487179487 0.333730399608612
101.019230769231 0.341765999794006
107.179487179487 0.335815727710724
113.711538461538 0.350803971290588
120.641025641026 0.328938961029053
127.99358974359 0.355592638254166
135.801282051282 0.317004650831223
144.076923076923 0.339187502861023
152.858974358974 0.320859670639038
162.179487179487 0.333070933818817
172.064102564103 0.319953858852386
182.551282051282 0.323918223381042
193.679487179487 0.327313810586929
205.487179487179 0.311169564723969
218.012820512821 0.324245572090149
231.301282051282 0.330102980136871
245.403846153846 0.332239091396332
260.358974358974 0.324193030595779
276.230769230769 0.303581953048706
293.070512820513 0.321353673934937
310.935897435897 0.342518299818039
329.884615384615 0.315311431884766
350 0.315075218677521
};
\addplot [, color0, opacity=0.6, mark=diamond*, mark size=0.5, mark options={solid}, only marks, forget plot]
table {%
1 0.931593477725983
1.05769230769231 0.926207721233368
1.12179487179487 0.912877380847931
1.19230769230769 0.919090569019318
1.26282051282051 0.867645561695099
1.33974358974359 0.833866715431213
1.42307692307692 0.758153975009918
1.51282051282051 0.708062410354614
1.6025641025641 0.7771115899086
1.69871794871795 0.756313025951385
1.80128205128205 0.766495823860168
1.91666666666667 0.692455112934113
2.03205128205128 0.758466243743896
2.15384615384615 0.724766194820404
2.28846153846154 0.733971893787384
2.42307692307692 0.745933830738068
2.57692307692308 0.729028463363647
2.73076923076923 0.711665034294128
2.8974358974359 0.757590472698212
3.07692307692308 0.728971183300018
3.26282051282051 0.698746144771576
3.46153846153846 0.77627569437027
3.67307692307692 0.762063205242157
3.8974358974359 0.709684729576111
4.13461538461539 0.755113303661346
4.38461538461539 0.753587305545807
4.65384615384615 0.748975038528442
4.93589743589744 0.74861878156662
5.23717948717949 0.734969317913055
5.55769230769231 0.708567082881927
5.8974358974359 0.680561423301697
6.25641025641026 0.6420037150383
6.64102564102564 0.678409934043884
7.04487179487179 0.664624452590942
7.47435897435897 0.644250571727753
7.92948717948718 0.648670017719269
8.41025641025641 0.646698772907257
8.92307692307692 0.612875938415527
9.46794871794872 0.627766370773315
10.0448717948718 0.630701303482056
10.6602564102564 0.617495954036713
11.3076923076923 0.591489255428314
12 0.573533296585083
12.7307692307692 0.590802013874054
13.5064102564103 0.575067460536957
14.3333333333333 0.576078712940216
15.2051282051282 0.553795635700226
16.1346153846154 0.56362110376358
17.1153846153846 0.545875549316406
18.1602564102564 0.553661704063416
19.2692307692308 0.53053081035614
20.4423076923077 0.533121287822723
21.6858974358974 0.502185106277466
23.0128205128205 0.527657032012939
24.4102564102564 0.507937550544739
25.9038461538462 0.504244804382324
27.4807692307692 0.499019831418991
29.1538461538462 0.498383820056915
30.9358974358974 0.48573836684227
32.8205128205128 0.494745820760727
34.8205128205128 0.485493689775467
36.9423076923077 0.471847534179688
39.1923076923077 0.474322766065598
41.5833333333333 0.477566987276077
44.1153846153846 0.470634132623672
46.8076923076923 0.459482640028
49.6602564102564 0.460596233606339
52.6858974358974 0.447112411260605
55.8974358974359 0.459837168455124
59.3076923076923 0.463533759117126
62.9230769230769 0.460386127233505
66.7564102564103 0.4404416680336
70.8269230769231 0.436283707618713
75.1474358974359 0.441671788692474
79.7307692307692 0.423863679170609
84.5897435897436 0.463803201913834
89.7435897435897 0.448863208293915
95.2179487179487 0.418738424777985
101.019230769231 0.420676022768021
107.179487179487 0.409487068653107
113.711538461538 0.414679318666458
120.641025641026 0.410559445619583
127.99358974359 0.390869528055191
135.801282051282 0.42719954252243
144.076923076923 0.418708264827728
152.858974358974 0.401182919740677
162.179487179487 0.435155034065247
172.064102564103 0.423938602209091
182.551282051282 0.413651496171951
193.679487179487 0.436474442481995
205.487179487179 0.398996412754059
218.012820512821 0.407286524772644
231.301282051282 0.387385159730911
245.403846153846 0.394710391759872
260.358974358974 0.413315504789352
276.230769230769 0.436504036188126
293.070512820513 0.427225321531296
310.935897435897 0.39719220995903
329.884615384615 0.405796110630035
350 0.419167995452881
};
\addplot [, color1, opacity=0.6, mark=square*, mark size=0.5, mark options={solid}, only marks]
table {%
1 0.933815598487854
1.05769230769231 0.944805562496185
1.12179487179487 0.92095673084259
1.19230769230769 0.906801760196686
1.26282051282051 0.882634401321411
1.33974358974359 0.865252494812012
1.42307692307692 0.861879944801331
1.51282051282051 0.858514070510864
1.6025641025641 0.87509948015213
1.69871794871795 0.85326224565506
1.80128205128205 0.849676191806793
1.91666666666667 0.870068967342377
2.03205128205128 0.850356757640839
2.15384615384615 0.873829960823059
2.28846153846154 0.851638495922089
2.42307692307692 0.854382395744324
2.57692307692308 0.849307179450989
2.73076923076923 0.839812994003296
2.8974358974359 0.856421768665314
3.07692307692308 0.845251083374023
3.26282051282051 0.84521222114563
3.46153846153846 0.881579279899597
3.67307692307692 0.862463593482971
3.8974358974359 0.843414604663849
4.13461538461539 0.887287735939026
4.38461538461539 0.874706268310547
4.65384615384615 0.879142582416534
4.93589743589744 0.866982221603394
5.23717948717949 0.871930241584778
5.55769230769231 0.879445791244507
5.8974358974359 0.842308938503265
6.25641025641026 0.850890636444092
6.64102564102564 0.885281145572662
7.04487179487179 0.849931061267853
7.47435897435897 0.860218346118927
7.92948717948718 0.844499051570892
8.41025641025641 0.867016494274139
8.92307692307692 0.866753160953522
9.46794871794872 0.844691753387451
10.0448717948718 0.845953941345215
10.6602564102564 0.863474726676941
11.3076923076923 0.847902119159698
12 0.830853879451752
12.7307692307692 0.844477534294128
13.5064102564103 0.842575192451477
14.3333333333333 0.851885199546814
15.2051282051282 0.841178178787231
16.1346153846154 0.863195657730103
17.1153846153846 0.843936383724213
18.1602564102564 0.842709183692932
19.2692307692308 0.839828729629517
20.4423076923077 0.841419219970703
21.6858974358974 0.819625556468964
23.0128205128205 0.843691468238831
24.4102564102564 0.829348564147949
25.9038461538462 0.82373458147049
27.4807692307692 0.825295209884644
29.1538461538462 0.816047668457031
30.9358974358974 0.811614215373993
32.8205128205128 0.817498445510864
34.8205128205128 0.825966477394104
36.9423076923077 0.805321455001831
39.1923076923077 0.823720693588257
41.5833333333333 0.809052705764771
44.1153846153846 0.791631937026978
46.8076923076923 0.816176295280457
49.6602564102564 0.810070753097534
52.6858974358974 0.802140653133392
55.8974358974359 0.810381889343262
59.3076923076923 0.805741846561432
62.9230769230769 0.793288111686707
66.7564102564103 0.799246072769165
70.8269230769231 0.801728367805481
75.1474358974359 0.817611992359161
79.7307692307692 0.801945805549622
84.5897435897436 0.810497105121613
89.7435897435897 0.77552992105484
95.2179487179487 0.787108302116394
101.019230769231 0.792248547077179
107.179487179487 0.792010247707367
113.711538461538 0.775051116943359
120.641025641026 0.779722094535828
127.99358974359 0.775363147258759
135.801282051282 0.774426400661469
144.076923076923 0.774412512779236
152.858974358974 0.771445453166962
162.179487179487 0.748274803161621
172.064102564103 0.767704665660858
182.551282051282 0.770076096057892
193.679487179487 0.733468890190125
205.487179487179 0.783152222633362
218.012820512821 0.782109379768372
231.301282051282 0.758421719074249
245.403846153846 0.782041013240814
260.358974358974 0.761578977108002
276.230769230769 0.756611287593842
293.070512820513 0.734513461589813
310.935897435897 0.769268572330475
329.884615384615 0.734050571918488
350 0.780656397342682
};
\addlegendentry{mb 128, mc 10}
\addplot [, color1, opacity=0.6, mark=square*, mark size=0.5, mark options={solid}, only marks, forget plot]
table {%
1 0.913733065128326
1.05769230769231 0.927339017391205
1.12179487179487 0.929070234298706
1.19230769230769 0.91342693567276
1.26282051282051 0.889622986316681
1.33974358974359 0.884113729000092
1.42307692307692 0.867553472518921
1.51282051282051 0.846061050891876
1.6025641025641 0.859427154064178
1.69871794871795 0.851585388183594
1.80128205128205 0.849848926067352
1.91666666666667 0.85142570734024
2.03205128205128 0.870018422603607
2.15384615384615 0.878290057182312
2.28846153846154 0.865154147148132
2.42307692307692 0.890442252159119
2.57692307692308 0.870899200439453
2.73076923076923 0.868977308273315
2.8974358974359 0.857962608337402
3.07692307692308 0.855769157409668
3.26282051282051 0.843807280063629
3.46153846153846 0.861048877239227
3.67307692307692 0.840531885623932
3.8974358974359 0.847744822502136
4.13461538461539 0.892828643321991
4.38461538461539 0.85648649930954
4.65384615384615 0.857701241970062
4.93589743589744 0.855412364006042
5.23717948717949 0.853311419487
5.55769230769231 0.874678373336792
5.8974358974359 0.843499660491943
6.25641025641026 0.856220960617065
6.64102564102564 0.843905031681061
7.04487179487179 0.857165992259979
7.47435897435897 0.859025716781616
7.92948717948718 0.859907686710358
8.41025641025641 0.855236947536469
8.92307692307692 0.860185742378235
9.46794871794872 0.846794128417969
10.0448717948718 0.847592890262604
10.6602564102564 0.846441805362701
11.3076923076923 0.853031277656555
12 0.844592094421387
12.7307692307692 0.852076530456543
13.5064102564103 0.836326122283936
14.3333333333333 0.844270467758179
15.2051282051282 0.809728860855103
16.1346153846154 0.833934783935547
17.1153846153846 0.822240889072418
18.1602564102564 0.829793214797974
19.2692307692308 0.82714718580246
20.4423076923077 0.818221390247345
21.6858974358974 0.836787641048431
23.0128205128205 0.814081847667694
24.4102564102564 0.841322898864746
25.9038461538462 0.801062047481537
27.4807692307692 0.827709019184113
29.1538461538462 0.802042841911316
30.9358974358974 0.825116574764252
32.8205128205128 0.840176701545715
34.8205128205128 0.823870360851288
36.9423076923077 0.820996522903442
39.1923076923077 0.820534944534302
41.5833333333333 0.827105820178986
44.1153846153846 0.829782485961914
46.8076923076923 0.822691321372986
49.6602564102564 0.823975682258606
52.6858974358974 0.815109848976135
55.8974358974359 0.834745764732361
59.3076923076923 0.813283681869507
62.9230769230769 0.814307689666748
66.7564102564103 0.793612480163574
70.8269230769231 0.798652112483978
75.1474358974359 0.831894040107727
79.7307692307692 0.799334228038788
84.5897435897436 0.8141810297966
89.7435897435897 0.792703866958618
95.2179487179487 0.796551287174225
101.019230769231 0.785683035850525
107.179487179487 0.796531796455383
113.711538461538 0.805339813232422
120.641025641026 0.782087385654449
127.99358974359 0.765975177288055
135.801282051282 0.792820811271667
144.076923076923 0.797229290008545
152.858974358974 0.767368197441101
162.179487179487 0.787981688976288
172.064102564103 0.761406064033508
182.551282051282 0.762562096118927
193.679487179487 0.779470324516296
205.487179487179 0.767640054225922
218.012820512821 0.750213742256165
231.301282051282 0.759168744087219
245.403846153846 0.772728323936462
260.358974358974 0.773756563663483
276.230769230769 0.78218537569046
293.070512820513 0.748607754707336
310.935897435897 0.751474797725677
329.884615384615 0.781770288944244
350 0.730762302875519
};
\addplot [, color1, opacity=0.6, mark=square*, mark size=0.5, mark options={solid}, only marks, forget plot]
table {%
1 0.916542947292328
1.05769230769231 0.931691467761993
1.12179487179487 0.900379300117493
1.19230769230769 0.895990669727325
1.26282051282051 0.899417042732239
1.33974358974359 0.880560457706451
1.42307692307692 0.850313246250153
1.51282051282051 0.85344409942627
1.6025641025641 0.862961053848267
1.69871794871795 0.871634066104889
1.80128205128205 0.838926672935486
1.91666666666667 0.827135801315308
2.03205128205128 0.861618936061859
2.15384615384615 0.851835012435913
2.28846153846154 0.849676191806793
2.42307692307692 0.844133377075195
2.57692307692308 0.837679326534271
2.73076923076923 0.84752231836319
2.8974358974359 0.858229279518127
3.07692307692308 0.833664059638977
3.26282051282051 0.856213986873627
3.46153846153846 0.859650015830994
3.67307692307692 0.85422271490097
3.8974358974359 0.849334836006165
4.13461538461539 0.868466317653656
4.38461538461539 0.869411706924438
4.65384615384615 0.88206022977829
4.93589743589744 0.868793249130249
5.23717948717949 0.876614987850189
5.55769230769231 0.868694126605988
5.8974358974359 0.846091270446777
6.25641025641026 0.853488206863403
6.64102564102564 0.842763364315033
7.04487179487179 0.864092528820038
7.47435897435897 0.861700296401978
7.92948717948718 0.825179755687714
8.41025641025641 0.845180034637451
8.92307692307692 0.830293118953705
9.46794871794872 0.849012970924377
10.0448717948718 0.84914243221283
10.6602564102564 0.841433107852936
11.3076923076923 0.838192582130432
12 0.828528702259064
12.7307692307692 0.839729309082031
13.5064102564103 0.839915156364441
14.3333333333333 0.844936072826385
15.2051282051282 0.823081314563751
16.1346153846154 0.85566371679306
17.1153846153846 0.828394889831543
18.1602564102564 0.822549402713776
19.2692307692308 0.826797008514404
20.4423076923077 0.834057688713074
21.6858974358974 0.812938809394836
23.0128205128205 0.828636288642883
24.4102564102564 0.819133341312408
25.9038461538462 0.832253396511078
27.4807692307692 0.820967674255371
29.1538461538462 0.816878974437714
30.9358974358974 0.818017661571503
32.8205128205128 0.833658277988434
34.8205128205128 0.798024117946625
36.9423076923077 0.804520070552826
39.1923076923077 0.804291486740112
41.5833333333333 0.812767148017883
44.1153846153846 0.807976007461548
46.8076923076923 0.80330091714859
49.6602564102564 0.787629663944244
52.6858974358974 0.788661658763885
55.8974358974359 0.796341180801392
59.3076923076923 0.785801231861115
62.9230769230769 0.773977637290955
66.7564102564103 0.786816596984863
70.8269230769231 0.782921373844147
75.1474358974359 0.797882676124573
79.7307692307692 0.790797412395477
84.5897435897436 0.789383709430695
89.7435897435897 0.787706911563873
95.2179487179487 0.785875737667084
101.019230769231 0.777137517929077
107.179487179487 0.794098198413849
113.711538461538 0.785645127296448
120.641025641026 0.789956510066986
127.99358974359 0.782077789306641
135.801282051282 0.787630498409271
144.076923076923 0.790004193782806
152.858974358974 0.7827108502388
162.179487179487 0.778348088264465
172.064102564103 0.758339703083038
182.551282051282 0.772092819213867
193.679487179487 0.775286376476288
205.487179487179 0.801550447940826
218.012820512821 0.757523477077484
231.301282051282 0.788908243179321
245.403846153846 0.744982123374939
260.358974358974 0.762783169746399
276.230769230769 0.761912822723389
293.070512820513 0.749617457389832
310.935897435897 0.738620102405548
329.884615384615 0.745811283588409
350 0.739320516586304
};
\addplot [, color1, opacity=0.6, mark=square*, mark size=0.5, mark options={solid}, only marks, forget plot]
table {%
1 0.930921018123627
1.05769230769231 0.937215268611908
1.12179487179487 0.923762857913971
1.19230769230769 0.880602538585663
1.26282051282051 0.881431877613068
1.33974358974359 0.868075668811798
1.42307692307692 0.851955711841583
1.51282051282051 0.828129172325134
1.6025641025641 0.861566603183746
1.69871794871795 0.86156439781189
1.80128205128205 0.858050525188446
1.91666666666667 0.841123580932617
2.03205128205128 0.866239428520203
2.15384615384615 0.868610084056854
2.28846153846154 0.871552228927612
2.42307692307692 0.850849568843842
2.57692307692308 0.843051731586456
2.73076923076923 0.874859750270844
2.8974358974359 0.83734130859375
3.07692307692308 0.860913217067719
3.26282051282051 0.849828958511353
3.46153846153846 0.876880943775177
3.67307692307692 0.863349914550781
3.8974358974359 0.868480980396271
4.13461538461539 0.858756542205811
4.38461538461539 0.868277132511139
4.65384615384615 0.863823354244232
4.93589743589744 0.866208612918854
5.23717948717949 0.877836883068085
5.55769230769231 0.868503093719482
5.8974358974359 0.847486853599548
6.25641025641026 0.857813000679016
6.64102564102564 0.866164982318878
7.04487179487179 0.861320912837982
7.47435897435897 0.861942291259766
7.92948717948718 0.846291422843933
8.41025641025641 0.861376106739044
8.92307692307692 0.855874836444855
9.46794871794872 0.851385354995728
10.0448717948718 0.864171147346497
10.6602564102564 0.826976537704468
11.3076923076923 0.834113597869873
12 0.832267284393311
12.7307692307692 0.839188694953918
13.5064102564103 0.827201187610626
14.3333333333333 0.842165887355804
15.2051282051282 0.82310026884079
16.1346153846154 0.828837692737579
17.1153846153846 0.827936410903931
18.1602564102564 0.823190152645111
19.2692307692308 0.828522145748138
20.4423076923077 0.819186508655548
21.6858974358974 0.808682978153229
23.0128205128205 0.811718702316284
24.4102564102564 0.803722679615021
25.9038461538462 0.812884092330933
27.4807692307692 0.805591404438019
29.1538461538462 0.822291374206543
30.9358974358974 0.795220613479614
32.8205128205128 0.817628443241119
34.8205128205128 0.81766551733017
36.9423076923077 0.820332467556
39.1923076923077 0.83106517791748
41.5833333333333 0.823542892932892
44.1153846153846 0.818914771080017
46.8076923076923 0.832840263843536
49.6602564102564 0.795054316520691
52.6858974358974 0.804426550865173
55.8974358974359 0.819401621818542
59.3076923076923 0.794214904308319
62.9230769230769 0.795349538326263
66.7564102564103 0.797185122966766
70.8269230769231 0.799373745918274
75.1474358974359 0.81201171875
79.7307692307692 0.801687479019165
84.5897435897436 0.789735078811646
89.7435897435897 0.795115053653717
95.2179487179487 0.787289261817932
101.019230769231 0.791907131671906
107.179487179487 0.786551952362061
113.711538461538 0.788482785224915
120.641025641026 0.785990417003632
127.99358974359 0.773228585720062
135.801282051282 0.783946335315704
144.076923076923 0.760813593864441
152.858974358974 0.798906564712524
162.179487179487 0.784055531024933
172.064102564103 0.759214997291565
182.551282051282 0.761335492134094
193.679487179487 0.735792815685272
205.487179487179 0.77546614408493
218.012820512821 0.755618870258331
231.301282051282 0.757085084915161
245.403846153846 0.764986395835876
260.358974358974 0.755032241344452
276.230769230769 0.743644714355469
293.070512820513 0.755886435508728
310.935897435897 0.732439696788788
329.884615384615 0.754682123661041
350 0.744815647602081
};
\addplot [, color1, opacity=0.6, mark=square*, mark size=0.5, mark options={solid}, only marks, forget plot]
table {%
1 0.912054240703583
1.05769230769231 0.929876983165741
1.12179487179487 0.92319518327713
1.19230769230769 0.88997483253479
1.26282051282051 0.880996227264404
1.33974358974359 0.854848623275757
1.42307692307692 0.839143335819244
1.51282051282051 0.821508646011353
1.6025641025641 0.866652727127075
1.69871794871795 0.853063464164734
1.80128205128205 0.836174547672272
1.91666666666667 0.824054062366486
2.03205128205128 0.859745442867279
2.15384615384615 0.876610219478607
2.28846153846154 0.856221914291382
2.42307692307692 0.849147617816925
2.57692307692308 0.834784805774689
2.73076923076923 0.865638434886932
2.8974358974359 0.860926508903503
3.07692307692308 0.872248649597168
3.26282051282051 0.847927212715149
3.46153846153846 0.862634539604187
3.67307692307692 0.845707654953003
3.8974358974359 0.862292289733887
4.13461538461539 0.861827373504639
4.38461538461539 0.862881004810333
4.65384615384615 0.885019361972809
4.93589743589744 0.884880781173706
5.23717948717949 0.858442962169647
5.55769230769231 0.865406334400177
5.8974358974359 0.855949819087982
6.25641025641026 0.851126074790955
6.64102564102564 0.838666498661041
7.04487179487179 0.8563392162323
7.47435897435897 0.859594643115997
7.92948717948718 0.841901838779449
8.41025641025641 0.858897745609283
8.92307692307692 0.853223264217377
9.46794871794872 0.845200657844543
10.0448717948718 0.853466749191284
10.6602564102564 0.830646634101868
11.3076923076923 0.853885769844055
12 0.819429814815521
12.7307692307692 0.836726188659668
13.5064102564103 0.825124442577362
14.3333333333333 0.841787338256836
15.2051282051282 0.833626747131348
16.1346153846154 0.843654334545135
17.1153846153846 0.828947186470032
18.1602564102564 0.839121699333191
19.2692307692308 0.832643330097198
20.4423076923077 0.838855862617493
21.6858974358974 0.818031430244446
23.0128205128205 0.818793296813965
24.4102564102564 0.825878739356995
25.9038461538462 0.820686995983124
27.4807692307692 0.821729123592377
29.1538461538462 0.825300335884094
30.9358974358974 0.813980519771576
32.8205128205128 0.832571387290955
34.8205128205128 0.826457321643829
36.9423076923077 0.825701713562012
39.1923076923077 0.81638365983963
41.5833333333333 0.823321044445038
44.1153846153846 0.804219961166382
46.8076923076923 0.815376341342926
49.6602564102564 0.805506885051727
52.6858974358974 0.790695488452911
55.8974358974359 0.783774554729462
59.3076923076923 0.784997701644897
62.9230769230769 0.7868412733078
66.7564102564103 0.79473614692688
70.8269230769231 0.815131962299347
75.1474358974359 0.798743546009064
79.7307692307692 0.79505980014801
84.5897435897436 0.778395056724548
89.7435897435897 0.786633968353271
95.2179487179487 0.807973742485046
101.019230769231 0.774679601192474
107.179487179487 0.78580242395401
113.711538461538 0.788279354572296
120.641025641026 0.775962352752686
127.99358974359 0.780393958091736
135.801282051282 0.791627764701843
144.076923076923 0.779008448123932
152.858974358974 0.796540677547455
162.179487179487 0.794141829013824
172.064102564103 0.745811223983765
182.551282051282 0.771310389041901
193.679487179487 0.788514792919159
205.487179487179 0.763978719711304
218.012820512821 0.777109205722809
231.301282051282 0.743472218513489
245.403846153846 0.779436767101288
260.358974358974 0.773873269557953
276.230769230769 0.764389455318451
293.070512820513 0.777651488780975
310.935897435897 0.787836134433746
329.884615384615 0.787735521793365
350 0.708586394786835
};
\addplot [, color2, opacity=0.6, mark=triangle*, mark size=0.5, mark options={solid,rotate=180}, only marks]
table {%
1 0.693138241767883
1.05769230769231 0.681042015552521
1.12179487179487 0.675624072551727
1.19230769230769 0.634541809558868
1.26282051282051 0.577827870845795
1.33974358974359 0.551983416080475
1.42307692307692 0.571359038352966
1.51282051282051 0.581796407699585
1.6025641025641 0.58854740858078
1.69871794871795 0.579913794994354
1.80128205128205 0.541118741035461
1.91666666666667 0.63287615776062
2.03205128205128 0.560380637645721
2.15384615384615 0.630942225456238
2.28846153846154 0.569148480892181
2.42307692307692 0.588764309883118
2.57692307692308 0.534227252006531
2.73076923076923 0.497168719768524
2.8974358974359 0.507247626781464
3.07692307692308 0.547014653682709
3.26282051282051 0.539292335510254
3.46153846153846 0.564400553703308
3.67307692307692 0.539557874202728
3.8974358974359 0.529395580291748
4.13461538461539 0.570440292358398
4.38461538461539 0.536126255989075
4.65384615384615 0.514625787734985
4.93589743589744 0.504750549793243
5.23717948717949 0.504397630691528
5.55769230769231 0.494057983160019
5.8974358974359 0.475596100091934
6.25641025641026 0.470050722360611
6.64102564102564 0.496611773967743
7.04487179487179 0.48316878080368
7.47435897435897 0.484239876270294
7.92948717948718 0.451668381690979
8.41025641025641 0.472718715667725
8.92307692307692 0.48405459523201
9.46794871794872 0.445079296827316
10.0448717948718 0.486413985490799
10.6602564102564 0.454417645931244
11.3076923076923 0.479628622531891
12 0.468721300363541
12.7307692307692 0.478107064962387
13.5064102564103 0.49614953994751
14.3333333333333 0.49348258972168
15.2051282051282 0.480019927024841
16.1346153846154 0.453595221042633
17.1153846153846 0.445897936820984
18.1602564102564 0.435354232788086
19.2692307692308 0.505371153354645
20.4423076923077 0.421740263700485
21.6858974358974 0.448441833257675
23.0128205128205 0.478776067495346
24.4102564102564 0.442039132118225
25.9038461538462 0.382141321897507
27.4807692307692 0.457672476768494
29.1538461538462 0.422127217054367
30.9358974358974 0.414038151502609
32.8205128205128 0.420136779546738
34.8205128205128 0.387849807739258
36.9423076923077 0.405749529600143
39.1923076923077 0.407015830278397
41.5833333333333 0.404880970716476
44.1153846153846 0.394809663295746
46.8076923076923 0.404225826263428
49.6602564102564 0.394925743341446
52.6858974358974 0.398270040750504
55.8974358974359 0.341135621070862
59.3076923076923 0.386650830507278
62.9230769230769 0.371445089578629
66.7564102564103 0.339403837919235
70.8269230769231 0.399158626794815
75.1474358974359 0.394701570272446
79.7307692307692 0.450529783964157
84.5897435897436 0.39716973900795
89.7435897435897 0.420387119054794
95.2179487179487 0.377454817295074
101.019230769231 0.432577133178711
107.179487179487 0.417753130197525
113.711538461538 0.368040144443512
120.641025641026 0.38361930847168
127.99358974359 0.449813067913055
135.801282051282 0.434487611055374
144.076923076923 0.440443187952042
152.858974358974 0.409933835268021
162.179487179487 0.373032242059708
172.064102564103 0.381757378578186
182.551282051282 0.42695289850235
193.679487179487 0.388760209083557
205.487179487179 0.373908221721649
218.012820512821 0.42933714389801
231.301282051282 0.434458523988724
245.403846153846 0.436883270740509
260.358974358974 0.399091213941574
276.230769230769 0.478462129831314
293.070512820513 0.447796583175659
310.935897435897 0.438323974609375
329.884615384615 0.496525436639786
350 0.452746331691742
};
\addlegendentry{sub 16, mc 10}
\addplot [, color2, opacity=0.6, mark=triangle*, mark size=0.5, mark options={solid,rotate=180}, only marks, forget plot]
table {%
1 0.687081277370453
1.05769230769231 0.672315657138824
1.12179487179487 0.677501380443573
1.19230769230769 0.634153008460999
1.26282051282051 0.557675719261169
1.33974358974359 0.572105705738068
1.42307692307692 0.560509264469147
1.51282051282051 0.528055429458618
1.6025641025641 0.527817487716675
1.69871794871795 0.53928142786026
1.80128205128205 0.548639953136444
1.91666666666667 0.578828036785126
2.03205128205128 0.590433180332184
2.15384615384615 0.549251019954681
2.28846153846154 0.55649471282959
2.42307692307692 0.573173522949219
2.57692307692308 0.547353029251099
2.73076923076923 0.53761225938797
2.8974358974359 0.538672804832458
3.07692307692308 0.529254913330078
3.26282051282051 0.50942724943161
3.46153846153846 0.539938569068909
3.67307692307692 0.5129114985466
3.8974358974359 0.499785006046295
4.13461538461539 0.547426342964172
4.38461538461539 0.503083467483521
4.65384615384615 0.50875198841095
4.93589743589744 0.499886780977249
5.23717948717949 0.495695292949677
5.55769230769231 0.513577282428741
5.8974358974359 0.44480100274086
6.25641025641026 0.450750261545181
6.64102564102564 0.473553657531738
7.04487179487179 0.454709529876709
7.47435897435897 0.434951782226562
7.92948717948718 0.462751805782318
8.41025641025641 0.446643143892288
8.92307692307692 0.42886084318161
9.46794871794872 0.422554016113281
10.0448717948718 0.438911736011505
10.6602564102564 0.437137216329575
11.3076923076923 0.440397173166275
12 0.435668796300888
12.7307692307692 0.422609597444534
13.5064102564103 0.445613324642181
14.3333333333333 0.433985888957977
15.2051282051282 0.404658734798431
16.1346153846154 0.434535712003708
17.1153846153846 0.438529133796692
18.1602564102564 0.448061734437943
19.2692307692308 0.452483236789703
20.4423076923077 0.441666096448898
21.6858974358974 0.432728558778763
23.0128205128205 0.45370826125145
24.4102564102564 0.443094313144684
25.9038461538462 0.439303129911423
27.4807692307692 0.426286369562149
29.1538461538462 0.434934824705124
30.9358974358974 0.425615459680557
32.8205128205128 0.470377117395401
34.8205128205128 0.465028434991837
36.9423076923077 0.425170689821243
39.1923076923077 0.438710957765579
41.5833333333333 0.424863219261169
44.1153846153846 0.395597130060196
46.8076923076923 0.381609559059143
49.6602564102564 0.4154032766819
52.6858974358974 0.423592358827591
55.8974358974359 0.380350708961487
59.3076923076923 0.370253056287766
62.9230769230769 0.434573292732239
66.7564102564103 0.382105618715286
70.8269230769231 0.41710638999939
75.1474358974359 0.387198388576508
79.7307692307692 0.335515737533569
84.5897435897436 0.430916368961334
89.7435897435897 0.402806490659714
95.2179487179487 0.420210331678391
101.019230769231 0.370447218418121
107.179487179487 0.370435237884521
113.711538461538 0.397159576416016
120.641025641026 0.371892720460892
127.99358974359 0.394521981477737
135.801282051282 0.375628918409348
144.076923076923 0.42793396115303
152.858974358974 0.419587939977646
162.179487179487 0.398572593927383
172.064102564103 0.389492243528366
182.551282051282 0.387873560190201
193.679487179487 0.376347541809082
205.487179487179 0.368850827217102
218.012820512821 0.399107098579407
231.301282051282 0.38762366771698
245.403846153846 0.394781708717346
260.358974358974 0.395743101835251
276.230769230769 0.406860917806625
293.070512820513 0.390224248170853
310.935897435897 0.426386028528214
329.884615384615 0.403529733419418
350 0.422126531600952
};
\addplot [, color2, opacity=0.6, mark=triangle*, mark size=0.5, mark options={solid,rotate=180}, only marks, forget plot]
table {%
1 0.688969075679779
1.05769230769231 0.679952621459961
1.12179487179487 0.669662892818451
1.19230769230769 0.617559969425201
1.26282051282051 0.569102227687836
1.33974358974359 0.535609185695648
1.42307692307692 0.49536520242691
1.51282051282051 0.542816579341888
1.6025641025641 0.554614067077637
1.69871794871795 0.567099153995514
1.80128205128205 0.5490842461586
1.91666666666667 0.554242551326752
2.03205128205128 0.540292859077454
2.15384615384615 0.556371033191681
2.28846153846154 0.509446024894714
2.42307692307692 0.533646166324615
2.57692307692308 0.525023341178894
2.73076923076923 0.53517609834671
2.8974358974359 0.553733766078949
3.07692307692308 0.516591012477875
3.26282051282051 0.536545217037201
3.46153846153846 0.539089441299438
3.67307692307692 0.494358897209167
3.8974358974359 0.53903329372406
4.13461538461539 0.53004914522171
4.38461538461539 0.522826194763184
4.65384615384615 0.531924664974213
4.93589743589744 0.533518671989441
5.23717948717949 0.520134329795837
5.55769230769231 0.505403578281403
5.8974358974359 0.469720244407654
6.25641025641026 0.468608379364014
6.64102564102564 0.472414493560791
7.04487179487179 0.476742804050446
7.47435897435897 0.466355949640274
7.92948717948718 0.488100916147232
8.41025641025641 0.443433165550232
8.92307692307692 0.460924625396729
9.46794871794872 0.454436719417572
10.0448717948718 0.498462587594986
10.6602564102564 0.4600989818573
11.3076923076923 0.494536340236664
12 0.439301908016205
12.7307692307692 0.447703778743744
13.5064102564103 0.485385924577713
14.3333333333333 0.430742263793945
15.2051282051282 0.428184121847153
16.1346153846154 0.46201503276825
17.1153846153846 0.444124221801758
18.1602564102564 0.430262356996536
19.2692307692308 0.428375393152237
20.4423076923077 0.459827780723572
21.6858974358974 0.454247266054153
23.0128205128205 0.418356686830521
24.4102564102564 0.421659827232361
25.9038461538462 0.409606605768204
27.4807692307692 0.441813766956329
29.1538461538462 0.425736159086227
30.9358974358974 0.41644349694252
32.8205128205128 0.456481128931046
34.8205128205128 0.40263107419014
36.9423076923077 0.442845821380615
39.1923076923077 0.421634584665298
41.5833333333333 0.401465356349945
44.1153846153846 0.371108859777451
46.8076923076923 0.4473757147789
49.6602564102564 0.444438964128494
52.6858974358974 0.347885727882385
55.8974358974359 0.340153962373734
59.3076923076923 0.382212907075882
62.9230769230769 0.342095017433167
66.7564102564103 0.399036943912506
70.8269230769231 0.383522927761078
75.1474358974359 0.403634488582611
79.7307692307692 0.392440468072891
84.5897435897436 0.408627927303314
89.7435897435897 0.428527981042862
95.2179487179487 0.405874371528625
101.019230769231 0.396040439605713
107.179487179487 0.419080406427383
113.711538461538 0.366154253482819
120.641025641026 0.374714314937592
127.99358974359 0.431790232658386
135.801282051282 0.402045518159866
144.076923076923 0.429907381534576
152.858974358974 0.439356178045273
162.179487179487 0.458303362131119
172.064102564103 0.442227095365524
182.551282051282 0.381209164857864
193.679487179487 0.388818949460983
205.487179487179 0.443259954452515
218.012820512821 0.483285278081894
231.301282051282 0.486429393291473
245.403846153846 0.474618971347809
260.358974358974 0.387579768896103
276.230769230769 0.459088802337646
293.070512820513 0.440504223108292
310.935897435897 0.499051839113235
329.884615384615 0.430749744176865
350 0.442314833402634
};
\addplot [, color2, opacity=0.6, mark=triangle*, mark size=0.5, mark options={solid,rotate=180}, only marks, forget plot]
table {%
1 0.700301170349121
1.05769230769231 0.683213472366333
1.12179487179487 0.676782965660095
1.19230769230769 0.598848938941956
1.26282051282051 0.569230556488037
1.33974358974359 0.530528783798218
1.42307692307692 0.528291523456573
1.51282051282051 0.559408366680145
1.6025641025641 0.573561310768127
1.69871794871795 0.587468326091766
1.80128205128205 0.56756180524826
1.91666666666667 0.556513786315918
2.03205128205128 0.590825438499451
2.15384615384615 0.555381536483765
2.28846153846154 0.571130275726318
2.42307692307692 0.534062683582306
2.57692307692308 0.532010436058044
2.73076923076923 0.520153105258942
2.8974358974359 0.511096119880676
3.07692307692308 0.537450611591339
3.26282051282051 0.522713541984558
3.46153846153846 0.545213103294373
3.67307692307692 0.487725138664246
3.8974358974359 0.491964489221573
4.13461538461539 0.516360700130463
4.38461538461539 0.476814568042755
4.65384615384615 0.522997856140137
4.93589743589744 0.512531280517578
5.23717948717949 0.515496075153351
5.55769230769231 0.533098876476288
5.8974358974359 0.472192406654358
6.25641025641026 0.477420121431351
6.64102564102564 0.475430250167847
7.04487179487179 0.43487086892128
7.47435897435897 0.45912703871727
7.92948717948718 0.439550459384918
8.41025641025641 0.434064447879791
8.92307692307692 0.442903578281403
9.46794871794872 0.420441508293152
10.0448717948718 0.462268888950348
10.6602564102564 0.430078387260437
11.3076923076923 0.455352455377579
12 0.42685067653656
12.7307692307692 0.422642081975937
13.5064102564103 0.488189429044724
14.3333333333333 0.462108135223389
15.2051282051282 0.421396017074585
16.1346153846154 0.434545367956161
17.1153846153846 0.435870349407196
18.1602564102564 0.426973700523376
19.2692307692308 0.459770381450653
20.4423076923077 0.415913224220276
21.6858974358974 0.454703539609909
23.0128205128205 0.435327529907227
24.4102564102564 0.390322178602219
25.9038461538462 0.405323773622513
27.4807692307692 0.425645291805267
29.1538461538462 0.415071934461594
30.9358974358974 0.363535225391388
32.8205128205128 0.381225049495697
34.8205128205128 0.388436436653137
36.9423076923077 0.39557620882988
39.1923076923077 0.394836962223053
41.5833333333333 0.414462894201279
44.1153846153846 0.41438215970993
46.8076923076923 0.389716297388077
49.6602564102564 0.365045219659805
52.6858974358974 0.420643717050552
55.8974358974359 0.425945222377777
59.3076923076923 0.4164779484272
62.9230769230769 0.410282552242279
66.7564102564103 0.440558224916458
70.8269230769231 0.373686730861664
75.1474358974359 0.441965848207474
79.7307692307692 0.41503980755806
84.5897435897436 0.417506486177444
89.7435897435897 0.391325682401657
95.2179487179487 0.360861569643021
101.019230769231 0.376767337322235
107.179487179487 0.400949329137802
113.711538461538 0.42396005988121
120.641025641026 0.437763571739197
127.99358974359 0.300637006759644
135.801282051282 0.367055803537369
144.076923076923 0.366498440504074
152.858974358974 0.432730048894882
162.179487179487 0.390829294919968
172.064102564103 0.358492285013199
182.551282051282 0.349722743034363
193.679487179487 0.44255518913269
205.487179487179 0.377709716558456
218.012820512821 0.357883244752884
231.301282051282 0.363071888685226
245.403846153846 0.43322280049324
260.358974358974 0.402835309505463
276.230769230769 0.44892543554306
293.070512820513 0.397932261228561
310.935897435897 0.418766856193542
329.884615384615 0.440651774406433
350 0.455139845609665
};
\addplot [, color2, opacity=0.6, mark=triangle*, mark size=0.5, mark options={solid,rotate=180}, only marks, forget plot]
table {%
1 0.692043364048004
1.05769230769231 0.686571478843689
1.12179487179487 0.671840190887451
1.19230769230769 0.651533007621765
1.26282051282051 0.589914083480835
1.33974358974359 0.56506735086441
1.42307692307692 0.537195801734924
1.51282051282051 0.523496150970459
1.6025641025641 0.626989901065826
1.69871794871795 0.582525551319122
1.80128205128205 0.533255279064178
1.91666666666667 0.560566902160645
2.03205128205128 0.547539055347443
2.15384615384615 0.561704277992249
2.28846153846154 0.545092403888702
2.42307692307692 0.54508900642395
2.57692307692308 0.514264583587646
2.73076923076923 0.559745609760284
2.8974358974359 0.532319962978363
3.07692307692308 0.529731035232544
3.26282051282051 0.512587368488312
3.46153846153846 0.55156010389328
3.67307692307692 0.514769911766052
3.8974358974359 0.513727843761444
4.13461538461539 0.520922243595123
4.38461538461539 0.520170152187347
4.65384615384615 0.510625839233398
4.93589743589744 0.564834654331207
5.23717948717949 0.497774869203568
5.55769230769231 0.509535491466522
5.8974358974359 0.446477204561234
6.25641025641026 0.474521487951279
6.64102564102564 0.470100551843643
7.04487179487179 0.483276933431625
7.47435897435897 0.481226921081543
7.92948717948718 0.453453898429871
8.41025641025641 0.459974378347397
8.92307692307692 0.440266132354736
9.46794871794872 0.438427954912186
10.0448717948718 0.441567778587341
10.6602564102564 0.451656877994537
11.3076923076923 0.447946816682816
12 0.432173252105713
12.7307692307692 0.452603965997696
13.5064102564103 0.426710486412048
14.3333333333333 0.45390647649765
15.2051282051282 0.423283517360687
16.1346153846154 0.440656661987305
17.1153846153846 0.436660766601562
18.1602564102564 0.393552005290985
19.2692307692308 0.412356942892075
20.4423076923077 0.432027131319046
21.6858974358974 0.436044543981552
23.0128205128205 0.426562637090683
24.4102564102564 0.416479468345642
25.9038461538462 0.415522068738937
27.4807692307692 0.435087084770203
29.1538461538462 0.453155398368835
30.9358974358974 0.420662999153137
32.8205128205128 0.426566302776337
34.8205128205128 0.415403991937637
36.9423076923077 0.392911523580551
39.1923076923077 0.417550772428513
41.5833333333333 0.442709118127823
44.1153846153846 0.395485907793045
46.8076923076923 0.421615779399872
49.6602564102564 0.420668005943298
52.6858974358974 0.430721282958984
55.8974358974359 0.45966449379921
59.3076923076923 0.396939277648926
62.9230769230769 0.404462814331055
66.7564102564103 0.431695550680161
70.8269230769231 0.411477267742157
75.1474358974359 0.438480287790298
79.7307692307692 0.398809045553207
84.5897435897436 0.41124564409256
89.7435897435897 0.390255659818649
95.2179487179487 0.355010837316513
101.019230769231 0.389070570468903
107.179487179487 0.3691266477108
113.711538461538 0.453323036432266
120.641025641026 0.380797773599625
127.99358974359 0.391756922006607
135.801282051282 0.393813014030457
144.076923076923 0.395882099866867
152.858974358974 0.396503925323486
162.179487179487 0.446147084236145
172.064102564103 0.424268394708633
182.551282051282 0.431434243917465
193.679487179487 0.413708955049515
205.487179487179 0.454949915409088
218.012820512821 0.385057926177979
231.301282051282 0.358635485172272
245.403846153846 0.512732923030853
260.358974358974 0.511533737182617
276.230769230769 0.431084781885147
293.070512820513 0.464394360780716
310.935897435897 0.48440158367157
329.884615384615 0.538196563720703
350 0.472001552581787
};
\end{axis}

\end{tikzpicture}

    \tikzexternaldisable
  \end{minipage}
\end{subfigure}
\caption{\textbf{\bfvivit{} versus mini-batch \ggn{}.} Overlap between the top-$C$
  eigenspaces of different \ggn approximations and the mini-batch \ggn during
  training for all test problems. Each approximation is evaluated on $5$
  different mini-batches.} \label{vivit::fig:vivit_vs_mini_batch_ggn}
\end{figure*}

%%% Local Variables:
%%% mode: latex
%%% TeX-master: "../../thesis"
%%% End:


%%% Local Variables:
%%% mode: latex
%%% TeX-master: "../thesis"
%%% End:
