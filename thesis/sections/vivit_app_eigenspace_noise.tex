%========== mb and further approximations versus full-batch GGN
\subsubsection{Procedure (1)}

We use the checkpoints and the definition of overlaps between eigenspaces from
\Cref{vivit::sec:ggn_vs_hessian}. For the approximation of the \ggn{}, we consider the
cases listed in \Cref{vivit::tab:cases_full_batch}.

\begin{table*}[ht]
  \centering
  \caption{ \textbf{Considered cases for approximation of the eigenspace:} We
    use a different set of cases for the approximation of the \ggn{}'s
    full-batch eigenspace depending on the test problem. For the test problems
    with $C=10$, we use $M=1$ \mc-sample, for the \cifarhun \allcnnc test
    problem ($C=100$), we use $M=10$ \mc-samples in order to reduce the
    computational costs by the same factor. }
  \label{vivit::tab:cases_full_batch}
  \vspace{1ex}
  \begin{footnotesize}
    \begin{tabular}{ll}
      \toprule
      \textbf{Problem}
      & \textbf{Cases} \\
      \midrule
      \makecell[tl]{
      \fmnist \twoctwod \\
      \cifarten \threecthreed and \\
      \cifarten \resnetthirtytwo}
      & \makecell[tl]{
        \textbf{mb, exact} with mini-batch sizes $N \in \{2, 8, 32, 128\}$\\
      \textbf{mb, mc} with $N=128$ and $M=1$ \mc{}-sample\\
      \textbf{sub, exact} using $16$ samples from the mini-batch\\
      \textbf{sub, mc} using $16$ samples from the mini-batch and $M=1$ \mc{}-sample
      }
      \\
      \midrule
      \cifarhun \allcnnc
      & \makecell[tl]{
        \textbf{mb, exact} with mini-batch sizes $N \in \{2, 8, 32, 128\}$\\
      \textbf{mb, mc} with $N=128$ and $M=10$ \mc{}-samples\\
      \textbf{sub, exact} using $16$ samples from the mini-batch\\
      \textbf{sub, mc} using $16$ samples from the mini-batch and $M=10$ \mc{}-samples
      }
      \\
      \bottomrule
    \end{tabular}
  \end{footnotesize}
\end{table*}

For every checkpoint and case, we compute the top-$C$ eigenvectors of the
respective approximation to the \ggn{}. The eigenvectors are either computed
directly using \vivit (by transforming eigenvectors of the Gram matrix into
parameter space, see \Cref{vivit::sec:computing-full-ggn-eigenspectrum}) or, if not
applicable (because memory requirements exceed $N_\text{crit}$, see
\Cref{vivit::subsec:scalability}), using an iterative matrix-free approach. The overlap
is computed in reference to the \ggn{}'s full-batch top-$C$ eigenspace (see
\Cref{vivit::sec:ggn_vs_hessian}). We extract $5$ mini-batches from the training data
and repeat the above procedure for each mini-batch (\ie we obtain $5$ overlap
measurements for every checkpoint and case). The same $5$ mini-batches are used
over all checkpoints and cases.

\subsubsection{Results (1)}

The results can be found in
\Cref{vivit::fig:vivit_vs_full_batch_ggn_1} and \ref{vivit::fig:vivit_vs_full_batch_ggn_2}.
All test problems show the same characteristics: with decreasing computational
effort, the approximation carries less and less structure of its full-batch
counterpart, as indicated by dropping overlaps. In addition, for a fixed
approximation method, a decrease in approximation quality can be observed over
the course of training.

\begin{figure*}[p]
  \centering
  \begin{minipage}[t]{0.495\linewidth}
    \centering
    {\footnotesize Impact of batch size}
  \end{minipage}\hfill
  \begin{minipage}[t]{0.495\linewidth}
    \centering
    {\footnotesize Impact of batch size \& approximations}
  \end{minipage}

  \begin{subfigure}[t]{\linewidth}
    \centering
    \caption{\fmnist \twoctwod \sgd}
    \begin{minipage}{0.50\linewidth}
      \centering
      % defines the pgfplots style "eigspacedefault"
\pgfkeys{/pgfplots/eigspacedefault/.style={
    width=1.0\linewidth,
    height=0.6\linewidth,
    every axis plot/.append style={line width = 1.5pt},
    tick pos = left,
    ylabel near ticks,
    xlabel near ticks,
    xtick align = inside,
    ytick align = inside,
    legend cell align = left,
    legend columns = 4,
    legend pos = south east,
    legend style = {
      fill opacity = 1,
      text opacity = 1,
      font = \footnotesize,
      at={(1, 1.025)},
      anchor=south east,
      column sep=0.25cm,
    },
    legend image post style={scale=2.5},
    xticklabel style = {font = \footnotesize},
    xlabel style = {font = \footnotesize},
    axis line style = {black},
    yticklabel style = {font = \footnotesize},
    ylabel style = {font = \footnotesize},
    title style = {font = \footnotesize},
    grid = major,
    grid style = {dashed}
  }
}

\pgfkeys{/pgfplots/eigspacedefaultapp/.style={
    eigspacedefault,
    height=0.6\linewidth,
    legend columns = 2,
  }
}

\pgfkeys{/pgfplots/eigspacenolegend/.style={
    legend image post style = {scale=0},
    legend style = {
      fill opacity = 0,
      draw opacity = 0,
      text opacity = 0,
      font = \footnotesize,
      at={(1, 1.025)},
      anchor=south east,
      column sep=0.25cm,
    },
  }
}
%%% Local Variables:
%%% mode: latex
%%% TeX-master: "../../thesis"
%%% End:

      \pgfkeys{/pgfplots/zmystyle/.style={
          eigspacedefaultapp,
        }}
      \tikzexternalenable
      % This file was created by tikzplotlib v0.9.7.
\begin{tikzpicture}

\definecolor{color0}{rgb}{0.501960784313725,0.184313725490196,0.6}
\definecolor{color1}{rgb}{0.870588235294118,0.623529411764706,0.0862745098039216}
\definecolor{color2}{rgb}{0.274509803921569,0.6,0.564705882352941}

\begin{axis}[
axis line style={white!10!black},
legend columns=2,
legend style={fill opacity=0.8, draw opacity=1, text opacity=1, draw=white!80!black},
log basis x={10},
tick pos=left,
xlabel={epoch (log scale)},
xmajorgrids,
xmin=0.794328234724281, xmax=125.892541179417,
xmode=log,
ylabel={overlap},
ymajorgrids,
ymin=-0.05, ymax=1.05,
zmystyle
]
\addplot [, white!10!black, dashed, forget plot]
table {%
0.794328234724281 1
125.892541179417 1
};
\addplot [, white!10!black, dashed, forget plot]
table {%
0.794328234724281 0
125.892541179417 0
};
\addplot [, color0, opacity=0.6, mark=triangle*, mark size=0.5, mark options={solid,rotate=180}, only marks]
table {%
1 0.801415264606476
1.04615384615385 0.607483446598053
1.0974358974359 0.521924734115601
1.14871794871795 0.483781486749649
1.2025641025641 0.433026611804962
1.26153846153846 0.430521786212921
1.32051282051282 0.436059206724167
1.38461538461538 0.413618057966232
1.44871794871795 0.396258562803268
1.51794871794872 0.370242446660995
1.58974358974359 0.374122470617294
1.66666666666667 0.354302853345871
1.74615384615385 0.355698853731155
1.82820512820513 0.309842079877853
1.91538461538462 0.305968075990677
2.00769230769231 0.35800102353096
2.1025641025641 0.332394927740097
2.20512820512821 0.343383640050888
2.30769230769231 0.354219764471054
2.41794871794872 0.345604091882706
2.53333333333333 0.308200478553772
2.65384615384615 0.314388424158096
2.78205128205128 0.270658940076828
2.91282051282051 0.299510717391968
3.05384615384615 0.293736308813095
3.1974358974359 0.319352596998215
3.35128205128205 0.270169585943222
3.51025641025641 0.280201882123947
3.67692307692308 0.294581472873688
3.85128205128205 0.29918098449707
4.03589743589744 0.262917369604111
4.22820512820513 0.279215306043625
4.42820512820513 0.273021399974823
4.64102564102564 0.263854593038559
4.86153846153846 0.225722789764404
5.09230769230769 0.245010569691658
5.33589743589744 0.244667962193489
5.58974358974359 0.228388622403145
5.85641025641026 0.254273146390915
6.13589743589744 0.252182066440582
6.42564102564103 0.21746364235878
6.73333333333333 0.215382859110832
7.05384615384615 0.228019520640373
7.38974358974359 0.214989185333252
7.74102564102564 0.190821006894112
8.11025641025641 0.202031210064888
8.4974358974359 0.205202057957649
8.9 0.181607082486153
9.32564102564103 0.199000790715218
9.76923076923077 0.164740383625031
10.2333333333333 0.201417282223701
10.7205128205128 0.145797714591026
11.2307692307692 0.177284613251686
11.7666666666667 0.167158111929893
12.3282051282051 0.122983612120152
12.9153846153846 0.152693763375282
13.5282051282051 0.138780981302261
14.174358974359 0.146117478609085
14.8487179487179 0.122104935348034
15.5564102564103 0.127807006239891
16.2974358974359 0.11899458616972
17.0717948717949 0.104685358703136
17.8846153846154 0.126593083143234
18.7358974358974 0.0763556584715843
19.6282051282051 0.106158912181854
20.5641025641026 0.092409186065197
21.5435897435897 0.11849470436573
22.5692307692308 0.0752997621893883
23.6435897435897 0.0836492925882339
24.7692307692308 0.0670356079936028
25.9487179487179 0.0753859430551529
27.1846153846154 0.0754093527793884
28.4794871794872 0.0782561227679253
29.8358974358974 0.0799943804740906
31.2564102564103 0.0866561308503151
32.7435897435897 0.0878168717026711
34.3025641025641 0.0794418826699257
35.9358974358974 0.0935513749718666
37.648717948718 0.0985497161746025
39.4410256410256 0.10114947706461
41.3179487179487 0.0993190631270409
43.2871794871795 0.107431314885616
45.3487179487179 0.105417720973492
47.5076923076923 0.103016532957554
49.7692307692308 0.107014678418636
52.1384615384615 0.103487983345985
54.6205128205128 0.105550192296505
57.2230769230769 0.105371907353401
59.9461538461538 0.100185729563236
62.8025641025641 0.0988427028059959
65.7923076923077 0.0678555890917778
68.925641025641 0.100860260426998
72.2076923076923 0.0939852148294449
75.6461538461539 0.100418977439404
79.2461538461538 0.0722875967621803
83.0205128205128 0.0983309522271156
86.974358974359 0.100937724113464
91.1153846153846 0.101440265774727
95.4538461538462 0.0974875092506409
100 0.0879229456186295
};
\addlegendentry{mb 2, exact}
\addplot [, color0, opacity=0.6, mark=triangle*, mark size=0.5, mark options={solid,rotate=180}, only marks, forget plot]
table {%
1 0.860156238079071
1.04615384615385 0.55435436964035
1.0974358974359 0.513881921768188
1.14871794871795 0.412069708108902
1.2025641025641 0.45983561873436
1.26153846153846 0.425520986318588
1.32051282051282 0.406377077102661
1.38461538461538 0.376234829425812
1.44871794871795 0.375154554843903
1.51794871794872 0.357694566249847
1.58974358974359 0.330127000808716
1.66666666666667 0.341553747653961
1.74615384615385 0.325628370046616
1.82820512820513 0.319416642189026
1.91538461538462 0.317777693271637
2.00769230769231 0.307725995779037
2.1025641025641 0.302972376346588
2.20512820512821 0.301707893610001
2.30769230769231 0.324722826480865
2.41794871794872 0.30107507109642
2.53333333333333 0.298221319913864
2.65384615384615 0.31688466668129
2.78205128205128 0.27636981010437
2.91282051282051 0.294025987386703
3.05384615384615 0.290811151266098
3.1974358974359 0.314726978540421
3.35128205128205 0.283009141683578
3.51025641025641 0.273170202970505
3.67692307692308 0.267809838056564
3.85128205128205 0.273654520511627
4.03589743589744 0.2597835958004
4.22820512820513 0.254717320203781
4.42820512820513 0.234138160943985
4.64102564102564 0.250003844499588
4.86153846153846 0.23247991502285
5.09230769230769 0.209158346056938
5.33589743589744 0.225532203912735
5.58974358974359 0.187979474663734
5.85641025641026 0.237729176878929
6.13589743589744 0.223293021321297
6.42564102564103 0.186790868639946
6.73333333333333 0.193418055772781
7.05384615384615 0.199886053800583
7.38974358974359 0.163161769509315
7.74102564102564 0.163973733782768
8.11025641025641 0.19355908036232
8.4974358974359 0.155432239174843
8.9 0.150776222348213
9.32564102564103 0.152059733867645
9.76923076923077 0.150382995605469
10.2333333333333 0.155687972903252
10.7205128205128 0.125480398535728
11.2307692307692 0.141938626766205
11.7666666666667 0.12253849953413
12.3282051282051 0.129056051373482
12.9153846153846 0.133234724402428
13.5282051282051 0.121513299643993
14.174358974359 0.10918714851141
14.8487179487179 0.10938274115324
15.5564102564103 0.104502141475677
16.2974358974359 0.103428743779659
17.0717948717949 0.0765472948551178
17.8846153846154 0.104869417846203
18.7358974358974 0.0754273384809494
19.6282051282051 0.105332016944885
20.5641025641026 0.0841090083122253
21.5435897435897 0.0928294956684113
22.5692307692308 0.0668514594435692
23.6435897435897 0.0763514041900635
24.7692307692308 0.0650515928864479
25.9487179487179 0.0677694380283356
27.1846153846154 0.0660687461495399
28.4794871794872 0.079690508544445
29.8358974358974 0.0798264071345329
31.2564102564103 0.0796260610222816
32.7435897435897 0.0804776176810265
34.3025641025641 0.0574897415935993
35.9358974358974 0.0852011069655418
37.648717948718 0.0875793769955635
39.4410256410256 0.0882190018892288
41.3179487179487 0.092965267598629
43.2871794871795 0.0930351987481117
45.3487179487179 0.0921454802155495
47.5076923076923 0.0912022814154625
49.7692307692308 0.0991519242525101
52.1384615384615 0.0876434221863747
54.6205128205128 0.0882805436849594
57.2230769230769 0.0974830389022827
59.9461538461538 0.0903393253684044
62.8025641025641 0.0921153947710991
65.7923076923077 0.0644759759306908
68.925641025641 0.0931715592741966
72.2076923076923 0.0863079577684402
75.6461538461539 0.0878442749381065
79.2461538461538 0.0642864033579826
83.0205128205128 0.0932041481137276
86.974358974359 0.0953576266765594
91.1153846153846 0.0953752174973488
95.4538461538462 0.0814596340060234
100 0.0834057480096817
};
\addplot [, color0, opacity=0.6, mark=triangle*, mark size=0.5, mark options={solid,rotate=180}, only marks, forget plot]
table {%
1 0.815615475177765
1.04615384615385 0.567070484161377
1.0974358974359 0.551407098770142
1.14871794871795 0.456358253955841
1.2025641025641 0.403757393360138
1.26153846153846 0.382956117391586
1.32051282051282 0.357266515493393
1.38461538461538 0.302408188581467
1.44871794871795 0.294763624668121
1.51794871794872 0.306659787893295
1.58974358974359 0.274574756622314
1.66666666666667 0.272709220647812
1.74615384615385 0.250370502471924
1.82820512820513 0.264001697301865
1.91538461538462 0.256459206342697
2.00769230769231 0.227220058441162
2.1025641025641 0.237213283777237
2.20512820512821 0.231595903635025
2.30769230769231 0.230742856860161
2.41794871794872 0.217478662729263
2.53333333333333 0.2250796854496
2.65384615384615 0.240192458033562
2.78205128205128 0.198347091674805
2.91282051282051 0.214276030659676
3.05384615384615 0.222010999917984
3.1974358974359 0.207402259111404
3.35128205128205 0.20348171889782
3.51025641025641 0.189477786421776
3.67692307692308 0.20351879298687
3.85128205128205 0.193052187561989
4.03589743589744 0.185597166419029
4.22820512820513 0.200690731406212
4.42820512820513 0.18082882463932
4.64102564102564 0.168138459324837
4.86153846153846 0.166707307100296
5.09230769230769 0.153511270880699
5.33589743589744 0.15968219935894
5.58974358974359 0.15216164290905
5.85641025641026 0.155449464917183
6.13589743589744 0.144037678837776
6.42564102564103 0.12876832485199
6.73333333333333 0.124823115766048
7.05384615384615 0.128740936517715
7.38974358974359 0.10589575022459
7.74102564102564 0.115572668612003
8.11025641025641 0.0903870835900307
8.4974358974359 0.110242538154125
8.9 0.0873178318142891
9.32564102564103 0.0939392000436783
9.76923076923077 0.0797256380319595
10.2333333333333 0.0894343256950378
10.7205128205128 0.0765309482812881
11.2307692307692 0.0642874166369438
11.7666666666667 0.0785159692168236
12.3282051282051 0.0853806063532829
12.9153846153846 0.0580415837466717
13.5282051282051 0.0628701224923134
14.174358974359 0.0553828477859497
14.8487179487179 0.053193736821413
15.5564102564103 0.0537879467010498
16.2974358974359 0.0602776519954205
17.0717948717949 0.0535960979759693
17.8846153846154 0.0536823235452175
18.7358974358974 0.0433096811175346
19.6282051282051 0.0404500402510166
20.5641025641026 0.0499706938862801
21.5435897435897 0.0467126928269863
22.5692307692308 0.0394668988883495
23.6435897435897 0.0571500323712826
24.7692307692308 0.0449243262410164
25.9487179487179 0.0374329872429371
27.1846153846154 0.0433085858821869
28.4794871794872 0.0449031554162502
29.8358974358974 0.0471164472401142
31.2564102564103 0.052008718252182
32.7435897435897 0.0550362654030323
34.3025641025641 0.0538701787590981
35.9358974358974 0.0540409199893475
37.648717948718 0.0557979010045528
39.4410256410256 0.0534689426422119
41.3179487179487 0.0528673641383648
43.2871794871795 0.0520407073199749
45.3487179487179 0.0558118708431721
47.5076923076923 0.0589669123291969
49.7692307692308 0.0585004761815071
52.1384615384615 0.0568538010120392
54.6205128205128 0.0564343929290771
57.2230769230769 0.0525758378207684
59.9461538461538 0.0584896206855774
62.8025641025641 0.0563374422490597
65.7923076923077 0.0597976222634315
68.925641025641 0.0593527555465698
72.2076923076923 0.0612475387752056
75.6461538461539 0.0623884201049805
79.2461538461538 0.0648704692721367
83.0205128205128 0.0583639703691006
86.974358974359 0.0614294707775116
91.1153846153846 0.0625423118472099
95.4538461538462 0.0605409219861031
100 0.0640914440155029
};
\addplot [, color0, opacity=0.6, mark=triangle*, mark size=0.5, mark options={solid,rotate=180}, only marks, forget plot]
table {%
1 0.760009109973907
1.04615384615385 0.495772808790207
1.0974358974359 0.40398508310318
1.14871794871795 0.402524322271347
1.2025641025641 0.366293728351593
1.26153846153846 0.336805075407028
1.32051282051282 0.311819106340408
1.38461538461538 0.313955277204514
1.44871794871795 0.281458854675293
1.51794871794872 0.273651272058487
1.58974358974359 0.298134863376617
1.66666666666667 0.274847507476807
1.74615384615385 0.267755895853043
1.82820512820513 0.247221097350121
1.91538461538462 0.240712553262711
2.00769230769231 0.271234691143036
2.1025641025641 0.263810843229294
2.20512820512821 0.241067752242088
2.30769230769231 0.239540010690689
2.41794871794872 0.234699159860611
2.53333333333333 0.231165170669556
2.65384615384615 0.212441831827164
2.78205128205128 0.210340365767479
2.91282051282051 0.22141495347023
3.05384615384615 0.248174741864204
3.1974358974359 0.214089900255203
3.35128205128205 0.192129775881767
3.51025641025641 0.209225222468376
3.67692307692308 0.190959021449089
3.85128205128205 0.200231105089188
4.03589743589744 0.214431315660477
4.22820512820513 0.196474686264992
4.42820512820513 0.183813735842705
4.64102564102564 0.190985813736916
4.86153846153846 0.164435371756554
5.09230769230769 0.182868897914886
5.33589743589744 0.159407287836075
5.58974358974359 0.169824823737144
5.85641025641026 0.170012265443802
6.13589743589744 0.180737838149071
6.42564102564103 0.133585020899773
6.73333333333333 0.140251979231834
7.05384615384615 0.15931724011898
7.38974358974359 0.138338789343834
7.74102564102564 0.138370558619499
8.11025641025641 0.111581161618233
8.4974358974359 0.147724539041519
8.9 0.144258335232735
9.32564102564103 0.135258302092552
9.76923076923077 0.0954814851284027
10.2333333333333 0.133621647953987
10.7205128205128 0.0888520106673241
11.2307692307692 0.0946044698357582
11.7666666666667 0.132985517382622
12.3282051282051 0.110128663480282
12.9153846153846 0.065111868083477
13.5282051282051 0.0699034109711647
14.174358974359 0.0816421359777451
14.8487179487179 0.0666072741150856
15.5564102564103 0.0622351244091988
16.2974358974359 0.0644682422280312
17.0717948717949 0.0631838664412498
17.8846153846154 0.0663440227508545
18.7358974358974 0.0554499588906765
19.6282051282051 0.0563324689865112
20.5641025641026 0.0555087812244892
21.5435897435897 0.0662388578057289
22.5692307692308 0.0528650060296059
23.6435897435897 0.0595816150307655
24.7692307692308 0.0510871522128582
25.9487179487179 0.0584490969777107
27.1846153846154 0.0540111474692822
28.4794871794872 0.0597152002155781
29.8358974358974 0.0603261366486549
31.2564102564103 0.0578301846981049
32.7435897435897 0.0612560771405697
34.3025641025641 0.0616822503507137
35.9358974358974 0.0620362050831318
37.648717948718 0.0578932240605354
39.4410256410256 0.0616488456726074
41.3179487179487 0.0627674758434296
43.2871794871795 0.0644053220748901
45.3487179487179 0.0617533884942532
47.5076923076923 0.0617232583463192
49.7692307692308 0.060748279094696
52.1384615384615 0.0638751164078712
54.6205128205128 0.0579171255230904
57.2230769230769 0.0633584409952164
59.9461538461538 0.0589615292847157
62.8025641025641 0.0619599111378193
65.7923076923077 0.0516028888523579
68.925641025641 0.0635199323296547
72.2076923076923 0.0592545866966248
75.6461538461539 0.062089528888464
79.2461538461538 0.0578220002353191
83.0205128205128 0.0586968436837196
86.974358974359 0.0577326230704784
91.1153846153846 0.0632943287491798
95.4538461538462 0.0615967288613319
100 0.0553217492997646
};
\addplot [, color0, opacity=0.6, mark=triangle*, mark size=0.5, mark options={solid,rotate=180}, only marks, forget plot]
table {%
1 0.764424324035645
1.04615384615385 0.507153034210205
1.0974358974359 0.437207221984863
1.14871794871795 0.455385893583298
1.2025641025641 0.41924974322319
1.26153846153846 0.40088152885437
1.32051282051282 0.361143827438354
1.38461538461538 0.33158478140831
1.44871794871795 0.322057247161865
1.51794871794872 0.322249889373779
1.58974358974359 0.307981163263321
1.66666666666667 0.302958190441132
1.74615384615385 0.289200693368912
1.82820512820513 0.275680035352707
1.91538461538462 0.274577915668488
2.00769230769231 0.268611639738083
2.1025641025641 0.284532994031906
2.20512820512821 0.264379590749741
2.30769230769231 0.266236543655396
2.41794871794872 0.231843382120132
2.53333333333333 0.251180410385132
2.65384615384615 0.236394554376602
2.78205128205128 0.225451350212097
2.91282051282051 0.228133633732796
3.05384615384615 0.230999708175659
3.1974358974359 0.236868858337402
3.35128205128205 0.216068223118782
3.51025641025641 0.207708120346069
3.67692307692308 0.214693456888199
3.85128205128205 0.230287119746208
4.03589743589744 0.221523195505142
4.22820512820513 0.21559739112854
4.42820512820513 0.205271273851395
4.64102564102564 0.198951914906502
4.86153846153846 0.184730216860771
5.09230769230769 0.17639784514904
5.33589743589744 0.194086581468582
5.58974358974359 0.169973999261856
5.85641025641026 0.173025086522102
6.13589743589744 0.18039134144783
6.42564102564103 0.154392570257187
6.73333333333333 0.148063972592354
7.05384615384615 0.158064499497414
7.38974358974359 0.149865821003914
7.74102564102564 0.132009074091911
8.11025641025641 0.124777369201183
8.4974358974359 0.136563569307327
8.9 0.121448516845703
9.32564102564103 0.130249053239822
9.76923076923077 0.114281557500362
10.2333333333333 0.121043525636196
10.7205128205128 0.109863579273224
11.2307692307692 0.107566393911839
11.7666666666667 0.103225186467171
12.3282051282051 0.100380085408688
12.9153846153846 0.0885907039046288
13.5282051282051 0.0905454158782959
14.174358974359 0.0987218022346497
14.8487179487179 0.0835083872079849
15.5564102564103 0.0860051214694977
16.2974358974359 0.0826871320605278
17.0717948717949 0.066032312810421
17.8846153846154 0.0827623531222343
18.7358974358974 0.061570692807436
19.6282051282051 0.081868015229702
20.5641025641026 0.0724557936191559
21.5435897435897 0.0787748694419861
22.5692307692308 0.0609469525516033
23.6435897435897 0.0620013065636158
24.7692307692308 0.0551753714680672
25.9487179487179 0.0552719533443451
27.1846153846154 0.0581364519894123
28.4794871794872 0.0692370608448982
29.8358974358974 0.0803605541586876
31.2564102564103 0.0800658613443375
32.7435897435897 0.0858613029122353
34.3025641025641 0.0840422734618187
35.9358974358974 0.0905877351760864
37.648717948718 0.0898654833436012
39.4410256410256 0.0970813930034637
41.3179487179487 0.0989624410867691
43.2871794871795 0.0980437248945236
45.3487179487179 0.0969468131661415
47.5076923076923 0.0939667150378227
49.7692307692308 0.0953759104013443
52.1384615384615 0.0980751365423203
54.6205128205128 0.0955007895827293
57.2230769230769 0.0995697751641273
59.9461538461538 0.0925079509615898
62.8025641025641 0.0934563428163528
65.7923076923077 0.0757078304886818
68.925641025641 0.104566052556038
72.2076923076923 0.0955099388957024
75.6461538461539 0.0976598188281059
79.2461538461538 0.0803440138697624
83.0205128205128 0.0950003787875175
86.974358974359 0.0985060557723045
91.1153846153846 0.0990588366985321
95.4538461538462 0.0915067568421364
100 0.0856600925326347
};
\addplot [, color1, opacity=0.6, mark=square*, mark size=0.5, mark options={solid}, only marks]
table {%
1 0.907389640808105
1.04615384615385 0.6194988489151
1.0974358974359 0.591697812080383
1.14871794871795 0.541236042976379
1.2025641025641 0.547634065151215
1.26153846153846 0.533943474292755
1.32051282051282 0.504132568836212
1.38461538461538 0.442704498767853
1.44871794871795 0.445635080337524
1.51794871794872 0.473265737295151
1.58974358974359 0.437359243631363
1.66666666666667 0.451649397611618
1.74615384615385 0.441180050373077
1.82820512820513 0.440214157104492
1.91538461538462 0.405761241912842
2.00769230769231 0.374343484640121
2.1025641025641 0.409262627363205
2.20512820512821 0.413294404745102
2.30769230769231 0.410807579755783
2.41794871794872 0.361013740301132
2.53333333333333 0.38526725769043
2.65384615384615 0.356925696134567
2.78205128205128 0.347686141729355
2.91282051282051 0.365585505962372
3.05384615384615 0.348160713911057
3.1974358974359 0.333882957696915
3.35128205128205 0.334554821252823
3.51025641025641 0.375004380941391
3.67692307692308 0.32976245880127
3.85128205128205 0.341194778680801
4.03589743589744 0.322816044092178
4.22820512820513 0.327320873737335
4.42820512820513 0.318063855171204
4.64102564102564 0.303957998752594
4.86153846153846 0.309131890535355
5.09230769230769 0.276707977056503
5.33589743589744 0.311907529830933
5.58974358974359 0.269659489393234
5.85641025641026 0.29499888420105
6.13589743589744 0.261791437864304
6.42564102564103 0.246105536818504
6.73333333333333 0.26513996720314
7.05384615384615 0.247661858797073
7.38974358974359 0.261545300483704
7.74102564102564 0.228782221674919
8.11025641025641 0.249234676361084
8.4974358974359 0.238306000828743
8.9 0.223387762904167
9.32564102564103 0.234601765871048
9.76923076923077 0.208964273333549
10.2333333333333 0.241891220211983
10.7205128205128 0.245600372552872
11.2307692307692 0.208606049418449
11.7666666666667 0.203398749232292
12.3282051282051 0.229819253087044
12.9153846153846 0.21005642414093
13.5282051282051 0.201158598065376
14.174358974359 0.186826303601265
14.8487179487179 0.174245625734329
15.5564102564103 0.166709333658218
16.2974358974359 0.179308727383614
17.0717948717949 0.150481432676315
17.8846153846154 0.147546917200089
18.7358974358974 0.136059507727623
19.6282051282051 0.136614561080933
20.5641025641026 0.145237728953362
21.5435897435897 0.116574838757515
22.5692307692308 0.111598208546638
23.6435897435897 0.134054273366928
24.7692307692308 0.110279142856598
25.9487179487179 0.0905516669154167
27.1846153846154 0.0905214473605156
28.4794871794872 0.128983780741692
29.8358974358974 0.11542820930481
31.2564102564103 0.116055212914944
32.7435897435897 0.115370623767376
34.3025641025641 0.116512574255466
35.9358974358974 0.122918203473091
37.648717948718 0.121471405029297
39.4410256410256 0.127213001251221
41.3179487179487 0.126380234956741
43.2871794871795 0.125004708766937
45.3487179487179 0.130344584584236
47.5076923076923 0.1276516020298
49.7692307692308 0.133093640208244
52.1384615384615 0.131242319941521
54.6205128205128 0.131510838866234
57.2230769230769 0.129437372088432
59.9461538461538 0.125982210040092
62.8025641025641 0.133149400353432
65.7923076923077 0.110172562301159
68.925641025641 0.134220227599144
72.2076923076923 0.132600918412209
75.6461538461539 0.126475676894188
79.2461538461538 0.109258413314819
83.0205128205128 0.134777441620827
86.974358974359 0.13208569586277
91.1153846153846 0.13604561984539
95.4538461538462 0.127197042107582
100 0.125623852014542
};
\addlegendentry{mb 8, exact}
\addplot [, color1, opacity=0.6, mark=square*, mark size=0.5, mark options={solid}, only marks, forget plot]
table {%
1 0.90475195646286
1.04615384615385 0.681745648384094
1.0974358974359 0.64810174703598
1.14871794871795 0.539335429668427
1.2025641025641 0.532479763031006
1.26153846153846 0.551333844661713
1.32051282051282 0.530624210834503
1.38461538461538 0.504585206508636
1.44871794871795 0.504612565040588
1.51794871794872 0.418496608734131
1.58974358974359 0.436021894216537
1.66666666666667 0.429719835519791
1.74615384615385 0.435489267110825
1.82820512820513 0.401511490345001
1.91538461538462 0.41080829501152
2.00769230769231 0.385743349790573
2.1025641025641 0.405422419309616
2.20512820512821 0.311486095190048
2.30769230769231 0.384747713804245
2.41794871794872 0.374130636453629
2.53333333333333 0.357934683561325
2.65384615384615 0.372507631778717
2.78205128205128 0.338233649730682
2.91282051282051 0.340213745832443
3.05384615384615 0.359621286392212
3.1974358974359 0.309204310178757
3.35128205128205 0.28189879655838
3.51025641025641 0.292741626501083
3.67692307692308 0.293220102787018
3.85128205128205 0.293835014104843
4.03589743589744 0.306105583906174
4.22820512820513 0.290420860052109
4.42820512820513 0.276310682296753
4.64102564102564 0.287255853414536
4.86153846153846 0.248742774128914
5.09230769230769 0.280360996723175
5.33589743589744 0.277948081493378
5.58974358974359 0.251640856266022
5.85641025641026 0.267946690320969
6.13589743589744 0.249070838093758
6.42564102564103 0.246293380856514
6.73333333333333 0.244803234934807
7.05384615384615 0.221763610839844
7.38974358974359 0.207306072115898
7.74102564102564 0.210675939917564
8.11025641025641 0.195802137255669
8.4974358974359 0.197491094470024
8.9 0.218272969126701
9.32564102564103 0.172065198421478
9.76923076923077 0.196629703044891
10.2333333333333 0.179159983992577
10.7205128205128 0.179334044456482
11.2307692307692 0.178953260183334
11.7666666666667 0.160516038537025
12.3282051282051 0.178809762001038
12.9153846153846 0.147335723042488
13.5282051282051 0.152266472578049
14.174358974359 0.143801629543304
14.8487179487179 0.159118697047234
15.5564102564103 0.151612713932991
16.2974358974359 0.132236376404762
17.0717948717949 0.13957579433918
17.8846153846154 0.128952726721764
18.7358974358974 0.137620404362679
19.6282051282051 0.120383039116859
20.5641025641026 0.126915216445923
21.5435897435897 0.0987429171800613
22.5692307692308 0.11805534362793
23.6435897435897 0.114323332905769
24.7692307692308 0.0893370136618614
25.9487179487179 0.117119088768959
27.1846153846154 0.0802415087819099
28.4794871794872 0.105015218257904
29.8358974358974 0.10173662006855
31.2564102564103 0.0966549292206764
32.7435897435897 0.0944023355841637
34.3025641025641 0.0923084542155266
35.9358974358974 0.0968215838074684
37.648717948718 0.0929040089249611
39.4410256410256 0.0961397066712379
41.3179487179487 0.0962869748473167
43.2871794871795 0.0920645743608475
45.3487179487179 0.0992545187473297
47.5076923076923 0.0962629243731499
49.7692307692308 0.091681532561779
52.1384615384615 0.0995702147483826
54.6205128205128 0.098388247191906
57.2230769230769 0.0948848277330399
59.9461538461538 0.0989256501197815
62.8025641025641 0.103955686092377
65.7923076923077 0.0920670852065086
68.925641025641 0.101510487496853
72.2076923076923 0.0980883538722992
75.6461538461539 0.0964685529470444
79.2461538461538 0.0928686186671257
83.0205128205128 0.097955621778965
86.974358974359 0.0960741490125656
91.1153846153846 0.0983426049351692
95.4538461538462 0.0960203558206558
100 0.0947395861148834
};
\addplot [, color1, opacity=0.6, mark=square*, mark size=0.5, mark options={solid}, only marks, forget plot]
table {%
1 0.881046295166016
1.04615384615385 0.6625856757164
1.0974358974359 0.587013363838196
1.14871794871795 0.566249132156372
1.2025641025641 0.435399532318115
1.26153846153846 0.430656433105469
1.32051282051282 0.47309073805809
1.38461538461538 0.48922061920166
1.44871794871795 0.467534154653549
1.51794871794872 0.402398824691772
1.58974358974359 0.385412305593491
1.66666666666667 0.433475881814957
1.74615384615385 0.369408458471298
1.82820512820513 0.408005952835083
1.91538461538462 0.386181801557541
2.00769230769231 0.449086755514145
2.1025641025641 0.383452743291855
2.20512820512821 0.410401672124863
2.30769230769231 0.414294332265854
2.41794871794872 0.429985046386719
2.53333333333333 0.346962839365005
2.65384615384615 0.41933137178421
2.78205128205128 0.339547634124756
2.91282051282051 0.307453185319901
3.05384615384615 0.333435297012329
3.1974358974359 0.351337343454361
3.35128205128205 0.294329404830933
3.51025641025641 0.349201887845993
3.67692307692308 0.349346399307251
3.85128205128205 0.29557204246521
4.03589743589744 0.316316187381744
4.22820512820513 0.30788466334343
4.42820512820513 0.342187017202377
4.64102564102564 0.276845514774323
4.86153846153846 0.264276653528214
5.09230769230769 0.282047718763351
5.33589743589744 0.28504490852356
5.58974358974359 0.2558753490448
5.85641025641026 0.288881450891495
6.13589743589744 0.253408908843994
6.42564102564103 0.246300846338272
6.73333333333333 0.274892598390579
7.05384615384615 0.24406872689724
7.38974358974359 0.268667906522751
7.74102564102564 0.220961004495621
8.11025641025641 0.279090017080307
8.4974358974359 0.255142092704773
8.9 0.212555631995201
9.32564102564103 0.219062566757202
9.76923076923077 0.226809456944466
10.2333333333333 0.228582665324211
10.7205128205128 0.21075701713562
11.2307692307692 0.233912378549576
11.7666666666667 0.204219490289688
12.3282051282051 0.187726989388466
12.9153846153846 0.178686410188675
13.5282051282051 0.240624234080315
14.174358974359 0.20317630469799
14.8487179487179 0.223036393523216
15.5564102564103 0.181342720985413
16.2974358974359 0.173855498433113
17.0717948717949 0.1631138920784
17.8846153846154 0.179864481091499
18.7358974358974 0.156614825129509
19.6282051282051 0.162090584635735
20.5641025641026 0.160578414797783
21.5435897435897 0.150482803583145
22.5692307692308 0.154282316565514
23.6435897435897 0.156566143035889
24.7692307692308 0.16075237095356
25.9487179487179 0.14496223628521
27.1846153846154 0.150568753480911
28.4794871794872 0.132746696472168
29.8358974358974 0.140075191855431
31.2564102564103 0.15561580657959
32.7435897435897 0.133770987391472
34.3025641025641 0.135414272546768
35.9358974358974 0.141078665852547
37.648717948718 0.141235813498497
39.4410256410256 0.135670229792595
41.3179487179487 0.129877328872681
43.2871794871795 0.13569863140583
45.3487179487179 0.142739176750183
47.5076923076923 0.13614284992218
49.7692307692308 0.140865325927734
52.1384615384615 0.134853079915047
54.6205128205128 0.153905361890793
57.2230769230769 0.153097659349442
59.9461538461538 0.144257590174675
62.8025641025641 0.141404375433922
65.7923076923077 0.153063759207726
68.925641025641 0.145923539996147
72.2076923076923 0.138038024306297
75.6461538461539 0.136320158839226
79.2461538461538 0.138301581144333
83.0205128205128 0.140976741909981
86.974358974359 0.132904037833214
91.1153846153846 0.145182281732559
95.4538461538462 0.147321581840515
100 0.13508328795433
};
\addplot [, color1, opacity=0.6, mark=square*, mark size=0.5, mark options={solid}, only marks, forget plot]
table {%
1 0.866020023822784
1.04615384615385 0.744210720062256
1.0974358974359 0.544375419616699
1.14871794871795 0.495109051465988
1.2025641025641 0.48607674241066
1.26153846153846 0.48066121339798
1.32051282051282 0.400053083896637
1.38461538461538 0.354795843362808
1.44871794871795 0.352775186300278
1.51794871794872 0.432038635015488
1.58974358974359 0.354846596717834
1.66666666666667 0.358901232481003
1.74615384615385 0.33120521903038
1.82820512820513 0.360823005437851
1.91538461538462 0.358723551034927
2.00769230769231 0.352127075195312
2.1025641025641 0.328231066465378
2.20512820512821 0.33050349354744
2.30769230769231 0.359756886959076
2.41794871794872 0.363259702920914
2.53333333333333 0.334725856781006
2.65384615384615 0.350171744823456
2.78205128205128 0.317881911993027
2.91282051282051 0.332164525985718
3.05384615384615 0.3006571829319
3.1974358974359 0.345649898052216
3.35128205128205 0.297827810049057
3.51025641025641 0.33513531088829
3.67692307692308 0.313300043344498
3.85128205128205 0.295180767774582
4.03589743589744 0.306694179773331
4.22820512820513 0.307981342077255
4.42820512820513 0.277676045894623
4.64102564102564 0.294829457998276
4.86153846153846 0.257381826639175
5.09230769230769 0.259939968585968
5.33589743589744 0.268885731697083
5.58974358974359 0.233838751912117
5.85641025641026 0.287585467100143
6.13589743589744 0.258958786725998
6.42564102564103 0.227832421660423
6.73333333333333 0.27102854847908
7.05384615384615 0.252081453800201
7.38974358974359 0.234896823763847
7.74102564102564 0.222188040614128
8.11025641025641 0.243903398513794
8.4974358974359 0.216594025492668
8.9 0.194569259881973
9.32564102564103 0.195392325520515
9.76923076923077 0.220135360956192
10.2333333333333 0.232761219143867
10.7205128205128 0.190155908465385
11.2307692307692 0.209902077913284
11.7666666666667 0.175749212503433
12.3282051282051 0.175862282514572
12.9153846153846 0.198912262916565
13.5282051282051 0.19781431555748
14.174358974359 0.161701902747154
14.8487179487179 0.160961493849754
15.5564102564103 0.15851953625679
16.2974358974359 0.161576703190804
17.0717948717949 0.132688209414482
17.8846153846154 0.150422140955925
18.7358974358974 0.112727999687195
19.6282051282051 0.14239327609539
20.5641025641026 0.135196536779404
21.5435897435897 0.130408719182014
22.5692307692308 0.128572687506676
23.6435897435897 0.124093532562256
24.7692307692308 0.109631851315498
25.9487179487179 0.109347440302372
27.1846153846154 0.100770018994808
28.4794871794872 0.132307276129723
29.8358974358974 0.119477890431881
31.2564102564103 0.125213176012039
32.7435897435897 0.131021991372108
34.3025641025641 0.126025557518005
35.9358974358974 0.137377068400383
37.648717948718 0.140868708491325
39.4410256410256 0.140819489955902
41.3179487179487 0.141196444630623
43.2871794871795 0.143327802419662
45.3487179487179 0.147089287638664
47.5076923076923 0.140889167785645
49.7692307692308 0.142619088292122
52.1384615384615 0.14415131509304
54.6205128205128 0.143941327929497
57.2230769230769 0.144952267408371
59.9461538461538 0.14106522500515
62.8025641025641 0.143287017941475
65.7923076923077 0.115470699965954
68.925641025641 0.146567389369011
72.2076923076923 0.139720112085342
75.6461538461539 0.141928240656853
79.2461538461538 0.117552518844604
83.0205128205128 0.144683137536049
86.974358974359 0.144276812672615
91.1153846153846 0.142074108123779
95.4538461538462 0.134899660944939
100 0.126484632492065
};
\addplot [, color1, opacity=0.6, mark=square*, mark size=0.5, mark options={solid}, only marks, forget plot]
table {%
1 0.870053291320801
1.04615384615385 0.651790618896484
1.0974358974359 0.609957873821259
1.14871794871795 0.608211517333984
1.2025641025641 0.530338227748871
1.26153846153846 0.463695734739304
1.32051282051282 0.448801606893539
1.38461538461538 0.451421350240707
1.44871794871795 0.428424119949341
1.51794871794872 0.38992714881897
1.58974358974359 0.438740700483322
1.66666666666667 0.391558080911636
1.74615384615385 0.393378734588623
1.82820512820513 0.372445702552795
1.91538461538462 0.359768033027649
2.00769230769231 0.361801713705063
2.1025641025641 0.365439534187317
2.20512820512821 0.380638211965561
2.30769230769231 0.344528406858444
2.41794871794872 0.354793190956116
2.53333333333333 0.366121053695679
2.65384615384615 0.331303209066391
2.78205128205128 0.309357970952988
2.91282051282051 0.319646179676056
3.05384615384615 0.332941144704819
3.1974358974359 0.321307241916656
3.35128205128205 0.291066944599152
3.51025641025641 0.286476075649261
3.67692307692308 0.28660923242569
3.85128205128205 0.284531384706497
4.03589743589744 0.306107372045517
4.22820512820513 0.280641078948975
4.42820512820513 0.294458389282227
4.64102564102564 0.277068465948105
4.86153846153846 0.260461896657944
5.09230769230769 0.268589973449707
5.33589743589744 0.266766250133514
5.58974358974359 0.25491127371788
5.85641025641026 0.249260529875755
6.13589743589744 0.251606374979019
6.42564102564103 0.236575126647949
6.73333333333333 0.242537006735802
7.05384615384615 0.240375190973282
7.38974358974359 0.239931777119637
7.74102564102564 0.22276246547699
8.11025641025641 0.241700693964958
8.4974358974359 0.257013946771622
8.9 0.225586220622063
9.32564102564103 0.211085274815559
9.76923076923077 0.230865433812141
10.2333333333333 0.197173476219177
10.7205128205128 0.2105373442173
11.2307692307692 0.205831721425056
11.7666666666667 0.198125228285789
12.3282051282051 0.201948687434196
12.9153846153846 0.178610324859619
13.5282051282051 0.188315212726593
14.174358974359 0.187878057360649
14.8487179487179 0.189431920647621
15.5564102564103 0.164421603083611
16.2974358974359 0.148001119494438
17.0717948717949 0.167398795485497
17.8846153846154 0.148899897933006
18.7358974358974 0.152325555682182
19.6282051282051 0.130599066615105
20.5641025641026 0.134041517972946
21.5435897435897 0.145330667495728
22.5692307692308 0.141567260026932
23.6435897435897 0.152409553527832
24.7692307692308 0.133440718054771
25.9487179487179 0.142448902130127
27.1846153846154 0.135158956050873
28.4794871794872 0.139913827180862
29.8358974358974 0.144831329584122
31.2564102564103 0.134451180696487
32.7435897435897 0.12746886909008
34.3025641025641 0.132421910762787
35.9358974358974 0.131143122911453
37.648717948718 0.132908821105957
39.4410256410256 0.135891541838646
41.3179487179487 0.135343596339226
43.2871794871795 0.140629485249519
45.3487179487179 0.138643607497215
47.5076923076923 0.12605245411396
49.7692307692308 0.13885098695755
52.1384615384615 0.143437892198563
54.6205128205128 0.141857013106346
57.2230769230769 0.142550155520439
59.9461538461538 0.146272510290146
62.8025641025641 0.141094952821732
65.7923076923077 0.137161120772362
68.925641025641 0.143922746181488
72.2076923076923 0.140580207109451
75.6461538461539 0.1409612596035
79.2461538461538 0.141659289598465
83.0205128205128 0.141465216875076
86.974358974359 0.140640243887901
91.1153846153846 0.143936082720757
95.4538461538462 0.148850798606873
100 0.141544908285141
};
\addplot [, color2, opacity=0.6, mark=diamond*, mark size=0.5, mark options={solid}, only marks]
table {%
1 0.943935215473175
1.04615384615385 0.93809586763382
1.0974358974359 0.863481462001801
1.14871794871795 0.809195935726166
1.2025641025641 0.716499924659729
1.26153846153846 0.672814667224884
1.32051282051282 0.674195468425751
1.38461538461538 0.61115950345993
1.44871794871795 0.554157555103302
1.51794871794872 0.517096042633057
1.58974358974359 0.583943486213684
1.66666666666667 0.554685294628143
1.74615384615385 0.585835099220276
1.82820512820513 0.581447184085846
1.91538461538462 0.491412252187729
2.00769230769231 0.513256371021271
2.1025641025641 0.466538190841675
2.20512820512821 0.419932454824448
2.30769230769231 0.462421238422394
2.41794871794872 0.550066113471985
2.53333333333333 0.432728379964828
2.65384615384615 0.438035875558853
2.78205128205128 0.443331331014633
2.91282051282051 0.413767904043198
3.05384615384615 0.462775051593781
3.1974358974359 0.411445826292038
3.35128205128205 0.405708640813828
3.51025641025641 0.382480472326279
3.67692307692308 0.343834936618805
3.85128205128205 0.361708790063858
4.03589743589744 0.366903513669968
4.22820512820513 0.328749448060989
4.42820512820513 0.360914528369904
4.64102564102564 0.32132750749588
4.86153846153846 0.301758527755737
5.09230769230769 0.28556427359581
5.33589743589744 0.322734415531158
5.58974358974359 0.300632476806641
5.85641025641026 0.307924538850784
6.13589743589744 0.280431300401688
6.42564102564103 0.234646365046501
6.73333333333333 0.261673420667648
7.05384615384615 0.255128473043442
7.38974358974359 0.263618737459183
7.74102564102564 0.243399649858475
8.11025641025641 0.23943804204464
8.4974358974359 0.252470999956131
8.9 0.252415269613266
9.32564102564103 0.235213041305542
9.76923076923077 0.167791470885277
10.2333333333333 0.212722018361092
10.7205128205128 0.22503824532032
11.2307692307692 0.213979676365852
11.7666666666667 0.212488636374474
12.3282051282051 0.195831701159477
12.9153846153846 0.190964981913567
13.5282051282051 0.156290277838707
14.174358974359 0.171311467885971
14.8487179487179 0.148083359003067
15.5564102564103 0.167847022414207
16.2974358974359 0.145258232951164
17.0717948717949 0.162274688482285
17.8846153846154 0.125705704092979
18.7358974358974 0.117599628865719
19.6282051282051 0.119428873062134
20.5641025641026 0.119699731469154
21.5435897435897 0.108113542199135
22.5692307692308 0.128212913870811
23.6435897435897 0.118496350944042
24.7692307692308 0.115302324295044
25.9487179487179 0.110586442053318
27.1846153846154 0.104785278439522
28.4794871794872 0.107117533683777
29.8358974358974 0.110078647732735
31.2564102564103 0.12357447296381
32.7435897435897 0.125097438693047
34.3025641025641 0.124516658484936
35.9358974358974 0.124255135655403
37.648717948718 0.124257139861584
39.4410256410256 0.129489660263062
41.3179487179487 0.130851075053215
43.2871794871795 0.124657861888409
45.3487179487179 0.123863458633423
47.5076923076923 0.132932931184769
49.7692307692308 0.128452658653259
52.1384615384615 0.120793655514717
54.6205128205128 0.125049456954002
57.2230769230769 0.130498483777046
59.9461538461538 0.127292200922966
62.8025641025641 0.124834470450878
65.7923076923077 0.135479301214218
68.925641025641 0.126250475645065
72.2076923076923 0.131709963083267
75.6461538461539 0.131237283349037
79.2461538461538 0.14096799492836
83.0205128205128 0.128384202718735
86.974358974359 0.134099423885345
91.1153846153846 0.130148530006409
95.4538461538462 0.132049843668938
100 0.140854462981224
};
\addlegendentry{mb 32, exact}
\addplot [, color2, opacity=0.6, mark=diamond*, mark size=0.5, mark options={solid}, only marks, forget plot]
table {%
1 0.971033215522766
1.04615384615385 0.855105221271515
1.0974358974359 0.828849613666534
1.14871794871795 0.856320381164551
1.2025641025641 0.742308795452118
1.26153846153846 0.756269574165344
1.32051282051282 0.736871182918549
1.38461538461538 0.694280624389648
1.44871794871795 0.638373672962189
1.51794871794872 0.612413227558136
1.58974358974359 0.646262109279633
1.66666666666667 0.60685133934021
1.74615384615385 0.60811710357666
1.82820512820513 0.592874884605408
1.91538461538462 0.590896248817444
2.00769230769231 0.567390143871307
2.1025641025641 0.562204658985138
2.20512820512821 0.553973197937012
2.30769230769231 0.552109062671661
2.41794871794872 0.496430695056915
2.53333333333333 0.548240184783936
2.65384615384615 0.498174875974655
2.78205128205128 0.527097702026367
2.91282051282051 0.536133766174316
3.05384615384615 0.559592306613922
3.1974358974359 0.499201208353043
3.35128205128205 0.4661885201931
3.51025641025641 0.520262539386749
3.67692307692308 0.44572925567627
3.85128205128205 0.455332368612289
4.03589743589744 0.452195376157761
4.22820512820513 0.388251274824142
4.42820512820513 0.400893688201904
4.64102564102564 0.394688278436661
4.86153846153846 0.438380628824234
5.09230769230769 0.472913414239883
5.33589743589744 0.3497174680233
5.58974358974359 0.415527075529099
5.85641025641026 0.385145097970963
6.13589743589744 0.416904300451279
6.42564102564103 0.382672607898712
6.73333333333333 0.384039700031281
7.05384615384615 0.338867664337158
7.38974358974359 0.375862807035446
7.74102564102564 0.357526212930679
8.11025641025641 0.350244492292404
8.4974358974359 0.339758068323135
8.9 0.352619498968124
9.32564102564103 0.296506851911545
9.76923076923077 0.352627962827682
10.2333333333333 0.309264123439789
10.7205128205128 0.28862139582634
11.2307692307692 0.282109171152115
11.7666666666667 0.232528686523438
12.3282051282051 0.306256204843521
12.9153846153846 0.343152076005936
13.5282051282051 0.290458261966705
14.174358974359 0.266379177570343
14.8487179487179 0.25775334239006
15.5564102564103 0.280398845672607
16.2974358974359 0.218723580241203
17.0717948717949 0.224613472819328
17.8846153846154 0.202945902943611
18.7358974358974 0.23304095864296
19.6282051282051 0.207478478550911
20.5641025641026 0.205012604594231
21.5435897435897 0.232806250452995
22.5692307692308 0.211458131670952
23.6435897435897 0.21272599697113
24.7692307692308 0.199055060744286
25.9487179487179 0.21211750805378
27.1846153846154 0.203120738267899
28.4794871794872 0.194165214896202
29.8358974358974 0.184643283486366
31.2564102564103 0.224875926971436
32.7435897435897 0.17747437953949
34.3025641025641 0.175188392400742
35.9358974358974 0.187530279159546
37.648717948718 0.187939405441284
39.4410256410256 0.181197762489319
41.3179487179487 0.167179569602013
43.2871794871795 0.18540047109127
45.3487179487179 0.185234352946281
47.5076923076923 0.178569987416267
49.7692307692308 0.177806377410889
52.1384615384615 0.184861943125725
54.6205128205128 0.18407940864563
57.2230769230769 0.191357091069221
59.9461538461538 0.189812898635864
62.8025641025641 0.190422877669334
65.7923076923077 0.194178983569145
68.925641025641 0.185439452528954
72.2076923076923 0.182717114686966
75.6461538461539 0.190831691026688
79.2461538461538 0.186565324664116
83.0205128205128 0.185521528124809
86.974358974359 0.172846481204033
91.1153846153846 0.190859153866768
95.4538461538462 0.193146005272865
100 0.19368639588356
};
\addplot [, color2, opacity=0.6, mark=diamond*, mark size=0.5, mark options={solid}, only marks, forget plot]
table {%
1 0.940145313739777
1.04615384615385 0.915837287902832
1.0974358974359 0.760414302349091
1.14871794871795 0.784418284893036
1.2025641025641 0.683900773525238
1.26153846153846 0.671772301197052
1.32051282051282 0.698531806468964
1.38461538461538 0.671330869197845
1.44871794871795 0.656006634235382
1.51794871794872 0.601931393146515
1.58974358974359 0.668590486049652
1.66666666666667 0.601351857185364
1.74615384615385 0.607471942901611
1.82820512820513 0.553194046020508
1.91538461538462 0.557714581489563
2.00769230769231 0.669593155384064
2.1025641025641 0.589406430721283
2.20512820512821 0.522189974784851
2.30769230769231 0.584576606750488
2.41794871794872 0.604787230491638
2.53333333333333 0.595297634601593
2.65384615384615 0.562601268291473
2.78205128205128 0.550758183002472
2.91282051282051 0.495357125997543
3.05384615384615 0.59639036655426
3.1974358974359 0.441741287708282
3.35128205128205 0.47147998213768
3.51025641025641 0.534143924713135
3.67692307692308 0.460953056812286
3.85128205128205 0.404619216918945
4.03589743589744 0.439873903989792
4.22820512820513 0.437759697437286
4.42820512820513 0.408349812030792
4.64102564102564 0.422312319278717
4.86153846153846 0.431376427412033
5.09230769230769 0.447295278310776
5.33589743589744 0.360042542219162
5.58974358974359 0.435308456420898
5.85641025641026 0.372612088918686
6.13589743589744 0.366876840591431
6.42564102564103 0.333227902650833
6.73333333333333 0.332951992750168
7.05384615384615 0.331768691539764
7.38974358974359 0.345480591058731
7.74102564102564 0.334746956825256
8.11025641025641 0.36106738448143
8.4974358974359 0.305079281330109
8.9 0.352365016937256
9.32564102564103 0.276012152433395
9.76923076923077 0.267902374267578
10.2333333333333 0.256920248270035
10.7205128205128 0.313522547483444
11.2307692307692 0.249882265925407
11.7666666666667 0.270653635263443
12.3282051282051 0.198130831122398
12.9153846153846 0.268580824136734
13.5282051282051 0.230753466486931
14.174358974359 0.282653719186783
14.8487179487179 0.172790333628654
15.5564102564103 0.181390762329102
16.2974358974359 0.165674433112144
17.0717948717949 0.187824964523315
17.8846153846154 0.184125930070877
18.7358974358974 0.158549174666405
19.6282051282051 0.162581279873848
20.5641025641026 0.160407692193985
21.5435897435897 0.121001996099949
22.5692307692308 0.167027562856674
23.6435897435897 0.171023637056351
24.7692307692308 0.149309948086739
25.9487179487179 0.156651124358177
27.1846153846154 0.150214299559593
28.4794871794872 0.146325632929802
29.8358974358974 0.128571316599846
31.2564102564103 0.138370752334595
32.7435897435897 0.135339483618736
34.3025641025641 0.125864446163177
35.9358974358974 0.12948440015316
37.648717948718 0.125074177980423
39.4410256410256 0.122908115386963
41.3179487179487 0.124825529754162
43.2871794871795 0.125261172652245
45.3487179487179 0.123443029820919
47.5076923076923 0.119707964360714
49.7692307692308 0.122062347829342
52.1384615384615 0.124072648584843
54.6205128205128 0.126632049679756
57.2230769230769 0.122397996485233
59.9461538461538 0.126160979270935
62.8025641025641 0.129853948950768
65.7923076923077 0.124782159924507
68.925641025641 0.105842135846615
72.2076923076923 0.121117070317268
75.6461538461539 0.1229647397995
79.2461538461538 0.12495543807745
83.0205128205128 0.123325608670712
86.974358974359 0.12383458763361
91.1153846153846 0.113452292978764
95.4538461538462 0.122287772595882
100 0.12540590763092
};
\addplot [, color2, opacity=0.6, mark=diamond*, mark size=0.5, mark options={solid}, only marks, forget plot]
table {%
1 0.975739657878876
1.04615384615385 0.92137736082077
1.0974358974359 0.760971546173096
1.14871794871795 0.730047702789307
1.2025641025641 0.682512521743774
1.26153846153846 0.662022054195404
1.32051282051282 0.68131411075592
1.38461538461538 0.633741676807404
1.44871794871795 0.610767066478729
1.51794871794872 0.633241236209869
1.58974358974359 0.627216160297394
1.66666666666667 0.636717617511749
1.74615384615385 0.609613418579102
1.82820512820513 0.553987503051758
1.91538461538462 0.577254712581635
2.00769230769231 0.552401959896088
2.1025641025641 0.568991839885712
2.20512820512821 0.553679645061493
2.30769230769231 0.570137500762939
2.41794871794872 0.514746129512787
2.53333333333333 0.520656108856201
2.65384615384615 0.494966894388199
2.78205128205128 0.496744602918625
2.91282051282051 0.505426108837128
3.05384615384615 0.508021473884583
3.1974358974359 0.462987989187241
3.35128205128205 0.481229305267334
3.51025641025641 0.425920069217682
3.67692307692308 0.440699398517609
3.85128205128205 0.362289577722549
4.03589743589744 0.394796669483185
4.22820512820513 0.307014316320419
4.42820512820513 0.414540112018585
4.64102564102564 0.379845947027206
4.86153846153846 0.367331475019455
5.09230769230769 0.380717009305954
5.33589743589744 0.348068177700043
5.58974358974359 0.358842700719833
5.85641025641026 0.324923455715179
6.13589743589744 0.277106523513794
6.42564102564103 0.301837652921677
6.73333333333333 0.293822288513184
7.05384615384615 0.237879186868668
7.38974358974359 0.289032697677612
7.74102564102564 0.328571051359177
8.11025641025641 0.311625152826309
8.4974358974359 0.258209556341171
8.9 0.289274603128433
9.32564102564103 0.191274911165237
9.76923076923077 0.214456543326378
10.2333333333333 0.243872210383415
10.7205128205128 0.218859985470772
11.2307692307692 0.183602631092072
11.7666666666667 0.157648608088493
12.3282051282051 0.166819259524345
12.9153846153846 0.198277026414871
13.5282051282051 0.198359549045563
14.174358974359 0.228740409016609
14.8487179487179 0.141407400369644
15.5564102564103 0.146006748080254
16.2974358974359 0.169589444994926
17.0717948717949 0.134069487452507
17.8846153846154 0.158391788601875
18.7358974358974 0.155896559357643
19.6282051282051 0.161858081817627
20.5641025641026 0.137317910790443
21.5435897435897 0.165189608931541
22.5692307692308 0.177792847156525
23.6435897435897 0.145831435918808
24.7692307692308 0.157753735780716
25.9487179487179 0.130138948559761
27.1846153846154 0.139035806059837
28.4794871794872 0.120551683008671
29.8358974358974 0.128459483385086
31.2564102564103 0.146938160061836
32.7435897435897 0.129980906844139
34.3025641025641 0.139989212155342
35.9358974358974 0.142062425613403
37.648717948718 0.141513586044312
39.4410256410256 0.136231660842896
41.3179487179487 0.121527530252934
43.2871794871795 0.133489981293678
45.3487179487179 0.127508848905563
47.5076923076923 0.123433530330658
49.7692307692308 0.118494726717472
52.1384615384615 0.127787098288536
54.6205128205128 0.126921191811562
57.2230769230769 0.126834213733673
59.9461538461538 0.134276896715164
62.8025641025641 0.128825858235359
65.7923076923077 0.126181244850159
68.925641025641 0.132749676704407
72.2076923076923 0.126358702778816
75.6461538461539 0.123605251312256
79.2461538461538 0.131610229611397
83.0205128205128 0.126625806093216
86.974358974359 0.118473224341869
91.1153846153846 0.131264537572861
95.4538461538462 0.131634294986725
100 0.13226030766964
};
\addplot [, color2, opacity=0.6, mark=diamond*, mark size=0.5, mark options={solid}, only marks, forget plot]
table {%
1 0.952633082866669
1.04615384615385 0.903267502784729
1.0974358974359 0.793491542339325
1.14871794871795 0.780839145183563
1.2025641025641 0.743993282318115
1.26153846153846 0.734797894954681
1.32051282051282 0.701653003692627
1.38461538461538 0.680005967617035
1.44871794871795 0.613892734050751
1.51794871794872 0.667347311973572
1.58974358974359 0.584311842918396
1.66666666666667 0.490297466516495
1.74615384615385 0.593291878700256
1.82820512820513 0.616912662982941
1.91538461538462 0.524292528629303
2.00769230769231 0.549291551113129
2.1025641025641 0.49665841460228
2.20512820512821 0.493371814489365
2.30769230769231 0.608119189739227
2.41794871794872 0.568088054656982
2.53333333333333 0.548240840435028
2.65384615384615 0.539231956005096
2.78205128205128 0.508765876293182
2.91282051282051 0.465526014566422
3.05384615384615 0.532103061676025
3.1974358974359 0.508229315280914
3.35128205128205 0.405637174844742
3.51025641025641 0.417821228504181
3.67692307692308 0.476782947778702
3.85128205128205 0.412455946207047
4.03589743589744 0.402409791946411
4.22820512820513 0.400019943714142
4.42820512820513 0.453623354434967
4.64102564102564 0.376035183668137
4.86153846153846 0.438058346509933
5.09230769230769 0.392801612615585
5.33589743589744 0.371996134519577
5.58974358974359 0.387209236621857
5.85641025641026 0.370586693286896
6.13589743589744 0.347138255834579
6.42564102564103 0.316300123929977
6.73333333333333 0.337422460317612
7.05384615384615 0.284922808408737
7.38974358974359 0.304535716772079
7.74102564102564 0.315229028463364
8.11025641025641 0.333064407110214
8.4974358974359 0.261631667613983
8.9 0.258829772472382
9.32564102564103 0.259235799312592
9.76923076923077 0.265514850616455
10.2333333333333 0.242858558893204
10.7205128205128 0.247738406062126
11.2307692307692 0.238764196634293
11.7666666666667 0.192954197525978
12.3282051282051 0.26243382692337
12.9153846153846 0.267645567655563
13.5282051282051 0.214400559663773
14.174358974359 0.222137793898582
14.8487179487179 0.254658311605453
15.5564102564103 0.207747802138329
16.2974358974359 0.228937581181526
17.0717948717949 0.224327564239502
17.8846153846154 0.195281565189362
18.7358974358974 0.183671712875366
19.6282051282051 0.171105787158012
20.5641025641026 0.163975432515144
21.5435897435897 0.164558157324791
22.5692307692308 0.151736214756966
23.6435897435897 0.172348961234093
24.7692307692308 0.150696203112602
25.9487179487179 0.148418828845024
27.1846153846154 0.142323762178421
28.4794871794872 0.159958094358444
29.8358974358974 0.125118598341942
31.2564102564103 0.126357704401016
32.7435897435897 0.130173832178116
34.3025641025641 0.133099660277367
35.9358974358974 0.135432153940201
37.648717948718 0.135681137442589
39.4410256410256 0.135910242795944
41.3179487179487 0.134965822100639
43.2871794871795 0.141728326678276
45.3487179487179 0.140509724617004
47.5076923076923 0.136040702462196
49.7692307692308 0.135799676179886
52.1384615384615 0.14285534620285
54.6205128205128 0.142967462539673
57.2230769230769 0.13880954682827
59.9461538461538 0.138778001070023
62.8025641025641 0.144370123744011
65.7923076923077 0.136582404375076
68.925641025641 0.13115020096302
72.2076923076923 0.14208011329174
75.6461538461539 0.136833533644676
79.2461538461538 0.138107299804688
83.0205128205128 0.14589624106884
86.974358974359 0.139933660626411
91.1153846153846 0.142590686678886
95.4538461538462 0.144812390208244
100 0.148029610514641
};
\addplot [, black, opacity=0.6, mark=*, mark size=0.5, mark options={solid}, only marks]
table {%
1 0.994974613189697
1.04615384615385 0.978533208370209
1.0974358974359 0.88761568069458
1.14871794871795 0.882425129413605
1.2025641025641 0.912472188472748
1.26153846153846 0.891978442668915
1.32051282051282 0.833248555660248
1.38461538461538 0.843169629573822
1.44871794871795 0.887269198894501
1.51794871794872 0.808582723140717
1.58974358974359 0.806749761104584
1.66666666666667 0.831486701965332
1.74615384615385 0.871429443359375
1.82820512820513 0.798410594463348
1.91538461538462 0.858729362487793
2.00769230769231 0.825686156749725
2.1025641025641 0.834156632423401
2.20512820512821 0.741746604442596
2.30769230769231 0.809187889099121
2.41794871794872 0.77706116437912
2.53333333333333 0.754463970661163
2.65384615384615 0.777244567871094
2.78205128205128 0.808225929737091
2.91282051282051 0.821924209594727
3.05384615384615 0.769788682460785
3.1974358974359 0.730509221553802
3.35128205128205 0.710287272930145
3.51025641025641 0.754907011985779
3.67692307692308 0.645866870880127
3.85128205128205 0.679824709892273
4.03589743589744 0.729696452617645
4.22820512820513 0.643503487110138
4.42820512820513 0.718096852302551
4.64102564102564 0.646313488483429
4.86153846153846 0.663933455944061
5.09230769230769 0.641498863697052
5.33589743589744 0.623622834682465
5.58974358974359 0.605701625347137
5.85641025641026 0.641050636768341
6.13589743589744 0.568261325359344
6.42564102564103 0.465463548898697
6.73333333333333 0.582907557487488
7.05384615384615 0.492966383695602
7.38974358974359 0.485041230916977
7.74102564102564 0.479545027017593
8.11025641025641 0.487361639738083
8.4974358974359 0.469846457242966
8.9 0.429182916879654
9.32564102564103 0.393158972263336
9.76923076923077 0.324559181928635
10.2333333333333 0.366457283496857
10.7205128205128 0.331745952367783
11.2307692307692 0.335001915693283
11.7666666666667 0.357869982719421
12.3282051282051 0.319116503000259
12.9153846153846 0.299164205789566
13.5282051282051 0.272829353809357
14.174358974359 0.211098477244377
14.8487179487179 0.337871581315994
15.5564102564103 0.21628013253212
16.2974358974359 0.207223922014236
17.0717948717949 0.219653114676476
17.8846153846154 0.219717964529991
18.7358974358974 0.240350350737572
19.6282051282051 0.192147731781006
20.5641025641026 0.181930392980576
21.5435897435897 0.211427316069603
22.5692307692308 0.205162152647972
23.6435897435897 0.181810811161995
24.7692307692308 0.151470884680748
25.9487179487179 0.177943721413612
27.1846153846154 0.166482970118523
28.4794871794872 0.166222751140594
29.8358974358974 0.163495108485222
31.2564102564103 0.158467754721642
32.7435897435897 0.162745118141174
34.3025641025641 0.155539974570274
35.9358974358974 0.158219367265701
37.648717948718 0.149359732866287
39.4410256410256 0.151034161448479
41.3179487179487 0.149515345692635
43.2871794871795 0.152691870927811
45.3487179487179 0.150481268763542
47.5076923076923 0.159203767776489
49.7692307692308 0.154682442545891
52.1384615384615 0.157055050134659
54.6205128205128 0.156430557370186
57.2230769230769 0.150397777557373
59.9461538461538 0.161128625273705
62.8025641025641 0.15912076830864
65.7923076923077 0.156661733984947
68.925641025641 0.163768723607063
72.2076923076923 0.154776498675346
75.6461538461539 0.157061859965324
79.2461538461538 0.153998404741287
83.0205128205128 0.159850403666496
86.974358974359 0.156407311558723
91.1153846153846 0.161182299256325
95.4538461538462 0.154676482081413
100 0.151814788579941
};
\addlegendentry{mb 128, exact}
\addplot [, black, opacity=0.6, mark=*, mark size=0.5, mark options={solid}, only marks, forget plot]
table {%
1 0.994340121746063
1.04615384615385 0.9846071600914
1.0974358974359 0.961117684841156
1.14871794871795 0.880270183086395
1.2025641025641 0.858643174171448
1.26153846153846 0.843821942806244
1.32051282051282 0.837051093578339
1.38461538461538 0.816023290157318
1.44871794871795 0.819119274616241
1.51794871794872 0.828187644481659
1.58974358974359 0.808079063892365
1.66666666666667 0.774262189865112
1.74615384615385 0.796475648880005
1.82820512820513 0.76874703168869
1.91538461538462 0.762240171432495
2.00769230769231 0.792746722698212
2.1025641025641 0.77276623249054
2.20512820512821 0.696559131145477
2.30769230769231 0.736120164394379
2.41794871794872 0.780815660953522
2.53333333333333 0.76865541934967
2.65384615384615 0.704696595668793
2.78205128205128 0.720338344573975
2.91282051282051 0.698709309101105
3.05384615384615 0.771089911460876
3.1974358974359 0.678792059421539
3.35128205128205 0.649837136268616
3.51025641025641 0.646670281887054
3.67692307692308 0.635070383548737
3.85128205128205 0.606635987758636
4.03589743589744 0.615378975868225
4.22820512820513 0.565634667873383
4.42820512820513 0.586093127727509
4.64102564102564 0.589771211147308
4.86153846153846 0.559933960437775
5.09230769230769 0.575185239315033
5.33589743589744 0.536265969276428
5.58974358974359 0.495523363351822
5.85641025641026 0.508306205272675
6.13589743589744 0.520278632640839
6.42564102564103 0.434947460889816
6.73333333333333 0.464853763580322
7.05384615384615 0.443529337644577
7.38974358974359 0.461599558591843
7.74102564102564 0.379453837871552
8.11025641025641 0.434795707464218
8.4974358974359 0.417380720376968
8.9 0.446479886770248
9.32564102564103 0.368054538965225
9.76923076923077 0.402794510126114
10.2333333333333 0.38363441824913
10.7205128205128 0.287693023681641
11.2307692307692 0.300360232591629
11.7666666666667 0.273304224014282
12.3282051282051 0.333551704883575
12.9153846153846 0.287632375955582
13.5282051282051 0.272924721240997
14.174358974359 0.223980471491814
14.8487179487179 0.295608103275299
15.5564102564103 0.20181143283844
16.2974358974359 0.187264427542686
17.0717948717949 0.255556106567383
17.8846153846154 0.202711209654808
18.7358974358974 0.213712975382805
19.6282051282051 0.215359315276146
20.5641025641026 0.17779241502285
21.5435897435897 0.176156684756279
22.5692307692308 0.161686912178993
23.6435897435897 0.179302975535393
24.7692307692308 0.156765133142471
25.9487179487179 0.167811349034309
27.1846153846154 0.184907600283623
28.4794871794872 0.13999992609024
29.8358974358974 0.153640702366829
31.2564102564103 0.172931119799614
32.7435897435897 0.170961573719978
34.3025641025641 0.173879787325859
35.9358974358974 0.173524871468544
37.648717948718 0.164736077189445
39.4410256410256 0.169582083821297
41.3179487179487 0.167429536581039
43.2871794871795 0.166251912713051
45.3487179487179 0.171472921967506
47.5076923076923 0.166811868548393
49.7692307692308 0.169324770569801
52.1384615384615 0.169742107391357
54.6205128205128 0.170091733336449
57.2230769230769 0.167961075901985
59.9461538461538 0.17472617328167
62.8025641025641 0.162149146199226
65.7923076923077 0.177537754178047
68.925641025641 0.17832787334919
72.2076923076923 0.169299483299255
75.6461538461539 0.170478478074074
79.2461538461538 0.181363344192505
83.0205128205128 0.172071397304535
86.974358974359 0.176639840006828
91.1153846153846 0.182558760046959
95.4538461538462 0.185364082455635
100 0.186870440840721
};
\addplot [, black, opacity=0.6, mark=*, mark size=0.5, mark options={solid}, only marks, forget plot]
table {%
1 0.995283544063568
1.04615384615385 0.97786283493042
1.0974358974359 0.968671143054962
1.14871794871795 0.938183307647705
1.2025641025641 0.840490758419037
1.26153846153846 0.863937199115753
1.32051282051282 0.877496540546417
1.38461538461538 0.888198018074036
1.44871794871795 0.896355092525482
1.51794871794872 0.845336735248566
1.58974358974359 0.816052913665771
1.66666666666667 0.843279838562012
1.74615384615385 0.866602540016174
1.82820512820513 0.813989162445068
1.91538461538462 0.795453488826752
2.00769230769231 0.812335789203644
2.1025641025641 0.845061302185059
2.20512820512821 0.796937644481659
2.30769230769231 0.781130492687225
2.41794871794872 0.832133948802948
2.53333333333333 0.765977799892426
2.65384615384615 0.740174949169159
2.78205128205128 0.761707782745361
2.91282051282051 0.790108382701874
3.05384615384615 0.709866881370544
3.1974358974359 0.738801777362823
3.35128205128205 0.700328409671783
3.51025641025641 0.697395980358124
3.67692307692308 0.646369516849518
3.85128205128205 0.712675750255585
4.03589743589744 0.657976448535919
4.22820512820513 0.618278920650482
4.42820512820513 0.724980652332306
4.64102564102564 0.632344245910645
4.86153846153846 0.548663258552551
5.09230769230769 0.61760002374649
5.33589743589744 0.649069130420685
5.58974358974359 0.562905728816986
5.85641025641026 0.647084593772888
6.13589743589744 0.599597871303558
6.42564102564103 0.553373336791992
6.73333333333333 0.577198684215546
7.05384615384615 0.513204395771027
7.38974358974359 0.520675301551819
7.74102564102564 0.498397439718246
8.11025641025641 0.531229197978973
8.4974358974359 0.45010033249855
8.9 0.46335506439209
9.32564102564103 0.461040705442429
9.76923076923077 0.452410906553268
10.2333333333333 0.48134633898735
10.7205128205128 0.416533797979355
11.2307692307692 0.383014917373657
11.7666666666667 0.4499531686306
12.3282051282051 0.368944108486176
12.9153846153846 0.248056203126907
13.5282051282051 0.388906687498093
14.174358974359 0.27390855550766
14.8487179487179 0.349796384572983
15.5564102564103 0.29123517870903
16.2974358974359 0.273118168115616
17.0717948717949 0.274765461683273
17.8846153846154 0.334249973297119
18.7358974358974 0.258772313594818
19.6282051282051 0.29042261838913
20.5641025641026 0.234134063124657
21.5435897435897 0.220668867230415
22.5692307692308 0.240460395812988
23.6435897435897 0.217838242650032
24.7692307692308 0.199632585048676
25.9487179487179 0.250699013471603
27.1846153846154 0.22636666893959
28.4794871794872 0.234674692153931
29.8358974358974 0.23102231323719
31.2564102564103 0.246151849627495
32.7435897435897 0.243723392486572
34.3025641025641 0.248331293463707
35.9358974358974 0.247135922312737
37.648717948718 0.243963196873665
39.4410256410256 0.241786867380142
41.3179487179487 0.255196064710617
43.2871794871795 0.252404510974884
45.3487179487179 0.249061897397041
47.5076923076923 0.248514130711555
49.7692307692308 0.252460092306137
52.1384615384615 0.245044574141502
54.6205128205128 0.255158364772797
57.2230769230769 0.249576166272163
59.9461538461538 0.25620111823082
62.8025641025641 0.260589182376862
65.7923076923077 0.255043596029282
68.925641025641 0.246378093957901
72.2076923076923 0.254867076873779
75.6461538461539 0.258190810680389
79.2461538461538 0.241877987980843
83.0205128205128 0.243310138583183
86.974358974359 0.252282232046127
91.1153846153846 0.24267315864563
95.4538461538462 0.252032667398453
100 0.24825993180275
};
\addplot [, black, opacity=0.6, mark=*, mark size=0.5, mark options={solid}, only marks, forget plot]
table {%
1 0.989522635936737
1.04615384615385 0.978588998317719
1.0974358974359 0.951145589351654
1.14871794871795 0.937792003154755
1.2025641025641 0.824235558509827
1.26153846153846 0.833244502544403
1.32051282051282 0.804740369319916
1.38461538461538 0.90880674123764
1.44871794871795 0.828184545040131
1.51794871794872 0.8201904296875
1.58974358974359 0.806869685649872
1.66666666666667 0.729628086090088
1.74615384615385 0.795067250728607
1.82820512820513 0.726777374744415
1.91538461538462 0.698961675167084
2.00769230769231 0.746545135974884
2.1025641025641 0.750320851802826
2.20512820512821 0.707176148891449
2.30769230769231 0.71510124206543
2.41794871794872 0.71117240190506
2.53333333333333 0.696636855602264
2.65384615384615 0.690979599952698
2.78205128205128 0.659699618816376
2.91282051282051 0.713985204696655
3.05384615384615 0.702777028083801
3.1974358974359 0.702119112014771
3.35128205128205 0.719510018825531
3.51025641025641 0.686610400676727
3.67692307692308 0.613040328025818
3.85128205128205 0.644203186035156
4.03589743589744 0.612169444561005
4.22820512820513 0.635552108287811
4.42820512820513 0.651890516281128
4.64102564102564 0.577337622642517
4.86153846153846 0.612019419670105
5.09230769230769 0.587266862392426
5.33589743589744 0.598931431770325
5.58974358974359 0.589041769504547
5.85641025641026 0.6244837641716
6.13589743589744 0.64569628238678
6.42564102564103 0.492092132568359
6.73333333333333 0.559301495552063
7.05384615384615 0.48268935084343
7.38974358974359 0.499605178833008
7.74102564102564 0.452725142240524
8.11025641025641 0.431591659784317
8.4974358974359 0.400140911340714
8.9 0.460161030292511
9.32564102564103 0.470592975616455
9.76923076923077 0.457668155431747
10.2333333333333 0.441314458847046
10.7205128205128 0.425982862710953
11.2307692307692 0.359665930271149
11.7666666666667 0.37989416718483
12.3282051282051 0.305234730243683
12.9153846153846 0.329457938671112
13.5282051282051 0.381025075912476
14.174358974359 0.328660279512405
14.8487179487179 0.298563003540039
15.5564102564103 0.273264318704605
16.2974358974359 0.204250976443291
17.0717948717949 0.283313870429993
17.8846153846154 0.278328269720078
18.7358974358974 0.219872832298279
19.6282051282051 0.207532078027725
20.5641025641026 0.20194636285305
21.5435897435897 0.202694699168205
22.5692307692308 0.22084866464138
23.6435897435897 0.199643597006798
24.7692307692308 0.191859245300293
25.9487179487179 0.238249495625496
27.1846153846154 0.192389577627182
28.4794871794872 0.233926609158516
29.8358974358974 0.240598991513252
31.2564102564103 0.222840264439583
32.7435897435897 0.211264804005623
34.3025641025641 0.224346950650215
35.9358974358974 0.222244456410408
37.648717948718 0.22749562561512
39.4410256410256 0.217957451939583
41.3179487179487 0.217454582452774
43.2871794871795 0.228226110339165
45.3487179487179 0.218844994902611
47.5076923076923 0.216697797179222
49.7692307692308 0.22492328286171
52.1384615384615 0.227226331830025
54.6205128205128 0.236445888876915
57.2230769230769 0.222115233540535
59.9461538461538 0.223236322402954
62.8025641025641 0.22771954536438
65.7923076923077 0.230958417057991
68.925641025641 0.210628420114517
72.2076923076923 0.226284772157669
75.6461538461539 0.223151594400406
79.2461538461538 0.230750843882561
83.0205128205128 0.229029461741447
86.974358974359 0.229126363992691
91.1153846153846 0.218544393777847
95.4538461538462 0.228257089853287
100 0.231624409556389
};
\addplot [, black, opacity=0.6, mark=*, mark size=0.5, mark options={solid}, only marks, forget plot]
table {%
1 0.9946249127388
1.04615384615385 0.98387622833252
1.0974358974359 0.958846390247345
1.14871794871795 0.868250846862793
1.2025641025641 0.904783070087433
1.26153846153846 0.878515899181366
1.32051282051282 0.83079594373703
1.38461538461538 0.820999324321747
1.44871794871795 0.822868347167969
1.51794871794872 0.848405778408051
1.58974358974359 0.848834216594696
1.66666666666667 0.853343963623047
1.74615384615385 0.892821311950684
1.82820512820513 0.798114776611328
1.91538461538462 0.809952735900879
2.00769230769231 0.789812326431274
2.1025641025641 0.733191013336182
2.20512820512821 0.73733252286911
2.30769230769231 0.734106719493866
2.41794871794872 0.808441579341888
2.53333333333333 0.751391053199768
2.65384615384615 0.705909073352814
2.78205128205128 0.684674382209778
2.91282051282051 0.729891717433929
3.05384615384615 0.781607627868652
3.1974358974359 0.693151891231537
3.35128205128205 0.669645607471466
3.51025641025641 0.724934041500092
3.67692307692308 0.673100709915161
3.85128205128205 0.685150504112244
4.03589743589744 0.629621684551239
4.22820512820513 0.640624940395355
4.42820512820513 0.662707030773163
4.64102564102564 0.632343590259552
4.86153846153846 0.675061702728271
5.09230769230769 0.614578425884247
5.33589743589744 0.586445271968842
5.58974358974359 0.566637516021729
5.85641025641026 0.598741710186005
6.13589743589744 0.592288970947266
6.42564102564103 0.578337252140045
6.73333333333333 0.573352515697479
7.05384615384615 0.539848744869232
7.38974358974359 0.566887974739075
7.74102564102564 0.511598587036133
8.11025641025641 0.540198981761932
8.4974358974359 0.492360502481461
8.9 0.513024687767029
9.32564102564103 0.438485682010651
9.76923076923077 0.463172763586044
10.2333333333333 0.448384284973145
10.7205128205128 0.412151902914047
11.2307692307692 0.430864870548248
11.7666666666667 0.403329193592072
12.3282051282051 0.26922282576561
12.9153846153846 0.35053226351738
13.5282051282051 0.349812805652618
14.174358974359 0.307074159383774
14.8487179487179 0.31959941983223
15.5564102564103 0.245153948664665
16.2974358974359 0.232444286346436
17.0717948717949 0.262086480855942
17.8846153846154 0.238577768206596
18.7358974358974 0.222327932715416
19.6282051282051 0.210955336689949
20.5641025641026 0.2162756472826
21.5435897435897 0.214320048689842
22.5692307692308 0.201260715723038
23.6435897435897 0.167045146226883
24.7692307692308 0.171784326434135
25.9487179487179 0.184942781925201
27.1846153846154 0.170699194073677
28.4794871794872 0.183226272463799
29.8358974358974 0.196478009223938
31.2564102564103 0.199185237288475
32.7435897435897 0.202754661440849
34.3025641025641 0.207423999905586
35.9358974358974 0.202543377876282
37.648717948718 0.199705168604851
39.4410256410256 0.193694368004799
41.3179487179487 0.198338225483894
43.2871794871795 0.188535913825035
45.3487179487179 0.191093489527702
47.5076923076923 0.190952986478806
49.7692307692308 0.195279598236084
52.1384615384615 0.203201323747635
54.6205128205128 0.190620705485344
57.2230769230769 0.188365027308464
59.9461538461538 0.197189748287201
62.8025641025641 0.191405817866325
65.7923076923077 0.197903588414192
68.925641025641 0.185508459806442
72.2076923076923 0.19719360768795
75.6461538461539 0.195181638002396
79.2461538461538 0.202272057533264
83.0205128205128 0.19380310177803
86.974358974359 0.196492001414299
91.1153846153846 0.191515639424324
95.4538461538462 0.194435328245163
100 0.1979900598526
};
\end{axis}

\end{tikzpicture}

      \tikzexternaldisable
    \end{minipage}\hfill
    \begin{minipage}{0.50\linewidth}
      \centering
      % defines the pgfplots style "eigspacedefault"
\pgfkeys{/pgfplots/eigspacedefault/.style={
    width=1.0\linewidth,
    height=0.6\linewidth,
    every axis plot/.append style={line width = 1.5pt},
    tick pos = left,
    ylabel near ticks,
    xlabel near ticks,
    xtick align = inside,
    ytick align = inside,
    legend cell align = left,
    legend columns = 4,
    legend pos = south east,
    legend style = {
      fill opacity = 1,
      text opacity = 1,
      font = \footnotesize,
      at={(1, 1.025)},
      anchor=south east,
      column sep=0.25cm,
    },
    legend image post style={scale=2.5},
    xticklabel style = {font = \footnotesize},
    xlabel style = {font = \footnotesize},
    axis line style = {black},
    yticklabel style = {font = \footnotesize},
    ylabel style = {font = \footnotesize},
    title style = {font = \footnotesize},
    grid = major,
    grid style = {dashed}
  }
}

\pgfkeys{/pgfplots/eigspacedefaultapp/.style={
    eigspacedefault,
    height=0.6\linewidth,
    legend columns = 2,
  }
}

\pgfkeys{/pgfplots/eigspacenolegend/.style={
    legend image post style = {scale=0},
    legend style = {
      fill opacity = 0,
      draw opacity = 0,
      text opacity = 0,
      font = \footnotesize,
      at={(1, 1.025)},
      anchor=south east,
      column sep=0.25cm,
    },
  }
}
%%% Local Variables:
%%% mode: latex
%%% TeX-master: "../../thesis"
%%% End:

      \pgfkeys{/pgfplots/zmystyle/.style={
          eigspacedefaultapp,
        }}
      \tikzexternalenable
      % This file was created by tikzplotlib v0.9.7.
\begin{tikzpicture}

\definecolor{color0}{rgb}{0.274509803921569,0.6,0.564705882352941}
\definecolor{color1}{rgb}{0.870588235294118,0.623529411764706,0.0862745098039216}
\definecolor{color2}{rgb}{0.501960784313725,0.184313725490196,0.6}

\begin{axis}[
axis line style={white!10!black},
legend columns=2,
legend style={fill opacity=0.8, draw opacity=1, text opacity=1, draw=white!80!black},
log basis x={10},
tick pos=left,
xlabel={epoch (log scale)},
xmajorgrids,
xmin=0.794328234724281, xmax=125.892541179417,
xmode=log,
ylabel={overlap},
ymajorgrids,
ymin=-0.05, ymax=1.05,
zmystyle
]
\addplot [, white!10!black, dashed, forget plot]
table {%
0.794328234724281 1
125.892541179417 1
};
\addplot [, white!10!black, dashed, forget plot]
table {%
0.794328234724281 0
125.892541179417 0
};
\addplot [, black, opacity=0.6, mark=*, mark size=0.5, mark options={solid}, only marks]
table {%
1 0.994974613189697
1.04615384615385 0.978533208370209
1.0974358974359 0.88761568069458
1.14871794871795 0.882425129413605
1.2025641025641 0.912472188472748
1.26153846153846 0.891978442668915
1.32051282051282 0.833248555660248
1.38461538461538 0.843169629573822
1.44871794871795 0.887269198894501
1.51794871794872 0.808582723140717
1.58974358974359 0.806749761104584
1.66666666666667 0.831486701965332
1.74615384615385 0.871429443359375
1.82820512820513 0.798410594463348
1.91538461538462 0.858729362487793
2.00769230769231 0.825686156749725
2.1025641025641 0.834156632423401
2.20512820512821 0.741746604442596
2.30769230769231 0.809187889099121
2.41794871794872 0.77706116437912
2.53333333333333 0.754463970661163
2.65384615384615 0.777244567871094
2.78205128205128 0.808225929737091
2.91282051282051 0.821924209594727
3.05384615384615 0.769788682460785
3.1974358974359 0.730509221553802
3.35128205128205 0.710287272930145
3.51025641025641 0.754907011985779
3.67692307692308 0.645866870880127
3.85128205128205 0.679824709892273
4.03589743589744 0.729696452617645
4.22820512820513 0.643503487110138
4.42820512820513 0.718096852302551
4.64102564102564 0.646313488483429
4.86153846153846 0.663933455944061
5.09230769230769 0.641498863697052
5.33589743589744 0.623622834682465
5.58974358974359 0.605701625347137
5.85641025641026 0.641050636768341
6.13589743589744 0.568261325359344
6.42564102564103 0.465463548898697
6.73333333333333 0.582907557487488
7.05384615384615 0.492966383695602
7.38974358974359 0.485041230916977
7.74102564102564 0.479545027017593
8.11025641025641 0.487361639738083
8.4974358974359 0.469846457242966
8.9 0.429182916879654
9.32564102564103 0.393158972263336
9.76923076923077 0.324559181928635
10.2333333333333 0.366457283496857
10.7205128205128 0.331745952367783
11.2307692307692 0.335001915693283
11.7666666666667 0.357869982719421
12.3282051282051 0.319116503000259
12.9153846153846 0.299164205789566
13.5282051282051 0.272829353809357
14.174358974359 0.211098477244377
14.8487179487179 0.337871581315994
15.5564102564103 0.21628013253212
16.2974358974359 0.207223922014236
17.0717948717949 0.219653114676476
17.8846153846154 0.219717964529991
18.7358974358974 0.240350350737572
19.6282051282051 0.192147731781006
20.5641025641026 0.181930392980576
21.5435897435897 0.211427316069603
22.5692307692308 0.205162152647972
23.6435897435897 0.181810811161995
24.7692307692308 0.151470884680748
25.9487179487179 0.177943721413612
27.1846153846154 0.166482970118523
28.4794871794872 0.166222751140594
29.8358974358974 0.163495108485222
31.2564102564103 0.158467754721642
32.7435897435897 0.162745118141174
34.3025641025641 0.155539974570274
35.9358974358974 0.158219367265701
37.648717948718 0.149359732866287
39.4410256410256 0.151034161448479
41.3179487179487 0.149515345692635
43.2871794871795 0.152691870927811
45.3487179487179 0.150481268763542
47.5076923076923 0.159203767776489
49.7692307692308 0.154682442545891
52.1384615384615 0.157055050134659
54.6205128205128 0.156430557370186
57.2230769230769 0.150397777557373
59.9461538461538 0.161128625273705
62.8025641025641 0.15912076830864
65.7923076923077 0.156661733984947
68.925641025641 0.163768723607063
72.2076923076923 0.154776498675346
75.6461538461539 0.157061859965324
79.2461538461538 0.153998404741287
83.0205128205128 0.159850403666496
86.974358974359 0.156407311558723
91.1153846153846 0.161182299256325
95.4538461538462 0.154676482081413
100 0.151814788579941
};
\addlegendentry{mb 128, exact}
\addplot [, black, opacity=0.6, mark=*, mark size=0.5, mark options={solid}, only marks, forget plot]
table {%
1 0.994340121746063
1.04615384615385 0.9846071600914
1.0974358974359 0.961117684841156
1.14871794871795 0.880270183086395
1.2025641025641 0.858643174171448
1.26153846153846 0.843821942806244
1.32051282051282 0.837051093578339
1.38461538461538 0.816023290157318
1.44871794871795 0.819119274616241
1.51794871794872 0.828187644481659
1.58974358974359 0.808079063892365
1.66666666666667 0.774262189865112
1.74615384615385 0.796475648880005
1.82820512820513 0.76874703168869
1.91538461538462 0.762240171432495
2.00769230769231 0.792746722698212
2.1025641025641 0.77276623249054
2.20512820512821 0.696559131145477
2.30769230769231 0.736120164394379
2.41794871794872 0.780815660953522
2.53333333333333 0.76865541934967
2.65384615384615 0.704696595668793
2.78205128205128 0.720338344573975
2.91282051282051 0.698709309101105
3.05384615384615 0.771089911460876
3.1974358974359 0.678792059421539
3.35128205128205 0.649837136268616
3.51025641025641 0.646670281887054
3.67692307692308 0.635070383548737
3.85128205128205 0.606635987758636
4.03589743589744 0.615378975868225
4.22820512820513 0.565634667873383
4.42820512820513 0.586093127727509
4.64102564102564 0.589771211147308
4.86153846153846 0.559933960437775
5.09230769230769 0.575185239315033
5.33589743589744 0.536265969276428
5.58974358974359 0.495523363351822
5.85641025641026 0.508306205272675
6.13589743589744 0.520278632640839
6.42564102564103 0.434947460889816
6.73333333333333 0.464853763580322
7.05384615384615 0.443529337644577
7.38974358974359 0.461599558591843
7.74102564102564 0.379453837871552
8.11025641025641 0.434795707464218
8.4974358974359 0.417380720376968
8.9 0.446479886770248
9.32564102564103 0.368054538965225
9.76923076923077 0.402794510126114
10.2333333333333 0.38363441824913
10.7205128205128 0.287693023681641
11.2307692307692 0.300360232591629
11.7666666666667 0.273304224014282
12.3282051282051 0.333551704883575
12.9153846153846 0.287632375955582
13.5282051282051 0.272924721240997
14.174358974359 0.223980471491814
14.8487179487179 0.295608103275299
15.5564102564103 0.20181143283844
16.2974358974359 0.187264427542686
17.0717948717949 0.255556106567383
17.8846153846154 0.202711209654808
18.7358974358974 0.213712975382805
19.6282051282051 0.215359315276146
20.5641025641026 0.17779241502285
21.5435897435897 0.176156684756279
22.5692307692308 0.161686912178993
23.6435897435897 0.179302975535393
24.7692307692308 0.156765133142471
25.9487179487179 0.167811349034309
27.1846153846154 0.184907600283623
28.4794871794872 0.13999992609024
29.8358974358974 0.153640702366829
31.2564102564103 0.172931119799614
32.7435897435897 0.170961573719978
34.3025641025641 0.173879787325859
35.9358974358974 0.173524871468544
37.648717948718 0.164736077189445
39.4410256410256 0.169582083821297
41.3179487179487 0.167429536581039
43.2871794871795 0.166251912713051
45.3487179487179 0.171472921967506
47.5076923076923 0.166811868548393
49.7692307692308 0.169324770569801
52.1384615384615 0.169742107391357
54.6205128205128 0.170091733336449
57.2230769230769 0.167961075901985
59.9461538461538 0.17472617328167
62.8025641025641 0.162149146199226
65.7923076923077 0.177537754178047
68.925641025641 0.17832787334919
72.2076923076923 0.169299483299255
75.6461538461539 0.170478478074074
79.2461538461538 0.181363344192505
83.0205128205128 0.172071397304535
86.974358974359 0.176639840006828
91.1153846153846 0.182558760046959
95.4538461538462 0.185364082455635
100 0.186870440840721
};
\addplot [, black, opacity=0.6, mark=*, mark size=0.5, mark options={solid}, only marks, forget plot]
table {%
1 0.995283544063568
1.04615384615385 0.97786283493042
1.0974358974359 0.968671143054962
1.14871794871795 0.938183307647705
1.2025641025641 0.840490758419037
1.26153846153846 0.863937199115753
1.32051282051282 0.877496540546417
1.38461538461538 0.888198018074036
1.44871794871795 0.896355092525482
1.51794871794872 0.845336735248566
1.58974358974359 0.816052913665771
1.66666666666667 0.843279838562012
1.74615384615385 0.866602540016174
1.82820512820513 0.813989162445068
1.91538461538462 0.795453488826752
2.00769230769231 0.812335789203644
2.1025641025641 0.845061302185059
2.20512820512821 0.796937644481659
2.30769230769231 0.781130492687225
2.41794871794872 0.832133948802948
2.53333333333333 0.765977799892426
2.65384615384615 0.740174949169159
2.78205128205128 0.761707782745361
2.91282051282051 0.790108382701874
3.05384615384615 0.709866881370544
3.1974358974359 0.738801777362823
3.35128205128205 0.700328409671783
3.51025641025641 0.697395980358124
3.67692307692308 0.646369516849518
3.85128205128205 0.712675750255585
4.03589743589744 0.657976448535919
4.22820512820513 0.618278920650482
4.42820512820513 0.724980652332306
4.64102564102564 0.632344245910645
4.86153846153846 0.548663258552551
5.09230769230769 0.61760002374649
5.33589743589744 0.649069130420685
5.58974358974359 0.562905728816986
5.85641025641026 0.647084593772888
6.13589743589744 0.599597871303558
6.42564102564103 0.553373336791992
6.73333333333333 0.577198684215546
7.05384615384615 0.513204395771027
7.38974358974359 0.520675301551819
7.74102564102564 0.498397439718246
8.11025641025641 0.531229197978973
8.4974358974359 0.45010033249855
8.9 0.46335506439209
9.32564102564103 0.461040705442429
9.76923076923077 0.452410906553268
10.2333333333333 0.48134633898735
10.7205128205128 0.416533797979355
11.2307692307692 0.383014917373657
11.7666666666667 0.4499531686306
12.3282051282051 0.368944108486176
12.9153846153846 0.248056203126907
13.5282051282051 0.388906687498093
14.174358974359 0.27390855550766
14.8487179487179 0.349796384572983
15.5564102564103 0.29123517870903
16.2974358974359 0.273118168115616
17.0717948717949 0.274765461683273
17.8846153846154 0.334249973297119
18.7358974358974 0.258772313594818
19.6282051282051 0.29042261838913
20.5641025641026 0.234134063124657
21.5435897435897 0.220668867230415
22.5692307692308 0.240460395812988
23.6435897435897 0.217838242650032
24.7692307692308 0.199632585048676
25.9487179487179 0.250699013471603
27.1846153846154 0.22636666893959
28.4794871794872 0.234674692153931
29.8358974358974 0.23102231323719
31.2564102564103 0.246151849627495
32.7435897435897 0.243723392486572
34.3025641025641 0.248331293463707
35.9358974358974 0.247135922312737
37.648717948718 0.243963196873665
39.4410256410256 0.241786867380142
41.3179487179487 0.255196064710617
43.2871794871795 0.252404510974884
45.3487179487179 0.249061897397041
47.5076923076923 0.248514130711555
49.7692307692308 0.252460092306137
52.1384615384615 0.245044574141502
54.6205128205128 0.255158364772797
57.2230769230769 0.249576166272163
59.9461538461538 0.25620111823082
62.8025641025641 0.260589182376862
65.7923076923077 0.255043596029282
68.925641025641 0.246378093957901
72.2076923076923 0.254867076873779
75.6461538461539 0.258190810680389
79.2461538461538 0.241877987980843
83.0205128205128 0.243310138583183
86.974358974359 0.252282232046127
91.1153846153846 0.24267315864563
95.4538461538462 0.252032667398453
100 0.24825993180275
};
\addplot [, black, opacity=0.6, mark=*, mark size=0.5, mark options={solid}, only marks, forget plot]
table {%
1 0.989522635936737
1.04615384615385 0.978588998317719
1.0974358974359 0.951145589351654
1.14871794871795 0.937792003154755
1.2025641025641 0.824235558509827
1.26153846153846 0.833244502544403
1.32051282051282 0.804740369319916
1.38461538461538 0.90880674123764
1.44871794871795 0.828184545040131
1.51794871794872 0.8201904296875
1.58974358974359 0.806869685649872
1.66666666666667 0.729628086090088
1.74615384615385 0.795067250728607
1.82820512820513 0.726777374744415
1.91538461538462 0.698961675167084
2.00769230769231 0.746545135974884
2.1025641025641 0.750320851802826
2.20512820512821 0.707176148891449
2.30769230769231 0.71510124206543
2.41794871794872 0.71117240190506
2.53333333333333 0.696636855602264
2.65384615384615 0.690979599952698
2.78205128205128 0.659699618816376
2.91282051282051 0.713985204696655
3.05384615384615 0.702777028083801
3.1974358974359 0.702119112014771
3.35128205128205 0.719510018825531
3.51025641025641 0.686610400676727
3.67692307692308 0.613040328025818
3.85128205128205 0.644203186035156
4.03589743589744 0.612169444561005
4.22820512820513 0.635552108287811
4.42820512820513 0.651890516281128
4.64102564102564 0.577337622642517
4.86153846153846 0.612019419670105
5.09230769230769 0.587266862392426
5.33589743589744 0.598931431770325
5.58974358974359 0.589041769504547
5.85641025641026 0.6244837641716
6.13589743589744 0.64569628238678
6.42564102564103 0.492092132568359
6.73333333333333 0.559301495552063
7.05384615384615 0.48268935084343
7.38974358974359 0.499605178833008
7.74102564102564 0.452725142240524
8.11025641025641 0.431591659784317
8.4974358974359 0.400140911340714
8.9 0.460161030292511
9.32564102564103 0.470592975616455
9.76923076923077 0.457668155431747
10.2333333333333 0.441314458847046
10.7205128205128 0.425982862710953
11.2307692307692 0.359665930271149
11.7666666666667 0.37989416718483
12.3282051282051 0.305234730243683
12.9153846153846 0.329457938671112
13.5282051282051 0.381025075912476
14.174358974359 0.328660279512405
14.8487179487179 0.298563003540039
15.5564102564103 0.273264318704605
16.2974358974359 0.204250976443291
17.0717948717949 0.283313870429993
17.8846153846154 0.278328269720078
18.7358974358974 0.219872832298279
19.6282051282051 0.207532078027725
20.5641025641026 0.20194636285305
21.5435897435897 0.202694699168205
22.5692307692308 0.22084866464138
23.6435897435897 0.199643597006798
24.7692307692308 0.191859245300293
25.9487179487179 0.238249495625496
27.1846153846154 0.192389577627182
28.4794871794872 0.233926609158516
29.8358974358974 0.240598991513252
31.2564102564103 0.222840264439583
32.7435897435897 0.211264804005623
34.3025641025641 0.224346950650215
35.9358974358974 0.222244456410408
37.648717948718 0.22749562561512
39.4410256410256 0.217957451939583
41.3179487179487 0.217454582452774
43.2871794871795 0.228226110339165
45.3487179487179 0.218844994902611
47.5076923076923 0.216697797179222
49.7692307692308 0.22492328286171
52.1384615384615 0.227226331830025
54.6205128205128 0.236445888876915
57.2230769230769 0.222115233540535
59.9461538461538 0.223236322402954
62.8025641025641 0.22771954536438
65.7923076923077 0.230958417057991
68.925641025641 0.210628420114517
72.2076923076923 0.226284772157669
75.6461538461539 0.223151594400406
79.2461538461538 0.230750843882561
83.0205128205128 0.229029461741447
86.974358974359 0.229126363992691
91.1153846153846 0.218544393777847
95.4538461538462 0.228257089853287
100 0.231624409556389
};
\addplot [, black, opacity=0.6, mark=*, mark size=0.5, mark options={solid}, only marks, forget plot]
table {%
1 0.9946249127388
1.04615384615385 0.98387622833252
1.0974358974359 0.958846390247345
1.14871794871795 0.868250846862793
1.2025641025641 0.904783070087433
1.26153846153846 0.878515899181366
1.32051282051282 0.83079594373703
1.38461538461538 0.820999324321747
1.44871794871795 0.822868347167969
1.51794871794872 0.848405778408051
1.58974358974359 0.848834216594696
1.66666666666667 0.853343963623047
1.74615384615385 0.892821311950684
1.82820512820513 0.798114776611328
1.91538461538462 0.809952735900879
2.00769230769231 0.789812326431274
2.1025641025641 0.733191013336182
2.20512820512821 0.73733252286911
2.30769230769231 0.734106719493866
2.41794871794872 0.808441579341888
2.53333333333333 0.751391053199768
2.65384615384615 0.705909073352814
2.78205128205128 0.684674382209778
2.91282051282051 0.729891717433929
3.05384615384615 0.781607627868652
3.1974358974359 0.693151891231537
3.35128205128205 0.669645607471466
3.51025641025641 0.724934041500092
3.67692307692308 0.673100709915161
3.85128205128205 0.685150504112244
4.03589743589744 0.629621684551239
4.22820512820513 0.640624940395355
4.42820512820513 0.662707030773163
4.64102564102564 0.632343590259552
4.86153846153846 0.675061702728271
5.09230769230769 0.614578425884247
5.33589743589744 0.586445271968842
5.58974358974359 0.566637516021729
5.85641025641026 0.598741710186005
6.13589743589744 0.592288970947266
6.42564102564103 0.578337252140045
6.73333333333333 0.573352515697479
7.05384615384615 0.539848744869232
7.38974358974359 0.566887974739075
7.74102564102564 0.511598587036133
8.11025641025641 0.540198981761932
8.4974358974359 0.492360502481461
8.9 0.513024687767029
9.32564102564103 0.438485682010651
9.76923076923077 0.463172763586044
10.2333333333333 0.448384284973145
10.7205128205128 0.412151902914047
11.2307692307692 0.430864870548248
11.7666666666667 0.403329193592072
12.3282051282051 0.26922282576561
12.9153846153846 0.35053226351738
13.5282051282051 0.349812805652618
14.174358974359 0.307074159383774
14.8487179487179 0.31959941983223
15.5564102564103 0.245153948664665
16.2974358974359 0.232444286346436
17.0717948717949 0.262086480855942
17.8846153846154 0.238577768206596
18.7358974358974 0.222327932715416
19.6282051282051 0.210955336689949
20.5641025641026 0.2162756472826
21.5435897435897 0.214320048689842
22.5692307692308 0.201260715723038
23.6435897435897 0.167045146226883
24.7692307692308 0.171784326434135
25.9487179487179 0.184942781925201
27.1846153846154 0.170699194073677
28.4794871794872 0.183226272463799
29.8358974358974 0.196478009223938
31.2564102564103 0.199185237288475
32.7435897435897 0.202754661440849
34.3025641025641 0.207423999905586
35.9358974358974 0.202543377876282
37.648717948718 0.199705168604851
39.4410256410256 0.193694368004799
41.3179487179487 0.198338225483894
43.2871794871795 0.188535913825035
45.3487179487179 0.191093489527702
47.5076923076923 0.190952986478806
49.7692307692308 0.195279598236084
52.1384615384615 0.203201323747635
54.6205128205128 0.190620705485344
57.2230769230769 0.188365027308464
59.9461538461538 0.197189748287201
62.8025641025641 0.191405817866325
65.7923076923077 0.197903588414192
68.925641025641 0.185508459806442
72.2076923076923 0.19719360768795
75.6461538461539 0.195181638002396
79.2461538461538 0.202272057533264
83.0205128205128 0.19380310177803
86.974358974359 0.196492001414299
91.1153846153846 0.191515639424324
95.4538461538462 0.194435328245163
100 0.1979900598526
};
\addplot [, color0, opacity=0.6, mark=diamond*, mark size=0.5, mark options={solid}, only marks]
table {%
1 0.89774763584137
1.04615384615385 0.811991989612579
1.0974358974359 0.697114884853363
1.14871794871795 0.764490783214569
1.2025641025641 0.613322734832764
1.26153846153846 0.630103051662445
1.32051282051282 0.597980201244354
1.38461538461538 0.502379357814789
1.44871794871795 0.520622074604034
1.51794871794872 0.431868761777878
1.58974358974359 0.477500975131989
1.66666666666667 0.47976142168045
1.74615384615385 0.538797676563263
1.82820512820513 0.437455087900162
1.91538461538462 0.441238641738892
2.00769230769231 0.461073011159897
2.1025641025641 0.450649082660675
2.20512820512821 0.43111926317215
2.30769230769231 0.400425732135773
2.41794871794872 0.454840749502182
2.53333333333333 0.43067142367363
2.65384615384615 0.383395999670029
2.78205128205128 0.382495671510696
2.91282051282051 0.375303685665131
3.05384615384615 0.402107536792755
3.1974358974359 0.35690575838089
3.35128205128205 0.375647574663162
3.51025641025641 0.353715091943741
3.67692307692308 0.337284177541733
3.85128205128205 0.339240819215775
4.03589743589744 0.315695703029633
4.22820512820513 0.297926753759384
4.42820512820513 0.275798261165619
4.64102564102564 0.331105023622513
4.86153846153846 0.289203375577927
5.09230769230769 0.275263905525208
5.33589743589744 0.281766384840012
5.58974358974359 0.262508809566498
5.85641025641026 0.249967440962791
6.13589743589744 0.278786182403564
6.42564102564103 0.243985012173653
6.73333333333333 0.264500379562378
7.05384615384615 0.224218606948853
7.38974358974359 0.244120359420776
7.74102564102564 0.232223317027092
8.11025641025641 0.165382534265518
8.4974358974359 0.225899800658226
8.9 0.216854438185692
9.32564102564103 0.219742879271507
9.76923076923077 0.188700720667839
10.2333333333333 0.219812631607056
10.7205128205128 0.18480621278286
11.2307692307692 0.187444880604744
11.7666666666667 0.209047719836235
12.3282051282051 0.200455456972122
12.9153846153846 0.164004877209663
13.5282051282051 0.158666044473648
14.174358974359 0.152521297335625
14.8487179487179 0.18116469681263
15.5564102564103 0.139524579048157
16.2974358974359 0.143905326724052
17.0717948717949 0.130216836929321
17.8846153846154 0.131738498806953
18.7358974358974 0.141731962561607
19.6282051282051 0.122466802597046
20.5641025641026 0.122289933264256
21.5435897435897 0.133704096078873
22.5692307692308 0.14700098335743
23.6435897435897 0.135767325758934
24.7692307692308 0.129570707678795
25.9487179487179 0.123587660491467
27.1846153846154 0.110138893127441
28.4794871794872 0.131294801831245
29.8358974358974 0.140054494142532
31.2564102564103 0.123152732849121
32.7435897435897 0.111564792692661
34.3025641025641 0.122461453080177
35.9358974358974 0.104622237384319
37.648717948718 0.107112310826778
39.4410256410256 0.113947942852974
41.3179487179487 0.116408444941044
43.2871794871795 0.115040935575962
45.3487179487179 0.111755505204201
47.5076923076923 0.108950786292553
49.7692307692308 0.109097637236118
52.1384615384615 0.113797806203365
54.6205128205128 0.105325698852539
57.2230769230769 0.114359311759472
59.9461538461538 0.108095400035381
62.8025641025641 0.111909307539463
65.7923076923077 0.106563925743103
68.925641025641 0.109510079026222
72.2076923076923 0.115052677690983
75.6461538461539 0.113139010965824
79.2461538461538 0.110817968845367
83.0205128205128 0.11194820702076
86.974358974359 0.112169705331326
91.1153846153846 0.110001064836979
95.4538461538462 0.11061280220747
100 0.108829155564308
};
\addlegendentry{sub 16, exact}
\addplot [, color0, opacity=0.6, mark=diamond*, mark size=0.5, mark options={solid}, only marks, forget plot]
table {%
1 0.912424087524414
1.04615384615385 0.866924107074738
1.0974358974359 0.800122261047363
1.14871794871795 0.732704043388367
1.2025641025641 0.665773630142212
1.26153846153846 0.668897569179535
1.32051282051282 0.651383817195892
1.38461538461538 0.574334383010864
1.44871794871795 0.557068347930908
1.51794871794872 0.541315495967865
1.58974358974359 0.524676322937012
1.66666666666667 0.502680718898773
1.74615384615385 0.564239203929901
1.82820512820513 0.495104610919952
1.91538461538462 0.501410663127899
2.00769230769231 0.535513341426849
2.1025641025641 0.511602401733398
2.20512820512821 0.44703397154808
2.30769230769231 0.466646194458008
2.41794871794872 0.448714733123779
2.53333333333333 0.45231905579567
2.65384615384615 0.433494299650192
2.78205128205128 0.402695268392563
2.91282051282051 0.387213319540024
3.05384615384615 0.419345289468765
3.1974358974359 0.383439004421234
3.35128205128205 0.389457315206528
3.51025641025641 0.36727836728096
3.67692307692308 0.357728481292725
3.85128205128205 0.314862877130508
4.03589743589744 0.337143093347549
4.22820512820513 0.33668726682663
4.42820512820513 0.334983348846436
4.64102564102564 0.270255535840988
4.86153846153846 0.290667027235031
5.09230769230769 0.286500781774521
5.33589743589744 0.312222719192505
5.58974358974359 0.315952122211456
5.85641025641026 0.236026674509048
6.13589743589744 0.263042747974396
6.42564102564103 0.2591672539711
6.73333333333333 0.24444542825222
7.05384615384615 0.231169506907463
7.38974358974359 0.25750595331192
7.74102564102564 0.233187109231949
8.11025641025641 0.223097279667854
8.4974358974359 0.24994383752346
8.9 0.216609999537468
9.32564102564103 0.223279342055321
9.76923076923077 0.225968405604362
10.2333333333333 0.186237990856171
10.7205128205128 0.198625758290291
11.2307692307692 0.200976952910423
11.7666666666667 0.214397951960564
12.3282051282051 0.19800677895546
12.9153846153846 0.155444160103798
13.5282051282051 0.202379509806633
14.174358974359 0.177456766366959
14.8487179487179 0.188591524958611
15.5564102564103 0.151705875992775
16.2974358974359 0.167377054691315
17.0717948717949 0.162710726261139
17.8846153846154 0.170241713523865
18.7358974358974 0.164222598075867
19.6282051282051 0.174826323986053
20.5641025641026 0.159764692187309
21.5435897435897 0.13172510266304
22.5692307692308 0.136665105819702
23.6435897435897 0.141137436032295
24.7692307692308 0.179056718945503
25.9487179487179 0.154509335756302
27.1846153846154 0.16156505048275
28.4794871794872 0.156191155314445
29.8358974358974 0.141897469758987
31.2564102564103 0.171452924609184
32.7435897435897 0.155998855829239
34.3025641025641 0.166379496455193
35.9358974358974 0.171604603528976
37.648717948718 0.173747688531876
39.4410256410256 0.162788584828377
41.3179487179487 0.161029919981956
43.2871794871795 0.17175917327404
45.3487179487179 0.174101039767265
47.5076923076923 0.179085746407509
49.7692307692308 0.173245534300804
52.1384615384615 0.167853876948357
54.6205128205128 0.178223952651024
57.2230769230769 0.170795515179634
59.9461538461538 0.176816523075104
62.8025641025641 0.166052147746086
65.7923076923077 0.187605127692223
68.925641025641 0.176452964544296
72.2076923076923 0.17225569486618
75.6461538461539 0.169929221272469
79.2461538461538 0.19031810760498
83.0205128205128 0.180894672870636
86.974358974359 0.172792464494705
91.1153846153846 0.180908888578415
95.4538461538462 0.182058960199356
100 0.183407410979271
};
\addplot [, color0, opacity=0.6, mark=diamond*, mark size=0.5, mark options={solid}, only marks, forget plot]
table {%
1 0.963827550411224
1.04615384615385 0.818356335163116
1.0974358974359 0.701636135578156
1.14871794871795 0.685225903987885
1.2025641025641 0.624708771705627
1.26153846153846 0.645189225673676
1.32051282051282 0.514663696289062
1.38461538461538 0.477619737386703
1.44871794871795 0.4549660384655
1.51794871794872 0.510686755180359
1.58974358974359 0.506451010704041
1.66666666666667 0.472172349691391
1.74615384615385 0.484474003314972
1.82820512820513 0.462175279855728
1.91538461538462 0.466709822416306
2.00769230769231 0.445034176111221
2.1025641025641 0.425559610128403
2.20512820512821 0.408659696578979
2.30769230769231 0.422151476144791
2.41794871794872 0.368097215890884
2.53333333333333 0.399953186511993
2.65384615384615 0.400472164154053
2.78205128205128 0.382501214742661
2.91282051282051 0.373369544744492
3.05384615384615 0.399668902158737
3.1974358974359 0.391961306333542
3.35128205128205 0.376461893320084
3.51025641025641 0.369284391403198
3.67692307692308 0.355394184589386
3.85128205128205 0.362631320953369
4.03589743589744 0.33654460310936
4.22820512820513 0.351982086896896
4.42820512820513 0.347300380468369
4.64102564102564 0.369035392999649
4.86153846153846 0.33823624253273
5.09230769230769 0.337843775749207
5.33589743589744 0.316310852766037
5.58974358974359 0.328718632459641
5.85641025641026 0.321899473667145
6.13589743589744 0.291397750377655
6.42564102564103 0.285479128360748
6.73333333333333 0.30653128027916
7.05384615384615 0.313486069440842
7.38974358974359 0.279577523469925
7.74102564102564 0.264827162027359
8.11025641025641 0.246498107910156
8.4974358974359 0.244983717799187
8.9 0.273148894309998
9.32564102564103 0.201263144612312
9.76923076923077 0.213747292757034
10.2333333333333 0.200740933418274
10.7205128205128 0.185734689235687
11.2307692307692 0.173381209373474
11.7666666666667 0.162100374698639
12.3282051282051 0.183247476816177
12.9153846153846 0.14571051299572
13.5282051282051 0.150157332420349
14.174358974359 0.138690158724785
14.8487179487179 0.176801040768623
15.5564102564103 0.136294394731522
16.2974358974359 0.134519666433334
17.0717948717949 0.12659740447998
17.8846153846154 0.123166061937809
18.7358974358974 0.145817935466766
19.6282051282051 0.138627082109451
20.5641025641026 0.126391693949699
21.5435897435897 0.108399547636509
22.5692307692308 0.0955685302615166
23.6435897435897 0.0991191267967224
24.7692307692308 0.105921439826488
25.9487179487179 0.107289969921112
27.1846153846154 0.102392844855785
28.4794871794872 0.10167620331049
29.8358974358974 0.107756435871124
31.2564102564103 0.103794418275356
32.7435897435897 0.115066401660442
34.3025641025641 0.11018443107605
35.9358974358974 0.113663077354431
37.648717948718 0.11089926213026
39.4410256410256 0.107884339988232
41.3179487179487 0.114959955215454
43.2871794871795 0.103905789554119
45.3487179487179 0.104830384254456
47.5076923076923 0.109780214726925
49.7692307692308 0.10284473747015
52.1384615384615 0.108061671257019
54.6205128205128 0.0984899625182152
57.2230769230769 0.108428359031677
59.9461538461538 0.105027988553047
62.8025641025641 0.116188459098339
65.7923076923077 0.092069610953331
68.925641025641 0.112094581127167
72.2076923076923 0.107126377522945
75.6461538461539 0.108458258211613
79.2461538461538 0.0998389720916748
83.0205128205128 0.105684101581573
86.974358974359 0.107798650860786
91.1153846153846 0.107153035700321
95.4538461538462 0.100834049284458
100 0.0968785881996155
};
\addplot [, color0, opacity=0.6, mark=diamond*, mark size=0.5, mark options={solid}, only marks, forget plot]
table {%
1 0.950647532939911
1.04615384615385 0.851631760597229
1.0974358974359 0.724289655685425
1.14871794871795 0.631197273731232
1.2025641025641 0.604303538799286
1.26153846153846 0.483979135751724
1.32051282051282 0.573766052722931
1.38461538461538 0.500865876674652
1.44871794871795 0.503856122493744
1.51794871794872 0.48355832695961
1.58974358974359 0.406964272260666
1.66666666666667 0.459552526473999
1.74615384615385 0.439813137054443
1.82820512820513 0.451313078403473
1.91538461538462 0.453810602426529
2.00769230769231 0.420349210500717
2.1025641025641 0.386937588453293
2.20512820512821 0.399334400892258
2.30769230769231 0.402429789304733
2.41794871794872 0.380543112754822
2.53333333333333 0.336925655603409
2.65384615384615 0.365827292203903
2.78205128205128 0.340268641710281
2.91282051282051 0.316996604204178
3.05384615384615 0.371348530054092
3.1974358974359 0.306850254535675
3.35128205128205 0.288736969232559
3.51025641025641 0.312628298997879
3.67692307692308 0.312006533145905
3.85128205128205 0.312524408102036
4.03589743589744 0.275888055562973
4.22820512820513 0.275465369224548
4.42820512820513 0.30283710360527
4.64102564102564 0.291304409503937
4.86153846153846 0.267265170812607
5.09230769230769 0.233507797122002
5.33589743589744 0.251975417137146
5.58974358974359 0.301395982503891
5.85641025641026 0.259168595075607
6.13589743589744 0.213997513055801
6.42564102564103 0.209526643157005
6.73333333333333 0.225482329726219
7.05384615384615 0.206535056233406
7.38974358974359 0.220500573515892
7.74102564102564 0.246576890349388
8.11025641025641 0.268462747335434
8.4974358974359 0.190491870045662
8.9 0.219602331519127
9.32564102564103 0.189597874879837
9.76923076923077 0.191489726305008
10.2333333333333 0.175933837890625
10.7205128205128 0.199954196810722
11.2307692307692 0.160996302962303
11.7666666666667 0.150740042328835
12.3282051282051 0.175718888640404
12.9153846153846 0.157140493392944
13.5282051282051 0.163869023323059
14.174358974359 0.15583561360836
14.8487179487179 0.151000633835793
15.5564102564103 0.157324895262718
16.2974358974359 0.14666286110878
17.0717948717949 0.158681109547615
17.8846153846154 0.133713960647583
18.7358974358974 0.164872527122498
19.6282051282051 0.122986912727356
20.5641025641026 0.13736517727375
21.5435897435897 0.119259439408779
22.5692307692308 0.129762172698975
23.6435897435897 0.129223302006721
24.7692307692308 0.119075514376163
25.9487179487179 0.118029691278934
27.1846153846154 0.133316680788994
28.4794871794872 0.130420431494713
29.8358974358974 0.134379893541336
31.2564102564103 0.118010617792606
32.7435897435897 0.136424586176872
34.3025641025641 0.137845784425735
35.9358974358974 0.138827592134476
37.648717948718 0.145938560366631
39.4410256410256 0.14497934281826
41.3179487179487 0.140773072838783
43.2871794871795 0.144314214587212
45.3487179487179 0.141407534480095
47.5076923076923 0.14108544588089
49.7692307692308 0.146301433444023
52.1384615384615 0.148774370551109
54.6205128205128 0.147023916244507
57.2230769230769 0.141800343990326
59.9461538461538 0.147409111261368
62.8025641025641 0.140414640307426
65.7923076923077 0.155733838677406
68.925641025641 0.15041184425354
72.2076923076923 0.141927540302277
75.6461538461539 0.147163733839989
79.2461538461538 0.156320706009865
83.0205128205128 0.145944312214851
86.974358974359 0.147633075714111
91.1153846153846 0.146355494856834
95.4538461538462 0.147160097956657
100 0.155663013458252
};
\addplot [, color0, opacity=0.6, mark=diamond*, mark size=0.5, mark options={solid}, only marks, forget plot]
table {%
1 0.90020889043808
1.04615384615385 0.773889660835266
1.0974358974359 0.739956498146057
1.14871794871795 0.684716939926147
1.2025641025641 0.655635356903076
1.26153846153846 0.61608350276947
1.32051282051282 0.645255029201508
1.38461538461538 0.634983479976654
1.44871794871795 0.514991700649261
1.51794871794872 0.56666773557663
1.58974358974359 0.597849547863007
1.66666666666667 0.532172501087189
1.74615384615385 0.607634603977203
1.82820512820513 0.579104125499725
1.91538461538462 0.570293366909027
2.00769230769231 0.50946033000946
2.1025641025641 0.463100731372833
2.20512820512821 0.447565138339996
2.30769230769231 0.49855962395668
2.41794871794872 0.476393043994904
2.53333333333333 0.48551082611084
2.65384615384615 0.438859552145004
2.78205128205128 0.491731971502304
2.91282051282051 0.442050278186798
3.05384615384615 0.490895360708237
3.1974358974359 0.437945038080215
3.35128205128205 0.428974777460098
3.51025641025641 0.385170191526413
3.67692307692308 0.360115051269531
3.85128205128205 0.394575595855713
4.03589743589744 0.411074012517929
4.22820512820513 0.390928000211716
4.42820512820513 0.365122616291046
4.64102564102564 0.405967950820923
4.86153846153846 0.348401039838791
5.09230769230769 0.365499794483185
5.33589743589744 0.383140742778778
5.58974358974359 0.345789521932602
5.85641025641026 0.320457547903061
6.13589743589744 0.339897781610489
6.42564102564103 0.321882307529449
6.73333333333333 0.344376295804977
7.05384615384615 0.334324032068253
7.38974358974359 0.289160162210464
7.74102564102564 0.299462080001831
8.11025641025641 0.271845549345016
8.4974358974359 0.325926095247269
8.9 0.300592720508575
9.32564102564103 0.293459683656693
9.76923076923077 0.270787209272385
10.2333333333333 0.267871826887131
10.7205128205128 0.236225560307503
11.2307692307692 0.256585836410522
11.7666666666667 0.285549700260162
12.3282051282051 0.268809974193573
12.9153846153846 0.239286929368973
13.5282051282051 0.201522156596184
14.174358974359 0.22749924659729
14.8487179487179 0.198641255497932
15.5564102564103 0.173223122954369
16.2974358974359 0.193418145179749
17.0717948717949 0.174427583813667
17.8846153846154 0.177893236279488
18.7358974358974 0.169689700007439
19.6282051282051 0.173014208674431
20.5641025641026 0.183251783251762
21.5435897435897 0.175375774502754
22.5692307692308 0.156216844916344
23.6435897435897 0.149422749876976
24.7692307692308 0.14630089700222
25.9487179487179 0.16559810936451
27.1846153846154 0.144438877701759
28.4794871794872 0.165667995810509
29.8358974358974 0.151184737682343
31.2564102564103 0.143659427762032
32.7435897435897 0.141771838068962
34.3025641025641 0.148513719439507
35.9358974358974 0.148332148790359
37.648717948718 0.146108150482178
39.4410256410256 0.140366941690445
41.3179487179487 0.144228890538216
43.2871794871795 0.141514882445335
45.3487179487179 0.146151259541512
47.5076923076923 0.146759539842606
49.7692307692308 0.145066142082214
52.1384615384615 0.14434988796711
54.6205128205128 0.148825153708458
57.2230769230769 0.14118929207325
59.9461538461538 0.15072312951088
62.8025641025641 0.143442064523697
65.7923076923077 0.15042670071125
68.925641025641 0.148506164550781
72.2076923076923 0.14380732178688
75.6461538461539 0.143079578876495
79.2461538461538 0.149821758270264
83.0205128205128 0.145909398794174
86.974358974359 0.147227048873901
91.1153846153846 0.143932089209557
95.4538461538462 0.14796045422554
100 0.148235023021698
};
\addplot [, color1, opacity=0.6, mark=square*, mark size=0.5, mark options={solid}, only marks]
table {%
1 0.957085609436035
1.04615384615385 0.7833571434021
1.0974358974359 0.783753335475922
1.14871794871795 0.738804996013641
1.2025641025641 0.658386647701263
1.26153846153846 0.578368842601776
1.32051282051282 0.607825458049774
1.38461538461538 0.631368160247803
1.44871794871795 0.58179771900177
1.51794871794872 0.546043574810028
1.58974358974359 0.583394825458527
1.66666666666667 0.563975811004639
1.74615384615385 0.486912548542023
1.82820512820513 0.458968609571457
1.91538461538462 0.491104602813721
2.00769230769231 0.571270704269409
2.1025641025641 0.470895290374756
2.20512820512821 0.502003371715546
2.30769230769231 0.458049297332764
2.41794871794872 0.415270149707794
2.53333333333333 0.446558952331543
2.65384615384615 0.457586377859116
2.78205128205128 0.437759608030319
2.91282051282051 0.429179280996323
3.05384615384615 0.383780121803284
3.1974358974359 0.337601155042648
3.35128205128205 0.354704469442368
3.51025641025641 0.348070234060287
3.67692307692308 0.40059420466423
3.85128205128205 0.344485849142075
4.03589743589744 0.363784164190292
4.22820512820513 0.335498064756393
4.42820512820513 0.342067092657089
4.64102564102564 0.33266693353653
4.86153846153846 0.34915217757225
5.09230769230769 0.392207205295563
5.33589743589744 0.296846628189087
5.58974358974359 0.33403342962265
5.85641025641026 0.314203143119812
6.13589743589744 0.291137129068375
6.42564102564103 0.304193466901779
6.73333333333333 0.288982301950455
7.05384615384615 0.26018562912941
7.38974358974359 0.259481728076935
7.74102564102564 0.299889236688614
8.11025641025641 0.255394786596298
8.4974358974359 0.279480427503586
8.9 0.235690876841545
9.32564102564103 0.276370525360107
9.76923076923077 0.287420272827148
10.2333333333333 0.255710929632187
10.7205128205128 0.282788693904877
11.2307692307692 0.284754991531372
11.7666666666667 0.240816548466682
12.3282051282051 0.252412647008896
12.9153846153846 0.251680463552475
13.5282051282051 0.229075655341148
14.174358974359 0.201370820403099
14.8487179487179 0.23576819896698
15.5564102564103 0.219633728265762
16.2974358974359 0.233126923441887
17.0717948717949 0.209264516830444
17.8846153846154 0.257837146520615
18.7358974358974 0.240336254239082
19.6282051282051 0.232350215315819
20.5641025641026 0.21199107170105
21.5435897435897 0.210631892085075
22.5692307692308 0.212942883372307
23.6435897435897 0.203719526529312
24.7692307692308 0.195222601294518
25.9487179487179 0.174915358424187
27.1846153846154 0.221433311700821
28.4794871794872 0.207618579268456
29.8358974358974 0.178089126944542
31.2564102564103 0.194746598601341
32.7435897435897 0.22751584649086
34.3025641025641 0.206713870167732
35.9358974358974 0.226016089320183
37.648717948718 0.226487323641777
39.4410256410256 0.226911827921867
41.3179487179487 0.227331355214119
43.2871794871795 0.229304790496826
45.3487179487179 0.228372648358345
47.5076923076923 0.237947508692741
49.7692307692308 0.228976160287857
52.1384615384615 0.225764185190201
54.6205128205128 0.237342521548271
57.2230769230769 0.232407912611961
59.9461538461538 0.234806969761848
62.8025641025641 0.235932350158691
65.7923076923077 0.209168240427971
68.925641025641 0.224647998809814
72.2076923076923 0.227354809641838
75.6461538461539 0.228796437382698
79.2461538461538 0.198701962828636
83.0205128205128 0.206330612301826
86.974358974359 0.231464967131615
91.1153846153846 0.224629327654839
95.4538461538462 0.219008788466454
100 0.210140183568001
};
\addlegendentry{mb 128, mc 1}
\addplot [, color1, opacity=0.6, mark=square*, mark size=0.5, mark options={solid}, only marks, forget plot]
table {%
1 0.951353669166565
1.04615384615385 0.848787426948547
1.0974358974359 0.782859802246094
1.14871794871795 0.794858813285828
1.2025641025641 0.731209278106689
1.26153846153846 0.698558747768402
1.32051282051282 0.685271263122559
1.38461538461538 0.69742739200592
1.44871794871795 0.653735458850861
1.51794871794872 0.592855989933014
1.58974358974359 0.671345174312592
1.66666666666667 0.571296393871307
1.74615384615385 0.575442254543304
1.82820512820513 0.553687870502472
1.91538461538462 0.526578545570374
2.00769230769231 0.582945823669434
2.1025641025641 0.548028767108917
2.20512820512821 0.559426963329315
2.30769230769231 0.542582213878632
2.41794871794872 0.551207005977631
2.53333333333333 0.43452587723732
2.65384615384615 0.512720584869385
2.78205128205128 0.512793302536011
2.91282051282051 0.50345104932785
3.05384615384615 0.500760853290558
3.1974358974359 0.503704488277435
3.35128205128205 0.501065313816071
3.51025641025641 0.525647819042206
3.67692307692308 0.439083158969879
3.85128205128205 0.477543413639069
4.03589743589744 0.479429394006729
4.22820512820513 0.377364665269852
4.42820512820513 0.465704351663589
4.64102564102564 0.417709559202194
4.86153846153846 0.4542256295681
5.09230769230769 0.42603799700737
5.33589743589744 0.421557903289795
5.58974358974359 0.468836694955826
5.85641025641026 0.440942049026489
6.13589743589744 0.418345898389816
6.42564102564103 0.255386561155319
6.73333333333333 0.378200858831406
7.05384615384615 0.337275266647339
7.38974358974359 0.372408777475357
7.74102564102564 0.368918389081955
8.11025641025641 0.326210498809814
8.4974358974359 0.335245221853256
8.9 0.337673008441925
9.32564102564103 0.247381925582886
9.76923076923077 0.277219891548157
10.2333333333333 0.267387479543686
10.7205128205128 0.290920197963715
11.2307692307692 0.264012843370438
11.7666666666667 0.298721969127655
12.3282051282051 0.274841755628586
12.9153846153846 0.294467926025391
13.5282051282051 0.232391342520714
14.174358974359 0.224403530359268
14.8487179487179 0.283504158258438
15.5564102564103 0.204802021384239
16.2974358974359 0.172391474246979
17.0717948717949 0.189470931887627
17.8846153846154 0.235602378845215
18.7358974358974 0.209105178713799
19.6282051282051 0.260483235120773
20.5641025641026 0.202073097229004
21.5435897435897 0.241353183984756
22.5692307692308 0.207280904054642
23.6435897435897 0.189780160784721
24.7692307692308 0.216544136404991
25.9487179487179 0.204612255096436
27.1846153846154 0.222099736332893
28.4794871794872 0.216946586966515
29.8358974358974 0.206639006733894
31.2564102564103 0.182822421193123
32.7435897435897 0.197523221373558
34.3025641025641 0.18877200782299
35.9358974358974 0.20431213080883
37.648717948718 0.194340661168098
39.4410256410256 0.189367935061455
41.3179487179487 0.191542729735374
43.2871794871795 0.193148136138916
45.3487179487179 0.197765052318573
47.5076923076923 0.200085639953613
49.7692307692308 0.205207392573357
52.1384615384615 0.207890748977661
54.6205128205128 0.207110688090324
57.2230769230769 0.191971823573112
59.9461538461538 0.205288246273994
62.8025641025641 0.209531143307686
65.7923076923077 0.19284999370575
68.925641025641 0.186818778514862
72.2076923076923 0.205473348498344
75.6461538461539 0.195249199867249
79.2461538461538 0.195900157094002
83.0205128205128 0.203106909990311
86.974358974359 0.204171374440193
91.1153846153846 0.1923438757658
95.4538461538462 0.19181402027607
100 0.199948325753212
};
\addplot [, color1, opacity=0.6, mark=square*, mark size=0.5, mark options={solid}, only marks, forget plot]
table {%
1 0.94314444065094
1.04615384615385 0.878027439117432
1.0974358974359 0.834002435207367
1.14871794871795 0.762448251247406
1.2025641025641 0.735769391059875
1.26153846153846 0.756207942962646
1.32051282051282 0.655962109565735
1.38461538461538 0.675584435462952
1.44871794871795 0.633192539215088
1.51794871794872 0.582743108272552
1.58974358974359 0.601026058197021
1.66666666666667 0.584083437919617
1.74615384615385 0.619395196437836
1.82820512820513 0.526982247829437
1.91538461538462 0.562084674835205
2.00769230769231 0.552991926670074
2.1025641025641 0.593074917793274
2.20512820512821 0.559264838695526
2.30769230769231 0.537013232707977
2.41794871794872 0.563763856887817
2.53333333333333 0.504997909069061
2.65384615384615 0.53549587726593
2.78205128205128 0.418310642242432
2.91282051282051 0.521261751651764
3.05384615384615 0.487166315317154
3.1974358974359 0.425396651029587
3.35128205128205 0.403994053602219
3.51025641025641 0.475722998380661
3.67692307692308 0.45844241976738
3.85128205128205 0.448366731405258
4.03589743589744 0.418455690145493
4.22820512820513 0.383652776479721
4.42820512820513 0.38519811630249
4.64102564102564 0.416651874780655
4.86153846153846 0.388901054859161
5.09230769230769 0.341777145862579
5.33589743589744 0.371748298406601
5.58974358974359 0.327646434307098
5.85641025641026 0.339859962463379
6.13589743589744 0.35237842798233
6.42564102564103 0.32008969783783
6.73333333333333 0.359890878200531
7.05384615384615 0.311016410589218
7.38974358974359 0.350318670272827
7.74102564102564 0.243932485580444
8.11025641025641 0.323245346546173
8.4974358974359 0.31336909532547
8.9 0.20470018684864
9.32564102564103 0.279900133609772
9.76923076923077 0.287450939416885
10.2333333333333 0.238610550761223
10.7205128205128 0.256125539541245
11.2307692307692 0.265954822301865
11.7666666666667 0.286762088537216
12.3282051282051 0.250280946493149
12.9153846153846 0.245642051100731
13.5282051282051 0.255082219839096
14.174358974359 0.232527256011963
14.8487179487179 0.239209339022636
15.5564102564103 0.232363685965538
16.2974358974359 0.212738424539566
17.0717948717949 0.182886853814125
17.8846153846154 0.180536210536957
18.7358974358974 0.21282921731472
19.6282051282051 0.216188654303551
20.5641025641026 0.182023033499718
21.5435897435897 0.203476533293724
22.5692307692308 0.20749743282795
23.6435897435897 0.200640872120857
24.7692307692308 0.201778054237366
25.9487179487179 0.168631747364998
27.1846153846154 0.210743591189384
28.4794871794872 0.217313095927238
29.8358974358974 0.191079422831535
31.2564102564103 0.190542727708817
32.7435897435897 0.188908323645592
34.3025641025641 0.191500410437584
35.9358974358974 0.196251451969147
37.648717948718 0.188730224967003
39.4410256410256 0.185151144862175
41.3179487179487 0.186565309762955
43.2871794871795 0.191795751452446
45.3487179487179 0.190847754478455
47.5076923076923 0.213262319564819
49.7692307692308 0.187982007861137
52.1384615384615 0.194876298308372
54.6205128205128 0.196254134178162
57.2230769230769 0.195358082652092
59.9461538461538 0.201501041650772
62.8025641025641 0.185982942581177
65.7923076923077 0.202561020851135
68.925641025641 0.205834254622459
72.2076923076923 0.187708243727684
75.6461538461539 0.194874629378319
79.2461538461538 0.204500511288643
83.0205128205128 0.203210160136223
86.974358974359 0.193565636873245
91.1153846153846 0.203831151127815
95.4538461538462 0.206028819084167
100 0.193084120750427
};
\addplot [, color1, opacity=0.6, mark=square*, mark size=0.5, mark options={solid}, only marks, forget plot]
table {%
1 0.952087700366974
1.04615384615385 0.851473808288574
1.0974358974359 0.756118059158325
1.14871794871795 0.734308421611786
1.2025641025641 0.615283131599426
1.26153846153846 0.667478919029236
1.32051282051282 0.643469214439392
1.38461538461538 0.613355576992035
1.44871794871795 0.500065982341766
1.51794871794872 0.577403843402863
1.58974358974359 0.511420130729675
1.66666666666667 0.539280593395233
1.74615384615385 0.495524555444717
1.82820512820513 0.51640784740448
1.91538461538462 0.446843057870865
2.00769230769231 0.57584422826767
2.1025641025641 0.519806206226349
2.20512820512821 0.495853424072266
2.30769230769231 0.420267969369888
2.41794871794872 0.546257615089417
2.53333333333333 0.46806412935257
2.65384615384615 0.490392357110977
2.78205128205128 0.390024065971375
2.91282051282051 0.438383907079697
3.05384615384615 0.475009351968765
3.1974358974359 0.472061485052109
3.35128205128205 0.470431804656982
3.51025641025641 0.468948185443878
3.67692307692308 0.458625376224518
3.85128205128205 0.449766606092453
4.03589743589744 0.385365903377533
4.22820512820513 0.402597993612289
4.42820512820513 0.466330915689468
4.64102564102564 0.376293480396271
4.86153846153846 0.381225079298019
5.09230769230769 0.386646419763565
5.33589743589744 0.366262882947922
5.58974358974359 0.384399086236954
5.85641025641026 0.385435223579407
6.13589743589744 0.418990612030029
6.42564102564103 0.389209300279617
6.73333333333333 0.385857462882996
7.05384615384615 0.340535491704941
7.38974358974359 0.395870357751846
7.74102564102564 0.297870218753815
8.11025641025641 0.338627904653549
8.4974358974359 0.274785667657852
8.9 0.301378756761551
9.32564102564103 0.266261965036392
9.76923076923077 0.283947020769119
10.2333333333333 0.28601861000061
10.7205128205128 0.292900770902634
11.2307692307692 0.186878487467766
11.7666666666667 0.23845486342907
12.3282051282051 0.231998160481453
12.9153846153846 0.291744858026505
13.5282051282051 0.264301329851151
14.174358974359 0.247165247797966
14.8487179487179 0.259509414434433
15.5564102564103 0.169722884893417
16.2974358974359 0.214497834444046
17.0717948717949 0.183511406183243
17.8846153846154 0.19771783053875
18.7358974358974 0.203337460756302
19.6282051282051 0.217142447829247
20.5641025641026 0.214294537901878
21.5435897435897 0.198293700814247
22.5692307692308 0.21113632619381
23.6435897435897 0.199126750230789
24.7692307692308 0.223727509379387
25.9487179487179 0.208256244659424
27.1846153846154 0.190873250365257
28.4794871794872 0.198511093854904
29.8358974358974 0.176179319620132
31.2564102564103 0.17244391143322
32.7435897435897 0.175159439444542
34.3025641025641 0.18366651237011
35.9358974358974 0.168487176299095
37.648717948718 0.158721312880516
39.4410256410256 0.167859122157097
41.3179487179487 0.167948767542839
43.2871794871795 0.168893322348595
45.3487179487179 0.169981107115746
47.5076923076923 0.173543497920036
49.7692307692308 0.174078151583672
52.1384615384615 0.171341702342033
54.6205128205128 0.169610857963562
57.2230769230769 0.173227429389954
59.9461538461538 0.168631166219711
62.8025641025641 0.168033465743065
65.7923076923077 0.183105632662773
68.925641025641 0.174579188227654
72.2076923076923 0.180778726935387
75.6461538461539 0.176532313227654
79.2461538461538 0.175625368952751
83.0205128205128 0.176984906196594
86.974358974359 0.182233199477196
91.1153846153846 0.176004275679588
95.4538461538462 0.179439887404442
100 0.179329439997673
};
\addplot [, color1, opacity=0.6, mark=square*, mark size=0.5, mark options={solid}, only marks, forget plot]
table {%
1 0.883962631225586
1.04615384615385 0.821612000465393
1.0974358974359 0.726090729236603
1.14871794871795 0.732755303382874
1.2025641025641 0.640961050987244
1.26153846153846 0.65021163225174
1.32051282051282 0.64508581161499
1.38461538461538 0.585134208202362
1.44871794871795 0.58390086889267
1.51794871794872 0.584287583827972
1.58974358974359 0.591844379901886
1.66666666666667 0.545236527919769
1.74615384615385 0.527501046657562
1.82820512820513 0.51330155134201
1.91538461538462 0.526581466197968
2.00769230769231 0.527162373065948
2.1025641025641 0.483817636966705
2.20512820512821 0.481213331222534
2.30769230769231 0.494928181171417
2.41794871794872 0.464383512735367
2.53333333333333 0.479005098342896
2.65384615384615 0.494285672903061
2.78205128205128 0.468895673751831
2.91282051282051 0.4904425740242
3.05384615384615 0.520952224731445
3.1974358974359 0.437317371368408
3.35128205128205 0.476098746061325
3.51025641025641 0.480490893125534
3.67692307692308 0.485720723867416
3.85128205128205 0.457436621189117
4.03589743589744 0.402869999408722
4.22820512820513 0.431202471256256
4.42820512820513 0.499218195676804
4.64102564102564 0.380300372838974
4.86153846153846 0.354769587516785
5.09230769230769 0.454367250204086
5.33589743589744 0.403119951486588
5.58974358974359 0.429007112979889
5.85641025641026 0.369609445333481
6.13589743589744 0.414407104253769
6.42564102564103 0.371585518121719
6.73333333333333 0.321092367172241
7.05384615384615 0.33825221657753
7.38974358974359 0.392834067344666
7.74102564102564 0.341760635375977
8.11025641025641 0.397423535585403
8.4974358974359 0.298042327165604
8.9 0.299722462892532
9.32564102564103 0.297729343175888
9.76923076923077 0.331728547811508
10.2333333333333 0.309782773256302
10.7205128205128 0.319245636463165
11.2307692307692 0.249386548995972
11.7666666666667 0.242586046457291
12.3282051282051 0.247887685894966
12.9153846153846 0.302817225456238
13.5282051282051 0.238790795207024
14.174358974359 0.298683255910873
14.8487179487179 0.261032074689865
15.5564102564103 0.249317690730095
16.2974358974359 0.243071749806404
17.0717948717949 0.222953662276268
17.8846153846154 0.203998312354088
18.7358974358974 0.223294049501419
19.6282051282051 0.250179916620255
20.5641025641026 0.220883801579475
21.5435897435897 0.202190548181534
22.5692307692308 0.230601027607918
23.6435897435897 0.230539754033089
24.7692307692308 0.210913613438606
25.9487179487179 0.241388902068138
27.1846153846154 0.233454510569572
28.4794871794872 0.209408804774284
29.8358974358974 0.216055750846863
31.2564102564103 0.229049205780029
32.7435897435897 0.245132401585579
34.3025641025641 0.245368957519531
35.9358974358974 0.238367468118668
37.648717948718 0.232314988970757
39.4410256410256 0.23215015232563
41.3179487179487 0.227359816431999
43.2871794871795 0.231317237019539
45.3487179487179 0.241165682673454
47.5076923076923 0.247437000274658
49.7692307692308 0.23125171661377
52.1384615384615 0.238304331898689
54.6205128205128 0.23528690636158
57.2230769230769 0.228216990828514
59.9461538461538 0.233564123511314
62.8025641025641 0.237465664744377
65.7923076923077 0.230828449130058
68.925641025641 0.215068623423576
72.2076923076923 0.234367802739143
75.6461538461539 0.229822069406509
79.2461538461538 0.226759865880013
83.0205128205128 0.232041999697685
86.974358974359 0.230709552764893
91.1153846153846 0.224057897925377
95.4538461538462 0.229751512408257
100 0.22603277862072
};
\addplot [, color2, opacity=0.6, mark=triangle*, mark size=0.5, mark options={solid,rotate=180}, only marks]
table {%
1 0.676551043987274
1.04615384615385 0.558850169181824
1.0974358974359 0.447052866220474
1.14871794871795 0.440687417984009
1.2025641025641 0.357560604810715
1.26153846153846 0.436119079589844
1.32051282051282 0.424261182546616
1.38461538461538 0.378135740756989
1.44871794871795 0.345074385404587
1.51794871794872 0.33459261059761
1.58974358974359 0.413366615772247
1.66666666666667 0.390501827001572
1.74615384615385 0.358498156070709
1.82820512820513 0.361860275268555
1.91538461538462 0.361782759428024
2.00769230769231 0.3602135181427
2.1025641025641 0.380427330732346
2.20512820512821 0.336662471294403
2.30769230769231 0.30333623290062
2.41794871794872 0.375387161970139
2.53333333333333 0.353107452392578
2.65384615384615 0.360527634620667
2.78205128205128 0.327173382043839
2.91282051282051 0.357004165649414
3.05384615384615 0.404513120651245
3.1974358974359 0.307355493307114
3.35128205128205 0.346837639808655
3.51025641025641 0.318865150213242
3.67692307692308 0.342413336038589
3.85128205128205 0.348639696836472
4.03589743589744 0.333893120288849
4.22820512820513 0.327952593564987
4.42820512820513 0.304024994373322
4.64102564102564 0.280841737985611
4.86153846153846 0.291388839483261
5.09230769230769 0.321111887693405
5.33589743589744 0.254120588302612
5.58974358974359 0.31567856669426
5.85641025641026 0.289551973342896
6.13589743589744 0.28338611125946
6.42564102564103 0.274564325809479
6.73333333333333 0.272933185100555
7.05384615384615 0.277342349290848
7.38974358974359 0.279231160879135
7.74102564102564 0.285906076431274
8.11025641025641 0.240175947546959
8.4974358974359 0.276408433914185
8.9 0.258193612098694
9.32564102564103 0.20531490445137
9.76923076923077 0.221170231699944
10.2333333333333 0.230737596750259
10.7205128205128 0.233005151152611
11.2307692307692 0.184022933244705
11.7666666666667 0.215324595570564
12.3282051282051 0.220871448516846
12.9153846153846 0.194548770785332
13.5282051282051 0.191822409629822
14.174358974359 0.189144283533096
14.8487179487179 0.178241714835167
15.5564102564103 0.205667451024055
16.2974358974359 0.147527918219566
17.0717948717949 0.165735676884651
17.8846153846154 0.153517231345177
18.7358974358974 0.172739997506142
19.6282051282051 0.159281179308891
20.5641025641026 0.129344791173935
21.5435897435897 0.161494642496109
22.5692307692308 0.154987305402756
23.6435897435897 0.143239036202431
24.7692307692308 0.142195105552673
25.9487179487179 0.119264282286167
27.1846153846154 0.136388346552849
28.4794871794872 nan
29.8358974358974 0.135893791913986
31.2564102564103 0.128348931670189
32.7435897435897 0.163149952888489
34.3025641025641 0.13164846599102
35.9358974358974 0.118180111050606
37.648717948718 0.118189372122288
39.4410256410256 0.126107886433601
41.3179487179487 0.13415265083313
43.2871794871795 0.136948943138123
45.3487179487179 0.12887716293335
47.5076923076923 0.141988724470139
49.7692307692308 0.156711339950562
52.1384615384615 0.118708707392216
54.6205128205128 0.139659494161606
57.2230769230769 0.138171002268791
59.9461538461538 0.115267090499401
62.8025641025641 0.142104133963585
65.7923076923077 0.145176991820335
68.925641025641 0.144272729754448
72.2076923076923 0.136945769190788
75.6461538461539 0.12254686653614
79.2461538461538 0.130334123969078
83.0205128205128 0.140411332249641
86.974358974359 0.13503110408783
91.1153846153846 0.132440879940987
95.4538461538462 0.142012879252434
100 0.125302404165268
};
\addlegendentry{sub 16, mc 1}
\addplot [, color2, opacity=0.6, mark=triangle*, mark size=0.5, mark options={solid,rotate=180}, only marks, forget plot]
table {%
1 0.671783626079559
1.04615384615385 0.505284786224365
1.0974358974359 0.458832085132599
1.14871794871795 0.505989849567413
1.2025641025641 0.529340445995331
1.26153846153846 0.527984440326691
1.32051282051282 0.458387941122055
1.38461538461538 0.415793716907501
1.44871794871795 0.454811811447144
1.51794871794872 0.404343992471695
1.58974358974359 0.490310162305832
1.66666666666667 0.450088232755661
1.74615384615385 0.460371643304825
1.82820512820513 0.453847169876099
1.91538461538462 0.416225582361221
2.00769230769231 0.370100110769272
2.1025641025641 0.426927626132965
2.20512820512821 0.373123645782471
2.30769230769231 0.411863625049591
2.41794871794872 0.401575177907944
2.53333333333333 0.400212287902832
2.65384615384615 0.397910922765732
2.78205128205128 0.404545694589615
2.91282051282051 0.362990975379944
3.05384615384615 0.34640645980835
3.1974358974359 0.348813742399216
3.35128205128205 0.391629308462143
3.51025641025641 0.352775007486343
3.67692307692308 0.36701163649559
3.85128205128205 0.344758570194244
4.03589743589744 0.361366420984268
4.22820512820513 0.335376709699631
4.42820512820513 0.341839611530304
4.64102564102564 0.353082984685898
4.86153846153846 0.402326971292496
5.09230769230769 0.354775041341782
5.33589743589744 0.316700607538223
5.58974358974359 0.299357026815414
5.85641025641026 0.327631443738937
6.13589743589744 0.302467077970505
6.42564102564103 0.277208626270294
6.73333333333333 0.344156950712204
7.05384615384615 0.31401914358139
7.38974358974359 0.310857534408569
7.74102564102564 0.302806705236435
8.11025641025641 0.300694048404694
8.4974358974359 0.293616265058517
8.9 0.264475643634796
9.32564102564103 0.286252498626709
9.76923076923077 0.265223443508148
10.2333333333333 0.259864062070847
10.7205128205128 0.275671809911728
11.2307692307692 0.24678210914135
11.7666666666667 0.245274141430855
12.3282051282051 0.228547722101212
12.9153846153846 0.245811372995377
13.5282051282051 0.241189405322075
14.174358974359 0.21425960958004
14.8487179487179 0.223179861903191
15.5564102564103 0.219830706715584
16.2974358974359 0.221437931060791
17.0717948717949 0.173424556851387
17.8846153846154 0.20410218834877
18.7358974358974 0.151107177138329
19.6282051282051 0.212303265929222
20.5641025641026 0.184299930930138
21.5435897435897 0.213498398661613
22.5692307692308 0.175351068377495
23.6435897435897 0.176450952887535
24.7692307692308 0.138841673731804
25.9487179487179 0.136357486248016
27.1846153846154 0.145325258374214
28.4794871794872 0.181260377168655
29.8358974358974 0.163854479789734
31.2564102564103 0.158661663532257
32.7435897435897 0.188699170947075
34.3025641025641 0.156929895281792
35.9358974358974 0.229201152920723
37.648717948718 0.177789032459259
39.4410256410256 0.174195408821106
41.3179487179487 0.199251964688301
43.2871794871795 0.184299185872078
45.3487179487179 0.212480410933495
47.5076923076923 0.215195879340172
49.7692307692308 0.186304077506065
52.1384615384615 0.175381779670715
54.6205128205128 0.194567188620567
57.2230769230769 0.184516221284866
59.9461538461538 0.171081930398941
62.8025641025641 0.178510814905167
65.7923076923077 0.138838216662407
68.925641025641 0.181173667311668
72.2076923076923 0.166090101003647
75.6461538461539 0.178260982036591
79.2461538461538 0.133622586727142
83.0205128205128 0.168394014239311
86.974358974359 0.182156607508659
91.1153846153846 0.162774428725243
95.4538461538462 0.16312375664711
100 0.137678056955338
};
\addplot [, color2, opacity=0.6, mark=triangle*, mark size=0.5, mark options={solid,rotate=180}, only marks, forget plot]
table {%
1 0.668105781078339
1.04615384615385 0.606866776943207
1.0974358974359 0.547902822494507
1.14871794871795 0.448069900274277
1.2025641025641 0.442476123571396
1.26153846153846 0.356631487607956
1.32051282051282 0.422628223896027
1.38461538461538 0.389240592718124
1.44871794871795 0.344063520431519
1.51794871794872 0.42520609498024
1.58974358974359 0.366117388010025
1.66666666666667 0.366915315389633
1.74615384615385 0.326338291168213
1.82820512820513 0.39484316110611
1.91538461538462 0.342982113361359
2.00769230769231 0.373261779546738
2.1025641025641 0.319478005170822
2.20512820512821 0.382022142410278
2.30769230769231 0.379527181386948
2.41794871794872 0.348466008901596
2.53333333333333 0.292351633310318
2.65384615384615 0.347259193658829
2.78205128205128 0.286688894033432
2.91282051282051 0.29981729388237
3.05384615384615 0.301280677318573
3.1974358974359 0.298230141401291
3.35128205128205 0.296242088079453
3.51025641025641 0.308493137359619
3.67692307692308 0.284030050039291
3.85128205128205 0.31193470954895
4.03589743589744 0.271461099386215
4.22820512820513 0.323495835065842
4.42820512820513 0.282878845930099
4.64102564102564 0.28743839263916
4.86153846153846 0.227854654192924
5.09230769230769 0.243989899754524
5.33589743589744 0.25860857963562
5.58974358974359 0.257818907499313
5.85641025641026 0.250191777944565
6.13589743589744 0.229697301983833
6.42564102564103 0.250173777341843
6.73333333333333 0.226392939686775
7.05384615384615 0.223544031381607
7.38974358974359 0.225553795695305
7.74102564102564 0.178182229399681
8.11025641025641 0.26079922914505
8.4974358974359 0.21364538371563
8.9 0.233213424682617
9.32564102564103 0.205759152770042
9.76923076923077 0.204782605171204
10.2333333333333 0.178935676813126
10.7205128205128 0.21323473751545
11.2307692307692 0.194919303059578
11.7666666666667 0.179332002997398
12.3282051282051 0.154195949435234
12.9153846153846 0.207958534359932
13.5282051282051 0.210385128855705
14.174358974359 0.188082292675972
14.8487179487179 0.178102448582649
15.5564102564103 0.181144878268242
16.2974358974359 0.153649255633354
17.0717948717949 0.156821474432945
17.8846153846154 0.143210932612419
18.7358974358974 0.143490344285965
19.6282051282051 0.113059304654598
20.5641025641026 0.115274667739868
21.5435897435897 0.128510102629662
22.5692307692308 0.130600348114967
23.6435897435897 0.116746261715889
24.7692307692308 0.171260073781013
25.9487179487179 0.10647114366293
27.1846153846154 0.124705985188484
28.4794871794872 0.112940095365047
29.8358974358974 0.111317336559296
31.2564102564103 0.122751250863075
32.7435897435897 0.141641348600388
34.3025641025641 0.135428011417389
35.9358974358974 0.12311464548111
37.648717948718 0.148271784186363
39.4410256410256 0.155179649591446
41.3179487179487 0.131509825587273
43.2871794871795 0.206141665577888
45.3487179487179 0.144996121525764
47.5076923076923 0.146132573485374
49.7692307692308 0.152700215578079
52.1384615384615 nan
54.6205128205128 0.153043434023857
57.2230769230769 nan
59.9461538461538 0.138972148299217
62.8025641025641 nan
65.7923076923077 0.13996933400631
68.925641025641 0.155394896864891
72.2076923076923 0.151966333389282
75.6461538461539 0.146564677357674
79.2461538461538 0.155147880315781
83.0205128205128 nan
86.974358974359 0.15384116768837
91.1153846153846 0.147026196122169
95.4538461538462 nan
100 nan
};
\addplot [, color2, opacity=0.6, mark=triangle*, mark size=0.5, mark options={solid,rotate=180}, only marks, forget plot]
table {%
1 0.615677535533905
1.04615384615385 0.575619161128998
1.0974358974359 0.481071799993515
1.14871794871795 0.469633162021637
1.2025641025641 0.468039602041245
1.26153846153846 0.437566012144089
1.32051282051282 0.521445333957672
1.38461538461538 0.511587560176849
1.44871794871795 0.477251678705215
1.51794871794872 0.476244449615479
1.58974358974359 0.419147312641144
1.66666666666667 0.455418974161148
1.74615384615385 0.443711191415787
1.82820512820513 0.450084924697876
1.91538461538462 0.472706884145737
2.00769230769231 0.423730373382568
2.1025641025641 0.465923935174942
2.20512820512821 0.412185430526733
2.30769230769231 0.396199136972427
2.41794871794872 0.37475848197937
2.53333333333333 0.357270151376724
2.65384615384615 0.404358625411987
2.78205128205128 0.365301489830017
2.91282051282051 0.372201383113861
3.05384615384615 0.362326920032501
3.1974358974359 0.35415330529213
3.35128205128205 0.362852334976196
3.51025641025641 0.397027939558029
3.67692307692308 0.395925343036652
3.85128205128205 0.377637445926666
4.03589743589744 0.372683495283127
4.22820512820513 0.374494522809982
4.42820512820513 0.358805686235428
4.64102564102564 0.366358131170273
4.86153846153846 0.333145380020142
5.09230769230769 0.310403317213058
5.33589743589744 0.282032310962677
5.58974358974359 0.334339708089828
5.85641025641026 0.307100266218185
6.13589743589744 0.34337118268013
6.42564102564103 0.308436006307602
6.73333333333333 0.280871242284775
7.05384615384615 0.299999564886093
7.38974358974359 0.26824277639389
7.74102564102564 0.250234067440033
8.11025641025641 0.251548767089844
8.4974358974359 0.265734285116196
8.9 0.245422348380089
9.32564102564103 0.246694207191467
9.76923076923077 0.221586942672729
10.2333333333333 0.244649603962898
10.7205128205128 0.253265917301178
11.2307692307692 0.225742861628532
11.7666666666667 0.21065691113472
12.3282051282051 0.222794130444527
12.9153846153846 0.210330486297607
13.5282051282051 0.220243722200394
14.174358974359 0.200260519981384
14.8487179487179 0.199104577302933
15.5564102564103 0.204837709665298
16.2974358974359 0.181041285395622
17.0717948717949 0.178633809089661
17.8846153846154 0.179618418216705
18.7358974358974 0.174616679549217
19.6282051282051 0.136612102389336
20.5641025641026 0.172676369547844
21.5435897435897 0.176127731800079
22.5692307692308 0.164758041501045
23.6435897435897 0.165222629904747
24.7692307692308 0.144084960222244
25.9487179487179 0.128045380115509
27.1846153846154 0.143569737672806
28.4794871794872 0.156573846936226
29.8358974358974 0.164780303835869
31.2564102564103 0.154991552233696
32.7435897435897 0.159140855073929
34.3025641025641 0.165725871920586
35.9358974358974 0.166457489132881
37.648717948718 0.138983845710754
39.4410256410256 0.155618071556091
41.3179487179487 0.150751128792763
43.2871794871795 0.145147487521172
45.3487179487179 0.156824707984924
47.5076923076923 0.153228878974915
49.7692307692308 0.147701069712639
52.1384615384615 0.151280328631401
54.6205128205128 0.148865565657616
57.2230769230769 0.154629334807396
59.9461538461538 0.151499673724174
62.8025641025641 0.157926365733147
65.7923076923077 0.133195430040359
68.925641025641 0.150156065821648
72.2076923076923 0.151872485876083
75.6461538461539 0.152401596307755
79.2461538461538 0.134366005659103
83.0205128205128 0.152773126959801
86.974358974359 0.153459548950195
91.1153846153846 0.153775840997696
95.4538461538462 0.14441804587841
100 0.147179841995239
};
\addplot [, color2, opacity=0.6, mark=triangle*, mark size=0.5, mark options={solid,rotate=180}, only marks, forget plot]
table {%
1 0.598997294902802
1.04615384615385 0.579464256763458
1.0974358974359 0.522017002105713
1.14871794871795 0.547317504882812
1.2025641025641 0.48540273308754
1.26153846153846 0.482969760894775
1.32051282051282 0.471073240041733
1.38461538461538 0.483331024646759
1.44871794871795 0.428619861602783
1.51794871794872 0.428117573261261
1.58974358974359 0.423566430807114
1.66666666666667 0.43544265627861
1.74615384615385 0.426893383264542
1.82820512820513 0.398061901330948
1.91538461538462 0.390319645404816
2.00769230769231 0.398380726575851
2.1025641025641 0.398483514785767
2.20512820512821 0.378004819154739
2.30769230769231 0.381623506546021
2.41794871794872 0.340520352125168
2.53333333333333 0.369493454694748
2.65384615384615 0.348771214485168
2.78205128205128 0.355172455310822
2.91282051282051 0.349037796258926
3.05384615384615 0.35663315653801
3.1974358974359 0.352074921131134
3.35128205128205 0.349082410335541
3.51025641025641 0.322955280542374
3.67692307692308 0.306901842355728
3.85128205128205 0.335221439599991
4.03589743589744 0.335403859615326
4.22820512820513 0.289544433355331
4.42820512820513 0.326517075300217
4.64102564102564 0.309474408626556
4.86153846153846 0.271066397428513
5.09230769230769 0.306891292333603
5.33589743589744 0.280282080173492
5.58974358974359 0.322211712598801
5.85641025641026 0.283241808414459
6.13589743589744 0.294103533029556
6.42564102564103 0.283559054136276
6.73333333333333 0.285782724618912
7.05384615384615 0.248952344059944
7.38974358974359 0.284105271100998
7.74102564102564 0.260426461696625
8.11025641025641 0.273803234100342
8.4974358974359 0.232144460082054
8.9 0.266088098287582
9.32564102564103 0.215588733553886
9.76923076923077 0.22667133808136
10.2333333333333 0.249775439500809
10.7205128205128 0.209087207913399
11.2307692307692 0.219331458210945
11.7666666666667 0.168273791670799
12.3282051282051 0.173602551221848
12.9153846153846 0.183285087347031
13.5282051282051 0.194999411702156
14.174358974359 0.173191770911217
14.8487179487179 0.216533899307251
15.5564102564103 0.182793170213699
16.2974358974359 0.1442901045084
17.0717948717949 0.148383110761642
17.8846153846154 0.159980639815331
18.7358974358974 0.118591539561749
19.6282051282051 0.117590427398682
20.5641025641026 0.114912271499634
21.5435897435897 0.134512975811958
22.5692307692308 0.116278529167175
23.6435897435897 0.0841376110911369
24.7692307692308 0.0958095118403435
25.9487179487179 0.118440322577953
27.1846153846154 0.105373077094555
28.4794871794872 0.11006011068821
29.8358974358974 0.152702316641808
31.2564102564103 0.154493078589439
32.7435897435897 0.168595910072327
34.3025641025641 0.121543765068054
35.9358974358974 0.168162956833839
37.648717948718 0.144966885447502
39.4410256410256 0.177102148532867
41.3179487179487 0.218352898955345
43.2871794871795 0.180778548121452
45.3487179487179 0.176139429211617
47.5076923076923 0.177709773182869
49.7692307692308 0.269155323505402
52.1384615384615 0.177435249090195
54.6205128205128 0.582909107208252
57.2230769230769 0.213609844446182
59.9461538461538 nan
62.8025641025641 nan
65.7923076923077 nan
68.925641025641 nan
72.2076923076923 0.152585551142693
75.6461538461539 0.184297829866409
79.2461538461538 0.108339883387089
83.0205128205128 0.165134772658348
86.974358974359 0.195962026715279
91.1153846153846 nan
95.4538461538462 0.164008229970932
100 nan
};
\end{axis}

\end{tikzpicture}

      \tikzexternaldisable
    \end{minipage}
  \end{subfigure}

  \begin{subfigure}[t]{\linewidth}
    \centering
    \caption{\fmnist \twoctwod \adam}
    \begin{minipage}{0.50\linewidth}
      \centering
      % defines the pgfplots style "eigspacedefault"
\pgfkeys{/pgfplots/eigspacedefault/.style={
    width=1.0\linewidth,
    height=0.6\linewidth,
    every axis plot/.append style={line width = 1.5pt},
    tick pos = left,
    ylabel near ticks,
    xlabel near ticks,
    xtick align = inside,
    ytick align = inside,
    legend cell align = left,
    legend columns = 4,
    legend pos = south east,
    legend style = {
      fill opacity = 1,
      text opacity = 1,
      font = \footnotesize,
      at={(1, 1.025)},
      anchor=south east,
      column sep=0.25cm,
    },
    legend image post style={scale=2.5},
    xticklabel style = {font = \footnotesize},
    xlabel style = {font = \footnotesize},
    axis line style = {black},
    yticklabel style = {font = \footnotesize},
    ylabel style = {font = \footnotesize},
    title style = {font = \footnotesize},
    grid = major,
    grid style = {dashed}
  }
}

\pgfkeys{/pgfplots/eigspacedefaultapp/.style={
    eigspacedefault,
    height=0.6\linewidth,
    legend columns = 2,
  }
}

\pgfkeys{/pgfplots/eigspacenolegend/.style={
    legend image post style = {scale=0},
    legend style = {
      fill opacity = 0,
      draw opacity = 0,
      text opacity = 0,
      font = \footnotesize,
      at={(1, 1.025)},
      anchor=south east,
      column sep=0.25cm,
    },
  }
}
%%% Local Variables:
%%% mode: latex
%%% TeX-master: "../../thesis"
%%% End:

      \pgfkeys{/pgfplots/zmystyle/.style={
          eigspacedefaultapp,
          eigspacenolegend,
        }}
      \tikzexternalenable
      \vspace{-6ex}
      % This file was created by tikzplotlib v0.9.7.
\begin{tikzpicture}

\definecolor{color0}{rgb}{0.501960784313725,0.184313725490196,0.6}
\definecolor{color1}{rgb}{0.870588235294118,0.623529411764706,0.0862745098039216}
\definecolor{color2}{rgb}{0.274509803921569,0.6,0.564705882352941}

\begin{axis}[
axis line style={white!10!black},
legend columns=2,
legend style={fill opacity=0.8, draw opacity=1, text opacity=1, at={(0.03,0.03)}, anchor=south west, draw=white!80!black},
log basis x={10},
tick pos=left,
xlabel={epoch (log scale)},
xmajorgrids,
xmin=0.794328234724281, xmax=125.892541179417,
xmode=log,
ylabel={overlap},
ymajorgrids,
ymin=-0.05, ymax=1.05,
zmystyle
]
\addplot [, white!10!black, dashed, forget plot]
table {%
0.794328234724281 1
125.892541179417 1
};
\addplot [, white!10!black, dashed, forget plot]
table {%
0.794328234724281 0
125.892541179417 0
};
\addplot [, color0, opacity=0.6, mark=triangle*, mark size=0.5, mark options={solid,rotate=180}, only marks]
table {%
1 0.808234393596649
1.04615384615385 0.814441680908203
1.0974358974359 0.784222722053528
1.14871794871795 0.771526396274567
1.2025641025641 0.718335092067719
1.26153846153846 0.746628105640411
1.32051282051282 0.739859223365784
1.38461538461538 0.74479216337204
1.44871794871795 0.684758186340332
1.51794871794872 0.72736781835556
1.58974358974359 0.72905296087265
1.66666666666667 0.717505395412445
1.74615384615385 0.661047279834747
1.82820512820513 0.650282561779022
1.91538461538462 0.645517289638519
2.00769230769231 0.645488560199738
2.1025641025641 0.645490348339081
2.20512820512821 0.655073165893555
2.30769230769231 0.611764788627625
2.41794871794872 0.635281026363373
2.53333333333333 0.634524285793304
2.65384615384615 0.617921113967896
2.78205128205128 0.619671523571014
2.91282051282051 0.601190388202667
3.05384615384615 0.607100129127502
3.1974358974359 0.60045713186264
3.35128205128205 0.583312809467316
3.51025641025641 0.591297745704651
3.67692307692308 0.583189427852631
3.85128205128205 0.583251297473907
4.03589743589744 0.564994096755981
4.22820512820513 0.579496324062347
4.42820512820513 0.556842565536499
4.64102564102564 0.557604491710663
4.86153846153846 0.526198387145996
5.09230769230769 0.540805757045746
5.33589743589744 0.555407643318176
5.58974358974359 0.554232537746429
5.85641025641026 0.543539047241211
6.13589743589744 0.535767495632172
6.42564102564103 0.527427673339844
6.73333333333333 0.523266732692719
7.05384615384615 0.503336131572723
7.38974358974359 0.516434669494629
7.74102564102564 0.505918681621552
8.11025641025641 0.504561364650726
8.4974358974359 0.486116230487823
8.9 0.503565013408661
9.32564102564103 0.480114549398422
9.76923076923077 0.473926156759262
10.2333333333333 0.478184133768082
10.7205128205128 0.452562242746353
11.2307692307692 0.466318994760513
11.7666666666667 0.465026468038559
12.3282051282051 0.459674030542374
12.9153846153846 0.443214982748032
13.5282051282051 0.426184892654419
14.174358974359 0.435368835926056
14.8487179487179 0.420685142278671
15.5564102564103 0.423262745141983
16.2974358974359 0.422932863235474
17.0717948717949 0.417033582925797
17.8846153846154 0.411274045705795
18.7358974358974 0.412772655487061
19.6282051282051 0.398178100585938
20.5641025641026 0.393288105726242
21.5435897435897 0.397180467844009
22.5692307692308 0.374125808477402
23.6435897435897 0.371305555105209
24.7692307692308 0.3786401450634
25.9487179487179 0.364118069410324
27.1846153846154 0.343166500329971
28.4794871794872 0.34480357170105
29.8358974358974 0.359858959913254
31.2564102564103 0.344883918762207
32.7435897435897 0.334812730550766
34.3025641025641 0.314797103404999
35.9358974358974 0.313951343297958
37.648717948718 0.312117397785187
39.4410256410256 0.285068809986115
41.3179487179487 0.29733020067215
43.2871794871795 0.285193651914597
45.3487179487179 0.269207030534744
47.5076923076923 0.271868467330933
49.7692307692308 0.259117543697357
52.1384615384615 0.268166393041611
54.6205128205128 0.241761490702629
57.2230769230769 0.244455441832542
59.9461538461538 0.232144474983215
62.8025641025641 0.2481429874897
65.7923076923077 0.233521461486816
68.925641025641 0.261493533849716
72.2076923076923 0.223874136805534
75.6461538461539 0.228239774703979
79.2461538461538 0.210948824882507
83.0205128205128 0.200878545641899
86.974358974359 0.227578803896904
91.1153846153846 0.238096117973328
95.4538461538462 0.218133479356766
100 0.254273593425751
};
\addlegendentry{mb 2, exact}
\addplot [, color0, opacity=0.6, mark=triangle*, mark size=0.5, mark options={solid,rotate=180}, only marks, forget plot]
table {%
1 0.79055780172348
1.04615384615385 0.835837006568909
1.0974358974359 0.774148404598236
1.14871794871795 0.758996903896332
1.2025641025641 0.734639585018158
1.26153846153846 0.726513087749481
1.32051282051282 0.724688708782196
1.38461538461538 0.714115142822266
1.44871794871795 0.683018863201141
1.51794871794872 0.711012661457062
1.58974358974359 0.705869138240814
1.66666666666667 0.678169250488281
1.74615384615385 0.67950314283371
1.82820512820513 0.656161665916443
1.91538461538462 0.63233894109726
2.00769230769231 0.647424697875977
2.1025641025641 0.647088825702667
2.20512820512821 0.640747606754303
2.30769230769231 0.620717942714691
2.41794871794872 0.623995006084442
2.53333333333333 0.611788392066956
2.65384615384615 0.603595554828644
2.78205128205128 0.610742807388306
2.91282051282051 0.599359095096588
3.05384615384615 0.597523868083954
3.1974358974359 0.583313584327698
3.35128205128205 0.569667458534241
3.51025641025641 0.578501045703888
3.67692307692308 0.571110725402832
3.85128205128205 0.561229348182678
4.03589743589744 0.554800927639008
4.22820512820513 0.557954430580139
4.42820512820513 0.527956485748291
4.64102564102564 0.52942556142807
4.86153846153846 0.523506104946136
5.09230769230769 0.529175937175751
5.33589743589744 0.522575378417969
5.58974358974359 0.522923111915588
5.85641025641026 0.515051484107971
6.13589743589744 0.508777618408203
6.42564102564103 0.502796173095703
6.73333333333333 0.489854425191879
7.05384615384615 0.492081165313721
7.38974358974359 0.498406171798706
7.74102564102564 0.485449314117432
8.11025641025641 0.47223025560379
8.4974358974359 0.478645950555801
8.9 0.486450672149658
9.32564102564103 0.451381355524063
9.76923076923077 0.455326080322266
10.2333333333333 0.462427139282227
10.7205128205128 0.450097769498825
11.2307692307692 0.448395937681198
11.7666666666667 0.448977679014206
12.3282051282051 0.449051678180695
12.9153846153846 0.435306787490845
13.5282051282051 0.430788040161133
14.174358974359 0.430222988128662
14.8487179487179 0.427929699420929
15.5564102564103 0.415283888578415
16.2974358974359 0.427997976541519
17.0717948717949 0.412310808897018
17.8846153846154 0.430178731679916
18.7358974358974 0.408814817667007
19.6282051282051 0.39656800031662
20.5641025641026 0.407884836196899
21.5435897435897 0.402655839920044
22.5692307692308 0.40256929397583
23.6435897435897 0.391357630491257
24.7692307692308 0.399322956800461
25.9487179487179 0.380915850400925
27.1846153846154 0.374530881643295
28.4794871794872 0.370508939027786
29.8358974358974 0.379287540912628
31.2564102564103 0.378953844308853
32.7435897435897 0.362848371267319
34.3025641025641 0.361208200454712
35.9358974358974 0.348764985799789
37.648717948718 0.354621261358261
39.4410256410256 0.319443076848984
41.3179487179487 0.33972293138504
43.2871794871795 0.319564431905746
45.3487179487179 0.317569881677628
47.5076923076923 0.314404547214508
49.7692307692308 0.329425394535065
52.1384615384615 0.289404600858688
54.6205128205128 0.301279753446579
57.2230769230769 0.294533401727676
59.9461538461538 0.280472278594971
62.8025641025641 0.270920962095261
65.7923076923077 0.276520222425461
68.925641025641 0.27612093091011
72.2076923076923 0.292937994003296
75.6461538461539 0.263089507818222
79.2461538461538 0.248909160494804
83.0205128205128 0.252015918493271
86.974358974359 0.282352685928345
91.1153846153846 0.270006328821182
95.4538461538462 0.277660846710205
100 0.287074536085129
};
\addplot [, color0, opacity=0.6, mark=triangle*, mark size=0.5, mark options={solid,rotate=180}, only marks, forget plot]
table {%
1 0.778681814670563
1.04615384615385 0.796188473701477
1.0974358974359 0.771596848964691
1.14871794871795 0.756519377231598
1.2025641025641 0.74484246969223
1.26153846153846 0.739646852016449
1.32051282051282 0.747567296028137
1.38461538461538 0.737904846668243
1.44871794871795 0.729647815227509
1.51794871794872 0.716303765773773
1.58974358974359 0.710456967353821
1.66666666666667 0.702198505401611
1.74615384615385 0.676780879497528
1.82820512820513 0.672272503376007
1.91538461538462 0.683867752552032
2.00769230769231 0.639134347438812
2.1025641025641 0.650525033473969
2.20512820512821 0.624929547309875
2.30769230769231 0.618377387523651
2.41794871794872 0.631933510303497
2.53333333333333 0.614215672016144
2.65384615384615 0.612018048763275
2.78205128205128 0.617526948451996
2.91282051282051 0.587545871734619
3.05384615384615 0.549250781536102
3.1974358974359 0.588209629058838
3.35128205128205 0.534993469715118
3.51025641025641 0.568743705749512
3.67692307692308 0.525312840938568
3.85128205128205 0.567443013191223
4.03589743589744 0.497439056634903
4.22820512820513 0.550377547740936
4.42820512820513 0.509293735027313
4.64102564102564 0.539196193218231
4.86153846153846 0.486315816640854
5.09230769230769 0.47906881570816
5.33589743589744 0.512502014636993
5.58974358974359 0.46828880906105
5.85641025641026 0.472108453512192
6.13589743589744 0.468376249074936
6.42564102564103 0.457832723855972
6.73333333333333 0.461091905832291
7.05384615384615 0.418823927640915
7.38974358974359 0.441470921039581
7.74102564102564 0.43730217218399
8.11025641025641 0.447497457265854
8.4974358974359 0.424672275781631
8.9 0.429768651723862
9.32564102564103 0.412982225418091
9.76923076923077 0.404866427183151
10.2333333333333 0.431028693914413
10.7205128205128 0.396504312753677
11.2307692307692 0.389548391103745
11.7666666666667 0.392287939786911
12.3282051282051 0.41322985291481
12.9153846153846 0.379399746656418
13.5282051282051 0.391152769327164
14.174358974359 0.382687479257584
14.8487179487179 0.377626717090607
15.5564102564103 0.379625648260117
16.2974358974359 0.387306541204453
17.0717948717949 0.366683065891266
17.8846153846154 0.363910347223282
18.7358974358974 0.361574947834015
19.6282051282051 0.32548114657402
20.5641025641026 0.347565442323685
21.5435897435897 0.346453994512558
22.5692307692308 0.34487196803093
23.6435897435897 0.340571373701096
24.7692307692308 0.334503352642059
25.9487179487179 0.3096062541008
27.1846153846154 0.317999660968781
28.4794871794872 0.29731872677803
29.8358974358974 0.305761307477951
31.2564102564103 0.292183250188828
32.7435897435897 0.283099144697189
34.3025641025641 0.291526943445206
35.9358974358974 0.279750317335129
37.648717948718 0.284726291894913
39.4410256410256 0.27138414978981
41.3179487179487 0.268549978733063
43.2871794871795 0.259110778570175
45.3487179487179 0.245779186487198
47.5076923076923 0.259223222732544
49.7692307692308 0.245217517018318
52.1384615384615 0.260645836591721
54.6205128205128 0.227887585759163
57.2230769230769 0.231268510222435
59.9461538461538 0.241674542427063
62.8025641025641 0.217485100030899
65.7923076923077 0.243862494826317
68.925641025641 0.232624083757401
72.2076923076923 0.230930358171463
75.6461538461539 0.244882300496101
79.2461538461538 0.244741678237915
83.0205128205128 0.219275519251823
86.974358974359 0.273074001073837
91.1153846153846 0.254893600940704
95.4538461538462 0.261643499135971
100 0.243339940905571
};
\addplot [, color0, opacity=0.6, mark=triangle*, mark size=0.5, mark options={solid,rotate=180}, only marks, forget plot]
table {%
1 0.852809906005859
1.04615384615385 0.902458190917969
1.0974358974359 0.84799861907959
1.14871794871795 0.840321481227875
1.2025641025641 0.832461953163147
1.26153846153846 0.836555480957031
1.32051282051282 0.840920567512512
1.38461538461538 0.838614881038666
1.44871794871795 0.802835643291473
1.51794871794872 0.817599892616272
1.58974358974359 0.809925973415375
1.66666666666667 0.812386155128479
1.74615384615385 0.800566017627716
1.82820512820513 0.787081182003021
1.91538461538462 0.785676181316376
2.00769230769231 0.771923005580902
2.1025641025641 0.780412495136261
2.20512820512821 0.766233444213867
2.30769230769231 0.779554963111877
2.41794871794872 0.753394663333893
2.53333333333333 0.729988992214203
2.65384615384615 0.748980939388275
2.78205128205128 0.750984489917755
2.91282051282051 0.728402554988861
3.05384615384615 0.723730683326721
3.1974358974359 0.74420839548111
3.35128205128205 0.700461685657501
3.51025641025641 0.709021985530853
3.67692307692308 0.689152598381042
3.85128205128205 0.696139752864838
4.03589743589744 0.669522702693939
4.22820512820513 0.677830457687378
4.42820512820513 0.664898037910461
4.64102564102564 0.667094230651855
4.86153846153846 0.662079989910126
5.09230769230769 0.647459506988525
5.33589743589744 0.656356751918793
5.58974358974359 0.638907730579376
5.85641025641026 0.644940555095673
6.13589743589744 0.641561329364777
6.42564102564103 0.625549077987671
6.73333333333333 0.633641064167023
7.05384615384615 0.59866189956665
7.38974358974359 0.616982460021973
7.74102564102564 0.606602132320404
8.11025641025641 0.597058296203613
8.4974358974359 0.583808839321136
8.9 0.584692537784576
9.32564102564103 0.56304943561554
9.76923076923077 0.575246453285217
10.2333333333333 0.580220699310303
10.7205128205128 0.555246531963348
11.2307692307692 0.559150695800781
11.7666666666667 0.559811413288116
12.3282051282051 0.555599927902222
12.9153846153846 0.536781489849091
13.5282051282051 0.538514733314514
14.174358974359 0.533681094646454
14.8487179487179 0.523644745349884
15.5564102564103 0.520727336406708
16.2974358974359 0.518002033233643
17.0717948717949 0.51174795627594
17.8846153846154 0.520687878131866
18.7358974358974 0.494105249643326
19.6282051282051 0.500032126903534
20.5641025641026 0.492022424936295
21.5435897435897 0.477710545063019
22.5692307692308 0.491215944290161
23.6435897435897 0.486072540283203
24.7692307692308 0.505723714828491
25.9487179487179 0.466185390949249
27.1846153846154 0.462150394916534
28.4794871794872 0.466278940439224
29.8358974358974 0.456505864858627
31.2564102564103 0.465964525938034
32.7435897435897 0.419994175434113
34.3025641025641 0.424814790487289
35.9358974358974 0.440754473209381
37.648717948718 0.433234214782715
39.4410256410256 0.413598358631134
41.3179487179487 0.416296780109406
43.2871794871795 0.402021139860153
45.3487179487179 0.385588258504868
47.5076923076923 0.413345903158188
49.7692307692308 0.412215769290924
52.1384615384615 0.36579829454422
54.6205128205128 0.367924630641937
57.2230769230769 0.354888528585434
59.9461538461538 0.359230428934097
62.8025641025641 0.368169546127319
65.7923076923077 0.348488420248032
68.925641025641 0.395449429750443
72.2076923076923 0.335189551115036
75.6461538461539 0.376108229160309
79.2461538461538 0.333802282810211
83.0205128205128 0.361840724945068
86.974358974359 0.37465438246727
91.1153846153846 0.37936145067215
95.4538461538462 0.340010106563568
100 0.347100824117661
};
\addplot [, color0, opacity=0.6, mark=triangle*, mark size=0.5, mark options={solid,rotate=180}, only marks, forget plot]
table {%
1 0.829403042793274
1.04615384615385 0.85082483291626
1.0974358974359 0.790084600448608
1.14871794871795 0.803588092327118
1.2025641025641 0.797669947147369
1.26153846153846 0.78161633014679
1.32051282051282 0.792710304260254
1.38461538461538 0.784815311431885
1.44871794871795 0.770520567893982
1.51794871794872 0.773061573505402
1.58974358974359 0.766344249248505
1.66666666666667 0.766078114509583
1.74615384615385 0.762807846069336
1.82820512820513 0.747422814369202
1.91538461538462 0.733741104602814
2.00769230769231 0.746840238571167
2.1025641025641 0.747446715831757
2.20512820512821 0.746390461921692
2.30769230769231 0.734932601451874
2.41794871794872 0.725378513336182
2.53333333333333 0.702951431274414
2.65384615384615 0.707222938537598
2.78205128205128 0.682768583297729
2.91282051282051 0.689033329486847
3.05384615384615 0.692261457443237
3.1974358974359 0.638579487800598
3.35128205128205 0.670956313610077
3.51025641025641 0.652121007442474
3.67692307692308 0.661463975906372
3.85128205128205 0.599360406398773
4.03589743589744 0.628019332885742
4.22820512820513 0.589810848236084
4.42820512820513 0.581816792488098
4.64102564102564 0.56853175163269
4.86153846153846 0.571031510829926
5.09230769230769 0.600275337696075
5.33589743589744 0.566527843475342
5.58974358974359 0.563900887966156
5.85641025641026 0.558314502239227
6.13589743589744 0.562927603721619
6.42564102564103 0.547851622104645
6.73333333333333 0.546609699726105
7.05384615384615 0.539586007595062
7.38974358974359 0.537352919578552
7.74102564102564 0.533601105213165
8.11025641025641 0.529511630535126
8.4974358974359 0.52748441696167
8.9 0.523904919624329
9.32564102564103 0.511582553386688
9.76923076923077 0.515571653842926
10.2333333333333 0.51347541809082
10.7205128205128 0.506551444530487
11.2307692307692 0.513745903968811
11.7666666666667 0.520946204662323
12.3282051282051 0.499731928110123
12.9153846153846 0.494951069355011
13.5282051282051 0.485607475042343
14.174358974359 0.495152801275253
14.8487179487179 0.490538418292999
15.5564102564103 0.510125339031219
16.2974358974359 0.470432728528976
17.0717948717949 0.499118536710739
17.8846153846154 0.466171562671661
18.7358974358974 0.45924636721611
19.6282051282051 0.482692718505859
20.5641025641026 0.483265489339828
21.5435897435897 0.467855930328369
22.5692307692308 0.458946049213409
23.6435897435897 0.464540958404541
24.7692307692308 0.465851783752441
25.9487179487179 0.446991175413132
27.1846153846154 0.418299347162247
28.4794871794872 0.446560770273209
29.8358974358974 0.436821430921555
31.2564102564103 0.41838151216507
32.7435897435897 0.417758852243423
34.3025641025641 0.417736530303955
35.9358974358974 0.423868864774704
37.648717948718 0.412972211837769
39.4410256410256 0.370976537466049
41.3179487179487 0.400528728961945
43.2871794871795 0.347640305757523
45.3487179487179 0.360057890415192
47.5076923076923 0.365131348371506
49.7692307692308 0.329356104135513
52.1384615384615 0.344886749982834
54.6205128205128 0.332347214221954
57.2230769230769 0.31575533747673
59.9461538461538 0.294059574604034
62.8025641025641 0.349527031183243
65.7923076923077 0.306697994470596
68.925641025641 0.3601094186306
72.2076923076923 0.307067394256592
75.6461538461539 0.294974476099014
79.2461538461538 0.273388475179672
83.0205128205128 0.290026158094406
86.974358974359 0.325860649347305
91.1153846153846 0.344316720962524
95.4538461538462 0.30957880616188
100 0.309582084417343
};
\addplot [, color1, opacity=0.6, mark=square*, mark size=0.5, mark options={solid}, only marks]
table {%
1 0.931219756603241
1.04615384615385 0.938552498817444
1.0974358974359 0.890907466411591
1.14871794871795 0.810727059841156
1.2025641025641 0.755685269832611
1.26153846153846 0.783188164234161
1.32051282051282 0.765978634357452
1.38461538461538 0.734972596168518
1.44871794871795 0.671775639057159
1.51794871794872 0.677355468273163
1.58974358974359 0.677028656005859
1.66666666666667 0.720286548137665
1.74615384615385 0.691295742988586
1.82820512820513 0.720942318439484
1.91538461538462 0.710050582885742
2.00769230769231 0.638265311717987
2.1025641025641 0.637081027030945
2.20512820512821 0.63757985830307
2.30769230769231 0.648163437843323
2.41794871794872 0.619206249713898
2.53333333333333 0.6226646900177
2.65384615384615 0.631678283214569
2.78205128205128 0.625747859477997
2.91282051282051 0.630714356899261
3.05384615384615 0.589415013790131
3.1974358974359 0.630532741546631
3.35128205128205 0.618497788906097
3.51025641025641 0.606923639774323
3.67692307692308 0.596537113189697
3.85128205128205 0.589339196681976
4.03589743589744 0.569254875183105
4.22820512820513 0.592426121234894
4.42820512820513 0.57182502746582
4.64102564102564 0.587085366249084
4.86153846153846 0.556819319725037
5.09230769230769 0.559150040149689
5.33589743589744 0.533205449581146
5.58974358974359 0.484764188528061
5.85641025641026 0.508373200893402
6.13589743589744 0.49961844086647
6.42564102564103 0.497666448354721
6.73333333333333 0.56260746717453
7.05384615384615 0.501350700855255
7.38974358974359 0.500495076179504
7.74102564102564 0.479002773761749
8.11025641025641 0.47927251458168
8.4974358974359 0.488924413919449
8.9 0.460226446390152
9.32564102564103 0.52442342042923
9.76923076923077 0.448273718357086
10.2333333333333 0.466817677021027
10.7205128205128 0.507460951805115
11.2307692307692 0.487673491239548
11.7666666666667 0.507846057415009
12.3282051282051 0.496849209070206
12.9153846153846 0.466641813516617
13.5282051282051 0.496458828449249
14.174358974359 0.469734907150269
14.8487179487179 0.52368289232254
15.5564102564103 0.46599555015564
16.2974358974359 0.481386482715607
17.0717948717949 0.479541033506393
17.8846153846154 0.443948358297348
18.7358974358974 0.466985285282135
19.6282051282051 0.463107019662857
20.5641025641026 0.465161472558975
21.5435897435897 0.480455547571182
22.5692307692308 0.455472469329834
23.6435897435897 0.455263614654541
24.7692307692308 0.448694795370102
25.9487179487179 0.430062592029572
27.1846153846154 0.427770584821701
28.4794871794872 0.414981603622437
29.8358974358974 0.397700041532516
31.2564102564103 0.395449787378311
32.7435897435897 0.35232949256897
34.3025641025641 0.334531843662262
35.9358974358974 0.356740683317184
37.648717948718 0.384150862693787
39.4410256410256 0.362026661634445
41.3179487179487 0.347302258014679
43.2871794871795 0.318587392568588
45.3487179487179 0.333518028259277
47.5076923076923 0.342090845108032
49.7692307692308 0.31196802854538
52.1384615384615 0.325693219900131
54.6205128205128 0.322067230939865
57.2230769230769 0.32719898223877
59.9461538461538 0.279565364122391
62.8025641025641 0.298547118902206
65.7923076923077 0.278484553098679
68.925641025641 0.287834376096725
72.2076923076923 0.279089033603668
75.6461538461539 0.300672829151154
79.2461538461538 0.291357815265656
83.0205128205128 0.288427919149399
86.974358974359 0.297085583209991
91.1153846153846 0.315802663564682
95.4538461538462 0.304337561130524
100 0.29423999786377
};
\addlegendentry{mb 8, exact}
\addplot [, color1, opacity=0.6, mark=square*, mark size=0.5, mark options={solid}, only marks, forget plot]
table {%
1 0.881440579891205
1.04615384615385 0.94662743806839
1.0974358974359 0.924689471721649
1.14871794871795 0.835634827613831
1.2025641025641 0.789496719837189
1.26153846153846 0.775716423988342
1.32051282051282 0.701003730297089
1.38461538461538 0.682543873786926
1.44871794871795 0.664491474628448
1.51794871794872 0.685344219207764
1.58974358974359 0.698144018650055
1.66666666666667 0.683916032314301
1.74615384615385 0.646517217159271
1.82820512820513 0.65081399679184
1.91538461538462 0.624968945980072
2.00769230769231 0.652618706226349
2.1025641025641 0.641045749187469
2.20512820512821 0.64350289106369
2.30769230769231 0.626960396766663
2.41794871794872 0.629917860031128
2.53333333333333 0.627935409545898
2.65384615384615 0.610577523708344
2.78205128205128 0.60753470659256
2.91282051282051 0.623652100563049
3.05384615384615 0.629461467266083
3.1974358974359 0.605173408985138
3.35128205128205 0.6130690574646
3.51025641025641 0.607473075389862
3.67692307692308 0.615136325359344
3.85128205128205 0.612684190273285
4.03589743589744 0.653011918067932
4.22820512820513 0.632123947143555
4.42820512820513 0.628488063812256
4.64102564102564 0.619285583496094
4.86153846153846 0.618396878242493
5.09230769230769 0.60612279176712
5.33589743589744 0.604854762554169
5.58974358974359 0.626900494098663
5.85641025641026 0.621904909610748
6.13589743589744 0.621305704116821
6.42564102564103 0.610524475574493
6.73333333333333 0.603407680988312
7.05384615384615 0.608661830425262
7.38974358974359 0.603932499885559
7.74102564102564 0.600690305233002
8.11025641025641 0.597454965114594
8.4974358974359 0.593428790569305
8.9 0.599718570709229
9.32564102564103 0.590509235858917
9.76923076923077 0.580784618854523
10.2333333333333 0.600134074687958
10.7205128205128 0.559286832809448
11.2307692307692 0.580739319324493
11.7666666666667 0.601877868175507
12.3282051282051 0.583369374275208
12.9153846153846 0.57044905424118
13.5282051282051 0.534032762050629
14.174358974359 0.535520493984222
14.8487179487179 0.505136966705322
15.5564102564103 0.531824588775635
16.2974358974359 0.513083100318909
17.0717948717949 0.538535237312317
17.8846153846154 0.548664569854736
18.7358974358974 0.498684018850327
19.6282051282051 0.496884435415268
20.5641025641026 0.525985896587372
21.5435897435897 0.459019035100937
22.5692307692308 0.463468462228775
23.6435897435897 0.462646543979645
24.7692307692308 0.46071720123291
25.9487179487179 0.43871545791626
27.1846153846154 0.413642257452011
28.4794871794872 0.426181226968765
29.8358974358974 0.459961146116257
31.2564102564103 0.421937227249146
32.7435897435897 0.437764644622803
34.3025641025641 0.44767752289772
35.9358974358974 0.422536462545395
37.648717948718 0.414428442716599
39.4410256410256 0.376765966415405
41.3179487179487 0.392683804035187
43.2871794871795 0.377142906188965
45.3487179487179 0.348110526800156
47.5076923076923 0.367960214614868
49.7692307692308 0.373824566602707
52.1384615384615 0.35050430893898
54.6205128205128 0.36563116312027
57.2230769230769 0.37675467133522
59.9461538461538 0.345403581857681
62.8025641025641 0.336667627096176
65.7923076923077 0.326251745223999
68.925641025641 0.34341761469841
72.2076923076923 0.312843292951584
75.6461538461539 0.320300042629242
79.2461538461538 0.315620571374893
83.0205128205128 0.322441071271896
86.974358974359 0.333726048469543
91.1153846153846 0.333687216043472
95.4538461538462 0.308729410171509
100 0.316134423017502
};
\addplot [, color1, opacity=0.6, mark=square*, mark size=0.5, mark options={solid}, only marks, forget plot]
table {%
1 0.894263863563538
1.04615384615385 0.959433376789093
1.0974358974359 0.937016308307648
1.14871794871795 0.861499607563019
1.2025641025641 0.848293244838715
1.26153846153846 0.828951954841614
1.32051282051282 0.821348667144775
1.38461538461538 0.818443477153778
1.44871794871795 0.730387687683105
1.51794871794872 0.826015174388885
1.58974358974359 0.794581830501556
1.66666666666667 0.751512050628662
1.74615384615385 0.738975167274475
1.82820512820513 0.694208562374115
1.91538461538462 0.72522759437561
2.00769230769231 0.681678235530853
2.1025641025641 0.728880822658539
2.20512820512821 0.668086469173431
2.30769230769231 0.705592811107635
2.41794871794872 0.652164340019226
2.53333333333333 0.660251617431641
2.65384615384615 0.715629398822784
2.78205128205128 0.638193547725677
2.91282051282051 0.677141785621643
3.05384615384615 0.663918495178223
3.1974358974359 0.689548134803772
3.35128205128205 0.590310037136078
3.51025641025641 0.636868834495544
3.67692307692308 0.647593915462494
3.85128205128205 0.625584721565247
4.03589743589744 0.584068477153778
4.22820512820513 0.621828973293304
4.42820512820513 0.641550540924072
4.64102564102564 0.653449356555939
4.86153846153846 0.618636071681976
5.09230769230769 0.60366815328598
5.33589743589744 0.605089366436005
5.58974358974359 0.601563394069672
5.85641025641026 0.61776202917099
6.13589743589744 0.608955979347229
6.42564102564103 0.596537888050079
6.73333333333333 0.610709488391876
7.05384615384615 0.585004866123199
7.38974358974359 0.592336118221283
7.74102564102564 0.577484607696533
8.11025641025641 0.553489029407501
8.4974358974359 0.576184451580048
8.9 0.55870509147644
9.32564102564103 0.561996459960938
9.76923076923077 0.526448845863342
10.2333333333333 0.568302631378174
10.7205128205128 0.51144278049469
11.2307692307692 0.527129828929901
11.7666666666667 0.515419065952301
12.3282051282051 0.508583009243011
12.9153846153846 0.500686407089233
13.5282051282051 0.527856826782227
14.174358974359 0.476732641458511
14.8487179487179 0.494505316019058
15.5564102564103 0.473263472318649
16.2974358974359 0.484562069177628
17.0717948717949 0.464240878820419
17.8846153846154 0.458909600973129
18.7358974358974 0.429908961057663
19.6282051282051 0.472223520278931
20.5641025641026 0.445273488759995
21.5435897435897 0.449259579181671
22.5692307692308 0.42881116271019
23.6435897435897 0.42021980881691
24.7692307692308 0.41767555475235
25.9487179487179 0.409188270568848
27.1846153846154 0.407317638397217
28.4794871794872 0.410273462533951
29.8358974358974 0.394507497549057
31.2564102564103 0.408607959747314
32.7435897435897 0.386587053537369
34.3025641025641 0.40024670958519
35.9358974358974 0.396598428487778
37.648717948718 0.382131308317184
39.4410256410256 0.371677041053772
41.3179487179487 0.404308557510376
43.2871794871795 0.373610019683838
45.3487179487179 0.367470920085907
47.5076923076923 0.380684852600098
49.7692307692308 0.327809125185013
52.1384615384615 0.3394955098629
54.6205128205128 0.344230741262436
57.2230769230769 0.32224914431572
59.9461538461538 0.324064761400223
62.8025641025641 0.346891343593597
65.7923076923077 0.327413707971573
68.925641025641 0.322406113147736
72.2076923076923 0.324845135211945
75.6461538461539 0.329490512609482
79.2461538461538 0.311690002679825
83.0205128205128 0.33866485953331
86.974358974359 0.316508293151855
91.1153846153846 0.313484609127045
95.4538461538462 0.344537228345871
100 0.314980447292328
};
\addplot [, color1, opacity=0.6, mark=square*, mark size=0.5, mark options={solid}, only marks, forget plot]
table {%
1 0.858411729335785
1.04615384615385 0.922185122966766
1.0974358974359 0.875420093536377
1.14871794871795 0.780671417713165
1.2025641025641 0.749508082866669
1.26153846153846 0.745001316070557
1.32051282051282 0.729262232780457
1.38461538461538 0.735662162303925
1.44871794871795 0.700793266296387
1.51794871794872 0.728383243083954
1.58974358974359 0.713000416755676
1.66666666666667 0.715414226055145
1.74615384615385 0.703270375728607
1.82820512820513 0.613368630409241
1.91538461538462 0.631699979305267
2.00769230769231 0.661597013473511
2.1025641025641 0.611375272274017
2.20512820512821 0.651740491390228
2.30769230769231 0.66800183057785
2.41794871794872 0.611727058887482
2.53333333333333 0.61631315946579
2.65384615384615 0.587713479995728
2.78205128205128 0.59673660993576
2.91282051282051 0.516008019447327
3.05384615384615 0.600525796413422
3.1974358974359 0.577473104000092
3.35128205128205 0.507554948329926
3.51025641025641 0.528603672981262
3.67692307692308 0.516451239585876
3.85128205128205 0.555493474006653
4.03589743589744 0.595083236694336
4.22820512820513 0.529721200466156
4.42820512820513 0.502953350543976
4.64102564102564 0.516888618469238
4.86153846153846 0.548647224903107
5.09230769230769 0.506748676300049
5.33589743589744 0.506345450878143
5.58974358974359 0.532819271087646
5.85641025641026 0.533593475818634
6.13589743589744 0.55156672000885
6.42564102564103 0.496413379907608
6.73333333333333 0.503746151924133
7.05384615384615 0.515898883342743
7.38974358974359 0.517915189266205
7.74102564102564 0.538012146949768
8.11025641025641 0.53562605381012
8.4974358974359 0.506936192512512
8.9 0.502864360809326
9.32564102564103 0.501699149608612
9.76923076923077 0.531750619411469
10.2333333333333 0.544679999351501
10.7205128205128 0.495446890592575
11.2307692307692 0.509904384613037
11.7666666666667 0.507467687129974
12.3282051282051 0.489986658096313
12.9153846153846 0.51212877035141
13.5282051282051 0.495780229568481
14.174358974359 0.495168894529343
14.8487179487179 0.499858945608139
15.5564102564103 0.479568213224411
16.2974358974359 0.481370836496353
17.0717948717949 0.505684077739716
17.8846153846154 0.509363055229187
18.7358974358974 0.466297715902328
19.6282051282051 0.467836141586304
20.5641025641026 0.466001331806183
21.5435897435897 0.465101540088654
22.5692307692308 0.455548018217087
23.6435897435897 0.467467457056046
24.7692307692308 0.450624197721481
25.9487179487179 0.434904813766479
27.1846153846154 0.426611006259918
28.4794871794872 0.464780062437057
29.8358974358974 0.425886362791061
31.2564102564103 0.439367979764938
32.7435897435897 0.417939901351929
34.3025641025641 0.424343794584274
35.9358974358974 0.427402317523956
37.648717948718 0.41078445315361
39.4410256410256 0.393739998340607
41.3179487179487 0.408158749341965
43.2871794871795 0.402427345514297
45.3487179487179 0.383394420146942
47.5076923076923 0.39565297961235
49.7692307692308 0.397716283798218
52.1384615384615 0.359107106924057
54.6205128205128 0.371755510568619
57.2230769230769 0.349292397499084
59.9461538461538 0.372075766324997
62.8025641025641 0.36311537027359
65.7923076923077 0.353772610425949
68.925641025641 0.374901860952377
72.2076923076923 0.354693353176117
75.6461538461539 0.343413829803467
79.2461538461538 0.339499861001968
83.0205128205128 0.357199400663376
86.974358974359 0.330552190542221
91.1153846153846 0.327753454446793
95.4538461538462 0.359686464071274
100 0.341812580823898
};
\addplot [, color1, opacity=0.6, mark=square*, mark size=0.5, mark options={solid}, only marks, forget plot]
table {%
1 0.879087865352631
1.04615384615385 0.955782532691956
1.0974358974359 0.921057343482971
1.14871794871795 0.86590713262558
1.2025641025641 0.842905819416046
1.26153846153846 0.842508733272552
1.32051282051282 0.834515750408173
1.38461538461538 0.825624465942383
1.44871794871795 0.813926637172699
1.51794871794872 0.81736433506012
1.58974358974359 0.819806694984436
1.66666666666667 0.806031405925751
1.74615384615385 0.787164866924286
1.82820512820513 0.7727952003479
1.91538461538462 0.772171974182129
2.00769230769231 0.800150573253632
2.1025641025641 0.769172191619873
2.20512820512821 0.777040660381317
2.30769230769231 0.73067182302475
2.41794871794872 0.76647424697876
2.53333333333333 0.782059848308563
2.65384615384615 0.741178035736084
2.78205128205128 0.714724361896515
2.91282051282051 0.732543170452118
3.05384615384615 0.75428718328476
3.1974358974359 0.701618194580078
3.35128205128205 0.683889210224152
3.51025641025641 0.718561947345734
3.67692307692308 0.725149750709534
3.85128205128205 0.689409196376801
4.03589743589744 0.701379239559174
4.22820512820513 0.696684181690216
4.42820512820513 0.634046792984009
4.64102564102564 0.64969265460968
4.86153846153846 0.666185557842255
5.09230769230769 0.685490310192108
5.33589743589744 0.674837291240692
5.58974358974359 0.678739011287689
5.85641025641026 0.668293416500092
6.13589743589744 0.632384121417999
6.42564102564103 0.63215845823288
6.73333333333333 0.622243106365204
7.05384615384615 0.616085648536682
7.38974358974359 0.650606334209442
7.74102564102564 0.649821102619171
8.11025641025641 0.638319671154022
8.4974358974359 0.606434464454651
8.9 0.642669796943665
9.32564102564103 0.580452442169189
9.76923076923077 0.611084580421448
10.2333333333333 0.562269330024719
10.7205128205128 0.606864809989929
11.2307692307692 0.602475762367249
11.7666666666667 0.583047866821289
12.3282051282051 0.607788681983948
12.9153846153846 0.593363285064697
13.5282051282051 0.533334493637085
14.174358974359 0.561380565166473
14.8487179487179 0.508560001850128
15.5564102564103 0.574156403541565
16.2974358974359 0.544704139232635
17.0717948717949 0.493126213550568
17.8846153846154 0.513526916503906
18.7358974358974 0.519072949886322
19.6282051282051 0.464987337589264
20.5641025641026 0.467118263244629
21.5435897435897 0.505775630474091
22.5692307692308 0.429682016372681
23.6435897435897 0.429888635873795
24.7692307692308 0.429671853780746
25.9487179487179 0.388179689645767
27.1846153846154 0.389951467514038
28.4794871794872 0.37622532248497
29.8358974358974 0.370806783437729
31.2564102564103 0.35928001999855
32.7435897435897 0.368641376495361
34.3025641025641 0.336508274078369
35.9358974358974 0.327075809240341
37.648717948718 0.358507722616196
39.4410256410256 0.312730133533478
41.3179487179487 0.332308143377304
43.2871794871795 0.315338104963303
45.3487179487179 0.306391894817352
47.5076923076923 0.29587659239769
49.7692307692308 0.246833130717278
52.1384615384615 0.267963320016861
54.6205128205128 0.248806551098824
57.2230769230769 0.261061042547226
59.9461538461538 0.263615369796753
62.8025641025641 0.270659476518631
65.7923076923077 0.229120180010796
68.925641025641 0.259430855512619
72.2076923076923 0.220604851841927
75.6461538461539 0.221078544855118
79.2461538461538 0.230034977197647
83.0205128205128 0.226181849837303
86.974358974359 0.22065606713295
91.1153846153846 0.202148541808128
95.4538461538462 0.256481885910034
100 0.227521568536758
};
\addplot [, color2, opacity=0.6, mark=diamond*, mark size=0.5, mark options={solid}, only marks]
table {%
1 0.943844974040985
1.04615384615385 0.971084773540497
1.0974358974359 0.967758595943451
1.14871794871795 0.955388247966766
1.2025641025641 0.945259511470795
1.26153846153846 0.938938736915588
1.32051282051282 0.9174684882164
1.38461538461538 0.919074237346649
1.44871794871795 0.824280083179474
1.51794871794872 0.884285151958466
1.58974358974359 0.870779156684875
1.66666666666667 0.851372838020325
1.74615384615385 0.818207561969757
1.82820512820513 0.820935845375061
1.91538461538462 0.810178220272064
2.00769230769231 0.7984619140625
2.1025641025641 0.808937191963196
2.20512820512821 0.781802952289581
2.30769230769231 0.808371961116791
2.41794871794872 0.751280784606934
2.53333333333333 0.721506774425507
2.65384615384615 0.722407519817352
2.78205128205128 0.725331127643585
2.91282051282051 0.765226244926453
3.05384615384615 0.774378955364227
3.1974358974359 0.671276986598969
3.35128205128205 0.703243553638458
3.51025641025641 0.6994269490242
3.67692307692308 0.641810953617096
3.85128205128205 0.639429807662964
4.03589743589744 0.706676602363586
4.22820512820513 0.671658337116241
4.42820512820513 0.66302216053009
4.64102564102564 0.608235836029053
4.86153846153846 0.63863343000412
5.09230769230769 0.676633775234222
5.33589743589744 0.589571297168732
5.58974358974359 0.658732891082764
5.85641025641026 0.657675266265869
6.13589743589744 0.627225458621979
6.42564102564103 0.643112897872925
6.73333333333333 0.597774624824524
7.05384615384615 0.573954939842224
7.38974358974359 0.555100560188293
7.74102564102564 0.6294264793396
8.11025641025641 0.577346503734589
8.4974358974359 0.593102753162384
8.9 0.582305729389191
9.32564102564103 0.538364946842194
9.76923076923077 0.584087550640106
10.2333333333333 0.530272424221039
10.7205128205128 0.502035558223724
11.2307692307692 0.602535426616669
11.7666666666667 0.529303431510925
12.3282051282051 0.495833963155746
12.9153846153846 0.546002089977264
13.5282051282051 0.529553890228271
14.174358974359 0.518087387084961
14.8487179487179 0.497969537973404
15.5564102564103 0.483196973800659
16.2974358974359 0.501104831695557
17.0717948717949 0.523100078105927
17.8846153846154 0.463704496622086
18.7358974358974 0.466202825307846
19.6282051282051 0.491770029067993
20.5641025641026 0.491607248783112
21.5435897435897 0.471135526895523
22.5692307692308 0.425479710102081
23.6435897435897 0.457247346639633
24.7692307692308 0.453832626342773
25.9487179487179 0.433128327131271
27.1846153846154 0.421893209218979
28.4794871794872 0.433476120233536
29.8358974358974 0.422991186380386
31.2564102564103 0.387902110815048
32.7435897435897 0.430641859769821
34.3025641025641 0.387364268302917
35.9358974358974 0.415891170501709
37.648717948718 0.382126063108444
39.4410256410256 0.405609846115112
41.3179487179487 0.410250008106232
43.2871794871795 0.383808434009552
45.3487179487179 0.389727473258972
47.5076923076923 0.376445919275284
49.7692307692308 0.393565863370895
52.1384615384615 0.370025396347046
54.6205128205128 0.392594903707504
57.2230769230769 0.392967313528061
59.9461538461538 0.339236944913864
62.8025641025641 0.34085088968277
65.7923076923077 0.366973161697388
68.925641025641 0.348763912916183
72.2076923076923 0.353554904460907
75.6461538461539 0.37717792391777
79.2461538461538 0.368026822805405
83.0205128205128 0.371374815702438
86.974358974359 0.376520097255707
91.1153846153846 0.369079262018204
95.4538461538462 0.362662643194199
100 0.361982882022858
};
\addlegendentry{mb 32, exact}
\addplot [, color2, opacity=0.6, mark=diamond*, mark size=0.5, mark options={solid}, only marks, forget plot]
table {%
1 0.974966704845428
1.04615384615385 0.986925899982452
1.0974358974359 0.982369244098663
1.14871794871795 0.969873130321503
1.2025641025641 0.954151272773743
1.26153846153846 0.952183902263641
1.32051282051282 0.923992097377777
1.38461538461538 0.896844685077667
1.44871794871795 0.897412419319153
1.51794871794872 0.859017193317413
1.58974358974359 0.865112483501434
1.66666666666667 0.847846925258636
1.74615384615385 0.832067608833313
1.82820512820513 0.803391635417938
1.91538461538462 0.830955684185028
2.00769230769231 0.820868790149689
2.1025641025641 0.762305557727814
2.20512820512821 0.762502253055573
2.30769230769231 0.766699910163879
2.41794871794872 0.825389862060547
2.53333333333333 0.776411890983582
2.65384615384615 0.740496754646301
2.78205128205128 0.762155532836914
2.91282051282051 0.717496037483215
3.05384615384615 0.7623011469841
3.1974358974359 0.714483678340912
3.35128205128205 0.730660617351532
3.51025641025641 0.710584938526154
3.67692307692308 0.723518311977386
3.85128205128205 0.704967319965363
4.03589743589744 0.749762952327728
4.22820512820513 0.705255389213562
4.42820512820513 0.718030631542206
4.64102564102564 0.722050786018372
4.86153846153846 0.717856228351593
5.09230769230769 0.715230882167816
5.33589743589744 0.724322378635406
5.58974358974359 0.718904793262482
5.85641025641026 0.709397971630096
6.13589743589744 0.719545304775238
6.42564102564103 0.701848208904266
6.73333333333333 0.707812368869781
7.05384615384615 0.699753701686859
7.38974358974359 0.639587342739105
7.74102564102564 0.663300454616547
8.11025641025641 0.629519999027252
8.4974358974359 0.680780827999115
8.9 0.67025226354599
9.32564102564103 0.658375203609467
9.76923076923077 0.609099686145782
10.2333333333333 0.662804782390594
10.7205128205128 0.641968190670013
11.2307692307692 0.611481845378876
11.7666666666667 0.619741559028625
12.3282051282051 0.602335274219513
12.9153846153846 0.502338349819183
13.5282051282051 0.558291852474213
14.174358974359 0.522624015808105
14.8487179487179 0.541693866252899
15.5564102564103 0.468200981616974
16.2974358974359 0.543175876140594
17.0717948717949 0.519932448863983
17.8846153846154 0.476579576730728
18.7358974358974 0.514367043972015
19.6282051282051 0.533517181873322
20.5641025641026 0.497596979141235
21.5435897435897 0.516900539398193
22.5692307692308 0.450045317411423
23.6435897435897 0.453379541635513
24.7692307692308 0.474255293607712
25.9487179487179 0.401887893676758
27.1846153846154 0.417249292135239
28.4794871794872 0.404639929533005
29.8358974358974 0.42485499382019
31.2564102564103 0.395195722579956
32.7435897435897 0.402237176895142
34.3025641025641 0.430076569318771
35.9358974358974 0.409534364938736
37.648717948718 0.40605965256691
39.4410256410256 0.42812642455101
41.3179487179487 0.432953268289566
43.2871794871795 0.40650463104248
45.3487179487179 0.406787604093552
47.5076923076923 0.383463948965073
49.7692307692308 0.423800200223923
52.1384615384615 0.414671510457993
54.6205128205128 0.422278791666031
57.2230769230769 0.418708473443985
59.9461538461538 0.380831152200699
62.8025641025641 0.409465044736862
65.7923076923077 0.392057657241821
68.925641025641 0.366047710180283
72.2076923076923 0.381768435239792
75.6461538461539 0.385424345731735
79.2461538461538 0.381890833377838
83.0205128205128 0.361380308866501
86.974358974359 0.381403893232346
91.1153846153846 0.387464106082916
95.4538461538462 0.365528881549835
100 0.384541600942612
};
\addplot [, color2, opacity=0.6, mark=diamond*, mark size=0.5, mark options={solid}, only marks, forget plot]
table {%
1 0.978291153907776
1.04615384615385 0.986281216144562
1.0974358974359 0.977453529834747
1.14871794871795 0.958560466766357
1.2025641025641 0.904912292957306
1.26153846153846 0.885174095630646
1.32051282051282 0.868678569793701
1.38461538461538 0.85496711730957
1.44871794871795 0.836746215820312
1.51794871794872 0.780582845211029
1.58974358974359 0.790078818798065
1.66666666666667 0.795597016811371
1.74615384615385 0.807866513729095
1.82820512820513 0.78180468082428
1.91538461538462 0.789427697658539
2.00769230769231 0.820644378662109
2.1025641025641 0.799163043498993
2.20512820512821 0.815995991230011
2.30769230769231 0.782320916652679
2.41794871794872 0.817291915416718
2.53333333333333 0.817368149757385
2.65384615384615 0.725181996822357
2.78205128205128 0.736786246299744
2.91282051282051 0.717587351799011
3.05384615384615 0.810422241687775
3.1974358974359 0.666558921337128
3.35128205128205 0.705657601356506
3.51025641025641 0.718513607978821
3.67692307692308 0.707189202308655
3.85128205128205 0.688362538814545
4.03589743589744 0.722273528575897
4.22820512820513 0.711739063262939
4.42820512820513 0.670523107051849
4.64102564102564 0.677993714809418
4.86153846153846 0.682985782623291
5.09230769230769 0.711651504039764
5.33589743589744 0.669672548770905
5.58974358974359 0.720462381839752
5.85641025641026 0.710029900074005
6.13589743589744 0.707176864147186
6.42564102564103 0.673777103424072
6.73333333333333 0.668092906475067
7.05384615384615 0.671148121356964
7.38974358974359 0.67328554391861
7.74102564102564 0.693152368068695
8.11025641025641 0.623264014720917
8.4974358974359 0.71677166223526
8.9 0.703075706958771
9.32564102564103 0.647290050983429
9.76923076923077 0.657804608345032
10.2333333333333 0.678853094577789
10.7205128205128 0.645462989807129
11.2307692307692 0.666936099529266
11.7666666666667 0.679803788661957
12.3282051282051 0.643230438232422
12.9153846153846 0.652233779430389
13.5282051282051 0.640233218669891
14.174358974359 0.637354373931885
14.8487179487179 0.623065292835236
15.5564102564103 0.589201152324677
16.2974358974359 0.639467358589172
17.0717948717949 0.618112981319427
17.8846153846154 0.625036299228668
18.7358974358974 0.585777103900909
19.6282051282051 0.582693338394165
20.5641025641026 0.564969003200531
21.5435897435897 0.558882117271423
22.5692307692308 0.548121869564056
23.6435897435897 0.572050094604492
24.7692307692308 0.554678738117218
25.9487179487179 0.527874171733856
27.1846153846154 0.521270215511322
28.4794871794872 0.525101482868195
29.8358974358974 0.51184493303299
31.2564102564103 0.536672830581665
32.7435897435897 0.485875904560089
34.3025641025641 0.502935349941254
35.9358974358974 0.475533872842789
37.648717948718 0.493133276700974
39.4410256410256 0.475215643644333
41.3179487179487 0.483727931976318
43.2871794871795 0.456807821989059
45.3487179487179 0.477699279785156
47.5076923076923 0.444924890995026
49.7692307692308 0.464029401540756
52.1384615384615 0.456087201833725
54.6205128205128 0.470060735940933
57.2230769230769 0.464056879281998
59.9461538461538 0.386837184429169
62.8025641025641 0.458399504423141
65.7923076923077 0.438194513320923
68.925641025641 0.455814450979233
72.2076923076923 0.450689136981964
75.6461538461539 0.42634579539299
79.2461538461538 0.406415849924088
83.0205128205128 0.413827151060104
86.974358974359 0.445946276187897
91.1153846153846 0.416579931974411
95.4538461538462 0.376824736595154
100 0.40446662902832
};
\addplot [, color2, opacity=0.6, mark=diamond*, mark size=0.5, mark options={solid}, only marks, forget plot]
table {%
1 0.974978148937225
1.04615384615385 0.987509071826935
1.0974358974359 0.979561626911163
1.14871794871795 0.971732318401337
1.2025641025641 0.964554309844971
1.26153846153846 0.965589344501495
1.32051282051282 0.949798047542572
1.38461538461538 0.945639252662659
1.44871794871795 0.879720330238342
1.51794871794872 0.876995742321014
1.58974358974359 0.878391683101654
1.66666666666667 0.845077216625214
1.74615384615385 0.886143147945404
1.82820512820513 0.847055733203888
1.91538461538462 0.826511383056641
2.00769230769231 0.827994823455811
2.1025641025641 0.81048721075058
2.20512820512821 0.853602051734924
2.30769230769231 0.801980197429657
2.41794871794872 0.811821281909943
2.53333333333333 0.815644264221191
2.65384615384615 0.792277932167053
2.78205128205128 0.779132485389709
2.91282051282051 0.796534836292267
3.05384615384615 0.823108851909637
3.1974358974359 0.737587094306946
3.35128205128205 0.785548686981201
3.51025641025641 0.780364632606506
3.67692307692308 0.796206653118134
3.85128205128205 0.733248949050903
4.03589743589744 0.816868126392365
4.22820512820513 0.781794726848602
4.42820512820513 0.752005100250244
4.64102564102564 0.736138224601746
4.86153846153846 0.752399146556854
5.09230769230769 0.756493508815765
5.33589743589744 0.670547246932983
5.58974358974359 0.731155633926392
5.85641025641026 0.719452083110809
6.13589743589744 0.695888698101044
6.42564102564103 0.676805913448334
6.73333333333333 0.670286357402802
7.05384615384615 0.658385872840881
7.38974358974359 0.623065292835236
7.74102564102564 0.685787618160248
8.11025641025641 0.634613513946533
8.4974358974359 0.650742828845978
8.9 0.669146239757538
9.32564102564103 0.664628803730011
9.76923076923077 0.618369281291962
10.2333333333333 0.628285527229309
10.7205128205128 0.62917947769165
11.2307692307692 0.590499103069305
11.7666666666667 0.643097102642059
12.3282051282051 0.588124692440033
12.9153846153846 0.592422962188721
13.5282051282051 0.595660865306854
14.174358974359 0.577473640441895
14.8487179487179 0.530136406421661
15.5564102564103 0.5153449177742
16.2974358974359 0.56602156162262
17.0717948717949 0.502659738063812
17.8846153846154 0.580907046794891
18.7358974358974 0.538406133651733
19.6282051282051 0.490233898162842
20.5641025641026 0.487264364957809
21.5435897435897 0.522262632846832
22.5692307692308 0.511374413967133
23.6435897435897 0.473469227552414
24.7692307692308 0.477207183837891
25.9487179487179 0.478228479623795
27.1846153846154 0.416349023580551
28.4794871794872 0.445866316556931
29.8358974358974 0.499762207269669
31.2564102564103 0.398218154907227
32.7435897435897 0.501901090145111
34.3025641025641 0.423102587461472
35.9358974358974 0.423115164041519
37.648717948718 0.420467674732208
39.4410256410256 0.405836433172226
41.3179487179487 0.442472606897354
43.2871794871795 0.42910099029541
45.3487179487179 0.401703923940659
47.5076923076923 0.384409576654434
49.7692307692308 0.393980503082275
52.1384615384615 0.402078598737717
54.6205128205128 0.383759886026382
57.2230769230769 0.395275264978409
59.9461538461538 0.415912300348282
62.8025641025641 0.370508164167404
65.7923076923077 0.377696603536606
68.925641025641 0.354013651609421
72.2076923076923 0.385656923055649
75.6461538461539 0.345911055803299
79.2461538461538 0.327401548624039
83.0205128205128 0.395484238862991
86.974358974359 0.363360643386841
91.1153846153846 0.338960617780685
95.4538461538462 0.36671182513237
100 0.342184722423553
};
\addplot [, color2, opacity=0.6, mark=diamond*, mark size=0.5, mark options={solid}, only marks, forget plot]
table {%
1 0.9708571434021
1.04615384615385 0.988922297954559
1.0974358974359 0.982923448085785
1.14871794871795 0.970705986022949
1.2025641025641 0.95971554517746
1.26153846153846 0.963769435882568
1.32051282051282 0.953489303588867
1.38461538461538 0.931393921375275
1.44871794871795 0.847586274147034
1.51794871794872 0.86174076795578
1.58974358974359 0.815381467342377
1.66666666666667 0.845152854919434
1.74615384615385 0.838447391986847
1.82820512820513 0.841055512428284
1.91538461538462 0.829017639160156
2.00769230769231 0.859940528869629
2.1025641025641 0.821684300899506
2.20512820512821 0.769925117492676
2.30769230769231 0.856572926044464
2.41794871794872 0.773224592208862
2.53333333333333 0.801861703395844
2.65384615384615 0.728741884231567
2.78205128205128 0.804679989814758
2.91282051282051 0.704732835292816
3.05384615384615 0.714294075965881
3.1974358974359 0.751589238643646
3.35128205128205 0.692211806774139
3.51025641025641 0.686902463436127
3.67692307692308 0.667775273323059
3.85128205128205 0.675200641155243
4.03589743589744 0.77333390712738
4.22820512820513 0.669722497463226
4.42820512820513 0.683193147182465
4.64102564102564 0.680682182312012
4.86153846153846 0.651868581771851
5.09230769230769 0.701175153255463
5.33589743589744 0.640350043773651
5.58974358974359 0.673046052455902
5.85641025641026 0.681954562664032
6.13589743589744 0.70149165391922
6.42564102564103 0.645831882953644
6.73333333333333 0.640626132488251
7.05384615384615 0.668260037899017
7.38974358974359 0.598867535591125
7.74102564102564 0.624545991420746
8.11025641025641 0.550290524959564
8.4974358974359 0.670787453651428
8.9 0.690401554107666
9.32564102564103 0.643333375453949
9.76923076923077 0.61131477355957
10.2333333333333 0.624626457691193
10.7205128205128 0.620091378688812
11.2307692307692 0.64387458562851
11.7666666666667 0.641688048839569
12.3282051282051 0.621330916881561
12.9153846153846 0.606647431850433
13.5282051282051 0.615661561489105
14.174358974359 0.578915536403656
14.8487179487179 0.612586677074432
15.5564102564103 0.581579983234406
16.2974358974359 0.578804612159729
17.0717948717949 0.608300089836121
17.8846153846154 0.563666641712189
18.7358974358974 0.563624978065491
19.6282051282051 0.569707691669464
20.5641025641026 0.589820265769958
21.5435897435897 0.57231616973877
22.5692307692308 0.526030719280243
23.6435897435897 0.536434471607208
24.7692307692308 0.511388540267944
25.9487179487179 0.532278835773468
27.1846153846154 0.511272549629211
28.4794871794872 0.504650831222534
29.8358974358974 0.505767524242401
31.2564102564103 0.484255999326706
32.7435897435897 0.509239017963409
34.3025641025641 0.479953497648239
35.9358974358974 0.524575233459473
37.648717948718 0.475012063980103
39.4410256410256 0.464483171701431
41.3179487179487 0.461944103240967
43.2871794871795 0.462664991617203
45.3487179487179 0.428922653198242
47.5076923076923 0.455129146575928
49.7692307692308 0.47074094414711
52.1384615384615 0.449475049972534
54.6205128205128 0.441178530454636
57.2230769230769 0.39079675078392
59.9461538461538 0.472016245126724
62.8025641025641 0.403713673353195
65.7923076923077 0.439165741205215
68.925641025641 0.419466257095337
72.2076923076923 0.426515489816666
75.6461538461539 0.405182808637619
79.2461538461538 0.414919912815094
83.0205128205128 0.392548322677612
86.974358974359 0.399612039327621
91.1153846153846 0.379395604133606
95.4538461538462 0.403797924518585
100 0.386635661125183
};
\addplot [, black, opacity=0.6, mark=*, mark size=0.5, mark options={solid}, only marks]
table {%
1 0.994028568267822
1.04615384615385 0.995532929897308
1.0974358974359 0.994038760662079
1.14871794871795 0.989708364009857
1.2025641025641 0.985826671123505
1.26153846153846 0.983290195465088
1.32051282051282 0.971213161945343
1.38461538461538 0.954515874385834
1.44871794871795 0.917149066925049
1.51794871794872 0.943449199199677
1.58974358974359 0.952398896217346
1.66666666666667 0.946468532085419
1.74615384615385 0.954476833343506
1.82820512820513 0.91742068529129
1.91538461538462 0.878035485744476
2.00769230769231 0.913759410381317
2.1025641025641 0.881194293498993
2.20512820512821 0.909050405025482
2.30769230769231 0.921557247638702
2.41794871794872 0.916325032711029
2.53333333333333 0.931946575641632
2.65384615384615 0.871881484985352
2.78205128205128 0.872420608997345
2.91282051282051 0.873710811138153
3.05384615384615 0.899538815021515
3.1974358974359 0.859496057033539
3.35128205128205 0.876919746398926
3.51025641025641 0.870399117469788
3.67692307692308 0.868935227394104
3.85128205128205 0.853279709815979
4.03589743589744 0.872560977935791
4.22820512820513 0.862492084503174
4.42820512820513 0.885929107666016
4.64102564102564 0.85664576292038
4.86153846153846 0.864387691020966
5.09230769230769 0.872125089168549
5.33589743589744 0.851608574390411
5.58974358974359 0.898318767547607
5.85641025641026 0.901290118694305
6.13589743589744 0.861707150936127
6.42564102564103 0.869136154651642
6.73333333333333 0.853763520717621
7.05384615384615 0.865958213806152
7.38974358974359 0.811963498592377
7.74102564102564 0.866204738616943
8.11025641025641 0.81992894411087
8.4974358974359 0.869678616523743
8.9 0.836935222148895
9.32564102564103 0.838403701782227
9.76923076923077 0.830814003944397
10.2333333333333 0.808399856090546
10.7205128205128 0.799348652362823
11.2307692307692 0.707185447216034
11.7666666666667 0.798606395721436
12.3282051282051 0.795615196228027
12.9153846153846 0.769974589347839
13.5282051282051 0.799137711524963
14.174358974359 0.732741832733154
14.8487179487179 0.714588701725006
15.5564102564103 0.710794746875763
16.2974358974359 0.759609222412109
17.0717948717949 0.718109428882599
17.8846153846154 0.727861106395721
18.7358974358974 0.721374452114105
19.6282051282051 0.680262744426727
20.5641025641026 0.607560455799103
21.5435897435897 0.687684893608093
22.5692307692308 0.592553317546844
23.6435897435897 0.617078900337219
24.7692307692308 0.642934739589691
25.9487179487179 0.608250737190247
27.1846153846154 0.602096199989319
28.4794871794872 0.611754238605499
29.8358974358974 0.5932297706604
31.2564102564103 0.536778628826141
32.7435897435897 0.565309226512909
34.3025641025641 0.583264768123627
35.9358974358974 0.561901032924652
37.648717948718 0.486840099096298
39.4410256410256 0.503694653511047
41.3179487179487 0.522434175014496
43.2871794871795 0.438699632883072
45.3487179487179 0.48332816362381
47.5076923076923 0.477622598409653
49.7692307692308 0.466665834188461
52.1384615384615 0.454295009374619
54.6205128205128 0.481028288602829
57.2230769230769 0.467037290334702
59.9461538461538 0.447650402784348
62.8025641025641 0.431758940219879
65.7923076923077 0.433361858129501
68.925641025641 0.420180141925812
72.2076923076923 0.383561998605728
75.6461538461539 0.403963953256607
79.2461538461538 0.433933794498444
83.0205128205128 0.414360195398331
86.974358974359 0.406529635190964
91.1153846153846 0.392766624689102
95.4538461538462 0.373715311288834
100 0.37746998667717
};
\addlegendentry{mb 128, exact}
\addplot [, black, opacity=0.6, mark=*, mark size=0.5, mark options={solid}, only marks, forget plot]
table {%
1 0.991706550121307
1.04615384615385 0.997022807598114
1.0974358974359 0.995591342449188
1.14871794871795 0.991665303707123
1.2025641025641 0.987361431121826
1.26153846153846 0.985481083393097
1.32051282051282 0.983776390552521
1.38461538461538 0.979556083679199
1.44871794871795 0.966664135456085
1.51794871794872 0.958603799343109
1.58974358974359 0.957813203334808
1.66666666666667 0.948358535766602
1.74615384615385 0.941102027893066
1.82820512820513 0.918793499469757
1.91538461538462 0.926381289958954
2.00769230769231 0.941750526428223
2.1025641025641 0.939816117286682
2.20512820512821 0.95298308134079
2.30769230769231 0.943270862102509
2.41794871794872 0.948424339294434
2.53333333333333 0.947727680206299
2.65384615384615 0.938095688819885
2.78205128205128 0.929375290870667
2.91282051282051 0.911316394805908
3.05384615384615 0.910702347755432
3.1974358974359 0.893549382686615
3.35128205128205 0.907097160816193
3.51025641025641 0.895476162433624
3.67692307692308 0.889096736907959
3.85128205128205 0.878571033477783
4.03589743589744 0.861399829387665
4.22820512820513 0.874711632728577
4.42820512820513 0.87043172121048
4.64102564102564 0.85120689868927
4.86153846153846 0.863661408424377
5.09230769230769 0.856592297554016
5.33589743589744 0.881087124347687
5.58974358974359 0.856142461299896
5.85641025641026 0.915935933589935
6.13589743589744 0.861045360565186
6.42564102564103 0.857509076595306
6.73333333333333 0.882289350032806
7.05384615384615 0.890490233898163
7.38974358974359 0.843625664710999
7.74102564102564 0.899800896644592
8.11025641025641 0.900576412677765
8.4974358974359 0.83291095495224
8.9 0.85979175567627
9.32564102564103 0.861006081104279
9.76923076923077 0.858018696308136
10.2333333333333 0.855575978755951
10.7205128205128 0.832142293453217
11.2307692307692 0.828982949256897
11.7666666666667 0.821417272090912
12.3282051282051 0.811684787273407
12.9153846153846 0.818069875240326
13.5282051282051 0.804020881652832
14.174358974359 0.755106568336487
14.8487179487179 0.775312840938568
15.5564102564103 0.765172600746155
16.2974358974359 0.741469025611877
17.0717948717949 0.744258999824524
17.8846153846154 0.783281803131104
18.7358974358974 0.749780595302582
19.6282051282051 0.737543523311615
20.5641025641026 0.723122358322144
21.5435897435897 0.726794242858887
22.5692307692308 0.660220384597778
23.6435897435897 0.656032025814056
24.7692307692308 0.683268010616302
25.9487179487179 0.706610321998596
27.1846153846154 0.669201493263245
28.4794871794872 0.723521411418915
29.8358974358974 0.691527962684631
31.2564102564103 0.546140193939209
32.7435897435897 0.604870438575745
34.3025641025641 0.570982575416565
35.9358974358974 0.628670036792755
37.648717948718 0.562267959117889
39.4410256410256 0.575533807277679
41.3179487179487 0.556773126125336
43.2871794871795 0.524787604808807
45.3487179487179 0.508279383182526
47.5076923076923 0.510513424873352
49.7692307692308 0.502310574054718
52.1384615384615 0.488432139158249
54.6205128205128 0.494626432657242
57.2230769230769 0.461378633975983
59.9461538461538 0.443095922470093
62.8025641025641 0.426553338766098
65.7923076923077 0.438673079013824
68.925641025641 0.417844444513321
72.2076923076923 0.398368179798126
75.6461538461539 0.448960274457932
79.2461538461538 0.406714916229248
83.0205128205128 0.387248367071152
86.974358974359 0.424945324659348
91.1153846153846 0.423113822937012
95.4538461538462 0.402803331613541
100 0.436538696289062
};
\addplot [, black, opacity=0.6, mark=*, mark size=0.5, mark options={solid}, only marks, forget plot]
table {%
1 0.992731511592865
1.04615384615385 0.996937453746796
1.0974358974359 0.99521940946579
1.14871794871795 0.991841316223145
1.2025641025641 0.988503456115723
1.26153846153846 0.987410724163055
1.32051282051282 0.982374131679535
1.38461538461538 0.980637967586517
1.44871794871795 0.926065385341644
1.51794871794872 0.938482582569122
1.58974358974359 0.941411018371582
1.66666666666667 0.948573768138885
1.74615384615385 0.964373230934143
1.82820512820513 0.96762627363205
1.91538461538462 0.968980431556702
2.00769230769231 0.973561763763428
2.1025641025641 0.97044175863266
2.20512820512821 0.962897300720215
2.30769230769231 0.967041194438934
2.41794871794872 0.964375674724579
2.53333333333333 0.963510632514954
2.65384615384615 0.95413464307785
2.78205128205128 0.956467926502228
2.91282051282051 0.946218967437744
3.05384615384615 0.953634440898895
3.1974358974359 0.883608639240265
3.35128205128205 0.929512798786163
3.51025641025641 0.937088668346405
3.67692307692308 0.921314239501953
3.85128205128205 0.881207585334778
4.03589743589744 0.928280651569366
4.22820512820513 0.917051732540131
4.42820512820513 0.911088407039642
4.64102564102564 0.899767518043518
4.86153846153846 0.906779110431671
5.09230769230769 0.912506997585297
5.33589743589744 0.907955348491669
5.58974358974359 0.916836440563202
5.85641025641026 0.907773673534393
6.13589743589744 0.907256722450256
6.42564102564103 0.904419422149658
6.73333333333333 0.903717041015625
7.05384615384615 0.900552928447723
7.38974358974359 0.900534570217133
7.74102564102564 0.903845310211182
8.11025641025641 0.906479835510254
8.4974358974359 0.870250225067139
8.9 0.885904729366302
9.32564102564103 0.882759511470795
9.76923076923077 0.857885539531708
10.2333333333333 0.842376887798309
10.7205128205128 0.854266464710236
11.2307692307692 0.846369206905365
11.7666666666667 0.849085986614227
12.3282051282051 0.818421483039856
12.9153846153846 0.836941182613373
13.5282051282051 0.781291902065277
14.174358974359 0.77932870388031
14.8487179487179 0.778231561183929
15.5564102564103 0.817240536212921
16.2974358974359 0.782917201519012
17.0717948717949 0.784587681293488
17.8846153846154 0.786919772624969
18.7358974358974 0.778592586517334
19.6282051282051 0.76498681306839
20.5641025641026 0.74370539188385
21.5435897435897 0.743267416954041
22.5692307692308 0.753884851932526
23.6435897435897 0.7350794672966
24.7692307692308 0.73886626958847
25.9487179487179 0.746078968048096
27.1846153846154 0.725281834602356
28.4794871794872 0.687323212623596
29.8358974358974 0.676326751708984
31.2564102564103 0.696645677089691
32.7435897435897 0.649566292762756
34.3025641025641 0.670870006084442
35.9358974358974 0.632062375545502
37.648717948718 0.638341069221497
39.4410256410256 0.608850419521332
41.3179487179487 0.602052807807922
43.2871794871795 0.588266789913177
45.3487179487179 0.592150628566742
47.5076923076923 0.533456921577454
49.7692307692308 0.568814337253571
52.1384615384615 0.542746484279633
54.6205128205128 0.523156583309174
57.2230769230769 0.533422946929932
59.9461538461538 0.469847679138184
62.8025641025641 0.497374445199966
65.7923076923077 0.516054809093475
68.925641025641 0.479160875082016
72.2076923076923 0.466017067432404
75.6461538461539 0.48551407456398
79.2461538461538 0.449299424886703
83.0205128205128 0.43969988822937
86.974358974359 0.448358535766602
91.1153846153846 0.475047647953033
95.4538461538462 0.474965631961823
100 0.45005851984024
};
\addplot [, black, opacity=0.6, mark=*, mark size=0.5, mark options={solid}, only marks, forget plot]
table {%
1 0.993377029895782
1.04615384615385 0.996295154094696
1.0974358974359 0.995041191577911
1.14871794871795 0.991120755672455
1.2025641025641 0.986871540546417
1.26153846153846 0.985466659069061
1.32051282051282 0.983344495296478
1.38461538461538 0.968218803405762
1.44871794871795 0.941327035427094
1.51794871794872 0.888970494270325
1.58974358974359 0.906764507293701
1.66666666666667 0.912162601947784
1.74615384615385 0.911395490169525
1.82820512820513 0.921511173248291
1.91538461538462 0.911652684211731
2.00769230769231 0.931042492389679
2.1025641025641 0.916641414165497
2.20512820512821 0.928853213787079
2.30769230769231 0.957143127918243
2.41794871794872 0.901934444904327
2.53333333333333 0.898184239864349
2.65384615384615 0.915748775005341
2.78205128205128 0.909198939800262
2.91282051282051 0.921803176403046
3.05384615384615 0.938034355640411
3.1974358974359 0.898952484130859
3.35128205128205 0.915366649627686
3.51025641025641 0.911019325256348
3.67692307692308 0.92304915189743
3.85128205128205 0.912662506103516
4.03589743589744 0.91961282491684
4.22820512820513 0.918498635292053
4.42820512820513 0.931189060211182
4.64102564102564 0.907854676246643
4.86153846153846 0.898814857006073
5.09230769230769 0.926773965358734
5.33589743589744 0.893811821937561
5.58974358974359 0.921594440937042
5.85641025641026 0.918610990047455
6.13589743589744 0.901191711425781
6.42564102564103 0.895778954029083
6.73333333333333 0.885952591896057
7.05384615384615 0.88738214969635
7.38974358974359 0.848188519477844
7.74102564102564 0.865455269813538
8.11025641025641 0.835410892963409
8.4974358974359 0.88863480091095
8.9 0.862096428871155
9.32564102564103 0.83354264497757
9.76923076923077 0.838313698768616
10.2333333333333 0.868589401245117
10.7205128205128 0.792487025260925
11.2307692307692 0.82845538854599
11.7666666666667 0.81723290681839
12.3282051282051 0.815092742443085
12.9153846153846 0.826624691486359
13.5282051282051 0.760144889354706
14.174358974359 0.811581075191498
14.8487179487179 0.771771013736725
15.5564102564103 0.788926005363464
16.2974358974359 0.774461925029755
17.0717948717949 0.759759068489075
17.8846153846154 0.789153814315796
18.7358974358974 0.715040385723114
19.6282051282051 0.726680874824524
20.5641025641026 0.707466721534729
21.5435897435897 0.702968537807465
22.5692307692308 0.647055745124817
23.6435897435897 0.661322712898254
24.7692307692308 0.65437525510788
25.9487179487179 0.601299524307251
27.1846153846154 0.631032466888428
28.4794871794872 0.620732605457306
29.8358974358974 0.59369307756424
31.2564102564103 0.605117201805115
32.7435897435897 0.540046334266663
34.3025641025641 0.553744852542877
35.9358974358974 0.513679146766663
37.648717948718 0.513655841350555
39.4410256410256 0.524393498897552
41.3179487179487 0.46834072470665
43.2871794871795 0.451448261737823
45.3487179487179 0.46662101149559
47.5076923076923 0.395513892173767
49.7692307692308 0.431278049945831
52.1384615384615 0.464940756559372
54.6205128205128 0.428158730268478
57.2230769230769 0.437923163175583
59.9461538461538 0.43643045425415
62.8025641025641 0.412132829427719
65.7923076923077 0.360656827688217
68.925641025641 0.407859951257706
72.2076923076923 0.359054952859879
75.6461538461539 0.389569967985153
79.2461538461538 0.326503723859787
83.0205128205128 0.344079583883286
86.974358974359 0.378715366125107
91.1153846153846 0.333244949579239
95.4538461538462 0.385921329259872
100 0.385322153568268
};
\addplot [, black, opacity=0.6, mark=*, mark size=0.5, mark options={solid}, only marks, forget plot]
table {%
1 0.98498409986496
1.04615384615385 0.995264947414398
1.0974358974359 0.995081722736359
1.14871794871795 0.991622567176819
1.2025641025641 0.989661335945129
1.26153846153846 0.988938748836517
1.32051282051282 0.985916912555695
1.38461538461538 0.9855917096138
1.44871794871795 0.979935824871063
1.51794871794872 0.969918429851532
1.58974358974359 0.966214835643768
1.66666666666667 0.962151527404785
1.74615384615385 0.959599912166595
1.82820512820513 0.947773635387421
1.91538461538462 0.94240015745163
2.00769230769231 0.964753329753876
2.1025641025641 0.955136895179749
2.20512820512821 0.93643993139267
2.30769230769231 0.95389711856842
2.41794871794872 0.94644957780838
2.53333333333333 0.94797271490097
2.65384615384615 0.869753658771515
2.78205128205128 0.865962624549866
2.91282051282051 0.871525704860687
3.05384615384615 0.90149062871933
3.1974358974359 0.869857788085938
3.35128205128205 0.921446800231934
3.51025641025641 0.889092087745667
3.67692307692308 0.899171769618988
3.85128205128205 0.852371692657471
4.03589743589744 0.916679978370667
4.22820512820513 0.89212954044342
4.42820512820513 0.917803585529327
4.64102564102564 0.866815030574799
4.86153846153846 0.901726365089417
5.09230769230769 0.920952916145325
5.33589743589744 0.904334723949432
5.58974358974359 0.919918656349182
5.85641025641026 0.917941093444824
6.13589743589744 0.832037568092346
6.42564102564103 0.833254456520081
6.73333333333333 0.905938625335693
7.05384615384615 0.830616295337677
7.38974358974359 0.891521096229553
7.74102564102564 0.900047779083252
8.11025641025641 0.870960056781769
8.4974358974359 0.833582818508148
8.9 0.862875759601593
9.32564102564103 0.833034157752991
9.76923076923077 0.864289283752441
10.2333333333333 0.818034172058105
10.7205128205128 0.842998623847961
11.2307692307692 0.866581916809082
11.7666666666667 0.870030224323273
12.3282051282051 0.800829589366913
12.9153846153846 0.811843693256378
13.5282051282051 0.816003799438477
14.174358974359 0.763382315635681
14.8487179487179 0.795010566711426
15.5564102564103 0.79793655872345
16.2974358974359 0.790882050991058
17.0717948717949 0.779683291912079
17.8846153846154 0.757333874702454
18.7358974358974 0.757314383983612
19.6282051282051 0.737408101558685
20.5641025641026 0.736309051513672
21.5435897435897 0.70807945728302
22.5692307692308 0.710383176803589
23.6435897435897 0.66688746213913
24.7692307692308 0.727880597114563
25.9487179487179 0.595376491546631
27.1846153846154 0.62880551815033
28.4794871794872 0.632704555988312
29.8358974358974 0.626671493053436
31.2564102564103 0.576481401920319
32.7435897435897 0.541564285755157
34.3025641025641 0.498997211456299
35.9358974358974 0.495587438344955
37.648717948718 0.496497631072998
39.4410256410256 0.505476593971252
41.3179487179487 0.474248707294464
43.2871794871795 0.49158450961113
45.3487179487179 0.474433273077011
47.5076923076923 0.455764681100845
49.7692307692308 0.516174972057343
52.1384615384615 0.460770815610886
54.6205128205128 0.432695299386978
57.2230769230769 0.411215603351593
59.9461538461538 0.440224885940552
62.8025641025641 0.431605488061905
65.7923076923077 0.418190628290176
68.925641025641 0.477815449237823
72.2076923076923 0.414630025625229
75.6461538461539 0.428045839071274
79.2461538461538 0.411519855260849
83.0205128205128 0.373872131109238
86.974358974359 0.405171245336533
91.1153846153846 0.419591248035431
95.4538461538462 0.393636614084244
100 0.41078644990921
};
\end{axis}

\end{tikzpicture}

      \tikzexternaldisable
    \end{minipage}\hfill
    \begin{minipage}{0.50\linewidth}
      \centering
      % defines the pgfplots style "eigspacedefault"
\pgfkeys{/pgfplots/eigspacedefault/.style={
    width=1.0\linewidth,
    height=0.6\linewidth,
    every axis plot/.append style={line width = 1.5pt},
    tick pos = left,
    ylabel near ticks,
    xlabel near ticks,
    xtick align = inside,
    ytick align = inside,
    legend cell align = left,
    legend columns = 4,
    legend pos = south east,
    legend style = {
      fill opacity = 1,
      text opacity = 1,
      font = \footnotesize,
      at={(1, 1.025)},
      anchor=south east,
      column sep=0.25cm,
    },
    legend image post style={scale=2.5},
    xticklabel style = {font = \footnotesize},
    xlabel style = {font = \footnotesize},
    axis line style = {black},
    yticklabel style = {font = \footnotesize},
    ylabel style = {font = \footnotesize},
    title style = {font = \footnotesize},
    grid = major,
    grid style = {dashed}
  }
}

\pgfkeys{/pgfplots/eigspacedefaultapp/.style={
    eigspacedefault,
    height=0.6\linewidth,
    legend columns = 2,
  }
}

\pgfkeys{/pgfplots/eigspacenolegend/.style={
    legend image post style = {scale=0},
    legend style = {
      fill opacity = 0,
      draw opacity = 0,
      text opacity = 0,
      font = \footnotesize,
      at={(1, 1.025)},
      anchor=south east,
      column sep=0.25cm,
    },
  }
}
%%% Local Variables:
%%% mode: latex
%%% TeX-master: "../../thesis"
%%% End:

      \pgfkeys{/pgfplots/zmystyle/.style={
          eigspacedefaultapp,
          eigspacenolegend,
        }}
      \tikzexternalenable
      \vspace{-6ex}
      % This file was created by tikzplotlib v0.9.7.
\begin{tikzpicture}

\definecolor{color0}{rgb}{0.274509803921569,0.6,0.564705882352941}
\definecolor{color1}{rgb}{0.870588235294118,0.623529411764706,0.0862745098039216}
\definecolor{color2}{rgb}{0.501960784313725,0.184313725490196,0.6}

\begin{axis}[
axis line style={white!10!black},
legend columns=2,
legend style={fill opacity=0.8, draw opacity=1, text opacity=1, at={(0.03,0.03)}, anchor=south west, draw=white!80!black},
log basis x={10},
tick pos=left,
xlabel={epoch (log scale)},
xmajorgrids,
xmin=0.794328234724281, xmax=125.892541179417,
xmode=log,
ylabel={overlap},
ymajorgrids,
ymin=-0.05, ymax=1.05,
zmystyle
]
\addplot [, white!10!black, dashed, forget plot]
table {%
0.794328234724281 1
125.892541179417 1
};
\addplot [, white!10!black, dashed, forget plot]
table {%
0.794328234724281 0
125.892541179417 0
};
\addplot [, black, opacity=0.6, mark=*, mark size=0.5, mark options={solid}, only marks]
table {%
1 0.994028568267822
1.04615384615385 0.995532929897308
1.0974358974359 0.994038760662079
1.14871794871795 0.989708364009857
1.2025641025641 0.985826671123505
1.26153846153846 0.983290195465088
1.32051282051282 0.971213161945343
1.38461538461538 0.954515874385834
1.44871794871795 0.917149066925049
1.51794871794872 0.943449199199677
1.58974358974359 0.952398896217346
1.66666666666667 0.946468532085419
1.74615384615385 0.954476833343506
1.82820512820513 0.91742068529129
1.91538461538462 0.878035485744476
2.00769230769231 0.913759410381317
2.1025641025641 0.881194293498993
2.20512820512821 0.909050405025482
2.30769230769231 0.921557247638702
2.41794871794872 0.916325032711029
2.53333333333333 0.931946575641632
2.65384615384615 0.871881484985352
2.78205128205128 0.872420608997345
2.91282051282051 0.873710811138153
3.05384615384615 0.899538815021515
3.1974358974359 0.859496057033539
3.35128205128205 0.876919746398926
3.51025641025641 0.870399117469788
3.67692307692308 0.868935227394104
3.85128205128205 0.853279709815979
4.03589743589744 0.872560977935791
4.22820512820513 0.862492084503174
4.42820512820513 0.885929107666016
4.64102564102564 0.85664576292038
4.86153846153846 0.864387691020966
5.09230769230769 0.872125089168549
5.33589743589744 0.851608574390411
5.58974358974359 0.898318767547607
5.85641025641026 0.901290118694305
6.13589743589744 0.861707150936127
6.42564102564103 0.869136154651642
6.73333333333333 0.853763520717621
7.05384615384615 0.865958213806152
7.38974358974359 0.811963498592377
7.74102564102564 0.866204738616943
8.11025641025641 0.81992894411087
8.4974358974359 0.869678616523743
8.9 0.836935222148895
9.32564102564103 0.838403701782227
9.76923076923077 0.830814003944397
10.2333333333333 0.808399856090546
10.7205128205128 0.799348652362823
11.2307692307692 0.707185447216034
11.7666666666667 0.798606395721436
12.3282051282051 0.795615196228027
12.9153846153846 0.769974589347839
13.5282051282051 0.799137711524963
14.174358974359 0.732741832733154
14.8487179487179 0.714588701725006
15.5564102564103 0.710794746875763
16.2974358974359 0.759609222412109
17.0717948717949 0.718109428882599
17.8846153846154 0.727861106395721
18.7358974358974 0.721374452114105
19.6282051282051 0.680262744426727
20.5641025641026 0.607560455799103
21.5435897435897 0.687684893608093
22.5692307692308 0.592553317546844
23.6435897435897 0.617078900337219
24.7692307692308 0.642934739589691
25.9487179487179 0.608250737190247
27.1846153846154 0.602096199989319
28.4794871794872 0.611754238605499
29.8358974358974 0.5932297706604
31.2564102564103 0.536778628826141
32.7435897435897 0.565309226512909
34.3025641025641 0.583264768123627
35.9358974358974 0.561901032924652
37.648717948718 0.486840099096298
39.4410256410256 0.503694653511047
41.3179487179487 0.522434175014496
43.2871794871795 0.438699632883072
45.3487179487179 0.48332816362381
47.5076923076923 0.477622598409653
49.7692307692308 0.466665834188461
52.1384615384615 0.454295009374619
54.6205128205128 0.481028288602829
57.2230769230769 0.467037290334702
59.9461538461538 0.447650402784348
62.8025641025641 0.431758940219879
65.7923076923077 0.433361858129501
68.925641025641 0.420180141925812
72.2076923076923 0.383561998605728
75.6461538461539 0.403963953256607
79.2461538461538 0.433933794498444
83.0205128205128 0.414360195398331
86.974358974359 0.406529635190964
91.1153846153846 0.392766624689102
95.4538461538462 0.373715311288834
100 0.37746998667717
};
\addlegendentry{mb 128, exact}
\addplot [, black, opacity=0.6, mark=*, mark size=0.5, mark options={solid}, only marks, forget plot]
table {%
1 0.991706550121307
1.04615384615385 0.997022807598114
1.0974358974359 0.995591342449188
1.14871794871795 0.991665303707123
1.2025641025641 0.987361431121826
1.26153846153846 0.985481083393097
1.32051282051282 0.983776390552521
1.38461538461538 0.979556083679199
1.44871794871795 0.966664135456085
1.51794871794872 0.958603799343109
1.58974358974359 0.957813203334808
1.66666666666667 0.948358535766602
1.74615384615385 0.941102027893066
1.82820512820513 0.918793499469757
1.91538461538462 0.926381289958954
2.00769230769231 0.941750526428223
2.1025641025641 0.939816117286682
2.20512820512821 0.95298308134079
2.30769230769231 0.943270862102509
2.41794871794872 0.948424339294434
2.53333333333333 0.947727680206299
2.65384615384615 0.938095688819885
2.78205128205128 0.929375290870667
2.91282051282051 0.911316394805908
3.05384615384615 0.910702347755432
3.1974358974359 0.893549382686615
3.35128205128205 0.907097160816193
3.51025641025641 0.895476162433624
3.67692307692308 0.889096736907959
3.85128205128205 0.878571033477783
4.03589743589744 0.861399829387665
4.22820512820513 0.874711632728577
4.42820512820513 0.87043172121048
4.64102564102564 0.85120689868927
4.86153846153846 0.863661408424377
5.09230769230769 0.856592297554016
5.33589743589744 0.881087124347687
5.58974358974359 0.856142461299896
5.85641025641026 0.915935933589935
6.13589743589744 0.861045360565186
6.42564102564103 0.857509076595306
6.73333333333333 0.882289350032806
7.05384615384615 0.890490233898163
7.38974358974359 0.843625664710999
7.74102564102564 0.899800896644592
8.11025641025641 0.900576412677765
8.4974358974359 0.83291095495224
8.9 0.85979175567627
9.32564102564103 0.861006081104279
9.76923076923077 0.858018696308136
10.2333333333333 0.855575978755951
10.7205128205128 0.832142293453217
11.2307692307692 0.828982949256897
11.7666666666667 0.821417272090912
12.3282051282051 0.811684787273407
12.9153846153846 0.818069875240326
13.5282051282051 0.804020881652832
14.174358974359 0.755106568336487
14.8487179487179 0.775312840938568
15.5564102564103 0.765172600746155
16.2974358974359 0.741469025611877
17.0717948717949 0.744258999824524
17.8846153846154 0.783281803131104
18.7358974358974 0.749780595302582
19.6282051282051 0.737543523311615
20.5641025641026 0.723122358322144
21.5435897435897 0.726794242858887
22.5692307692308 0.660220384597778
23.6435897435897 0.656032025814056
24.7692307692308 0.683268010616302
25.9487179487179 0.706610321998596
27.1846153846154 0.669201493263245
28.4794871794872 0.723521411418915
29.8358974358974 0.691527962684631
31.2564102564103 0.546140193939209
32.7435897435897 0.604870438575745
34.3025641025641 0.570982575416565
35.9358974358974 0.628670036792755
37.648717948718 0.562267959117889
39.4410256410256 0.575533807277679
41.3179487179487 0.556773126125336
43.2871794871795 0.524787604808807
45.3487179487179 0.508279383182526
47.5076923076923 0.510513424873352
49.7692307692308 0.502310574054718
52.1384615384615 0.488432139158249
54.6205128205128 0.494626432657242
57.2230769230769 0.461378633975983
59.9461538461538 0.443095922470093
62.8025641025641 0.426553338766098
65.7923076923077 0.438673079013824
68.925641025641 0.417844444513321
72.2076923076923 0.398368179798126
75.6461538461539 0.448960274457932
79.2461538461538 0.406714916229248
83.0205128205128 0.387248367071152
86.974358974359 0.424945324659348
91.1153846153846 0.423113822937012
95.4538461538462 0.402803331613541
100 0.436538696289062
};
\addplot [, black, opacity=0.6, mark=*, mark size=0.5, mark options={solid}, only marks, forget plot]
table {%
1 0.992731511592865
1.04615384615385 0.996937453746796
1.0974358974359 0.99521940946579
1.14871794871795 0.991841316223145
1.2025641025641 0.988503456115723
1.26153846153846 0.987410724163055
1.32051282051282 0.982374131679535
1.38461538461538 0.980637967586517
1.44871794871795 0.926065385341644
1.51794871794872 0.938482582569122
1.58974358974359 0.941411018371582
1.66666666666667 0.948573768138885
1.74615384615385 0.964373230934143
1.82820512820513 0.96762627363205
1.91538461538462 0.968980431556702
2.00769230769231 0.973561763763428
2.1025641025641 0.97044175863266
2.20512820512821 0.962897300720215
2.30769230769231 0.967041194438934
2.41794871794872 0.964375674724579
2.53333333333333 0.963510632514954
2.65384615384615 0.95413464307785
2.78205128205128 0.956467926502228
2.91282051282051 0.946218967437744
3.05384615384615 0.953634440898895
3.1974358974359 0.883608639240265
3.35128205128205 0.929512798786163
3.51025641025641 0.937088668346405
3.67692307692308 0.921314239501953
3.85128205128205 0.881207585334778
4.03589743589744 0.928280651569366
4.22820512820513 0.917051732540131
4.42820512820513 0.911088407039642
4.64102564102564 0.899767518043518
4.86153846153846 0.906779110431671
5.09230769230769 0.912506997585297
5.33589743589744 0.907955348491669
5.58974358974359 0.916836440563202
5.85641025641026 0.907773673534393
6.13589743589744 0.907256722450256
6.42564102564103 0.904419422149658
6.73333333333333 0.903717041015625
7.05384615384615 0.900552928447723
7.38974358974359 0.900534570217133
7.74102564102564 0.903845310211182
8.11025641025641 0.906479835510254
8.4974358974359 0.870250225067139
8.9 0.885904729366302
9.32564102564103 0.882759511470795
9.76923076923077 0.857885539531708
10.2333333333333 0.842376887798309
10.7205128205128 0.854266464710236
11.2307692307692 0.846369206905365
11.7666666666667 0.849085986614227
12.3282051282051 0.818421483039856
12.9153846153846 0.836941182613373
13.5282051282051 0.781291902065277
14.174358974359 0.77932870388031
14.8487179487179 0.778231561183929
15.5564102564103 0.817240536212921
16.2974358974359 0.782917201519012
17.0717948717949 0.784587681293488
17.8846153846154 0.786919772624969
18.7358974358974 0.778592586517334
19.6282051282051 0.76498681306839
20.5641025641026 0.74370539188385
21.5435897435897 0.743267416954041
22.5692307692308 0.753884851932526
23.6435897435897 0.7350794672966
24.7692307692308 0.73886626958847
25.9487179487179 0.746078968048096
27.1846153846154 0.725281834602356
28.4794871794872 0.687323212623596
29.8358974358974 0.676326751708984
31.2564102564103 0.696645677089691
32.7435897435897 0.649566292762756
34.3025641025641 0.670870006084442
35.9358974358974 0.632062375545502
37.648717948718 0.638341069221497
39.4410256410256 0.608850419521332
41.3179487179487 0.602052807807922
43.2871794871795 0.588266789913177
45.3487179487179 0.592150628566742
47.5076923076923 0.533456921577454
49.7692307692308 0.568814337253571
52.1384615384615 0.542746484279633
54.6205128205128 0.523156583309174
57.2230769230769 0.533422946929932
59.9461538461538 0.469847679138184
62.8025641025641 0.497374445199966
65.7923076923077 0.516054809093475
68.925641025641 0.479160875082016
72.2076923076923 0.466017067432404
75.6461538461539 0.48551407456398
79.2461538461538 0.449299424886703
83.0205128205128 0.43969988822937
86.974358974359 0.448358535766602
91.1153846153846 0.475047647953033
95.4538461538462 0.474965631961823
100 0.45005851984024
};
\addplot [, black, opacity=0.6, mark=*, mark size=0.5, mark options={solid}, only marks, forget plot]
table {%
1 0.993377029895782
1.04615384615385 0.996295154094696
1.0974358974359 0.995041191577911
1.14871794871795 0.991120755672455
1.2025641025641 0.986871540546417
1.26153846153846 0.985466659069061
1.32051282051282 0.983344495296478
1.38461538461538 0.968218803405762
1.44871794871795 0.941327035427094
1.51794871794872 0.888970494270325
1.58974358974359 0.906764507293701
1.66666666666667 0.912162601947784
1.74615384615385 0.911395490169525
1.82820512820513 0.921511173248291
1.91538461538462 0.911652684211731
2.00769230769231 0.931042492389679
2.1025641025641 0.916641414165497
2.20512820512821 0.928853213787079
2.30769230769231 0.957143127918243
2.41794871794872 0.901934444904327
2.53333333333333 0.898184239864349
2.65384615384615 0.915748775005341
2.78205128205128 0.909198939800262
2.91282051282051 0.921803176403046
3.05384615384615 0.938034355640411
3.1974358974359 0.898952484130859
3.35128205128205 0.915366649627686
3.51025641025641 0.911019325256348
3.67692307692308 0.92304915189743
3.85128205128205 0.912662506103516
4.03589743589744 0.91961282491684
4.22820512820513 0.918498635292053
4.42820512820513 0.931189060211182
4.64102564102564 0.907854676246643
4.86153846153846 0.898814857006073
5.09230769230769 0.926773965358734
5.33589743589744 0.893811821937561
5.58974358974359 0.921594440937042
5.85641025641026 0.918610990047455
6.13589743589744 0.901191711425781
6.42564102564103 0.895778954029083
6.73333333333333 0.885952591896057
7.05384615384615 0.88738214969635
7.38974358974359 0.848188519477844
7.74102564102564 0.865455269813538
8.11025641025641 0.835410892963409
8.4974358974359 0.88863480091095
8.9 0.862096428871155
9.32564102564103 0.83354264497757
9.76923076923077 0.838313698768616
10.2333333333333 0.868589401245117
10.7205128205128 0.792487025260925
11.2307692307692 0.82845538854599
11.7666666666667 0.81723290681839
12.3282051282051 0.815092742443085
12.9153846153846 0.826624691486359
13.5282051282051 0.760144889354706
14.174358974359 0.811581075191498
14.8487179487179 0.771771013736725
15.5564102564103 0.788926005363464
16.2974358974359 0.774461925029755
17.0717948717949 0.759759068489075
17.8846153846154 0.789153814315796
18.7358974358974 0.715040385723114
19.6282051282051 0.726680874824524
20.5641025641026 0.707466721534729
21.5435897435897 0.702968537807465
22.5692307692308 0.647055745124817
23.6435897435897 0.661322712898254
24.7692307692308 0.65437525510788
25.9487179487179 0.601299524307251
27.1846153846154 0.631032466888428
28.4794871794872 0.620732605457306
29.8358974358974 0.59369307756424
31.2564102564103 0.605117201805115
32.7435897435897 0.540046334266663
34.3025641025641 0.553744852542877
35.9358974358974 0.513679146766663
37.648717948718 0.513655841350555
39.4410256410256 0.524393498897552
41.3179487179487 0.46834072470665
43.2871794871795 0.451448261737823
45.3487179487179 0.46662101149559
47.5076923076923 0.395513892173767
49.7692307692308 0.431278049945831
52.1384615384615 0.464940756559372
54.6205128205128 0.428158730268478
57.2230769230769 0.437923163175583
59.9461538461538 0.43643045425415
62.8025641025641 0.412132829427719
65.7923076923077 0.360656827688217
68.925641025641 0.407859951257706
72.2076923076923 0.359054952859879
75.6461538461539 0.389569967985153
79.2461538461538 0.326503723859787
83.0205128205128 0.344079583883286
86.974358974359 0.378715366125107
91.1153846153846 0.333244949579239
95.4538461538462 0.385921329259872
100 0.385322153568268
};
\addplot [, black, opacity=0.6, mark=*, mark size=0.5, mark options={solid}, only marks, forget plot]
table {%
1 0.98498409986496
1.04615384615385 0.995264947414398
1.0974358974359 0.995081722736359
1.14871794871795 0.991622567176819
1.2025641025641 0.989661335945129
1.26153846153846 0.988938748836517
1.32051282051282 0.985916912555695
1.38461538461538 0.9855917096138
1.44871794871795 0.979935824871063
1.51794871794872 0.969918429851532
1.58974358974359 0.966214835643768
1.66666666666667 0.962151527404785
1.74615384615385 0.959599912166595
1.82820512820513 0.947773635387421
1.91538461538462 0.94240015745163
2.00769230769231 0.964753329753876
2.1025641025641 0.955136895179749
2.20512820512821 0.93643993139267
2.30769230769231 0.95389711856842
2.41794871794872 0.94644957780838
2.53333333333333 0.94797271490097
2.65384615384615 0.869753658771515
2.78205128205128 0.865962624549866
2.91282051282051 0.871525704860687
3.05384615384615 0.90149062871933
3.1974358974359 0.869857788085938
3.35128205128205 0.921446800231934
3.51025641025641 0.889092087745667
3.67692307692308 0.899171769618988
3.85128205128205 0.852371692657471
4.03589743589744 0.916679978370667
4.22820512820513 0.89212954044342
4.42820512820513 0.917803585529327
4.64102564102564 0.866815030574799
4.86153846153846 0.901726365089417
5.09230769230769 0.920952916145325
5.33589743589744 0.904334723949432
5.58974358974359 0.919918656349182
5.85641025641026 0.917941093444824
6.13589743589744 0.832037568092346
6.42564102564103 0.833254456520081
6.73333333333333 0.905938625335693
7.05384615384615 0.830616295337677
7.38974358974359 0.891521096229553
7.74102564102564 0.900047779083252
8.11025641025641 0.870960056781769
8.4974358974359 0.833582818508148
8.9 0.862875759601593
9.32564102564103 0.833034157752991
9.76923076923077 0.864289283752441
10.2333333333333 0.818034172058105
10.7205128205128 0.842998623847961
11.2307692307692 0.866581916809082
11.7666666666667 0.870030224323273
12.3282051282051 0.800829589366913
12.9153846153846 0.811843693256378
13.5282051282051 0.816003799438477
14.174358974359 0.763382315635681
14.8487179487179 0.795010566711426
15.5564102564103 0.79793655872345
16.2974358974359 0.790882050991058
17.0717948717949 0.779683291912079
17.8846153846154 0.757333874702454
18.7358974358974 0.757314383983612
19.6282051282051 0.737408101558685
20.5641025641026 0.736309051513672
21.5435897435897 0.70807945728302
22.5692307692308 0.710383176803589
23.6435897435897 0.66688746213913
24.7692307692308 0.727880597114563
25.9487179487179 0.595376491546631
27.1846153846154 0.62880551815033
28.4794871794872 0.632704555988312
29.8358974358974 0.626671493053436
31.2564102564103 0.576481401920319
32.7435897435897 0.541564285755157
34.3025641025641 0.498997211456299
35.9358974358974 0.495587438344955
37.648717948718 0.496497631072998
39.4410256410256 0.505476593971252
41.3179487179487 0.474248707294464
43.2871794871795 0.49158450961113
45.3487179487179 0.474433273077011
47.5076923076923 0.455764681100845
49.7692307692308 0.516174972057343
52.1384615384615 0.460770815610886
54.6205128205128 0.432695299386978
57.2230769230769 0.411215603351593
59.9461538461538 0.440224885940552
62.8025641025641 0.431605488061905
65.7923076923077 0.418190628290176
68.925641025641 0.477815449237823
72.2076923076923 0.414630025625229
75.6461538461539 0.428045839071274
79.2461538461538 0.411519855260849
83.0205128205128 0.373872131109238
86.974358974359 0.405171245336533
91.1153846153846 0.419591248035431
95.4538461538462 0.393636614084244
100 0.41078644990921
};
\addplot [, color0, opacity=0.6, mark=diamond*, mark size=0.5, mark options={solid}, only marks]
table {%
1 0.894292175769806
1.04615384615385 0.972469806671143
1.0974358974359 0.947438716888428
1.14871794871795 0.89987576007843
1.2025641025641 0.887459099292755
1.26153846153846 0.888091683387756
1.32051282051282 0.862513065338135
1.38461538461538 0.865094602108002
1.44871794871795 0.845361351966858
1.51794871794872 0.82983410358429
1.58974358974359 0.810975193977356
1.66666666666667 0.813776791095734
1.74615384615385 0.798110544681549
1.82820512820513 0.799910008907318
1.91538461538462 0.778843402862549
2.00769230769231 0.788788020610809
2.1025641025641 0.778457343578339
2.20512820512821 0.786975085735321
2.30769230769231 0.77516907453537
2.41794871794872 0.778852760791779
2.53333333333333 0.787159144878387
2.65384615384615 0.762507140636444
2.78205128205128 0.761295139789581
2.91282051282051 0.762674450874329
3.05384615384615 0.773209691047668
3.1974358974359 0.656807005405426
3.35128205128205 0.765810906887054
3.51025641025641 0.758271336555481
3.67692307692308 0.674658894538879
3.85128205128205 0.640068233013153
4.03589743589744 0.755290269851685
4.22820512820513 0.725640475749969
4.42820512820513 0.681573629379272
4.64102564102564 0.593750357627869
4.86153846153846 0.647236287593842
5.09230769230769 0.71817934513092
5.33589743589744 0.656009435653687
5.58974358974359 0.717175662517548
5.85641025641026 0.66793817281723
6.13589743589744 0.607986867427826
6.42564102564103 0.656503677368164
6.73333333333333 0.583166301250458
7.05384615384615 0.593563973903656
7.38974358974359 0.635855853557587
7.74102564102564 0.646994292736053
8.11025641025641 0.603814899921417
8.4974358974359 0.564028739929199
8.9 0.637327432632446
9.32564102564103 0.536960542201996
9.76923076923077 0.584807515144348
10.2333333333333 0.554325878620148
10.7205128205128 0.563835442066193
11.2307692307692 0.612780332565308
11.7666666666667 0.609938740730286
12.3282051282051 0.582379221916199
12.9153846153846 0.529603898525238
13.5282051282051 0.572533190250397
14.174358974359 0.512361705303192
14.8487179487179 0.520527958869934
15.5564102564103 0.514104664325714
16.2974358974359 0.517350316047668
17.0717948717949 0.52533483505249
17.8846153846154 0.51115471124649
18.7358974358974 0.507031440734863
19.6282051282051 0.486072838306427
20.5641025641026 0.52396297454834
21.5435897435897 0.545292437076569
22.5692307692308 0.478127628564835
23.6435897435897 0.46242219209671
24.7692307692308 0.483935445547104
25.9487179487179 0.471822649240494
27.1846153846154 0.473260968923569
28.4794871794872 0.482050001621246
29.8358974358974 0.435260772705078
31.2564102564103 0.43684133887291
32.7435897435897 0.417220413684845
34.3025641025641 0.442694425582886
35.9358974358974 0.4463230073452
37.648717948718 0.435208767652512
39.4410256410256 0.398578584194183
41.3179487179487 0.404708385467529
43.2871794871795 0.438088655471802
45.3487179487179 0.384614676237106
47.5076923076923 0.381035536527634
49.7692307692308 0.400491863489151
52.1384615384615 0.403228968381882
54.6205128205128 0.389372408390045
57.2230769230769 0.342048794031143
59.9461538461538 0.395254284143448
62.8025641025641 0.324947208166122
65.7923076923077 0.360273808240891
68.925641025641 0.346446961164474
72.2076923076923 0.32426443696022
75.6461538461539 0.34700933098793
79.2461538461538 0.332311928272247
83.0205128205128 0.322209089994431
86.974358974359 0.330642700195312
91.1153846153846 0.327917486429214
95.4538461538462 0.349909752607346
100 0.352484554052353
};
\addlegendentry{sub 16, exact}
\addplot [, color0, opacity=0.6, mark=diamond*, mark size=0.5, mark options={solid}, only marks, forget plot]
table {%
1 0.937871932983398
1.04615384615385 0.966511845588684
1.0974358974359 0.947273433208466
1.14871794871795 0.920068085193634
1.2025641025641 0.873619556427002
1.26153846153846 0.799409627914429
1.32051282051282 0.822388112545013
1.38461538461538 0.786675870418549
1.44871794871795 0.743740797042847
1.51794871794872 0.776160180568695
1.58974358974359 0.74654358625412
1.66666666666667 0.777979016304016
1.74615384615385 0.728194534778595
1.82820512820513 0.72405332326889
1.91538461538462 0.723877489566803
2.00769230769231 0.633299112319946
2.1025641025641 0.63270491361618
2.20512820512821 0.598059058189392
2.30769230769231 0.635112464427948
2.41794871794872 0.612896084785461
2.53333333333333 0.555071175098419
2.65384615384615 0.62331485748291
2.78205128205128 0.556564629077911
2.91282051282051 0.532693207263947
3.05384615384615 0.544160187244415
3.1974358974359 0.547952592372894
3.35128205128205 0.531652927398682
3.51025641025641 0.535271823406219
3.67692307692308 0.52444976568222
3.85128205128205 0.543479382991791
4.03589743589744 0.526860415935516
4.22820512820513 0.54000198841095
4.42820512820513 0.512355506420135
4.64102564102564 0.537785649299622
4.86153846153846 0.530326902866364
5.09230769230769 0.517478108406067
5.33589743589744 0.533158719539642
5.58974358974359 0.501226365566254
5.85641025641026 0.473785400390625
6.13589743589744 0.467469036579132
6.42564102564103 0.488970816135406
6.73333333333333 0.490879535675049
7.05384615384615 0.451627492904663
7.38974358974359 0.45061182975769
7.74102564102564 0.478557556867599
8.11025641025641 0.470948427915573
8.4974358974359 0.424247831106186
8.9 0.426872938871384
9.32564102564103 0.426707953214645
9.76923076923077 0.441901206970215
10.2333333333333 0.37063804268837
10.7205128205128 0.435557693243027
11.2307692307692 0.375869154930115
11.7666666666667 0.37189981341362
12.3282051282051 0.368880242109299
12.9153846153846 0.392696022987366
13.5282051282051 0.367002785205841
14.174358974359 0.361753404140472
14.8487179487179 0.369499057531357
15.5564102564103 0.375822305679321
16.2974358974359 0.375621765851974
17.0717948717949 0.346447855234146
17.8846153846154 0.35730853676796
18.7358974358974 0.357271432876587
19.6282051282051 0.351854771375656
20.5641025641026 0.357336670160294
21.5435897435897 0.377481937408447
22.5692307692308 0.350055515766144
23.6435897435897 0.356687039136887
24.7692307692308 0.335395336151123
25.9487179487179 0.359566181898117
27.1846153846154 0.349491715431213
28.4794871794872 0.348519295454025
29.8358974358974 0.352035224437714
31.2564102564103 0.327411890029907
32.7435897435897 0.334774941205978
34.3025641025641 0.330968588590622
35.9358974358974 0.311406403779984
37.648717948718 0.340340226888657
39.4410256410256 0.315250843763351
41.3179487179487 0.300972700119019
43.2871794871795 0.296130418777466
45.3487179487179 0.306997776031494
47.5076923076923 0.310792833566666
49.7692307692308 0.268420428037643
52.1384615384615 0.290537416934967
54.6205128205128 0.284612029790878
57.2230769230769 0.273978114128113
59.9461538461538 0.257259786128998
62.8025641025641 0.261825889348984
65.7923076923077 0.2614706158638
68.925641025641 0.268711894750595
72.2076923076923 0.258431196212769
75.6461538461539 0.264866322278976
79.2461538461538 0.251891702413559
83.0205128205128 0.264396727085114
86.974358974359 0.26787468791008
91.1153846153846 0.273233324289322
95.4538461538462 0.278928935527802
100 0.277541309595108
};
\addplot [, color0, opacity=0.6, mark=diamond*, mark size=0.5, mark options={solid}, only marks, forget plot]
table {%
1 0.960266411304474
1.04615384615385 0.978347957134247
1.0974358974359 0.966863751411438
1.14871794871795 0.945090770721436
1.2025641025641 0.91665917634964
1.26153846153846 0.89801561832428
1.32051282051282 0.849474430084229
1.38461538461538 0.909526169300079
1.44871794871795 0.7823725938797
1.51794871794872 0.836174011230469
1.58974358974359 0.804586410522461
1.66666666666667 0.794012486934662
1.74615384615385 0.731601238250732
1.82820512820513 0.731308341026306
1.91538461538462 0.718372344970703
2.00769230769231 0.704994380474091
2.1025641025641 0.696251749992371
2.20512820512821 0.708422541618347
2.30769230769231 0.75321352481842
2.41794871794872 0.70017009973526
2.53333333333333 0.696180284023285
2.65384615384615 0.701153457164764
2.78205128205128 0.681400299072266
2.91282051282051 0.712050139904022
3.05384615384615 0.697172403335571
3.1974358974359 0.706725776195526
3.35128205128205 0.694370210170746
3.51025641025641 0.699902534484863
3.67692307692308 0.685197174549103
3.85128205128205 0.693837642669678
4.03589743589744 0.68349277973175
4.22820512820513 0.66054505109787
4.42820512820513 0.679781436920166
4.64102564102564 0.713679730892181
4.86153846153846 0.709384918212891
5.09230769230769 0.675502419471741
5.33589743589744 0.669074654579163
5.58974358974359 0.681433856487274
5.85641025641026 0.687188923358917
6.13589743589744 0.712440133094788
6.42564102564103 0.660122871398926
6.73333333333333 0.671196281909943
7.05384615384615 0.62133115530014
7.38974358974359 0.656698167324066
7.74102564102564 0.639075040817261
8.11025641025641 0.631854653358459
8.4974358974359 0.590351700782776
8.9 0.627540588378906
9.32564102564103 0.610369861125946
9.76923076923077 0.609297633171082
10.2333333333333 0.65010267496109
10.7205128205128 0.556419551372528
11.2307692307692 0.557029128074646
11.7666666666667 0.58081316947937
12.3282051282051 0.555552184581757
12.9153846153846 0.579140603542328
13.5282051282051 0.556365787982941
14.174358974359 0.505998373031616
14.8487179487179 0.56506872177124
15.5564102564103 0.54519122838974
16.2974358974359 0.523997366428375
17.0717948717949 0.540237247943878
17.8846153846154 0.559704601764679
18.7358974358974 0.5171138048172
19.6282051282051 0.566868662834167
20.5641025641026 0.565583288669586
21.5435897435897 0.492735952138901
22.5692307692308 0.556258082389832
23.6435897435897 0.492953687906265
24.7692307692308 0.507158815860748
25.9487179487179 0.487246036529541
27.1846153846154 0.500939071178436
28.4794871794872 0.517161667346954
29.8358974358974 0.523092925548553
31.2564102564103 0.462834268808365
32.7435897435897 0.455427467823029
34.3025641025641 0.490421533584595
35.9358974358974 0.475949764251709
37.648717948718 0.477249145507812
39.4410256410256 0.463688850402832
41.3179487179487 0.477311432361603
43.2871794871795 0.436379909515381
45.3487179487179 0.469934672117233
47.5076923076923 0.479200273752213
49.7692307692308 0.449030607938766
52.1384615384615 0.478973120450974
54.6205128205128 0.450396537780762
57.2230769230769 0.44285175204277
59.9461538461538 0.386257261037827
62.8025641025641 0.425237953662872
65.7923076923077 0.434485197067261
68.925641025641 0.405945599079132
72.2076923076923 0.422116279602051
75.6461538461539 0.421222656965256
79.2461538461538 0.391128122806549
83.0205128205128 0.369596719741821
86.974358974359 0.422086536884308
91.1153846153846 0.394227117300034
95.4538461538462 0.347776859998703
100 0.400567382574081
};
\addplot [, color0, opacity=0.6, mark=diamond*, mark size=0.5, mark options={solid}, only marks, forget plot]
table {%
1 0.887853443622589
1.04615384615385 0.969360828399658
1.0974358974359 0.9516841173172
1.14871794871795 0.860351264476776
1.2025641025641 0.831176936626434
1.26153846153846 0.833877682685852
1.32051282051282 0.778636753559113
1.38461538461538 0.814163208007812
1.44871794871795 0.729141235351562
1.51794871794872 0.776322305202484
1.58974358974359 0.758827149868011
1.66666666666667 0.720944702625275
1.74615384615385 0.694678843021393
1.82820512820513 0.688821315765381
1.91538461538462 0.676903426647186
2.00769230769231 0.663655936717987
2.1025641025641 0.645662784576416
2.20512820512821 0.712193310260773
2.30769230769231 0.699776947498322
2.41794871794872 0.641402721405029
2.53333333333333 0.581464767456055
2.65384615384615 0.670631110668182
2.78205128205128 0.600662469863892
2.91282051282051 0.659061372280121
3.05384615384615 0.659701943397522
3.1974358974359 0.685934484004974
3.35128205128205 0.657721221446991
3.51025641025641 0.671900451183319
3.67692307692308 0.642916858196259
3.85128205128205 0.679837703704834
4.03589743589744 0.597101151943207
4.22820512820513 0.606660068035126
4.42820512820513 0.653286755084991
4.64102564102564 0.596137344837189
4.86153846153846 0.603598058223724
5.09230769230769 0.622271418571472
5.33589743589744 0.618937373161316
5.58974358974359 0.581950664520264
5.85641025641026 0.623457312583923
6.13589743589744 0.58910459280014
6.42564102564103 0.593604683876038
6.73333333333333 0.572100579738617
7.05384615384615 0.534694492816925
7.38974358974359 0.597541928291321
7.74102564102564 0.545645415782928
8.11025641025641 0.571467518806458
8.4974358974359 0.497472196817398
8.9 0.532867610454559
9.32564102564103 0.479271024465561
9.76923076923077 0.533183693885803
10.2333333333333 0.578067779541016
10.7205128205128 0.489571064710617
11.2307692307692 0.533953607082367
11.7666666666667 0.473247617483139
12.3282051282051 0.480268239974976
12.9153846153846 0.518171787261963
13.5282051282051 0.460941761732101
14.174358974359 0.470835655927658
14.8487179487179 0.494384735822678
15.5564102564103 0.512182176113129
16.2974358974359 0.489991009235382
17.0717948717949 0.483718723058701
17.8846153846154 0.481522619724274
18.7358974358974 0.439413756132126
19.6282051282051 0.477486342191696
20.5641025641026 0.444119274616241
21.5435897435897 0.471204429864883
22.5692307692308 0.431290715932846
23.6435897435897 0.458059787750244
24.7692307692308 0.451231777667999
25.9487179487179 0.463759511709213
27.1846153846154 0.428120344877243
28.4794871794872 0.455883979797363
29.8358974358974 0.386818826198578
31.2564102564103 0.389222532510757
32.7435897435897 0.392107218503952
34.3025641025641 0.414949625730515
35.9358974358974 0.405768483877182
37.648717948718 0.35748490691185
39.4410256410256 0.353134095668793
41.3179487179487 0.38958278298378
43.2871794871795 0.363987177610397
45.3487179487179 0.360392659902573
47.5076923076923 0.386482089757919
49.7692307692308 0.350660771131516
52.1384615384615 0.359409183263779
54.6205128205128 0.346349000930786
57.2230769230769 0.359323799610138
59.9461538461538 0.339452117681503
62.8025641025641 0.334946244955063
65.7923076923077 0.353606998920441
68.925641025641 0.329991787672043
72.2076923076923 0.315354079008102
75.6461538461539 0.317504793405533
79.2461538461538 0.313193053007126
83.0205128205128 0.293039858341217
86.974358974359 0.318567782640457
91.1153846153846 0.325044423341751
95.4538461538462 0.309735864400864
100 0.331141889095306
};
\addplot [, color0, opacity=0.6, mark=diamond*, mark size=0.5, mark options={solid}, only marks, forget plot]
table {%
1 0.919136822223663
1.04615384615385 0.956780076026917
1.0974358974359 0.934567451477051
1.14871794871795 0.90130341053009
1.2025641025641 0.853542923927307
1.26153846153846 0.896739184856415
1.32051282051282 0.792188286781311
1.38461538461538 0.831690490245819
1.44871794871795 0.707315921783447
1.51794871794872 0.76215785741806
1.58974358974359 0.75478583574295
1.66666666666667 0.689023673534393
1.74615384615385 0.63350385427475
1.82820512820513 0.651887893676758
1.91538461538462 0.690997242927551
2.00769230769231 0.632666051387787
2.1025641025641 0.598920822143555
2.20512820512821 0.586931943893433
2.30769230769231 0.617872357368469
2.41794871794872 0.574691355228424
2.53333333333333 0.570960164070129
2.65384615384615 0.601252496242523
2.78205128205128 0.609649300575256
2.91282051282051 0.53608238697052
3.05384615384615 0.535062432289124
3.1974358974359 0.616815745830536
3.35128205128205 0.507342636585236
3.51025641025641 0.528733730316162
3.67692307692308 0.530211269855499
3.85128205128205 0.539565563201904
4.03589743589744 0.515758335590363
4.22820512820513 0.550780415534973
4.42820512820513 0.560219049453735
4.64102564102564 0.566166341304779
4.86153846153846 0.578617513179779
5.09230769230769 0.522977769374847
5.33589743589744 0.540034770965576
5.58974358974359 0.497629731893539
5.85641025641026 0.535738885402679
6.13589743589744 0.545096814632416
6.42564102564103 0.490604311227798
6.73333333333333 0.541644394397736
7.05384615384615 0.469212770462036
7.38974358974359 0.520162880420685
7.74102564102564 0.491019070148468
8.11025641025641 0.505435466766357
8.4974358974359 0.438703387975693
8.9 0.42508265376091
9.32564102564103 0.482284367084503
9.76923076923077 0.472338497638702
10.2333333333333 0.549849689006805
10.7205128205128 0.460575670003891
11.2307692307692 0.484586328268051
11.7666666666667 0.437227815389633
12.3282051282051 0.465654283761978
12.9153846153846 0.467743396759033
13.5282051282051 0.480415105819702
14.174358974359 0.444572418928146
14.8487179487179 0.48501268029213
15.5564102564103 0.470740288496017
16.2974358974359 0.446141541004181
17.0717948717949 0.42813840508461
17.8846153846154 0.413984388113022
18.7358974358974 0.440965503454208
19.6282051282051 0.400839626789093
20.5641025641026 0.447611153125763
21.5435897435897 0.42215433716774
22.5692307692308 0.42366486787796
23.6435897435897 0.416759550571442
24.7692307692308 0.417089194059372
25.9487179487179 0.399985074996948
27.1846153846154 0.426470845937729
28.4794871794872 0.422306209802628
29.8358974358974 0.406944513320923
31.2564102564103 0.404196470975876
32.7435897435897 0.378771781921387
34.3025641025641 0.38513657450676
35.9358974358974 0.391237199306488
37.648717948718 0.380964249372482
39.4410256410256 0.390865623950958
41.3179487179487 0.397978037595749
43.2871794871795 0.393022865056992
45.3487179487179 0.393382519483566
47.5076923076923 0.412140846252441
49.7692307692308 0.398451328277588
52.1384615384615 0.371394693851471
54.6205128205128 0.403497517108917
57.2230769230769 0.389385432004929
59.9461538461538 0.398190975189209
62.8025641025641 0.398137658834457
65.7923076923077 0.391780763864517
68.925641025641 0.341281145811081
72.2076923076923 0.387008160352707
75.6461538461539 0.361102342605591
79.2461538461538 0.326677322387695
83.0205128205128 0.358928948640823
86.974358974359 0.362865030765533
91.1153846153846 0.362548261880875
95.4538461538462 0.341159552335739
100 0.352523386478424
};
\addplot [, color1, opacity=0.6, mark=square*, mark size=0.5, mark options={solid}, only marks]
table {%
1 0.939294278621674
1.04615384615385 0.977964997291565
1.0974358974359 0.94040435552597
1.14871794871795 0.920980930328369
1.2025641025641 0.907797336578369
1.26153846153846 0.896046280860901
1.32051282051282 0.831510841846466
1.38461538461538 0.802809655666351
1.44871794871795 0.892022430896759
1.51794871794872 0.716478407382965
1.58974358974359 0.801111042499542
1.66666666666667 0.711639702320099
1.74615384615385 0.742418646812439
1.82820512820513 0.764108002185822
1.91538461538462 0.738203346729279
2.00769230769231 0.802215993404388
2.1025641025641 0.751209437847137
2.20512820512821 0.77171516418457
2.30769230769231 0.723629176616669
2.41794871794872 0.798041820526123
2.53333333333333 0.774130463600159
2.65384615384615 0.798518598079681
2.78205128205128 0.72570925951004
2.91282051282051 0.750353634357452
3.05384615384615 0.788025557994843
3.1974358974359 0.711539506912231
3.35128205128205 0.78863537311554
3.51025641025641 0.738683760166168
3.67692307692308 0.733033180236816
3.85128205128205 0.698983609676361
4.03589743589744 0.717253744602203
4.22820512820513 0.667680740356445
4.42820512820513 0.704868972301483
4.64102564102564 0.677909374237061
4.86153846153846 0.653310418128967
5.09230769230769 0.673263549804688
5.33589743589744 0.664958357810974
5.58974358974359 0.740078270435333
5.85641025641026 0.67911970615387
6.13589743589744 0.675512969493866
6.42564102564103 0.709936678409576
6.73333333333333 0.673476219177246
7.05384615384615 0.602884232997894
7.38974358974359 0.708616197109222
7.74102564102564 0.641999900341034
8.11025641025641 0.691263198852539
8.4974358974359 0.646536886692047
8.9 0.687775731086731
9.32564102564103 0.593701541423798
9.76923076923077 0.637765347957611
10.2333333333333 0.580083250999451
10.7205128205128 0.575261414051056
11.2307692307692 0.569583415985107
11.7666666666667 0.554789483547211
12.3282051282051 0.599509537220001
12.9153846153846 0.550595760345459
13.5282051282051 0.599240243434906
14.174358974359 0.479354858398438
14.8487179487179 0.576224982738495
15.5564102564103 0.509478688240051
16.2974358974359 0.54613870382309
17.0717948717949 0.51023930311203
17.8846153846154 0.485695332288742
18.7358974358974 0.5328568816185
19.6282051282051 0.467842787504196
20.5641025641026 0.501362144947052
21.5435897435897 0.475559234619141
22.5692307692308 0.421992123126984
23.6435897435897 0.487018823623657
24.7692307692308 0.474580019712448
25.9487179487179 0.522583186626434
27.1846153846154 0.437956959009171
28.4794871794872 0.425327867269516
29.8358974358974 0.457215875387192
31.2564102564103 0.492900520563126
32.7435897435897 0.465334236621857
34.3025641025641 0.446900367736816
35.9358974358974 0.435136795043945
37.648717948718 0.406459152698517
39.4410256410256 0.427557945251465
41.3179487179487 0.423603683710098
43.2871794871795 0.446978956460953
45.3487179487179 0.408322066068649
47.5076923076923 0.412912458181381
49.7692307692308 0.351442098617554
52.1384615384615 0.414702415466309
54.6205128205128 0.366444438695908
57.2230769230769 0.356674909591675
59.9461538461538 0.396075695753098
62.8025641025641 0.429944336414337
65.7923076923077 0.433523505926132
68.925641025641 0.415269285440445
72.2076923076923 0.339855641126633
75.6461538461539 0.367901295423508
79.2461538461538 0.397590756416321
83.0205128205128 0.363113909959793
86.974358974359 0.414443105459213
91.1153846153846 0.377236694097519
95.4538461538462 0.420357525348663
100 0.423177063465118
};
\addlegendentry{mb 128, mc 1}
\addplot [, color1, opacity=0.6, mark=square*, mark size=0.5, mark options={solid}, only marks, forget plot]
table {%
1 0.947470963001251
1.04615384615385 0.966112911701202
1.0974358974359 0.918744087219238
1.14871794871795 0.876689076423645
1.2025641025641 0.817752301692963
1.26153846153846 0.83651202917099
1.32051282051282 0.840488076210022
1.38461538461538 0.839366257190704
1.44871794871795 0.859943211078644
1.51794871794872 0.758049964904785
1.58974358974359 0.78158575296402
1.66666666666667 0.780109524726868
1.74615384615385 0.78438526391983
1.82820512820513 0.809303402900696
1.91538461538462 0.800793349742889
2.00769230769231 0.792854428291321
2.1025641025641 0.777150273323059
2.20512820512821 0.751656711101532
2.30769230769231 0.736739158630371
2.41794871794872 0.798648416996002
2.53333333333333 0.738127648830414
2.65384615384615 0.751147210597992
2.78205128205128 0.751523673534393
2.91282051282051 0.762617766857147
3.05384615384615 0.6952223777771
3.1974358974359 0.699176967144012
3.35128205128205 0.706318974494934
3.51025641025641 0.732494056224823
3.67692307692308 0.734067440032959
3.85128205128205 0.757981717586517
4.03589743589744 0.700197398662567
4.22820512820513 0.679717540740967
4.42820512820513 0.719720304012299
4.64102564102564 0.697799026966095
4.86153846153846 0.685786247253418
5.09230769230769 0.689217567443848
5.33589743589744 0.729216277599335
5.58974358974359 0.687619090080261
5.85641025641026 0.692617177963257
6.13589743589744 0.681533932685852
6.42564102564103 0.664286255836487
6.73333333333333 0.710330307483673
7.05384615384615 0.69409191608429
7.38974358974359 0.671842098236084
7.74102564102564 0.672704458236694
8.11025641025641 0.640570104122162
8.4974358974359 0.597455143928528
8.9 0.619799554347992
9.32564102564103 0.587779641151428
9.76923076923077 0.658781766891479
10.2333333333333 0.604233264923096
10.7205128205128 0.625307857990265
11.2307692307692 0.622653126716614
11.7666666666667 0.546595215797424
12.3282051282051 0.496441841125488
12.9153846153846 0.580526053905487
13.5282051282051 0.569672584533691
14.174358974359 0.577690303325653
14.8487179487179 0.548286080360413
15.5564102564103 0.545208692550659
16.2974358974359 0.573623299598694
17.0717948717949 0.535030841827393
17.8846153846154 0.547889411449432
18.7358974358974 0.470319658517838
19.6282051282051 0.482513725757599
20.5641025641026 0.521117866039276
21.5435897435897 0.494018942117691
22.5692307692308 0.522136628627777
23.6435897435897 0.479879111051559
24.7692307692308 0.487193584442139
25.9487179487179 0.482185661792755
27.1846153846154 0.520685315132141
28.4794871794872 0.445722341537476
29.8358974358974 0.359223455190659
31.2564102564103 0.42126128077507
32.7435897435897 0.328919976949692
34.3025641025641 0.435489475727081
35.9358974358974 0.455002307891846
37.648717948718 0.391554176807404
39.4410256410256 0.367151230573654
41.3179487179487 0.424471467733383
43.2871794871795 0.394061088562012
45.3487179487179 0.309266567230225
47.5076923076923 0.419086664915085
49.7692307692308 0.369549244642258
52.1384615384615 0.334955304861069
54.6205128205128 0.325259745121002
57.2230769230769 0.380344092845917
59.9461538461538 0.314428120851517
62.8025641025641 0.350103110074997
65.7923076923077 0.312439680099487
68.925641025641 0.295607149600983
72.2076923076923 0.311895579099655
75.6461538461539 0.298463016748428
79.2461538461538 0.312000036239624
83.0205128205128 0.266155302524567
86.974358974359 0.323455572128296
91.1153846153846 0.297432869672775
95.4538461538462 0.341381669044495
100 0.322819143533707
};
\addplot [, color1, opacity=0.6, mark=square*, mark size=0.5, mark options={solid}, only marks, forget plot]
table {%
1 0.960217773914337
1.04615384615385 0.970201969146729
1.0974358974359 0.941101968288422
1.14871794871795 0.915100693702698
1.2025641025641 0.868541359901428
1.26153846153846 0.87178510427475
1.32051282051282 0.852006137371063
1.38461538461538 0.855007827281952
1.44871794871795 0.825238704681396
1.51794871794872 0.819959163665771
1.58974358974359 0.841568171977997
1.66666666666667 0.804981529712677
1.74615384615385 0.840112149715424
1.82820512820513 0.762884795665741
1.91538461538462 0.758971810340881
2.00769230769231 0.802488327026367
2.1025641025641 0.796353757381439
2.20512820512821 0.774918794631958
2.30769230769231 0.796801507472992
2.41794871794872 0.742369592189789
2.53333333333333 0.767824828624725
2.65384615384615 0.7176114320755
2.78205128205128 0.684917569160461
2.91282051282051 0.740081429481506
3.05384615384615 0.726754665374756
3.1974358974359 0.720325589179993
3.35128205128205 0.704555332660675
3.51025641025641 0.672992706298828
3.67692307692308 0.744775295257568
3.85128205128205 0.689330399036407
4.03589743589744 0.660865068435669
4.22820512820513 0.732797086238861
4.42820512820513 0.641087114810944
4.64102564102564 0.739834666252136
4.86153846153846 0.641162037849426
5.09230769230769 0.614982306957245
5.33589743589744 0.663978695869446
5.58974358974359 0.648840546607971
5.85641025641026 0.611065804958344
6.13589743589744 0.628268599510193
6.42564102564103 0.630123376846313
6.73333333333333 0.623091995716095
7.05384615384615 0.628239452838898
7.38974358974359 0.628620862960815
7.74102564102564 0.603569030761719
8.11025641025641 0.623159885406494
8.4974358974359 0.669319331645966
8.9 0.55152553319931
9.32564102564103 0.574156999588013
9.76923076923077 0.562036871910095
10.2333333333333 0.693737149238586
10.7205128205128 0.559950411319733
11.2307692307692 0.590838849544525
11.7666666666667 0.55485337972641
12.3282051282051 0.597092747688293
12.9153846153846 0.562725245952606
13.5282051282051 0.546250462532043
14.174358974359 0.533869206905365
14.8487179487179 0.560153186321259
15.5564102564103 0.547911107540131
16.2974358974359 0.523443400859833
17.0717948717949 0.498696058988571
17.8846153846154 0.472961634397507
18.7358974358974 0.555120646953583
19.6282051282051 0.505126774311066
20.5641025641026 0.488472074270248
21.5435897435897 0.451197624206543
22.5692307692308 0.492244094610214
23.6435897435897 0.482890039682388
24.7692307692308 0.431083649396896
25.9487179487179 0.438216894865036
27.1846153846154 0.463581085205078
28.4794871794872 0.409791052341461
29.8358974358974 0.378432899713516
31.2564102564103 0.392110794782639
32.7435897435897 0.372618645429611
34.3025641025641 0.361735820770264
35.9358974358974 0.350009679794312
37.648717948718 0.323416113853455
39.4410256410256 0.344369620084763
41.3179487179487 0.34957355260849
43.2871794871795 0.343703806400299
45.3487179487179 0.328064620494843
47.5076923076923 0.321782827377319
49.7692307692308 0.373643130064011
52.1384615384615 0.368402391672134
54.6205128205128 0.324808806180954
57.2230769230769 0.357163429260254
59.9461538461538 0.367937654256821
62.8025641025641 0.38791236281395
65.7923076923077 0.361591309309006
68.925641025641 0.337745666503906
72.2076923076923 0.350223362445831
75.6461538461539 0.340193957090378
79.2461538461538 0.321741163730621
83.0205128205128 0.326778590679169
86.974358974359 0.380625694990158
91.1153846153846 0.320085257291794
95.4538461538462 0.319973438978195
100 0.32417818903923
};
\addplot [, color1, opacity=0.6, mark=square*, mark size=0.5, mark options={solid}, only marks, forget plot]
table {%
1 0.907842338085175
1.04615384615385 0.968755543231964
1.0974358974359 0.930206477642059
1.14871794871795 0.873743534088135
1.2025641025641 0.838400065898895
1.26153846153846 0.831649303436279
1.32051282051282 0.84936648607254
1.38461538461538 0.873037159442902
1.44871794871795 0.827675640583038
1.51794871794872 0.752366006374359
1.58974358974359 0.771559417247772
1.66666666666667 0.759345710277557
1.74615384615385 0.757333695888519
1.82820512820513 0.747791111469269
1.91538461538462 0.748569905757904
2.00769230769231 0.781522333621979
2.1025641025641 0.753518760204315
2.20512820512821 0.736226737499237
2.30769230769231 0.73338919878006
2.41794871794872 0.761233031749725
2.53333333333333 0.746887624263763
2.65384615384615 0.68424379825592
2.78205128205128 0.7564896941185
2.91282051282051 0.687841355800629
3.05384615384615 0.715307772159576
3.1974358974359 0.669129312038422
3.35128205128205 0.72155898809433
3.51025641025641 0.701709568500519
3.67692307692308 0.669896304607391
3.85128205128205 0.67002010345459
4.03589743589744 0.763750553131104
4.22820512820513 0.712338387966156
4.42820512820513 0.61361825466156
4.64102564102564 0.66160923242569
4.86153846153846 0.68230277299881
5.09230769230769 0.696723759174347
5.33589743589744 0.607913970947266
5.58974358974359 0.641638815402985
5.85641025641026 0.703728139400482
6.13589743589744 0.672037303447723
6.42564102564103 0.560131669044495
6.73333333333333 0.604505896568298
7.05384615384615 0.610172092914581
7.38974358974359 0.568655133247375
7.74102564102564 0.642987430095673
8.11025641025641 0.543044328689575
8.4974358974359 0.542860150337219
8.9 0.645459353923798
9.32564102564103 0.665132462978363
9.76923076923077 0.543042361736298
10.2333333333333 0.531050145626068
10.7205128205128 0.550489485263824
11.2307692307692 0.550434708595276
11.7666666666667 0.509460210800171
12.3282051282051 0.545035898685455
12.9153846153846 0.535925984382629
13.5282051282051 0.62906402349472
14.174358974359 0.538611769676208
14.8487179487179 0.53367292881012
15.5564102564103 0.535244107246399
16.2974358974359 0.533724784851074
17.0717948717949 0.560788452625275
17.8846153846154 0.523629188537598
18.7358974358974 0.522252678871155
19.6282051282051 0.57112979888916
20.5641025641026 0.50555545091629
21.5435897435897 0.472986787557602
22.5692307692308 0.46057054400444
23.6435897435897 0.485085219144821
24.7692307692308 0.492561429738998
25.9487179487179 0.448657810688019
27.1846153846154 0.440075486898422
28.4794871794872 0.526698112487793
29.8358974358974 0.413478910923004
31.2564102564103 0.490939050912857
32.7435897435897 0.425613224506378
34.3025641025641 0.42024627327919
35.9358974358974 0.454601973295212
37.648717948718 0.440044164657593
39.4410256410256 0.395279169082642
41.3179487179487 0.377603799104691
43.2871794871795 0.371036261320114
45.3487179487179 0.393351674079895
47.5076923076923 0.401940017938614
49.7692307692308 0.364361196756363
52.1384615384615 0.398925513029099
54.6205128205128 0.387061089277267
57.2230769230769 0.427358448505402
59.9461538461538 0.353567570447922
62.8025641025641 0.408110916614532
65.7923076923077 0.367673844099045
68.925641025641 0.392758935689926
72.2076923076923 0.353232234716415
75.6461538461539 0.354623466730118
79.2461538461538 0.31494864821434
83.0205128205128 0.292269051074982
86.974358974359 0.366107136011124
91.1153846153846 0.298583000898361
95.4538461538462 0.361128807067871
100 0.294713586568832
};
\addplot [, color1, opacity=0.6, mark=square*, mark size=0.5, mark options={solid}, only marks, forget plot]
table {%
1 0.945137917995453
1.04615384615385 0.971043705940247
1.0974358974359 0.923480808734894
1.14871794871795 0.914975762367249
1.2025641025641 0.893874108791351
1.26153846153846 0.868954122066498
1.32051282051282 0.854865252971649
1.38461538461538 0.869256198406219
1.44871794871795 0.846488416194916
1.51794871794872 0.838169097900391
1.58974358974359 0.77527904510498
1.66666666666667 0.780908763408661
1.74615384615385 0.757070004940033
1.82820512820513 0.827676236629486
1.91538461538462 0.804859757423401
2.00769230769231 0.778474450111389
2.1025641025641 0.760150849819183
2.20512820512821 0.762512385845184
2.30769230769231 0.75742119550705
2.41794871794872 0.755813002586365
2.53333333333333 0.720780372619629
2.65384615384615 0.739524781703949
2.78205128205128 0.745750904083252
2.91282051282051 0.757349610328674
3.05384615384615 0.721915900707245
3.1974358974359 0.72061550617218
3.35128205128205 0.726920545101166
3.51025641025641 0.689169526100159
3.67692307692308 0.681348919868469
3.85128205128205 0.667631447315216
4.03589743589744 0.65816742181778
4.22820512820513 0.657005310058594
4.42820512820513 0.67544412612915
4.64102564102564 0.726266503334045
4.86153846153846 0.721706986427307
5.09230769230769 0.657370626926422
5.33589743589744 0.660770118236542
5.58974358974359 0.662311196327209
5.85641025641026 0.66662722826004
6.13589743589744 0.640422284603119
6.42564102564103 0.663360059261322
6.73333333333333 0.618137061595917
7.05384615384615 0.686238050460815
7.38974358974359 0.577328622341156
7.74102564102564 0.59385746717453
8.11025641025641 0.641297519207001
8.4974358974359 0.592081487178802
8.9 0.604330122470856
9.32564102564103 0.581207692623138
9.76923076923077 0.591372430324554
10.2333333333333 0.543565452098846
10.7205128205128 0.555243790149689
11.2307692307692 0.608644425868988
11.7666666666667 0.570199608802795
12.3282051282051 0.521475493907928
12.9153846153846 0.587798655033112
13.5282051282051 0.495764791965485
14.174358974359 0.529648542404175
14.8487179487179 0.536366403102875
15.5564102564103 0.533021926879883
16.2974358974359 0.460989773273468
17.0717948717949 0.538099944591522
17.8846153846154 0.428724497556686
18.7358974358974 0.496068865060806
19.6282051282051 0.482276499271393
20.5641025641026 0.445119291543961
21.5435897435897 0.485555231571198
22.5692307692308 0.415550321340561
23.6435897435897 0.419009268283844
24.7692307692308 0.435795366764069
25.9487179487179 0.443683475255966
27.1846153846154 0.472515791654587
28.4794871794872 0.456331938505173
29.8358974358974 0.379474818706512
31.2564102564103 0.462079614400864
32.7435897435897 0.398469656705856
34.3025641025641 0.382890552282333
35.9358974358974 0.43034440279007
37.648717948718 0.409101396799088
39.4410256410256 0.41038104891777
41.3179487179487 0.414515793323517
43.2871794871795 0.412768661975861
45.3487179487179 0.365672796964645
47.5076923076923 0.435240805149078
49.7692307692308 0.372149616479874
52.1384615384615 0.42024764418602
54.6205128205128 0.395211309194565
57.2230769230769 0.372151345014572
59.9461538461538 0.394231766462326
62.8025641025641 0.373547375202179
65.7923076923077 0.414023786783218
68.925641025641 0.402355581521988
72.2076923076923 0.376826733350754
75.6461538461539 0.387321382761002
79.2461538461538 0.382611244916916
83.0205128205128 0.352304458618164
86.974358974359 0.365824639797211
91.1153846153846 0.382898658514023
95.4538461538462 0.365406572818756
100 0.317458242177963
};
\addplot [, color2, opacity=0.6, mark=triangle*, mark size=0.5, mark options={solid,rotate=180}, only marks]
table {%
1 0.693727612495422
1.04615384615385 0.680780351161957
1.0974358974359 0.770604729652405
1.14871794871795 0.725186288356781
1.2025641025641 0.706702768802643
1.26153846153846 0.664879500865936
1.32051282051282 0.708377301692963
1.38461538461538 0.653382539749146
1.44871794871795 0.631286203861237
1.51794871794872 0.673494279384613
1.58974358974359 0.678348243236542
1.66666666666667 0.683851897716522
1.74615384615385 0.682752728462219
1.82820512820513 0.541607081890106
1.91538461538462 0.637619197368622
2.00769230769231 0.645008087158203
2.1025641025641 0.634111344814301
2.20512820512821 0.636279106140137
2.30769230769231 0.624882519245148
2.41794871794872 0.619108855724335
2.53333333333333 0.628581464290619
2.65384615384615 0.559481799602509
2.78205128205128 0.577151417732239
2.91282051282051 0.570199906826019
3.05384615384615 0.617224276065826
3.1974358974359 0.599003493785858
3.35128205128205 0.622608780860901
3.51025641025641 0.553388595581055
3.67692307692308 0.604640662670135
3.85128205128205 0.540587902069092
4.03589743589744 0.562505662441254
4.22820512820513 0.53203934431076
4.42820512820513 0.623155117034912
4.64102564102564 0.56870311498642
4.86153846153846 0.536044895648956
5.09230769230769 0.596262454986572
5.33589743589744 0.576530575752258
5.58974358974359 0.591780841350555
5.85641025641026 0.592126548290253
6.13589743589744 0.555578112602234
6.42564102564103 0.564535558223724
6.73333333333333 0.569076538085938
7.05384615384615 0.58214008808136
7.38974358974359 0.551502704620361
7.74102564102564 0.54630321264267
8.11025641025641 0.566515147686005
8.4974358974359 0.56314218044281
8.9 0.534356594085693
9.32564102564103 0.530383706092834
9.76923076923077 0.528336822986603
10.2333333333333 0.546503722667694
10.7205128205128 0.509297788143158
11.2307692307692 0.494511902332306
11.7666666666667 0.490456074476242
12.3282051282051 0.509564816951752
12.9153846153846 0.504638195037842
13.5282051282051 0.498807162046432
14.174358974359 0.497449368238449
14.8487179487179 0.495622873306274
15.5564102564103 0.488597393035889
16.2974358974359 0.471165716648102
17.0717948717949 0.484007924795151
17.8846153846154 0.500007152557373
18.7358974358974 0.471365183591843
19.6282051282051 0.483115345239639
20.5641025641026 0.469110697507858
21.5435897435897 0.453604310750961
22.5692307692308 0.453824996948242
23.6435897435897 0.430432707071304
24.7692307692308 0.43325462937355
25.9487179487179 0.446426600217819
27.1846153846154 0.442343056201935
28.4794871794872 0.446410715579987
29.8358974358974 0.432268470525742
31.2564102564103 0.4156434237957
32.7435897435897 0.401817560195923
34.3025641025641 0.437322914600372
35.9358974358974 0.442858308553696
37.648717948718 0.407794207334518
39.4410256410256 0.404664009809494
41.3179487179487 0.408350229263306
43.2871794871795 0.384977161884308
45.3487179487179 0.37384158372879
47.5076923076923 0.445147961378098
49.7692307692308 0.370937585830688
52.1384615384615 0.385302841663361
54.6205128205128 0.392190337181091
57.2230769230769 0.355943411588669
59.9461538461538 0.370930045843124
62.8025641025641 0.329294204711914
65.7923076923077 0.359216660261154
68.925641025641 0.379535675048828
72.2076923076923 0.358621925115585
75.6461538461539 0.335759788751602
79.2461538461538 0.349384635686874
83.0205128205128 0.332648426294327
86.974358974359 0.299899101257324
91.1153846153846 0.29847115278244
95.4538461538462 0.330428689718246
100 0.371293187141418
};
\addlegendentry{sub 16, mc 1}
\addplot [, color2, opacity=0.6, mark=triangle*, mark size=0.5, mark options={solid,rotate=180}, only marks, forget plot]
table {%
1 0.653675019741058
1.04615384615385 0.636393427848816
1.0974358974359 0.648579895496368
1.14871794871795 0.61937403678894
1.2025641025641 0.593536019325256
1.26153846153846 0.602782368659973
1.32051282051282 0.538430869579315
1.38461538461538 0.615821361541748
1.44871794871795 0.497967541217804
1.51794871794872 0.64619255065918
1.58974358974359 0.565708458423615
1.66666666666667 0.599131882190704
1.74615384615385 0.593147218227386
1.82820512820513 0.511482357978821
1.91538461538462 0.560302317142487
2.00769230769231 0.578881204128265
2.1025641025641 0.628819465637207
2.20512820512821 0.503145158290863
2.30769230769231 0.578563094139099
2.41794871794872 0.533103466033936
2.53333333333333 0.560707986354828
2.65384615384615 0.532446444034576
2.78205128205128 0.555656850337982
2.91282051282051 0.528802812099457
3.05384615384615 0.544940173625946
3.1974358974359 0.57191675901413
3.35128205128205 0.52626758813858
3.51025641025641 0.531772077083588
3.67692307692308 0.5312819480896
3.85128205128205 0.518880486488342
4.03589743589744 0.511874616146088
4.22820512820513 0.527622520923615
4.42820512820513 0.521969139575958
4.64102564102564 0.525750458240509
4.86153846153846 0.516772270202637
5.09230769230769 0.518696248531342
5.33589743589744 0.503518521785736
5.58974358974359 0.502795815467834
5.85641025641026 0.499732404947281
6.13589743589744 0.511925339698792
6.42564102564103 0.487289577722549
6.73333333333333 0.499607563018799
7.05384615384615 0.496742576360703
7.38974358974359 0.481126070022583
7.74102564102564 0.466178148984909
8.11025641025641 0.437175273895264
8.4974358974359 0.485646635293961
8.9 0.469920367002487
9.32564102564103 0.46212312579155
9.76923076923077 0.467639207839966
10.2333333333333 0.485023319721222
10.7205128205128 0.440464466810226
11.2307692307692 0.447988599538803
11.7666666666667 0.452294170856476
12.3282051282051 0.442107111215591
12.9153846153846 0.430669993162155
13.5282051282051 0.433440178632736
14.174358974359 0.418857961893082
14.8487179487179 0.454310148954391
15.5564102564103 0.422545194625854
16.2974358974359 0.425638258457184
17.0717948717949 0.406928777694702
17.8846153846154 0.422529608011246
18.7358974358974 0.419487863779068
19.6282051282051 0.430408865213394
20.5641025641026 0.399328947067261
21.5435897435897 0.396209448575974
22.5692307692308 0.406517565250397
23.6435897435897 0.385463535785675
24.7692307692308 0.397652357816696
25.9487179487179 0.41415199637413
27.1846153846154 0.412353962659836
28.4794871794872 0.346190810203552
29.8358974358974 0.381460756063461
31.2564102564103 0.40917444229126
32.7435897435897 0.340854555368423
34.3025641025641 0.362202018499374
35.9358974358974 0.395376294851303
37.648717948718 0.322113841772079
39.4410256410256 0.379490941762924
41.3179487179487 0.333208292722702
43.2871794871795 0.343972593545914
45.3487179487179 0.32807594537735
47.5076923076923 0.332019627094269
49.7692307692308 0.269908726215363
52.1384615384615 0.290175199508667
54.6205128205128 0.287988275289536
57.2230769230769 0.326751947402954
59.9461538461538 0.277277708053589
62.8025641025641 0.280847728252411
65.7923076923077 0.290741205215454
68.925641025641 0.32089176774025
72.2076923076923 0.262967586517334
75.6461538461539 0.272496789693832
79.2461538461538 0.278938323259354
83.0205128205128 0.288222461938858
86.974358974359 0.280288130044937
91.1153846153846 0.276302635669708
95.4538461538462 0.26351797580719
100 0.273569613695145
};
\addplot [, color2, opacity=0.6, mark=triangle*, mark size=0.5, mark options={solid,rotate=180}, only marks, forget plot]
table {%
1 0.620134949684143
1.04615384615385 0.723954617977142
1.0974358974359 0.596045792102814
1.14871794871795 0.633840978145599
1.2025641025641 0.588489055633545
1.26153846153846 0.69552606344223
1.32051282051282 0.698261559009552
1.38461538461538 0.605440020561218
1.44871794871795 0.566089808940887
1.51794871794872 0.616282105445862
1.58974358974359 0.603321433067322
1.66666666666667 0.588808298110962
1.74615384615385 0.630506157875061
1.82820512820513 0.597203731536865
1.91538461538462 0.591791331768036
2.00769230769231 0.621661365032196
2.1025641025641 0.579618632793427
2.20512820512821 0.607517123222351
2.30769230769231 0.577914655208588
2.41794871794872 0.576551735401154
2.53333333333333 0.57029801607132
2.65384615384615 0.581252872943878
2.78205128205128 0.554794013500214
2.91282051282051 0.554378509521484
3.05384615384615 0.564366936683655
3.1974358974359 0.557165145874023
3.35128205128205 0.519036412239075
3.51025641025641 0.561954736709595
3.67692307692308 0.553743302822113
3.85128205128205 0.553749263286591
4.03589743589744 0.51231974363327
4.22820512820513 0.551812529563904
4.42820512820513 0.508747339248657
4.64102564102564 0.522983193397522
4.86153846153846 0.523935914039612
5.09230769230769 0.511667966842651
5.33589743589744 0.534750938415527
5.58974358974359 0.537909924983978
5.85641025641026 0.501629590988159
6.13589743589744 0.49638968706131
6.42564102564103 0.506797969341278
6.73333333333333 0.474414646625519
7.05384615384615 0.506680548191071
7.38974358974359 0.508407413959503
7.74102564102564 0.47770968079567
8.11025641025641 0.502065777778625
8.4974358974359 0.499860852956772
8.9 0.451256334781647
9.32564102564103 0.467289209365845
9.76923076923077 0.484345197677612
10.2333333333333 0.450974524021149
10.7205128205128 0.486675232648849
11.2307692307692 0.472440004348755
11.7666666666667 0.49336177110672
12.3282051282051 0.490280836820602
12.9153846153846 0.422113984823227
13.5282051282051 0.462146252393723
14.174358974359 0.451734364032745
14.8487179487179 0.433115780353546
15.5564102564103 0.424235820770264
16.2974358974359 0.448445469141006
17.0717948717949 0.429114431142807
17.8846153846154 0.438704818487167
18.7358974358974 0.460261642932892
19.6282051282051 0.429112017154694
20.5641025641026 0.45067235827446
21.5435897435897 0.441748946905136
22.5692307692308 0.420592039823532
23.6435897435897 0.424013912677765
24.7692307692308 0.409102648496628
25.9487179487179 0.417770475149155
27.1846153846154 0.406411081552505
28.4794871794872 0.405434429645538
29.8358974358974 0.4036865234375
31.2564102564103 0.396220237016678
32.7435897435897 0.393568247556686
34.3025641025641 0.3850357234478
35.9358974358974 0.370021909475327
37.648717948718 0.401499032974243
39.4410256410256 0.349017381668091
41.3179487179487 0.401208013296127
43.2871794871795 0.404588043689728
45.3487179487179 0.342751801013947
47.5076923076923 0.350426286458969
49.7692307692308 0.396591007709503
52.1384615384615 0.330476850271225
54.6205128205128 0.352481424808502
57.2230769230769 0.299628347158432
59.9461538461538 0.369509279727936
62.8025641025641 0.317663043737411
65.7923076923077 0.356819480657578
68.925641025641 0.351463049650192
72.2076923076923 0.381566107273102
75.6461538461539 0.311359971761703
79.2461538461538 0.299783617258072
83.0205128205128 0.28909769654274
86.974358974359 0.331129848957062
91.1153846153846 0.309157490730286
95.4538461538462 0.318864762783051
100 0.333490818738937
};
\addplot [, color2, opacity=0.6, mark=triangle*, mark size=0.5, mark options={solid,rotate=180}, only marks, forget plot]
table {%
1 0.643575668334961
1.04615384615385 0.775696694850922
1.0974358974359 0.712574899196625
1.14871794871795 0.593177318572998
1.2025641025641 0.654149830341339
1.26153846153846 0.674933850765228
1.32051282051282 0.628009736537933
1.38461538461538 0.662072479724884
1.44871794871795 0.642270267009735
1.51794871794872 0.587342441082001
1.58974358974359 0.65009593963623
1.66666666666667 0.586813628673553
1.74615384615385 0.568937301635742
1.82820512820513 0.492892175912857
1.91538461538462 0.581144034862518
2.00769230769231 0.570952892303467
2.1025641025641 0.494144588708878
2.20512820512821 0.507708191871643
2.30769230769231 0.482706636190414
2.41794871794872 0.505551755428314
2.53333333333333 0.57016795873642
2.65384615384615 0.493417799472809
2.78205128205128 0.550698697566986
2.91282051282051 0.494295835494995
3.05384615384615 0.490532219409943
3.1974358974359 0.512477397918701
3.35128205128205 0.486635982990265
3.51025641025641 0.51180237531662
3.67692307692308 0.503197193145752
3.85128205128205 0.552313506603241
4.03589743589744 0.509155333042145
4.22820512820513 0.515761971473694
4.42820512820513 0.468440622091293
4.64102564102564 0.500194489955902
4.86153846153846 0.480341732501984
5.09230769230769 0.500829696655273
5.33589743589744 0.507397949695587
5.58974358974359 0.527890145778656
5.85641025641026 0.498423546552658
6.13589743589744 0.465162128210068
6.42564102564103 0.478064268827438
6.73333333333333 0.46371927857399
7.05384615384615 0.517676711082458
7.38974358974359 0.476526081562042
7.74102564102564 0.473904818296432
8.11025641025641 0.478507608175278
8.4974358974359 0.449889004230499
8.9 0.457254409790039
9.32564102564103 0.432021349668503
9.76923076923077 0.460717022418976
10.2333333333333 0.458454757928848
10.7205128205128 0.450693607330322
11.2307692307692 0.432268291711807
11.7666666666667 0.410072803497314
12.3282051282051 0.449987322092056
12.9153846153846 0.441637814044952
13.5282051282051 0.447544068098068
14.174358974359 0.432095438241959
14.8487179487179 0.408235609531403
15.5564102564103 0.432640612125397
16.2974358974359 0.421347141265869
17.0717948717949 0.410881102085114
17.8846153846154 0.420873612165451
18.7358974358974 0.413749039173126
19.6282051282051 0.3735631108284
20.5641025641026 0.381828755140305
21.5435897435897 0.424183666706085
22.5692307692308 0.402295887470245
23.6435897435897 0.38370344042778
24.7692307692308 0.384390532970428
25.9487179487179 0.397337555885315
27.1846153846154 0.406966596841812
28.4794871794872 0.388192027807236
29.8358974358974 0.371990829706192
31.2564102564103 0.376241654157639
32.7435897435897 0.360650151968002
34.3025641025641 0.369140028953552
35.9358974358974 0.373061239719391
37.648717948718 0.376948863267899
39.4410256410256 0.380572885274887
41.3179487179487 0.379489868879318
43.2871794871795 0.380249440670013
45.3487179487179 0.350042730569839
47.5076923076923 0.354917824268341
49.7692307692308 0.345423191785812
52.1384615384615 0.340282410383224
54.6205128205128 0.338409632444382
57.2230769230769 0.348179697990417
59.9461538461538 0.359174579381943
62.8025641025641 0.336934596300125
65.7923076923077 0.338908165693283
68.925641025641 0.354305565357208
72.2076923076923 0.358900040388107
75.6461538461539 0.358384281396866
79.2461538461538 0.321666091680527
83.0205128205128 0.347705125808716
86.974358974359 0.355066418647766
91.1153846153846 0.350439935922623
95.4538461538462 0.357577055692673
100 0.33313250541687
};
\addplot [, color2, opacity=0.6, mark=triangle*, mark size=0.5, mark options={solid,rotate=180}, only marks, forget plot]
table {%
1 0.670116126537323
1.04615384615385 0.688552796840668
1.0974358974359 0.727224946022034
1.14871794871795 0.666293144226074
1.2025641025641 0.658023834228516
1.26153846153846 0.639449715614319
1.32051282051282 0.6471146941185
1.38461538461538 0.675963699817657
1.44871794871795 0.669123768806458
1.51794871794872 0.638483166694641
1.58974358974359 0.619391024112701
1.66666666666667 0.64858865737915
1.74615384615385 0.649833858013153
1.82820512820513 0.606956660747528
1.91538461538462 0.655722200870514
2.00769230769231 0.615619599819183
2.1025641025641 0.621951699256897
2.20512820512821 0.669163346290588
2.30769230769231 0.628867328166962
2.41794871794872 0.643801629543304
2.53333333333333 0.575005710124969
2.65384615384615 0.684292912483215
2.78205128205128 0.612332046031952
2.91282051282051 0.680157124996185
3.05384615384615 0.701790869235992
3.1974358974359 0.691975295543671
3.35128205128205 0.656712651252747
3.51025641025641 0.684622347354889
3.67692307692308 0.651035249233246
3.85128205128205 0.661327123641968
4.03589743589744 0.64604377746582
4.22820512820513 0.649456918239594
4.42820512820513 0.645676910877228
4.64102564102564 0.633455991744995
4.86153846153846 0.637354969978333
5.09230769230769 0.611588776111603
5.33589743589744 0.643035650253296
5.58974358974359 0.648472964763641
5.85641025641026 0.640162825584412
6.13589743589744 0.586988151073456
6.42564102564103 0.603223860263824
6.73333333333333 0.609087586402893
7.05384615384615 0.603633224964142
7.38974358974359 0.613547146320343
7.74102564102564 0.566059231758118
8.11025641025641 0.587567985057831
8.4974358974359 0.609107196331024
8.9 0.5555419921875
9.32564102564103 0.575416505336761
9.76923076923077 0.56263655424118
10.2333333333333 0.551961660385132
10.7205128205128 0.560971677303314
11.2307692307692 0.506049931049347
11.7666666666667 0.528604209423065
12.3282051282051 0.512854099273682
12.9153846153846 0.53241491317749
13.5282051282051 0.517832458019257
14.174358974359 0.505244731903076
14.8487179487179 0.517298340797424
15.5564102564103 0.518386900424957
16.2974358974359 0.515102744102478
17.0717948717949 0.501561105251312
17.8846153846154 0.515463411808014
18.7358974358974 0.508116662502289
19.6282051282051 0.497651398181915
20.5641025641026 0.50120609998703
21.5435897435897 0.484664887189865
22.5692307692308 0.487663477659225
23.6435897435897 0.462733238935471
24.7692307692308 0.485623180866241
25.9487179487179 0.472351759672165
27.1846153846154 0.466819137334824
28.4794871794872 0.480812132358551
29.8358974358974 0.473771184682846
31.2564102564103 0.459099024534225
32.7435897435897 0.471851348876953
34.3025641025641 0.453896969556808
35.9358974358974 0.462344080209732
37.648717948718 0.432960242033005
39.4410256410256 0.40609273314476
41.3179487179487 0.477459996938705
43.2871794871795 0.479015976190567
45.3487179487179 0.401993662118912
47.5076923076923 0.42195662856102
49.7692307692308 0.410141080617905
52.1384615384615 0.413759469985962
54.6205128205128 0.414904177188873
57.2230769230769 0.37058898806572
59.9461538461538 0.451622396707535
62.8025641025641 0.365902870893478
65.7923076923077 0.448089510202408
68.925641025641 0.412727177143097
72.2076923076923 0.435296833515167
75.6461538461539 0.403671652078629
79.2461538461538 0.349900543689728
83.0205128205128 0.426113426685333
86.974358974359 0.394461363554001
91.1153846153846 0.396772444248199
95.4538461538462 0.383325695991516
100 0.384916871786118
};
\end{axis}

\end{tikzpicture}

      \tikzexternaldisable
    \end{minipage}
  \end{subfigure}

  \begin{subfigure}[t]{\linewidth}
    \centering
    \caption{\cifarten \threecthreed \sgd}
    \begin{minipage}{0.50\linewidth}
      \centering
      % defines the pgfplots style "eigspacedefault"
\pgfkeys{/pgfplots/eigspacedefault/.style={
    width=1.0\linewidth,
    height=0.6\linewidth,
    every axis plot/.append style={line width = 1.5pt},
    tick pos = left,
    ylabel near ticks,
    xlabel near ticks,
    xtick align = inside,
    ytick align = inside,
    legend cell align = left,
    legend columns = 4,
    legend pos = south east,
    legend style = {
      fill opacity = 1,
      text opacity = 1,
      font = \footnotesize,
      at={(1, 1.025)},
      anchor=south east,
      column sep=0.25cm,
    },
    legend image post style={scale=2.5},
    xticklabel style = {font = \footnotesize},
    xlabel style = {font = \footnotesize},
    axis line style = {black},
    yticklabel style = {font = \footnotesize},
    ylabel style = {font = \footnotesize},
    title style = {font = \footnotesize},
    grid = major,
    grid style = {dashed}
  }
}

\pgfkeys{/pgfplots/eigspacedefaultapp/.style={
    eigspacedefault,
    height=0.6\linewidth,
    legend columns = 2,
  }
}

\pgfkeys{/pgfplots/eigspacenolegend/.style={
    legend image post style = {scale=0},
    legend style = {
      fill opacity = 0,
      draw opacity = 0,
      text opacity = 0,
      font = \footnotesize,
      at={(1, 1.025)},
      anchor=south east,
      column sep=0.25cm,
    },
  }
}
%%% Local Variables:
%%% mode: latex
%%% TeX-master: "../../thesis"
%%% End:

      \pgfkeys{/pgfplots/zmystyle/.style={
          eigspacedefaultapp,
          eigspacenolegend,
        }}
      \tikzexternalenable
      \vspace{-6ex}
      % This file was created by tikzplotlib v0.9.7.
\begin{tikzpicture}

\definecolor{color0}{rgb}{0.501960784313725,0.184313725490196,0.6}
\definecolor{color1}{rgb}{0.870588235294118,0.623529411764706,0.0862745098039216}
\definecolor{color2}{rgb}{0.274509803921569,0.6,0.564705882352941}

\begin{axis}[
axis line style={white!10!black},
legend columns=2,
legend style={fill opacity=0.8, draw opacity=1, text opacity=1, at={(0.03,0.03)}, anchor=south west, draw=white!80!black},
log basis x={10},
tick pos=left,
xlabel={epoch (log scale)},
xmajorgrids,
xmin=0.794328234724281, xmax=125.892541179417,
xmode=log,
ylabel={overlap},
ymajorgrids,
ymin=-0.05, ymax=1.05,
zmystyle
]
\addplot [, white!10!black, dashed, forget plot]
table {%
0.794328234724281 1
125.892541179417 1
};
\addplot [, white!10!black, dashed, forget plot]
table {%
0.794328234724281 0
125.892541179417 0
};
\addplot [, color0, opacity=0.6, mark=triangle*, mark size=0.5, mark options={solid,rotate=180}, only marks]
table {%
1 0.633321225643158
1.04487179487179 0.633167922496796
1.09615384615385 0.645319879055023
1.1474358974359 0.696935772895813
1.20192307692308 0.735691547393799
1.25961538461538 0.743004858493805
1.32051282051282 0.663159072399139
1.38461538461538 0.578848659992218
1.44871794871795 0.566265523433685
1.51923076923077 0.511307239532471
1.58974358974359 0.553184747695923
1.66666666666667 0.507019340991974
1.74679487179487 0.527087032794952
1.83012820512821 0.489433765411377
1.91666666666667 0.52100533246994
2.00641025641026 0.487207233905792
2.1025641025641 0.516088545322418
2.20512820512821 0.523843228816986
2.30769230769231 0.506053566932678
2.41987179487179 0.543740272521973
2.53525641025641 0.500154674053192
2.65384615384615 0.497678607702255
2.78205128205128 0.518427073955536
2.91346153846154 0.490070313215256
3.05128205128205 0.478386849164963
3.19871794871795 0.484398126602173
3.34935897435897 0.461323797702789
3.50961538461538 0.456510841846466
3.67628205128205 0.400044173002243
3.8525641025641 0.404926776885986
4.03525641025641 0.426442533731461
4.2275641025641 0.415361940860748
4.42948717948718 0.38182258605957
4.64102564102564 0.365217059850693
4.86217948717949 0.365998953580856
5.09294871794872 0.351479113101959
5.33653846153846 0.366690158843994
5.58974358974359 0.331267327070236
5.85576923076923 0.361305773258209
6.13461538461539 0.32453528046608
6.42628205128205 0.343815416097641
6.73397435897436 0.304473489522934
7.05448717948718 0.287028342485428
7.38782051282051 0.304264098405838
7.74038461538461 0.291326016187668
8.10897435897436 0.310010522603989
8.49679487179487 0.287607759237289
8.90064102564103 0.270657986402512
9.32371794871795 0.272140860557556
9.76923076923077 0.291391462087631
10.2339743589744 0.288339495658875
10.7211538461538 0.265323370695114
11.2307692307692 0.268832862377167
11.7660256410256 0.28240692615509
12.3269230769231 0.261321276426315
12.9134615384615 0.211589977145195
13.5288461538462 0.254175752401352
14.1730769230769 0.26838681101799
14.849358974359 0.239705845713615
15.5544871794872 0.268844097852707
16.2948717948718 0.263189524412155
17.0705128205128 0.248800024390221
17.8846153846154 0.234545215964317
18.7371794871795 0.243242412805557
19.6282051282051 0.237384006381035
20.5641025641026 0.232320502400398
21.5416666666667 0.21445694565773
22.5673076923077 0.203809887170792
23.6442307692308 0.211414203047752
24.7692307692308 0.192375466227531
25.9487179487179 0.207051798701286
27.1826923076923 0.184000551700592
28.4775641025641 0.186780527234077
29.8333333333333 0.209896236658096
31.2564102564103 0.193499848246574
32.7435897435897 0.187499046325684
34.3044871794872 0.177010402083397
35.9358974358974 0.178522393107414
37.6474358974359 0.184974029660225
39.4391025641026 0.175946339964867
41.3173076923077 0.168570205569267
43.2852564102564 0.150046691298485
45.3461538461538 0.160229966044426
47.5064102564103 0.15055838227272
49.7692307692308 0.151106879115105
52.1378205128205 0.149630650877953
54.6217948717949 0.159002006053925
57.2211538461538 0.151609182357788
59.9455128205128 0.14301697909832
62.8012820512821 0.140859991312027
65.7916666666667 0.114369906485081
68.9230769230769 0.127198085188866
72.2051282051282 0.117595888674259
75.6442307692308 0.133174225687981
79.2467948717949 0.114469043910503
83.0192307692308 0.127625197172165
86.974358974359 0.120496347546577
91.1153846153846 0.0918874368071556
95.4519230769231 0.114998415112495
100 0.101283550262451
};
\addlegendentry{mb 2, exact}
\addplot [, color0, opacity=0.6, mark=triangle*, mark size=0.5, mark options={solid,rotate=180}, only marks, forget plot]
table {%
1 0.494709342718124
1.04487179487179 0.508072674274445
1.09615384615385 0.472669273614883
1.1474358974359 0.4533970952034
1.20192307692308 0.487139940261841
1.25961538461538 0.536234498023987
1.32051282051282 0.51693856716156
1.38461538461538 0.508000075817108
1.44871794871795 0.532252967357635
1.51923076923077 0.521783828735352
1.58974358974359 0.515431880950928
1.66666666666667 0.483936071395874
1.74679487179487 0.511323273181915
1.83012820512821 0.463768869638443
1.91666666666667 0.501585304737091
2.00641025641026 0.491608619689941
2.1025641025641 0.502568542957306
2.20512820512821 0.48529776930809
2.30769230769231 0.425935506820679
2.41987179487179 0.493673413991928
2.53525641025641 0.466302067041397
2.65384615384615 0.40349555015564
2.78205128205128 0.4183629155159
2.91346153846154 0.443614333868027
3.05128205128205 0.400776952505112
3.19871794871795 0.397828459739685
3.34935897435897 0.40813159942627
3.50961538461538 0.371348828077316
3.67628205128205 0.370048195123672
3.8525641025641 0.338275015354156
4.03525641025641 0.371874868869781
4.2275641025641 0.349921315908432
4.42948717948718 0.33645948767662
4.64102564102564 0.342290461063385
4.86217948717949 0.342065185308456
5.09294871794872 0.343239516019821
5.33653846153846 0.304060429334641
5.58974358974359 0.302288204431534
5.85576923076923 0.348987132310867
6.13461538461539 0.307548046112061
6.42628205128205 0.293838262557983
6.73397435897436 0.298730999231339
7.05448717948718 0.288828045129776
7.38782051282051 0.291887164115906
7.74038461538461 0.271417856216431
8.10897435897436 0.291929572820663
8.49679487179487 0.239717289805412
8.90064102564103 0.256766855716705
9.32371794871795 0.255834400653839
9.76923076923077 0.235459715127945
10.2339743589744 0.249298378825188
10.7211538461538 0.238867193460464
11.2307692307692 0.261288821697235
11.7660256410256 0.239302918314934
12.3269230769231 0.252336323261261
12.9134615384615 0.21277180314064
13.5288461538462 0.241548523306847
14.1730769230769 0.214700922369957
14.849358974359 0.220108270645142
15.5544871794872 0.233450695872307
16.2948717948718 0.220849707722664
17.0705128205128 0.232942014932632
17.8846153846154 0.2052371352911
18.7371794871795 0.208879426121712
19.6282051282051 0.201221585273743
20.5641025641026 0.228594049811363
21.5416666666667 0.201344445347786
22.5673076923077 0.182441547513008
23.6442307692308 0.194797322154045
24.7692307692308 0.189139351248741
25.9487179487179 0.192408084869385
27.1826923076923 0.210514530539513
28.4775641025641 0.197941929101944
29.8333333333333 0.183976471424103
31.2564102564103 0.200855448842049
32.7435897435897 0.202761322259903
34.3044871794872 0.165803536772728
35.9358974358974 0.170795083045959
37.6474358974359 0.16943621635437
39.4391025641026 0.188188686966896
41.3173076923077 0.168842270970345
43.2852564102564 0.170913800597191
45.3461538461538 0.15562430024147
47.5064102564103 0.161235079169273
49.7692307692308 0.146307706832886
52.1378205128205 0.129889413714409
54.6217948717949 0.160507008433342
57.2211538461538 0.143643841147423
59.9455128205128 0.13008388876915
62.8012820512821 0.12173768132925
65.7916666666667 0.131866618990898
68.9230769230769 0.121484279632568
72.2051282051282 0.114791117608547
75.6442307692308 0.132824599742889
79.2467948717949 0.122266843914986
83.0192307692308 0.0980019569396973
86.974358974359 0.0903954803943634
91.1153846153846 0.111445426940918
95.4519230769231 0.121098451316357
100 0.103437542915344
};
\addplot [, color0, opacity=0.6, mark=triangle*, mark size=0.5, mark options={solid,rotate=180}, only marks, forget plot]
table {%
1 0.545524299144745
1.04487179487179 0.531452834606171
1.09615384615385 0.595713555812836
1.1474358974359 0.604683220386505
1.20192307692308 0.612162947654724
1.25961538461538 0.601134896278381
1.32051282051282 0.598043084144592
1.38461538461538 0.58671897649765
1.44871794871795 0.643461883068085
1.51923076923077 0.584388673305511
1.58974358974359 0.603388786315918
1.66666666666667 0.542600691318512
1.74679487179487 0.56792676448822
1.83012820512821 0.542393863201141
1.91666666666667 0.532415986061096
2.00641025641026 0.550632297992706
2.1025641025641 0.503512680530548
2.20512820512821 0.522281169891357
2.30769230769231 0.520580768585205
2.41987179487179 0.526549100875854
2.53525641025641 0.547998547554016
2.65384615384615 0.477871954441071
2.78205128205128 0.502317786216736
2.91346153846154 0.455182939767838
3.05128205128205 0.465587705373764
3.19871794871795 0.455523103475571
3.34935897435897 0.462501853704453
3.50961538461538 0.43492203950882
3.67628205128205 0.40585994720459
3.8525641025641 0.409336149692535
4.03525641025641 0.440798580646515
4.2275641025641 0.402796983718872
4.42948717948718 0.403908252716064
4.64102564102564 0.391839325428009
4.86217948717949 0.360024124383926
5.09294871794872 0.385159879922867
5.33653846153846 0.368329137563705
5.58974358974359 0.352680623531342
5.85576923076923 0.370293498039246
6.13461538461539 0.354883462190628
6.42628205128205 0.348915994167328
6.73397435897436 0.332288801670074
7.05448717948718 0.291275918483734
7.38782051282051 0.276949733495712
7.74038461538461 0.283589214086533
8.10897435897436 0.314339220523834
8.49679487179487 0.275656849145889
8.90064102564103 0.275349378585815
9.32371794871795 0.281154692173004
9.76923076923077 0.274319857358932
10.2339743589744 0.217059180140495
10.7211538461538 0.239347815513611
11.2307692307692 0.265282303094864
11.7660256410256 0.263832896947861
12.3269230769231 0.239591643214226
12.9134615384615 0.207162380218506
13.5288461538462 0.215716153383255
14.1730769230769 0.232497081160545
14.849358974359 0.242441937327385
15.5544871794872 0.202556610107422
16.2948717948718 0.210512146353722
17.0705128205128 0.202129125595093
17.8846153846154 0.192183658480644
18.7371794871795 0.196333274245262
19.6282051282051 0.211474731564522
20.5641025641026 0.203455209732056
21.5416666666667 0.195467472076416
22.5673076923077 0.176268443465233
23.6442307692308 0.173090502619743
24.7692307692308 0.179466798901558
25.9487179487179 0.154363766312599
27.1826923076923 0.176492363214493
28.4775641025641 0.135159716010094
29.8333333333333 0.157441377639771
31.2564102564103 0.138464614748955
32.7435897435897 0.139906913042068
34.3044871794872 0.122646294534206
35.9358974358974 0.145737156271935
37.6474358974359 0.133732527494431
39.4391025641026 0.142999321222305
41.3173076923077 0.132487401366234
43.2852564102564 0.118065379559994
45.3461538461538 0.127987772226334
47.5064102564103 0.138686999678612
49.7692307692308 0.117941595613956
52.1378205128205 0.114847496151924
54.6217948717949 0.124358415603638
57.2211538461538 0.123346768319607
59.9455128205128 0.108971193432808
62.8012820512821 0.0969991162419319
65.7916666666667 0.111605107784271
68.9230769230769 0.0979296118021011
72.2051282051282 0.0991314128041267
75.6442307692308 0.0898933187127113
79.2467948717949 0.0954435020685196
83.0192307692308 0.0925521999597549
86.974358974359 0.104363583028316
91.1153846153846 0.108041144907475
95.4519230769231 0.0872776731848717
100 0.091015987098217
};
\addplot [, color0, opacity=0.6, mark=triangle*, mark size=0.5, mark options={solid,rotate=180}, only marks, forget plot]
table {%
1 0.523422837257385
1.04487179487179 0.517627477645874
1.09615384615385 0.528109967708588
1.1474358974359 0.520947277545929
1.20192307692308 0.494197100400925
1.25961538461538 0.441493839025497
1.32051282051282 0.421509712934494
1.38461538461538 0.397586405277252
1.44871794871795 0.484627097845078
1.51923076923077 0.474800407886505
1.58974358974359 0.428153336048126
1.66666666666667 0.458623945713043
1.74679487179487 0.433956056833267
1.83012820512821 0.451023578643799
1.91666666666667 0.422371596097946
2.00641025641026 0.41279274225235
2.1025641025641 0.418320804834366
2.20512820512821 0.408322244882584
2.30769230769231 0.391086012125015
2.41987179487179 0.413396567106247
2.53525641025641 0.459487289190292
2.65384615384615 0.341690868139267
2.78205128205128 0.389077931642532
2.91346153846154 0.366262018680573
3.05128205128205 0.369984745979309
3.19871794871795 0.345057189464569
3.34935897435897 0.35873094201088
3.50961538461538 0.344358503818512
3.67628205128205 0.333692252635956
3.8525641025641 0.328528970479965
4.03525641025641 0.331492394208908
4.2275641025641 0.313480824232101
4.42948717948718 0.343261927366257
4.64102564102564 0.340480923652649
4.86217948717949 0.342256844043732
5.09294871794872 0.302556663751602
5.33653846153846 0.313608795404434
5.58974358974359 0.302308350801468
5.85576923076923 0.337082266807556
6.13461538461539 0.311828523874283
6.42628205128205 0.271950960159302
6.73397435897436 0.298972755670547
7.05448717948718 0.266834586858749
7.38782051282051 0.288553267717361
7.74038461538461 0.286214828491211
8.10897435897436 0.295535951852798
8.49679487179487 0.248173668980598
8.90064102564103 0.275844842195511
9.32371794871795 0.27056673169136
9.76923076923077 0.255090594291687
10.2339743589744 0.242637678980827
10.7211538461538 0.251264572143555
11.2307692307692 0.256876617670059
11.7660256410256 0.258264362812042
12.3269230769231 0.247272834181786
12.9134615384615 0.20845952630043
13.5288461538462 0.232171177864075
14.1730769230769 0.224469527602196
14.849358974359 0.236231431365013
15.5544871794872 0.220951840281487
16.2948717948718 0.209961369633675
17.0705128205128 0.229115352034569
17.8846153846154 0.200397402048111
18.7371794871795 0.194268450140953
19.6282051282051 0.191540837287903
20.5641025641026 0.20839436352253
21.5416666666667 0.190234184265137
22.5673076923077 0.18439456820488
23.6442307692308 0.164636000990868
24.7692307692308 0.173214182257652
25.9487179487179 0.180703267455101
27.1826923076923 0.171099841594696
28.4775641025641 0.164580777287483
29.8333333333333 0.188525423407555
31.2564102564103 0.168585330247879
32.7435897435897 0.15962065756321
34.3044871794872 0.150746837258339
35.9358974358974 0.145215258002281
37.6474358974359 0.121914483606815
39.4391025641026 0.128806188702583
41.3173076923077 0.136148855090141
43.2852564102564 0.136276051402092
45.3461538461538 0.141158908605576
47.5064102564103 0.114671371877193
49.7692307692308 0.119016073644161
52.1378205128205 0.120105646550655
54.6217948717949 0.116573847830296
57.2211538461538 0.122383452951908
59.9455128205128 0.116962432861328
62.8012820512821 0.105055682361126
65.7916666666667 0.107373334467411
68.9230769230769 0.0942141935229301
72.2051282051282 0.0986168012022972
75.6442307692308 0.0988930389285088
79.2467948717949 0.0927614718675613
83.0192307692308 0.0953556448221207
86.974358974359 0.0847640186548233
91.1153846153846 0.0852355360984802
95.4519230769231 0.0992736965417862
100 0.0802000463008881
};
\addplot [, color0, opacity=0.6, mark=triangle*, mark size=0.5, mark options={solid,rotate=180}, only marks, forget plot]
table {%
1 0.605333626270294
1.04487179487179 0.597685515880585
1.09615384615385 0.667412877082825
1.1474358974359 0.641108334064484
1.20192307692308 0.659764111042023
1.25961538461538 0.584210574626923
1.32051282051282 0.504951179027557
1.38461538461538 0.468591302633286
1.44871794871795 0.491764754056931
1.51923076923077 0.445441067218781
1.58974358974359 0.461829245090485
1.66666666666667 0.396699368953705
1.74679487179487 0.431269645690918
1.83012820512821 0.396298855543137
1.91666666666667 0.418356865644455
2.00641025641026 0.396757781505585
2.1025641025641 0.407334387302399
2.20512820512821 0.437328070402145
2.30769230769231 0.436087816953659
2.41987179487179 0.443477481603622
2.53525641025641 0.421981781721115
2.65384615384615 0.393724977970123
2.78205128205128 0.412965029478073
2.91346153846154 0.396501153707504
3.05128205128205 0.384052693843842
3.19871794871795 0.419008076190948
3.34935897435897 0.374358713626862
3.50961538461538 0.370988756418228
3.67628205128205 0.373596727848053
3.8525641025641 0.365131318569183
4.03525641025641 0.359287589788437
4.2275641025641 0.357491701841354
4.42948717948718 0.33229061961174
4.64102564102564 0.326798528432846
4.86217948717949 0.311623483896255
5.09294871794872 0.388948082923889
5.33653846153846 0.328317314386368
5.58974358974359 0.314519166946411
5.85576923076923 0.343756437301636
6.13461538461539 0.347886890172958
6.42628205128205 0.328506678342819
6.73397435897436 0.301367402076721
7.05448717948718 0.306648939847946
7.38782051282051 0.297799736261368
7.74038461538461 0.297945588827133
8.10897435897436 0.296382278203964
8.49679487179487 0.255705922842026
8.90064102564103 0.276719629764557
9.32371794871795 0.252005964517593
9.76923076923077 0.245079860091209
10.2339743589744 0.280048936605453
10.7211538461538 0.259938299655914
11.2307692307692 0.242932483553886
11.7660256410256 0.231496959924698
12.3269230769231 0.240419551730156
12.9134615384615 0.234377935528755
13.5288461538462 0.236405208706856
14.1730769230769 0.236070021986961
14.849358974359 0.223111420869827
15.5544871794872 0.234789177775383
16.2948717948718 0.24242801964283
17.0705128205128 0.231832429766655
17.8846153846154 0.205523729324341
18.7371794871795 0.229432567954063
19.6282051282051 0.221449330449104
20.5641025641026 0.212581545114517
21.5416666666667 0.19132462143898
22.5673076923077 0.207483127713203
23.6442307692308 0.191727444529533
24.7692307692308 0.185348510742188
25.9487179487179 0.169403895735741
27.1826923076923 0.191393688321114
28.4775641025641 0.181521043181419
29.8333333333333 0.191006451845169
31.2564102564103 0.184154734015465
32.7435897435897 0.176578268408775
34.3044871794872 0.160316348075867
35.9358974358974 0.171448990702629
37.6474358974359 0.186104118824005
39.4391025641026 0.165023446083069
41.3173076923077 0.187238797545433
43.2852564102564 0.156787440180779
45.3461538461538 0.154439687728882
47.5064102564103 0.140450239181519
49.7692307692308 0.145386666059494
52.1378205128205 0.121340773999691
54.6217948717949 0.143825083971024
57.2211538461538 0.130993768572807
59.9455128205128 0.141369804739952
62.8012820512821 0.135775536298752
65.7916666666667 0.1395603120327
68.9230769230769 0.12280859798193
72.2051282051282 0.126340761780739
75.6442307692308 0.122518993914127
79.2467948717949 0.122738234698772
83.0192307692308 0.111788801848888
86.974358974359 0.10277109593153
91.1153846153846 0.114183381199837
95.4519230769231 0.111956752836704
100 0.1044527515769
};
\addplot [, color1, opacity=0.6, mark=square*, mark size=0.5, mark options={solid}, only marks]
table {%
1 0.777247369289398
1.04487179487179 0.770448386669159
1.09615384615385 0.803281962871552
1.1474358974359 0.834520161151886
1.20192307692308 0.872211098670959
1.25961538461538 0.871958553791046
1.32051282051282 0.841903150081635
1.38461538461538 0.8604736328125
1.44871794871795 0.879635751247406
1.51923076923077 0.848080456256866
1.58974358974359 0.836516857147217
1.66666666666667 0.818723857402802
1.74679487179487 0.792382717132568
1.83012820512821 0.787184357643127
1.91666666666667 0.766308486461639
2.00641025641026 0.756807565689087
2.1025641025641 0.750781238079071
2.20512820512821 0.732520520687103
2.30769230769231 0.711338639259338
2.41987179487179 0.687405049800873
2.53525641025641 0.661563098430634
2.65384615384615 0.628782272338867
2.78205128205128 0.613285064697266
2.91346153846154 0.596566617488861
3.05128205128205 0.570539176464081
3.19871794871795 0.536603391170502
3.34935897435897 0.559958875179291
3.50961538461538 0.559679687023163
3.67628205128205 0.488923400640488
3.8525641025641 0.484778791666031
4.03525641025641 0.500897228717804
4.2275641025641 0.45257568359375
4.42948717948718 0.438570111989975
4.64102564102564 0.44877552986145
4.86217948717949 0.429499715566635
5.09294871794872 0.439673900604248
5.33653846153846 0.397564023733139
5.58974358974359 0.383366823196411
5.85576923076923 0.391532182693481
6.13461538461539 0.394979059696198
6.42628205128205 0.397225320339203
6.73397435897436 0.357703119516373
7.05448717948718 0.325975149869919
7.38782051282051 0.341578394174576
7.74038461538461 0.360099494457245
8.10897435897436 0.360215574502945
8.49679487179487 0.299640864133835
8.90064102564103 0.333866506814957
9.32371794871795 0.30286979675293
9.76923076923077 0.301265954971313
10.2339743589744 0.301431447267532
10.7211538461538 0.295453041791916
11.2307692307692 0.278884559869766
11.7660256410256 0.308207601308823
12.3269230769231 0.278986752033234
12.9134615384615 0.26557445526123
13.5288461538462 0.292152255773544
14.1730769230769 0.247988864779472
14.849358974359 0.249606162309647
15.5544871794872 0.279840379953384
16.2948717948718 0.284491121768951
17.0705128205128 0.282015144824982
17.8846153846154 0.265656620264053
18.7371794871795 0.258210599422455
19.6282051282051 0.24965600669384
20.5641025641026 0.257215738296509
21.5416666666667 0.225708723068237
22.5673076923077 0.23987241089344
23.6442307692308 0.212257727980614
24.7692307692308 0.215380147099495
25.9487179487179 0.19194233417511
27.1826923076923 0.230384543538094
28.4775641025641 0.186996221542358
29.8333333333333 0.20021764934063
31.2564102564103 0.173916697502136
32.7435897435897 0.157003805041313
34.3044871794872 0.162724554538727
35.9358974358974 0.199153184890747
37.6474358974359 0.1612488925457
39.4391025641026 0.169204756617546
41.3173076923077 0.169399812817574
43.2852564102564 0.156952604651451
45.3461538461538 0.13326258957386
47.5064102564103 0.157900258898735
49.7692307692308 0.151678755879402
52.1378205128205 0.130881577730179
54.6217948717949 0.172865629196167
57.2211538461538 0.10993454605341
59.9455128205128 0.140342459082603
62.8012820512821 0.111761748790741
65.7916666666667 0.135233223438263
68.9230769230769 0.0985468775033951
72.2051282051282 0.107280112802982
75.6442307692308 0.121876381337643
79.2467948717949 0.0944937095046043
83.0192307692308 0.105866812169552
86.974358974359 0.101951658725739
91.1153846153846 0.102934278547764
95.4519230769231 0.109189748764038
100 0.0980146899819374
};
\addlegendentry{mb 8, exact}
\addplot [, color1, opacity=0.6, mark=square*, mark size=0.5, mark options={solid}, only marks, forget plot]
table {%
1 0.80650120973587
1.04487179487179 0.821713924407959
1.09615384615385 0.846667110919952
1.1474358974359 0.879475891590118
1.20192307692308 0.886012375354767
1.25961538461538 0.885229706764221
1.32051282051282 0.866720676422119
1.38461538461538 0.878308951854706
1.44871794871795 0.889364838600159
1.51923076923077 0.857218861579895
1.58974358974359 0.857651233673096
1.66666666666667 0.771526575088501
1.74679487179487 0.848896622657776
1.83012820512821 0.774302363395691
1.91666666666667 0.749237477779388
2.00641025641026 0.775936841964722
2.1025641025641 0.711972415447235
2.20512820512821 0.707637846469879
2.30769230769231 0.744617640972137
2.41987179487179 0.687649488449097
2.53525641025641 0.663406848907471
2.65384615384615 0.661353290081024
2.78205128205128 0.661773800849915
2.91346153846154 0.627224624156952
3.05128205128205 0.598433494567871
3.19871794871795 0.553872883319855
3.34935897435897 0.579225957393646
3.50961538461538 0.541127860546112
3.67628205128205 0.494524478912354
3.8525641025641 0.500620543956757
4.03525641025641 0.460244804620743
4.2275641025641 0.51121586561203
4.42948717948718 0.480776607990265
4.64102564102564 0.45650252699852
4.86217948717949 0.438486099243164
5.09294871794872 0.417048275470734
5.33653846153846 0.444022655487061
5.58974358974359 0.404520004987717
5.85576923076923 0.399827271699905
6.13461538461539 0.394353955984116
6.42628205128205 0.419809192419052
6.73397435897436 0.371141403913498
7.05448717948718 0.392766296863556
7.38782051282051 0.372695922851562
7.74038461538461 0.362761110067368
8.10897435897436 0.376328468322754
8.49679487179487 0.368702739477158
8.90064102564103 0.364896714687347
9.32371794871795 0.347817301750183
9.76923076923077 0.369773596525192
10.2339743589744 0.32089838385582
10.7211538461538 0.287603288888931
11.2307692307692 0.326994985342026
11.7660256410256 0.324010074138641
12.3269230769231 0.303202301263809
12.9134615384615 0.305205971002579
13.5288461538462 0.289823979139328
14.1730769230769 0.31696605682373
14.849358974359 0.28064638376236
15.5544871794872 0.277780413627625
16.2948717948718 0.289385050535202
17.0705128205128 0.262940227985382
17.8846153846154 0.278089195489883
18.7371794871795 0.27421972155571
19.6282051282051 0.267389208078384
20.5641025641026 0.240693598985672
21.5416666666667 0.258440762758255
22.5673076923077 0.23048098385334
23.6442307692308 0.231325432658195
24.7692307692308 0.248114019632339
25.9487179487179 0.224651649594307
27.1826923076923 0.23150072991848
28.4775641025641 0.212005808949471
29.8333333333333 0.230271726846695
31.2564102564103 0.213032558560371
32.7435897435897 0.208778187632561
34.3044871794872 0.204019069671631
35.9358974358974 0.203365445137024
37.6474358974359 0.215405061841011
39.4391025641026 0.189480572938919
41.3173076923077 0.182646736502647
43.2852564102564 0.163511916995049
45.3461538461538 0.202956065535545
47.5064102564103 0.181064739823341
49.7692307692308 0.181889921426773
52.1378205128205 0.172009155154228
54.6217948717949 0.154074549674988
57.2211538461538 0.157865285873413
59.9455128205128 0.15925757586956
62.8012820512821 0.15394501388073
65.7916666666667 0.146935254335403
68.9230769230769 0.136101558804512
72.2051282051282 0.132477253675461
75.6442307692308 0.129364117980003
79.2467948717949 0.137396246194839
83.0192307692308 0.145724207162857
86.974358974359 0.156431078910828
91.1153846153846 0.152589112520218
95.4519230769231 0.112779669463634
100 0.128511294722557
};
\addplot [, color1, opacity=0.6, mark=square*, mark size=0.5, mark options={solid}, only marks, forget plot]
table {%
1 0.822732448577881
1.04487179487179 0.824281334877014
1.09615384615385 0.862076282501221
1.1474358974359 0.886173248291016
1.20192307692308 0.915951728820801
1.25961538461538 0.907984673976898
1.32051282051282 0.869018733501434
1.38461538461538 0.867047488689423
1.44871794871795 0.884066700935364
1.51923076923077 0.863296031951904
1.58974358974359 0.798871695995331
1.66666666666667 0.79884934425354
1.74679487179487 0.745267391204834
1.83012820512821 0.741233289241791
1.91666666666667 0.721401631832123
2.00641025641026 0.693054139614105
2.1025641025641 0.699892103672028
2.20512820512821 0.660626590251923
2.30769230769231 0.583081126213074
2.41987179487179 0.612829327583313
2.53525641025641 0.631756186485291
2.65384615384615 0.537929356098175
2.78205128205128 0.607859075069427
2.91346153846154 0.529702663421631
3.05128205128205 0.569072782993317
3.19871794871795 0.508898198604584
3.34935897435897 0.53626012802124
3.50961538461538 0.51501053571701
3.67628205128205 0.514332592487335
3.8525641025641 0.479987055063248
4.03525641025641 0.485254853963852
4.2275641025641 0.491164892911911
4.42948717948718 0.499634563922882
4.64102564102564 0.456337213516235
4.86217948717949 0.413573116064072
5.09294871794872 0.432276219129562
5.33653846153846 0.418748944997787
5.58974358974359 0.438474148511887
5.85576923076923 0.397523790597916
6.13461538461539 0.435563653707504
6.42628205128205 0.428228199481964
6.73397435897436 0.400451749563217
7.05448717948718 0.376968204975128
7.38782051282051 0.378808587789536
7.74038461538461 0.367855787277222
8.10897435897436 0.390940517187119
8.49679487179487 0.340040653944016
8.90064102564103 0.346644431352615
9.32371794871795 0.335524648427963
9.76923076923077 0.344602167606354
10.2339743589744 0.313341945409775
10.7211538461538 0.342174410820007
11.2307692307692 0.313525289297104
11.7660256410256 0.317241787910461
12.3269230769231 0.302724123001099
12.9134615384615 0.300145953893661
13.5288461538462 0.268837749958038
14.1730769230769 0.274399518966675
14.849358974359 0.27524733543396
15.5544871794872 0.254760235548019
16.2948717948718 0.244571357965469
17.0705128205128 0.267859518527985
17.8846153846154 0.278651714324951
18.7371794871795 0.249213144183159
19.6282051282051 0.252094268798828
20.5641025641026 0.225551083683968
21.5416666666667 0.237385079264641
22.5673076923077 0.215372905135155
23.6442307692308 0.230469092726707
24.7692307692308 0.219263538718224
25.9487179487179 0.178676664829254
27.1826923076923 0.196067586541176
28.4775641025641 0.192209377884865
29.8333333333333 0.232647761702538
31.2564102564103 0.207955032587051
32.7435897435897 0.191949233412743
34.3044871794872 0.18684609234333
35.9358974358974 0.158464059233665
37.6474358974359 0.162177518010139
39.4391025641026 0.166888028383255
41.3173076923077 0.149999007582664
43.2852564102564 0.153630018234253
45.3461538461538 0.165447860956192
47.5064102564103 0.143311947584152
49.7692307692308 0.147649720311165
52.1378205128205 0.131863236427307
54.6217948717949 0.139704301953316
57.2211538461538 0.115432880818844
59.9455128205128 0.107081584632397
62.8012820512821 0.136034920811653
65.7916666666667 0.129533633589745
68.9230769230769 0.129929646849632
72.2051282051282 0.115943811833858
75.6442307692308 0.107810162007809
79.2467948717949 0.105069078505039
83.0192307692308 0.0990055128931999
86.974358974359 0.0766728147864342
91.1153846153846 0.0973830297589302
95.4519230769231 0.0907079502940178
100 0.0839100256562233
};
\addplot [, color1, opacity=0.6, mark=square*, mark size=0.5, mark options={solid}, only marks, forget plot]
table {%
1 0.832188427448273
1.04487179487179 0.819915950298309
1.09615384615385 0.854742050170898
1.1474358974359 0.878585994243622
1.20192307692308 0.884367763996124
1.25961538461538 0.892147064208984
1.32051282051282 0.795580983161926
1.38461538461538 0.783761441707611
1.44871794871795 0.791585922241211
1.51923076923077 0.739203274250031
1.58974358974359 0.687588512897491
1.66666666666667 0.716758489608765
1.74679487179487 0.734555542469025
1.83012820512821 0.643817782402039
1.91666666666667 0.710172474384308
2.00641025641026 0.667308747768402
2.1025641025641 0.628762304782867
2.20512820512821 0.68874591588974
2.30769230769231 0.664266049861908
2.41987179487179 0.688416302204132
2.53525641025641 0.600337088108063
2.65384615384615 0.615456581115723
2.78205128205128 0.574759602546692
2.91346153846154 0.588202595710754
3.05128205128205 0.546380221843719
3.19871794871795 0.605657815933228
3.34935897435897 0.579697549343109
3.50961538461538 0.543127000331879
3.67628205128205 0.560913503170013
3.8525641025641 0.532085835933685
4.03525641025641 0.561569571495056
4.2275641025641 0.523038327693939
4.42948717948718 0.49079155921936
4.64102564102564 0.497922241687775
4.86217948717949 0.490234583616257
5.09294871794872 0.488036453723907
5.33653846153846 0.465779215097427
5.58974358974359 0.495301306247711
5.85576923076923 0.493101507425308
6.13461538461539 0.439215958118439
6.42628205128205 0.47782489657402
6.73397435897436 0.407924383878708
7.05448717948718 0.444955259561539
7.38782051282051 0.449209421873093
7.74038461538461 0.408889144659042
8.10897435897436 0.377460926771164
8.49679487179487 0.394536972045898
8.90064102564103 0.397503942251205
9.32371794871795 0.363994807004929
9.76923076923077 0.341746091842651
10.2339743589744 0.347708225250244
10.7211538461538 0.352140575647354
11.2307692307692 0.337055265903473
11.7660256410256 0.349001258611679
12.3269230769231 0.318394392728806
12.9134615384615 0.297825664281845
13.5288461538462 0.330868780612946
14.1730769230769 0.294612646102905
14.849358974359 0.298680365085602
15.5544871794872 0.307645857334137
16.2948717948718 0.332239598035812
17.0705128205128 0.326354563236237
17.8846153846154 0.286406844854355
18.7371794871795 0.296731948852539
19.6282051282051 0.301868259906769
20.5641025641026 0.283962309360504
21.5416666666667 0.278897285461426
22.5673076923077 0.268107652664185
23.6442307692308 0.247674375772476
24.7692307692308 0.258186638355255
25.9487179487179 0.259999960660934
27.1826923076923 0.238257318735123
28.4775641025641 0.222829386591911
29.8333333333333 0.265736550092697
31.2564102564103 0.247794985771179
32.7435897435897 0.210168316960335
34.3044871794872 0.223555564880371
35.9358974358974 0.196300163865089
37.6474358974359 0.218035131692886
39.4391025641026 0.21666131913662
41.3173076923077 0.201963424682617
43.2852564102564 0.161626935005188
45.3461538461538 0.178665056824684
47.5064102564103 0.18268945813179
49.7692307692308 0.158858358860016
52.1378205128205 0.18066842854023
54.6217948717949 0.211129054427147
57.2211538461538 0.164845615625381
59.9455128205128 0.155562683939934
62.8012820512821 0.151147231459618
65.7916666666667 0.137063190340996
68.9230769230769 0.130773141980171
72.2051282051282 0.124488286674023
75.6442307692308 0.118389323353767
79.2467948717949 0.128397688269615
83.0192307692308 0.112234406173229
86.974358974359 0.129150375723839
91.1153846153846 0.15471588075161
95.4519230769231 0.1021488904953
100 0.12485633045435
};
\addplot [, color1, opacity=0.6, mark=square*, mark size=0.5, mark options={solid}, only marks, forget plot]
table {%
1 0.729271829128265
1.04487179487179 0.755219519138336
1.09615384615385 0.775568544864655
1.1474358974359 0.810066521167755
1.20192307692308 0.844181180000305
1.25961538461538 0.863144874572754
1.32051282051282 0.842780530452728
1.38461538461538 0.834303319454193
1.44871794871795 0.869764745235443
1.51923076923077 0.832343876361847
1.58974358974359 0.818791806697845
1.66666666666667 0.792705655097961
1.74679487179487 0.779376447200775
1.83012820512821 0.724467575550079
1.91666666666667 0.721819937229156
2.00641025641026 0.766333758831024
2.1025641025641 0.687975108623505
2.20512820512821 0.692751228809357
2.30769230769231 0.653091728687286
2.41987179487179 0.658458530902863
2.53525641025641 0.613855719566345
2.65384615384615 0.578324735164642
2.78205128205128 0.617970168590546
2.91346153846154 0.626808941364288
3.05128205128205 0.577496647834778
3.19871794871795 0.556955456733704
3.34935897435897 0.557727217674255
3.50961538461538 0.558555424213409
3.67628205128205 0.533319890499115
3.8525641025641 0.509102284908295
4.03525641025641 0.497951835393906
4.2275641025641 0.502053439617157
4.42948717948718 0.494830101728439
4.64102564102564 0.445264160633087
4.86217948717949 0.456398814916611
5.09294871794872 0.424147427082062
5.33653846153846 0.439424127340317
5.58974358974359 0.406697273254395
5.85576923076923 0.422736018896103
6.13461538461539 0.437953561544418
6.42628205128205 0.413424491882324
6.73397435897436 0.363414376974106
7.05448717948718 0.369961351156235
7.38782051282051 0.370476692914963
7.74038461538461 0.380969017744064
8.10897435897436 0.404207855463028
8.49679487179487 0.363017469644547
8.90064102564103 0.353272914886475
9.32371794871795 0.325083076953888
9.76923076923077 0.342595815658569
10.2339743589744 0.305660337209702
10.7211538461538 0.34252792596817
11.2307692307692 0.299053519964218
11.7660256410256 0.339979976415634
12.3269230769231 0.327261954545975
12.9134615384615 0.278658360242844
13.5288461538462 0.314245194196701
14.1730769230769 0.306252837181091
14.849358974359 0.299396336078644
15.5544871794872 0.315703809261322
16.2948717948718 0.296998023986816
17.0705128205128 0.288243621587753
17.8846153846154 0.284331321716309
18.7371794871795 0.274262011051178
19.6282051282051 0.284875363111496
20.5641025641026 0.262186765670776
21.5416666666667 0.263358652591705
22.5673076923077 0.251400768756866
23.6442307692308 0.259766846895218
24.7692307692308 0.225276932120323
25.9487179487179 0.259557038545609
27.1826923076923 0.243178889155388
28.4775641025641 0.243529915809631
29.8333333333333 0.221827939152718
31.2564102564103 0.233973607420921
32.7435897435897 0.228265836834908
34.3044871794872 0.212328478693962
35.9358974358974 0.228460907936096
37.6474358974359 0.194003537297249
39.4391025641026 0.206472739577293
41.3173076923077 0.198425635695457
43.2852564102564 0.213042497634888
45.3461538461538 0.187539830803871
47.5064102564103 0.173514053225517
49.7692307692308 0.160965517163277
52.1378205128205 0.157441332936287
54.6217948717949 0.166003942489624
57.2211538461538 0.176121026277542
59.9455128205128 0.163126260042191
62.8012820512821 0.142935663461685
65.7916666666667 0.134364292025566
68.9230769230769 0.143608137965202
72.2051282051282 0.130777627229691
75.6442307692308 0.136319518089294
79.2467948717949 0.141721427440643
83.0192307692308 0.130325794219971
86.974358974359 0.129972085356712
91.1153846153846 0.118830107152462
95.4519230769231 0.125920921564102
100 0.126060917973518
};
\addplot [, color2, opacity=0.6, mark=diamond*, mark size=0.5, mark options={solid}, only marks]
table {%
1 0.932628273963928
1.04487179487179 0.938291192054749
1.09615384615385 0.95775955915451
1.1474358974359 0.968070209026337
1.20192307692308 0.974394500255585
1.25961538461538 0.973689079284668
1.32051282051282 0.966567993164062
1.38461538461538 0.963431000709534
1.44871794871795 0.966016590595245
1.51923076923077 0.94951456785202
1.58974358974359 0.950101315975189
1.66666666666667 0.88587874174118
1.74679487179487 0.940914332866669
1.83012820512821 0.84552401304245
1.91666666666667 0.913096070289612
2.00641025641026 0.922969281673431
2.1025641025641 0.909432411193848
2.20512820512821 0.893574893474579
2.30769230769231 0.884677529335022
2.41987179487179 0.877425014972687
2.53525641025641 0.880024135112762
2.65384615384615 0.889153122901917
2.78205128205128 0.802811801433563
2.91346153846154 0.790535628795624
3.05128205128205 0.766136646270752
3.19871794871795 0.76026064157486
3.34935897435897 0.735775649547577
3.50961538461538 0.750852048397064
3.67628205128205 0.78460294008255
3.8525641025641 0.770893275737762
4.03525641025641 0.747582137584686
4.2275641025641 0.701965272426605
4.42948717948718 0.698621392250061
4.64102564102564 0.728967428207397
4.86217948717949 0.711494028568268
5.09294871794872 0.744661927223206
5.33653846153846 0.720024466514587
5.58974358974359 0.675424695014954
5.85576923076923 0.653375327587128
6.13461538461539 0.675180852413177
6.42628205128205 0.664807617664337
6.73397435897436 0.660577178001404
7.05448717948718 0.672150552272797
7.38782051282051 0.617719173431396
7.74038461538461 0.639324367046356
8.10897435897436 0.584364354610443
8.49679487179487 0.633070945739746
8.90064102564103 0.643986284732819
9.32371794871795 0.544728577136993
9.76923076923077 0.521876454353333
10.2339743589744 0.54596072435379
10.7211538461538 0.558124899864197
11.2307692307692 0.602697789669037
11.7660256410256 0.552951753139496
12.3269230769231 0.537913501262665
12.9134615384615 0.587661921977997
13.5288461538462 0.457116335630417
14.1730769230769 0.534262597560883
14.849358974359 0.506164252758026
15.5544871794872 0.430409163236618
16.2948717948718 0.452758133411407
17.0705128205128 0.471149921417236
17.8846153846154 0.440868765115738
18.7371794871795 0.411999464035034
19.6282051282051 0.399665117263794
20.5641025641026 0.407514631748199
21.5416666666667 0.459341675043106
22.5673076923077 0.387651532888412
23.6442307692308 0.345926523208618
24.7692307692308 0.354522913694382
25.9487179487179 0.352641344070435
27.1826923076923 0.37148904800415
28.4775641025641 0.328543424606323
29.8333333333333 0.344101250171661
31.2564102564103 0.32806059718132
32.7435897435897 0.294717878103256
34.3044871794872 0.263706922531128
35.9358974358974 0.258626490831375
37.6474358974359 0.273335069417953
39.4391025641026 0.310210436582565
41.3173076923077 0.273879289627075
43.2852564102564 0.263814598321915
45.3461538461538 0.228401228785515
47.5064102564103 0.209037065505981
49.7692307692308 0.228399440646172
52.1378205128205 0.222367271780968
54.6217948717949 0.19845524430275
57.2211538461538 0.219551682472229
59.9455128205128 0.188205301761627
62.8012820512821 0.214487314224243
65.7916666666667 0.14883154630661
68.9230769230769 0.188908979296684
72.2051282051282 0.173554807901382
75.6442307692308 0.159559071063995
79.2467948717949 0.14506408572197
83.0192307692308 0.17661364376545
86.974358974359 0.17196224629879
91.1153846153846 0.154570445418358
95.4519230769231 0.127721652388573
100 0.140174746513367
};
\addlegendentry{mb 32, exact}
\addplot [, color2, opacity=0.6, mark=diamond*, mark size=0.5, mark options={solid}, only marks, forget plot]
table {%
1 0.92177402973175
1.04487179487179 0.931659519672394
1.09615384615385 0.952114701271057
1.1474358974359 0.961091816425323
1.20192307692308 0.969603180885315
1.25961538461538 0.963872909545898
1.32051282051282 0.961275100708008
1.38461538461538 0.960747718811035
1.44871794871795 0.967851579189301
1.51923076923077 0.955938518047333
1.58974358974359 0.949824512004852
1.66666666666667 0.940075874328613
1.74679487179487 0.937249958515167
1.83012820512821 0.931238174438477
1.91666666666667 0.930560529232025
2.00641025641026 0.925332248210907
2.1025641025641 0.918525159358978
2.20512820512821 0.919639050960541
2.30769230769231 0.902421474456787
2.41987179487179 0.898367047309875
2.53525641025641 0.895472168922424
2.65384615384615 0.870235085487366
2.78205128205128 0.813703000545502
2.91346153846154 0.848914563655853
3.05128205128205 0.829177856445312
3.19871794871795 0.841319680213928
3.34935897435897 0.795029997825623
3.50961538461538 0.80362743139267
3.67628205128205 0.728380918502808
3.8525641025641 0.759848058223724
4.03525641025641 0.778432786464691
4.2275641025641 0.765615165233612
4.42948717948718 0.716013610363007
4.64102564102564 0.729758441448212
4.86217948717949 0.743451416492462
5.09294871794872 0.713679730892181
5.33653846153846 0.732860624790192
5.58974358974359 0.719551026821136
5.85576923076923 0.673462510108948
6.13461538461539 0.667547583580017
6.42628205128205 0.668962240219116
6.73397435897436 0.641673862934113
7.05448717948718 0.661937892436981
7.38782051282051 0.642151653766632
7.74038461538461 0.648819148540497
8.10897435897436 0.627889037132263
8.49679487179487 0.647981643676758
8.90064102564103 0.651549279689789
9.32371794871795 0.643559575080872
9.76923076923077 0.571582794189453
10.2339743589744 0.562951982021332
10.7211538461538 0.558732450008392
11.2307692307692 0.480571955442429
11.7660256410256 0.538775503635406
12.3269230769231 0.52058619260788
12.9134615384615 0.513228118419647
13.5288461538462 0.502796471118927
14.1730769230769 0.487495422363281
14.849358974359 0.507695853710175
15.5544871794872 0.472815662622452
16.2948717948718 0.481665134429932
17.0705128205128 0.487552642822266
17.8846153846154 0.462441056966782
18.7371794871795 0.410538733005524
19.6282051282051 0.39987388253212
20.5641025641026 0.401975303888321
21.5416666666667 0.37185201048851
22.5673076923077 0.395497411489487
23.6442307692308 0.375535815954208
24.7692307692308 0.337764263153076
25.9487179487179 0.340410709381104
27.1826923076923 0.318518370389938
28.4775641025641 0.35831543803215
29.8333333333333 0.340592473745346
31.2564102564103 0.329118221998215
32.7435897435897 0.291967242956161
34.3044871794872 0.284279346466064
35.9358974358974 0.309867918491364
37.6474358974359 0.296899706125259
39.4391025641026 0.285943508148193
41.3173076923077 0.243596956133842
43.2852564102564 0.233159527182579
45.3461538461538 0.267334848642349
47.5064102564103 0.256620615720749
49.7692307692308 0.24321885406971
52.1378205128205 0.223377659916878
54.6217948717949 0.23184834420681
57.2211538461538 0.172045648097992
59.9455128205128 0.178952977061272
62.8012820512821 0.177083283662796
65.7916666666667 0.179162889719009
68.9230769230769 0.18260882794857
72.2051282051282 0.165032982826233
75.6442307692308 0.154753610491753
79.2467948717949 0.131483241915703
83.0192307692308 0.169182419776917
86.974358974359 0.149082258343697
91.1153846153846 0.129810333251953
95.4519230769231 0.153520584106445
100 0.150085091590881
};
\addplot [, color2, opacity=0.6, mark=diamond*, mark size=0.5, mark options={solid}, only marks, forget plot]
table {%
1 0.902037560939789
1.04487179487179 0.870559334754944
1.09615384615385 0.944737076759338
1.1474358974359 0.951055824756622
1.20192307692308 0.962268054485321
1.25961538461538 0.961669385433197
1.32051282051282 0.946449220180511
1.38461538461538 0.94780695438385
1.44871794871795 0.951630592346191
1.51923076923077 0.933424413204193
1.58974358974359 0.92816162109375
1.66666666666667 0.922088444232941
1.74679487179487 0.917732238769531
1.83012820512821 0.910565555095673
1.91666666666667 0.913563430309296
2.00641025641026 0.892433166503906
2.1025641025641 0.898369014263153
2.20512820512821 0.897662937641144
2.30769230769231 0.859035491943359
2.41987179487179 0.84860897064209
2.53525641025641 0.865632176399231
2.65384615384615 0.836091697216034
2.78205128205128 0.846567571163177
2.91346153846154 0.763507723808289
3.05128205128205 0.801495552062988
3.19871794871795 0.764559388160706
3.34935897435897 0.749117136001587
3.50961538461538 0.781613945960999
3.67628205128205 0.788118302822113
3.8525641025641 0.762565493583679
4.03525641025641 0.734131634235382
4.2275641025641 0.731942594051361
4.42948717948718 0.757953405380249
4.64102564102564 0.696383655071259
4.86217948717949 0.715134859085083
5.09294871794872 0.674463748931885
5.33653846153846 0.686055183410645
5.58974358974359 0.684452950954437
5.85576923076923 0.633180260658264
6.13461538461539 0.637839615345001
6.42628205128205 0.641892850399017
6.73397435897436 0.606902897357941
7.05448717948718 0.576881349086761
7.38782051282051 0.628829598426819
7.74038461538461 0.580028712749481
8.10897435897436 0.56657201051712
8.49679487179487 0.597490787506104
8.90064102564103 0.60525780916214
9.32371794871795 0.599415004253387
9.76923076923077 0.556484639644623
10.2339743589744 0.557806253433228
10.7211538461538 0.50328004360199
11.2307692307692 0.51010799407959
11.7660256410256 0.536683022975922
12.3269230769231 0.532055497169495
12.9134615384615 0.493945330381393
13.5288461538462 0.48908343911171
14.1730769230769 0.438534557819366
14.849358974359 0.494656294584274
15.5544871794872 0.452489197254181
16.2948717948718 0.44525671005249
17.0705128205128 0.46257096529007
17.8846153846154 0.433210283517838
18.7371794871795 0.430849313735962
19.6282051282051 0.375973552465439
20.5641025641026 0.361198425292969
21.5416666666667 0.470883578062057
22.5673076923077 0.392141073942184
23.6442307692308 0.366648644208908
24.7692307692308 0.383997410535812
25.9487179487179 0.378357857465744
27.1826923076923 0.364077657461166
28.4775641025641 0.318030089139938
29.8333333333333 0.319372326135635
31.2564102564103 0.293523639440536
32.7435897435897 0.309991121292114
34.3044871794872 0.328725308179855
35.9358974358974 0.255352228879929
37.6474358974359 0.242363438010216
39.4391025641026 0.222953990101814
41.3173076923077 0.251070499420166
43.2852564102564 0.235842660069466
45.3461538461538 0.268515825271606
47.5064102564103 0.256174296140671
49.7692307692308 0.209148988127708
52.1378205128205 0.212478324770927
54.6217948717949 0.196045160293579
57.2211538461538 0.196020394563675
59.9455128205128 0.186025947332382
62.8012820512821 0.216279983520508
65.7916666666667 0.167298391461372
68.9230769230769 0.171931236982346
72.2051282051282 0.148505434393883
75.6442307692308 0.148019716143608
79.2467948717949 0.155865892767906
83.0192307692308 0.137825831770897
86.974358974359 0.143041059374809
91.1153846153846 0.137221693992615
95.4519230769231 0.136360466480255
100 0.120750807225704
};
\addplot [, color2, opacity=0.6, mark=diamond*, mark size=0.5, mark options={solid}, only marks, forget plot]
table {%
1 0.93377411365509
1.04487179487179 0.933065116405487
1.09615384615385 0.954798221588135
1.1474358974359 0.961155593395233
1.20192307692308 0.964705288410187
1.25961538461538 0.963135242462158
1.32051282051282 0.951974093914032
1.38461538461538 0.95636510848999
1.44871794871795 0.965109348297119
1.51923076923077 0.951481640338898
1.58974358974359 0.950816571712494
1.66666666666667 0.944420754909515
1.74679487179487 0.942672252655029
1.83012820512821 0.94032746553421
1.91666666666667 0.926631569862366
2.00641025641026 0.932477653026581
2.1025641025641 0.903216779232025
2.20512820512821 0.90234911441803
2.30769230769231 0.887151718139648
2.41987179487179 0.88903933763504
2.53525641025641 0.889262974262238
2.65384615384615 0.883346021175385
2.78205128205128 0.862769424915314
2.91346153846154 0.832723259925842
3.05128205128205 0.829456925392151
3.19871794871795 0.828689515590668
3.34935897435897 0.810957849025726
3.50961538461538 0.79102611541748
3.67628205128205 0.783592879772186
3.8525641025641 0.757906556129456
4.03525641025641 0.734292447566986
4.2275641025641 0.773223519325256
4.42948717948718 0.74751877784729
4.64102564102564 0.699628531932831
4.86217948717949 0.703737437725067
5.09294871794872 0.765421867370605
5.33653846153846 0.791190683841705
5.58974358974359 0.734164893627167
5.85576923076923 0.657411515712738
6.13461538461539 0.6817946434021
6.42628205128205 0.637957811355591
6.73397435897436 0.696737170219421
7.05448717948718 0.672457695007324
7.38782051282051 0.642049610614777
7.74038461538461 0.644884288311005
8.10897435897436 0.56655877828598
8.49679487179487 0.645265579223633
8.90064102564103 0.597652673721313
9.32371794871795 0.64038360118866
9.76923076923077 0.55820620059967
10.2339743589744 0.546125471591949
10.7211538461538 0.508544087409973
11.2307692307692 0.440067201852798
11.7660256410256 0.453912973403931
12.3269230769231 0.450243383646011
12.9134615384615 0.466993302106857
13.5288461538462 0.401077032089233
14.1730769230769 0.384100407361984
14.849358974359 0.362000167369843
15.5544871794872 0.36279234290123
16.2948717948718 0.3930604159832
17.0705128205128 0.38768607378006
17.8846153846154 0.367034196853638
18.7371794871795 0.345895737409592
19.6282051282051 0.31819748878479
20.5641025641026 0.334131568670273
21.5416666666667 0.338995784521103
22.5673076923077 0.274976491928101
23.6442307692308 0.336225718259811
24.7692307692308 0.301344275474548
25.9487179487179 0.287796378135681
27.1826923076923 0.322814971208572
28.4775641025641 0.309925615787506
29.8333333333333 0.285752892494202
31.2564102564103 0.251562595367432
32.7435897435897 0.248492434620857
34.3044871794872 0.265980839729309
35.9358974358974 0.258852332830429
37.6474358974359 0.291488826274872
39.4391025641026 0.241829872131348
41.3173076923077 0.205918431282043
43.2852564102564 0.213039681315422
45.3461538461538 0.203236728906631
47.5064102564103 0.220217376947403
49.7692307692308 0.203176781535149
52.1378205128205 0.18151481449604
54.6217948717949 0.156794488430023
57.2211538461538 0.224233508110046
59.9455128205128 0.153899446129799
62.8012820512821 0.169163212180138
65.7916666666667 0.166928142309189
68.9230769230769 0.155261501669884
72.2051282051282 0.168304517865181
75.6442307692308 0.175465226173401
79.2467948717949 0.122561097145081
83.0192307692308 0.149315819144249
86.974358974359 0.164364844560623
91.1153846153846 0.13106207549572
95.4519230769231 0.135219275951385
100 0.159893423318863
};
\addplot [, color2, opacity=0.6, mark=diamond*, mark size=0.5, mark options={solid}, only marks, forget plot]
table {%
1 0.918709754943848
1.04487179487179 0.895618438720703
1.09615384615385 0.944429397583008
1.1474358974359 0.952463150024414
1.20192307692308 0.961049675941467
1.25961538461538 0.955068707466125
1.32051282051282 0.948335349559784
1.38461538461538 0.949211716651917
1.44871794871795 0.960704624652863
1.51923076923077 0.949916064739227
1.58974358974359 0.942548930644989
1.66666666666667 0.931734025478363
1.74679487179487 0.929324924945831
1.83012820512821 0.926349639892578
1.91666666666667 0.903291702270508
2.00641025641026 0.920504868030548
2.1025641025641 0.899483621120453
2.20512820512821 0.889619529247284
2.30769230769231 0.859867691993713
2.41987179487179 0.86351090669632
2.53525641025641 0.836765229701996
2.65384615384615 0.815252482891083
2.78205128205128 0.825437963008881
2.91346153846154 0.783214330673218
3.05128205128205 0.78251326084137
3.19871794871795 0.7662553191185
3.34935897435897 0.777276813983917
3.50961538461538 0.78879177570343
3.67628205128205 0.763471245765686
3.8525641025641 0.772683084011078
4.03525641025641 0.747499287128448
4.2275641025641 0.724232017993927
4.42948717948718 0.745960593223572
4.64102564102564 0.704512298107147
4.86217948717949 0.719797253608704
5.09294871794872 0.705552995204926
5.33653846153846 0.712149500846863
5.58974358974359 0.719823360443115
5.85576923076923 0.658820450305939
6.13461538461539 0.661200225353241
6.42628205128205 0.717771112918854
6.73397435897436 0.652535259723663
7.05448717948718 0.613766133785248
7.38782051282051 0.619120538234711
7.74038461538461 0.658773124217987
8.10897435897436 0.608645617961884
8.49679487179487 0.632529437541962
8.90064102564103 0.608493030071259
9.32371794871795 0.586467504501343
9.76923076923077 0.569081962108612
10.2339743589744 0.560609817504883
10.7211538461538 0.536305010318756
11.2307692307692 0.545849800109863
11.7660256410256 0.552339255809784
12.3269230769231 0.499271959066391
12.9134615384615 0.49711799621582
13.5288461538462 0.452871292829514
14.1730769230769 0.50910872220993
14.849358974359 0.461415857076645
15.5544871794872 0.506064653396606
16.2948717948718 0.466512680053711
17.0705128205128 0.48555326461792
17.8846153846154 0.458046734333038
18.7371794871795 0.397574007511139
19.6282051282051 0.441966742277145
20.5641025641026 0.459091275930405
21.5416666666667 0.408522427082062
22.5673076923077 0.371674627065659
23.6442307692308 0.35003998875618
24.7692307692308 0.37718865275383
25.9487179487179 0.413035213947296
27.1826923076923 0.353285223245621
28.4775641025641 0.368711769580841
29.8333333333333 0.330677002668381
31.2564102564103 0.326293230056763
32.7435897435897 0.304729074239731
34.3044871794872 0.304218143224716
35.9358974358974 0.302728235721588
37.6474358974359 0.315792948007584
39.4391025641026 0.24323458969593
41.3173076923077 0.27931135892868
43.2852564102564 0.269490569829941
45.3461538461538 0.230944752693176
47.5064102564103 0.261680334806442
49.7692307692308 0.231959626078606
52.1378205128205 0.216957196593285
54.6217948717949 0.191049367189407
57.2211538461538 0.19061179459095
59.9455128205128 0.181484803557396
62.8012820512821 0.195404812693596
65.7916666666667 0.173414841294289
68.9230769230769 0.17659604549408
72.2051282051282 0.173661828041077
75.6442307692308 0.156608209013939
79.2467948717949 0.143463164567947
83.0192307692308 0.161770209670067
86.974358974359 0.130946978926659
91.1153846153846 0.129203006625175
95.4519230769231 0.136560201644897
100 0.14145527780056
};
\addplot [, black, opacity=0.6, mark=*, mark size=0.5, mark options={solid}, only marks]
table {%
1 0.981747090816498
1.04487179487179 0.98182338476181
1.09615384615385 0.988020360469818
1.1474358974359 0.990345180034637
1.20192307692308 0.991780400276184
1.25961538461538 0.99025171995163
1.32051282051282 0.988504528999329
1.38461538461538 0.98907083272934
1.44871794871795 0.990697205066681
1.51923076923077 0.988103091716766
1.58974358974359 0.985977172851562
1.66666666666667 0.98432582616806
1.74679487179487 0.983087539672852
1.83012820512821 0.981574535369873
1.91666666666667 0.980182111263275
2.00641025641026 0.979159355163574
2.1025641025641 0.978555619716644
2.20512820512821 0.974648296833038
2.30769230769231 0.973026692867279
2.41987179487179 0.970817983150482
2.53525641025641 0.967321872711182
2.65384615384615 0.957172393798828
2.78205128205128 0.96312552690506
2.91346153846154 0.932680547237396
3.05128205128205 0.902878761291504
3.19871794871795 0.872137367725372
3.34935897435897 0.914182841777802
3.50961538461538 0.912082493305206
3.67628205128205 0.924856305122375
3.8525641025641 0.878294587135315
4.03525641025641 0.910778641700745
4.2275641025641 0.842510998249054
4.42948717948718 0.867378413677216
4.64102564102564 0.819855988025665
4.86217948717949 0.837713837623596
5.09294871794872 0.864742696285248
5.33653846153846 0.874320983886719
5.58974358974359 0.825038135051727
5.85576923076923 0.856293499469757
6.13461538461539 0.864015102386475
6.42628205128205 0.861823260784149
6.73397435897436 0.88582843542099
7.05448717948718 0.859955608844757
7.38782051282051 0.815200984477997
7.74038461538461 0.806585609912872
8.10897435897436 0.842734754085541
8.49679487179487 0.815576016902924
8.90064102564103 0.82606166601181
9.32371794871795 0.858394265174866
9.76923076923077 0.754096686840057
10.2339743589744 0.807255387306213
10.7211538461538 0.784341633319855
11.2307692307692 0.759694039821625
11.7660256410256 0.759757816791534
12.3269230769231 0.785604655742645
12.9134615384615 0.772199630737305
13.5288461538462 0.677051663398743
14.1730769230769 0.699923515319824
14.849358974359 0.711808145046234
15.5544871794872 0.689821422100067
16.2948717948718 0.719122231006622
17.0705128205128 0.755955517292023
17.8846153846154 0.685849487781525
18.7371794871795 0.674093425273895
19.6282051282051 0.663297057151794
20.5641025641026 0.651888072490692
21.5416666666667 0.679920017719269
22.5673076923077 0.673120439052582
23.6442307692308 0.54052209854126
24.7692307692308 0.550203800201416
25.9487179487179 0.58035671710968
27.1826923076923 0.525099635124207
28.4775641025641 0.566456913948059
29.8333333333333 0.538295209407806
31.2564102564103 0.560106933116913
32.7435897435897 0.494608640670776
34.3044871794872 0.478361934423447
35.9358974358974 0.481842517852783
37.6474358974359 0.464411735534668
39.4391025641026 0.383963644504547
41.3173076923077 0.414829462766647
43.2852564102564 0.420306831598282
45.3461538461538 0.395040482282639
47.5064102564103 0.368936538696289
49.7692307692308 0.369207710027695
52.1378205128205 0.315311104059219
54.6217948717949 0.306672900915146
57.2211538461538 0.310634672641754
59.9455128205128 0.262401551008224
62.8012820512821 0.281118243932724
65.7916666666667 0.27843850851059
68.9230769230769 0.289015680551529
72.2051282051282 0.271875828504562
75.6442307692308 0.217997744679451
79.2467948717949 0.236163005232811
83.0192307692308 0.199122712016106
86.974358974359 0.217396453022957
91.1153846153846 0.212340220808983
95.4519230769231 0.166822001338005
100 0.159118488430977
};
\addlegendentry{mb 128, exact}
\addplot [, black, opacity=0.6, mark=*, mark size=0.5, mark options={solid}, only marks, forget plot]
table {%
1 0.982020795345306
1.04487179487179 0.953780591487885
1.09615384615385 0.987308979034424
1.1474358974359 0.990592777729034
1.20192307692308 0.992946445941925
1.25961538461538 0.991203784942627
1.32051282051282 0.989980518817902
1.38461538461538 0.990488648414612
1.44871794871795 0.99211198091507
1.51923076923077 0.990089416503906
1.58974358974359 0.988253116607666
1.66666666666667 0.987675189971924
1.74679487179487 0.986547589302063
1.83012820512821 0.98565673828125
1.91666666666667 0.985000550746918
2.00641025641026 0.983888268470764
2.1025641025641 0.980955302715302
2.20512820512821 0.981683909893036
2.30769230769231 0.975873172283173
2.41987179487179 0.978211998939514
2.53525641025641 0.974315464496613
2.65384615384615 0.967435479164124
2.78205128205128 0.967437088489532
2.91346153846154 0.953267276287079
3.05128205128205 0.949415385723114
3.19871794871795 0.875376641750336
3.34935897435897 0.924704968929291
3.50961538461538 0.895968735218048
3.67628205128205 0.92753928899765
3.8525641025641 0.923917770385742
4.03525641025641 0.917343556880951
4.2275641025641 0.846921145915985
4.42948717948718 0.902848660945892
4.64102564102564 0.898207008838654
4.86217948717949 0.904129326343536
5.09294871794872 0.891416072845459
5.33653846153846 0.925188660621643
5.58974358974359 0.896750271320343
5.85576923076923 0.880420506000519
6.13461538461539 0.877892136573792
6.42628205128205 0.904029667377472
6.73397435897436 0.839462876319885
7.05448717948718 0.85766077041626
7.38782051282051 0.849320232868195
7.74038461538461 0.863644242286682
8.10897435897436 0.847573757171631
8.49679487179487 0.859029591083527
8.90064102564103 0.847772538661957
9.32371794871795 0.852554261684418
9.76923076923077 0.770186126232147
10.2339743589744 0.763655602931976
10.7211538461538 0.726811528205872
11.2307692307692 0.776473820209503
11.7660256410256 0.761313438415527
12.3269230769231 0.79255622625351
12.9134615384615 0.743525981903076
13.5288461538462 0.68277233839035
14.1730769230769 0.739355981349945
14.849358974359 0.724602162837982
15.5544871794872 0.686427116394043
16.2948717948718 0.693134009838104
17.0705128205128 0.739706635475159
17.8846153846154 0.696715474128723
18.7371794871795 0.68716698884964
19.6282051282051 0.61690491437912
20.5641025641026 0.614159882068634
21.5416666666667 0.721479594707489
22.5673076923077 0.648724734783173
23.6442307692308 0.585017144680023
24.7692307692308 0.590175628662109
25.9487179487179 0.524260938167572
27.1826923076923 0.562411189079285
28.4775641025641 0.575817584991455
29.8333333333333 0.535391569137573
31.2564102564103 0.604447066783905
32.7435897435897 0.502214431762695
34.3044871794872 0.502260327339172
35.9358974358974 0.454190403223038
37.6474358974359 0.520226001739502
39.4391025641026 0.448612034320831
41.3173076923077 0.427878677845001
43.2852564102564 0.384945422410965
45.3461538461538 0.419080346822739
47.5064102564103 0.331002056598663
49.7692307692308 0.307637602090836
52.1378205128205 0.359945029020309
54.6217948717949 0.30944037437439
57.2211538461538 0.262731164693832
59.9455128205128 0.280141055583954
62.8012820512821 0.229214191436768
65.7916666666667 0.199491798877716
68.9230769230769 0.216717630624771
72.2051282051282 0.192475125193596
75.6442307692308 0.257141083478928
79.2467948717949 0.148212775588036
83.0192307692308 0.18046860396862
86.974358974359 0.187112584710121
91.1153846153846 0.136196181178093
95.4519230769231 0.145475029945374
100 0.141011148691177
};
\addplot [, black, opacity=0.6, mark=*, mark size=0.5, mark options={solid}, only marks, forget plot]
table {%
1 0.984170615673065
1.04487179487179 0.962684333324432
1.09615384615385 0.990463376045227
1.1474358974359 0.992333114147186
1.20192307692308 0.994071006774902
1.25961538461538 0.993533909320831
1.32051282051282 0.991840958595276
1.38461538461538 0.991823792457581
1.44871794871795 0.993282914161682
1.51923076923077 0.990627884864807
1.58974358974359 0.990484058856964
1.66666666666667 0.988605320453644
1.74679487179487 0.988952457904816
1.83012820512821 0.986732482910156
1.91666666666667 0.98617947101593
2.00641025641026 0.986214578151703
2.1025641025641 0.982601761817932
2.20512820512821 0.983159720897675
2.30769230769231 0.981453120708466
2.41987179487179 0.982396304607391
2.53525641025641 0.979023456573486
2.65384615384615 0.975050568580627
2.78205128205128 0.960993766784668
2.91346153846154 0.952524185180664
3.05128205128205 0.926346242427826
3.19871794871795 0.874208569526672
3.34935897435897 0.890124142169952
3.50961538461538 0.870679318904877
3.67628205128205 0.927426934242249
3.8525641025641 0.894946277141571
4.03525641025641 0.848156630992889
4.2275641025641 0.932078003883362
4.42948717948718 0.869904816150665
4.64102564102564 0.875738322734833
4.86217948717949 0.840713024139404
5.09294871794872 0.840531945228577
5.33653846153846 0.923191964626312
5.58974358974359 0.840282380580902
5.85576923076923 0.817100942134857
6.13461538461539 0.834160268306732
6.42628205128205 0.857244312763214
6.73397435897436 0.873536288738251
7.05448717948718 0.870953977108002
7.38782051282051 0.872702240943909
7.74038461538461 0.864868760108948
8.10897435897436 0.791886925697327
8.49679487179487 0.808820724487305
8.90064102564103 0.809190213680267
9.32371794871795 0.802600562572479
9.76923076923077 0.809189915657043
10.2339743589744 0.766855239868164
10.7211538461538 0.817921936511993
11.2307692307692 0.780461728572845
11.7660256410256 0.73890221118927
12.3269230769231 0.792637348175049
12.9134615384615 0.758075952529907
13.5288461538462 0.795771420001984
14.1730769230769 0.707125425338745
14.849358974359 0.722668051719666
15.5544871794872 0.65700376033783
16.2948717948718 0.729147553443909
17.0705128205128 0.699276089668274
17.8846153846154 0.688258171081543
18.7371794871795 0.630997002124786
19.6282051282051 0.641006469726562
20.5641025641026 0.641031563282013
21.5416666666667 0.611485123634338
22.5673076923077 0.636362671852112
23.6442307692308 0.579290926456451
24.7692307692308 0.580309808254242
25.9487179487179 0.571253001689911
27.1826923076923 0.566135585308075
28.4775641025641 0.604471206665039
29.8333333333333 0.568235576152802
31.2564102564103 0.54334819316864
32.7435897435897 0.529070019721985
34.3044871794872 0.544546067714691
35.9358974358974 0.482943207025528
37.6474358974359 0.538393676280975
39.4391025641026 0.463052660226822
41.3173076923077 0.472070604562759
43.2852564102564 0.442756474018097
45.3461538461538 0.433039039373398
47.5064102564103 0.403144657611847
49.7692307692308 0.368089973926544
52.1378205128205 0.375291228294373
54.6217948717949 0.373930543661118
57.2211538461538 0.284169286489487
59.9455128205128 0.306993454694748
62.8012820512821 0.282681196928024
65.7916666666667 0.288691371679306
68.9230769230769 0.285261958837509
72.2051282051282 0.256960988044739
75.6442307692308 0.230094075202942
79.2467948717949 0.2147067040205
83.0192307692308 0.226162105798721
86.974358974359 0.186426684260368
91.1153846153846 0.155625566840172
95.4519230769231 0.171306014060974
100 0.173411771655083
};
\addplot [, black, opacity=0.6, mark=*, mark size=0.5, mark options={solid}, only marks, forget plot]
table {%
1 0.984564304351807
1.04487179487179 0.975940704345703
1.09615384615385 0.990538895130157
1.1474358974359 0.991618752479553
1.20192307692308 0.993819415569305
1.25961538461538 0.993273675441742
1.32051282051282 0.991371631622314
1.38461538461538 0.991176605224609
1.44871794871795 0.993327617645264
1.51923076923077 0.990107357501984
1.58974358974359 0.98944091796875
1.66666666666667 0.987054467201233
1.74679487179487 0.987636685371399
1.83012820512821 0.98361474275589
1.91666666666667 0.983842968940735
2.00641025641026 0.983477294445038
2.1025641025641 0.97759610414505
2.20512820512821 0.981981098651886
2.30769230769231 0.977427303791046
2.41987179487179 0.94308990240097
2.53525641025641 0.885430157184601
2.65384615384615 0.940607070922852
2.78205128205128 0.879908382892609
2.91346153846154 0.963673532009125
3.05128205128205 0.902095317840576
3.19871794871795 0.928976535797119
3.34935897435897 0.907859265804291
3.50961538461538 0.874574661254883
3.67628205128205 0.870367050170898
3.8525641025641 0.943110883235931
4.03525641025641 0.854616105556488
4.2275641025641 0.914628028869629
4.42948717948718 0.825600624084473
4.64102564102564 0.887604176998138
4.86217948717949 0.835559844970703
5.09294871794872 0.886755168437958
5.33653846153846 0.882330358028412
5.58974358974359 0.816561877727509
5.85576923076923 0.81672191619873
6.13461538461539 0.833069980144501
6.42628205128205 0.844195187091827
6.73397435897436 0.819805800914764
7.05448717948718 0.808649480342865
7.38782051282051 0.807758152484894
7.74038461538461 0.759325206279755
8.10897435897436 0.799039661884308
8.49679487179487 0.880455434322357
8.90064102564103 0.810913264751434
9.32371794871795 0.796833634376526
9.76923076923077 0.757974803447723
10.2339743589744 0.751160323619843
10.7211538461538 0.791345059871674
11.2307692307692 0.769242703914642
11.7660256410256 0.763507902622223
12.3269230769231 0.753553450107574
12.9134615384615 0.752502381801605
13.5288461538462 0.724954068660736
14.1730769230769 0.736444771289825
14.849358974359 0.718579232692719
15.5544871794872 0.71355402469635
16.2948717948718 0.689074337482452
17.0705128205128 0.689262390136719
17.8846153846154 0.716107070446014
18.7371794871795 0.738085210323334
19.6282051282051 0.694091975688934
20.5641025641026 0.601486027240753
21.5416666666667 0.648961961269379
22.5673076923077 0.697732865810394
23.6442307692308 0.650158584117889
24.7692307692308 0.592214584350586
25.9487179487179 0.591780006885529
27.1826923076923 0.588055074214935
28.4775641025641 0.569010436534882
29.8333333333333 0.524317264556885
31.2564102564103 0.545189797878265
32.7435897435897 0.537986934185028
34.3044871794872 0.493544101715088
35.9358974358974 0.51635879278183
37.6474358974359 0.495911508798599
39.4391025641026 0.452186733484268
41.3173076923077 0.451370060443878
43.2852564102564 0.443919867277145
45.3461538461538 0.395095527172089
47.5064102564103 0.411824524402618
49.7692307692308 0.371033251285553
52.1378205128205 0.385174244642258
54.6217948717949 0.360418766736984
57.2211538461538 0.331921488046646
59.9455128205128 0.321492403745651
62.8012820512821 0.306410759687424
65.7916666666667 0.375531643629074
68.9230769230769 0.285832017660141
72.2051282051282 0.273515850305557
75.6442307692308 0.241822242736816
79.2467948717949 0.254934042692184
83.0192307692308 0.242678984999657
86.974358974359 0.241615250706673
91.1153846153846 0.212708428502083
95.4519230769231 0.185712978243828
100 0.191799759864807
};
\addplot [, black, opacity=0.6, mark=*, mark size=0.5, mark options={solid}, only marks, forget plot]
table {%
1 0.982150375843048
1.04487179487179 0.948779225349426
1.09615384615385 0.988999545574188
1.1474358974359 0.99137419462204
1.20192307692308 0.993538856506348
1.25961538461538 0.99172830581665
1.32051282051282 0.990062236785889
1.38461538461538 0.989916980266571
1.44871794871795 0.99199241399765
1.51923076923077 0.989880263805389
1.58974358974359 0.988528251647949
1.66666666666667 0.987747013568878
1.74679487179487 0.987243294715881
1.83012820512821 0.986089169979095
1.91666666666667 0.984426498413086
2.00641025641026 0.984810292720795
2.1025641025641 0.981416404247284
2.20512820512821 0.980676293373108
2.30769230769231 0.975415885448456
2.41987179487179 0.971500217914581
2.53525641025641 0.977910220623016
2.65384615384615 0.962833404541016
2.78205128205128 0.961658298969269
2.91346153846154 0.872323989868164
3.05128205128205 0.957341969013214
3.19871794871795 0.867337822914124
3.34935897435897 0.877156853675842
3.50961538461538 0.894341468811035
3.67628205128205 0.941383183002472
3.8525641025641 0.887681901454926
4.03525641025641 0.903548419475555
4.2275641025641 0.870131611824036
4.42948717948718 0.906260013580322
4.64102564102564 0.911598980426788
4.86217948717949 0.929806411266327
5.09294871794872 0.909204006195068
5.33653846153846 0.913798332214355
5.58974358974359 0.885871529579163
5.85576923076923 0.89288866519928
6.13461538461539 0.887316048145294
6.42628205128205 0.899638175964355
6.73397435897436 0.886183440685272
7.05448717948718 0.895258843898773
7.38782051282051 0.876403033733368
7.74038461538461 0.866112351417542
8.10897435897436 0.843766212463379
8.49679487179487 0.86292439699173
8.90064102564103 0.811742007732391
9.32371794871795 0.844043731689453
9.76923076923077 0.791018426418304
10.2339743589744 0.809145092964172
10.7211538461538 0.744627714157104
11.2307692307692 0.77544641494751
11.7660256410256 0.759141147136688
12.3269230769231 0.774510502815247
12.9134615384615 0.784630239009857
13.5288461538462 0.709848046302795
14.1730769230769 0.729115843772888
14.849358974359 0.771539866924286
15.5544871794872 0.777998089790344
16.2948717948718 0.736065566539764
17.0705128205128 0.726726353168488
17.8846153846154 0.71113646030426
18.7371794871795 0.703156292438507
19.6282051282051 0.703533232212067
20.5641025641026 0.665829658508301
21.5416666666667 0.686701118946075
22.5673076923077 0.672307372093201
23.6442307692308 0.647614181041718
24.7692307692308 0.611117362976074
25.9487179487179 0.640250205993652
27.1826923076923 0.631130695343018
28.4775641025641 0.613438129425049
29.8333333333333 0.572669923305511
31.2564102564103 0.557980358600616
32.7435897435897 0.562351405620575
34.3044871794872 0.532477676868439
35.9358974358974 0.478374540805817
37.6474358974359 0.473836660385132
39.4391025641026 0.408859342336655
41.3173076923077 0.420732021331787
43.2852564102564 0.414896339178085
45.3461538461538 0.41843256354332
47.5064102564103 0.381112337112427
49.7692307692308 0.402549088001251
52.1378205128205 0.326619625091553
54.6217948717949 0.366044819355011
57.2211538461538 0.355786979198456
59.9455128205128 0.324082583189011
62.8012820512821 0.283750951290131
65.7916666666667 0.292596429586411
68.9230769230769 0.273279219865799
72.2051282051282 0.233821824193001
75.6442307692308 0.210522279143333
79.2467948717949 0.2241500467062
83.0192307692308 0.253349304199219
86.974358974359 0.174529269337654
91.1153846153846 0.180669337511063
95.4519230769231 0.163924336433411
100 0.149136736989021
};
\end{axis}

\end{tikzpicture}

      \tikzexternaldisable
    \end{minipage}\hfill
    \begin{minipage}{0.50\linewidth}
      \centering
      % defines the pgfplots style "eigspacedefault"
\pgfkeys{/pgfplots/eigspacedefault/.style={
    width=1.0\linewidth,
    height=0.6\linewidth,
    every axis plot/.append style={line width = 1.5pt},
    tick pos = left,
    ylabel near ticks,
    xlabel near ticks,
    xtick align = inside,
    ytick align = inside,
    legend cell align = left,
    legend columns = 4,
    legend pos = south east,
    legend style = {
      fill opacity = 1,
      text opacity = 1,
      font = \footnotesize,
      at={(1, 1.025)},
      anchor=south east,
      column sep=0.25cm,
    },
    legend image post style={scale=2.5},
    xticklabel style = {font = \footnotesize},
    xlabel style = {font = \footnotesize},
    axis line style = {black},
    yticklabel style = {font = \footnotesize},
    ylabel style = {font = \footnotesize},
    title style = {font = \footnotesize},
    grid = major,
    grid style = {dashed}
  }
}

\pgfkeys{/pgfplots/eigspacedefaultapp/.style={
    eigspacedefault,
    height=0.6\linewidth,
    legend columns = 2,
  }
}

\pgfkeys{/pgfplots/eigspacenolegend/.style={
    legend image post style = {scale=0},
    legend style = {
      fill opacity = 0,
      draw opacity = 0,
      text opacity = 0,
      font = \footnotesize,
      at={(1, 1.025)},
      anchor=south east,
      column sep=0.25cm,
    },
  }
}
%%% Local Variables:
%%% mode: latex
%%% TeX-master: "../../thesis"
%%% End:

      \pgfkeys{/pgfplots/zmystyle/.style={
          eigspacedefaultapp,
          eigspacenolegend,
        }}
      \tikzexternalenable
      \vspace{-6ex}
      % This file was created by tikzplotlib v0.9.7.
\begin{tikzpicture}

\definecolor{color0}{rgb}{0.274509803921569,0.6,0.564705882352941}
\definecolor{color1}{rgb}{0.870588235294118,0.623529411764706,0.0862745098039216}
\definecolor{color2}{rgb}{0.501960784313725,0.184313725490196,0.6}

\begin{axis}[
axis line style={white!10!black},
legend columns=2,
legend style={fill opacity=0.8, draw opacity=1, text opacity=1, at={(0.03,0.03)}, anchor=south west, draw=white!80!black},
log basis x={10},
tick pos=left,
xlabel={epoch (log scale)},
xmajorgrids,
xmin=0.794328234724281, xmax=125.892541179417,
xmode=log,
ylabel={overlap},
ymajorgrids,
ymin=-0.05, ymax=1.05,
zmystyle
]
\addplot [, white!10!black, dashed, forget plot]
table {%
0.794328234724281 1
125.892541179417 1
};
\addplot [, white!10!black, dashed, forget plot]
table {%
0.794328234724281 0
125.892541179417 0
};
\addplot [, black, opacity=0.6, mark=*, mark size=0.5, mark options={solid}, only marks]
table {%
1 0.981747090816498
1.04487179487179 0.98182338476181
1.09615384615385 0.988020360469818
1.1474358974359 0.990345180034637
1.20192307692308 0.991780400276184
1.25961538461538 0.99025171995163
1.32051282051282 0.988504528999329
1.38461538461538 0.98907083272934
1.44871794871795 0.990697205066681
1.51923076923077 0.988103091716766
1.58974358974359 0.985977172851562
1.66666666666667 0.98432582616806
1.74679487179487 0.983087539672852
1.83012820512821 0.981574535369873
1.91666666666667 0.980182111263275
2.00641025641026 0.979159355163574
2.1025641025641 0.978555619716644
2.20512820512821 0.974648296833038
2.30769230769231 0.973026692867279
2.41987179487179 0.970817983150482
2.53525641025641 0.967321872711182
2.65384615384615 0.957172393798828
2.78205128205128 0.96312552690506
2.91346153846154 0.932680547237396
3.05128205128205 0.902878761291504
3.19871794871795 0.872137367725372
3.34935897435897 0.914182841777802
3.50961538461538 0.912082493305206
3.67628205128205 0.924856305122375
3.8525641025641 0.878294587135315
4.03525641025641 0.910778641700745
4.2275641025641 0.842510998249054
4.42948717948718 0.867378413677216
4.64102564102564 0.819855988025665
4.86217948717949 0.837713837623596
5.09294871794872 0.864742696285248
5.33653846153846 0.874320983886719
5.58974358974359 0.825038135051727
5.85576923076923 0.856293499469757
6.13461538461539 0.864015102386475
6.42628205128205 0.861823260784149
6.73397435897436 0.88582843542099
7.05448717948718 0.859955608844757
7.38782051282051 0.815200984477997
7.74038461538461 0.806585609912872
8.10897435897436 0.842734754085541
8.49679487179487 0.815576016902924
8.90064102564103 0.82606166601181
9.32371794871795 0.858394265174866
9.76923076923077 0.754096686840057
10.2339743589744 0.807255387306213
10.7211538461538 0.784341633319855
11.2307692307692 0.759694039821625
11.7660256410256 0.759757816791534
12.3269230769231 0.785604655742645
12.9134615384615 0.772199630737305
13.5288461538462 0.677051663398743
14.1730769230769 0.699923515319824
14.849358974359 0.711808145046234
15.5544871794872 0.689821422100067
16.2948717948718 0.719122231006622
17.0705128205128 0.755955517292023
17.8846153846154 0.685849487781525
18.7371794871795 0.674093425273895
19.6282051282051 0.663297057151794
20.5641025641026 0.651888072490692
21.5416666666667 0.679920017719269
22.5673076923077 0.673120439052582
23.6442307692308 0.54052209854126
24.7692307692308 0.550203800201416
25.9487179487179 0.58035671710968
27.1826923076923 0.525099635124207
28.4775641025641 0.566456913948059
29.8333333333333 0.538295209407806
31.2564102564103 0.560106933116913
32.7435897435897 0.494608640670776
34.3044871794872 0.478361934423447
35.9358974358974 0.481842517852783
37.6474358974359 0.464411735534668
39.4391025641026 0.383963644504547
41.3173076923077 0.414829462766647
43.2852564102564 0.420306831598282
45.3461538461538 0.395040482282639
47.5064102564103 0.368936538696289
49.7692307692308 0.369207710027695
52.1378205128205 0.315311104059219
54.6217948717949 0.306672900915146
57.2211538461538 0.310634672641754
59.9455128205128 0.262401551008224
62.8012820512821 0.281118243932724
65.7916666666667 0.27843850851059
68.9230769230769 0.289015680551529
72.2051282051282 0.271875828504562
75.6442307692308 0.217997744679451
79.2467948717949 0.236163005232811
83.0192307692308 0.199122712016106
86.974358974359 0.217396453022957
91.1153846153846 0.212340220808983
95.4519230769231 0.166822001338005
100 0.159118488430977
};
\addlegendentry{mb 128, exact}
\addplot [, black, opacity=0.6, mark=*, mark size=0.5, mark options={solid}, only marks, forget plot]
table {%
1 0.982020795345306
1.04487179487179 0.953780591487885
1.09615384615385 0.987308979034424
1.1474358974359 0.990592777729034
1.20192307692308 0.992946445941925
1.25961538461538 0.991203784942627
1.32051282051282 0.989980518817902
1.38461538461538 0.990488648414612
1.44871794871795 0.99211198091507
1.51923076923077 0.990089416503906
1.58974358974359 0.988253116607666
1.66666666666667 0.987675189971924
1.74679487179487 0.986547589302063
1.83012820512821 0.98565673828125
1.91666666666667 0.985000550746918
2.00641025641026 0.983888268470764
2.1025641025641 0.980955302715302
2.20512820512821 0.981683909893036
2.30769230769231 0.975873172283173
2.41987179487179 0.978211998939514
2.53525641025641 0.974315464496613
2.65384615384615 0.967435479164124
2.78205128205128 0.967437088489532
2.91346153846154 0.953267276287079
3.05128205128205 0.949415385723114
3.19871794871795 0.875376641750336
3.34935897435897 0.924704968929291
3.50961538461538 0.895968735218048
3.67628205128205 0.92753928899765
3.8525641025641 0.923917770385742
4.03525641025641 0.917343556880951
4.2275641025641 0.846921145915985
4.42948717948718 0.902848660945892
4.64102564102564 0.898207008838654
4.86217948717949 0.904129326343536
5.09294871794872 0.891416072845459
5.33653846153846 0.925188660621643
5.58974358974359 0.896750271320343
5.85576923076923 0.880420506000519
6.13461538461539 0.877892136573792
6.42628205128205 0.904029667377472
6.73397435897436 0.839462876319885
7.05448717948718 0.85766077041626
7.38782051282051 0.849320232868195
7.74038461538461 0.863644242286682
8.10897435897436 0.847573757171631
8.49679487179487 0.859029591083527
8.90064102564103 0.847772538661957
9.32371794871795 0.852554261684418
9.76923076923077 0.770186126232147
10.2339743589744 0.763655602931976
10.7211538461538 0.726811528205872
11.2307692307692 0.776473820209503
11.7660256410256 0.761313438415527
12.3269230769231 0.79255622625351
12.9134615384615 0.743525981903076
13.5288461538462 0.68277233839035
14.1730769230769 0.739355981349945
14.849358974359 0.724602162837982
15.5544871794872 0.686427116394043
16.2948717948718 0.693134009838104
17.0705128205128 0.739706635475159
17.8846153846154 0.696715474128723
18.7371794871795 0.68716698884964
19.6282051282051 0.61690491437912
20.5641025641026 0.614159882068634
21.5416666666667 0.721479594707489
22.5673076923077 0.648724734783173
23.6442307692308 0.585017144680023
24.7692307692308 0.590175628662109
25.9487179487179 0.524260938167572
27.1826923076923 0.562411189079285
28.4775641025641 0.575817584991455
29.8333333333333 0.535391569137573
31.2564102564103 0.604447066783905
32.7435897435897 0.502214431762695
34.3044871794872 0.502260327339172
35.9358974358974 0.454190403223038
37.6474358974359 0.520226001739502
39.4391025641026 0.448612034320831
41.3173076923077 0.427878677845001
43.2852564102564 0.384945422410965
45.3461538461538 0.419080346822739
47.5064102564103 0.331002056598663
49.7692307692308 0.307637602090836
52.1378205128205 0.359945029020309
54.6217948717949 0.30944037437439
57.2211538461538 0.262731164693832
59.9455128205128 0.280141055583954
62.8012820512821 0.229214191436768
65.7916666666667 0.199491798877716
68.9230769230769 0.216717630624771
72.2051282051282 0.192475125193596
75.6442307692308 0.257141083478928
79.2467948717949 0.148212775588036
83.0192307692308 0.18046860396862
86.974358974359 0.187112584710121
91.1153846153846 0.136196181178093
95.4519230769231 0.145475029945374
100 0.141011148691177
};
\addplot [, black, opacity=0.6, mark=*, mark size=0.5, mark options={solid}, only marks, forget plot]
table {%
1 0.984170615673065
1.04487179487179 0.962684333324432
1.09615384615385 0.990463376045227
1.1474358974359 0.992333114147186
1.20192307692308 0.994071006774902
1.25961538461538 0.993533909320831
1.32051282051282 0.991840958595276
1.38461538461538 0.991823792457581
1.44871794871795 0.993282914161682
1.51923076923077 0.990627884864807
1.58974358974359 0.990484058856964
1.66666666666667 0.988605320453644
1.74679487179487 0.988952457904816
1.83012820512821 0.986732482910156
1.91666666666667 0.98617947101593
2.00641025641026 0.986214578151703
2.1025641025641 0.982601761817932
2.20512820512821 0.983159720897675
2.30769230769231 0.981453120708466
2.41987179487179 0.982396304607391
2.53525641025641 0.979023456573486
2.65384615384615 0.975050568580627
2.78205128205128 0.960993766784668
2.91346153846154 0.952524185180664
3.05128205128205 0.926346242427826
3.19871794871795 0.874208569526672
3.34935897435897 0.890124142169952
3.50961538461538 0.870679318904877
3.67628205128205 0.927426934242249
3.8525641025641 0.894946277141571
4.03525641025641 0.848156630992889
4.2275641025641 0.932078003883362
4.42948717948718 0.869904816150665
4.64102564102564 0.875738322734833
4.86217948717949 0.840713024139404
5.09294871794872 0.840531945228577
5.33653846153846 0.923191964626312
5.58974358974359 0.840282380580902
5.85576923076923 0.817100942134857
6.13461538461539 0.834160268306732
6.42628205128205 0.857244312763214
6.73397435897436 0.873536288738251
7.05448717948718 0.870953977108002
7.38782051282051 0.872702240943909
7.74038461538461 0.864868760108948
8.10897435897436 0.791886925697327
8.49679487179487 0.808820724487305
8.90064102564103 0.809190213680267
9.32371794871795 0.802600562572479
9.76923076923077 0.809189915657043
10.2339743589744 0.766855239868164
10.7211538461538 0.817921936511993
11.2307692307692 0.780461728572845
11.7660256410256 0.73890221118927
12.3269230769231 0.792637348175049
12.9134615384615 0.758075952529907
13.5288461538462 0.795771420001984
14.1730769230769 0.707125425338745
14.849358974359 0.722668051719666
15.5544871794872 0.65700376033783
16.2948717948718 0.729147553443909
17.0705128205128 0.699276089668274
17.8846153846154 0.688258171081543
18.7371794871795 0.630997002124786
19.6282051282051 0.641006469726562
20.5641025641026 0.641031563282013
21.5416666666667 0.611485123634338
22.5673076923077 0.636362671852112
23.6442307692308 0.579290926456451
24.7692307692308 0.580309808254242
25.9487179487179 0.571253001689911
27.1826923076923 0.566135585308075
28.4775641025641 0.604471206665039
29.8333333333333 0.568235576152802
31.2564102564103 0.54334819316864
32.7435897435897 0.529070019721985
34.3044871794872 0.544546067714691
35.9358974358974 0.482943207025528
37.6474358974359 0.538393676280975
39.4391025641026 0.463052660226822
41.3173076923077 0.472070604562759
43.2852564102564 0.442756474018097
45.3461538461538 0.433039039373398
47.5064102564103 0.403144657611847
49.7692307692308 0.368089973926544
52.1378205128205 0.375291228294373
54.6217948717949 0.373930543661118
57.2211538461538 0.284169286489487
59.9455128205128 0.306993454694748
62.8012820512821 0.282681196928024
65.7916666666667 0.288691371679306
68.9230769230769 0.285261958837509
72.2051282051282 0.256960988044739
75.6442307692308 0.230094075202942
79.2467948717949 0.2147067040205
83.0192307692308 0.226162105798721
86.974358974359 0.186426684260368
91.1153846153846 0.155625566840172
95.4519230769231 0.171306014060974
100 0.173411771655083
};
\addplot [, black, opacity=0.6, mark=*, mark size=0.5, mark options={solid}, only marks, forget plot]
table {%
1 0.984564304351807
1.04487179487179 0.975940704345703
1.09615384615385 0.990538895130157
1.1474358974359 0.991618752479553
1.20192307692308 0.993819415569305
1.25961538461538 0.993273675441742
1.32051282051282 0.991371631622314
1.38461538461538 0.991176605224609
1.44871794871795 0.993327617645264
1.51923076923077 0.990107357501984
1.58974358974359 0.98944091796875
1.66666666666667 0.987054467201233
1.74679487179487 0.987636685371399
1.83012820512821 0.98361474275589
1.91666666666667 0.983842968940735
2.00641025641026 0.983477294445038
2.1025641025641 0.97759610414505
2.20512820512821 0.981981098651886
2.30769230769231 0.977427303791046
2.41987179487179 0.94308990240097
2.53525641025641 0.885430157184601
2.65384615384615 0.940607070922852
2.78205128205128 0.879908382892609
2.91346153846154 0.963673532009125
3.05128205128205 0.902095317840576
3.19871794871795 0.928976535797119
3.34935897435897 0.907859265804291
3.50961538461538 0.874574661254883
3.67628205128205 0.870367050170898
3.8525641025641 0.943110883235931
4.03525641025641 0.854616105556488
4.2275641025641 0.914628028869629
4.42948717948718 0.825600624084473
4.64102564102564 0.887604176998138
4.86217948717949 0.835559844970703
5.09294871794872 0.886755168437958
5.33653846153846 0.882330358028412
5.58974358974359 0.816561877727509
5.85576923076923 0.81672191619873
6.13461538461539 0.833069980144501
6.42628205128205 0.844195187091827
6.73397435897436 0.819805800914764
7.05448717948718 0.808649480342865
7.38782051282051 0.807758152484894
7.74038461538461 0.759325206279755
8.10897435897436 0.799039661884308
8.49679487179487 0.880455434322357
8.90064102564103 0.810913264751434
9.32371794871795 0.796833634376526
9.76923076923077 0.757974803447723
10.2339743589744 0.751160323619843
10.7211538461538 0.791345059871674
11.2307692307692 0.769242703914642
11.7660256410256 0.763507902622223
12.3269230769231 0.753553450107574
12.9134615384615 0.752502381801605
13.5288461538462 0.724954068660736
14.1730769230769 0.736444771289825
14.849358974359 0.718579232692719
15.5544871794872 0.71355402469635
16.2948717948718 0.689074337482452
17.0705128205128 0.689262390136719
17.8846153846154 0.716107070446014
18.7371794871795 0.738085210323334
19.6282051282051 0.694091975688934
20.5641025641026 0.601486027240753
21.5416666666667 0.648961961269379
22.5673076923077 0.697732865810394
23.6442307692308 0.650158584117889
24.7692307692308 0.592214584350586
25.9487179487179 0.591780006885529
27.1826923076923 0.588055074214935
28.4775641025641 0.569010436534882
29.8333333333333 0.524317264556885
31.2564102564103 0.545189797878265
32.7435897435897 0.537986934185028
34.3044871794872 0.493544101715088
35.9358974358974 0.51635879278183
37.6474358974359 0.495911508798599
39.4391025641026 0.452186733484268
41.3173076923077 0.451370060443878
43.2852564102564 0.443919867277145
45.3461538461538 0.395095527172089
47.5064102564103 0.411824524402618
49.7692307692308 0.371033251285553
52.1378205128205 0.385174244642258
54.6217948717949 0.360418766736984
57.2211538461538 0.331921488046646
59.9455128205128 0.321492403745651
62.8012820512821 0.306410759687424
65.7916666666667 0.375531643629074
68.9230769230769 0.285832017660141
72.2051282051282 0.273515850305557
75.6442307692308 0.241822242736816
79.2467948717949 0.254934042692184
83.0192307692308 0.242678984999657
86.974358974359 0.241615250706673
91.1153846153846 0.212708428502083
95.4519230769231 0.185712978243828
100 0.191799759864807
};
\addplot [, black, opacity=0.6, mark=*, mark size=0.5, mark options={solid}, only marks, forget plot]
table {%
1 0.982150375843048
1.04487179487179 0.948779225349426
1.09615384615385 0.988999545574188
1.1474358974359 0.99137419462204
1.20192307692308 0.993538856506348
1.25961538461538 0.99172830581665
1.32051282051282 0.990062236785889
1.38461538461538 0.989916980266571
1.44871794871795 0.99199241399765
1.51923076923077 0.989880263805389
1.58974358974359 0.988528251647949
1.66666666666667 0.987747013568878
1.74679487179487 0.987243294715881
1.83012820512821 0.986089169979095
1.91666666666667 0.984426498413086
2.00641025641026 0.984810292720795
2.1025641025641 0.981416404247284
2.20512820512821 0.980676293373108
2.30769230769231 0.975415885448456
2.41987179487179 0.971500217914581
2.53525641025641 0.977910220623016
2.65384615384615 0.962833404541016
2.78205128205128 0.961658298969269
2.91346153846154 0.872323989868164
3.05128205128205 0.957341969013214
3.19871794871795 0.867337822914124
3.34935897435897 0.877156853675842
3.50961538461538 0.894341468811035
3.67628205128205 0.941383183002472
3.8525641025641 0.887681901454926
4.03525641025641 0.903548419475555
4.2275641025641 0.870131611824036
4.42948717948718 0.906260013580322
4.64102564102564 0.911598980426788
4.86217948717949 0.929806411266327
5.09294871794872 0.909204006195068
5.33653846153846 0.913798332214355
5.58974358974359 0.885871529579163
5.85576923076923 0.89288866519928
6.13461538461539 0.887316048145294
6.42628205128205 0.899638175964355
6.73397435897436 0.886183440685272
7.05448717948718 0.895258843898773
7.38782051282051 0.876403033733368
7.74038461538461 0.866112351417542
8.10897435897436 0.843766212463379
8.49679487179487 0.86292439699173
8.90064102564103 0.811742007732391
9.32371794871795 0.844043731689453
9.76923076923077 0.791018426418304
10.2339743589744 0.809145092964172
10.7211538461538 0.744627714157104
11.2307692307692 0.77544641494751
11.7660256410256 0.759141147136688
12.3269230769231 0.774510502815247
12.9134615384615 0.784630239009857
13.5288461538462 0.709848046302795
14.1730769230769 0.729115843772888
14.849358974359 0.771539866924286
15.5544871794872 0.777998089790344
16.2948717948718 0.736065566539764
17.0705128205128 0.726726353168488
17.8846153846154 0.71113646030426
18.7371794871795 0.703156292438507
19.6282051282051 0.703533232212067
20.5641025641026 0.665829658508301
21.5416666666667 0.686701118946075
22.5673076923077 0.672307372093201
23.6442307692308 0.647614181041718
24.7692307692308 0.611117362976074
25.9487179487179 0.640250205993652
27.1826923076923 0.631130695343018
28.4775641025641 0.613438129425049
29.8333333333333 0.572669923305511
31.2564102564103 0.557980358600616
32.7435897435897 0.562351405620575
34.3044871794872 0.532477676868439
35.9358974358974 0.478374540805817
37.6474358974359 0.473836660385132
39.4391025641026 0.408859342336655
41.3173076923077 0.420732021331787
43.2852564102564 0.414896339178085
45.3461538461538 0.41843256354332
47.5064102564103 0.381112337112427
49.7692307692308 0.402549088001251
52.1378205128205 0.326619625091553
54.6217948717949 0.366044819355011
57.2211538461538 0.355786979198456
59.9455128205128 0.324082583189011
62.8012820512821 0.283750951290131
65.7916666666667 0.292596429586411
68.9230769230769 0.273279219865799
72.2051282051282 0.233821824193001
75.6442307692308 0.210522279143333
79.2467948717949 0.2241500467062
83.0192307692308 0.253349304199219
86.974358974359 0.174529269337654
91.1153846153846 0.180669337511063
95.4519230769231 0.163924336433411
100 0.149136736989021
};
\addplot [, color0, opacity=0.6, mark=diamond*, mark size=0.5, mark options={solid}, only marks]
table {%
1 0.888289570808411
1.04487179487179 0.850245416164398
1.09615384615385 0.929787933826447
1.1474358974359 0.935880959033966
1.20192307692308 0.943499863147736
1.25961538461538 0.929302036762238
1.32051282051282 0.91467422246933
1.38461538461538 0.918963551521301
1.44871794871795 0.93555736541748
1.51923076923077 0.922814548015594
1.58974358974359 0.914416491985321
1.66666666666667 0.912550449371338
1.74679487179487 0.902347981929779
1.83012820512821 0.901371419429779
1.91666666666667 0.895730793476105
2.00641025641026 0.887700259685516
2.1025641025641 0.792950987815857
2.20512820512821 0.85385400056839
2.30769230769231 0.823620975017548
2.41987179487179 0.763083279132843
2.53525641025641 0.811955094337463
2.65384615384615 0.785270512104034
2.78205128205128 0.759598016738892
2.91346153846154 0.765279948711395
3.05128205128205 0.741091549396515
3.19871794871795 0.73689204454422
3.34935897435897 0.735926568508148
3.50961538461538 0.729260742664337
3.67628205128205 0.66669362783432
3.8525641025641 0.702483832836151
4.03525641025641 0.658973634243011
4.2275641025641 0.651246964931488
4.42948717948718 0.647138714790344
4.64102564102564 0.641364753246307
4.86217948717949 0.639494895935059
5.09294871794872 0.648367762565613
5.33653846153846 0.65986031293869
5.58974358974359 0.581650614738464
5.85576923076923 0.556316316127777
6.13461538461539 0.604955017566681
6.42628205128205 0.562110304832458
6.73397435897436 0.52776974439621
7.05448717948718 0.533655226230621
7.38782051282051 0.529252648353577
7.74038461538461 0.506701648235321
8.10897435897436 0.551192760467529
8.49679487179487 0.511806011199951
8.90064102564103 0.494422018527985
9.32371794871795 0.448191255331039
9.76923076923077 0.395153999328613
10.2339743589744 0.447132438421249
10.7211538461538 0.464736312627792
11.2307692307692 0.418795645236969
11.7660256410256 0.376513570547104
12.3269230769231 0.406778633594513
12.9134615384615 0.364027082920074
13.5288461538462 0.381784528493881
14.1730769230769 0.329464435577393
14.849358974359 0.35334438085556
15.5544871794872 0.4212606549263
16.2948717948718 0.355994403362274
17.0705128205128 0.352870553731918
17.8846153846154 0.343120992183685
18.7371794871795 0.272719025611877
19.6282051282051 0.329196721315384
20.5641025641026 0.286043673753738
21.5416666666667 0.334159225225449
22.5673076923077 0.293594181537628
23.6442307692308 0.286035478115082
24.7692307692308 0.308116465806961
25.9487179487179 0.244923546910286
27.1826923076923 0.239613577723503
28.4775641025641 0.274717658758163
29.8333333333333 0.269978433847427
31.2564102564103 0.247582659125328
32.7435897435897 0.2376539260149
34.3044871794872 0.218981549143791
35.9358974358974 0.221796855330467
37.6474358974359 0.223196133971214
39.4391025641026 0.208313748240471
41.3173076923077 0.215252637863159
43.2852564102564 0.19606702029705
45.3461538461538 0.181623503565788
47.5064102564103 0.17289674282074
49.7692307692308 0.160922631621361
52.1378205128205 0.162959381937981
54.6217948717949 0.163500472903252
57.2211538461538 0.164205715060234
59.9455128205128 0.175009906291962
62.8012820512821 0.140065833926201
65.7916666666667 0.138864755630493
68.9230769230769 0.140384361147881
72.2051282051282 0.141933500766754
75.6442307692308 0.139193296432495
79.2467948717949 0.146768793463707
83.0192307692308 0.128423228859901
86.974358974359 0.157832458615303
91.1153846153846 0.121040962636471
95.4519230769231 0.108292534947395
100 0.112434543669224
};
\addlegendentry{sub 16, exact}
\addplot [, color0, opacity=0.6, mark=diamond*, mark size=0.5, mark options={solid}, only marks, forget plot]
table {%
1 0.893791139125824
1.04487179487179 0.854173302650452
1.09615384615385 0.928434371948242
1.1474358974359 0.936124265193939
1.20192307692308 0.951777875423431
1.25961538461538 0.949469029903412
1.32051282051282 0.934574723243713
1.38461538461538 0.93833714723587
1.44871794871795 0.948976516723633
1.51923076923077 0.937324702739716
1.58974358974359 0.919515609741211
1.66666666666667 0.908123672008514
1.74679487179487 0.890411198139191
1.83012820512821 0.876589953899384
1.91666666666667 0.890449941158295
2.00641025641026 0.867175877094269
2.1025641025641 0.783797204494476
2.20512820512821 0.867137432098389
2.30769230769231 0.816940784454346
2.41987179487179 0.76775187253952
2.53525641025641 0.754798829555511
2.65384615384615 0.783145546913147
2.78205128205128 0.714383900165558
2.91346153846154 0.729553878307343
3.05128205128205 0.700503766536713
3.19871794871795 0.735346615314484
3.34935897435897 0.740528881549835
3.50961538461538 0.723609924316406
3.67628205128205 0.666880667209625
3.8525641025641 0.656340777873993
4.03525641025641 0.642926037311554
4.2275641025641 0.650541007518768
4.42948717948718 0.601711690425873
4.64102564102564 0.618374109268188
4.86217948717949 0.616057097911835
5.09294871794872 0.58877432346344
5.33653846153846 0.576996743679047
5.58974358974359 0.57886803150177
5.85576923076923 0.533031165599823
6.13461538461539 0.549373686313629
6.42628205128205 0.572380721569061
6.73397435897436 0.520179390907288
7.05448717948718 0.518650531768799
7.38782051282051 0.518807888031006
7.74038461538461 0.473583698272705
8.10897435897436 0.481939882040024
8.49679487179487 0.468410223722458
8.90064102564103 0.464531630277634
9.32371794871795 0.461068838834763
9.76923076923077 0.438083410263062
10.2339743589744 0.419620484113693
10.7211538461538 0.430715948343277
11.2307692307692 0.413104265928268
11.7660256410256 0.383857488632202
12.3269230769231 0.369482487440109
12.9134615384615 0.372629731893539
13.5288461538462 0.350613206624985
14.1730769230769 0.354642152786255
14.849358974359 0.375771790742874
15.5544871794872 0.367782175540924
16.2948717948718 0.326392233371735
17.0705128205128 0.345154494047165
17.8846153846154 0.356812238693237
18.7371794871795 0.324780136346817
19.6282051282051 0.335758477449417
20.5641025641026 0.316658705472946
21.5416666666667 0.336475849151611
22.5673076923077 0.314805507659912
23.6442307692308 0.283265024423599
24.7692307692308 0.285132855176926
25.9487179487179 0.288173645734787
27.1826923076923 0.288888901472092
28.4775641025641 0.290199965238571
29.8333333333333 0.269994080066681
31.2564102564103 0.271203547716141
32.7435897435897 0.248423621058464
34.3044871794872 0.245234161615372
35.9358974358974 0.241854622960091
37.6474358974359 0.199167862534523
39.4391025641026 0.2262252420187
41.3173076923077 0.213113889098167
43.2852564102564 0.20268402993679
45.3461538461538 0.192482188344002
47.5064102564103 0.186557650566101
49.7692307692308 0.191979929804802
52.1378205128205 0.18522472679615
54.6217948717949 0.198280215263367
57.2211538461538 0.178179949522018
59.9455128205128 0.178136929869652
62.8012820512821 0.178591907024384
65.7916666666667 0.157031685113907
68.9230769230769 0.17550702393055
72.2051282051282 0.14932969212532
75.6442307692308 0.154043957591057
79.2467948717949 0.147724434733391
83.0192307692308 0.154264211654663
86.974358974359 0.131414070725441
91.1153846153846 0.137487486004829
95.4519230769231 0.148206621408463
100 0.140420719981194
};
\addplot [, color0, opacity=0.6, mark=diamond*, mark size=0.5, mark options={solid}, only marks, forget plot]
table {%
1 0.891717374324799
1.04487179487179 0.869342982769012
1.09615384615385 0.919399440288544
1.1474358974359 0.933474004268646
1.20192307692308 0.946918129920959
1.25961538461538 0.93341988325119
1.32051282051282 0.916859447956085
1.38461538461538 0.923144817352295
1.44871794871795 0.936473965644836
1.51923076923077 0.929627239704132
1.58974358974359 0.902821183204651
1.66666666666667 0.89877837896347
1.74679487179487 0.891434490680695
1.83012820512821 0.876318395137787
1.91666666666667 0.880023181438446
2.00641025641026 0.863736927509308
2.1025641025641 0.855601489543915
2.20512820512821 0.851014077663422
2.30769230769231 0.808860599994659
2.41987179487179 0.816703021526337
2.53525641025641 0.831770539283752
2.65384615384615 0.75491589307785
2.78205128205128 0.755659341812134
2.91346153846154 0.737481713294983
3.05128205128205 0.70359867811203
3.19871794871795 0.668097019195557
3.34935897435897 0.68815416097641
3.50961538461538 0.632323503494263
3.67628205128205 0.611617684364319
3.8525641025641 0.65784627199173
4.03525641025641 0.63075464963913
4.2275641025641 0.621694028377533
4.42948717948718 0.607796907424927
4.64102564102564 0.565841376781464
4.86217948717949 0.586063385009766
5.09294871794872 0.576301455497742
5.33653846153846 0.590676605701447
5.58974358974359 0.52356892824173
5.85576923076923 0.521511495113373
6.13461538461539 0.554125010967255
6.42628205128205 0.513071358203888
6.73397435897436 0.517736494541168
7.05448717948718 0.496909350156784
7.38782051282051 0.506617426872253
7.74038461538461 0.479956835508347
8.10897435897436 0.469402849674225
8.49679487179487 0.461047023534775
8.90064102564103 0.462979704141617
9.32371794871795 0.486325353384018
9.76923076923077 0.418221771717072
10.2339743589744 0.390541791915894
10.7211538461538 0.439920097589493
11.2307692307692 0.37975537776947
11.7660256410256 0.459371298551559
12.3269230769231 0.398192316293716
12.9134615384615 0.354040682315826
13.5288461538462 0.351535648107529
14.1730769230769 0.344239860773087
14.849358974359 0.382947832345963
15.5544871794872 0.344797998666763
16.2948717948718 0.305987119674683
17.0705128205128 0.370094865560532
17.8846153846154 0.336247593164444
18.7371794871795 0.327902793884277
19.6282051282051 0.346059888601303
20.5641025641026 0.298206001520157
21.5416666666667 0.306540340185165
22.5673076923077 0.291296809911728
23.6442307692308 0.279941380023956
24.7692307692308 0.271569490432739
25.9487179487179 0.269172310829163
27.1826923076923 0.247764781117439
28.4775641025641 0.239833638072014
29.8333333333333 0.243390426039696
31.2564102564103 0.238371476531029
32.7435897435897 0.233770713210106
34.3044871794872 0.221039280295372
35.9358974358974 0.256289690732956
37.6474358974359 0.220509633421898
39.4391025641026 0.216097861528397
41.3173076923077 0.194299146533012
43.2852564102564 0.227774649858475
45.3461538461538 0.197210982441902
47.5064102564103 0.196835562586784
49.7692307692308 0.178263410925865
52.1378205128205 0.15285649895668
54.6217948717949 0.186899214982986
57.2211538461538 0.157098278403282
59.9455128205128 0.169110685586929
62.8012820512821 0.148695811629295
65.7916666666667 0.15706293284893
68.9230769230769 0.119244933128357
72.2051282051282 0.145460233092308
75.6442307692308 0.131833106279373
79.2467948717949 0.146838709712029
83.0192307692308 0.137946113944054
86.974358974359 0.124879911541939
91.1153846153846 0.119742766022682
95.4519230769231 0.119828082621098
100 0.123383574187756
};
\addplot [, color0, opacity=0.6, mark=diamond*, mark size=0.5, mark options={solid}, only marks, forget plot]
table {%
1 0.893024384975433
1.04487179487179 0.910530865192413
1.09615384615385 0.927744090557098
1.1474358974359 0.944096386432648
1.20192307692308 0.955252647399902
1.25961538461538 0.958257853984833
1.32051282051282 0.938665211200714
1.38461538461538 0.943126678466797
1.44871794871795 0.951784312725067
1.51923076923077 0.925331115722656
1.58974358974359 0.930963158607483
1.66666666666667 0.833288013935089
1.74679487179487 0.909896373748779
1.83012820512821 0.898792266845703
1.91666666666667 0.821410953998566
2.00641025641026 0.892078995704651
2.1025641025641 0.844248235225677
2.20512820512821 0.850242078304291
2.30769230769231 0.863114833831787
2.41987179487179 0.824883937835693
2.53525641025641 0.835236191749573
2.65384615384615 0.76833975315094
2.78205128205128 0.82883894443512
2.91346153846154 0.761045396327972
3.05128205128205 0.724739491939545
3.19871794871795 0.70199453830719
3.34935897435897 0.702546060085297
3.50961538461538 0.671069800853729
3.67628205128205 0.702909350395203
3.8525641025641 0.666581273078918
4.03525641025641 0.676584422588348
4.2275641025641 0.634079396724701
4.42948717948718 0.654314935207367
4.64102564102564 0.595455348491669
4.86217948717949 0.63051849603653
5.09294871794872 0.571516454219818
5.33653846153846 0.576733529567719
5.58974358974359 0.55988883972168
5.85576923076923 0.546994626522064
6.13461538461539 0.537533521652222
6.42628205128205 0.574197113513947
6.73397435897436 0.569141805171967
7.05448717948718 0.524285912513733
7.38782051282051 0.53972989320755
7.74038461538461 0.519838452339172
8.10897435897436 0.464834213256836
8.49679487179487 0.453800588846207
8.90064102564103 0.463179796934128
9.32371794871795 0.502031803131104
9.76923076923077 0.397626042366028
10.2339743589744 0.406128019094467
10.7211538461538 0.443699091672897
11.2307692307692 0.344123691320419
11.7660256410256 0.414508253335953
12.3269230769231 0.377609103918076
12.9134615384615 0.339062422513962
13.5288461538462 0.397669851779938
14.1730769230769 0.373747766017914
14.849358974359 0.369808584451675
15.5544871794872 0.361682713031769
16.2948717948718 0.33836567401886
17.0705128205128 0.347890764474869
17.8846153846154 0.314610540866852
18.7371794871795 0.328371733427048
19.6282051282051 0.321426630020142
20.5641025641026 0.319091945886612
21.5416666666667 0.307366997003555
22.5673076923077 0.272907763719559
23.6442307692308 0.281391710042953
24.7692307692308 0.288220018148422
25.9487179487179 0.269860476255417
27.1826923076923 0.265939146280289
28.4775641025641 0.236077472567558
29.8333333333333 0.240741729736328
31.2564102564103 0.219699457287788
32.7435897435897 0.274763643741608
34.3044871794872 0.266452521085739
35.9358974358974 0.224049806594849
37.6474358974359 0.200556501746178
39.4391025641026 0.233722046017647
41.3173076923077 0.23330469429493
43.2852564102564 0.200644060969353
45.3461538461538 0.241697236895561
47.5064102564103 0.200261399149895
49.7692307692308 0.177268266677856
52.1378205128205 0.186943978071213
54.6217948717949 0.17124892771244
57.2211538461538 0.162881061434746
59.9455128205128 0.169677838683128
62.8012820512821 0.166471898555756
65.7916666666667 0.169251278042793
68.9230769230769 0.129498943686485
72.2051282051282 0.148728296160698
75.6442307692308 0.136617735028267
79.2467948717949 0.153871297836304
83.0192307692308 0.13096396625042
86.974358974359 0.142919063568115
91.1153846153846 0.133864000439644
95.4519230769231 0.120287202298641
100 0.12443633377552
};
\addplot [, color0, opacity=0.6, mark=diamond*, mark size=0.5, mark options={solid}, only marks, forget plot]
table {%
1 0.845916569232941
1.04487179487179 0.842062175273895
1.09615384615385 0.883130252361298
1.1474358974359 0.866719722747803
1.20192307692308 0.932085931301117
1.25961538461538 0.907064616680145
1.32051282051282 0.91059821844101
1.38461538461538 0.915305554866791
1.44871794871795 0.930165946483612
1.51923076923077 0.923670768737793
1.58974358974359 0.91000759601593
1.66666666666667 0.909145176410675
1.74679487179487 0.898723781108856
1.83012820512821 0.891693711280823
1.91666666666667 0.896263718605042
2.00641025641026 0.864108681678772
2.1025641025641 0.858955085277557
2.20512820512821 0.875910699367523
2.30769230769231 0.840061843395233
2.41987179487179 0.794665455818176
2.53525641025641 0.783516585826874
2.65384615384615 0.77463561296463
2.78205128205128 0.77858829498291
2.91346153846154 0.730015754699707
3.05128205128205 0.738743484020233
3.19871794871795 0.690069735050201
3.34935897435897 0.72129899263382
3.50961538461538 0.687578856945038
3.67628205128205 0.641691863536835
3.8525641025641 0.645978629589081
4.03525641025641 0.677906334400177
4.2275641025641 0.61445951461792
4.42948717948718 0.622399926185608
4.64102564102564 0.644111275672913
4.86217948717949 0.6007000207901
5.09294871794872 0.616980731487274
5.33653846153846 0.566904187202454
5.58974358974359 0.60029661655426
5.85576923076923 0.575075745582581
6.13461538461539 0.603174209594727
6.42628205128205 0.585057556629181
6.73397435897436 0.508771896362305
7.05448717948718 0.515156269073486
7.38782051282051 0.510006844997406
7.74038461538461 0.515608489513397
8.10897435897436 0.480144828557968
8.49679487179487 0.522282242774963
8.90064102564103 0.480451196432114
9.32371794871795 0.489544928073883
9.76923076923077 0.445629805326462
10.2339743589744 0.423765182495117
10.7211538461538 0.423987001180649
11.2307692307692 0.430731594562531
11.7660256410256 0.405017465353012
12.3269230769231 0.439331352710724
12.9134615384615 0.392494261264801
13.5288461538462 0.372928828001022
14.1730769230769 0.371385663747787
14.849358974359 0.352378219366074
15.5544871794872 0.366654723882675
16.2948717948718 0.350931406021118
17.0705128205128 0.378520339727402
17.8846153846154 0.348050802946091
18.7371794871795 0.291749387979507
19.6282051282051 0.347710102796555
20.5641025641026 0.307074934244156
21.5416666666667 0.374862015247345
22.5673076923077 0.269711971282959
23.6442307692308 0.2686927318573
24.7692307692308 0.261369079351425
25.9487179487179 0.259682148694992
27.1826923076923 0.249882504343987
28.4775641025641 0.248703479766846
29.8333333333333 0.226085141301155
31.2564102564103 0.258959621191025
32.7435897435897 0.226390510797501
34.3044871794872 0.220880016684532
35.9358974358974 0.236545950174332
37.6474358974359 0.222899720072746
39.4391025641026 0.222450017929077
41.3173076923077 0.201570630073547
43.2852564102564 0.190586537122726
45.3461538461538 0.201367244124413
47.5064102564103 0.190268754959106
49.7692307692308 0.172049760818481
52.1378205128205 0.175165176391602
54.6217948717949 0.186083629727364
57.2211538461538 0.165031909942627
59.9455128205128 0.163544625043869
62.8012820512821 0.163207098841667
65.7916666666667 0.149850562214851
68.9230769230769 0.133799344301224
72.2051282051282 0.141920894384384
75.6442307692308 0.150656268000603
79.2467948717949 0.156617060303688
83.0192307692308 0.146520391106606
86.974358974359 0.126060798764229
91.1153846153846 0.131151244044304
95.4519230769231 0.125335350632668
100 0.136466324329376
};
\addplot [, color1, opacity=0.6, mark=square*, mark size=0.5, mark options={solid}, only marks]
table {%
1 0.823391258716583
1.04487179487179 0.842215180397034
1.09615384615385 0.889334976673126
1.1474358974359 0.902535855770111
1.20192307692308 0.935880661010742
1.25961538461538 0.924263179302216
1.32051282051282 0.920876502990723
1.38461538461538 0.927456080913544
1.44871794871795 0.933768689632416
1.51923076923077 0.918422698974609
1.58974358974359 0.910473167896271
1.66666666666667 0.921622395515442
1.74679487179487 0.897418618202209
1.83012820512821 0.885995090007782
1.91666666666667 0.877628803253174
2.00641025641026 0.901624858379364
2.1025641025641 0.827723920345306
2.20512820512821 0.905844151973724
2.30769230769231 0.89058381319046
2.41987179487179 0.849358081817627
2.53525641025641 0.852563083171844
2.65384615384615 0.858367085456848
2.78205128205128 0.768650770187378
2.91346153846154 0.777651727199554
3.05128205128205 0.755616664886475
3.19871794871795 0.764954686164856
3.34935897435897 0.758058190345764
3.50961538461538 0.774447023868561
3.67628205128205 0.762955844402313
3.8525641025641 0.739735305309296
4.03525641025641 0.729641914367676
4.2275641025641 0.758092403411865
4.42948717948718 0.715419411659241
4.64102564102564 0.63623720407486
4.86217948717949 0.686236143112183
5.09294871794872 0.710280776023865
5.33653846153846 0.694095611572266
5.58974358974359 0.62928718328476
5.85576923076923 0.666550040245056
6.13461538461539 0.600450098514557
6.42628205128205 0.638733327388763
6.73397435897436 0.625997245311737
7.05448717948718 0.595963895320892
7.38782051282051 0.550736963748932
7.74038461538461 0.620732188224792
8.10897435897436 0.582258999347687
8.49679487179487 0.603067696094513
8.90064102564103 0.600725829601288
9.32371794871795 0.57937616109848
9.76923076923077 0.532348215579987
10.2339743589744 0.524379074573517
10.7211538461538 0.473377704620361
11.2307692307692 0.495040386915207
11.7660256410256 0.514038503170013
12.3269230769231 0.478236585855484
12.9134615384615 0.492099970579147
13.5288461538462 0.475808918476105
14.1730769230769 0.412688463926315
14.849358974359 0.432154387235641
15.5544871794872 0.474344223737717
16.2948717948718 0.440344244241714
17.0705128205128 0.492268294095993
17.8846153846154 0.40320560336113
18.7371794871795 0.369570881128311
19.6282051282051 0.338529855012894
20.5641025641026 0.320983022451401
21.5416666666667 0.413980633020401
22.5673076923077 0.378960222005844
23.6442307692308 0.365000575780869
24.7692307692308 0.341664463281631
25.9487179487179 0.303045481443405
27.1826923076923 0.322080224752426
28.4775641025641 0.256156295537949
29.8333333333333 0.299985557794571
31.2564102564103 0.30941778421402
32.7435897435897 0.295577198266983
34.3044871794872 0.285488218069077
35.9358974358974 0.253482073545456
37.6474358974359 0.209031566977501
39.4391025641026 0.19727049767971
41.3173076923077 0.220738098025322
43.2852564102564 0.199967280030251
45.3461538461538 0.201494887471199
47.5064102564103 0.215584620833397
49.7692307692308 0.198384687304497
52.1378205128205 0.203752189874649
54.6217948717949 0.163193479180336
57.2211538461538 0.20632492005825
59.9455128205128 0.152495786547661
62.8012820512821 0.170067459344864
65.7916666666667 0.143556877970695
68.9230769230769 0.156829744577408
72.2051282051282 0.135373160243034
75.6442307692308 0.141417667269707
79.2467948717949 0.140521839261055
83.0192307692308 0.148325487971306
86.974358974359 0.142252922058105
91.1153846153846 0.127506420016289
95.4519230769231 0.133091166615486
100 0.140429183840752
};
\addlegendentry{mb 128, mc 1}
\addplot [, color1, opacity=0.6, mark=square*, mark size=0.5, mark options={solid}, only marks, forget plot]
table {%
1 0.87957763671875
1.04487179487179 0.828774571418762
1.09615384615385 0.874753594398499
1.1474358974359 0.900301098823547
1.20192307692308 0.927363097667694
1.25961538461538 0.951544106006622
1.32051282051282 0.913272857666016
1.38461538461538 0.937634110450745
1.44871794871795 0.94763058423996
1.51923076923077 0.907957077026367
1.58974358974359 0.86978942155838
1.66666666666667 0.91970694065094
1.74679487179487 0.911797165870667
1.83012820512821 0.890633583068848
1.91666666666667 0.834814250469208
2.00641025641026 0.917069256305695
2.1025641025641 0.786681532859802
2.20512820512821 0.870474815368652
2.30769230769231 0.82436865568161
2.41987179487179 0.85211580991745
2.53525641025641 0.859180271625519
2.65384615384615 0.833840370178223
2.78205128205128 0.834819436073303
2.91346153846154 0.825025022029877
3.05128205128205 0.791481733322144
3.19871794871795 0.772403836250305
3.34935897435897 0.74486231803894
3.50961538461538 0.75981992483139
3.67628205128205 0.708748042583466
3.8525641025641 0.7497239112854
4.03525641025641 0.794406831264496
4.2275641025641 0.697329759597778
4.42948717948718 0.7423135638237
4.64102564102564 0.708115935325623
4.86217948717949 0.6645787358284
5.09294871794872 0.651331543922424
5.33653846153846 0.644330263137817
5.58974358974359 0.65204781293869
5.85576923076923 0.690769135951996
6.13461538461539 0.641214609146118
6.42628205128205 0.625691056251526
6.73397435897436 0.64350837469101
7.05448717948718 0.547377705574036
7.38782051282051 0.57829350233078
7.74038461538461 0.589384257793427
8.10897435897436 0.620327174663544
8.49679487179487 0.55165559053421
8.90064102564103 0.605317234992981
9.32371794871795 0.561595678329468
9.76923076923077 0.545445740222931
10.2339743589744 0.519560754299164
10.7211538461538 0.531488597393036
11.2307692307692 0.574404835700989
11.7660256410256 0.544439494609833
12.3269230769231 0.552535355091095
12.9134615384615 0.464041203260422
13.5288461538462 0.536853134632111
14.1730769230769 0.463635921478271
14.849358974359 0.43095263838768
15.5544871794872 0.471528291702271
16.2948717948718 0.489900797605515
17.0705128205128 0.454616516828537
17.8846153846154 0.413832873106003
18.7371794871795 0.414517611265182
19.6282051282051 0.403109401464462
20.5641025641026 0.423843622207642
21.5416666666667 0.411006897687912
22.5673076923077 0.412874311208725
23.6442307692308 0.391848534345627
24.7692307692308 0.362223953008652
25.9487179487179 0.322522342205048
27.1826923076923 0.354929178953171
28.4775641025641 0.290924519300461
29.8333333333333 0.307508558034897
31.2564102564103 0.338540941476822
32.7435897435897 0.304098606109619
34.3044871794872 0.269175976514816
35.9358974358974 0.290908694267273
37.6474358974359 0.244592472910881
39.4391025641026 0.240803003311157
41.3173076923077 0.263549894094467
43.2852564102564 0.235744342207909
45.3461538461538 0.221570774912834
47.5064102564103 0.258336931467056
49.7692307692308 0.205432698130608
52.1378205128205 0.200462728738785
54.6217948717949 0.24289308488369
57.2211538461538 0.182994619011879
59.9455128205128 0.194484278559685
62.8012820512821 0.163613274693489
65.7916666666667 0.172761157155037
68.9230769230769 0.141867563128471
72.2051282051282 0.132742926478386
75.6442307692308 0.138381108641624
79.2467948717949 0.158747658133507
83.0192307692308 0.142110258340836
86.974358974359 0.119228236377239
91.1153846153846 0.137062087655067
95.4519230769231 0.122919373214245
100 0.141973420977592
};
\addplot [, color1, opacity=0.6, mark=square*, mark size=0.5, mark options={solid}, only marks, forget plot]
table {%
1 0.813920974731445
1.04487179487179 0.853579938411713
1.09615384615385 0.877396106719971
1.1474358974359 0.909783542156219
1.20192307692308 0.931008338928223
1.25961538461538 0.945797920227051
1.32051282051282 0.923625767230988
1.38461538461538 0.927043855190277
1.44871794871795 0.936709403991699
1.51923076923077 0.876821219921112
1.58974358974359 0.930028557777405
1.66666666666667 0.902787208557129
1.74679487179487 0.920415878295898
1.83012820512821 0.880231082439423
1.91666666666667 0.860368192195892
2.00641025641026 0.904169499874115
2.1025641025641 0.817207753658295
2.20512820512821 0.884664058685303
2.30769230769231 0.884621262550354
2.41987179487179 0.859479129314423
2.53525641025641 0.851247012615204
2.65384615384615 0.856026649475098
2.78205128205128 0.778792023658752
2.91346153846154 0.771400511264801
3.05128205128205 0.760350406169891
3.19871794871795 0.776646554470062
3.34935897435897 0.772492825984955
3.50961538461538 0.723271310329437
3.67628205128205 0.740569770336151
3.8525641025641 0.740213572978973
4.03525641025641 0.734779536724091
4.2275641025641 0.677603423595428
4.42948717948718 0.688316643238068
4.64102564102564 0.695041358470917
4.86217948717949 0.691567659378052
5.09294871794872 0.711704015731812
5.33653846153846 0.709236085414886
5.58974358974359 0.67598968744278
5.85576923076923 0.658975541591644
6.13461538461539 0.66428929567337
6.42628205128205 0.605591952800751
6.73397435897436 0.580184936523438
7.05448717948718 0.58957177400589
7.38782051282051 0.617193818092346
7.74038461538461 0.583066523075104
8.10897435897436 0.543685674667358
8.49679487179487 0.534812569618225
8.90064102564103 0.572847068309784
9.32371794871795 0.535838425159454
9.76923076923077 0.47879084944725
10.2339743589744 0.419375628232956
10.7211538461538 0.475016415119171
11.2307692307692 0.482770442962646
11.7660256410256 0.442036211490631
12.3269230769231 0.461029440164566
12.9134615384615 0.487050443887711
13.5288461538462 0.439131170511246
14.1730769230769 0.415038108825684
14.849358974359 0.381323337554932
15.5544871794872 0.371467232704163
16.2948717948718 0.412198930978775
17.0705128205128 0.373337090015411
17.8846153846154 0.356059223413467
18.7371794871795 0.350715607404709
19.6282051282051 0.390187859535217
20.5641025641026 0.403272300958633
21.5416666666667 0.354211896657944
22.5673076923077 0.334307998418808
23.6442307692308 0.315705060958862
24.7692307692308 0.327926367521286
25.9487179487179 0.318911999464035
27.1826923076923 0.318706005811691
28.4775641025641 0.284682840108871
29.8333333333333 0.309139788150787
31.2564102564103 0.272540628910065
32.7435897435897 0.309625297784805
34.3044871794872 0.278086453676224
35.9358974358974 0.244552060961723
37.6474358974359 0.239249810576439
39.4391025641026 0.252473384141922
41.3173076923077 0.198404505848885
43.2852564102564 0.191971883177757
45.3461538461538 0.244633108377457
47.5064102564103 0.256941765546799
49.7692307692308 0.198323920369148
52.1378205128205 0.180555522441864
54.6217948717949 0.216850519180298
57.2211538461538 0.175423145294189
59.9455128205128 0.159764602780342
62.8012820512821 0.155464872717857
65.7916666666667 0.166446432471275
68.9230769230769 0.142659857869148
72.2051282051282 0.158121258020401
75.6442307692308 0.149655967950821
79.2467948717949 0.135649248957634
83.0192307692308 0.122383452951908
86.974358974359 0.1365697234869
91.1153846153846 0.137557059526443
95.4519230769231 0.105647042393684
100 0.121991850435734
};
\addplot [, color1, opacity=0.6, mark=square*, mark size=0.5, mark options={solid}, only marks, forget plot]
table {%
1 0.814961731433868
1.04487179487179 0.861804664134979
1.09615384615385 0.903393268585205
1.1474358974359 0.902513027191162
1.20192307692308 0.931597650051117
1.25961538461538 0.939505755901337
1.32051282051282 0.915442645549774
1.38461538461538 0.927477836608887
1.44871794871795 0.934845566749573
1.51923076923077 0.909969985485077
1.58974358974359 0.908619225025177
1.66666666666667 0.916705548763275
1.74679487179487 0.927593231201172
1.83012820512821 0.89570564031601
1.91666666666667 0.884585678577423
2.00641025641026 0.879919052124023
2.1025641025641 0.828971683979034
2.20512820512821 0.904527008533478
2.30769230769231 0.892909169197083
2.41987179487179 0.886720657348633
2.53525641025641 0.855257928371429
2.65384615384615 0.79511547088623
2.78205128205128 0.796644926071167
2.91346153846154 0.846791684627533
3.05128205128205 0.766516208648682
3.19871794871795 0.777772605419159
3.34935897435897 0.777154088020325
3.50961538461538 0.689082384109497
3.67628205128205 0.724733591079712
3.8525641025641 0.772805392742157
4.03525641025641 0.699395358562469
4.2275641025641 0.693999946117401
4.42948717948718 0.681119680404663
4.64102564102564 0.593441426753998
4.86217948717949 0.650995373725891
5.09294871794872 0.613939881324768
5.33653846153846 0.624313652515411
5.58974358974359 0.607631504535675
5.85576923076923 0.569897353649139
6.13461538461539 0.580474555492401
6.42628205128205 0.586857736110687
6.73397435897436 0.55461311340332
7.05448717948718 0.559203565120697
7.38782051282051 0.603739559650421
7.74038461538461 0.548218727111816
8.10897435897436 0.579538643360138
8.49679487179487 0.56089448928833
8.90064102564103 0.548054337501526
9.32371794871795 0.518293976783752
9.76923076923077 0.516376674175262
10.2339743589744 0.535245835781097
10.7211538461538 0.472121924161911
11.2307692307692 0.469164818525314
11.7660256410256 0.494981676340103
12.3269230769231 0.474410593509674
12.9134615384615 0.47300997376442
13.5288461538462 0.461851984262466
14.1730769230769 0.43564772605896
14.849358974359 0.467223167419434
15.5544871794872 0.432653158903122
16.2948717948718 0.398156851530075
17.0705128205128 0.406810134649277
17.8846153846154 0.330373674631119
18.7371794871795 0.386685580015182
19.6282051282051 0.3945472240448
20.5641025641026 0.36268812417984
21.5416666666667 0.363277703523636
22.5673076923077 0.337811142206192
23.6442307692308 0.319479703903198
24.7692307692308 0.328717321157455
25.9487179487179 0.340767592191696
27.1826923076923 0.304143756628036
28.4775641025641 0.262426614761353
29.8333333333333 0.313884079456329
31.2564102564103 0.252789109945297
32.7435897435897 0.278071075677872
34.3044871794872 0.237848803400993
35.9358974358974 0.266281992197037
37.6474358974359 0.207716226577759
39.4391025641026 0.244884833693504
41.3173076923077 0.241774350404739
43.2852564102564 0.18808887898922
45.3461538461538 0.201867386698723
47.5064102564103 0.191822364926338
49.7692307692308 0.205799534916878
52.1378205128205 0.193779021501541
54.6217948717949 0.201593667268753
57.2211538461538 0.157081291079521
59.9455128205128 0.148374885320663
62.8012820512821 0.172693774104118
65.7916666666667 0.154709547758102
68.9230769230769 0.135667085647583
72.2051282051282 0.175849035382271
75.6442307692308 0.156277179718018
79.2467948717949 0.130936414003372
83.0192307692308 0.116735257208347
86.974358974359 0.164088845252991
91.1153846153846 0.117777347564697
95.4519230769231 0.162224695086479
100 0.127921149134636
};
\addplot [, color1, opacity=0.6, mark=square*, mark size=0.5, mark options={solid}, only marks, forget plot]
table {%
1 0.814086556434631
1.04487179487179 0.820041000843048
1.09615384615385 0.905407845973969
1.1474358974359 0.907537460327148
1.20192307692308 0.946788311004639
1.25961538461538 0.93783712387085
1.32051282051282 0.91689395904541
1.38461538461538 0.937966763973236
1.44871794871795 0.947003304958344
1.51923076923077 0.924761474132538
1.58974358974359 0.933577656745911
1.66666666666667 0.92704576253891
1.74679487179487 0.923090755939484
1.83012820512821 0.894181668758392
1.91666666666667 0.815649151802063
2.00641025641026 0.916873753070831
2.1025641025641 0.903815448284149
2.20512820512821 0.899733364582062
2.30769230769231 0.890616238117218
2.41987179487179 0.8893723487854
2.53525641025641 0.889072597026825
2.65384615384615 0.820434033870697
2.78205128205128 0.873179614543915
2.91346153846154 0.823922336101532
3.05128205128205 0.806040227413177
3.19871794871795 0.782665848731995
3.34935897435897 0.776334941387177
3.50961538461538 0.754034340381622
3.67628205128205 0.772092461585999
3.8525641025641 0.782798588275909
4.03525641025641 0.788489043712616
4.2275641025641 0.636767566204071
4.42948717948718 0.682346522808075
4.64102564102564 0.727462291717529
4.86217948717949 0.693143367767334
5.09294871794872 0.73149311542511
5.33653846153846 0.672623693943024
5.58974358974359 0.672357559204102
5.85576923076923 0.679366767406464
6.13461538461539 0.636976003646851
6.42628205128205 0.669324219226837
6.73397435897436 0.667176425457001
7.05448717948718 0.621859848499298
7.38782051282051 0.685414969921112
7.74038461538461 0.610198140144348
8.10897435897436 0.628699958324432
8.49679487179487 0.644458293914795
8.90064102564103 0.589770257472992
9.32371794871795 0.596592903137207
9.76923076923077 0.556481063365936
10.2339743589744 0.547998666763306
10.7211538461538 0.543423593044281
11.2307692307692 0.557931125164032
11.7660256410256 0.533280551433563
12.3269230769231 0.496736437082291
12.9134615384615 0.469264358282089
13.5288461538462 0.430152714252472
14.1730769230769 0.480765789747238
14.849358974359 0.476478397846222
15.5544871794872 0.475318640470505
16.2948717948718 0.382304817438126
17.0705128205128 0.461870104074478
17.8846153846154 0.417780965566635
18.7371794871795 0.379417896270752
19.6282051282051 0.425748020410538
20.5641025641026 0.342852592468262
21.5416666666667 0.405255615711212
22.5673076923077 0.352134764194489
23.6442307692308 0.319680124521255
24.7692307692308 0.315171450376511
25.9487179487179 0.290882915258408
27.1826923076923 0.268134653568268
28.4775641025641 0.330175787210464
29.8333333333333 0.325382441282272
31.2564102564103 0.309806555509567
32.7435897435897 0.255855947732925
34.3044871794872 0.257487446069717
35.9358974358974 0.243169099092484
37.6474358974359 0.240131288766861
39.4391025641026 0.243887767195702
41.3173076923077 0.239833876490593
43.2852564102564 0.228459239006042
45.3461538461538 0.213240057229996
47.5064102564103 0.215318068861961
49.7692307692308 0.207732677459717
52.1378205128205 0.235179737210274
54.6217948717949 0.179309815168381
57.2211538461538 0.165841445326805
59.9455128205128 0.189157351851463
62.8012820512821 0.169837236404419
65.7916666666667 0.152424365282059
68.9230769230769 0.160150915384293
72.2051282051282 0.151564314961433
75.6442307692308 0.175773069262505
79.2467948717949 0.117102935910225
83.0192307692308 0.151815339922905
86.974358974359 0.143327280879021
91.1153846153846 0.153160110116005
95.4519230769231 0.13850474357605
100 0.130394786596298
};
\addplot [, color2, opacity=0.6, mark=triangle*, mark size=0.5, mark options={solid,rotate=180}, only marks]
table {%
1 0.442987680435181
1.04487179487179 0.453514963388443
1.09615384615385 0.508857190608978
1.1474358974359 0.51631635427475
1.20192307692308 0.548053860664368
1.25961538461538 0.581723034381866
1.32051282051282 0.56914758682251
1.38461538461538 0.624051988124847
1.44871794871795 0.603230059146881
1.51923076923077 0.575822055339813
1.58974358974359 0.519580662250519
1.66666666666667 0.538453280925751
1.74679487179487 0.577455639839172
1.83012820512821 0.465112030506134
1.91666666666667 0.501900315284729
2.00641025641026 0.480522364377975
2.1025641025641 0.498138517141342
2.20512820512821 0.451758146286011
2.30769230769231 0.489146769046783
2.41987179487179 0.506605327129364
2.53525641025641 0.459589689970016
2.65384615384615 0.460887730121613
2.78205128205128 0.521577060222626
2.91346153846154 0.501647651195526
3.05128205128205 0.446675300598145
3.19871794871795 0.404471546411514
3.34935897435897 0.425048023462296
3.50961538461538 0.43471547961235
3.67628205128205 0.426186770200729
3.8525641025641 0.389112055301666
4.03525641025641 0.36141037940979
4.2275641025641 0.386157214641571
4.42948717948718 0.39362621307373
4.64102564102564 0.336796820163727
4.86217948717949 0.326784700155258
5.09294871794872 0.373699545860291
5.33653846153846 0.336952775716782
5.58974358974359 0.320444732904434
5.85576923076923 0.34437757730484
6.13461538461539 0.317411661148071
6.42628205128205 0.366957485675812
6.73397435897436 0.315403163433075
7.05448717948718 0.310861319303513
7.38782051282051 0.294764757156372
7.74038461538461 0.299458742141724
8.10897435897436 0.30516916513443
8.49679487179487 0.313449114561081
8.90064102564103 0.304549932479858
9.32371794871795 0.291255563497543
9.76923076923077 0.311374515295029
10.2339743589744 0.287690550088882
10.7211538461538 0.274033337831497
11.2307692307692 0.303581774234772
11.7660256410256 0.291517466306686
12.3269230769231 0.275168538093567
12.9134615384615 0.250420093536377
13.5288461538462 0.276546329259872
14.1730769230769 0.246168091893196
14.849358974359 0.265483766794205
15.5544871794872 0.259662568569183
16.2948717948718 0.26627379655838
17.0705128205128 0.270033121109009
17.8846153846154 0.240415379405022
18.7371794871795 0.241636261343956
19.6282051282051 0.242404818534851
20.5641025641026 0.206714436411858
21.5416666666667 0.195420429110527
22.5673076923077 0.221726089715958
23.6442307692308 0.205098509788513
24.7692307692308 0.225963115692139
25.9487179487179 0.215676531195641
27.1826923076923 0.201812982559204
28.4775641025641 0.23456946015358
29.8333333333333 0.248164817690849
31.2564102564103 0.195357233285904
32.7435897435897 0.213364630937576
34.3044871794872 0.205700397491455
35.9358974358974 0.203671649098396
37.6474358974359 0.197671562433243
39.4391025641026 0.174669116735458
41.3173076923077 0.174284800887108
43.2852564102564 0.155085369944572
45.3461538461538 0.183879807591438
47.5064102564103 0.195349678397179
49.7692307692308 0.162939965724945
52.1378205128205 0.168420717120171
54.6217948717949 0.178241923451424
57.2211538461538 0.195703938603401
59.9455128205128 0.149719923734665
62.8012820512821 0.155604839324951
65.7916666666667 0.159613460302353
68.9230769230769 0.144018188118935
72.2051282051282 0.1764245480299
75.6442307692308 0.157525792717934
79.2467948717949 0.163749203085899
83.0192307692308 0.155408576130867
86.974358974359 0.16846863925457
91.1153846153846 0.14734773337841
95.4519230769231 0.14391665160656
100 0.156635954976082
};
\addlegendentry{sub 16, mc 1}
\addplot [, color2, opacity=0.6, mark=triangle*, mark size=0.5, mark options={solid,rotate=180}, only marks, forget plot]
table {%
1 0.45664244890213
1.04487179487179 0.488359540700912
1.09615384615385 0.474117964506149
1.1474358974359 0.533434987068176
1.20192307692308 0.560124218463898
1.25961538461538 0.620501935482025
1.32051282051282 0.520655274391174
1.38461538461538 0.525057733058929
1.44871794871795 0.544446647167206
1.51923076923077 0.522458970546722
1.58974358974359 0.513312101364136
1.66666666666667 0.565965414047241
1.74679487179487 0.542752921581268
1.83012820512821 0.520931720733643
1.91666666666667 0.547880411148071
2.00641025641026 0.488321304321289
2.1025641025641 0.548984467983246
2.20512820512821 0.502498388290405
2.30769230769231 0.501622080802917
2.41987179487179 0.483981221914291
2.53525641025641 0.495440691709518
2.65384615384615 0.4845130443573
2.78205128205128 0.510759770870209
2.91346153846154 0.499755680561066
3.05128205128205 0.497873395681381
3.19871794871795 0.398508280515671
3.34935897435897 0.414653271436691
3.50961538461538 0.413383692502975
3.67628205128205 0.415128141641617
3.8525641025641 0.408534914255142
4.03525641025641 0.361922264099121
4.2275641025641 0.427779734134674
4.42948717948718 0.389261990785599
4.64102564102564 0.403603881597519
4.86217948717949 0.369113564491272
5.09294871794872 0.3641017973423
5.33653846153846 0.354526996612549
5.58974358974359 0.382056146860123
5.85576923076923 0.360566049814224
6.13461538461539 0.314578324556351
6.42628205128205 0.364496558904648
6.73397435897436 0.328855037689209
7.05448717948718 0.332928508520126
7.38782051282051 0.336692869663239
7.74038461538461 0.336801588535309
8.10897435897436 0.352983444929123
8.49679487179487 0.321911811828613
8.90064102564103 0.316495060920715
9.32371794871795 0.315565556287766
9.76923076923077 0.336298942565918
10.2339743589744 0.304185122251511
10.7211538461538 0.313432425260544
11.2307692307692 0.302625179290771
11.7660256410256 0.293559044599533
12.3269230769231 0.295323193073273
12.9134615384615 0.250570684671402
13.5288461538462 0.274021446704865
14.1730769230769 0.278924137353897
14.849358974359 0.285097599029541
15.5544871794872 0.270343154668808
16.2948717948718 0.283934772014618
17.0705128205128 0.281418293714523
17.8846153846154 0.237061783671379
18.7371794871795 0.253382921218872
19.6282051282051 0.252987951040268
20.5641025641026 0.248651131987572
21.5416666666667 0.2256178855896
22.5673076923077 0.246026083827019
23.6442307692308 0.227260306477547
24.7692307692308 0.220754146575928
25.9487179487179 0.216524228453636
27.1826923076923 0.225194498896599
28.4775641025641 0.218571096658707
29.8333333333333 0.241528496146202
31.2564102564103 0.243770122528076
32.7435897435897 0.227981239557266
34.3044871794872 0.236047431826591
35.9358974358974 0.214144572615623
37.6474358974359 0.219956785440445
39.4391025641026 0.176924183964729
41.3173076923077 0.203414246439934
43.2852564102564 0.193079113960266
45.3461538461538 0.206042915582657
47.5064102564103 0.171578079462051
49.7692307692308 0.17240546643734
52.1378205128205 0.197866842150688
54.6217948717949 0.185724526643753
57.2211538461538 0.186682730913162
59.9455128205128 0.179983541369438
62.8012820512821 0.167170464992523
65.7916666666667 0.15029376745224
68.9230769230769 0.154174849390984
72.2051282051282 0.147461250424385
75.6442307692308 0.151970133185387
79.2467948717949 0.160637184977531
83.0192307692308 0.15930163860321
86.974358974359 0.148638397455215
91.1153846153846 0.125940963625908
95.4519230769231 0.136837914586067
100 0.154181644320488
};
\addplot [, color2, opacity=0.6, mark=triangle*, mark size=0.5, mark options={solid,rotate=180}, only marks, forget plot]
table {%
1 0.463568300008774
1.04487179487179 0.44919815659523
1.09615384615385 0.496870100498199
1.1474358974359 0.496309250593185
1.20192307692308 0.559290587902069
1.25961538461538 0.601864516735077
1.32051282051282 0.527247428894043
1.38461538461538 0.55091518163681
1.44871794871795 0.581195533275604
1.51923076923077 0.517044365406036
1.58974358974359 0.506243526935577
1.66666666666667 0.537494778633118
1.74679487179487 0.528838157653809
1.83012820512821 0.552605807781219
1.91666666666667 0.517944037914276
2.00641025641026 0.531650841236115
2.1025641025641 0.503016173839569
2.20512820512821 0.538962006568909
2.30769230769231 0.482532739639282
2.41987179487179 0.529026806354523
2.53525641025641 0.549134075641632
2.65384615384615 0.551187694072723
2.78205128205128 0.492201328277588
2.91346153846154 0.458098709583282
3.05128205128205 0.438568264245987
3.19871794871795 0.394972234964371
3.34935897435897 0.381661385297775
3.50961538461538 0.442925453186035
3.67628205128205 0.491434097290039
3.8525641025641 0.360328018665314
4.03525641025641 0.396867126226425
4.2275641025641 0.336165577173233
4.42948717948718 0.404158651828766
4.64102564102564 0.372887700796127
4.86217948717949 0.354875952005386
5.09294871794872 0.389184504747391
5.33653846153846 0.314355581998825
5.58974358974359 0.334769457578659
5.85576923076923 0.397382169961929
6.13461538461539 0.330030888319016
6.42628205128205 0.362342089414597
6.73397435897436 0.32244861125946
7.05448717948718 0.288790076971054
7.38782051282051 0.286737591028214
7.74038461538461 0.314424753189087
8.10897435897436 0.328387588262558
8.49679487179487 0.265753000974655
8.90064102564103 0.266751646995544
9.32371794871795 0.29220649600029
9.76923076923077 0.276252955198288
10.2339743589744 0.271228075027466
10.7211538461538 0.259152263402939
11.2307692307692 0.279304176568985
11.7660256410256 0.27250599861145
12.3269230769231 0.258378833532333
12.9134615384615 0.218941077589989
13.5288461538462 0.243590518832207
14.1730769230769 0.248198136687279
14.849358974359 0.263823598623276
15.5544871794872 0.269834011793137
16.2948717948718 0.247258976101875
17.0705128205128 0.252733558416367
17.8846153846154 0.235700473189354
18.7371794871795 0.242205217480659
19.6282051282051 0.241946563124657
20.5641025641026 0.217965230345726
21.5416666666667 0.222919389605522
22.5673076923077 0.214803293347359
23.6442307692308 0.209262043237686
24.7692307692308 0.224295094609261
25.9487179487179 0.208931088447571
27.1826923076923 0.189168974757195
28.4775641025641 0.195302471518517
29.8333333333333 0.22994776070118
31.2564102564103 0.234125301241875
32.7435897435897 0.190059661865234
34.3044871794872 0.189065620303154
35.9358974358974 0.187740713357925
37.6474358974359 0.190043240785599
39.4391025641026 0.171962007880211
41.3173076923077 0.178120836615562
43.2852564102564 0.16480877995491
45.3461538461538 0.167153149843216
47.5064102564103 0.167318314313889
49.7692307692308 0.173778459429741
52.1378205128205 0.150978028774261
54.6217948717949 0.1663988083601
57.2211538461538 0.159672975540161
59.9455128205128 0.167956575751305
62.8012820512821 0.136348500847816
65.7916666666667 0.149875909090042
68.9230769230769 0.141034036874771
72.2051282051282 0.152741074562073
75.6442307692308 0.145118221640587
79.2467948717949 0.153723910450935
83.0192307692308 0.143347293138504
86.974358974359 0.131125003099442
91.1153846153846 0.128940671682358
95.4519230769231 0.123244978487492
100 0.136153683066368
};
\addplot [, color2, opacity=0.6, mark=triangle*, mark size=0.5, mark options={solid,rotate=180}, only marks, forget plot]
table {%
1 0.442890256643295
1.04487179487179 0.465807527303696
1.09615384615385 0.491499274969101
1.1474358974359 0.492543518543243
1.20192307692308 0.531104981899261
1.25961538461538 0.5337815284729
1.32051282051282 0.473546892404556
1.38461538461538 0.506591796875
1.44871794871795 0.505273580551147
1.51923076923077 0.494515031576157
1.58974358974359 0.486622780561447
1.66666666666667 0.481880396604538
1.74679487179487 0.51718008518219
1.83012820512821 0.474972784519196
1.91666666666667 0.544571101665497
2.00641025641026 0.450578898191452
2.1025641025641 0.497553914785385
2.20512820512821 0.521302163600922
2.30769230769231 0.45562544465065
2.41987179487179 0.450323671102524
2.53525641025641 0.486208409070969
2.65384615384615 0.412342995405197
2.78205128205128 0.374674588441849
2.91346153846154 0.360656172037125
3.05128205128205 0.412627846002579
3.19871794871795 0.382128715515137
3.34935897435897 0.407662868499756
3.50961538461538 0.398483484983444
3.67628205128205 0.374871581792831
3.8525641025641 0.38334047794342
4.03525641025641 0.364080220460892
4.2275641025641 0.389536947011948
4.42948717948718 0.335583060979843
4.64102564102564 0.374951839447021
4.86217948717949 0.332113832235336
5.09294871794872 0.362540572881699
5.33653846153846 0.310377717018127
5.58974358974359 0.306521087884903
5.85576923076923 0.342299312353134
6.13461538461539 0.325338542461395
6.42628205128205 0.29438716173172
6.73397435897436 0.298663437366486
7.05448717948718 0.32064101099968
7.38782051282051 0.278215140104294
7.74038461538461 0.30039244890213
8.10897435897436 0.276203364133835
8.49679487179487 0.282353401184082
8.90064102564103 0.291979849338531
9.32371794871795 0.265044778585434
9.76923076923077 0.297161608934402
10.2339743589744 0.306463867425919
10.7211538461538 0.26125904917717
11.2307692307692 0.279982060194016
11.7660256410256 0.23257015645504
12.3269230769231 0.257759869098663
12.9134615384615 0.273136675357819
13.5288461538462 0.30076465010643
14.1730769230769 0.25873601436615
14.849358974359 0.240730032324791
15.5544871794872 0.237923502922058
16.2948717948718 0.25894159078598
17.0705128205128 0.20793791115284
17.8846153846154 0.251441925764084
18.7371794871795 0.231198355555534
19.6282051282051 0.229582354426384
20.5641025641026 0.20737612247467
21.5416666666667 0.218416377902031
22.5673076923077 0.200692698359489
23.6442307692308 0.206666097044945
24.7692307692308 0.204730197787285
25.9487179487179 0.202501535415649
27.1826923076923 0.187782526016235
28.4775641025641 0.202057287096977
29.8333333333333 0.196097046136856
31.2564102564103 0.181255623698235
32.7435897435897 0.187923938035965
34.3044871794872 0.160223871469498
35.9358974358974 0.176091715693474
37.6474358974359 0.164523750543594
39.4391025641026 0.182492956519127
41.3173076923077 0.165994361042976
43.2852564102564 0.159150034189224
45.3461538461538 0.171510845422745
47.5064102564103 0.140902370214462
49.7692307692308 0.138725787401199
52.1378205128205 0.134748324751854
54.6217948717949 0.144759640097618
57.2211538461538 0.145405247807503
59.9455128205128 0.144622012972832
62.8012820512821 0.166542336344719
65.7916666666667 0.141166135668755
68.9230769230769 0.133803591132164
72.2051282051282 0.149031579494476
75.6442307692308 0.151126965880394
79.2467948717949 0.152791857719421
83.0192307692308 0.146502569317818
86.974358974359 0.133316144347191
91.1153846153846 0.133518055081367
95.4519230769231 0.0990371704101562
100 0.120995998382568
};
\addplot [, color2, opacity=0.6, mark=triangle*, mark size=0.5, mark options={solid,rotate=180}, only marks, forget plot]
table {%
1 0.424876511096954
1.04487179487179 0.418141752481461
1.09615384615385 0.500442504882812
1.1474358974359 0.528636634349823
1.20192307692308 0.523607909679413
1.25961538461538 0.550910234451294
1.32051282051282 0.545012235641479
1.38461538461538 0.546003341674805
1.44871794871795 0.570225059986115
1.51923076923077 0.578176140785217
1.58974358974359 0.630516290664673
1.66666666666667 0.528062164783478
1.74679487179487 0.571726977825165
1.83012820512821 0.511195123195648
1.91666666666667 0.512043595314026
2.00641025641026 0.574773490428925
2.1025641025641 0.582970261573792
2.20512820512821 0.52328634262085
2.30769230769231 0.514954626560211
2.41987179487179 0.530214905738831
2.53525641025641 0.481572359800339
2.65384615384615 0.48029550909996
2.78205128205128 0.506241321563721
2.91346153846154 0.464868366718292
3.05128205128205 0.416675060987473
3.19871794871795 0.454407036304474
3.34935897435897 0.41616615653038
3.50961538461538 0.41492372751236
3.67628205128205 0.409368246793747
3.8525641025641 0.433226585388184
4.03525641025641 0.380622059106827
4.2275641025641 0.404396146535873
4.42948717948718 0.4060919880867
4.64102564102564 0.379947513341904
4.86217948717949 0.349557608366013
5.09294871794872 0.38306587934494
5.33653846153846 0.3383928835392
5.58974358974359 0.383353680372238
5.85576923076923 0.369841545820236
6.13461538461539 0.339458912611008
6.42628205128205 0.312626928091049
6.73397435897436 0.343037724494934
7.05448717948718 0.313010424375534
7.38782051282051 0.332656651735306
7.74038461538461 0.341195791959763
8.10897435897436 0.309576332569122
8.49679487179487 0.273302286863327
8.90064102564103 0.312992870807648
9.32371794871795 0.283845007419586
9.76923076923077 0.295323699712753
10.2339743589744 0.283484101295471
10.7211538461538 0.299267441034317
11.2307692307692 0.293455898761749
11.7660256410256 0.276124209165573
12.3269230769231 0.300295472145081
12.9134615384615 0.263891845941544
13.5288461538462 0.275073677301407
14.1730769230769 0.263921111822128
14.849358974359 0.243724510073662
15.5544871794872 0.221858888864517
16.2948717948718 0.27247616648674
17.0705128205128 0.263088583946228
17.8846153846154 0.259578466415405
18.7371794871795 0.217894420027733
19.6282051282051 0.248406037688255
20.5641025641026 0.228343531489372
21.5416666666667 0.250413149595261
22.5673076923077 0.235506936907768
23.6442307692308 0.233039379119873
24.7692307692308 0.207214340567589
25.9487179487179 0.209854394197464
27.1826923076923 0.199583813548088
28.4775641025641 0.209045320749283
29.8333333333333 0.198848739266396
31.2564102564103 0.187825068831444
32.7435897435897 0.192919611930847
34.3044871794872 0.17939218878746
35.9358974358974 0.186725944280624
37.6474358974359 0.173520252108574
39.4391025641026 0.196665167808533
41.3173076923077 0.184170290827751
43.2852564102564 0.162272915244102
45.3461538461538 0.187145337462425
47.5064102564103 0.145138785243034
49.7692307692308 0.16075287759304
52.1378205128205 0.156466946005821
54.6217948717949 0.177348092198372
57.2211538461538 0.155200436711311
59.9455128205128 0.134817287325859
62.8012820512821 0.149735972285271
65.7916666666667 0.148916438221931
68.9230769230769 0.117412738502026
72.2051282051282 0.13433600962162
75.6442307692308 0.114476136863232
79.2467948717949 0.12873487174511
83.0192307692308 0.134172335267067
86.974358974359 0.151155397295952
91.1153846153846 0.137189537286758
95.4519230769231 0.126561284065247
100 0.129756525158882
};
\end{axis}

\end{tikzpicture}

      \tikzexternaldisable
    \end{minipage}
  \end{subfigure}

  \begin{subfigure}[t]{\linewidth}
    \centering
    \caption{\cifarten \threecthreed \adam}
    \begin{minipage}{0.50\linewidth}
      \centering
      % defines the pgfplots style "eigspacedefault"
\pgfkeys{/pgfplots/eigspacedefault/.style={
    width=1.0\linewidth,
    height=0.6\linewidth,
    every axis plot/.append style={line width = 1.5pt},
    tick pos = left,
    ylabel near ticks,
    xlabel near ticks,
    xtick align = inside,
    ytick align = inside,
    legend cell align = left,
    legend columns = 4,
    legend pos = south east,
    legend style = {
      fill opacity = 1,
      text opacity = 1,
      font = \footnotesize,
      at={(1, 1.025)},
      anchor=south east,
      column sep=0.25cm,
    },
    legend image post style={scale=2.5},
    xticklabel style = {font = \footnotesize},
    xlabel style = {font = \footnotesize},
    axis line style = {black},
    yticklabel style = {font = \footnotesize},
    ylabel style = {font = \footnotesize},
    title style = {font = \footnotesize},
    grid = major,
    grid style = {dashed}
  }
}

\pgfkeys{/pgfplots/eigspacedefaultapp/.style={
    eigspacedefault,
    height=0.6\linewidth,
    legend columns = 2,
  }
}

\pgfkeys{/pgfplots/eigspacenolegend/.style={
    legend image post style = {scale=0},
    legend style = {
      fill opacity = 0,
      draw opacity = 0,
      text opacity = 0,
      font = \footnotesize,
      at={(1, 1.025)},
      anchor=south east,
      column sep=0.25cm,
    },
  }
}
%%% Local Variables:
%%% mode: latex
%%% TeX-master: "../../thesis"
%%% End:

      \pgfkeys{/pgfplots/zmystyle/.style={
          eigspacedefaultapp,
          eigspacenolegend,
        }}
      \tikzexternalenable
      \vspace{-6ex}
      % This file was created by tikzplotlib v0.9.7.
\begin{tikzpicture}

\definecolor{color0}{rgb}{0.501960784313725,0.184313725490196,0.6}
\definecolor{color1}{rgb}{0.870588235294118,0.623529411764706,0.0862745098039216}
\definecolor{color2}{rgb}{0.274509803921569,0.6,0.564705882352941}

\begin{axis}[
axis line style={white!10!black},
legend columns=2,
legend style={fill opacity=0.8, draw opacity=1, text opacity=1, at={(0.03,0.03)}, anchor=south west, draw=white!80!black},
log basis x={10},
tick pos=left,
xlabel={epoch (log scale)},
xmajorgrids,
xmin=0.794328234724281, xmax=125.892541179417,
xmode=log,
ylabel={overlap},
ymajorgrids,
ymin=-0.05, ymax=1.05,
zmystyle
]
\addplot [, white!10!black, dashed, forget plot]
table {%
0.794328234724281 1
125.892541179417 1
};
\addplot [, white!10!black, dashed, forget plot]
table {%
0.794328234724281 0
125.892541179417 0
};
\addplot [, color0, opacity=0.6, mark=triangle*, mark size=0.5, mark options={solid,rotate=180}, only marks]
table {%
1 0.592307209968567
1.04487179487179 0.546835482120514
1.09615384615385 0.603555679321289
1.1474358974359 0.598651230335236
1.20192307692308 0.552099347114563
1.25961538461538 0.594812452793121
1.32051282051282 0.464624881744385
1.38461538461538 0.541784167289734
1.44871794871795 0.517004311084747
1.51923076923077 0.463415771722794
1.58974358974359 0.486117750406265
1.66666666666667 0.472998708486557
1.74679487179487 0.482929140329361
1.83012820512821 0.441577523946762
1.91666666666667 0.463936388492584
2.00641025641026 0.419153779745102
2.1025641025641 0.427711725234985
2.20512820512821 0.448628157377243
2.30769230769231 0.416244477033615
2.41987179487179 0.384994357824326
2.53525641025641 0.404327005147934
2.65384615384615 0.341067999601364
2.78205128205128 0.399928867816925
2.91346153846154 0.367521464824677
3.05128205128205 0.351166844367981
3.19871794871795 0.348866283893585
3.34935897435897 0.357875496149063
3.50961538461538 0.353610426187515
3.67628205128205 0.347389757633209
3.8525641025641 0.350051134824753
4.03525641025641 0.319423377513885
4.2275641025641 0.337003320455551
4.42948717948718 0.309750378131866
4.64102564102564 0.311457186937332
4.86217948717949 0.30822429060936
5.09294871794872 0.302430063486099
5.33653846153846 0.287228554487228
5.58974358974359 0.295301586389542
5.85576923076923 0.288435041904449
6.13461538461539 0.297879159450531
6.42628205128205 0.291455805301666
6.73397435897436 0.293273776769638
7.05448717948718 0.276805967092514
7.38782051282051 0.282050400972366
7.74038461538461 0.27923846244812
8.10897435897436 0.257671684026718
8.49679487179487 0.271702021360397
8.90064102564103 0.259701102972031
9.32371794871795 0.26940530538559
9.76923076923077 0.274723589420319
10.2339743589744 0.271373957395554
10.7211538461538 0.277475923299789
11.2307692307692 0.263631880283356
11.7660256410256 0.277493000030518
12.3269230769231 0.2483199685812
12.9134615384615 0.251137882471085
13.5288461538462 0.220538377761841
14.1730769230769 0.236719653010368
14.849358974359 0.218988224864006
15.5544871794872 0.253796190023422
16.2948717948718 0.249460130929947
17.0705128205128 0.250696420669556
17.8846153846154 0.259507358074188
18.7371794871795 0.210837155580521
19.6282051282051 0.206601023674011
20.5641025641026 0.220209822058678
21.5416666666667 0.224931001663208
22.5673076923077 0.211947128176689
23.6442307692308 0.215814828872681
24.7692307692308 0.204044133424759
25.9487179487179 0.207341432571411
27.1826923076923 0.192573472857475
28.4775641025641 0.221555098891258
29.8333333333333 0.205034300684929
31.2564102564103 0.181921154260635
32.7435897435897 0.189766556024551
34.3044871794872 0.175309851765633
35.9358974358974 0.187651917338371
37.6474358974359 0.191715285181999
39.4391025641026 0.170455917716026
41.3173076923077 0.186214789748192
43.2852564102564 0.183850541710854
45.3461538461538 0.179818972945213
47.5064102564103 0.176553323864937
49.7692307692308 0.166018590331078
52.1378205128205 0.191725924611092
54.6217948717949 0.164068296551704
57.2211538461538 0.161347240209579
59.9455128205128 0.159445092082024
62.8012820512821 0.156176343560219
65.7916666666667 0.179518550634384
68.9230769230769 0.171437412500381
72.2051282051282 0.159951567649841
75.6442307692308 0.145247757434845
79.2467948717949 0.145979598164558
83.0192307692308 0.142506837844849
86.974358974359 0.167011812329292
91.1153846153846 0.160471752285957
95.4519230769231 0.118968799710274
100 0.127207607030869
};
\addlegendentry{mb 2, exact}
\addplot [, color0, opacity=0.6, mark=triangle*, mark size=0.5, mark options={solid,rotate=180}, only marks, forget plot]
table {%
1 0.585482776165009
1.04487179487179 0.642820537090302
1.09615384615385 0.737456917762756
1.1474358974359 0.703723013401031
1.20192307692308 0.659144341945648
1.25961538461538 0.624452710151672
1.32051282051282 0.512674272060394
1.38461538461538 0.58567488193512
1.44871794871795 0.573139607906342
1.51923076923077 0.542164325714111
1.58974358974359 0.590914249420166
1.66666666666667 0.535554945468903
1.74679487179487 0.548782050609589
1.83012820512821 0.508130073547363
1.91666666666667 0.509879767894745
2.00641025641026 0.499742567539215
2.1025641025641 0.498545736074448
2.20512820512821 0.517250716686249
2.30769230769231 0.490232080221176
2.41987179487179 0.453849047422409
2.53525641025641 0.486332803964615
2.65384615384615 0.456932753324509
2.78205128205128 0.523271560668945
2.91346153846154 0.473270416259766
3.05128205128205 0.450023949146271
3.19871794871795 0.438610702753067
3.34935897435897 0.440023511648178
3.50961538461538 0.449272453784943
3.67628205128205 0.410219579935074
3.8525641025641 0.422479212284088
4.03525641025641 0.429505825042725
4.2275641025641 0.428436905145645
4.42948717948718 0.425813645124435
4.64102564102564 0.396275013685226
4.86217948717949 0.417201995849609
5.09294871794872 0.414662599563599
5.33653846153846 0.44944429397583
5.58974358974359 0.415950924158096
5.85576923076923 0.393380492925644
6.13461538461539 0.404739588499069
6.42628205128205 0.434069454669952
6.73397435897436 0.431607455015182
7.05448717948718 0.432514011859894
7.38782051282051 0.447400808334351
7.74038461538461 0.415784686803818
8.10897435897436 0.423454433679581
8.49679487179487 0.395981788635254
8.90064102564103 0.439824670553207
9.32371794871795 0.396825343370438
9.76923076923077 0.418368339538574
10.2339743589744 0.429487377405167
10.7211538461538 0.405466556549072
11.2307692307692 0.403895378112793
11.7660256410256 0.389219135046005
12.3269230769231 0.412729173898697
12.9134615384615 0.396542400121689
13.5288461538462 0.377960354089737
14.1730769230769 0.365768760442734
14.849358974359 0.36883607506752
15.5544871794872 0.377174288034439
16.2948717948718 0.376384526491165
17.0705128205128 0.375020951032639
17.8846153846154 0.375096052885056
18.7371794871795 0.368546962738037
19.6282051282051 0.347338914871216
20.5641025641026 0.353253185749054
21.5416666666667 0.31588938832283
22.5673076923077 0.345776796340942
23.6442307692308 0.356596559286118
24.7692307692308 0.315961897373199
25.9487179487179 0.320177555084229
27.1826923076923 0.345033973455429
28.4775641025641 0.329159826040268
29.8333333333333 0.307424038648605
31.2564102564103 0.315818428993225
32.7435897435897 0.34919410943985
34.3044871794872 0.347409009933472
35.9358974358974 0.289627909660339
37.6474358974359 0.320884436368942
39.4391025641026 0.358908504247665
41.3173076923077 0.371110260486603
43.2852564102564 0.345608711242676
45.3461538461538 0.346660017967224
47.5064102564103 0.333024233579636
49.7692307692308 0.339248389005661
52.1378205128205 0.34049266576767
54.6217948717949 0.268183052539825
57.2211538461538 0.351125538349152
59.9455128205128 0.334892570972443
62.8012820512821 0.326612502336502
65.7916666666667 0.290396451950073
68.9230769230769 0.292425721883774
72.2051282051282 0.29774335026741
75.6442307692308 0.294426590204239
79.2467948717949 0.294831722974777
83.0192307692308 0.317448437213898
86.974358974359 0.308747172355652
91.1153846153846 0.284468680620193
95.4519230769231 0.277747005224228
100 0.272795349359512
};
\addplot [, color0, opacity=0.6, mark=triangle*, mark size=0.5, mark options={solid,rotate=180}, only marks, forget plot]
table {%
1 0.502753794193268
1.04487179487179 0.489509582519531
1.09615384615385 0.533672332763672
1.1474358974359 0.497904062271118
1.20192307692308 0.495550841093063
1.25961538461538 0.488884180784225
1.32051282051282 0.442734003067017
1.38461538461538 0.463622242212296
1.44871794871795 0.401593655347824
1.51923076923077 0.388241201639175
1.58974358974359 0.422877222299576
1.66666666666667 0.40520578622818
1.74679487179487 0.388937205076218
1.83012820512821 0.374259740114212
1.91666666666667 0.351854026317596
2.00641025641026 0.354837030172348
2.1025641025641 0.36358967423439
2.20512820512821 0.312477201223373
2.30769230769231 0.303010433912277
2.41987179487179 0.317272245883942
2.53525641025641 0.326904624700546
2.65384615384615 0.306135922670364
2.78205128205128 0.306288063526154
2.91346153846154 0.317791074514389
3.05128205128205 0.300774574279785
3.19871794871795 0.297990441322327
3.34935897435897 0.284591346979141
3.50961538461538 0.272943556308746
3.67628205128205 0.266404122114182
3.8525641025641 0.267289847135544
4.03525641025641 0.253162771463394
4.2275641025641 0.241636142134666
4.42948717948718 0.267525851726532
4.64102564102564 0.232459455728531
4.86217948717949 0.253876715898514
5.09294871794872 0.259975969791412
5.33653846153846 0.217476710677147
5.58974358974359 0.24880413711071
5.85576923076923 0.24929404258728
6.13461538461539 0.254732519388199
6.42628205128205 0.236306115984917
6.73397435897436 0.253792762756348
7.05448717948718 0.230030134320259
7.38782051282051 0.246845826506615
7.74038461538461 0.247952803969383
8.10897435897436 0.251498281955719
8.49679487179487 0.228025481104851
8.90064102564103 0.254550039768219
9.32371794871795 0.235579594969749
9.76923076923077 0.234713315963745
10.2339743589744 0.225211337208748
10.7211538461538 0.232245117425919
11.2307692307692 0.216675281524658
11.7660256410256 0.247531533241272
12.3269230769231 0.238133653998375
12.9134615384615 0.21694914996624
13.5288461538462 0.217873960733414
14.1730769230769 0.206582620739937
14.849358974359 0.225143864750862
15.5544871794872 0.240500882267952
16.2948717948718 0.233887299895287
17.0705128205128 0.261169612407684
17.8846153846154 0.174442604184151
18.7371794871795 0.207801461219788
19.6282051282051 0.240677699446678
20.5641025641026 0.21805964410305
21.5416666666667 0.237472414970398
22.5673076923077 0.233217343688011
23.6442307692308 0.238422617316246
24.7692307692308 0.226135089993477
25.9487179487179 0.229801654815674
27.1826923076923 0.216780588030815
28.4775641025641 0.221086263656616
29.8333333333333 0.2468171864748
31.2564102564103 0.260253548622131
32.7435897435897 0.252157956361771
34.3044871794872 0.19179430603981
35.9358974358974 0.203749522566795
37.6474358974359 0.197102665901184
39.4391025641026 0.225069284439087
41.3173076923077 0.212511539459229
43.2852564102564 0.213074713945389
45.3461538461538 0.209655001759529
47.5064102564103 0.20054192841053
49.7692307692308 0.225561618804932
52.1378205128205 0.194253131747246
54.6217948717949 0.198165774345398
57.2211538461538 0.212529584765434
59.9455128205128 0.227554872632027
62.8012820512821 0.229551315307617
65.7916666666667 0.211195811629295
68.9230769230769 0.20320038497448
72.2051282051282 0.220923289656639
75.6442307692308 0.184507206082344
79.2467948717949 0.216113522648811
83.0192307692308 0.212736949324608
86.974358974359 0.235083058476448
91.1153846153846 0.252482503652573
95.4519230769231 0.239584118127823
100 0.171486154198647
};
\addplot [, color0, opacity=0.6, mark=triangle*, mark size=0.5, mark options={solid,rotate=180}, only marks, forget plot]
table {%
1 0.57866096496582
1.04487179487179 0.676907181739807
1.09615384615385 0.66103857755661
1.1474358974359 0.613086640834808
1.20192307692308 0.646507441997528
1.25961538461538 0.600063323974609
1.32051282051282 0.530238389968872
1.38461538461538 0.594143986701965
1.44871794871795 0.527702271938324
1.51923076923077 0.467513561248779
1.58974358974359 0.519423186779022
1.66666666666667 0.478883475065231
1.74679487179487 0.516321837902069
1.83012820512821 0.454561084508896
1.91666666666667 0.463756650686264
2.00641025641026 0.458988279104233
2.1025641025641 0.441328257322311
2.20512820512821 0.441762298345566
2.30769230769231 0.410736948251724
2.41987179487179 0.42197060585022
2.53525641025641 0.4247986972332
2.65384615384615 0.407556265592575
2.78205128205128 0.408221930265427
2.91346153846154 0.398326724767685
3.05128205128205 0.385015189647675
3.19871794871795 0.403363198041916
3.34935897435897 0.416102796792984
3.50961538461538 0.387670904397964
3.67628205128205 0.397609114646912
3.8525641025641 0.395324558019638
4.03525641025641 0.397075176239014
4.2275641025641 0.369644612073898
4.42948717948718 0.382799923419952
4.64102564102564 0.377935349941254
4.86217948717949 0.383544087409973
5.09294871794872 0.38995549082756
5.33653846153846 0.383177846670151
5.58974358974359 0.375000923871994
5.85576923076923 0.358095198869705
6.13461538461539 0.372890800237656
6.42628205128205 0.387697488069534
6.73397435897436 0.360351651906967
7.05448717948718 0.369450837373734
7.38782051282051 0.314811795949936
7.74038461538461 0.325424253940582
8.10897435897436 0.327002912759781
8.49679487179487 0.312997251749039
8.90064102564103 0.329159170389175
9.32371794871795 0.31680753827095
9.76923076923077 0.32023224234581
10.2339743589744 0.334200203418732
10.7211538461538 0.294420212507248
11.2307692307692 0.30793172121048
11.7660256410256 0.282094895839691
12.3269230769231 0.300957351922989
12.9134615384615 0.269127458333969
13.5288461538462 0.270617634057999
14.1730769230769 0.243981033563614
14.849358974359 0.252620220184326
15.5544871794872 0.25038480758667
16.2948717948718 0.256727129220963
17.0705128205128 0.263109624385834
17.8846153846154 0.276291131973267
18.7371794871795 0.223415330052376
19.6282051282051 0.244529083371162
20.5641025641026 0.262406677007675
21.5416666666667 0.290581375360489
22.5673076923077 0.295093774795532
23.6442307692308 0.221500709652901
24.7692307692308 0.240476086735725
25.9487179487179 0.237801656126976
27.1826923076923 0.243841007351875
28.4775641025641 0.258962541818619
29.8333333333333 0.23039798438549
31.2564102564103 0.200932174921036
32.7435897435897 0.225850105285645
34.3044871794872 0.20632591843605
35.9358974358974 0.228188320994377
37.6474358974359 0.220425948500633
39.4391025641026 0.200400859117508
41.3173076923077 0.222618132829666
43.2852564102564 0.184313341975212
45.3461538461538 0.181449458003044
47.5064102564103 0.20966449379921
49.7692307692308 0.201858326792717
52.1378205128205 0.19481460750103
54.6217948717949 0.203439623117447
57.2211538461538 0.191560849547386
59.9455128205128 0.191939070820808
62.8012820512821 0.188076540827751
65.7916666666667 0.167691454291344
68.9230769230769 0.19236995279789
72.2051282051282 0.171185702085495
75.6442307692308 0.166320368647575
79.2467948717949 0.171476975083351
83.0192307692308 0.167338952422142
86.974358974359 0.167468458414078
91.1153846153846 0.183997794985771
95.4519230769231 0.1688362210989
100 0.181777387857437
};
\addplot [, color0, opacity=0.6, mark=triangle*, mark size=0.5, mark options={solid,rotate=180}, only marks, forget plot]
table {%
1 0.573790490627289
1.04487179487179 0.631611168384552
1.09615384615385 0.674712359905243
1.1474358974359 0.662199676036835
1.20192307692308 0.661509990692139
1.25961538461538 0.618332922458649
1.32051282051282 0.542199194431305
1.38461538461538 0.626749217510223
1.44871794871795 0.56579601764679
1.51923076923077 0.491115391254425
1.58974358974359 0.55174320936203
1.66666666666667 0.508000075817108
1.74679487179487 0.532376170158386
1.83012820512821 0.477628946304321
1.91666666666667 0.480161011219025
2.00641025641026 0.474416941404343
2.1025641025641 0.46269017457962
2.20512820512821 0.461393177509308
2.30769230769231 0.420006722211838
2.41987179487179 0.466852813959122
2.53525641025641 0.462906748056412
2.65384615384615 0.434481918811798
2.78205128205128 0.434114694595337
2.91346153846154 0.440793335437775
3.05128205128205 0.387094587087631
3.19871794871795 0.411156088113785
3.34935897435897 0.406082630157471
3.50961538461538 0.387847155332565
3.67628205128205 0.357632160186768
3.8525641025641 0.343068480491638
4.03525641025641 0.3526291847229
4.2275641025641 0.33409383893013
4.42948717948718 0.33807572722435
4.64102564102564 0.299622297286987
4.86217948717949 0.334631592035294
5.09294871794872 0.332187294960022
5.33653846153846 0.313861310482025
5.58974358974359 0.31063038110733
5.85576923076923 0.30160865187645
6.13461538461539 0.353870958089828
6.42628205128205 0.288804739713669
6.73397435897436 0.348492741584778
7.05448717948718 0.296383887529373
7.38782051282051 0.3315050303936
7.74038461538461 0.329625159502029
8.10897435897436 0.282636851072311
8.49679487179487 0.336550265550613
8.90064102564103 0.301706373691559
9.32371794871795 0.344898223876953
9.76923076923077 0.326054573059082
10.2339743589744 0.267272859811783
10.7211538461538 0.304627150297165
11.2307692307692 0.312139838933945
11.7660256410256 0.297453254461288
12.3269230769231 0.264882892370224
12.9134615384615 0.267457962036133
13.5288461538462 0.29574978351593
14.1730769230769 0.261932522058487
14.849358974359 0.277858376502991
15.5544871794872 0.247788816690445
16.2948717948718 0.251994550228119
17.0705128205128 0.260666489601135
17.8846153846154 0.249128490686417
18.7371794871795 0.237147927284241
19.6282051282051 0.255518108606339
20.5641025641026 0.234200105071068
21.5416666666667 0.237606689333916
22.5673076923077 0.240628153085709
23.6442307692308 0.245363473892212
24.7692307692308 0.240520715713501
25.9487179487179 0.196251600980759
27.1826923076923 0.227656751871109
28.4775641025641 0.2418542355299
29.8333333333333 0.240934997797012
31.2564102564103 0.226873978972435
32.7435897435897 0.246357545256615
34.3044871794872 0.198945805430412
35.9358974358974 0.211172565817833
37.6474358974359 0.191067695617676
39.4391025641026 0.200824543833733
41.3173076923077 0.208135411143303
43.2852564102564 0.17915952205658
45.3461538461538 0.195072337985039
47.5064102564103 0.221192955970764
49.7692307692308 0.220792457461357
52.1378205128205 0.184967175126076
54.6217948717949 0.205503970384598
57.2211538461538 0.197166845202446
59.9455128205128 0.180766612291336
62.8012820512821 0.199647411704063
65.7916666666667 0.211226090788841
68.9230769230769 0.210257574915886
72.2051282051282 0.19671343266964
75.6442307692308 0.20431461930275
79.2467948717949 0.196685194969177
83.0192307692308 0.208932682871819
86.974358974359 0.205271273851395
91.1153846153846 0.188923090696335
95.4519230769231 0.185675904154778
100 0.171028003096581
};
\addplot [, color1, opacity=0.6, mark=square*, mark size=0.5, mark options={solid}, only marks]
table {%
1 0.775229752063751
1.04487179487179 0.834898114204407
1.09615384615385 0.861268818378448
1.1474358974359 0.802887916564941
1.20192307692308 0.786221385002136
1.25961538461538 0.688303172588348
1.32051282051282 0.674230217933655
1.38461538461538 0.697616338729858
1.44871794871795 0.689489006996155
1.51923076923077 0.6154465675354
1.58974358974359 0.674067914485931
1.66666666666667 0.593099772930145
1.74679487179487 0.593142926692963
1.83012820512821 0.585586965084076
1.91666666666667 0.580171287059784
2.00641025641026 0.566816091537476
2.1025641025641 0.596588909626007
2.20512820512821 0.538447856903076
2.30769230769231 0.52722692489624
2.41987179487179 0.543999195098877
2.53525641025641 0.54193514585495
2.65384615384615 0.516180813312531
2.78205128205128 0.515158653259277
2.91346153846154 0.521232545375824
3.05128205128205 0.51141893863678
3.19871794871795 0.51218169927597
3.34935897435897 0.514228701591492
3.50961538461538 0.539941251277924
3.67628205128205 0.509822010993958
3.8525641025641 0.536152243614197
4.03525641025641 0.502993583679199
4.2275641025641 0.498270511627197
4.42948717948718 0.468910366296768
4.64102564102564 0.478029251098633
4.86217948717949 0.487383127212524
5.09294871794872 0.472910732030869
5.33653846153846 0.456267118453979
5.58974358974359 0.441177576780319
5.85576923076923 0.468663185834885
6.13461538461539 0.487639337778091
6.42628205128205 0.461875528097153
6.73397435897436 0.451742231845856
7.05448717948718 0.440451920032501
7.38782051282051 0.459627538919449
7.74038461538461 0.447512924671173
8.10897435897436 0.482622921466827
8.49679487179487 0.472595304250717
8.90064102564103 0.439602762460709
9.32371794871795 0.424852281808853
9.76923076923077 0.430189818143845
10.2339743589744 0.393048822879791
10.7211538461538 0.405276209115982
11.2307692307692 0.427704960107803
11.7660256410256 0.410948187112808
12.3269230769231 0.411720752716064
12.9134615384615 0.460963070392609
13.5288461538462 0.402852147817612
14.1730769230769 0.452136486768723
14.849358974359 0.390006691217422
15.5544871794872 0.375693142414093
16.2948717948718 0.393276125192642
17.0705128205128 0.374880760908127
17.8846153846154 0.458240807056427
18.7371794871795 0.427198141813278
19.6282051282051 0.416810512542725
20.5641025641026 0.405216991901398
21.5416666666667 0.389627307653427
22.5673076923077 0.387713372707367
23.6442307692308 0.414250046014786
24.7692307692308 0.40398508310318
25.9487179487179 0.367044627666473
27.1826923076923 0.404527515172958
28.4775641025641 0.427076488733292
29.8333333333333 0.406216681003571
31.2564102564103 0.356632679700851
32.7435897435897 0.333788931369781
34.3044871794872 0.337432056665421
35.9358974358974 0.400461912155151
37.6474358974359 0.368151664733887
39.4391025641026 0.343468755483627
41.3173076923077 0.328131169080734
43.2852564102564 0.315718233585358
45.3461538461538 0.329084128141403
47.5064102564103 0.324349790811539
49.7692307692308 0.334141701459885
52.1378205128205 0.271560162305832
54.6217948717949 0.321716129779816
57.2211538461538 0.289410918951035
59.9455128205128 0.337118864059448
62.8012820512821 0.334038347005844
65.7916666666667 0.314336270093918
68.9230769230769 0.294853419065475
72.2051282051282 0.276419997215271
75.6442307692308 0.272727727890015
79.2467948717949 0.217798635363579
83.0192307692308 0.307780057191849
86.974358974359 0.273792952299118
91.1153846153846 0.233823254704475
95.4519230769231 0.281128346920013
100 0.274847507476807
};
\addlegendentry{mb 8, exact}
\addplot [, color1, opacity=0.6, mark=square*, mark size=0.5, mark options={solid}, only marks, forget plot]
table {%
1 0.817290484905243
1.04487179487179 0.833622395992279
1.09615384615385 0.802972435951233
1.1474358974359 0.793247818946838
1.20192307692308 0.798309087753296
1.25961538461538 0.727238595485687
1.32051282051282 0.677721679210663
1.38461538461538 0.723292052745819
1.44871794871795 0.711740672588348
1.51923076923077 0.705733954906464
1.58974358974359 0.735188901424408
1.66666666666667 0.697189748287201
1.74679487179487 0.674855649471283
1.83012820512821 0.632772326469421
1.91666666666667 0.649274170398712
2.00641025641026 0.645285308361053
2.1025641025641 0.623648703098297
2.20512820512821 0.61297744512558
2.30769230769231 0.572667181491852
2.41987179487179 0.597496509552002
2.53525641025641 0.569752812385559
2.65384615384615 0.530937790870667
2.78205128205128 0.591774582862854
2.91346153846154 0.528895735740662
3.05128205128205 0.540111541748047
3.19871794871795 0.538449764251709
3.34935897435897 0.5384281873703
3.50961538461538 0.519609093666077
3.67628205128205 0.54844331741333
3.8525641025641 0.492975771427155
4.03525641025641 0.5104119181633
4.2275641025641 0.522556245326996
4.42948717948718 0.465496361255646
4.64102564102564 0.481006145477295
4.86217948717949 0.507571160793304
5.09294871794872 0.449863642454147
5.33653846153846 0.505663394927979
5.58974358974359 0.487150967121124
5.85576923076923 0.447920173406601
6.13461538461539 0.440423458814621
6.42628205128205 0.492852210998535
6.73397435897436 0.466593712568283
7.05448717948718 0.493542015552521
7.38782051282051 0.425025224685669
7.74038461538461 0.440595239400864
8.10897435897436 0.440321981906891
8.49679487179487 0.423727095127106
8.90064102564103 0.469213098287582
9.32371794871795 0.435114115476608
9.76923076923077 0.413553237915039
10.2339743589744 0.440394133329391
10.7211538461538 0.422849744558334
11.2307692307692 0.407037258148193
11.7660256410256 0.4073785841465
12.3269230769231 0.38017874956131
12.9134615384615 0.376680374145508
13.5288461538462 0.344733774662018
14.1730769230769 0.376605361700058
14.849358974359 0.379657685756683
15.5544871794872 0.359738916158676
16.2948717948718 0.381080955266953
17.0705128205128 0.385384529829025
17.8846153846154 0.383078187704086
18.7371794871795 0.372798174619675
19.6282051282051 0.35928612947464
20.5641025641026 0.358325332403183
21.5416666666667 0.327907085418701
22.5673076923077 0.363713979721069
23.6442307692308 0.356115251779556
24.7692307692308 0.29966613650322
25.9487179487179 0.320610523223877
27.1826923076923 0.328413993120193
28.4775641025641 0.29500937461853
29.8333333333333 0.320685625076294
31.2564102564103 0.278470396995544
32.7435897435897 0.314703196287155
34.3044871794872 0.300110310316086
35.9358974358974 0.270798057317734
37.6474358974359 0.273097932338715
39.4391025641026 0.277230083942413
41.3173076923077 0.285015106201172
43.2852564102564 0.289374470710754
45.3461538461538 0.268408000469208
47.5064102564103 0.247292473912239
49.7692307692308 0.270607203245163
52.1378205128205 0.248081400990486
54.6217948717949 0.249355658888817
57.2211538461538 0.260912716388702
59.9455128205128 0.261591762304306
62.8012820512821 0.26254940032959
65.7916666666667 0.234893187880516
68.9230769230769 0.254209667444229
72.2051282051282 0.272329688072205
75.6442307692308 0.240091368556023
79.2467948717949 0.255794554948807
83.0192307692308 0.218131348490715
86.974358974359 0.229501590132713
91.1153846153846 0.23914460837841
95.4519230769231 0.249630644917488
100 0.230527922511101
};
\addplot [, color1, opacity=0.6, mark=square*, mark size=0.5, mark options={solid}, only marks, forget plot]
table {%
1 0.790875136852264
1.04487179487179 0.859308183193207
1.09615384615385 0.795816123485565
1.1474358974359 0.766025424003601
1.20192307692308 0.75420469045639
1.25961538461538 0.753512561321259
1.32051282051282 0.652438580989838
1.38461538461538 0.698777377605438
1.44871794871795 0.679062485694885
1.51923076923077 0.654459297657013
1.58974358974359 0.689158380031586
1.66666666666667 0.65529078245163
1.74679487179487 0.649127423763275
1.83012820512821 0.603583931922913
1.91666666666667 0.603108108043671
2.00641025641026 0.606627821922302
2.1025641025641 0.572260797023773
2.20512820512821 0.624123275279999
2.30769230769231 0.584132373332977
2.41987179487179 0.576422989368439
2.53525641025641 0.607409358024597
2.65384615384615 0.590995490550995
2.78205128205128 0.635689198970795
2.91346153846154 0.604482591152191
3.05128205128205 0.594979763031006
3.19871794871795 0.574485301971436
3.34935897435897 0.533998608589172
3.50961538461538 0.550468862056732
3.67628205128205 0.540239632129669
3.8525641025641 0.545129477977753
4.03525641025641 0.542207896709442
4.2275641025641 0.552607238292694
4.42948717948718 0.521578967571259
4.64102564102564 0.511578917503357
4.86217948717949 0.489605247974396
5.09294871794872 0.521907091140747
5.33653846153846 0.483219534158707
5.58974358974359 0.461021959781647
5.85576923076923 0.452543824911118
6.13461538461539 0.507509231567383
6.42628205128205 0.424778521060944
6.73397435897436 0.47260782122612
7.05448717948718 0.452501684427261
7.38782051282051 0.440578371286392
7.74038461538461 0.413476675748825
8.10897435897436 0.409554570913315
8.49679487179487 0.428780764341354
8.90064102564103 0.435083538293839
9.32371794871795 0.431083589792252
9.76923076923077 0.405536860227585
10.2339743589744 0.419180363416672
10.7211538461538 0.38334658741951
11.2307692307692 0.396318465471268
11.7660256410256 0.379234999418259
12.3269230769231 0.39835250377655
12.9134615384615 0.3829625248909
13.5288461538462 0.336646050214767
14.1730769230769 0.36872336268425
14.849358974359 0.357752025127411
15.5544871794872 0.366139948368073
16.2948717948718 0.356333762407303
17.0705128205128 0.306655675172806
17.8846153846154 0.321937918663025
18.7371794871795 0.3731988966465
19.6282051282051 0.347998142242432
20.5641025641026 0.351135164499283
21.5416666666667 0.342731863260269
22.5673076923077 0.340925723314285
23.6442307692308 0.358181715011597
24.7692307692308 0.33562096953392
25.9487179487179 0.30395969748497
27.1826923076923 0.347082197666168
28.4775641025641 0.278556674718857
29.8333333333333 0.319155544042587
31.2564102564103 0.324093341827393
32.7435897435897 0.286205381155014
34.3044871794872 0.280888348817825
35.9358974358974 0.30543065071106
37.6474358974359 0.287291467189789
39.4391025641026 0.270487934350967
41.3173076923077 0.276360034942627
43.2852564102564 0.244526296854019
45.3461538461538 0.250813335180283
47.5064102564103 0.243494674563408
49.7692307692308 0.288633614778519
52.1378205128205 0.245680764317513
54.6217948717949 0.243079587817192
57.2211538461538 0.24867208302021
59.9455128205128 0.255564987659454
62.8012820512821 0.231819987297058
65.7916666666667 0.25693079829216
68.9230769230769 0.255542725324631
72.2051282051282 0.238452419638634
75.6442307692308 0.228410348296165
79.2467948717949 0.143796756863594
83.0192307692308 0.228780224919319
86.974358974359 0.21432788670063
91.1153846153846 0.160383209586143
95.4519230769231 0.183212637901306
100 0.217475175857544
};
\addplot [, color1, opacity=0.6, mark=square*, mark size=0.5, mark options={solid}, only marks, forget plot]
table {%
1 0.84156322479248
1.04487179487179 0.810401141643524
1.09615384615385 0.849722683429718
1.1474358974359 0.869491994380951
1.20192307692308 0.863043606281281
1.25961538461538 0.772471606731415
1.32051282051282 0.752154290676117
1.38461538461538 0.718194425106049
1.44871794871795 0.787082135677338
1.51923076923077 0.745417356491089
1.58974358974359 0.700117409229279
1.66666666666667 0.718453586101532
1.74679487179487 0.637394607067108
1.83012820512821 0.682915329933167
1.91666666666667 0.64766663312912
2.00641025641026 0.651719093322754
2.1025641025641 0.629267394542694
2.20512820512821 0.580933511257172
2.30769230769231 0.603518664836884
2.41987179487179 0.572541236877441
2.53525641025641 0.611217439174652
2.65384615384615 0.610289216041565
2.78205128205128 0.583103001117706
2.91346153846154 0.582615852355957
3.05128205128205 0.553732395172119
3.19871794871795 0.544501900672913
3.34935897435897 0.524566113948822
3.50961538461538 0.512038052082062
3.67628205128205 0.496919542551041
3.8525641025641 0.520865440368652
4.03525641025641 0.537538349628448
4.2275641025641 0.473673671483994
4.42948717948718 0.487593948841095
4.64102564102564 0.453553736209869
4.86217948717949 0.434724718332291
5.09294871794872 0.487040668725967
5.33653846153846 0.428161293268204
5.58974358974359 0.497292727231979
5.85576923076923 0.44395238161087
6.13461538461539 0.43573209643364
6.42628205128205 0.454729288816452
6.73397435897436 0.467222779989243
7.05448717948718 0.482168763875961
7.38782051282051 0.425057500600815
7.74038461538461 0.469218701124191
8.10897435897436 0.466367244720459
8.49679487179487 0.454442977905273
8.90064102564103 0.45413276553154
9.32371794871795 0.425470739603043
9.76923076923077 0.407605648040771
10.2339743589744 0.384866714477539
10.7211538461538 0.433651059865952
11.2307692307692 0.415202707052231
11.7660256410256 0.375048875808716
12.3269230769231 0.441868305206299
12.9134615384615 0.419702529907227
13.5288461538462 0.462206423282623
14.1730769230769 0.431682169437408
14.849358974359 0.428505718708038
15.5544871794872 0.393610090017319
16.2948717948718 0.380222767591476
17.0705128205128 0.343656182289124
17.8846153846154 0.393765419721603
18.7371794871795 0.37301430106163
19.6282051282051 0.328802078962326
20.5641025641026 0.37728476524353
21.5416666666667 0.357637286186218
22.5673076923077 0.342895656824112
23.6442307692308 0.388205260038376
24.7692307692308 0.368149280548096
25.9487179487179 0.38114133477211
27.1826923076923 0.343823879957199
28.4775641025641 0.337090462446213
29.8333333333333 0.361041992902756
31.2564102564103 0.32559609413147
32.7435897435897 0.383866786956787
34.3044871794872 0.319305419921875
35.9358974358974 0.358463227748871
37.6474358974359 0.356724947690964
39.4391025641026 0.345072656869888
41.3173076923077 0.303710848093033
43.2852564102564 0.341316372156143
45.3461538461538 0.291898727416992
47.5064102564103 0.378540724515915
49.7692307692308 0.331479996442795
52.1378205128205 0.296035289764404
54.6217948717949 0.390150994062424
57.2211538461538 0.328662484884262
59.9455128205128 0.318418085575104
62.8012820512821 0.378795385360718
65.7916666666667 0.307457029819489
68.9230769230769 0.308516889810562
72.2051282051282 0.341237306594849
75.6442307692308 0.368962168693542
79.2467948717949 0.273258566856384
83.0192307692308 0.316354423761368
86.974358974359 0.312498897314072
91.1153846153846 0.285168647766113
95.4519230769231 0.29633504152298
100 0.244280815124512
};
\addplot [, color1, opacity=0.6, mark=square*, mark size=0.5, mark options={solid}, only marks, forget plot]
table {%
1 0.832199692726135
1.04487179487179 0.852664947509766
1.09615384615385 0.852144837379456
1.1474358974359 0.806996762752533
1.20192307692308 0.804085254669189
1.25961538461538 0.753994345664978
1.32051282051282 0.731126725673676
1.38461538461538 0.739906311035156
1.44871794871795 0.736095309257507
1.51923076923077 0.713423430919647
1.58974358974359 0.698448181152344
1.66666666666667 0.695625603199005
1.74679487179487 0.719115674495697
1.83012820512821 0.661738216876984
1.91666666666667 0.650789558887482
2.00641025641026 0.649781823158264
2.1025641025641 0.567904591560364
2.20512820512821 0.54837840795517
2.30769230769231 0.568767249584198
2.41987179487179 0.579926133155823
2.53525641025641 0.567223846912384
2.65384615384615 0.564363956451416
2.78205128205128 0.52381557226181
2.91346153846154 0.518091201782227
3.05128205128205 0.534168660640717
3.19871794871795 0.555109143257141
3.34935897435897 0.55482017993927
3.50961538461538 0.544473648071289
3.67628205128205 0.502694189548492
3.8525641025641 0.543829321861267
4.03525641025641 0.522872149944305
4.2275641025641 0.530649602413177
4.42948717948718 0.536355674266815
4.64102564102564 0.498820304870605
4.86217948717949 0.512147724628448
5.09294871794872 0.517255783081055
5.33653846153846 0.509665310382843
5.58974358974359 0.487054973840714
5.85576923076923 0.450135856866837
6.13461538461539 0.457638174295425
6.42628205128205 0.463554292917252
6.73397435897436 0.484735459089279
7.05448717948718 0.486353009939194
7.38782051282051 0.452537357807159
7.74038461538461 0.465028017759323
8.10897435897436 0.468230575323105
8.49679487179487 0.425179094076157
8.90064102564103 0.459640234708786
9.32371794871795 0.429241806268692
9.76923076923077 0.428711801767349
10.2339743589744 0.444723606109619
10.7211538461538 0.431533426046371
11.2307692307692 0.420778423547745
11.7660256410256 0.4031982421875
12.3269230769231 0.414184719324112
12.9134615384615 0.402811378240585
13.5288461538462 0.37811404466629
14.1730769230769 0.41451159119606
14.849358974359 0.394625872373581
15.5544871794872 0.391471922397614
16.2948717948718 0.417096823453903
17.0705128205128 0.411496847867966
17.8846153846154 0.389127194881439
18.7371794871795 0.376384884119034
19.6282051282051 0.378056198358536
20.5641025641026 0.396567046642303
21.5416666666667 0.39749550819397
22.5673076923077 0.402093976736069
23.6442307692308 0.3363878428936
24.7692307692308 0.384241670370102
25.9487179487179 0.384370416402817
27.1826923076923 0.321665376424789
28.4775641025641 0.301871031522751
29.8333333333333 0.304227203130722
31.2564102564103 0.33698371052742
32.7435897435897 0.357223361730576
34.3044871794872 0.372363537549973
35.9358974358974 0.273480325937271
37.6474358974359 0.307365745306015
39.4391025641026 0.27471661567688
41.3173076923077 0.28671795129776
43.2852564102564 0.33003157377243
45.3461538461538 0.291920512914658
47.5064102564103 0.304850727319717
49.7692307692308 0.283096015453339
52.1378205128205 0.306422799825668
54.6217948717949 0.270999163389206
57.2211538461538 0.332549631595612
59.9455128205128 0.300238400697708
62.8012820512821 0.312691807746887
65.7916666666667 0.248575523495674
68.9230769230769 0.318888962268829
72.2051282051282 0.277192234992981
75.6442307692308 0.262075334787369
79.2467948717949 0.291235148906708
83.0192307692308 0.306561678647995
86.974358974359 0.242272689938545
91.1153846153846 0.256353706121445
95.4519230769231 0.242180705070496
100 0.247388124465942
};
\addplot [, color2, opacity=0.6, mark=diamond*, mark size=0.5, mark options={solid}, only marks]
table {%
1 0.923596322536469
1.04487179487179 0.948826730251312
1.09615384615385 0.932184875011444
1.1474358974359 0.853067398071289
1.20192307692308 0.936140954494476
1.25961538461538 0.867386043071747
1.32051282051282 0.824814796447754
1.38461538461538 0.788733124732971
1.44871794871795 0.797811985015869
1.51923076923077 0.821383893489838
1.58974358974359 0.780341804027557
1.66666666666667 0.785732805728912
1.74679487179487 0.753855526447296
1.83012820512821 0.810369491577148
1.91666666666667 0.787722826004028
2.00641025641026 0.81292450428009
2.1025641025641 0.745684564113617
2.20512820512821 0.750877201557159
2.30769230769231 0.775173842906952
2.41987179487179 0.742673218250275
2.53525641025641 0.811068713665009
2.65384615384615 0.743331730365753
2.78205128205128 0.735200881958008
2.91346153846154 0.73801326751709
3.05128205128205 0.738528966903687
3.19871794871795 0.696457982063293
3.34935897435897 0.740277945995331
3.50961538461538 0.66485995054245
3.67628205128205 0.685679614543915
3.8525641025641 0.676301598548889
4.03525641025641 0.673131108283997
4.2275641025641 0.632294178009033
4.42948717948718 0.65690678358078
4.64102564102564 0.689555644989014
4.86217948717949 0.682468056678772
5.09294871794872 0.68321430683136
5.33653846153846 0.618815362453461
5.58974358974359 0.637872517108917
5.85576923076923 0.618439793586731
6.13461538461539 0.633284091949463
6.42628205128205 0.616554379463196
6.73397435897436 0.604224383831024
7.05448717948718 0.610110282897949
7.38782051282051 0.613191545009613
7.74038461538461 0.567644536495209
8.10897435897436 0.586249351501465
8.49679487179487 0.608028709888458
8.90064102564103 0.594076156616211
9.32371794871795 0.572559714317322
9.76923076923077 0.592096626758575
10.2339743589744 0.627251267433167
10.7211538461538 0.550368964672089
11.2307692307692 0.549485325813293
11.7660256410256 0.56054276227951
12.3269230769231 0.516850769519806
12.9134615384615 0.565336644649506
13.5288461538462 0.511038780212402
14.1730769230769 0.535507321357727
14.849358974359 0.536588847637177
15.5544871794872 0.569409012794495
16.2948717948718 0.52327960729599
17.0705128205128 0.527316629886627
17.8846153846154 0.528558790683746
18.7371794871795 0.475571364164352
19.6282051282051 0.503554701805115
20.5641025641026 0.463877171278
21.5416666666667 0.512375950813293
22.5673076923077 0.434387028217316
23.6442307692308 0.498825639486313
24.7692307692308 0.430453985929489
25.9487179487179 0.500141620635986
27.1826923076923 0.471800416707993
28.4775641025641 0.452335804700851
29.8333333333333 0.442075252532959
31.2564102564103 0.446137249469757
32.7435897435897 0.411787122488022
34.3044871794872 0.389358043670654
35.9358974358974 0.360162705183029
37.6474358974359 0.401142418384552
39.4391025641026 0.37937119603157
41.3173076923077 0.370106518268585
43.2852564102564 0.329326212406158
45.3461538461538 0.315330415964127
47.5064102564103 0.353149741888046
49.7692307692308 0.33413153886795
52.1378205128205 0.312396258115768
54.6217948717949 0.247092172503471
57.2211538461538 0.290685564279556
59.9455128205128 0.278045952320099
62.8012820512821 0.277869075536728
65.7916666666667 0.264151334762573
68.9230769230769 0.234184190630913
72.2051282051282 0.282983273267746
75.6442307692308 0.218035265803337
79.2467948717949 0.210621744394302
83.0192307692308 0.243516847491264
86.974358974359 0.241593942046165
91.1153846153846 0.291359573602676
95.4519230769231 0.193163186311722
100 0.243242546916008
};
\addlegendentry{mb 32, exact}
\addplot [, color2, opacity=0.6, mark=diamond*, mark size=0.5, mark options={solid}, only marks, forget plot]
table {%
1 0.909110009670258
1.04487179487179 0.91324371099472
1.09615384615385 0.943406522274017
1.1474358974359 0.93501341342926
1.20192307692308 0.935328900814056
1.25961538461538 0.864865005016327
1.32051282051282 0.847835958003998
1.38461538461538 0.897116124629974
1.44871794871795 0.832647442817688
1.51923076923077 0.82162743806839
1.58974358974359 0.830810964107513
1.66666666666667 0.84141606092453
1.74679487179487 0.810281097888947
1.83012820512821 0.787113606929779
1.91666666666667 0.795083463191986
2.00641025641026 0.743732988834381
2.1025641025641 0.737479329109192
2.20512820512821 0.733865201473236
2.30769230769231 0.753193378448486
2.41987179487179 0.735922515392303
2.53525641025641 0.756545960903168
2.65384615384615 0.702017903327942
2.78205128205128 0.761334836483002
2.91346153846154 0.670202374458313
3.05128205128205 0.733997464179993
3.19871794871795 0.677244544029236
3.34935897435897 0.711730778217316
3.50961538461538 0.695183753967285
3.67628205128205 0.651515185832977
3.8525641025641 0.668325126171112
4.03525641025641 0.637773990631104
4.2275641025641 0.663382589817047
4.42948717948718 0.69854748249054
4.64102564102564 0.604002714157104
4.86217948717949 0.624508023262024
5.09294871794872 0.62965977191925
5.33653846153846 0.671830236911774
5.58974358974359 0.664434373378754
5.85576923076923 0.605644404888153
6.13461538461539 0.628252685070038
6.42628205128205 0.629720866680145
6.73397435897436 0.616782307624817
7.05448717948718 0.645887315273285
7.38782051282051 0.622330844402313
7.74038461538461 0.589641928672791
8.10897435897436 0.63943886756897
8.49679487179487 0.636010468006134
8.90064102564103 0.615337193012238
9.32371794871795 0.560231328010559
9.76923076923077 0.567024648189545
10.2339743589744 0.606737196445465
10.7211538461538 0.577746570110321
11.2307692307692 0.513975083827972
11.7660256410256 0.605417191982269
12.3269230769231 0.573237717151642
12.9134615384615 0.560134053230286
13.5288461538462 0.520385026931763
14.1730769230769 0.552833735942841
14.849358974359 0.567299485206604
15.5544871794872 0.49006849527359
16.2948717948718 0.533996999263763
17.0705128205128 0.490483850240707
17.8846153846154 0.514454483985901
18.7371794871795 0.510523855686188
19.6282051282051 0.498424053192139
20.5641025641026 0.481738150119781
21.5416666666667 0.445556789636612
22.5673076923077 0.511438846588135
23.6442307692308 0.442458629608154
24.7692307692308 0.40256068110466
25.9487179487179 0.444501399993896
27.1826923076923 0.435781449079514
28.4775641025641 0.39306303858757
29.8333333333333 0.427590429782867
31.2564102564103 0.379556506872177
32.7435897435897 0.347934484481812
34.3044871794872 0.380713552236557
35.9358974358974 0.375784069299698
37.6474358974359 0.343354463577271
39.4391025641026 0.350786298513412
41.3173076923077 0.318407446146011
43.2852564102564 0.314587205648422
45.3461538461538 0.315156042575836
47.5064102564103 0.318936109542847
49.7692307692308 0.345795124769211
52.1378205128205 0.350533217191696
54.6217948717949 0.308702081441879
57.2211538461538 0.327336460351944
59.9455128205128 0.277222514152527
62.8012820512821 0.255620628595352
65.7916666666667 0.279867708683014
68.9230769230769 0.259991317987442
72.2051282051282 0.28376841545105
75.6442307692308 0.24960994720459
79.2467948717949 0.236764907836914
83.0192307692308 0.246803596615791
86.974358974359 0.263487964868546
91.1153846153846 0.19944603741169
95.4519230769231 0.222612023353577
100 0.236471846699715
};
\addplot [, color2, opacity=0.6, mark=diamond*, mark size=0.5, mark options={solid}, only marks, forget plot]
table {%
1 0.939186036586761
1.04487179487179 0.960581600666046
1.09615384615385 0.972133636474609
1.1474358974359 0.963629066944122
1.20192307692308 0.957696855068207
1.25961538461538 0.861648201942444
1.32051282051282 0.898684680461884
1.38461538461538 0.847900331020355
1.44871794871795 0.830069243907928
1.51923076923077 0.823318660259247
1.58974358974359 0.810707211494446
1.66666666666667 0.789223611354828
1.74679487179487 0.822940051555634
1.83012820512821 0.84147834777832
1.91666666666667 0.817030608654022
2.00641025641026 0.832604885101318
2.1025641025641 0.814369142055511
2.20512820512821 0.789141356945038
2.30769230769231 0.77628093957901
2.41987179487179 0.766906261444092
2.53525641025641 0.762664496898651
2.65384615384615 0.78702586889267
2.78205128205128 0.782090127468109
2.91346153846154 0.780976057052612
3.05128205128205 0.789249062538147
3.19871794871795 0.751678049564362
3.34935897435897 0.771072387695312
3.50961538461538 0.707693934440613
3.67628205128205 0.704369187355042
3.8525641025641 0.711309731006622
4.03525641025641 0.686074912548065
4.2275641025641 0.702359020709991
4.42948717948718 0.637596547603607
4.64102564102564 0.673157870769501
4.86217948717949 0.703161180019379
5.09294871794872 0.648456513881683
5.33653846153846 0.704696714878082
5.58974358974359 0.703649640083313
5.85576923076923 0.646322250366211
6.13461538461539 0.67479807138443
6.42628205128205 0.632371842861176
6.73397435897436 0.619209110736847
7.05448717948718 0.655877709388733
7.38782051282051 0.64228618144989
7.74038461538461 0.578345894813538
8.10897435897436 0.59954845905304
8.49679487179487 0.644607424736023
8.90064102564103 0.59172511100769
9.32371794871795 0.565201222896576
9.76923076923077 0.557239532470703
10.2339743589744 0.540659189224243
10.7211538461538 0.555424869060516
11.2307692307692 0.527645409107208
11.7660256410256 0.558537006378174
12.3269230769231 0.54282009601593
12.9134615384615 0.533755958080292
13.5288461538462 0.452355116605759
14.1730769230769 0.521063446998596
14.849358974359 0.49442520737648
15.5544871794872 0.451494455337524
16.2948717948718 0.41840335726738
17.0705128205128 0.406205862760544
17.8846153846154 0.435362428426743
18.7371794871795 0.387539386749268
19.6282051282051 0.432558447122574
20.5641025641026 0.43563437461853
21.5416666666667 0.418601661920547
22.5673076923077 0.348040133714676
23.6442307692308 0.329881995916367
24.7692307692308 0.318592876195908
25.9487179487179 0.356861919164658
27.1826923076923 0.345322757959366
28.4775641025641 0.360304206609726
29.8333333333333 0.318399399518967
31.2564102564103 0.356953412294388
32.7435897435897 0.332024395465851
34.3044871794872 0.284701973199844
35.9358974358974 0.286424994468689
37.6474358974359 0.285911500453949
39.4391025641026 0.247767686843872
41.3173076923077 0.246083840727806
43.2852564102564 0.267797708511353
45.3461538461538 0.262297958135605
47.5064102564103 0.267339199781418
49.7692307692308 0.231055855751038
52.1378205128205 0.206860780715942
54.6217948717949 0.199464365839958
57.2211538461538 0.234666749835014
59.9455128205128 0.248019799590111
62.8012820512821 0.200420722365379
65.7916666666667 0.265529900789261
68.9230769230769 0.218236789107323
72.2051282051282 0.224767878651619
75.6442307692308 0.226058706641197
79.2467948717949 0.212494567036629
83.0192307692308 0.195730149745941
86.974358974359 0.23101818561554
91.1153846153846 0.23189289867878
95.4519230769231 0.199117541313171
100 0.165166720747948
};
\addplot [, color2, opacity=0.6, mark=diamond*, mark size=0.5, mark options={solid}, only marks, forget plot]
table {%
1 0.870489716529846
1.04487179487179 0.935364246368408
1.09615384615385 0.964053153991699
1.1474358974359 0.958510041236877
1.20192307692308 0.958301961421967
1.25961538461538 0.921665489673615
1.32051282051282 0.842723548412323
1.38461538461538 0.913231790065765
1.44871794871795 0.833271026611328
1.51923076923077 0.800750732421875
1.58974358974359 0.863539516925812
1.66666666666667 0.829654812812805
1.74679487179487 0.841280102729797
1.83012820512821 0.81115859746933
1.91666666666667 0.819491803646088
2.00641025641026 0.816375911235809
2.1025641025641 0.774799764156342
2.20512820512821 0.786855638027191
2.30769230769231 0.791038334369659
2.41987179487179 0.758108615875244
2.53525641025641 0.782817363739014
2.65384615384615 0.724452912807465
2.78205128205128 0.807171642780304
2.91346153846154 0.728748023509979
3.05128205128205 0.78945118188858
3.19871794871795 0.789476811885834
3.34935897435897 0.764855682849884
3.50961538461538 0.739502370357513
3.67628205128205 0.758156895637512
3.8525641025641 0.74662834405899
4.03525641025641 0.731626212596893
4.2275641025641 0.736231744289398
4.42948717948718 0.723129510879517
4.64102564102564 0.663209617137909
4.86217948717949 0.680518090724945
5.09294871794872 0.68487948179245
5.33653846153846 0.673256456851959
5.58974358974359 0.651246070861816
5.85576923076923 0.609593570232391
6.13461538461539 0.638026893138885
6.42628205128205 0.618495583534241
6.73397435897436 0.605496168136597
7.05448717948718 0.65783703327179
7.38782051282051 0.637362122535706
7.74038461538461 0.6588214635849
8.10897435897436 0.595795571804047
8.49679487179487 0.584287583827972
8.90064102564103 0.588076889514923
9.32371794871795 0.551468312740326
9.76923076923077 0.537117481231689
10.2339743589744 0.554873049259186
10.7211538461538 0.56600433588028
11.2307692307692 0.554200947284698
11.7660256410256 0.517959713935852
12.3269230769231 0.556491076946259
12.9134615384615 0.541067600250244
13.5288461538462 0.46036496758461
14.1730769230769 0.496356099843979
14.849358974359 0.480073422193527
15.5544871794872 0.495301306247711
16.2948717948718 0.457664549350739
17.0705128205128 0.450729936361313
17.8846153846154 0.485267251729965
18.7371794871795 0.480254650115967
19.6282051282051 0.440441608428955
20.5641025641026 0.424300670623779
21.5416666666667 0.42279776930809
22.5673076923077 0.396065056324005
23.6442307692308 0.365687519311905
24.7692307692308 0.345124691724777
25.9487179487179 0.369284152984619
27.1826923076923 0.342686921358109
28.4775641025641 0.388363242149353
29.8333333333333 0.340485632419586
31.2564102564103 0.334116131067276
32.7435897435897 0.329119861125946
34.3044871794872 0.32552632689476
35.9358974358974 0.341098845005035
37.6474358974359 0.310205519199371
39.4391025641026 0.3306964635849
41.3173076923077 0.291908174753189
43.2852564102564 0.283678859472275
45.3461538461538 0.294953346252441
47.5064102564103 0.299306273460388
49.7692307692308 0.326312065124512
52.1378205128205 0.33295202255249
54.6217948717949 0.288879603147507
57.2211538461538 0.294006496667862
59.9455128205128 0.284581869840622
62.8012820512821 0.300698757171631
65.7916666666667 0.268146961927414
68.9230769230769 0.255880564451218
72.2051282051282 0.325813591480255
75.6442307692308 0.239172413945198
79.2467948717949 0.258913040161133
83.0192307692308 0.327995985746384
86.974358974359 0.229960232973099
91.1153846153846 0.259236067533493
95.4519230769231 0.218886658549309
100 0.256976455450058
};
\addplot [, color2, opacity=0.6, mark=diamond*, mark size=0.5, mark options={solid}, only marks, forget plot]
table {%
1 0.903891086578369
1.04487179487179 0.960523068904877
1.09615384615385 0.969154298305511
1.1474358974359 0.957531869411469
1.20192307692308 0.953636348247528
1.25961538461538 0.869651794433594
1.32051282051282 0.843073189258575
1.38461538461538 0.833102822303772
1.44871794871795 0.837513089179993
1.51923076923077 0.826095223426819
1.58974358974359 0.819530427455902
1.66666666666667 0.767179429531097
1.74679487179487 0.804807126522064
1.83012820512821 0.767197787761688
1.91666666666667 0.766448795795441
2.00641025641026 0.801736176013947
2.1025641025641 0.784634232521057
2.20512820512821 0.76101940870285
2.30769230769231 0.744242191314697
2.41987179487179 0.756465435028076
2.53525641025641 0.707584023475647
2.65384615384615 0.761402130126953
2.78205128205128 0.7372727394104
2.91346153846154 0.719739139080048
3.05128205128205 0.712562024593353
3.19871794871795 0.719883501529694
3.34935897435897 0.70027220249176
3.50961538461538 0.683774828910828
3.67628205128205 0.684413969516754
3.8525641025641 0.663357138633728
4.03525641025641 0.633002936840057
4.2275641025641 0.66645336151123
4.42948717948718 0.633135497570038
4.64102564102564 0.599463939666748
4.86217948717949 0.609501779079437
5.09294871794872 0.591087162494659
5.33653846153846 0.599561393260956
5.58974358974359 0.591695606708527
5.85576923076923 0.604640960693359
6.13461538461539 0.602146565914154
6.42628205128205 0.570812225341797
6.73397435897436 0.566800713539124
7.05448717948718 0.616502225399017
7.38782051282051 0.576175391674042
7.74038461538461 0.531761050224304
8.10897435897436 0.534079074859619
8.49679487179487 0.578649699687958
8.90064102564103 0.562284648418427
9.32371794871795 0.572964429855347
9.76923076923077 0.516609668731689
10.2339743589744 0.509173214435577
10.7211538461538 0.49501821398735
11.2307692307692 0.480398893356323
11.7660256410256 0.504102349281311
12.3269230769231 0.493724882602692
12.9134615384615 0.497609227895737
13.5288461538462 0.45828253030777
14.1730769230769 0.460320293903351
14.849358974359 0.452550798654556
15.5544871794872 0.456444352865219
16.2948717948718 0.475094527006149
17.0705128205128 0.431394189596176
17.8846153846154 0.497371882200241
18.7371794871795 0.438674211502075
19.6282051282051 0.424494653940201
20.5641025641026 0.466628313064575
21.5416666666667 0.443159580230713
22.5673076923077 0.413116216659546
23.6442307692308 0.421346008777618
24.7692307692308 0.429418653249741
25.9487179487179 0.363405764102936
27.1826923076923 0.387297719717026
28.4775641025641 0.393820881843567
29.8333333333333 0.338872283697128
31.2564102564103 0.369581669569016
32.7435897435897 0.355920374393463
34.3044871794872 0.341797351837158
35.9358974358974 0.353797197341919
37.6474358974359 0.304833739995956
39.4391025641026 0.309753388166428
41.3173076923077 0.263724654912949
43.2852564102564 0.312895238399506
45.3461538461538 0.270923256874084
47.5064102564103 0.291931629180908
49.7692307692308 0.337716162204742
52.1378205128205 0.29282534122467
54.6217948717949 0.256777554750443
57.2211538461538 0.259248822927475
59.9455128205128 0.27093893289566
62.8012820512821 0.309323042631149
65.7916666666667 0.267243236303329
68.9230769230769 0.257831424474716
72.2051282051282 0.243019253015518
75.6442307692308 0.273396700620651
79.2467948717949 0.257131069898605
83.0192307692308 0.246834471821785
86.974358974359 0.213358119130135
91.1153846153846 0.294928878545761
95.4519230769231 0.231307178735733
100 0.25795790553093
};
\addplot [, black, opacity=0.6, mark=*, mark size=0.5, mark options={solid}, only marks]
table {%
1 0.980172097682953
1.04487179487179 0.991716206073761
1.09615384615385 0.992607712745667
1.1474358974359 0.989956080913544
1.20192307692308 0.990800857543945
1.25961538461538 0.907791912555695
1.32051282051282 0.961326599121094
1.38461538461538 0.88730925321579
1.44871794871795 0.947700977325439
1.51923076923077 0.937219440937042
1.58974358974359 0.949622809886932
1.66666666666667 0.968756198883057
1.74679487179487 0.887838780879974
1.83012820512821 0.916164219379425
1.91666666666667 0.947528004646301
2.00641025641026 0.922192215919495
2.1025641025641 0.928771018981934
2.20512820512821 0.897602498531342
2.30769230769231 0.870130360126495
2.41987179487179 0.878030240535736
2.53525641025641 0.866652131080627
2.65384615384615 0.909153580665588
2.78205128205128 0.937457025051117
2.91346153846154 0.9390869140625
3.05128205128205 0.866176545619965
3.19871794871795 0.88209742307663
3.34935897435897 0.900237679481506
3.50961538461538 0.905459046363831
3.67628205128205 0.9186971783638
3.8525641025641 0.922338128089905
4.03525641025641 0.842959880828857
4.2275641025641 0.896037757396698
4.42948717948718 0.919768989086151
4.64102564102564 0.837957322597504
4.86217948717949 0.866225063800812
5.09294871794872 0.874295353889465
5.33653846153846 0.857604801654816
5.58974358974359 0.839182078838348
5.85576923076923 0.831863701343536
6.13461538461539 0.827021539211273
6.42628205128205 0.851464450359344
6.73397435897436 0.853148102760315
7.05448717948718 0.827787399291992
7.38782051282051 0.85309374332428
7.74038461538461 0.840215861797333
8.10897435897436 0.842476665973663
8.49679487179487 0.846230506896973
8.90064102564103 0.825722634792328
9.32371794871795 0.853772282600403
9.76923076923077 0.857580304145813
10.2339743589744 0.806088864803314
10.7211538461538 0.82514476776123
11.2307692307692 0.816884994506836
11.7660256410256 0.813710987567902
12.3269230769231 0.778618633747101
12.9134615384615 0.794024109840393
13.5288461538462 0.733671605587006
14.1730769230769 0.75925225019455
14.849358974359 0.770133435726166
15.5544871794872 0.770629107952118
16.2948717948718 0.775415062904358
17.0705128205128 0.771032631397247
17.8846153846154 0.748296916484833
18.7371794871795 0.683089137077332
19.6282051282051 0.702462375164032
20.5641025641026 0.673430383205414
21.5416666666667 0.739798545837402
22.5673076923077 0.656615376472473
23.6442307692308 0.712085545063019
24.7692307692308 0.661113440990448
25.9487179487179 0.686876118183136
27.1826923076923 0.673646330833435
28.4775641025641 0.661878705024719
29.8333333333333 0.658248126506805
31.2564102564103 0.601502895355225
32.7435897435897 0.635036945343018
34.3044871794872 0.598907768726349
35.9358974358974 0.649761617183685
37.6474358974359 0.624386310577393
39.4391025641026 0.583890557289124
41.3173076923077 0.602699160575867
43.2852564102564 0.562650620937347
45.3461538461538 0.520900547504425
47.5064102564103 0.571963310241699
49.7692307692308 0.508267223834991
52.1378205128205 0.49919381737709
54.6217948717949 0.480009287595749
57.2211538461538 0.501744508743286
59.9455128205128 0.439036667346954
62.8012820512821 0.425910860300064
65.7916666666667 0.406243652105331
68.9230769230769 0.426879316568375
72.2051282051282 0.436816990375519
75.6442307692308 0.393632560968399
79.2467948717949 0.367149382829666
83.0192307692308 0.35765129327774
86.974358974359 0.372109144926071
91.1153846153846 0.358631610870361
95.4519230769231 0.363734215497971
100 0.373095273971558
};
\addlegendentry{mb 128, exact}
\addplot [, black, opacity=0.6, mark=*, mark size=0.5, mark options={solid}, only marks, forget plot]
table {%
1 0.980207860469818
1.04487179487179 0.986687839031219
1.09615384615385 0.99109822511673
1.1474358974359 0.988957226276398
1.20192307692308 0.989828884601593
1.25961538461538 0.981170475482941
1.32051282051282 0.944415390491486
1.38461538461538 0.97414493560791
1.44871794871795 0.977206408977509
1.51923076923077 0.970506191253662
1.58974358974359 0.97232574224472
1.66666666666667 0.960947215557098
1.74679487179487 0.94966596364975
1.83012820512821 0.947515308856964
1.91666666666667 0.953044831752777
2.00641025641026 0.931983470916748
2.1025641025641 0.938220322132111
2.20512820512821 0.923636555671692
2.30769230769231 0.946711838245392
2.41987179487179 0.919902265071869
2.53525641025641 0.92106819152832
2.65384615384615 0.876817166805267
2.78205128205128 0.888949573040009
2.91346153846154 0.922625064849854
3.05128205128205 0.911627113819122
3.19871794871795 0.863408267498016
3.34935897435897 0.921932399272919
3.50961538461538 0.911691844463348
3.67628205128205 0.869503676891327
3.8525641025641 0.902216255664825
4.03525641025641 0.885700643062592
4.2275641025641 0.849653422832489
4.42948717948718 0.834793746471405
4.64102564102564 0.833202004432678
4.86217948717949 0.89985865354538
5.09294871794872 0.843555629253387
5.33653846153846 0.880906879901886
5.58974358974359 0.910552024841309
5.85576923076923 0.877913773059845
6.13461538461539 0.821998298168182
6.42628205128205 0.865674018859863
6.73397435897436 0.856091499328613
7.05448717948718 0.854511678218842
7.38782051282051 0.867451846599579
7.74038461538461 0.811384618282318
8.10897435897436 0.835431277751923
8.49679487179487 0.874100983142853
8.90064102564103 0.797595322132111
9.32371794871795 0.824461162090302
9.76923076923077 0.82661360502243
10.2339743589744 0.82679957151413
10.7211538461538 0.822871804237366
11.2307692307692 0.773120164871216
11.7660256410256 0.766072332859039
12.3269230769231 0.809707462787628
12.9134615384615 0.796929776668549
13.5288461538462 0.747281968593597
14.1730769230769 0.740639626979828
14.849358974359 0.753473103046417
15.5544871794872 0.73205977678299
16.2948717948718 0.766981542110443
17.0705128205128 0.712304592132568
17.8846153846154 0.791382789611816
18.7371794871795 0.703667104244232
19.6282051282051 0.689324736595154
20.5641025641026 0.644148707389832
21.5416666666667 0.634638488292694
22.5673076923077 0.67344206571579
23.6442307692308 0.653737664222717
24.7692307692308 0.664027214050293
25.9487179487179 0.686160087585449
27.1826923076923 0.632227838039398
28.4775641025641 0.638557136058807
29.8333333333333 0.659567296504974
31.2564102564103 0.581326365470886
32.7435897435897 0.600731611251831
34.3044871794872 0.56149685382843
35.9358974358974 0.593187272548676
37.6474358974359 0.560176849365234
39.4391025641026 0.536633670330048
41.3173076923077 0.543563067913055
43.2852564102564 0.568399846553802
45.3461538461538 0.481604427099228
47.5064102564103 0.538267314434052
49.7692307692308 0.473466485738754
52.1378205128205 0.518011033535004
54.6217948717949 0.448529452085495
57.2211538461538 0.464230448007584
59.9455128205128 0.472301870584488
62.8012820512821 0.444409996271133
65.7916666666667 0.427443742752075
68.9230769230769 0.413966983556747
72.2051282051282 0.409690469503403
75.6442307692308 0.459181159734726
79.2467948717949 0.353957533836365
83.0192307692308 0.398236364126205
86.974358974359 0.31950780749321
91.1153846153846 0.304738879203796
95.4519230769231 0.313452303409576
100 0.317403376102448
};
\addplot [, black, opacity=0.6, mark=*, mark size=0.5, mark options={solid}, only marks, forget plot]
table {%
1 0.983946800231934
1.04487179487179 0.988971710205078
1.09615384615385 0.992758393287659
1.1474358974359 0.990560352802277
1.20192307692308 0.989403367042542
1.25961538461538 0.973469197750092
1.32051282051282 0.899483203887939
1.38461538461538 0.972841441631317
1.44871794871795 0.971960842609406
1.51923076923077 0.968037068843842
1.58974358974359 0.967146873474121
1.66666666666667 0.912840843200684
1.74679487179487 0.959828197956085
1.83012820512821 0.939577400684357
1.91666666666667 0.936375319957733
2.00641025641026 0.937602162361145
2.1025641025641 0.946089446544647
2.20512820512821 0.927560746669769
2.30769230769231 0.925248324871063
2.41987179487179 0.926818192005157
2.53525641025641 0.936004757881165
2.65384615384615 0.900654315948486
2.78205128205128 0.908032417297363
2.91346153846154 0.909767746925354
3.05128205128205 0.936206161975861
3.19871794871795 0.888727605342865
3.34935897435897 0.905566990375519
3.50961538461538 0.903665363788605
3.67628205128205 0.893167674541473
3.8525641025641 0.904410183429718
4.03525641025641 0.899736523628235
4.2275641025641 0.855421245098114
4.42948717948718 0.84848165512085
4.64102564102564 0.849940001964569
4.86217948717949 0.838516414165497
5.09294871794872 0.821752548217773
5.33653846153846 0.874599099159241
5.58974358974359 0.886141002178192
5.85576923076923 0.879460513591766
6.13461538461539 0.883392512798309
6.42628205128205 0.852098405361176
6.73397435897436 0.825584411621094
7.05448717948718 0.805654466152191
7.38782051282051 0.85651171207428
7.74038461538461 0.826787769794464
8.10897435897436 0.813583970069885
8.49679487179487 0.853635966777802
8.90064102564103 0.787913739681244
9.32371794871795 0.808611512184143
9.76923076923077 0.77556037902832
10.2339743589744 0.760887801647186
10.7211538461538 0.785118699073792
11.2307692307692 0.791158854961395
11.7660256410256 0.775297701358795
12.3269230769231 0.764155805110931
12.9134615384615 0.810503602027893
13.5288461538462 0.77249950170517
14.1730769230769 0.756299197673798
14.849358974359 0.724047720432281
15.5544871794872 0.702045857906342
16.2948717948718 0.755908310413361
17.0705128205128 0.690865457057953
17.8846153846154 0.739099204540253
18.7371794871795 0.696106493473053
19.6282051282051 0.656948268413544
20.5641025641026 0.730032682418823
21.5416666666667 0.692141950130463
22.5673076923077 0.69680780172348
23.6442307692308 0.670672595500946
24.7692307692308 0.637449860572815
25.9487179487179 0.644334495067596
27.1826923076923 0.635448336601257
28.4775641025641 0.646008908748627
29.8333333333333 0.639594495296478
31.2564102564103 0.57536393404007
32.7435897435897 0.579100787639618
34.3044871794872 0.598610639572144
35.9358974358974 0.641373157501221
37.6474358974359 0.558124184608459
39.4391025641026 0.518050730228424
41.3173076923077 0.534603893756866
43.2852564102564 0.511987686157227
45.3461538461538 0.50470495223999
47.5064102564103 0.483739674091339
49.7692307692308 0.471539467573166
52.1378205128205 0.392315179109573
54.6217948717949 0.452086359262466
57.2211538461538 0.38785719871521
59.9455128205128 0.432793229818344
62.8012820512821 0.401318460702896
65.7916666666667 0.384503036737442
68.9230769230769 0.244176179170609
72.2051282051282 0.297735422849655
75.6442307692308 0.372241079807281
79.2467948717949 0.284189194440842
83.0192307692308 0.352907985448837
86.974358974359 0.332353800535202
91.1153846153846 0.307710289955139
95.4519230769231 0.33690333366394
100 0.282698422670364
};
\addplot [, black, opacity=0.6, mark=*, mark size=0.5, mark options={solid}, only marks, forget plot]
table {%
1 0.982988655567169
1.04487179487179 0.992447853088379
1.09615384615385 0.993656158447266
1.1474358974359 0.990851044654846
1.20192307692308 0.989253461360931
1.25961538461538 0.947678983211517
1.32051282051282 0.947290420532227
1.38461538461538 0.93720954656601
1.44871794871795 0.948299825191498
1.51923076923077 0.963698089122772
1.58974358974359 0.967891335487366
1.66666666666667 0.898256957530975
1.74679487179487 0.940048694610596
1.83012820512821 0.948034584522247
1.91666666666667 0.927294194698334
2.00641025641026 0.900033593177795
2.1025641025641 0.879429519176483
2.20512820512821 0.871008992195129
2.30769230769231 0.928948998451233
2.41987179487179 0.922504425048828
2.53525641025641 0.897674560546875
2.65384615384615 0.855482757091522
2.78205128205128 0.916048645973206
2.91346153846154 0.892417430877686
3.05128205128205 0.895148456096649
3.19871794871795 0.88008588552475
3.34935897435897 0.883312880992889
3.50961538461538 0.869988441467285
3.67628205128205 0.875664055347443
3.8525641025641 0.895642876625061
4.03525641025641 0.853973388671875
4.2275641025641 0.823393523693085
4.42948717948718 0.86317777633667
4.64102564102564 0.853497803211212
4.86217948717949 0.854592263698578
5.09294871794872 0.832616984844208
5.33653846153846 0.821280896663666
5.58974358974359 0.801458775997162
5.85576923076923 0.821565270423889
6.13461538461539 0.808993637561798
6.42628205128205 0.83081191778183
6.73397435897436 0.827543199062347
7.05448717948718 0.815712153911591
7.38782051282051 0.772370278835297
7.74038461538461 0.788699567317963
8.10897435897436 0.818086802959442
8.49679487179487 0.793797016143799
8.90064102564103 0.739077568054199
9.32371794871795 0.76023668050766
9.76923076923077 0.78579705953598
10.2339743589744 0.788220643997192
10.7211538461538 0.775658845901489
11.2307692307692 0.789422690868378
11.7660256410256 0.799331724643707
12.3269230769231 0.750704288482666
12.9134615384615 0.782992780208588
13.5288461538462 0.727857768535614
14.1730769230769 0.753467559814453
14.849358974359 0.715115964412689
15.5544871794872 0.71807450056076
16.2948717948718 0.719560086727142
17.0705128205128 0.712568938732147
17.8846153846154 0.773623049259186
18.7371794871795 0.725335478782654
19.6282051282051 0.712160050868988
20.5641025641026 0.708537459373474
21.5416666666667 0.71966141462326
22.5673076923077 0.670502722263336
23.6442307692308 0.675801575183868
24.7692307692308 0.722430825233459
25.9487179487179 0.691539704799652
27.1826923076923 0.679106056690216
28.4775641025641 0.644481837749481
29.8333333333333 0.676703870296478
31.2564102564103 0.677815973758698
32.7435897435897 0.662394165992737
34.3044871794872 0.585131764411926
35.9358974358974 0.593388736248016
37.6474358974359 0.601539433002472
39.4391025641026 0.586460292339325
41.3173076923077 0.527765691280365
43.2852564102564 0.563214004039764
45.3461538461538 0.503712356090546
47.5064102564103 0.531544148921967
49.7692307692308 0.531743943691254
52.1378205128205 0.483150869607925
54.6217948717949 0.520426452159882
57.2211538461538 0.445996850728989
59.9455128205128 0.488213866949081
62.8012820512821 0.411276251077652
65.7916666666667 0.443762362003326
68.9230769230769 0.447615712881088
72.2051282051282 0.410937458276749
75.6442307692308 0.370191097259521
79.2467948717949 0.346092462539673
83.0192307692308 0.437259286642075
86.974358974359 0.347940772771835
91.1153846153846 0.33807760477066
95.4519230769231 0.378379315137863
100 0.385885238647461
};
\addplot [, black, opacity=0.6, mark=*, mark size=0.5, mark options={solid}, only marks, forget plot]
table {%
1 0.98374742269516
1.04487179487179 0.99119645357132
1.09615384615385 0.993699729442596
1.1474358974359 0.99211448431015
1.20192307692308 0.990588963031769
1.25961538461538 0.986877083778381
1.32051282051282 0.97865241765976
1.38461538461538 0.972180366516113
1.44871794871795 0.95185375213623
1.51923076923077 0.970642864704132
1.58974358974359 0.96730625629425
1.66666666666667 0.977256953716278
1.74679487179487 0.942020893096924
1.83012820512821 0.953889489173889
1.91666666666667 0.960329711437225
2.00641025641026 0.972863674163818
2.1025641025641 0.954428672790527
2.20512820512821 0.941209018230438
2.30769230769231 0.938596367835999
2.41987179487179 0.945420861244202
2.53525641025641 0.92853319644928
2.65384615384615 0.896694481372833
2.78205128205128 0.949042141437531
2.91346153846154 0.932797908782959
3.05128205128205 0.931452929973602
3.19871794871795 0.939801871776581
3.34935897435897 0.919247806072235
3.50961538461538 0.930974185466766
3.67628205128205 0.929368615150452
3.8525641025641 0.92060798406601
4.03525641025641 0.896500587463379
4.2275641025641 0.896802723407745
4.42948717948718 0.908545315265656
4.64102564102564 0.859995663166046
4.86217948717949 0.881321370601654
5.09294871794872 0.890502572059631
5.33653846153846 0.845744609832764
5.58974358974359 0.849791526794434
5.85576923076923 0.823348045349121
6.13461538461539 0.846448540687561
6.42628205128205 0.854026317596436
6.73397435897436 0.871932685375214
7.05448717948718 0.865126073360443
7.38782051282051 0.870404899120331
7.74038461538461 0.837932050228119
8.10897435897436 0.855006158351898
8.49679487179487 0.83854866027832
8.90064102564103 0.777332901954651
9.32371794871795 0.803152740001678
9.76923076923077 0.85256427526474
10.2339743589744 0.77681827545166
10.7211538461538 0.762219071388245
11.2307692307692 0.827032208442688
11.7660256410256 0.762560188770294
12.3269230769231 0.791698157787323
12.9134615384615 0.779714524745941
13.5288461538462 0.720747232437134
14.1730769230769 0.766886711120605
14.849358974359 0.783985257148743
15.5544871794872 0.681840837001801
16.2948717948718 0.746723890304565
17.0705128205128 0.732045769691467
17.8846153846154 0.719694912433624
18.7371794871795 0.711341559886932
19.6282051282051 0.666857838630676
20.5641025641026 0.634184539318085
21.5416666666667 0.637069165706635
22.5673076923077 0.70036906003952
23.6442307692308 0.615962445735931
24.7692307692308 0.640998721122742
25.9487179487179 0.668058335781097
27.1826923076923 0.653937339782715
28.4775641025641 0.580042839050293
29.8333333333333 0.623236477375031
31.2564102564103 0.546408355236053
32.7435897435897 0.630167126655579
34.3044871794872 0.565100312232971
35.9358974358974 0.522231757640839
37.6474358974359 0.509883403778076
39.4391025641026 0.487043231725693
41.3173076923077 0.491316556930542
43.2852564102564 0.464536666870117
45.3461538461538 0.469104677438736
47.5064102564103 0.46718493103981
49.7692307692308 0.463799208402634
52.1378205128205 0.467945963144302
54.6217948717949 0.437769860029221
57.2211538461538 0.466181248426437
59.9455128205128 0.439740240573883
62.8012820512821 0.377387911081314
65.7916666666667 0.388364851474762
68.9230769230769 0.362410694360733
72.2051282051282 0.351156383752823
75.6442307692308 0.339453905820847
79.2467948717949 0.348567098379135
83.0192307692308 0.412407159805298
86.974358974359 0.329574853181839
91.1153846153846 0.317077040672302
95.4519230769231 0.365579962730408
100 0.345400899648666
};
\end{axis}

\end{tikzpicture}

      \tikzexternaldisable
    \end{minipage}\hfill
    \begin{minipage}{0.50\linewidth}
      \centering
      % defines the pgfplots style "eigspacedefault"
\pgfkeys{/pgfplots/eigspacedefault/.style={
    width=1.0\linewidth,
    height=0.6\linewidth,
    every axis plot/.append style={line width = 1.5pt},
    tick pos = left,
    ylabel near ticks,
    xlabel near ticks,
    xtick align = inside,
    ytick align = inside,
    legend cell align = left,
    legend columns = 4,
    legend pos = south east,
    legend style = {
      fill opacity = 1,
      text opacity = 1,
      font = \footnotesize,
      at={(1, 1.025)},
      anchor=south east,
      column sep=0.25cm,
    },
    legend image post style={scale=2.5},
    xticklabel style = {font = \footnotesize},
    xlabel style = {font = \footnotesize},
    axis line style = {black},
    yticklabel style = {font = \footnotesize},
    ylabel style = {font = \footnotesize},
    title style = {font = \footnotesize},
    grid = major,
    grid style = {dashed}
  }
}

\pgfkeys{/pgfplots/eigspacedefaultapp/.style={
    eigspacedefault,
    height=0.6\linewidth,
    legend columns = 2,
  }
}

\pgfkeys{/pgfplots/eigspacenolegend/.style={
    legend image post style = {scale=0},
    legend style = {
      fill opacity = 0,
      draw opacity = 0,
      text opacity = 0,
      font = \footnotesize,
      at={(1, 1.025)},
      anchor=south east,
      column sep=0.25cm,
    },
  }
}
%%% Local Variables:
%%% mode: latex
%%% TeX-master: "../../thesis"
%%% End:

      \pgfkeys{/pgfplots/zmystyle/.style={
          eigspacedefaultapp,
          eigspacenolegend,
        }}
      \tikzexternalenable
      \vspace{-6ex}
      % This file was created by tikzplotlib v0.9.7.
\begin{tikzpicture}

\definecolor{color0}{rgb}{0.274509803921569,0.6,0.564705882352941}
\definecolor{color1}{rgb}{0.870588235294118,0.623529411764706,0.0862745098039216}
\definecolor{color2}{rgb}{0.501960784313725,0.184313725490196,0.6}

\begin{axis}[
axis line style={white!10!black},
legend columns=2,
legend style={fill opacity=0.8, draw opacity=1, text opacity=1, at={(0.03,0.03)}, anchor=south west, draw=white!80!black},
log basis x={10},
tick pos=left,
xlabel={epoch (log scale)},
xmajorgrids,
xmin=0.794328234724281, xmax=125.892541179417,
xmode=log,
ylabel={overlap},
ymajorgrids,
ymin=-0.05, ymax=1.05,
zmystyle
]
\addplot [, white!10!black, dashed, forget plot]
table {%
0.794328234724281 1
125.892541179417 1
};
\addplot [, white!10!black, dashed, forget plot]
table {%
0.794328234724281 0
125.892541179417 0
};
\addplot [, black, opacity=0.6, mark=*, mark size=0.5, mark options={solid}, only marks]
table {%
1 0.980172097682953
1.04487179487179 0.991716206073761
1.09615384615385 0.992607712745667
1.1474358974359 0.989956080913544
1.20192307692308 0.990800857543945
1.25961538461538 0.907791912555695
1.32051282051282 0.961326599121094
1.38461538461538 0.88730925321579
1.44871794871795 0.947700977325439
1.51923076923077 0.937219440937042
1.58974358974359 0.949622809886932
1.66666666666667 0.968756198883057
1.74679487179487 0.887838780879974
1.83012820512821 0.916164219379425
1.91666666666667 0.947528004646301
2.00641025641026 0.922192215919495
2.1025641025641 0.928771018981934
2.20512820512821 0.897602498531342
2.30769230769231 0.870130360126495
2.41987179487179 0.878030240535736
2.53525641025641 0.866652131080627
2.65384615384615 0.909153580665588
2.78205128205128 0.937457025051117
2.91346153846154 0.9390869140625
3.05128205128205 0.866176545619965
3.19871794871795 0.88209742307663
3.34935897435897 0.900237679481506
3.50961538461538 0.905459046363831
3.67628205128205 0.9186971783638
3.8525641025641 0.922338128089905
4.03525641025641 0.842959880828857
4.2275641025641 0.896037757396698
4.42948717948718 0.919768989086151
4.64102564102564 0.837957322597504
4.86217948717949 0.866225063800812
5.09294871794872 0.874295353889465
5.33653846153846 0.857604801654816
5.58974358974359 0.839182078838348
5.85576923076923 0.831863701343536
6.13461538461539 0.827021539211273
6.42628205128205 0.851464450359344
6.73397435897436 0.853148102760315
7.05448717948718 0.827787399291992
7.38782051282051 0.85309374332428
7.74038461538461 0.840215861797333
8.10897435897436 0.842476665973663
8.49679487179487 0.846230506896973
8.90064102564103 0.825722634792328
9.32371794871795 0.853772282600403
9.76923076923077 0.857580304145813
10.2339743589744 0.806088864803314
10.7211538461538 0.82514476776123
11.2307692307692 0.816884994506836
11.7660256410256 0.813710987567902
12.3269230769231 0.778618633747101
12.9134615384615 0.794024109840393
13.5288461538462 0.733671605587006
14.1730769230769 0.75925225019455
14.849358974359 0.770133435726166
15.5544871794872 0.770629107952118
16.2948717948718 0.775415062904358
17.0705128205128 0.771032631397247
17.8846153846154 0.748296916484833
18.7371794871795 0.683089137077332
19.6282051282051 0.702462375164032
20.5641025641026 0.673430383205414
21.5416666666667 0.739798545837402
22.5673076923077 0.656615376472473
23.6442307692308 0.712085545063019
24.7692307692308 0.661113440990448
25.9487179487179 0.686876118183136
27.1826923076923 0.673646330833435
28.4775641025641 0.661878705024719
29.8333333333333 0.658248126506805
31.2564102564103 0.601502895355225
32.7435897435897 0.635036945343018
34.3044871794872 0.598907768726349
35.9358974358974 0.649761617183685
37.6474358974359 0.624386310577393
39.4391025641026 0.583890557289124
41.3173076923077 0.602699160575867
43.2852564102564 0.562650620937347
45.3461538461538 0.520900547504425
47.5064102564103 0.571963310241699
49.7692307692308 0.508267223834991
52.1378205128205 0.49919381737709
54.6217948717949 0.480009287595749
57.2211538461538 0.501744508743286
59.9455128205128 0.439036667346954
62.8012820512821 0.425910860300064
65.7916666666667 0.406243652105331
68.9230769230769 0.426879316568375
72.2051282051282 0.436816990375519
75.6442307692308 0.393632560968399
79.2467948717949 0.367149382829666
83.0192307692308 0.35765129327774
86.974358974359 0.372109144926071
91.1153846153846 0.358631610870361
95.4519230769231 0.363734215497971
100 0.373095273971558
};
\addlegendentry{mb 128, exact}
\addplot [, black, opacity=0.6, mark=*, mark size=0.5, mark options={solid}, only marks, forget plot]
table {%
1 0.980207860469818
1.04487179487179 0.986687839031219
1.09615384615385 0.99109822511673
1.1474358974359 0.988957226276398
1.20192307692308 0.989828884601593
1.25961538461538 0.981170475482941
1.32051282051282 0.944415390491486
1.38461538461538 0.97414493560791
1.44871794871795 0.977206408977509
1.51923076923077 0.970506191253662
1.58974358974359 0.97232574224472
1.66666666666667 0.960947215557098
1.74679487179487 0.94966596364975
1.83012820512821 0.947515308856964
1.91666666666667 0.953044831752777
2.00641025641026 0.931983470916748
2.1025641025641 0.938220322132111
2.20512820512821 0.923636555671692
2.30769230769231 0.946711838245392
2.41987179487179 0.919902265071869
2.53525641025641 0.92106819152832
2.65384615384615 0.876817166805267
2.78205128205128 0.888949573040009
2.91346153846154 0.922625064849854
3.05128205128205 0.911627113819122
3.19871794871795 0.863408267498016
3.34935897435897 0.921932399272919
3.50961538461538 0.911691844463348
3.67628205128205 0.869503676891327
3.8525641025641 0.902216255664825
4.03525641025641 0.885700643062592
4.2275641025641 0.849653422832489
4.42948717948718 0.834793746471405
4.64102564102564 0.833202004432678
4.86217948717949 0.89985865354538
5.09294871794872 0.843555629253387
5.33653846153846 0.880906879901886
5.58974358974359 0.910552024841309
5.85576923076923 0.877913773059845
6.13461538461539 0.821998298168182
6.42628205128205 0.865674018859863
6.73397435897436 0.856091499328613
7.05448717948718 0.854511678218842
7.38782051282051 0.867451846599579
7.74038461538461 0.811384618282318
8.10897435897436 0.835431277751923
8.49679487179487 0.874100983142853
8.90064102564103 0.797595322132111
9.32371794871795 0.824461162090302
9.76923076923077 0.82661360502243
10.2339743589744 0.82679957151413
10.7211538461538 0.822871804237366
11.2307692307692 0.773120164871216
11.7660256410256 0.766072332859039
12.3269230769231 0.809707462787628
12.9134615384615 0.796929776668549
13.5288461538462 0.747281968593597
14.1730769230769 0.740639626979828
14.849358974359 0.753473103046417
15.5544871794872 0.73205977678299
16.2948717948718 0.766981542110443
17.0705128205128 0.712304592132568
17.8846153846154 0.791382789611816
18.7371794871795 0.703667104244232
19.6282051282051 0.689324736595154
20.5641025641026 0.644148707389832
21.5416666666667 0.634638488292694
22.5673076923077 0.67344206571579
23.6442307692308 0.653737664222717
24.7692307692308 0.664027214050293
25.9487179487179 0.686160087585449
27.1826923076923 0.632227838039398
28.4775641025641 0.638557136058807
29.8333333333333 0.659567296504974
31.2564102564103 0.581326365470886
32.7435897435897 0.600731611251831
34.3044871794872 0.56149685382843
35.9358974358974 0.593187272548676
37.6474358974359 0.560176849365234
39.4391025641026 0.536633670330048
41.3173076923077 0.543563067913055
43.2852564102564 0.568399846553802
45.3461538461538 0.481604427099228
47.5064102564103 0.538267314434052
49.7692307692308 0.473466485738754
52.1378205128205 0.518011033535004
54.6217948717949 0.448529452085495
57.2211538461538 0.464230448007584
59.9455128205128 0.472301870584488
62.8012820512821 0.444409996271133
65.7916666666667 0.427443742752075
68.9230769230769 0.413966983556747
72.2051282051282 0.409690469503403
75.6442307692308 0.459181159734726
79.2467948717949 0.353957533836365
83.0192307692308 0.398236364126205
86.974358974359 0.31950780749321
91.1153846153846 0.304738879203796
95.4519230769231 0.313452303409576
100 0.317403376102448
};
\addplot [, black, opacity=0.6, mark=*, mark size=0.5, mark options={solid}, only marks, forget plot]
table {%
1 0.983946800231934
1.04487179487179 0.988971710205078
1.09615384615385 0.992758393287659
1.1474358974359 0.990560352802277
1.20192307692308 0.989403367042542
1.25961538461538 0.973469197750092
1.32051282051282 0.899483203887939
1.38461538461538 0.972841441631317
1.44871794871795 0.971960842609406
1.51923076923077 0.968037068843842
1.58974358974359 0.967146873474121
1.66666666666667 0.912840843200684
1.74679487179487 0.959828197956085
1.83012820512821 0.939577400684357
1.91666666666667 0.936375319957733
2.00641025641026 0.937602162361145
2.1025641025641 0.946089446544647
2.20512820512821 0.927560746669769
2.30769230769231 0.925248324871063
2.41987179487179 0.926818192005157
2.53525641025641 0.936004757881165
2.65384615384615 0.900654315948486
2.78205128205128 0.908032417297363
2.91346153846154 0.909767746925354
3.05128205128205 0.936206161975861
3.19871794871795 0.888727605342865
3.34935897435897 0.905566990375519
3.50961538461538 0.903665363788605
3.67628205128205 0.893167674541473
3.8525641025641 0.904410183429718
4.03525641025641 0.899736523628235
4.2275641025641 0.855421245098114
4.42948717948718 0.84848165512085
4.64102564102564 0.849940001964569
4.86217948717949 0.838516414165497
5.09294871794872 0.821752548217773
5.33653846153846 0.874599099159241
5.58974358974359 0.886141002178192
5.85576923076923 0.879460513591766
6.13461538461539 0.883392512798309
6.42628205128205 0.852098405361176
6.73397435897436 0.825584411621094
7.05448717948718 0.805654466152191
7.38782051282051 0.85651171207428
7.74038461538461 0.826787769794464
8.10897435897436 0.813583970069885
8.49679487179487 0.853635966777802
8.90064102564103 0.787913739681244
9.32371794871795 0.808611512184143
9.76923076923077 0.77556037902832
10.2339743589744 0.760887801647186
10.7211538461538 0.785118699073792
11.2307692307692 0.791158854961395
11.7660256410256 0.775297701358795
12.3269230769231 0.764155805110931
12.9134615384615 0.810503602027893
13.5288461538462 0.77249950170517
14.1730769230769 0.756299197673798
14.849358974359 0.724047720432281
15.5544871794872 0.702045857906342
16.2948717948718 0.755908310413361
17.0705128205128 0.690865457057953
17.8846153846154 0.739099204540253
18.7371794871795 0.696106493473053
19.6282051282051 0.656948268413544
20.5641025641026 0.730032682418823
21.5416666666667 0.692141950130463
22.5673076923077 0.69680780172348
23.6442307692308 0.670672595500946
24.7692307692308 0.637449860572815
25.9487179487179 0.644334495067596
27.1826923076923 0.635448336601257
28.4775641025641 0.646008908748627
29.8333333333333 0.639594495296478
31.2564102564103 0.57536393404007
32.7435897435897 0.579100787639618
34.3044871794872 0.598610639572144
35.9358974358974 0.641373157501221
37.6474358974359 0.558124184608459
39.4391025641026 0.518050730228424
41.3173076923077 0.534603893756866
43.2852564102564 0.511987686157227
45.3461538461538 0.50470495223999
47.5064102564103 0.483739674091339
49.7692307692308 0.471539467573166
52.1378205128205 0.392315179109573
54.6217948717949 0.452086359262466
57.2211538461538 0.38785719871521
59.9455128205128 0.432793229818344
62.8012820512821 0.401318460702896
65.7916666666667 0.384503036737442
68.9230769230769 0.244176179170609
72.2051282051282 0.297735422849655
75.6442307692308 0.372241079807281
79.2467948717949 0.284189194440842
83.0192307692308 0.352907985448837
86.974358974359 0.332353800535202
91.1153846153846 0.307710289955139
95.4519230769231 0.33690333366394
100 0.282698422670364
};
\addplot [, black, opacity=0.6, mark=*, mark size=0.5, mark options={solid}, only marks, forget plot]
table {%
1 0.982988655567169
1.04487179487179 0.992447853088379
1.09615384615385 0.993656158447266
1.1474358974359 0.990851044654846
1.20192307692308 0.989253461360931
1.25961538461538 0.947678983211517
1.32051282051282 0.947290420532227
1.38461538461538 0.93720954656601
1.44871794871795 0.948299825191498
1.51923076923077 0.963698089122772
1.58974358974359 0.967891335487366
1.66666666666667 0.898256957530975
1.74679487179487 0.940048694610596
1.83012820512821 0.948034584522247
1.91666666666667 0.927294194698334
2.00641025641026 0.900033593177795
2.1025641025641 0.879429519176483
2.20512820512821 0.871008992195129
2.30769230769231 0.928948998451233
2.41987179487179 0.922504425048828
2.53525641025641 0.897674560546875
2.65384615384615 0.855482757091522
2.78205128205128 0.916048645973206
2.91346153846154 0.892417430877686
3.05128205128205 0.895148456096649
3.19871794871795 0.88008588552475
3.34935897435897 0.883312880992889
3.50961538461538 0.869988441467285
3.67628205128205 0.875664055347443
3.8525641025641 0.895642876625061
4.03525641025641 0.853973388671875
4.2275641025641 0.823393523693085
4.42948717948718 0.86317777633667
4.64102564102564 0.853497803211212
4.86217948717949 0.854592263698578
5.09294871794872 0.832616984844208
5.33653846153846 0.821280896663666
5.58974358974359 0.801458775997162
5.85576923076923 0.821565270423889
6.13461538461539 0.808993637561798
6.42628205128205 0.83081191778183
6.73397435897436 0.827543199062347
7.05448717948718 0.815712153911591
7.38782051282051 0.772370278835297
7.74038461538461 0.788699567317963
8.10897435897436 0.818086802959442
8.49679487179487 0.793797016143799
8.90064102564103 0.739077568054199
9.32371794871795 0.76023668050766
9.76923076923077 0.78579705953598
10.2339743589744 0.788220643997192
10.7211538461538 0.775658845901489
11.2307692307692 0.789422690868378
11.7660256410256 0.799331724643707
12.3269230769231 0.750704288482666
12.9134615384615 0.782992780208588
13.5288461538462 0.727857768535614
14.1730769230769 0.753467559814453
14.849358974359 0.715115964412689
15.5544871794872 0.71807450056076
16.2948717948718 0.719560086727142
17.0705128205128 0.712568938732147
17.8846153846154 0.773623049259186
18.7371794871795 0.725335478782654
19.6282051282051 0.712160050868988
20.5641025641026 0.708537459373474
21.5416666666667 0.71966141462326
22.5673076923077 0.670502722263336
23.6442307692308 0.675801575183868
24.7692307692308 0.722430825233459
25.9487179487179 0.691539704799652
27.1826923076923 0.679106056690216
28.4775641025641 0.644481837749481
29.8333333333333 0.676703870296478
31.2564102564103 0.677815973758698
32.7435897435897 0.662394165992737
34.3044871794872 0.585131764411926
35.9358974358974 0.593388736248016
37.6474358974359 0.601539433002472
39.4391025641026 0.586460292339325
41.3173076923077 0.527765691280365
43.2852564102564 0.563214004039764
45.3461538461538 0.503712356090546
47.5064102564103 0.531544148921967
49.7692307692308 0.531743943691254
52.1378205128205 0.483150869607925
54.6217948717949 0.520426452159882
57.2211538461538 0.445996850728989
59.9455128205128 0.488213866949081
62.8012820512821 0.411276251077652
65.7916666666667 0.443762362003326
68.9230769230769 0.447615712881088
72.2051282051282 0.410937458276749
75.6442307692308 0.370191097259521
79.2467948717949 0.346092462539673
83.0192307692308 0.437259286642075
86.974358974359 0.347940772771835
91.1153846153846 0.33807760477066
95.4519230769231 0.378379315137863
100 0.385885238647461
};
\addplot [, black, opacity=0.6, mark=*, mark size=0.5, mark options={solid}, only marks, forget plot]
table {%
1 0.98374742269516
1.04487179487179 0.99119645357132
1.09615384615385 0.993699729442596
1.1474358974359 0.99211448431015
1.20192307692308 0.990588963031769
1.25961538461538 0.986877083778381
1.32051282051282 0.97865241765976
1.38461538461538 0.972180366516113
1.44871794871795 0.95185375213623
1.51923076923077 0.970642864704132
1.58974358974359 0.96730625629425
1.66666666666667 0.977256953716278
1.74679487179487 0.942020893096924
1.83012820512821 0.953889489173889
1.91666666666667 0.960329711437225
2.00641025641026 0.972863674163818
2.1025641025641 0.954428672790527
2.20512820512821 0.941209018230438
2.30769230769231 0.938596367835999
2.41987179487179 0.945420861244202
2.53525641025641 0.92853319644928
2.65384615384615 0.896694481372833
2.78205128205128 0.949042141437531
2.91346153846154 0.932797908782959
3.05128205128205 0.931452929973602
3.19871794871795 0.939801871776581
3.34935897435897 0.919247806072235
3.50961538461538 0.930974185466766
3.67628205128205 0.929368615150452
3.8525641025641 0.92060798406601
4.03525641025641 0.896500587463379
4.2275641025641 0.896802723407745
4.42948717948718 0.908545315265656
4.64102564102564 0.859995663166046
4.86217948717949 0.881321370601654
5.09294871794872 0.890502572059631
5.33653846153846 0.845744609832764
5.58974358974359 0.849791526794434
5.85576923076923 0.823348045349121
6.13461538461539 0.846448540687561
6.42628205128205 0.854026317596436
6.73397435897436 0.871932685375214
7.05448717948718 0.865126073360443
7.38782051282051 0.870404899120331
7.74038461538461 0.837932050228119
8.10897435897436 0.855006158351898
8.49679487179487 0.83854866027832
8.90064102564103 0.777332901954651
9.32371794871795 0.803152740001678
9.76923076923077 0.85256427526474
10.2339743589744 0.77681827545166
10.7211538461538 0.762219071388245
11.2307692307692 0.827032208442688
11.7660256410256 0.762560188770294
12.3269230769231 0.791698157787323
12.9134615384615 0.779714524745941
13.5288461538462 0.720747232437134
14.1730769230769 0.766886711120605
14.849358974359 0.783985257148743
15.5544871794872 0.681840837001801
16.2948717948718 0.746723890304565
17.0705128205128 0.732045769691467
17.8846153846154 0.719694912433624
18.7371794871795 0.711341559886932
19.6282051282051 0.666857838630676
20.5641025641026 0.634184539318085
21.5416666666667 0.637069165706635
22.5673076923077 0.70036906003952
23.6442307692308 0.615962445735931
24.7692307692308 0.640998721122742
25.9487179487179 0.668058335781097
27.1826923076923 0.653937339782715
28.4775641025641 0.580042839050293
29.8333333333333 0.623236477375031
31.2564102564103 0.546408355236053
32.7435897435897 0.630167126655579
34.3044871794872 0.565100312232971
35.9358974358974 0.522231757640839
37.6474358974359 0.509883403778076
39.4391025641026 0.487043231725693
41.3173076923077 0.491316556930542
43.2852564102564 0.464536666870117
45.3461538461538 0.469104677438736
47.5064102564103 0.46718493103981
49.7692307692308 0.463799208402634
52.1378205128205 0.467945963144302
54.6217948717949 0.437769860029221
57.2211538461538 0.466181248426437
59.9455128205128 0.439740240573883
62.8012820512821 0.377387911081314
65.7916666666667 0.388364851474762
68.9230769230769 0.362410694360733
72.2051282051282 0.351156383752823
75.6442307692308 0.339453905820847
79.2467948717949 0.348567098379135
83.0192307692308 0.412407159805298
86.974358974359 0.329574853181839
91.1153846153846 0.317077040672302
95.4519230769231 0.365579962730408
100 0.345400899648666
};
\addplot [, color0, opacity=0.6, mark=diamond*, mark size=0.5, mark options={solid}, only marks]
table {%
1 0.860992848873138
1.04487179487179 0.892344951629639
1.09615384615385 0.823918163776398
1.1474358974359 0.7441645860672
1.20192307692308 0.801132202148438
1.25961538461538 0.738307893276215
1.32051282051282 0.737305164337158
1.38461538461538 0.725374221801758
1.44871794871795 0.722569942474365
1.51923076923077 0.74552184343338
1.58974358974359 0.746620118618011
1.66666666666667 0.730974614620209
1.74679487179487 0.730906009674072
1.83012820512821 0.710865914821625
1.91666666666667 0.713170826435089
2.00641025641026 0.682346761226654
2.1025641025641 0.661310136318207
2.20512820512821 0.673846244812012
2.30769230769231 0.63198322057724
2.41987179487179 0.653586387634277
2.53525641025641 0.639173626899719
2.65384615384615 0.661026537418365
2.78205128205128 0.621147930622101
2.91346153846154 0.581974685192108
3.05128205128205 0.609411656856537
3.19871794871795 0.637974083423615
3.34935897435897 0.579538762569427
3.50961538461538 0.593585312366486
3.67628205128205 0.597751557826996
3.8525641025641 0.562667667865753
4.03525641025641 0.553296685218811
4.2275641025641 0.549773633480072
4.42948717948718 0.55252343416214
4.64102564102564 0.539897382259369
4.86217948717949 0.516569912433624
5.09294871794872 0.514274299144745
5.33653846153846 0.557376563549042
5.58974358974359 0.485806524753571
5.85576923076923 0.518746972084045
6.13461538461539 0.483038514852524
6.42628205128205 0.512336313724518
6.73397435897436 0.511140167713165
7.05448717948718 0.479569166898727
7.38782051282051 0.479413419961929
7.74038461538461 0.488739967346191
8.10897435897436 0.464967638254166
8.49679487179487 0.489871591329575
8.90064102564103 0.471107870340347
9.32371794871795 0.447679758071899
9.76923076923077 0.435437977313995
10.2339743589744 0.452548325061798
10.7211538461538 0.415754705667496
11.2307692307692 0.413579136133194
11.7660256410256 0.417902559041977
12.3269230769231 0.413538545370102
12.9134615384615 0.398396730422974
13.5288461538462 0.377598613500595
14.1730769230769 0.406299114227295
14.849358974359 0.372663199901581
15.5544871794872 0.363748759031296
16.2948717948718 0.364039808511734
17.0705128205128 0.365611404180527
17.8846153846154 0.345699548721313
18.7371794871795 0.302672922611237
19.6282051282051 0.3581922352314
20.5641025641026 0.336457967758179
21.5416666666667 0.348222225904465
22.5673076923077 0.361511439085007
23.6442307692308 0.312443792819977
24.7692307692308 0.340276718139648
25.9487179487179 0.351905554533005
27.1826923076923 0.323430836200714
28.4775641025641 0.333444774150848
29.8333333333333 0.304251849651337
31.2564102564103 0.347670465707779
32.7435897435897 0.320370823144913
34.3044871794872 0.31298753619194
35.9358974358974 0.313371866941452
37.6474358974359 0.344379812479019
39.4391025641026 0.305190354585648
41.3173076923077 0.339858621358871
43.2852564102564 0.283608019351959
45.3461538461538 0.270850449800491
47.5064102564103 0.331485599279404
49.7692307692308 0.302991956472397
52.1378205128205 0.281433582305908
54.6217948717949 0.315260171890259
57.2211538461538 0.262642502784729
59.9455128205128 0.290677845478058
62.8012820512821 0.313325792551041
65.7916666666667 0.266252160072327
68.9230769230769 0.30434376001358
72.2051282051282 0.229493170976639
75.6442307692308 0.249063447117805
79.2467948717949 0.269422829151154
83.0192307692308 0.310287296772003
86.974358974359 0.246017098426819
91.1153846153846 0.28729447722435
95.4519230769231 0.244613796472549
100 0.265447944402695
};
\addlegendentry{sub 16, exact}
\addplot [, color0, opacity=0.6, mark=diamond*, mark size=0.5, mark options={solid}, only marks, forget plot]
table {%
1 0.860444664955139
1.04487179487179 0.928392887115479
1.09615384615385 0.931609630584717
1.1474358974359 0.915550053119659
1.20192307692308 0.923747658729553
1.25961538461538 0.81793624162674
1.32051282051282 0.789078176021576
1.38461538461538 0.815545260906219
1.44871794871795 0.801478862762451
1.51923076923077 0.736773133277893
1.58974358974359 0.80506831407547
1.66666666666667 0.749329090118408
1.74679487179487 0.767842710018158
1.83012820512821 0.691979050636292
1.91666666666667 0.704763889312744
2.00641025641026 0.679268062114716
2.1025641025641 0.691880881786346
2.20512820512821 0.680962264537811
2.30769230769231 0.646706104278564
2.41987179487179 0.629142642021179
2.53525641025641 0.663328766822815
2.65384615384615 0.614461362361908
2.78205128205128 0.638552904129028
2.91346153846154 0.636721014976501
3.05128205128205 0.624222695827484
3.19871794871795 0.606026291847229
3.34935897435897 0.6509108543396
3.50961538461538 0.593663334846497
3.67628205128205 0.571529567241669
3.8525641025641 0.589094936847687
4.03525641025641 0.571939468383789
4.2275641025641 0.538677930831909
4.42948717948718 0.577403485774994
4.64102564102564 0.487995713949203
4.86217948717949 0.51754355430603
5.09294871794872 0.489810705184937
5.33653846153846 0.491236686706543
5.58974358974359 0.513736546039581
5.85576923076923 0.446437567472458
6.13461538461539 0.478004068136215
6.42628205128205 0.477623760700226
6.73397435897436 0.467316448688507
7.05448717948718 0.502695381641388
7.38782051282051 0.4743971824646
7.74038461538461 0.491034656763077
8.10897435897436 0.446462541818619
8.49679487179487 0.457213789224625
8.90064102564103 0.388940781354904
9.32371794871795 0.442613422870636
9.76923076923077 0.430276721715927
10.2339743589744 0.416165560483932
10.7211538461538 0.407509654760361
11.2307692307692 0.402035623788834
11.7660256410256 0.411157608032227
12.3269230769231 0.395867139101028
12.9134615384615 0.390715450048447
13.5288461538462 0.373276740312576
14.1730769230769 0.381951957941055
14.849358974359 0.407582819461823
15.5544871794872 0.383623600006104
16.2948717948718 0.373151868581772
17.0705128205128 0.374163955450058
17.8846153846154 0.40241551399231
18.7371794871795 0.347761392593384
19.6282051282051 0.304075241088867
20.5641025641026 0.352734059095383
21.5416666666667 0.375276327133179
22.5673076923077 0.392354637384415
23.6442307692308 0.355653643608093
24.7692307692308 0.340053766965866
25.9487179487179 0.312725871801376
27.1826923076923 0.309933662414551
28.4775641025641 0.279562562704086
29.8333333333333 0.306813716888428
31.2564102564103 0.276081442832947
32.7435897435897 0.251543939113617
34.3044871794872 0.289635092020035
35.9358974358974 0.267010658979416
37.6474358974359 0.244917392730713
39.4391025641026 0.239098504185677
41.3173076923077 0.265054553747177
43.2852564102564 0.26198336482048
45.3461538461538 0.233927875757217
47.5064102564103 0.244653224945068
49.7692307692308 0.247437506914139
52.1378205128205 0.253242939710617
54.6217948717949 0.234389111399651
57.2211538461538 0.179333835840225
59.9455128205128 0.215434908866882
62.8012820512821 0.223949149250984
65.7916666666667 0.220515176653862
68.9230769230769 0.229365661740303
72.2051282051282 0.201969012618065
75.6442307692308 0.212255865335464
79.2467948717949 0.224710181355476
83.0192307692308 0.217072889208794
86.974358974359 0.245852038264275
91.1153846153846 0.189456537365913
95.4519230769231 0.211614564061165
100 0.207913503050804
};
\addplot [, color0, opacity=0.6, mark=diamond*, mark size=0.5, mark options={solid}, only marks, forget plot]
table {%
1 0.890892803668976
1.04487179487179 0.938475251197815
1.09615384615385 0.95172780752182
1.1474358974359 0.93952351808548
1.20192307692308 0.938625454902649
1.25961538461538 0.88451874256134
1.32051282051282 0.817189395427704
1.38461538461538 0.863233745098114
1.44871794871795 0.837721049785614
1.51923076923077 0.792255580425262
1.58974358974359 0.797558665275574
1.66666666666667 0.805702209472656
1.74679487179487 0.788794875144958
1.83012820512821 0.752457559108734
1.91666666666667 0.777341246604919
2.00641025641026 0.757799386978149
2.1025641025641 0.718825995922089
2.20512820512821 0.733715057373047
2.30769230769231 0.730856597423553
2.41987179487179 0.689408898353577
2.53525641025641 0.708274304866791
2.65384615384615 0.698091328144073
2.78205128205128 0.700066685676575
2.91346153846154 0.646445035934448
3.05128205128205 0.668342113494873
3.19871794871795 0.617294311523438
3.34935897435897 0.661685228347778
3.50961538461538 0.642695307731628
3.67628205128205 0.62440550327301
3.8525641025641 0.63926762342453
4.03525641025641 0.638611495494843
4.2275641025641 0.617121458053589
4.42948717948718 0.606868386268616
4.64102564102564 0.586089789867401
4.86217948717949 0.612089693546295
5.09294871794872 0.58807498216629
5.33653846153846 0.5828777551651
5.58974358974359 0.575221717357635
5.85576923076923 0.589239537715912
6.13461538461539 0.5655717253685
6.42628205128205 0.521285951137543
6.73397435897436 0.592981994152069
7.05448717948718 0.572408974170685
7.38782051282051 0.549133121967316
7.74038461538461 0.548787772655487
8.10897435897436 0.515187859535217
8.49679487179487 0.572144150733948
8.90064102564103 0.501758754253387
9.32371794871795 0.474455803632736
9.76923076923077 0.497018337249756
10.2339743589744 0.469075977802277
10.7211538461538 0.526267051696777
11.2307692307692 0.506640434265137
11.7660256410256 0.493066042661667
12.3269230769231 0.470579475164413
12.9134615384615 0.515702426433563
13.5288461538462 0.451410204172134
14.1730769230769 0.428697645664215
14.849358974359 0.426507532596588
15.5544871794872 0.372526377439499
16.2948717948718 0.364423036575317
17.0705128205128 0.445634365081787
17.8846153846154 0.423056095838547
18.7371794871795 0.340795248746872
19.6282051282051 0.355673044919968
20.5641025641026 0.332795381546021
21.5416666666667 0.350276231765747
22.5673076923077 0.318451255559921
23.6442307692308 0.299259573221207
24.7692307692308 0.321578323841095
25.9487179487179 0.308305889368057
27.1826923076923 0.297343611717224
28.4775641025641 0.285895824432373
29.8333333333333 0.315602153539658
31.2564102564103 0.292469888925552
32.7435897435897 0.285817533731461
34.3044871794872 0.335625618696213
35.9358974358974 0.257306694984436
37.6474358974359 0.341187298297882
39.4391025641026 0.365381836891174
41.3173076923077 0.270805239677429
43.2852564102564 0.28551658987999
45.3461538461538 0.278515815734863
47.5064102564103 0.268705695867538
49.7692307692308 0.279886484146118
52.1378205128205 0.268429487943649
54.6217948717949 0.292847514152527
57.2211538461538 0.299233168363571
59.9455128205128 0.216808080673218
62.8012820512821 0.261133283376694
65.7916666666667 0.274537444114685
68.9230769230769 0.233089923858643
72.2051282051282 0.225452408194542
75.6442307692308 0.228130534291267
79.2467948717949 0.1875329464674
83.0192307692308 0.252193838357925
86.974358974359 0.218955546617508
91.1153846153846 0.241251900792122
95.4519230769231 0.170218393206596
100 0.220372557640076
};
\addplot [, color0, opacity=0.6, mark=diamond*, mark size=0.5, mark options={solid}, only marks, forget plot]
table {%
1 0.87019681930542
1.04487179487179 0.913722038269043
1.09615384615385 0.941634953022003
1.1474358974359 0.92918872833252
1.20192307692308 0.927006185054779
1.25961538461538 0.823446452617645
1.32051282051282 0.824841320514679
1.38461538461538 0.824603855609894
1.44871794871795 0.819951832294464
1.51923076923077 0.784321963787079
1.58974358974359 0.775236785411835
1.66666666666667 0.76844722032547
1.74679487179487 0.773922026157379
1.83012820512821 0.779536068439484
1.91666666666667 0.786202371120453
2.00641025641026 0.778388142585754
2.1025641025641 0.719281673431396
2.20512820512821 0.691725730895996
2.30769230769231 0.712027370929718
2.41987179487179 0.667527794837952
2.53525641025641 0.692149877548218
2.65384615384615 0.66925185918808
2.78205128205128 0.657330214977264
2.91346153846154 0.631011605262756
3.05128205128205 0.652633965015411
3.19871794871795 0.668539881706238
3.34935897435897 0.637573897838593
3.50961538461538 0.625498354434967
3.67628205128205 0.616036534309387
3.8525641025641 0.630297303199768
4.03525641025641 0.63679975271225
4.2275641025641 0.57340544462204
4.42948717948718 0.593968331813812
4.64102564102564 0.577471017837524
4.86217948717949 0.595125615596771
5.09294871794872 0.583095490932465
5.33653846153846 0.576386094093323
5.58974358974359 0.576989352703094
5.85576923076923 0.468805551528931
6.13461538461539 0.540302455425262
6.42628205128205 0.513910233974457
6.73397435897436 0.541025936603546
7.05448717948718 0.548005402088165
7.38782051282051 0.53130030632019
7.74038461538461 0.5748530626297
8.10897435897436 0.508989453315735
8.49679487179487 0.529834270477295
8.90064102564103 0.509609639644623
9.32371794871795 0.522071719169617
9.76923076923077 0.462595671415329
10.2339743589744 0.479950875043869
10.7211538461538 0.491844952106476
11.2307692307692 0.472049087285995
11.7660256410256 0.457933723926544
12.3269230769231 0.394485533237457
12.9134615384615 0.428000599145889
13.5288461538462 0.430046081542969
14.1730769230769 0.387380212545395
14.849358974359 0.403182804584503
15.5544871794872 0.431074440479279
16.2948717948718 0.412648499011993
17.0705128205128 0.390059471130371
17.8846153846154 0.416656076908112
18.7371794871795 0.385417103767395
19.6282051282051 0.375000089406967
20.5641025641026 0.370560646057129
21.5416666666667 0.382299244403839
22.5673076923077 0.397766441106796
23.6442307692308 0.40088739991188
24.7692307692308 0.358211189508438
25.9487179487179 0.419089883565903
27.1826923076923 0.364112466573715
28.4775641025641 0.353068977594376
29.8333333333333 0.35141795873642
31.2564102564103 0.340419977903366
32.7435897435897 0.333000183105469
34.3044871794872 0.369581431150436
35.9358974358974 0.306281536817551
37.6474358974359 0.278300285339355
39.4391025641026 0.295454651117325
41.3173076923077 0.296001881361008
43.2852564102564 0.280763477087021
45.3461538461538 0.301892399787903
47.5064102564103 0.294060915708542
49.7692307692308 0.299918740987778
52.1378205128205 0.286176532506943
54.6217948717949 0.29368194937706
57.2211538461538 0.276437371969223
59.9455128205128 0.262390345335007
62.8012820512821 0.297131538391113
65.7916666666667 0.256762385368347
68.9230769230769 0.220945030450821
72.2051282051282 0.269754260778427
75.6442307692308 0.276794761419296
79.2467948717949 0.275366395711899
83.0192307692308 0.325062245130539
86.974358974359 0.281228631734848
91.1153846153846 0.264606863260269
95.4519230769231 0.262997061014175
100 0.302665621042252
};
\addplot [, color0, opacity=0.6, mark=diamond*, mark size=0.5, mark options={solid}, only marks, forget plot]
table {%
1 0.889948487281799
1.04487179487179 0.93640673160553
1.09615384615385 0.94278758764267
1.1474358974359 0.924996018409729
1.20192307692308 0.928551197052002
1.25961538461538 0.804385125637054
1.32051282051282 0.781125485897064
1.38461538461538 0.784711062908173
1.44871794871795 0.815053164958954
1.51923076923077 0.750312030315399
1.58974358974359 0.76828521490097
1.66666666666667 0.738130748271942
1.74679487179487 0.759201645851135
1.83012820512821 0.706417858600616
1.91666666666667 0.726467311382294
2.00641025641026 0.70868045091629
2.1025641025641 0.687180161476135
2.20512820512821 0.684219002723694
2.30769230769231 0.67307710647583
2.41987179487179 0.607285141944885
2.53525641025641 0.592164158821106
2.65384615384615 0.608381450176239
2.78205128205128 0.659370422363281
2.91346153846154 0.620877206325531
3.05128205128205 0.609286785125732
3.19871794871795 0.61617386341095
3.34935897435897 0.541272103786469
3.50961538461538 0.575032651424408
3.67628205128205 0.559536039829254
3.8525641025641 0.567671239376068
4.03525641025641 0.52724426984787
4.2275641025641 0.540580749511719
4.42948717948718 0.50359833240509
4.64102564102564 0.466407686471939
4.86217948717949 0.524787068367004
5.09294871794872 0.449942797422409
5.33653846153846 0.438492119312286
5.58974358974359 0.448543637990952
5.85576923076923 0.429678648710251
6.13461538461539 0.420867294073105
6.42628205128205 0.443538278341293
6.73397435897436 0.40187931060791
7.05448717948718 0.442580133676529
7.38782051282051 0.404153645038605
7.74038461538461 0.397739857435226
8.10897435897436 0.366039544343948
8.49679487179487 0.403776556253433
8.90064102564103 0.349451959133148
9.32371794871795 0.368320673704147
9.76923076923077 0.372489392757416
10.2339743589744 0.380041271448135
10.7211538461538 0.335601091384888
11.2307692307692 0.348117619752884
11.7660256410256 0.360536903142929
12.3269230769231 0.337315946817398
12.9134615384615 0.330901563167572
13.5288461538462 0.318796157836914
14.1730769230769 0.317127853631973
14.849358974359 0.32819128036499
15.5544871794872 0.281036913394928
16.2948717948718 0.309944033622742
17.0705128205128 0.303658068180084
17.8846153846154 0.324016183614731
18.7371794871795 0.291385173797607
19.6282051282051 0.272760778665543
20.5641025641026 0.237963438034058
21.5416666666667 0.234408140182495
22.5673076923077 0.283042728900909
23.6442307692308 0.278632551431656
24.7692307692308 0.237166866660118
25.9487179487179 0.289711445569992
27.1826923076923 0.272768378257751
28.4775641025641 0.286693304777145
29.8333333333333 0.257146835327148
31.2564102564103 0.266930729150772
32.7435897435897 0.275360673666
34.3044871794872 0.204768419265747
35.9358974358974 0.235359773039818
37.6474358974359 0.190605014562607
39.4391025641026 0.243993237614632
41.3173076923077 0.236428543925285
43.2852564102564 0.204455927014351
45.3461538461538 0.217014417052269
47.5064102564103 0.21175654232502
49.7692307692308 0.208007797598839
52.1378205128205 0.183520570397377
54.6217948717949 0.213924080133438
57.2211538461538 0.221110582351685
59.9455128205128 0.2117590457201
62.8012820512821 0.189344838261604
65.7916666666667 0.179524168372154
68.9230769230769 0.169765502214432
72.2051282051282 0.201682046055794
75.6442307692308 0.173765867948532
79.2467948717949 0.217665299773216
83.0192307692308 0.22142219543457
86.974358974359 0.187772154808044
91.1153846153846 0.144345209002495
95.4519230769231 0.186378717422485
100 0.184396415948868
};
\addplot [, color1, opacity=0.6, mark=square*, mark size=0.5, mark options={solid}, only marks]
table {%
1 0.831718385219574
1.04487179487179 0.91532689332962
1.09615384615385 0.934202969074249
1.1474358974359 0.924113094806671
1.20192307692308 0.860385537147522
1.25961538461538 0.832666993141174
1.32051282051282 0.872560799121857
1.38461538461538 0.813702881336212
1.44871794871795 0.778506696224213
1.51923076923077 0.820350825786591
1.58974358974359 0.811107099056244
1.66666666666667 0.830908596515656
1.74679487179487 0.828177928924561
1.83012820512821 0.821011245250702
1.91666666666667 0.80619353055954
2.00641025641026 0.741983592510223
2.1025641025641 0.797460556030273
2.20512820512821 0.767449796199799
2.30769230769231 0.743414342403412
2.41987179487179 0.781626343727112
2.53525641025641 0.749984741210938
2.65384615384615 0.761724174022675
2.78205128205128 0.741706311702728
2.91346153846154 0.733337759971619
3.05128205128205 0.72812956571579
3.19871794871795 0.708706378936768
3.34935897435897 0.744928240776062
3.50961538461538 0.748154282569885
3.67628205128205 0.719311058521271
3.8525641025641 0.713043034076691
4.03525641025641 0.657571732997894
4.2275641025641 0.658721148967743
4.42948717948718 0.649153232574463
4.64102564102564 0.680747509002686
4.86217948717949 0.703750550746918
5.09294871794872 0.580794930458069
5.33653846153846 0.671773493289948
5.58974358974359 0.642318844795227
5.85576923076923 0.690593600273132
6.13461538461539 0.619023621082306
6.42628205128205 0.593171119689941
6.73397435897436 0.685479462146759
7.05448717948718 0.588415324687958
7.38782051282051 0.626609086990356
7.74038461538461 0.627931535243988
8.10897435897436 0.655384242534637
8.49679487179487 0.585829138755798
8.90064102564103 0.626469135284424
9.32371794871795 0.598152101039886
9.76923076923077 0.616691589355469
10.2339743589744 0.561780393123627
10.7211538461538 0.578710794448853
11.2307692307692 0.539949774742126
11.7660256410256 0.579552173614502
12.3269230769231 0.545426964759827
12.9134615384615 0.557892441749573
13.5288461538462 0.533496499061584
14.1730769230769 0.531644523143768
14.849358974359 0.510430037975311
15.5544871794872 0.545254349708557
16.2948717948718 0.521359086036682
17.0705128205128 0.505151093006134
17.8846153846154 0.551330506801605
18.7371794871795 0.398896545171738
19.6282051282051 0.499796390533447
20.5641025641026 0.449761211872101
21.5416666666667 0.474911898374557
22.5673076923077 0.459392130374908
23.6442307692308 0.46713000535965
24.7692307692308 0.404979914426804
25.9487179487179 0.459385007619858
27.1826923076923 0.451853483915329
28.4775641025641 0.457371920347214
29.8333333333333 0.425272911787033
31.2564102564103 0.395643532276154
32.7435897435897 0.407275408506393
34.3044871794872 0.379059761762619
35.9358974358974 0.408226251602173
37.6474358974359 0.382270961999893
39.4391025641026 0.322337061166763
41.3173076923077 0.377113163471222
43.2852564102564 0.318112105131149
45.3461538461538 0.322911113500595
47.5064102564103 0.325559467077255
49.7692307692308 0.258837223052979
52.1378205128205 0.323755979537964
54.6217948717949 0.264705806970596
57.2211538461538 0.272887051105499
59.9455128205128 0.282916009426117
62.8012820512821 0.278243690729141
65.7916666666667 0.29192391037941
68.9230769230769 0.280155420303345
72.2051282051282 0.224564895033836
75.6442307692308 0.257812678813934
79.2467948717949 0.242182806134224
83.0192307692308 0.278226792812347
86.974358974359 0.280973523855209
91.1153846153846 0.284133583307266
95.4519230769231 0.269515365362167
100 0.256322622299194
};
\addlegendentry{mb 128, mc 1}
\addplot [, color1, opacity=0.6, mark=square*, mark size=0.5, mark options={solid}, only marks, forget plot]
table {%
1 0.858656346797943
1.04487179487179 0.907752156257629
1.09615384615385 0.929432094097137
1.1474358974359 0.913423001766205
1.20192307692308 0.865697205066681
1.25961538461538 0.816353619098663
1.32051282051282 0.832290947437286
1.38461538461538 0.763093769550323
1.44871794871795 0.821062028408051
1.51923076923077 0.812784373760223
1.58974358974359 0.787710130214691
1.66666666666667 0.837903499603271
1.74679487179487 0.794794917106628
1.83012820512821 0.809977352619171
1.91666666666667 0.807985603809357
2.00641025641026 0.778876721858978
2.1025641025641 0.764621734619141
2.20512820512821 0.756167888641357
2.30769230769231 0.755573213100433
2.41987179487179 0.774213790893555
2.53525641025641 0.72566556930542
2.65384615384615 0.738050878047943
2.78205128205128 0.785201966762543
2.91346153846154 0.780164539813995
3.05128205128205 0.74937915802002
3.19871794871795 0.740462124347687
3.34935897435897 0.759673714637756
3.50961538461538 0.720031023025513
3.67628205128205 0.681498229503632
3.8525641025641 0.795181393623352
4.03525641025641 0.74979555606842
4.2275641025641 0.723177433013916
4.42948717948718 0.66851794719696
4.64102564102564 0.67188024520874
4.86217948717949 0.661979377269745
5.09294871794872 0.688683807849884
5.33653846153846 0.614040791988373
5.58974358974359 0.633781909942627
5.85576923076923 0.639364182949066
6.13461538461539 0.625432193279266
6.42628205128205 0.573632836341858
6.73397435897436 0.611762166023254
7.05448717948718 0.62417459487915
7.38782051282051 0.598569333553314
7.74038461538461 0.594687402248383
8.10897435897436 0.577013194561005
8.49679487179487 0.574857294559479
8.90064102564103 0.573448836803436
9.32371794871795 0.581501483917236
9.76923076923077 0.604185163974762
10.2339743589744 0.578490972518921
10.7211538461538 0.538881719112396
11.2307692307692 0.545246362686157
11.7660256410256 0.499398916959763
12.3269230769231 0.486677408218384
12.9134615384615 0.523888051509857
13.5288461538462 0.487390339374542
14.1730769230769 0.510754525661469
14.849358974359 0.462380319833755
15.5544871794872 0.458656698465347
16.2948717948718 0.502867341041565
17.0705128205128 0.480082273483276
17.8846153846154 0.502073705196381
18.7371794871795 0.424837559461594
19.6282051282051 0.391698509454727
20.5641025641026 0.442054957151413
21.5416666666667 0.400834530591965
22.5673076923077 0.417273253202438
23.6442307692308 0.383448511362076
24.7692307692308 0.345826327800751
25.9487179487179 0.427371174097061
27.1826923076923 0.372452944517136
28.4775641025641 0.371306329965591
29.8333333333333 0.385769486427307
31.2564102564103 0.361891120672226
32.7435897435897 0.362316250801086
34.3044871794872 0.384469091892242
35.9358974358974 0.339487075805664
37.6474358974359 0.352418392896652
39.4391025641026 0.317721992731094
41.3173076923077 0.339120447635651
43.2852564102564 0.250126928091049
45.3461538461538 0.311809539794922
47.5064102564103 0.375011265277863
49.7692307692308 0.263584464788437
52.1378205128205 0.314800947904587
54.6217948717949 0.296242415904999
57.2211538461538 0.24090464413166
59.9455128205128 0.284472644329071
62.8012820512821 0.281670391559601
65.7916666666667 0.294350832700729
68.9230769230769 0.309636890888214
72.2051282051282 0.276949107646942
75.6442307692308 0.261805027723312
79.2467948717949 0.293683022260666
83.0192307692308 0.286657452583313
86.974358974359 0.284863740205765
91.1153846153846 0.267405927181244
95.4519230769231 0.255504190921783
100 0.265985369682312
};
\addplot [, color1, opacity=0.6, mark=square*, mark size=0.5, mark options={solid}, only marks, forget plot]
table {%
1 0.828397452831268
1.04487179487179 0.907137215137482
1.09615384615385 0.931383907794952
1.1474358974359 0.907084107398987
1.20192307692308 0.932597577571869
1.25961538461538 0.832961857318878
1.32051282051282 0.818427681922913
1.38461538461538 0.798737645149231
1.44871794871795 0.834369778633118
1.51923076923077 0.805143654346466
1.58974358974359 0.742678821086884
1.66666666666667 0.785011470317841
1.74679487179487 0.770569443702698
1.83012820512821 0.753225982189178
1.91666666666667 0.749503016471863
2.00641025641026 0.755612075328827
2.1025641025641 0.732974231243134
2.20512820512821 0.750865161418915
2.30769230769231 0.730873346328735
2.41987179487179 0.719896912574768
2.53525641025641 0.742928266525269
2.65384615384615 0.725754082202911
2.78205128205128 0.728431224822998
2.91346153846154 0.749420821666718
3.05128205128205 0.764041244983673
3.19871794871795 0.725082635879517
3.34935897435897 0.73115473985672
3.50961538461538 0.703763246536255
3.67628205128205 0.675777554512024
3.8525641025641 0.732781112194061
4.03525641025641 0.703109920024872
4.2275641025641 0.661949813365936
4.42948717948718 0.636121928691864
4.64102564102564 0.668044209480286
4.86217948717949 0.627338528633118
5.09294871794872 0.628443896770477
5.33653846153846 0.695911824703217
5.58974358974359 0.656872391700745
5.85576923076923 0.66290694475174
6.13461538461539 0.642579317092896
6.42628205128205 0.623920381069183
6.73397435897436 0.636985838413239
7.05448717948718 0.670020818710327
7.38782051282051 0.601765811443329
7.74038461538461 0.621480703353882
8.10897435897436 0.627401769161224
8.49679487179487 0.685251712799072
8.90064102564103 0.603479981422424
9.32371794871795 0.589451968669891
9.76923076923077 0.505423724651337
10.2339743589744 0.585445582866669
10.7211538461538 0.615051209926605
11.2307692307692 0.531665444374084
11.7660256410256 0.560044288635254
12.3269230769231 0.559454083442688
12.9134615384615 0.536922097206116
13.5288461538462 0.585442125797272
14.1730769230769 0.538672566413879
14.849358974359 0.537347137928009
15.5544871794872 0.512918889522552
16.2948717948718 0.503639221191406
17.0705128205128 0.491594612598419
17.8846153846154 0.50579035282135
18.7371794871795 0.446386426687241
19.6282051282051 0.428650110960007
20.5641025641026 0.412720650434494
21.5416666666667 0.487561225891113
22.5673076923077 0.4489386677742
23.6442307692308 0.410760313272476
24.7692307692308 0.439415991306305
25.9487179487179 0.398690909147263
27.1826923076923 0.428749412298203
28.4775641025641 0.436763972043991
29.8333333333333 0.445739030838013
31.2564102564103 0.411928236484528
32.7435897435897 0.393013119697571
34.3044871794872 0.402959018945694
35.9358974358974 0.379105180501938
37.6474358974359 0.386829018592834
39.4391025641026 0.325447529554367
41.3173076923077 0.401644378900528
43.2852564102564 0.293905735015869
45.3461538461538 0.281503587961197
47.5064102564103 0.327936083078384
49.7692307692308 0.319226652383804
52.1378205128205 0.255844175815582
54.6217948717949 0.262244254350662
57.2211538461538 0.278933852910995
59.9455128205128 0.269355744123459
62.8012820512821 0.267186403274536
65.7916666666667 0.329450696706772
68.9230769230769 0.288878917694092
72.2051282051282 0.218272119760513
75.6442307692308 0.240003615617752
79.2467948717949 0.273418992757797
83.0192307692308 0.260252803564072
86.974358974359 0.223961398005486
91.1153846153846 0.27972999215126
95.4519230769231 0.254800707101822
100 0.247597649693489
};
\addplot [, color1, opacity=0.6, mark=square*, mark size=0.5, mark options={solid}, only marks, forget plot]
table {%
1 0.817034423351288
1.04487179487179 0.909705758094788
1.09615384615385 0.928344249725342
1.1474358974359 0.927618026733398
1.20192307692308 0.936348915100098
1.25961538461538 0.893659591674805
1.32051282051282 0.852953732013702
1.38461538461538 0.872898876667023
1.44871794871795 0.792310833930969
1.51923076923077 0.795319199562073
1.58974358974359 0.884269654750824
1.66666666666667 0.869260787963867
1.74679487179487 0.871471405029297
1.83012820512821 0.840520977973938
1.91666666666667 0.855638921260834
2.00641025641026 0.778341591358185
2.1025641025641 0.80114883184433
2.20512820512821 0.772222697734833
2.30769230769231 0.786143124103546
2.41987179487179 0.745745360851288
2.53525641025641 0.785748183727264
2.65384615384615 0.7447469830513
2.78205128205128 0.743236243724823
2.91346153846154 0.741190612316132
3.05128205128205 0.763479053974152
3.19871794871795 0.762459337711334
3.34935897435897 0.737563967704773
3.50961538461538 0.676618993282318
3.67628205128205 0.70228499174118
3.8525641025641 0.7459756731987
4.03525641025641 0.729778110980988
4.2275641025641 0.755058526992798
4.42948717948718 0.687171041965485
4.64102564102564 0.68555873632431
4.86217948717949 0.692193210124969
5.09294871794872 0.668319880962372
5.33653846153846 0.694530844688416
5.58974358974359 0.719569325447083
5.85576923076923 0.603907465934753
6.13461538461539 0.638790309429169
6.42628205128205 0.614583313465118
6.73397435897436 0.583330273628235
7.05448717948718 0.592248618602753
7.38782051282051 0.647828876972198
7.74038461538461 0.646674335002899
8.10897435897436 0.571582555770874
8.49679487179487 0.600022375583649
8.90064102564103 0.612543284893036
9.32371794871795 0.59609979391098
9.76923076923077 0.556222438812256
10.2339743589744 0.653578281402588
10.7211538461538 0.583583772182465
11.2307692307692 0.528942465782166
11.7660256410256 0.532447338104248
12.3269230769231 0.561110496520996
12.9134615384615 0.529454827308655
13.5288461538462 0.513769328594208
14.1730769230769 0.459704875946045
14.849358974359 0.481569766998291
15.5544871794872 0.526020348072052
16.2948717948718 0.477293103933334
17.0705128205128 0.458888620138168
17.8846153846154 0.483338177204132
18.7371794871795 0.487065851688385
19.6282051282051 0.48175048828125
20.5641025641026 0.463883638381958
21.5416666666667 0.49774631857872
22.5673076923077 0.415245026350021
23.6442307692308 0.415585517883301
24.7692307692308 0.392017811536789
25.9487179487179 0.454217731952667
27.1826923076923 0.458766847848892
28.4775641025641 0.367083847522736
29.8333333333333 0.407296001911163
31.2564102564103 0.417179673910141
32.7435897435897 0.362832278013229
34.3044871794872 0.333935111761093
35.9358974358974 0.324279189109802
37.6474358974359 0.309039562940598
39.4391025641026 0.324372977018356
41.3173076923077 0.333293884992599
43.2852564102564 0.28301340341568
45.3461538461538 0.317429780960083
47.5064102564103 0.357645660638809
49.7692307692308 0.297974050045013
52.1378205128205 0.241303682327271
54.6217948717949 0.312122374773026
57.2211538461538 0.282743990421295
59.9455128205128 0.283277601003647
62.8012820512821 0.247865840792656
65.7916666666667 0.26517516374588
68.9230769230769 0.27281329035759
72.2051282051282 0.253391712903976
75.6442307692308 0.252779692411423
79.2467948717949 0.238938376307487
83.0192307692308 0.268889456987381
86.974358974359 0.231116443872452
91.1153846153846 0.221858263015747
95.4519230769231 0.218844279646873
100 0.293239623308182
};
\addplot [, color1, opacity=0.6, mark=square*, mark size=0.5, mark options={solid}, only marks, forget plot]
table {%
1 0.827813267707825
1.04487179487179 0.846263885498047
1.09615384615385 0.925744235515594
1.1474358974359 0.910347163677216
1.20192307692308 0.938778102397919
1.25961538461538 0.797892451286316
1.32051282051282 0.812102913856506
1.38461538461538 0.814592480659485
1.44871794871795 0.860499203205109
1.51923076923077 0.834038555622101
1.58974358974359 0.853531301021576
1.66666666666667 0.834654271602631
1.74679487179487 0.848230183124542
1.83012820512821 0.78334367275238
1.91666666666667 0.790080189704895
2.00641025641026 0.834882915019989
2.1025641025641 0.763410806655884
2.20512820512821 0.815198123455048
2.30769230769231 0.798012912273407
2.41987179487179 0.746797442436218
2.53525641025641 0.769881069660187
2.65384615384615 0.724181592464447
2.78205128205128 0.776508569717407
2.91346153846154 0.761766135692596
3.05128205128205 0.751246869564056
3.19871794871795 0.714191257953644
3.34935897435897 0.731342732906342
3.50961538461538 0.699134528636932
3.67628205128205 0.655514001846313
3.8525641025641 0.738500773906708
4.03525641025641 0.757274568080902
4.2275641025641 0.771870136260986
4.42948717948718 0.742603302001953
4.64102564102564 0.691445350646973
4.86217948717949 0.699981510639191
5.09294871794872 0.729809939861298
5.33653846153846 0.698283731937408
5.58974358974359 0.681201934814453
5.85576923076923 0.688197731971741
6.13461538461539 0.654345154762268
6.42628205128205 0.678068518638611
6.73397435897436 0.631806790828705
7.05448717948718 0.677231311798096
7.38782051282051 0.644726872444153
7.74038461538461 0.618898332118988
8.10897435897436 0.635389745235443
8.49679487179487 0.62433522939682
8.90064102564103 0.682459652423859
9.32371794871795 0.587493538856506
9.76923076923077 0.593628764152527
10.2339743589744 0.592686474323273
10.7211538461538 0.537352859973907
11.2307692307692 0.571306884288788
11.7660256410256 0.602908313274384
12.3269230769231 0.577520549297333
12.9134615384615 0.578780472278595
13.5288461538462 0.541737139225006
14.1730769230769 0.521243035793304
14.849358974359 0.524567246437073
15.5544871794872 0.497612565755844
16.2948717948718 0.485525608062744
17.0705128205128 0.532859444618225
17.8846153846154 0.546761870384216
18.7371794871795 0.478527992963791
19.6282051282051 0.475883334875107
20.5641025641026 0.449024260044098
21.5416666666667 0.452493965625763
22.5673076923077 0.469335079193115
23.6442307692308 0.434750378131866
24.7692307692308 0.439308255910873
25.9487179487179 0.505074977874756
27.1826923076923 0.415506422519684
28.4775641025641 0.434371948242188
29.8333333333333 0.416286915540695
31.2564102564103 0.408123970031738
32.7435897435897 0.404401868581772
34.3044871794872 0.430812984704971
35.9358974358974 0.367878824472427
37.6474358974359 0.37580081820488
39.4391025641026 0.348804384469986
41.3173076923077 0.354967355728149
43.2852564102564 0.390192002058029
45.3461538461538 0.401204496622086
47.5064102564103 0.356531530618668
49.7692307692308 0.292953729629517
52.1378205128205 0.351745277643204
54.6217948717949 0.314463436603546
57.2211538461538 0.338084548711777
59.9455128205128 0.333658665418625
62.8012820512821 0.314126044511795
65.7916666666667 0.329912334680557
68.9230769230769 0.318909138441086
72.2051282051282 0.293680816888809
75.6442307692308 0.327805042266846
79.2467948717949 0.305316567420959
83.0192307692308 0.290589779615402
86.974358974359 0.252898305654526
91.1153846153846 0.297418087720871
95.4519230769231 0.26307338476181
100 0.256137877702713
};
\addplot [, color2, opacity=0.6, mark=triangle*, mark size=0.5, mark options={solid,rotate=180}, only marks]
table {%
1 0.403328388929367
1.04487179487179 0.500678479671478
1.09615384615385 0.531769931316376
1.1474358974359 0.580257713794708
1.20192307692308 0.610610008239746
1.25961538461538 0.491812318563461
1.32051282051282 0.455650806427002
1.38461538461538 0.508278131484985
1.44871794871795 0.49448749423027
1.51923076923077 0.509385526180267
1.58974358974359 0.554943978786469
1.66666666666667 0.475136965513229
1.74679487179487 0.524940133094788
1.83012820512821 0.446522623300552
1.91666666666667 0.507020592689514
2.00641025641026 0.498102575540543
2.1025641025641 0.428548812866211
2.20512820512821 0.45367032289505
2.30769230769231 0.44624862074852
2.41987179487179 0.410327345132828
2.53525641025641 0.493331730365753
2.65384615384615 0.412990570068359
2.78205128205128 0.447286337614059
2.91346153846154 0.448118031024933
3.05128205128205 0.46475550532341
3.19871794871795 0.422908514738083
3.34935897435897 0.45371276140213
3.50961538461538 0.465672463178635
3.67628205128205 0.38970759510994
3.8525641025641 0.408283233642578
4.03525641025641 0.376380890607834
4.2275641025641 0.375652551651001
4.42948717948718 0.378371775150299
4.64102564102564 0.391542166471481
4.86217948717949 0.383591413497925
5.09294871794872 0.353204280138016
5.33653846153846 0.385457992553711
5.58974358974359 0.365813255310059
5.85576923076923 0.34700021147728
6.13461538461539 0.361018687486649
6.42628205128205 0.390840113162994
6.73397435897436 0.335453361272812
7.05448717948718 0.324993103742599
7.38782051282051 0.334690302610397
7.74038461538461 0.325295120477676
8.10897435897436 0.368836730718613
8.49679487179487 0.311197131872177
8.90064102564103 0.390583276748657
9.32371794871795 0.322690486907959
9.76923076923077 0.325783461332321
10.2339743589744 0.348309844732285
10.7211538461538 0.317012161016464
11.2307692307692 0.345407873392105
11.7660256410256 0.323500156402588
12.3269230769231 0.321108788251877
12.9134615384615 0.309399098157883
13.5288461538462 0.300587087869644
14.1730769230769 0.314572960138321
14.849358974359 0.301639646291733
15.5544871794872 0.310845583677292
16.2948717948718 0.331483691930771
17.0705128205128 0.276376515626907
17.8846153846154 0.309700936079025
18.7371794871795 0.32034358382225
19.6282051282051 0.309875458478928
20.5641025641026 0.283959567546844
21.5416666666667 0.300912231206894
22.5673076923077 0.311990708112717
23.6442307692308 0.292921394109726
24.7692307692308 0.298220723867416
25.9487179487179 0.271746724843979
27.1826923076923 0.264651209115982
28.4775641025641 0.288961231708527
29.8333333333333 0.327028185129166
31.2564102564103 0.256611078977585
32.7435897435897 0.283818274736404
34.3044871794872 0.294581741094589
35.9358974358974 0.216543480753899
37.6474358974359 0.272851318120956
39.4391025641026 0.25249195098877
41.3173076923077 0.26741087436676
43.2852564102564 0.230801627039909
45.3461538461538 0.271301597356796
47.5064102564103 0.28218212723732
49.7692307692308 0.251521974802017
52.1378205128205 0.268582940101624
54.6217948717949 0.264646768569946
57.2211538461538 0.287348657846451
59.9455128205128 0.286088019609451
62.8012820512821 0.290025353431702
65.7916666666667 0.31400129199028
68.9230769230769 0.274712771177292
72.2051282051282 0.235672384500504
75.6442307692308 0.268685907125473
79.2467948717949 0.251919090747833
83.0192307692308 0.277584820985794
86.974358974359 0.253834784030914
91.1153846153846 0.217347726225853
95.4519230769231 0.275533050298691
100 0.261717021465302
};
\addlegendentry{sub 16, mc 1}
\addplot [, color2, opacity=0.6, mark=triangle*, mark size=0.5, mark options={solid,rotate=180}, only marks, forget plot]
table {%
1 0.407858848571777
1.04487179487179 0.502846360206604
1.09615384615385 0.58879154920578
1.1474358974359 0.562698841094971
1.20192307692308 0.610564887523651
1.25961538461538 0.606741726398468
1.32051282051282 0.523517429828644
1.38461538461538 0.571849465370178
1.44871794871795 0.525750577449799
1.51923076923077 0.570542454719543
1.58974358974359 0.579596042633057
1.66666666666667 0.533606171607971
1.74679487179487 0.529407441616058
1.83012820512821 0.591767549514771
1.91666666666667 0.566542744636536
2.00641025641026 0.564357101917267
2.1025641025641 0.494915813207626
2.20512820512821 0.519237458705902
2.30769230769231 0.501540124416351
2.41987179487179 0.486162006855011
2.53525641025641 0.548281788825989
2.65384615384615 0.495616108179092
2.78205128205128 0.521672368049622
2.91346153846154 0.487012833356857
3.05128205128205 0.50938481092453
3.19871794871795 0.509578108787537
3.34935897435897 0.485174089670181
3.50961538461538 0.469097435474396
3.67628205128205 0.466290056705475
3.8525641025641 0.450110912322998
4.03525641025641 0.496870517730713
4.2275641025641 0.464010715484619
4.42948717948718 0.444722175598145
4.64102564102564 0.424870729446411
4.86217948717949 0.434516102075577
5.09294871794872 0.360240131616592
5.33653846153846 0.43590459227562
5.58974358974359 0.416973441839218
5.85576923076923 0.418000906705856
6.13461538461539 0.397315680980682
6.42628205128205 0.388605982065201
6.73397435897436 0.384747713804245
7.05448717948718 0.396933168172836
7.38782051282051 0.380245625972748
7.74038461538461 0.316427290439606
8.10897435897436 0.37564691901207
8.49679487179487 0.337792128324509
8.90064102564103 0.376842498779297
9.32371794871795 0.363871157169342
9.76923076923077 0.343962907791138
10.2339743589744 0.407250612974167
10.7211538461538 0.311388432979584
11.2307692307692 0.325923323631287
11.7660256410256 0.336207300424576
12.3269230769231 0.374246150255203
12.9134615384615 0.324291884899139
13.5288461538462 0.341023206710815
14.1730769230769 0.330274105072021
14.849358974359 0.356853902339935
15.5544871794872 0.336841344833374
16.2948717948718 0.349818557500839
17.0705128205128 0.323360830545425
17.8846153846154 0.3030846118927
18.7371794871795 0.340853661298752
19.6282051282051 0.310323685407639
20.5641025641026 0.305444151163101
21.5416666666667 0.326009839773178
22.5673076923077 0.312898606061935
23.6442307692308 0.32131415605545
24.7692307692308 0.301416844129562
25.9487179487179 0.29797551035881
27.1826923076923 0.289734095335007
28.4775641025641 0.269061654806137
29.8333333333333 0.30263552069664
31.2564102564103 0.305902302265167
32.7435897435897 0.296522498130798
34.3044871794872 0.302342742681503
35.9358974358974 0.264042764902115
37.6474358974359 0.236106038093567
39.4391025641026 0.257474452257156
41.3173076923077 0.247835382819176
43.2852564102564 0.276715010404587
45.3461538461538 0.254599392414093
47.5064102564103 0.237850531935692
49.7692307692308 0.243172645568848
52.1378205128205 0.281688302755356
54.6217948717949 0.238183185458183
57.2211538461538 0.230761811137199
59.9455128205128 0.232109233736992
62.8012820512821 0.25855153799057
65.7916666666667 0.213054686784744
68.9230769230769 0.195758521556854
72.2051282051282 0.217609032988548
75.6442307692308 0.217123076319695
79.2467948717949 0.226714059710503
83.0192307692308 0.218353554606438
86.974358974359 0.229764610528946
91.1153846153846 0.190627917647362
95.4519230769231 0.180565983057022
100 0.213041737675667
};
\addplot [, color2, opacity=0.6, mark=triangle*, mark size=0.5, mark options={solid,rotate=180}, only marks, forget plot]
table {%
1 0.452566713094711
1.04487179487179 0.555111587047577
1.09615384615385 0.548881769180298
1.1474358974359 0.602046430110931
1.20192307692308 0.62760728597641
1.25961538461538 0.590451717376709
1.32051282051282 0.566168010234833
1.38461538461538 0.591725409030914
1.44871794871795 0.519594788551331
1.51923076923077 0.490444272756577
1.58974358974359 0.510723352432251
1.66666666666667 0.505447387695312
1.74679487179487 0.553880035877228
1.83012820512821 0.483377367258072
1.91666666666667 0.502935409545898
2.00641025641026 0.477682322263718
2.1025641025641 0.485909849405289
2.20512820512821 0.503089547157288
2.30769230769231 0.417315691709518
2.41987179487179 0.444171905517578
2.53525641025641 0.471772402524948
2.65384615384615 0.450422286987305
2.78205128205128 0.40151110291481
2.91346153846154 0.469295978546143
3.05128205128205 0.41649055480957
3.19871794871795 0.401228278875351
3.34935897435897 0.41104793548584
3.50961538461538 0.436288833618164
3.67628205128205 0.412791818380356
3.8525641025641 0.41339522600174
4.03525641025641 0.397449404001236
4.2275641025641 0.378023236989975
4.42948717948718 0.397342711687088
4.64102564102564 0.387521356344223
4.86217948717949 0.397853910923004
5.09294871794872 0.371277958154678
5.33653846153846 0.341769278049469
5.58974358974359 0.347661733627319
5.85576923076923 0.341797977685928
6.13461538461539 0.348016947507858
6.42628205128205 0.332917064428329
6.73397435897436 0.325384825468063
7.05448717948718 0.322986215353012
7.38782051282051 0.304545789957047
7.74038461538461 0.299807906150818
8.10897435897436 0.338314086198807
8.49679487179487 0.295009762048721
8.90064102564103 0.318485409021378
9.32371794871795 0.321156591176987
9.76923076923077 0.3239766061306
10.2339743589744 0.300059795379639
10.7211538461538 0.318225890398026
11.2307692307692 0.314446806907654
11.7660256410256 0.32953754067421
12.3269230769231 0.308620184659958
12.9134615384615 0.290620476007462
13.5288461538462 0.341040581464767
14.1730769230769 0.318509638309479
14.849358974359 0.293774664402008
15.5544871794872 0.31644532084465
16.2948717948718 0.322812646627426
17.0705128205128 0.323996514081955
17.8846153846154 0.248896673321724
18.7371794871795 0.284689337015152
19.6282051282051 0.266220986843109
20.5641025641026 0.299442619085312
21.5416666666667 0.299358010292053
22.5673076923077 0.311369359493256
23.6442307692308 0.279814869165421
24.7692307692308 0.271275460720062
25.9487179487179 0.258940100669861
27.1826923076923 0.279079526662827
28.4775641025641 0.293164074420929
29.8333333333333 0.279764801263809
31.2564102564103 0.272250711917877
32.7435897435897 0.307057350873947
34.3044871794872 0.257985562086105
35.9358974358974 0.259776264429092
37.6474358974359 0.274165600538254
39.4391025641026 0.29090029001236
41.3173076923077 0.284646093845367
43.2852564102564 0.269764959812164
45.3461538461538 0.273502916097641
47.5064102564103 0.255349904298782
49.7692307692308 0.252426445484161
52.1378205128205 0.271670788526535
54.6217948717949 0.251283258199692
57.2211538461538 0.275805801153183
59.9455128205128 0.246348097920418
62.8012820512821 0.255823075771332
65.7916666666667 0.220743849873543
68.9230769230769 0.223265796899796
72.2051282051282 0.22325225174427
75.6442307692308 0.240950584411621
79.2467948717949 0.22792737185955
83.0192307692308 0.246221944689751
86.974358974359 0.249203354120255
91.1153846153846 0.246047884225845
95.4519230769231 0.23921899497509
100 0.228757500648499
};
\addplot [, color2, opacity=0.6, mark=triangle*, mark size=0.5, mark options={solid,rotate=180}, only marks, forget plot]
table {%
1 0.476896584033966
1.04487179487179 0.573431849479675
1.09615384615385 0.636700332164764
1.1474358974359 0.609522044658661
1.20192307692308 0.64972972869873
1.25961538461538 0.606738030910492
1.32051282051282 0.553682088851929
1.38461538461538 0.580105006694794
1.44871794871795 0.570102632045746
1.51923076923077 0.516376554965973
1.58974358974359 0.510970175266266
1.66666666666667 0.502802550792694
1.74679487179487 0.53408020734787
1.83012820512821 0.467223465442657
1.91666666666667 0.498631209135056
2.00641025641026 0.414588451385498
2.1025641025641 0.464940786361694
2.20512820512821 0.481362104415894
2.30769230769231 0.456477969884872
2.41987179487179 0.503044545650482
2.53525641025641 0.467787325382233
2.65384615384615 0.436957120895386
2.78205128205128 0.465717226266861
2.91346153846154 0.440336138010025
3.05128205128205 0.460307896137238
3.19871794871795 0.385144352912903
3.34935897435897 0.443659693002701
3.50961538461538 0.39060378074646
3.67628205128205 0.448398679494858
3.8525641025641 0.414163589477539
4.03525641025641 0.427004337310791
4.2275641025641 0.396601498126984
4.42948717948718 0.43778458237648
4.64102564102564 0.391023099422455
4.86217948717949 0.393404215574265
5.09294871794872 0.423762142658234
5.33653846153846 0.363830745220184
5.58974358974359 0.388390690088272
5.85576923076923 0.402399629354477
6.13461538461539 0.33602187037468
6.42628205128205 0.390635848045349
6.73397435897436 0.407109260559082
7.05448717948718 0.391096740961075
7.38782051282051 0.374573141336441
7.74038461538461 0.415051311254501
8.10897435897436 0.383230149745941
8.49679487179487 0.358995825052261
8.90064102564103 0.378960222005844
9.32371794871795 0.382364094257355
9.76923076923077 0.364135712385178
10.2339743589744 0.344790995121002
10.7211538461538 0.360283225774765
11.2307692307692 0.394431531429291
11.7660256410256 0.345769733190536
12.3269230769231 0.35903987288475
12.9134615384615 0.380283683538437
13.5288461538462 0.336706936359406
14.1730769230769 0.32210049033165
14.849358974359 0.29973977804184
15.5544871794872 0.337621837854385
16.2948717948718 0.345727771520615
17.0705128205128 0.335712283849716
17.8846153846154 0.322233289480209
18.7371794871795 0.339669078588486
19.6282051282051 0.326494604349136
20.5641025641026 0.323687881231308
21.5416666666667 0.297789484262466
22.5673076923077 0.305610239505768
23.6442307692308 0.316064804792404
24.7692307692308 0.298534125089645
25.9487179487179 0.276820361614227
27.1826923076923 0.297135472297668
28.4775641025641 0.276917517185211
29.8333333333333 0.294411659240723
31.2564102564103 0.288006722927094
32.7435897435897 0.282210916280746
34.3044871794872 0.261523872613907
35.9358974358974 0.267753154039383
37.6474358974359 0.26441764831543
39.4391025641026 0.234028458595276
41.3173076923077 0.275474399328232
43.2852564102564 0.278287082910538
45.3461538461538 0.278962224721909
47.5064102564103 0.283633708953857
49.7692307692308 0.276561349630356
52.1378205128205 0.223923355340958
54.6217948717949 0.234238669276237
57.2211538461538 0.237907275557518
59.9455128205128 0.225008398294449
62.8012820512821 0.229689478874207
65.7916666666667 0.218459561467171
68.9230769230769 0.202610850334167
72.2051282051282 0.236436530947685
75.6442307692308 0.249004647135735
79.2467948717949 0.220484301447868
83.0192307692308 0.235340639948845
86.974358974359 0.24115352332592
91.1153846153846 0.214632660150528
95.4519230769231 0.210192039608955
100 0.286387622356415
};
\addplot [, color2, opacity=0.6, mark=triangle*, mark size=0.5, mark options={solid,rotate=180}, only marks, forget plot]
table {%
1 0.423201769590378
1.04487179487179 0.481868743896484
1.09615384615385 0.591870307922363
1.1474358974359 0.623361527919769
1.20192307692308 0.615780830383301
1.25961538461538 0.604755222797394
1.32051282051282 0.561985433101654
1.38461538461538 0.571284711360931
1.44871794871795 0.600662410259247
1.51923076923077 0.537410914897919
1.58974358974359 0.501823306083679
1.66666666666667 0.565169513225555
1.74679487179487 0.495580047369003
1.83012820512821 0.541629791259766
1.91666666666667 0.543665051460266
2.00641025641026 0.520067453384399
2.1025641025641 0.510983169078827
2.20512820512821 0.477101892232895
2.30769230769231 0.487212270498276
2.41987179487179 0.488206595182419
2.53525641025641 0.493708699941635
2.65384615384615 0.45949187874794
2.78205128205128 0.493779003620148
2.91346153846154 0.457040309906006
3.05128205128205 0.458070248365402
3.19871794871795 0.476939588785172
3.34935897435897 0.468901008367538
3.50961538461538 0.416797399520874
3.67628205128205 0.454002946615219
3.8525641025641 0.434823721647263
4.03525641025641 0.421017080545425
4.2275641025641 0.479255020618439
4.42948717948718 0.434119611978531
4.64102564102564 0.411223500967026
4.86217948717949 0.382410943508148
5.09294871794872 0.392111033201218
5.33653846153846 0.392803162336349
5.58974358974359 0.385303318500519
5.85576923076923 0.407132118940353
6.13461538461539 0.382713794708252
6.42628205128205 0.387118309736252
6.73397435897436 0.437141954898834
7.05448717948718 0.388004630804062
7.38782051282051 0.378326267004013
7.74038461538461 0.378219991922379
8.10897435897436 0.369479656219482
8.49679487179487 0.366816014051437
8.90064102564103 0.330578058958054
9.32371794871795 0.308064609766006
9.76923076923077 0.34880992770195
10.2339743589744 0.342554658651352
10.7211538461538 0.347850680351257
11.2307692307692 0.298551321029663
11.7660256410256 0.329081565141678
12.3269230769231 0.342234939336777
12.9134615384615 0.301361531019211
13.5288461538462 0.367499828338623
14.1730769230769 0.353368163108826
14.849358974359 0.333645015954971
15.5544871794872 0.297293245792389
16.2948717948718 0.304090827703476
17.0705128205128 0.359236419200897
17.8846153846154 0.311598837375641
18.7371794871795 0.302559196949005
19.6282051282051 0.302731931209564
20.5641025641026 0.323415726423264
21.5416666666667 0.282303541898727
22.5673076923077 0.317699640989304
23.6442307692308 0.322174191474915
24.7692307692308 0.30498930811882
25.9487179487179 0.277491241693497
27.1826923076923 0.278138041496277
28.4775641025641 0.279551595449448
29.8333333333333 0.275996893644333
31.2564102564103 0.278350114822388
32.7435897435897 0.287115514278412
34.3044871794872 0.275540202856064
35.9358974358974 0.273132294416428
37.6474358974359 0.252303838729858
39.4391025641026 0.258446902036667
41.3173076923077 0.233400583267212
43.2852564102564 0.291865080595016
45.3461538461538 0.239434525370598
47.5064102564103 0.287410646677017
49.7692307692308 0.282996237277985
52.1378205128205 0.260377079248428
54.6217948717949 0.255564272403717
57.2211538461538 0.251687854528427
59.9455128205128 0.271311551332474
62.8012820512821 0.255502074956894
65.7916666666667 0.241249233484268
68.9230769230769 0.267258614301682
72.2051282051282 0.23657500743866
75.6442307692308 0.244346842169762
79.2467948717949 0.252802848815918
83.0192307692308 0.228952839970589
86.974358974359 0.297624260187149
91.1153846153846 0.227431699633598
95.4519230769231 0.249187469482422
100 0.241004854440689
};
\end{axis}

\end{tikzpicture}

      \tikzexternaldisable
    \end{minipage}
  \end{subfigure}
  \caption{\textbf{\bfvivit{} versus full-batch \ggn (1).} Overlap between the
    top-$C$ eigenspaces of different \ggn approximations and the full-batch \ggn
    during training for all test problems. Each approximation is evaluated on
    $5$ different mini-batches.} \label{vivit::fig:vivit_vs_full_batch_ggn_1}
\end{figure*}

\begin{figure*}[p]
  \centering
  \begin{minipage}[t]{0.495\linewidth}
    \centering
    {\footnotesize Impact of batch size}
  \end{minipage}\hfill
  \begin{minipage}[t]{0.495\linewidth}
    \centering
    {\footnotesize Impact of batch size \& approximations}
  \end{minipage}

  \begin{subfigure}[t]{\linewidth}
    \centering
    \caption{\cifarten \resnetthirtytwo \sgd}
    \begin{minipage}{0.50\linewidth}
      \centering
      % defines the pgfplots style "eigspacedefault"
\pgfkeys{/pgfplots/eigspacedefault/.style={
    width=1.0\linewidth,
    height=0.6\linewidth,
    every axis plot/.append style={line width = 1.5pt},
    tick pos = left,
    ylabel near ticks,
    xlabel near ticks,
    xtick align = inside,
    ytick align = inside,
    legend cell align = left,
    legend columns = 4,
    legend pos = south east,
    legend style = {
      fill opacity = 1,
      text opacity = 1,
      font = \footnotesize,
      at={(1, 1.025)},
      anchor=south east,
      column sep=0.25cm,
    },
    legend image post style={scale=2.5},
    xticklabel style = {font = \footnotesize},
    xlabel style = {font = \footnotesize},
    axis line style = {black},
    yticklabel style = {font = \footnotesize},
    ylabel style = {font = \footnotesize},
    title style = {font = \footnotesize},
    grid = major,
    grid style = {dashed}
  }
}

\pgfkeys{/pgfplots/eigspacedefaultapp/.style={
    eigspacedefault,
    height=0.6\linewidth,
    legend columns = 2,
  }
}

\pgfkeys{/pgfplots/eigspacenolegend/.style={
    legend image post style = {scale=0},
    legend style = {
      fill opacity = 0,
      draw opacity = 0,
      text opacity = 0,
      font = \footnotesize,
      at={(1, 1.025)},
      anchor=south east,
      column sep=0.25cm,
    },
  }
}
%%% Local Variables:
%%% mode: latex
%%% TeX-master: "../../thesis"
%%% End:

      \pgfkeys{/pgfplots/zmystyle/.style={
          eigspacedefaultapp,
        }}
      \tikzexternalenable
      % This file was created by tikzplotlib v0.9.7.
\begin{tikzpicture}

\definecolor{color0}{rgb}{0.501960784313725,0.184313725490196,0.6}
\definecolor{color1}{rgb}{0.870588235294118,0.623529411764706,0.0862745098039216}
\definecolor{color2}{rgb}{0.274509803921569,0.6,0.564705882352941}

\begin{axis}[
axis line style={white!10!black},
legend columns=2,
legend style={fill opacity=0.8, draw opacity=1, text opacity=1, at={(0.97,0.03)}, anchor=south east, draw=white!80!black},
log basis x={10},
tick pos=left,
xlabel={epoch (log scale)},
xmajorgrids,
xmin=0.771323165184619, xmax=233.365219825747,
xmode=log,
ylabel={overlap},
ymajorgrids,
ymin=-0.05, ymax=1.05,
zmystyle
]
\addplot [, white!10!black, dashed, forget plot]
table {%
0.771323165184619 1
233.365219825748 1
};
\addplot [, white!10!black, dashed, forget plot]
table {%
0.771323165184619 0
233.365219825748 0
};
\addplot [, color0, opacity=0.6, mark=triangle*, mark size=0.5, mark options={solid,rotate=180}, only marks]
table {%
1 nan
1.05128205128205 0.578713059425354
1.10897435897436 0.0934028327465057
1.16987179487179 0.329118698835373
1.23076923076923 0.38268119096756
1.29807692307692 0.413004219532013
1.36858974358974 0.400552660226822
1.44230769230769 0.418150722980499
1.51923076923077 0.427569627761841
1.6025641025641 0.426072120666504
1.68910256410256 0.397210329771042
1.77884615384615 0.448423117399216
1.875 0.502146542072296
1.9775641025641 0.457458227872849
2.08333333333333 0.424997538328171
2.19551282051282 0.464976996183395
2.31410256410256 0.458824247121811
2.43910256410256 0.468046396970749
2.57051282051282 0.447967857122421
2.70833333333333 0.579184591770172
2.8525641025641 0.521476447582245
3.00641025641026 0.521829605102539
3.16987179487179 0.54497617483139
3.33974358974359 0.570933997631073
3.51923076923077 0.573860943317413
3.70833333333333 0.548757255077362
3.91025641025641 0.548858821392059
4.11858974358974 0.591545760631561
4.34294871794872 0.591292500495911
4.57692307692308 0.596419632434845
4.82371794871795 0.561238765716553
5.08333333333333 0.546018242835999
5.35576923076923 0.539287567138672
5.64423076923077 0.551958024501801
5.94871794871795 0.516928493976593
6.26923076923077 0.533680558204651
6.60576923076923 0.556977272033691
6.96153846153846 0.52448433637619
7.33653846153846 0.525449097156525
7.73397435897436 0.518015563488007
8.15064102564103 0.523552417755127
8.58974358974359 0.475385099649429
9.05128205128205 0.496413379907608
9.53846153846154 0.503262996673584
10.0512820512821 0.471064180135727
10.5929487179487 0.462177336215973
11.1634615384615 0.453263372182846
11.7660256410256 0.473769158124924
12.400641025641 0.494481950998306
13.0673076923077 0.483082115650177
13.7724358974359 0.468068987131119
14.5128205128205 0.45580068230629
15.2948717948718 0.402402728796005
16.1185897435897 0.432581096887589
16.9871794871795 0.442269414663315
17.900641025641 0.41350993514061
18.8653846153846 0.435066431760788
19.8814102564103 0.395771831274033
20.9519230769231 0.424014180898666
22.0801282051282 0.419073820114136
23.2692307692308 0.415346056222916
24.5224358974359 0.428518921136856
25.8429487179487 0.415148109197617
27.2371794871795 0.414286524057388
28.7019230769231 0.39297690987587
30.25 0.376833736896515
31.8782051282051 0.380725800991058
33.5961538461538 0.363679021596909
35.4038461538462 0.370450884103775
37.3108974358974 0.347859501838684
39.3205128205128 0.345148414373398
41.4391025641026 0.342918395996094
43.6698717948718 0.392895877361298
46.0224358974359 0.368138074874878
48.5 0.404512405395508
51.1121794871795 0.353839844465256
53.8653846153846 0.348737269639969
56.7660256410256 0.354974240064621
59.8237179487179 0.35675898194313
63.0448717948718 0.319194108247757
66.4391025641026 0.345811516046524
70.0192307692308 0.321964025497437
73.7884615384615 0.311834186315536
77.7628205128205 0.353135406970978
81.9519230769231 0.361716121435165
86.3653846153846 0.33873376250267
91.0160256410256 0.314845383167267
95.9166666666667 0.324146598577499
101.083333333333 0.318626046180725
106.525641025641 0.298507392406464
112.262820512821 0.326893299818039
118.310897435897 0.316762238740921
124.682692307692 0.276919454336166
131.397435897436 0.288800716400146
138.474358974359 0.300550431013107
145.929487179487 0.322014629840851
153.788461538462 0.321260511875153
162.070512820513 0.282265990972519
170.801282051282 0.304365873336792
180 0.330555379390717
};
\addlegendentry{mb 2, exact}
\addplot [, color0, opacity=0.6, mark=triangle*, mark size=0.5, mark options={solid,rotate=180}, only marks, forget plot]
table {%
1 nan
1.05128205128205 0.589084565639496
1.10897435897436 0.179002091288567
1.16987179487179 0.435307592153549
1.23076923076923 0.551977574825287
1.29807692307692 0.483564615249634
1.36858974358974 0.458113580942154
1.44230769230769 0.469464123249054
1.51923076923077 0.508115768432617
1.6025641025641 0.481718868017197
1.68910256410256 0.428747564554214
1.77884615384615 0.501877725124359
1.875 0.525018632411957
1.9775641025641 0.451663702726364
2.08333333333333 0.378535985946655
2.19551282051282 0.437135130167007
2.31410256410256 0.413330554962158
2.43910256410256 0.472152233123779
2.57051282051282 0.502956867218018
2.70833333333333 0.504203796386719
2.8525641025641 0.505079925060272
3.00641025641026 0.513301312923431
3.16987179487179 0.54655247926712
3.33974358974359 0.514442801475525
3.51923076923077 0.559629499912262
3.70833333333333 0.490669339895248
3.91025641025641 0.505240917205811
4.11858974358974 0.561023116111755
4.34294871794872 0.522759616374969
4.57692307692308 0.546562850475311
4.82371794871795 0.515803039073944
5.08333333333333 0.511971771717072
5.35576923076923 0.499823570251465
5.64423076923077 0.507944762706757
5.94871794871795 0.486094385385513
6.26923076923077 0.496177583932877
6.60576923076923 0.458743870258331
6.96153846153846 0.467527210712433
7.33653846153846 0.502422451972961
7.73397435897436 0.449869453907013
8.15064102564103 0.432740777730942
8.58974358974359 0.422981113195419
9.05128205128205 0.38680574297905
9.53846153846154 0.417795956134796
10.0512820512821 0.427678406238556
10.5929487179487 0.427213430404663
11.1634615384615 0.433729380369186
11.7660256410256 0.381713688373566
12.400641025641 0.393189996480942
13.0673076923077 0.404461294412613
13.7724358974359 0.360163152217865
14.5128205128205 0.401524364948273
15.2948717948718 0.412565290927887
16.1185897435897 0.386182278394699
16.9871794871795 0.365540027618408
17.900641025641 0.323081344366074
18.8653846153846 0.327490538358688
19.8814102564103 0.403825908899307
20.9519230769231 0.337230980396271
22.0801282051282 0.374308526515961
23.2692307692308 0.379625767469406
24.5224358974359 0.365510880947113
25.8429487179487 0.359859198331833
27.2371794871795 0.344518601894379
28.7019230769231 0.344554632902145
30.25 0.338027328252792
31.8782051282051 0.35016542673111
33.5961538461538 0.33881214261055
35.4038461538462 0.327060550451279
37.3108974358974 0.35591459274292
39.3205128205128 0.331728786230087
41.4391025641026 0.297040551900864
43.6698717948718 0.302970767021179
46.0224358974359 0.300436347723007
48.5 0.294877916574478
51.1121794871795 0.317533165216446
53.8653846153846 0.322497695684433
56.7660256410256 0.280668675899506
59.8237179487179 0.316414028406143
63.0448717948718 0.31861013174057
66.4391025641026 0.266756236553192
70.0192307692308 0.281009644269943
73.7884615384615 0.304712057113647
77.7628205128205 0.292962610721588
81.9519230769231 0.315772294998169
86.3653846153846 0.319815456867218
91.0160256410256 0.307550042867661
95.9166666666667 0.303526222705841
101.083333333333 0.263178944587708
106.525641025641 0.299746185541153
112.262820512821 0.307335287332535
118.310897435897 0.282227665185928
124.682692307692 0.31023246049881
131.397435897436 0.288241684436798
138.474358974359 0.315948188304901
145.929487179487 0.325845062732697
153.788461538462 0.283598989248276
162.070512820513 0.289772421121597
170.801282051282 0.289728850126266
180 0.273426681756973
};
\addplot [, color0, opacity=0.6, mark=triangle*, mark size=0.5, mark options={solid,rotate=180}, only marks, forget plot]
table {%
1 nan
1.05128205128205 0.611252784729004
1.10897435897436 0.47348889708519
1.16987179487179 0.381807386875153
1.23076923076923 0.345988839864731
1.29807692307692 0.353681117296219
1.36858974358974 0.422211855649948
1.44230769230769 0.369066089391708
1.51923076923077 0.363830715417862
1.6025641025641 0.387055605649948
1.68910256410256 0.427210628986359
1.77884615384615 0.3839410841465
1.875 0.307702422142029
1.9775641025641 0.405059635639191
2.08333333333333 0.396745711565018
2.19551282051282 0.386446326971054
2.31410256410256 0.388696938753128
2.43910256410256 0.377914637327194
2.57051282051282 0.394016474485397
2.70833333333333 0.461743116378784
2.8525641025641 0.45064902305603
3.00641025641026 0.485678255558014
3.16987179487179 0.488628476858139
3.33974358974359 0.510310769081116
3.51923076923077 0.497949808835983
3.70833333333333 0.485403627157211
3.91025641025641 0.463590443134308
4.11858974358974 0.456893414258957
4.34294871794872 0.51920610666275
4.57692307692308 0.478825718164444
4.82371794871795 0.492236584424973
5.08333333333333 0.4826680123806
5.35576923076923 0.488750070333481
5.64423076923077 0.485536396503448
5.94871794871795 0.45932149887085
6.26923076923077 0.460102707147598
6.60576923076923 0.487147897481918
6.96153846153846 0.416872322559357
7.33653846153846 0.508146703243256
7.73397435897436 0.42515966296196
8.15064102564103 0.474777936935425
8.58974358974359 0.460694074630737
9.05128205128205 0.424573391675949
9.53846153846154 0.423771619796753
10.0512820512821 0.448190271854401
10.5929487179487 0.421525865793228
11.1634615384615 0.444140762090683
11.7660256410256 0.435377687215805
12.400641025641 0.511880338191986
13.0673076923077 0.524013221263885
13.7724358974359 0.439306020736694
14.5128205128205 0.440678507089615
15.2948717948718 0.423192232847214
16.1185897435897 0.479658424854279
16.9871794871795 0.478051334619522
17.900641025641 0.425073057413101
18.8653846153846 0.422247976064682
19.8814102564103 0.413831382989883
20.9519230769231 0.387429386377335
22.0801282051282 0.404230117797852
23.2692307692308 0.3712238073349
24.5224358974359 0.432516157627106
25.8429487179487 0.380722492933273
27.2371794871795 0.400549381971359
28.7019230769231 0.401839941740036
30.25 0.369582623243332
31.8782051282051 0.380800575017929
33.5961538461538 0.377493470907211
35.4038461538462 0.392291396856308
37.3108974358974 0.389281719923019
39.3205128205128 0.40899658203125
41.4391025641026 0.354437112808228
43.6698717948718 0.41130143404007
46.0224358974359 0.385733634233475
48.5 0.380384176969528
51.1121794871795 0.37381711602211
53.8653846153846 0.393245697021484
56.7660256410256 0.36825355887413
59.8237179487179 0.388748854398727
63.0448717948718 0.369534611701965
66.4391025641026 0.344504654407501
70.0192307692308 0.356461673974991
73.7884615384615 0.367924630641937
77.7628205128205 0.375075817108154
81.9519230769231 0.372633427381516
86.3653846153846 0.365389674901962
91.0160256410256 0.361490190029144
95.9166666666667 0.367129147052765
101.083333333333 0.341300040483475
106.525641025641 0.34124681353569
112.262820512821 0.358804076910019
118.310897435897 0.347850561141968
124.682692307692 0.325235694646835
131.397435897436 0.319385081529617
138.474358974359 0.347527265548706
145.929487179487 0.332268625497818
153.788461538462 0.342211246490479
162.070512820513 0.350776463747025
170.801282051282 0.345442622900009
180 0.349085718393326
};
\addplot [, color0, opacity=0.6, mark=triangle*, mark size=0.5, mark options={solid,rotate=180}, only marks, forget plot]
table {%
1 nan
1.05128205128205 0.52709025144577
1.10897435897436 0.183314666152
1.16987179487179 0.420908272266388
1.23076923076923 0.513801336288452
1.29807692307692 0.481895744800568
1.36858974358974 0.459115028381348
1.44230769230769 0.500301778316498
1.51923076923077 0.507010638713837
1.6025641025641 0.500992059707642
1.68910256410256 0.4635309278965
1.77884615384615 0.495175182819366
1.875 0.56475430727005
1.9775641025641 0.484230041503906
2.08333333333333 0.442044258117676
2.19551282051282 0.479207724332809
2.31410256410256 0.479501634836197
2.43910256410256 0.501877307891846
2.57051282051282 0.45946404337883
2.70833333333333 0.525549113750458
2.8525641025641 0.488492012023926
3.00641025641026 0.480440229177475
3.16987179487179 0.503780364990234
3.33974358974359 0.491904467344284
3.51923076923077 0.542682290077209
3.70833333333333 0.513959050178528
3.91025641025641 0.54792183637619
4.11858974358974 0.519955098628998
4.34294871794872 0.5333451628685
4.57692307692308 0.573999226093292
4.82371794871795 0.573120713233948
5.08333333333333 0.568469822406769
5.35576923076923 0.561022102832794
5.64423076923077 0.529622197151184
5.94871794871795 0.507860124111176
6.26923076923077 0.549948394298553
6.60576923076923 0.548926055431366
6.96153846153846 0.409953266382217
7.33653846153846 0.503727316856384
7.73397435897436 0.487804979085922
8.15064102564103 0.531125605106354
8.58974358974359 0.509819984436035
9.05128205128205 0.482317537069321
9.53846153846154 0.512870490550995
10.0512820512821 0.417242586612701
10.5929487179487 0.485168606042862
11.1634615384615 0.470143377780914
11.7660256410256 0.476603269577026
12.400641025641 0.445653825998306
13.0673076923077 0.428042620420456
13.7724358974359 0.413312822580338
14.5128205128205 0.4316585958004
15.2948717948718 0.426947832107544
16.1185897435897 0.378752082586288
16.9871794871795 0.41690656542778
17.900641025641 0.380893528461456
18.8653846153846 0.330272495746613
19.8814102564103 0.414747059345245
20.9519230769231 0.337731450796127
22.0801282051282 0.353926151990891
23.2692307692308 0.40550622344017
24.5224358974359 0.353589922189713
25.8429487179487 0.330309450626373
27.2371794871795 0.327740997076035
28.7019230769231 0.279643535614014
30.25 0.346427917480469
31.8782051282051 0.316901981830597
33.5961538461538 0.367010861635208
35.4038461538462 0.290425062179565
37.3108974358974 0.323590785264969
39.3205128205128 0.343748390674591
41.4391025641026 0.304665088653564
43.6698717948718 0.295310795307159
46.0224358974359 0.326605409383774
48.5 0.310814082622528
51.1121794871795 0.303437858819962
53.8653846153846 0.321015506982803
56.7660256410256 0.34128212928772
59.8237179487179 0.341635525226593
63.0448717948718 0.325113356113434
66.4391025641026 0.283448755741119
70.0192307692308 0.306793451309204
73.7884615384615 0.275724500417709
77.7628205128205 0.297144174575806
81.9519230769231 0.281010329723358
86.3653846153846 0.291848868131638
91.0160256410256 0.289933919906616
95.9166666666667 0.308462709188461
101.083333333333 0.294346570968628
106.525641025641 0.296774804592133
112.262820512821 0.301042407751083
118.310897435897 0.29477071762085
124.682692307692 0.29257670044899
131.397435897436 0.280056446790695
138.474358974359 0.270216137170792
145.929487179487 0.272956371307373
153.788461538462 0.255688399076462
162.070512820513 0.266073286533356
170.801282051282 0.26547172665596
180 0.268197387456894
};
\addplot [, color0, opacity=0.6, mark=triangle*, mark size=0.5, mark options={solid,rotate=180}, only marks, forget plot]
table {%
1 nan
1.05128205128205 0.556252121925354
1.10897435897436 0.486529409885406
1.16987179487179 0.428912848234177
1.23076923076923 0.505236446857452
1.29807692307692 0.503711700439453
1.36858974358974 0.434563547372818
1.44230769230769 0.450196832418442
1.51923076923077 0.449138849973679
1.6025641025641 0.433089256286621
1.68910256410256 0.386711269617081
1.77884615384615 0.45623779296875
1.875 0.445711761713028
1.9775641025641 0.410528659820557
2.08333333333333 0.383908748626709
2.19551282051282 0.451963901519775
2.31410256410256 0.443912118673325
2.43910256410256 0.481337368488312
2.57051282051282 0.419943660497665
2.70833333333333 0.538575291633606
2.8525641025641 0.497646480798721
3.00641025641026 0.475170135498047
3.16987179487179 0.499726861715317
3.33974358974359 0.528785586357117
3.51923076923077 0.498629659414291
3.70833333333333 0.468739718198776
3.91025641025641 0.534878671169281
4.11858974358974 0.52656614780426
4.34294871794872 0.518789947032928
4.57692307692308 0.511142909526825
4.82371794871795 0.505921244621277
5.08333333333333 0.44672703742981
5.35576923076923 0.480505555868149
5.64423076923077 0.470739513635635
5.94871794871795 0.424978882074356
6.26923076923077 0.447140216827393
6.60576923076923 0.437528103590012
6.96153846153846 0.4213847219944
7.33653846153846 0.43936014175415
7.73397435897436 0.403134107589722
8.15064102564103 0.378988713026047
8.58974358974359 0.416613101959229
9.05128205128205 0.396192133426666
9.53846153846154 0.438064575195312
10.0512820512821 0.382639646530151
10.5929487179487 0.366181433200836
11.1634615384615 0.398280024528503
11.7660256410256 0.379292875528336
12.400641025641 0.4002705514431
13.0673076923077 0.387809902429581
13.7724358974359 0.381953835487366
14.5128205128205 0.364692687988281
15.2948717948718 0.360908776521683
16.1185897435897 0.361045479774475
16.9871794871795 0.391831278800964
17.900641025641 0.317135959863663
18.8653846153846 0.346383333206177
19.8814102564103 0.343174189329147
20.9519230769231 0.328036993741989
22.0801282051282 0.32736748456955
23.2692307692308 0.340071588754654
24.5224358974359 0.335529029369354
25.8429487179487 0.323638021945953
27.2371794871795 0.35327336192131
28.7019230769231 0.302592366933823
30.25 0.306465864181519
31.8782051282051 0.330135822296143
33.5961538461538 0.301556795835495
35.4038461538462 0.293707698583603
37.3108974358974 0.291509836912155
39.3205128205128 0.271966695785522
41.4391025641026 0.28978106379509
43.6698717948718 0.292409360408783
46.0224358974359 0.290118426084518
48.5 0.290832489728928
51.1121794871795 0.273476094007492
53.8653846153846 0.275300472974777
56.7660256410256 0.294930070638657
59.8237179487179 0.285774290561676
63.0448717948718 0.261557698249817
66.4391025641026 0.266904324293137
70.0192307692308 0.271077871322632
73.7884615384615 0.281912326812744
77.7628205128205 0.261884659528732
81.9519230769231 0.263548016548157
86.3653846153846 0.293557852506638
91.0160256410256 0.266898155212402
95.9166666666667 0.288737803697586
101.083333333333 0.246266290545464
106.525641025641 0.254637956619263
112.262820512821 0.25761279463768
118.310897435897 0.271103262901306
124.682692307692 0.255146950483322
131.397435897436 0.243096023797989
138.474358974359 0.241623803973198
145.929487179487 0.256644159555435
153.788461538462 0.259319692850113
162.070512820513 0.245536848902702
170.801282051282 0.255801498889923
180 0.259687513113022
};
\addplot [, color1, opacity=0.6, mark=square*, mark size=0.5, mark options={solid}, only marks]
table {%
1 nan
1.05128205128205 0.759853005409241
1.10897435897436 0.739465355873108
1.16987179487179 0.655319035053253
1.23076923076923 0.689782023429871
1.29807692307692 0.791783511638641
1.36858974358974 0.765750408172607
1.44230769230769 0.734462916851044
1.51923076923077 0.826384007930756
1.6025641025641 0.704811155796051
1.68910256410256 0.707638442516327
1.77884615384615 0.672464489936829
1.875 0.686907947063446
1.9775641025641 0.642514169216156
2.08333333333333 0.659444808959961
2.19551282051282 0.627987563610077
2.31410256410256 0.59365713596344
2.43910256410256 0.631751596927643
2.57051282051282 0.625321865081787
2.70833333333333 0.596621990203857
2.8525641025641 0.575415134429932
3.00641025641026 0.630328357219696
3.16987179487179 0.603658020496368
3.33974358974359 0.607584416866302
3.51923076923077 0.604437649250031
3.70833333333333 0.655057787895203
3.91025641025641 0.663909196853638
4.11858974358974 0.61564439535141
4.34294871794872 0.673425734043121
4.57692307692308 0.637520372867584
4.82371794871795 0.599974632263184
5.08333333333333 0.60954886674881
5.35576923076923 0.60318261384964
5.64423076923077 0.605795085430145
5.94871794871795 0.56388646364212
6.26923076923077 0.553519546985626
6.60576923076923 0.549197137355804
6.96153846153846 0.541161715984344
7.33653846153846 0.565719723701477
7.73397435897436 0.477035194635391
8.15064102564103 0.515192329883575
8.58974358974359 0.484665393829346
9.05128205128205 0.5072922706604
9.53846153846154 0.523773491382599
10.0512820512821 0.510842978954315
10.5929487179487 0.535758554935455
11.1634615384615 0.489917188882828
11.7660256410256 0.460014402866364
12.400641025641 0.449834644794464
13.0673076923077 0.54056453704834
13.7724358974359 0.455823093652725
14.5128205128205 0.527361869812012
15.2948717948718 0.389473348855972
16.1185897435897 0.420601695775986
16.9871794871795 0.454511702060699
17.900641025641 0.442410856485367
18.8653846153846 0.38717794418335
19.8814102564103 0.427576839923859
20.9519230769231 0.410636514425278
22.0801282051282 0.40265616774559
23.2692307692308 0.374734550714493
24.5224358974359 0.395021021366119
25.8429487179487 0.421092003583908
27.2371794871795 0.404414385557175
28.7019230769231 0.36478191614151
30.25 0.34477236866951
31.8782051282051 0.370018005371094
33.5961538461538 0.3438580930233
35.4038461538462 0.325561255216599
37.3108974358974 0.319328844547272
39.3205128205128 0.34404268860817
41.4391025641026 0.332022398710251
43.6698717948718 0.33311516046524
46.0224358974359 0.335747748613358
48.5 0.312749147415161
51.1121794871795 0.375161379575729
53.8653846153846 0.320454508066177
56.7660256410256 0.352085441350937
59.8237179487179 0.313081294298172
63.0448717948718 0.321374267339706
66.4391025641026 0.312866300344467
70.0192307692308 0.335195749998093
73.7884615384615 0.333652019500732
77.7628205128205 0.323419660329819
81.9519230769231 0.318400681018829
86.3653846153846 0.310409039258957
91.0160256410256 0.342231750488281
95.9166666666667 0.312175273895264
101.083333333333 0.324770033359528
106.525641025641 0.325606197118759
112.262820512821 0.322886973619461
118.310897435897 0.323247492313385
124.682692307692 0.317321985960007
131.397435897436 0.276141375303268
138.474358974359 0.286857187747955
145.929487179487 0.298978567123413
153.788461538462 0.290489494800568
162.070512820513 0.347342103719711
170.801282051282 0.28743788599968
180 0.272947996854782
};
\addlegendentry{mb 8, exact}
\addplot [, color1, opacity=0.6, mark=square*, mark size=0.5, mark options={solid}, only marks, forget plot]
table {%
1 nan
1.05128205128205 0.630015730857849
1.10897435897436 0.586316764354706
1.16987179487179 0.69194495677948
1.23076923076923 0.712859749794006
1.29807692307692 0.714089035987854
1.36858974358974 0.650891304016113
1.44230769230769 0.731243252754211
1.51923076923077 0.690023124217987
1.6025641025641 0.683657586574554
1.68910256410256 0.582908272743225
1.77884615384615 0.643375992774963
1.875 0.721822679042816
1.9775641025641 0.624764919281006
2.08333333333333 0.582251965999603
2.19551282051282 0.63753354549408
2.31410256410256 0.632959961891174
2.43910256410256 0.560804665088654
2.57051282051282 0.581508696079254
2.70833333333333 0.648610472679138
2.8525641025641 0.612318515777588
3.00641025641026 0.626333355903625
3.16987179487179 0.627617835998535
3.33974358974359 0.609947025775909
3.51923076923077 0.712938249111176
3.70833333333333 0.594122886657715
3.91025641025641 0.635198652744293
4.11858974358974 0.620364606380463
4.34294871794872 0.558561503887177
4.57692307692308 0.597885012626648
4.82371794871795 0.522135674953461
5.08333333333333 0.559117078781128
5.35576923076923 0.495645493268967
5.64423076923077 0.573888599872589
5.94871794871795 0.511902511119843
6.26923076923077 0.466156393289566
6.60576923076923 0.489395678043365
6.96153846153846 0.522248268127441
7.33653846153846 0.530638813972473
7.73397435897436 0.44711372256279
8.15064102564103 0.495113283395767
8.58974358974359 0.497966766357422
9.05128205128205 0.437306642532349
9.53846153846154 0.430157870054245
10.0512820512821 0.479808002710342
10.5929487179487 0.498164653778076
11.1634615384615 0.425306051969528
11.7660256410256 0.411353826522827
12.400641025641 0.456013679504395
13.0673076923077 0.459009081125259
13.7724358974359 0.409828513860703
14.5128205128205 0.401791572570801
15.2948717948718 0.412300676107407
16.1185897435897 0.384061336517334
16.9871794871795 0.366407841444016
17.900641025641 0.415155649185181
18.8653846153846 0.374263018369675
19.8814102564103 0.3875392973423
20.9519230769231 0.376439839601517
22.0801282051282 0.376433312892914
23.2692307692308 0.312576591968536
24.5224358974359 0.401218235492706
25.8429487179487 0.362413823604584
27.2371794871795 0.342090129852295
28.7019230769231 0.395111680030823
30.25 0.334289461374283
31.8782051282051 0.311215132474899
33.5961538461538 0.312302649021149
35.4038461538462 0.332122772932053
37.3108974358974 0.304359704256058
39.3205128205128 0.338009297847748
41.4391025641026 0.337452590465546
43.6698717948718 0.300906002521515
46.0224358974359 0.287134379148483
48.5 0.297240644693375
51.1121794871795 0.31259760260582
53.8653846153846 0.287588268518448
56.7660256410256 0.3152194917202
59.8237179487179 0.273815333843231
63.0448717948718 0.26623147726059
66.4391025641026 0.27480736374855
70.0192307692308 0.286302715539932
73.7884615384615 0.302410840988159
77.7628205128205 0.280670076608658
81.9519230769231 0.308882266283035
86.3653846153846 0.261998981237411
91.0160256410256 0.22730116546154
95.9166666666667 0.265390604734421
101.083333333333 0.288528740406036
106.525641025641 0.293999373912811
112.262820512821 0.263976186513901
118.310897435897 0.28909307718277
124.682692307692 0.271686643362045
131.397435897436 0.312261790037155
138.474358974359 0.275341212749481
145.929487179487 0.261945515871048
153.788461538462 0.281132638454437
162.070512820513 0.240827396512032
170.801282051282 0.248024389147758
180 0.258766859769821
};
\addplot [, color1, opacity=0.6, mark=square*, mark size=0.5, mark options={solid}, only marks, forget plot]
table {%
1 nan
1.05128205128205 0.72972697019577
1.10897435897436 0.510451138019562
1.16987179487179 0.58172607421875
1.23076923076923 0.650754749774933
1.29807692307692 0.649746716022491
1.36858974358974 0.63249659538269
1.44230769230769 0.635287046432495
1.51923076923077 0.67099791765213
1.6025641025641 0.628484725952148
1.68910256410256 0.609093010425568
1.77884615384615 0.642564594745636
1.875 0.682421982288361
1.9775641025641 0.651307940483093
2.08333333333333 0.602725982666016
2.19551282051282 0.671134412288666
2.31410256410256 0.598797619342804
2.43910256410256 0.612831830978394
2.57051282051282 0.578766465187073
2.70833333333333 0.638838708400726
2.8525641025641 0.575446307659149
3.00641025641026 0.61674177646637
3.16987179487179 0.55733984708786
3.33974358974359 0.575647532939911
3.51923076923077 0.618586957454681
3.70833333333333 0.589278399944305
3.91025641025641 0.662000477313995
4.11858974358974 0.603882491588593
4.34294871794872 0.611694812774658
4.57692307692308 0.59159654378891
4.82371794871795 0.626793026924133
5.08333333333333 0.5798379778862
5.35576923076923 0.60307502746582
5.64423076923077 0.556595742702484
5.94871794871795 0.585905969142914
6.26923076923077 0.637280404567719
6.60576923076923 0.558829545974731
6.96153846153846 0.588434040546417
7.33653846153846 0.525761485099792
7.73397435897436 0.54670661687851
8.15064102564103 0.579963028430939
8.58974358974359 0.503501355648041
9.05128205128205 0.523459732532501
9.53846153846154 0.526332199573517
10.0512820512821 0.468542098999023
10.5929487179487 0.497749418020248
11.1634615384615 0.507600963115692
11.7660256410256 0.4564328789711
12.400641025641 0.457263946533203
13.0673076923077 0.476798266172409
13.7724358974359 0.434778839349747
14.5128205128205 0.480497747659683
15.2948717948718 0.49758443236351
16.1185897435897 0.416402816772461
16.9871794871795 0.431260734796524
17.900641025641 0.455903202295303
18.8653846153846 0.45642963051796
19.8814102564103 0.466833829879761
20.9519230769231 0.440419018268585
22.0801282051282 0.433854788541794
23.2692307692308 0.40736135840416
24.5224358974359 0.399529665708542
25.8429487179487 0.409922689199448
27.2371794871795 0.400940090417862
28.7019230769231 0.408023804426193
30.25 0.409406900405884
31.8782051282051 0.398960262537003
33.5961538461538 0.357180327177048
35.4038461538462 0.338135629892349
37.3108974358974 0.362028896808624
39.3205128205128 0.364762872457504
41.4391025641026 0.355580568313599
43.6698717948718 0.382976144552231
46.0224358974359 0.380188673734665
48.5 0.329656839370728
51.1121794871795 0.389328151941299
53.8653846153846 0.347770422697067
56.7660256410256 0.352100372314453
59.8237179487179 0.357211351394653
63.0448717948718 0.374053388834
66.4391025641026 0.337434142827988
70.0192307692308 0.353070706129074
73.7884615384615 0.369711637496948
77.7628205128205 0.36659848690033
81.9519230769231 0.359281063079834
86.3653846153846 0.382545083761215
91.0160256410256 0.361530929803848
95.9166666666667 0.333948373794556
101.083333333333 0.383487075567245
106.525641025641 0.305918306112289
112.262820512821 0.371825605630875
118.310897435897 0.304268151521683
124.682692307692 0.346261948347092
131.397435897436 0.335128515958786
138.474358974359 0.35671991109848
145.929487179487 0.335923403501511
153.788461538462 0.360773950815201
162.070512820513 0.364931672811508
170.801282051282 0.332949846982956
180 0.351379603147507
};
\addplot [, color1, opacity=0.6, mark=square*, mark size=0.5, mark options={solid}, only marks, forget plot]
table {%
1 nan
1.05128205128205 0.648340344429016
1.10897435897436 0.696191489696503
1.16987179487179 0.616332232952118
1.23076923076923 0.641261041164398
1.29807692307692 0.564119517803192
1.36858974358974 0.681986153125763
1.44230769230769 0.683771252632141
1.51923076923077 0.653343617916107
1.6025641025641 0.625117659568787
1.68910256410256 0.594478130340576
1.77884615384615 0.57131028175354
1.875 0.633172690868378
1.9775641025641 0.597470819950104
2.08333333333333 0.649731338024139
2.19551282051282 0.558295011520386
2.31410256410256 0.590739727020264
2.43910256410256 0.600098609924316
2.57051282051282 0.52477753162384
2.70833333333333 0.573981702327728
2.8525641025641 0.542833030223846
3.00641025641026 0.561576366424561
3.16987179487179 0.570219218730927
3.33974358974359 0.572488188743591
3.51923076923077 0.560167789459229
3.70833333333333 0.515717446804047
3.91025641025641 0.586460947990417
4.11858974358974 0.555770576000214
4.34294871794872 0.57284677028656
4.57692307692308 0.555584967136383
4.82371794871795 0.553139507770538
5.08333333333333 0.535487532615662
5.35576923076923 0.477800101041794
5.64423076923077 0.588516533374786
5.94871794871795 0.549264907836914
6.26923076923077 0.476074039936066
6.60576923076923 0.584322869777679
6.96153846153846 0.560511946678162
7.33653846153846 0.538513362407684
7.73397435897436 0.531952083110809
8.15064102564103 0.56194657087326
8.58974358974359 0.534368455410004
9.05128205128205 0.504019677639008
9.53846153846154 0.465150892734528
10.0512820512821 0.501082539558411
10.5929487179487 0.492814630270004
11.1634615384615 0.4963039457798
11.7660256410256 0.506355941295624
12.400641025641 0.475838959217072
13.0673076923077 0.46166855096817
13.7724358974359 0.487527281045914
14.5128205128205 0.483817964792252
15.2948717948718 0.469904333353043
16.1185897435897 0.477266311645508
16.9871794871795 0.4444260597229
17.900641025641 0.451462954282761
18.8653846153846 0.449789822101593
19.8814102564103 0.465183734893799
20.9519230769231 0.434892565011978
22.0801282051282 0.433360904455185
23.2692307692308 0.401677697896957
24.5224358974359 0.417658239603043
25.8429487179487 0.381187915802002
27.2371794871795 0.39768922328949
28.7019230769231 0.385645717382431
30.25 0.374193966388702
31.8782051282051 0.355787813663483
33.5961538461538 0.365121692419052
35.4038461538462 0.360952913761139
37.3108974358974 0.369683563709259
39.3205128205128 0.384858042001724
41.4391025641026 0.346937417984009
43.6698717948718 0.349095106124878
46.0224358974359 0.37097504734993
48.5 0.349438399076462
51.1121794871795 0.405700981616974
53.8653846153846 0.351124346256256
56.7660256410256 0.360295534133911
59.8237179487179 0.368119537830353
63.0448717948718 0.325333923101425
66.4391025641026 0.326354950666428
70.0192307692308 0.347087204456329
73.7884615384615 0.330307841300964
77.7628205128205 0.318880498409271
81.9519230769231 0.334263622760773
86.3653846153846 0.333771407604218
91.0160256410256 0.306611776351929
95.9166666666667 0.327362686395645
101.083333333333 0.336138695478439
106.525641025641 0.329414755105972
112.262820512821 0.307609707117081
118.310897435897 0.372311443090439
124.682692307692 0.308635979890823
131.397435897436 0.294477492570877
138.474358974359 0.319994181394577
145.929487179487 0.340276926755905
153.788461538462 0.34143602848053
162.070512820513 0.335416465997696
170.801282051282 0.301829487085342
180 0.340263336896896
};
\addplot [, color1, opacity=0.6, mark=square*, mark size=0.5, mark options={solid}, only marks, forget plot]
table {%
1 0.317827463150024
1.05128205128205 0.687550663948059
1.10897435897436 0.319952785968781
1.16987179487179 0.552493095397949
1.23076923076923 0.651574790477753
1.29807692307692 0.611338317394257
1.36858974358974 0.629461288452148
1.44230769230769 0.632675647735596
1.51923076923077 0.692654609680176
1.6025641025641 0.621993839740753
1.68910256410256 0.621526300907135
1.77884615384615 0.644492268562317
1.875 0.687174916267395
1.9775641025641 0.622430324554443
2.08333333333333 0.586113929748535
2.19551282051282 0.584549427032471
2.31410256410256 0.600969910621643
2.43910256410256 0.576325356960297
2.57051282051282 0.573913753032684
2.70833333333333 0.569676518440247
2.8525641025641 0.61366081237793
3.00641025641026 0.56893515586853
3.16987179487179 0.557516098022461
3.33974358974359 0.588577091693878
3.51923076923077 0.612085521221161
3.70833333333333 0.519961655139923
3.91025641025641 0.559741020202637
4.11858974358974 0.592296957969666
4.34294871794872 0.542282342910767
4.57692307692308 0.688184916973114
4.82371794871795 0.654534518718719
5.08333333333333 0.606560707092285
5.35576923076923 0.630791127681732
5.64423076923077 0.658446848392487
5.94871794871795 0.588958203792572
6.26923076923077 0.561368107795715
6.60576923076923 0.586937785148621
6.96153846153846 0.539696574211121
7.33653846153846 0.537503838539124
7.73397435897436 0.554575622081757
8.15064102564103 0.587106943130493
8.58974358974359 0.541067779064178
9.05128205128205 0.421837151050568
9.53846153846154 0.547208249568939
10.0512820512821 0.524263918399811
10.5929487179487 0.408860683441162
11.1634615384615 0.460636347532272
11.7660256410256 0.419872015714645
12.400641025641 0.444108307361603
13.0673076923077 0.454688757658005
13.7724358974359 0.457754522562027
14.5128205128205 0.44317951798439
15.2948717948718 0.405481725931168
16.1185897435897 0.472763061523438
16.9871794871795 0.453780084848404
17.900641025641 0.395538985729218
18.8653846153846 0.471596240997314
19.8814102564103 0.404871791601181
20.9519230769231 0.402434915304184
22.0801282051282 0.43663477897644
23.2692307692308 0.443407624959946
24.5224358974359 0.425425052642822
25.8429487179487 0.391078948974609
27.2371794871795 0.372082442045212
28.7019230769231 0.35768985748291
30.25 0.337647974491119
31.8782051282051 0.378209680318832
33.5961538461538 0.346204966306686
35.4038461538462 0.361796945333481
37.3108974358974 0.381218999624252
39.3205128205128 0.341154634952545
41.4391025641026 0.379549264907837
43.6698717948718 0.325090557336807
46.0224358974359 0.34041890501976
48.5 0.311956435441971
51.1121794871795 0.32648241519928
53.8653846153846 0.352833807468414
56.7660256410256 0.300140142440796
59.8237179487179 0.300133913755417
63.0448717948718 0.298030227422714
66.4391025641026 0.317218452692032
70.0192307692308 0.252334326505661
73.7884615384615 0.301236301660538
77.7628205128205 0.328152507543564
81.9519230769231 0.318715900182724
86.3653846153846 0.310745149850845
91.0160256410256 0.278330057859421
95.9166666666667 0.332725375890732
101.083333333333 0.312249302864075
106.525641025641 0.297983318567276
112.262820512821 0.318392097949982
118.310897435897 0.301361173391342
124.682692307692 0.276366323232651
131.397435897436 0.284807026386261
138.474358974359 0.305565536022186
145.929487179487 0.286950558423996
153.788461538462 0.301304906606674
162.070512820513 0.289768785238266
170.801282051282 0.285021871328354
180 0.279938280582428
};
\addplot [, color2, opacity=0.6, mark=diamond*, mark size=0.5, mark options={solid}, only marks]
table {%
1 0.11981824785471
1.05128205128205 0.837295234203339
1.10897435897436 0.74955028295517
1.16987179487179 0.679739773273468
1.23076923076923 0.70360404253006
1.29807692307692 0.758731842041016
1.36858974358974 0.761583983898163
1.44230769230769 0.767446637153625
1.51923076923077 0.746459901332855
1.6025641025641 0.771988928318024
1.68910256410256 0.741749703884125
1.77884615384615 0.634516060352325
1.875 0.773816108703613
1.9775641025641 0.725350499153137
2.08333333333333 0.713078022003174
2.19551282051282 0.788644134998322
2.31410256410256 0.677223384380341
2.43910256410256 0.674014449119568
2.57051282051282 0.695789456367493
2.70833333333333 0.672382473945618
2.8525641025641 0.717797100543976
3.00641025641026 0.777785003185272
3.16987179487179 0.695268332958221
3.33974358974359 0.726305603981018
3.51923076923077 0.727250516414642
3.70833333333333 0.755482017993927
3.91025641025641 0.745897948741913
4.11858974358974 0.717685341835022
4.34294871794872 0.730643153190613
4.57692307692308 0.692432761192322
4.82371794871795 0.722373127937317
5.08333333333333 0.670852780342102
5.35576923076923 0.735814154148102
5.64423076923077 0.759610533714294
5.94871794871795 0.675690770149231
6.26923076923077 0.700427711009979
6.60576923076923 0.714502930641174
6.96153846153846 0.692158937454224
7.33653846153846 0.603394031524658
7.73397435897436 0.691804826259613
8.15064102564103 0.696665346622467
8.58974358974359 0.646570086479187
9.05128205128205 0.628552377223969
9.53846153846154 0.672821998596191
10.0512820512821 0.624726712703705
10.5929487179487 0.654639899730682
11.1634615384615 0.639744937419891
11.7660256410256 0.646105170249939
12.400641025641 0.597169876098633
13.0673076923077 0.604124963283539
13.7724358974359 0.63664847612381
14.5128205128205 0.556724190711975
15.2948717948718 0.585099995136261
16.1185897435897 0.582469403743744
16.9871794871795 0.56008118391037
17.900641025641 0.56195342540741
18.8653846153846 0.548291921615601
19.8814102564103 0.499452888965607
20.9519230769231 0.473575592041016
22.0801282051282 0.513980448246002
23.2692307692308 0.499806642532349
24.5224358974359 0.416852205991745
25.8429487179487 0.439953535795212
27.2371794871795 0.499034404754639
28.7019230769231 0.465262949466705
30.25 0.424325555562973
31.8782051282051 0.435800164937973
33.5961538461538 0.390791922807693
35.4038461538462 0.414607375860214
37.3108974358974 0.39752322435379
39.3205128205128 0.423370361328125
41.4391025641026 0.399825692176819
43.6698717948718 0.368824928998947
46.0224358974359 0.409465044736862
48.5 0.357581079006195
51.1121794871795 0.372737884521484
53.8653846153846 0.364838033914566
56.7660256410256 0.34876748919487
59.8237179487179 0.34413868188858
63.0448717948718 0.334741920232773
66.4391025641026 0.299217939376831
70.0192307692308 0.367687553167343
73.7884615384615 0.353821158409119
77.7628205128205 0.284915655851364
81.9519230769231 0.317193239927292
86.3653846153846 0.357921808958054
91.0160256410256 0.366287231445312
95.9166666666667 0.316520184278488
101.083333333333 0.30722114443779
106.525641025641 0.298834651708603
112.262820512821 0.332701027393341
118.310897435897 0.33405789732933
124.682692307692 0.275922983884811
131.397435897436 0.328634679317474
138.474358974359 0.320749461650848
145.929487179487 0.341055780649185
153.788461538462 0.324311941862106
162.070512820513 0.292864173650742
170.801282051282 0.299744695425034
180 0.346527129411697
};
\addlegendentry{mb 32, exact}
\addplot [, color2, opacity=0.6, mark=diamond*, mark size=0.5, mark options={solid}, only marks, forget plot]
table {%
1 0.196364372968674
1.05128205128205 0.778618097305298
1.10897435897436 0.767292678356171
1.16987179487179 0.929760575294495
1.23076923076923 0.924710214138031
1.29807692307692 0.866626918315887
1.36858974358974 0.923155128955841
1.44230769230769 0.859532356262207
1.51923076923077 0.864682018756866
1.6025641025641 0.813648879528046
1.68910256410256 0.822426736354828
1.77884615384615 0.772900521755219
1.875 0.899652481079102
1.9775641025641 0.819776475429535
2.08333333333333 0.824316799640656
2.19551282051282 0.803861618041992
2.31410256410256 0.802933692932129
2.43910256410256 0.866603970527649
2.57051282051282 0.762768983840942
2.70833333333333 0.74802577495575
2.8525641025641 0.710406482219696
3.00641025641026 0.747665584087372
3.16987179487179 0.736758649349213
3.33974358974359 0.727915227413177
3.51923076923077 0.766108870506287
3.70833333333333 0.761342823505402
3.91025641025641 0.779019773006439
4.11858974358974 0.795238852500916
4.34294871794872 0.739213645458221
4.57692307692308 0.753251969814301
4.82371794871795 0.752416729927063
5.08333333333333 0.798942506313324
5.35576923076923 0.799425721168518
5.64423076923077 0.75725519657135
5.94871794871795 0.753632843494415
6.26923076923077 0.747090995311737
6.60576923076923 0.763355135917664
6.96153846153846 0.746002018451691
7.33653846153846 0.676833152770996
7.73397435897436 0.742178618907928
8.15064102564103 0.703369498252869
8.58974358974359 0.697418868541718
9.05128205128205 0.655690133571625
9.53846153846154 0.661413192749023
10.0512820512821 0.641086757183075
10.5929487179487 0.56852787733078
11.1634615384615 0.580054461956024
11.7660256410256 0.650184154510498
12.400641025641 0.602141320705414
13.0673076923077 0.659935772418976
13.7724358974359 0.624140202999115
14.5128205128205 0.591449439525604
15.2948717948718 0.46243879199028
16.1185897435897 0.610901236534119
16.9871794871795 0.573297202587128
17.900641025641 0.555977880954742
18.8653846153846 0.600026488304138
19.8814102564103 0.556471645832062
20.9519230769231 0.544497966766357
22.0801282051282 0.47572460770607
23.2692307692308 0.543266236782074
24.5224358974359 0.479374885559082
25.8429487179487 0.407392591238022
27.2371794871795 0.426048964262009
28.7019230769231 0.409218847751617
30.25 0.396581947803497
31.8782051282051 0.43340665102005
33.5961538461538 0.394775867462158
35.4038461538462 0.382427603006363
37.3108974358974 0.359813034534454
39.3205128205128 0.441294401884079
41.4391025641026 0.338774681091309
43.6698717948718 0.40898123383522
46.0224358974359 0.395606368780136
48.5 0.361879736185074
51.1121794871795 0.391783058643341
53.8653846153846 0.354710251092911
56.7660256410256 0.350573986768723
59.8237179487179 0.368046909570694
63.0448717948718 0.331572264432907
66.4391025641026 0.343005567789078
70.0192307692308 0.325284481048584
73.7884615384615 0.327899545431137
77.7628205128205 0.346173375844955
81.9519230769231 0.359017491340637
86.3653846153846 0.333273261785507
91.0160256410256 0.343561500310898
95.9166666666667 0.345331728458405
101.083333333333 0.344956159591675
106.525641025641 0.347210139036179
112.262820512821 0.313642501831055
118.310897435897 0.339276939630508
124.682692307692 0.317081362009048
131.397435897436 0.290175408124924
138.474358974359 0.323442965745926
145.929487179487 0.289411276578903
153.788461538462 0.315129101276398
162.070512820513 0.276556491851807
170.801282051282 0.332854032516479
180 0.290457040071487
};
\addplot [, color2, opacity=0.6, mark=diamond*, mark size=0.5, mark options={solid}, only marks, forget plot]
table {%
1 0.28116312623024
1.05128205128205 0.899288356304169
1.10897435897436 0.803821086883545
1.16987179487179 0.916050851345062
1.23076923076923 0.896565735340118
1.29807692307692 0.889821469783783
1.36858974358974 0.869400441646576
1.44230769230769 0.857334733009338
1.51923076923077 0.834472835063934
1.6025641025641 0.811902642250061
1.68910256410256 0.904411137104034
1.77884615384615 0.787602424621582
1.875 0.812097489833832
1.9775641025641 0.788743913173676
2.08333333333333 0.804455935955048
2.19551282051282 0.861001431941986
2.31410256410256 0.768641412258148
2.43910256410256 0.724602162837982
2.57051282051282 0.762129783630371
2.70833333333333 0.720209896564484
2.8525641025641 0.701987385749817
3.00641025641026 0.774101376533508
3.16987179487179 0.838882625102997
3.33974358974359 0.794073104858398
3.51923076923077 0.832184970378876
3.70833333333333 0.755195617675781
3.91025641025641 0.795858025550842
4.11858974358974 0.788565814495087
4.34294871794872 0.789804637432098
4.57692307692308 0.799069225788116
4.82371794871795 0.827780544757843
5.08333333333333 0.779014885425568
5.35576923076923 0.759860873222351
5.64423076923077 0.726893723011017
5.94871794871795 0.74411529302597
6.26923076923077 0.777914762496948
6.60576923076923 0.769153594970703
6.96153846153846 0.757950007915497
7.33653846153846 0.71914404630661
7.73397435897436 0.643153786659241
8.15064102564103 0.70281857252121
8.58974358974359 0.658607423305511
9.05128205128205 0.701694130897522
9.53846153846154 0.638822674751282
10.0512820512821 0.597369313240051
10.5929487179487 0.638498961925507
11.1634615384615 0.661192953586578
11.7660256410256 0.615286767482758
12.400641025641 0.615429520606995
13.0673076923077 0.594968199729919
13.7724358974359 0.570953547954559
14.5128205128205 0.59306389093399
15.2948717948718 0.535050213336945
16.1185897435897 0.584794163703918
16.9871794871795 0.519201576709747
17.900641025641 0.564422786235809
18.8653846153846 0.520045220851898
19.8814102564103 0.516805171966553
20.9519230769231 0.497715711593628
22.0801282051282 0.546065866947174
23.2692307692308 0.494046032428741
24.5224358974359 0.551328659057617
25.8429487179487 0.415474325418472
27.2371794871795 0.424788802862167
28.7019230769231 0.484315395355225
30.25 0.46904793381691
31.8782051282051 0.426814079284668
33.5961538461538 0.424085140228271
35.4038461538462 0.41370764374733
37.3108974358974 0.357081174850464
39.3205128205128 0.361755073070526
41.4391025641026 0.410971254110336
43.6698717948718 0.413096904754639
46.0224358974359 0.344008147716522
48.5 0.358992397785187
51.1121794871795 0.340972989797592
53.8653846153846 0.348779618740082
56.7660256410256 0.353572368621826
59.8237179487179 0.3509820997715
63.0448717948718 0.362792789936066
66.4391025641026 0.341424286365509
70.0192307692308 0.359184712171555
73.7884615384615 0.347322881221771
77.7628205128205 0.332731306552887
81.9519230769231 0.293603122234344
86.3653846153846 0.373455017805099
91.0160256410256 0.29749384522438
95.9166666666667 0.319384574890137
101.083333333333 0.364244997501373
106.525641025641 0.294043928384781
112.262820512821 0.344189316034317
118.310897435897 0.319366931915283
124.682692307692 0.323834002017975
131.397435897436 0.301629394292831
138.474358974359 0.320218533277512
145.929487179487 0.317127704620361
153.788461538462 0.294097661972046
162.070512820513 0.312924951314926
170.801282051282 0.311868965625763
180 0.308166235685349
};
\addplot [, color2, opacity=0.6, mark=diamond*, mark size=0.5, mark options={solid}, only marks, forget plot]
table {%
1 0.121515668928623
1.05128205128205 0.845720291137695
1.10897435897436 0.829960644245148
1.16987179487179 0.84165495634079
1.23076923076923 0.917669713497162
1.29807692307692 0.86567634344101
1.36858974358974 0.847967743873596
1.44230769230769 0.875771939754486
1.51923076923077 0.892412304878235
1.6025641025641 0.854964673519135
1.68910256410256 0.912895202636719
1.77884615384615 0.855627477169037
1.875 0.877321064472198
1.9775641025641 0.831666171550751
2.08333333333333 0.826602578163147
2.19551282051282 0.867336213588715
2.31410256410256 0.770126760005951
2.43910256410256 0.778379917144775
2.57051282051282 0.810145974159241
2.70833333333333 0.744114100933075
2.8525641025641 0.714560449123383
3.00641025641026 0.780830860137939
3.16987179487179 0.785825908184052
3.33974358974359 0.76714289188385
3.51923076923077 0.760190486907959
3.70833333333333 0.766688823699951
3.91025641025641 0.78530091047287
4.11858974358974 0.762439668178558
4.34294871794872 0.739381432533264
4.57692307692308 0.762027442455292
4.82371794871795 0.788071930408478
5.08333333333333 0.766415119171143
5.35576923076923 0.78395813703537
5.64423076923077 0.798341274261475
5.94871794871795 0.695757806301117
6.26923076923077 0.676800906658173
6.60576923076923 0.724539577960968
6.96153846153846 0.75559538602829
7.33653846153846 0.745241224765778
7.73397435897436 0.687600791454315
8.15064102564103 0.654770374298096
8.58974358974359 0.70273345708847
9.05128205128205 0.646819353103638
9.53846153846154 0.656380653381348
10.0512820512821 0.67052549123764
10.5929487179487 0.695768058300018
11.1634615384615 0.628768026828766
11.7660256410256 0.698254883289337
12.400641025641 0.634375154972076
13.0673076923077 0.594621121883392
13.7724358974359 0.53405886888504
14.5128205128205 0.570012032985687
15.2948717948718 0.537223279476166
16.1185897435897 0.570352733135223
16.9871794871795 0.547322869300842
17.900641025641 0.526209354400635
18.8653846153846 0.54284805059433
19.8814102564103 0.536028563976288
20.9519230769231 0.486539840698242
22.0801282051282 0.52196741104126
23.2692307692308 0.437801361083984
24.5224358974359 0.472302347421646
25.8429487179487 0.551640927791595
27.2371794871795 0.467596828937531
28.7019230769231 0.483966082334518
30.25 0.418057352304459
31.8782051282051 0.384022772312164
33.5961538461538 0.416839987039566
35.4038461538462 0.38257309794426
37.3108974358974 0.365334689617157
39.3205128205128 0.436182469129562
41.4391025641026 0.370353937149048
43.6698717948718 0.397792249917984
46.0224358974359 0.380184143781662
48.5 0.317651182413101
51.1121794871795 0.363219022750854
53.8653846153846 0.36677297949791
56.7660256410256 0.369294136762619
59.8237179487179 0.340241253376007
63.0448717948718 0.394876480102539
66.4391025641026 0.32477530837059
70.0192307692308 0.334329515695572
73.7884615384615 0.385021775960922
77.7628205128205 0.354707002639771
81.9519230769231 0.385443419218063
86.3653846153846 0.384065359830856
91.0160256410256 0.374469041824341
95.9166666666667 0.318077862262726
101.083333333333 0.325967133045197
106.525641025641 0.332308948040009
112.262820512821 0.321619987487793
118.310897435897 0.340510368347168
124.682692307692 0.329611599445343
131.397435897436 0.342024594545364
138.474358974359 0.293127328157425
145.929487179487 0.325699627399445
153.788461538462 0.343083143234253
162.070512820513 0.352963358163834
170.801282051282 0.297198623418808
180 0.339042633771896
};
\addplot [, color2, opacity=0.6, mark=diamond*, mark size=0.5, mark options={solid}, only marks, forget plot]
table {%
1 0.285763323307037
1.05128205128205 0.795574128627777
1.10897435897436 0.845627129077911
1.16987179487179 0.849562108516693
1.23076923076923 0.857629418373108
1.29807692307692 0.837050914764404
1.36858974358974 0.887918174266815
1.44230769230769 0.837830364704132
1.51923076923077 0.906481564044952
1.6025641025641 0.83546656370163
1.68910256410256 0.791768074035645
1.77884615384615 0.775756001472473
1.875 0.87506240606308
1.9775641025641 0.810434639453888
2.08333333333333 0.810200691223145
2.19551282051282 0.812253177165985
2.31410256410256 0.772663533687592
2.43910256410256 0.748633682727814
2.57051282051282 0.753781735897064
2.70833333333333 0.663087546825409
2.8525641025641 0.736597716808319
3.00641025641026 0.808428943157196
3.16987179487179 0.77894651889801
3.33974358974359 0.761298358440399
3.51923076923077 0.72836047410965
3.70833333333333 0.785854756832123
3.91025641025641 0.798982918262482
4.11858974358974 0.78741180896759
4.34294871794872 0.803020119667053
4.57692307692308 0.7438023686409
4.82371794871795 0.7915980219841
5.08333333333333 0.728120744228363
5.35576923076923 0.777872502803802
5.64423076923077 0.815408885478973
5.94871794871795 0.715947449207306
6.26923076923077 0.761055648326874
6.60576923076923 0.758642494678497
6.96153846153846 0.675444722175598
7.33653846153846 0.672043144702911
7.73397435897436 0.790943264961243
8.15064102564103 0.664753079414368
8.58974358974359 0.628942668437958
9.05128205128205 0.653115749359131
9.53846153846154 0.674408495426178
10.0512820512821 0.679857015609741
10.5929487179487 0.635262608528137
11.1634615384615 0.600906848907471
11.7660256410256 0.591902673244476
12.400641025641 0.550721168518066
13.0673076923077 0.580403447151184
13.7724358974359 0.686040878295898
14.5128205128205 0.603906750679016
15.2948717948718 0.490711778402328
16.1185897435897 0.531691193580627
16.9871794871795 0.569947421550751
17.900641025641 0.505355060100555
18.8653846153846 0.583765923976898
19.8814102564103 0.506695866584778
20.9519230769231 0.554836213588715
22.0801282051282 0.522643864154816
23.2692307692308 0.559481561183929
24.5224358974359 0.488628596067429
25.8429487179487 0.467340797185898
27.2371794871795 0.509064853191376
28.7019230769231 0.484683126211166
30.25 0.439852923154831
31.8782051282051 0.477106779813766
33.5961538461538 0.437792867422104
35.4038461538462 0.508601367473602
37.3108974358974 0.425069481134415
39.3205128205128 0.395737707614899
41.4391025641026 0.383669525384903
43.6698717948718 0.385194808244705
46.0224358974359 0.420233726501465
48.5 0.3722163438797
51.1121794871795 0.347556322813034
53.8653846153846 0.344409853219986
56.7660256410256 0.392092168331146
59.8237179487179 0.371646374464035
63.0448717948718 0.345573008060455
66.4391025641026 0.346542745828629
70.0192307692308 0.366894334554672
73.7884615384615 0.350660562515259
77.7628205128205 0.383280515670776
81.9519230769231 0.365889549255371
86.3653846153846 0.322756052017212
91.0160256410256 0.338671535253525
95.9166666666667 0.354666024446487
101.083333333333 0.341141074895859
106.525641025641 0.337334394454956
112.262820512821 0.344331353902817
118.310897435897 0.346874266862869
124.682692307692 0.355634689331055
131.397435897436 0.334206491708755
138.474358974359 0.346375316381454
145.929487179487 0.345579475164413
153.788461538462 0.355228245258331
162.070512820513 0.35076367855072
170.801282051282 0.313125103712082
180 0.337030798196793
};
\addplot [, black, opacity=0.6, mark=*, mark size=0.5, mark options={solid}, only marks]
table {%
1 0.305525034666061
1.05128205128205 0.867510735988617
1.10897435897436 0.85441267490387
1.16987179487179 0.968456208705902
1.23076923076923 0.904888570308685
1.29807692307692 0.907327473163605
1.36858974358974 0.956023156642914
1.44230769230769 0.902581512928009
1.51923076923077 0.862817943096161
1.6025641025641 0.873826682567596
1.68910256410256 0.86531525850296
1.77884615384615 0.8625847697258
1.875 0.878564357757568
1.9775641025641 0.877597033977509
2.08333333333333 0.929476261138916
2.19551282051282 0.884415626525879
2.31410256410256 0.841836392879486
2.43910256410256 0.855202496051788
2.57051282051282 0.869396805763245
2.70833333333333 0.876039922237396
2.8525641025641 0.851332128047943
3.00641025641026 0.843810856342316
3.16987179487179 0.93546849489212
3.33974358974359 0.826354205608368
3.51923076923077 0.843248784542084
3.70833333333333 0.809187114238739
3.91025641025641 0.89957582950592
4.11858974358974 0.865671753883362
4.34294871794872 0.930556893348694
4.57692307692308 0.834340035915375
4.82371794871795 0.86573201417923
5.08333333333333 0.842539966106415
5.35576923076923 0.936816513538361
5.64423076923077 0.865322113037109
5.94871794871795 0.855805099010468
6.26923076923077 0.930491924285889
6.60576923076923 0.864096462726593
6.96153846153846 0.852414309978485
7.33653846153846 0.86074298620224
7.73397435897436 0.916081845760345
8.15064102564103 0.879328906536102
8.58974358974359 0.825703263282776
9.05128205128205 0.816291987895966
9.53846153846154 0.841651737689972
10.0512820512821 0.846462249755859
10.5929487179487 0.80458015203476
11.1634615384615 0.808720409870148
11.7660256410256 0.850522994995117
12.400641025641 0.779815375804901
13.0673076923077 0.834940373897552
13.7724358974359 0.769226729869843
14.5128205128205 0.835479378700256
15.2948717948718 0.738794803619385
16.1185897435897 0.739693462848663
16.9871794871795 0.70992386341095
17.900641025641 0.754636585712433
18.8653846153846 0.737872123718262
19.8814102564103 0.770030200481415
20.9519230769231 0.754261314868927
22.0801282051282 0.73608535528183
23.2692307692308 0.653751134872437
24.5224358974359 0.699576854705811
25.8429487179487 0.713391482830048
27.2371794871795 0.685398578643799
28.7019230769231 0.648679077625275
30.25 0.657928943634033
31.8782051282051 0.5930215716362
33.5961538461538 0.667793691158295
35.4038461538462 0.613419532775879
37.3108974358974 0.520879209041595
39.3205128205128 0.606424748897552
41.4391025641026 0.564800441265106
43.6698717948718 0.585728824138641
46.0224358974359 0.557176291942596
48.5 0.494693279266357
51.1121794871795 0.530408561229706
53.8653846153846 0.55501788854599
56.7660256410256 0.545189678668976
59.8237179487179 0.483681201934814
63.0448717948718 0.490001350641251
66.4391025641026 0.484029203653336
70.0192307692308 0.480315834283829
73.7884615384615 0.550341546535492
77.7628205128205 0.413457959890366
81.9519230769231 0.494478046894073
86.3653846153846 0.43643456697464
91.0160256410256 0.38387593626976
95.9166666666667 0.442472130060196
101.083333333333 0.357962042093277
106.525641025641 0.361237108707428
112.262820512821 0.410045772790909
118.310897435897 0.38868772983551
124.682692307692 0.391101658344269
131.397435897436 0.341840505599976
138.474358974359 0.341467618942261
145.929487179487 0.373020857572556
153.788461538462 0.342051088809967
162.070512820513 0.33715483546257
170.801282051282 0.307459622621536
180 0.353144615888596
};
\addlegendentry{mb 128, exact}
\addplot [, black, opacity=0.6, mark=*, mark size=0.5, mark options={solid}, only marks, forget plot]
table {%
1 0.273963123559952
1.05128205128205 0.906545460224152
1.10897435897436 0.86246919631958
1.16987179487179 0.979024350643158
1.23076923076923 0.957169234752655
1.29807692307692 0.900367200374603
1.36858974358974 0.95029604434967
1.44230769230769 0.936038196086884
1.51923076923077 0.945952415466309
1.6025641025641 0.871630311012268
1.68910256410256 0.945213615894318
1.77884615384615 0.916598260402679
1.875 0.970188438892365
1.9775641025641 0.884454190731049
2.08333333333333 0.92325097322464
2.19551282051282 0.946746468544006
2.31410256410256 0.90296745300293
2.43910256410256 0.883409440517426
2.57051282051282 0.889931380748749
2.70833333333333 0.861435115337372
2.8525641025641 0.831447064876556
3.00641025641026 0.829699993133545
3.16987179487179 0.946545422077179
3.33974358974359 0.874669969081879
3.51923076923077 0.922058284282684
3.70833333333333 0.863543629646301
3.91025641025641 0.928695380687714
4.11858974358974 0.928799271583557
4.34294871794872 0.949428975582123
4.57692307692308 0.954972684383392
4.82371794871795 0.962281048297882
5.08333333333333 0.933227002620697
5.35576923076923 0.9554363489151
5.64423076923077 0.957210540771484
5.94871794871795 0.903505265712738
6.26923076923077 0.90545791387558
6.60576923076923 0.941758930683136
6.96153846153846 0.945200562477112
7.33653846153846 0.944108426570892
7.73397435897436 0.895507335662842
8.15064102564103 0.916976928710938
8.58974358974359 0.898813247680664
9.05128205128205 0.874596238136292
9.53846153846154 0.89363044500351
10.0512820512821 0.899371922016144
10.5929487179487 0.859142124652863
11.1634615384615 0.90508633852005
11.7660256410256 0.914695203304291
12.400641025641 0.884411633014679
13.0673076923077 0.897495448589325
13.7724358974359 0.812052845954895
14.5128205128205 0.856431424617767
15.2948717948718 0.862488567829132
16.1185897435897 0.821696281433105
16.9871794871795 0.851251721382141
17.900641025641 0.841248989105225
18.8653846153846 0.803831696510315
19.8814102564103 0.810796558856964
20.9519230769231 0.820944964885712
22.0801282051282 0.786037623882294
23.2692307692308 0.756000816822052
24.5224358974359 0.730936169624329
25.8429487179487 0.727111995220184
27.2371794871795 0.723686218261719
28.7019230769231 0.690563201904297
30.25 0.652049005031586
31.8782051282051 0.654570281505585
33.5961538461538 0.688075721263885
35.4038461538462 0.691811680793762
37.3108974358974 0.639884412288666
39.3205128205128 0.664538204669952
41.4391025641026 0.60777348279953
43.6698717948718 0.595055043697357
46.0224358974359 0.598678946495056
48.5 0.526540458202362
51.1121794871795 0.531236112117767
53.8653846153846 0.538845241069794
56.7660256410256 0.590235352516174
59.8237179487179 0.501786649227142
63.0448717948718 0.487831026315689
66.4391025641026 0.416343986988068
70.0192307692308 0.487059593200684
73.7884615384615 0.384841561317444
77.7628205128205 0.417500257492065
81.9519230769231 0.435793697834015
86.3653846153846 0.34729591012001
91.0160256410256 0.411235630512238
95.9166666666667 0.375062018632889
101.083333333333 0.408260017633438
106.525641025641 0.345399349927902
112.262820512821 0.3559250831604
118.310897435897 0.346348255872726
124.682692307692 0.332870870828629
131.397435897436 0.306294977664948
138.474358974359 0.305483460426331
145.929487179487 0.376949965953827
153.788461538462 0.360994189977646
162.070512820513 0.289540618658066
170.801282051282 0.315924793481827
180 0.318314701318741
};
\addplot [, black, opacity=0.6, mark=*, mark size=0.5, mark options={solid}, only marks, forget plot]
table {%
1 0.206792756915092
1.05128205128205 0.923220276832581
1.10897435897436 0.829094707965851
1.16987179487179 0.888688504695892
1.23076923076923 0.978302001953125
1.29807692307692 0.952070415019989
1.36858974358974 0.972220599651337
1.44230769230769 0.972162842750549
1.51923076923077 0.960039138793945
1.6025641025641 0.927901864051819
1.68910256410256 0.935822129249573
1.77884615384615 0.87205046415329
1.875 0.968385636806488
1.9775641025641 0.933841168880463
2.08333333333333 0.950798630714417
2.19551282051282 0.965398967266083
2.31410256410256 0.916502416133881
2.43910256410256 0.9345743060112
2.57051282051282 0.915764808654785
2.70833333333333 0.893345296382904
2.8525641025641 0.849240481853485
3.00641025641026 0.919723331928253
3.16987179487179 0.878533184528351
3.33974358974359 0.931654930114746
3.51923076923077 0.868087887763977
3.70833333333333 0.922607600688934
3.91025641025641 0.881661593914032
4.11858974358974 0.925730407238007
4.34294871794872 0.936521828174591
4.57692307692308 0.951252937316895
4.82371794871795 0.94819450378418
5.08333333333333 0.945086300373077
5.35576923076923 0.857471883296967
5.64423076923077 0.950258076190948
5.94871794871795 0.91396427154541
6.26923076923077 0.92147558927536
6.60576923076923 0.938277244567871
6.96153846153846 0.943231225013733
7.33653846153846 0.884757459163666
7.73397435897436 0.905120372772217
8.15064102564103 0.907542824745178
8.58974358974359 0.89387172460556
9.05128205128205 0.868740379810333
9.53846153846154 0.878661155700684
10.0512820512821 0.901529610157013
10.5929487179487 0.82360565662384
11.1634615384615 0.81480085849762
11.7660256410256 0.799965918064117
12.400641025641 0.814549744129181
13.0673076923077 0.872050225734711
13.7724358974359 0.801087319850922
14.5128205128205 0.816910564899445
15.2948717948718 0.781413853168488
16.1185897435897 0.781532943248749
16.9871794871795 0.726291477680206
17.900641025641 0.76269143819809
18.8653846153846 0.774917244911194
19.8814102564103 0.791600406169891
20.9519230769231 0.742591381072998
22.0801282051282 0.758284866809845
23.2692307692308 0.731964349746704
24.5224358974359 0.705916523933411
25.8429487179487 0.742393672466278
27.2371794871795 0.7006756067276
28.7019230769231 0.651995241641998
30.25 0.652998745441437
31.8782051282051 0.645965993404388
33.5961538461538 0.613315343856812
35.4038461538462 0.573036372661591
37.3108974358974 0.594161152839661
39.3205128205128 0.533027112483978
41.4391025641026 0.574728190898895
43.6698717948718 0.575229704380035
46.0224358974359 0.582828521728516
48.5 0.541238963603973
51.1121794871795 0.462941557168961
53.8653846153846 0.535358846187592
56.7660256410256 0.494771093130112
59.8237179487179 0.448495715856552
63.0448717948718 0.439537018537521
66.4391025641026 0.433353424072266
70.0192307692308 0.393433421850204
73.7884615384615 0.437160789966583
77.7628205128205 0.345540583133698
81.9519230769231 0.368734806776047
86.3653846153846 0.415588915348053
91.0160256410256 0.347919195890427
95.9166666666667 0.380706697702408
101.083333333333 0.340928256511688
106.525641025641 0.353994935750961
112.262820512821 0.373788893222809
118.310897435897 0.33899986743927
124.682692307692 0.359279811382294
131.397435897436 0.360484182834625
138.474358974359 0.307782620191574
145.929487179487 0.342215925455093
153.788461538462 0.293697208166122
162.070512820513 0.339891225099564
170.801282051282 0.310110777616501
180 0.315504848957062
};
\addplot [, black, opacity=0.6, mark=*, mark size=0.5, mark options={solid}, only marks, forget plot]
table {%
1 0.297460794448853
1.05128205128205 0.882480442523956
1.10897435897436 0.93700760602951
1.16987179487179 0.97179251909256
1.23076923076923 0.974449336528778
1.29807692307692 0.896788239479065
1.36858974358974 0.964662730693817
1.44230769230769 0.962470471858978
1.51923076923077 0.963107883930206
1.6025641025641 0.918927192687988
1.68910256410256 0.952753365039825
1.77884615384615 0.948655545711517
1.875 0.961739361286163
1.9775641025641 0.896327972412109
2.08333333333333 0.953063189983368
2.19551282051282 0.922832131385803
2.31410256410256 0.850580036640167
2.43910256410256 0.840593934059143
2.57051282051282 0.852829396724701
2.70833333333333 0.862501740455627
2.8525641025641 0.937472343444824
3.00641025641026 0.875166356563568
3.16987179487179 0.935013949871063
3.33974358974359 0.890396773815155
3.51923076923077 0.882086098194122
3.70833333333333 0.858346164226532
3.91025641025641 0.946393609046936
4.11858974358974 0.883803188800812
4.34294871794872 0.862473905086517
4.57692307692308 0.864787399768829
4.82371794871795 0.865656793117523
5.08333333333333 0.851107776165009
5.35576923076923 0.887620151042938
5.64423076923077 0.945471286773682
5.94871794871795 0.864815533161163
6.26923076923077 0.873511731624603
6.60576923076923 0.849579513072968
6.96153846153846 0.868328869342804
7.33653846153846 0.923715770244598
7.73397435897436 0.848408341407776
8.15064102564103 0.891354858875275
8.58974358974359 0.856627106666565
9.05128205128205 0.854905307292938
9.53846153846154 0.880493760108948
10.0512820512821 0.908465385437012
10.5929487179487 0.901594758033752
11.1634615384615 0.901255786418915
11.7660256410256 0.853381276130676
12.400641025641 0.813972771167755
13.0673076923077 0.866974830627441
13.7724358974359 0.831236481666565
14.5128205128205 0.792718529701233
15.2948717948718 0.753373742103577
16.1185897435897 0.815903306007385
16.9871794871795 0.815936505794525
17.900641025641 0.776154458522797
18.8653846153846 0.791139721870422
19.8814102564103 0.70494681596756
20.9519230769231 0.77446460723877
22.0801282051282 0.727085828781128
23.2692307692308 0.734372615814209
24.5224358974359 0.708732783794403
25.8429487179487 0.73063725233078
27.2371794871795 0.721469044685364
28.7019230769231 0.753333568572998
30.25 0.678173005580902
31.8782051282051 0.620563507080078
33.5961538461538 0.603319823741913
35.4038461538462 0.688043117523193
37.3108974358974 0.672308743000031
39.3205128205128 0.62028980255127
41.4391025641026 0.582560658454895
43.6698717948718 0.636888563632965
46.0224358974359 0.564417660236359
48.5 0.576473832130432
51.1121794871795 0.578295826911926
53.8653846153846 0.563542604446411
56.7660256410256 0.504777610301971
59.8237179487179 0.471317201852798
63.0448717948718 0.465050220489502
66.4391025641026 0.478257179260254
70.0192307692308 0.468680053949356
73.7884615384615 0.455231875181198
77.7628205128205 0.445824831724167
81.9519230769231 0.422278493642807
86.3653846153846 0.386949330568314
91.0160256410256 0.404509156942368
95.9166666666667 0.411584466695786
101.083333333333 0.415474504232407
106.525641025641 0.405273526906967
112.262820512821 0.354053020477295
118.310897435897 0.389715611934662
124.682692307692 0.367266356945038
131.397435897436 0.3234843313694
138.474358974359 0.359880536794662
145.929487179487 0.346529930830002
153.788461538462 0.353375673294067
162.070512820513 0.34207546710968
170.801282051282 0.405372351408005
180 0.314199268817902
};
\addplot [, black, opacity=0.6, mark=*, mark size=0.5, mark options={solid}, only marks, forget plot]
table {%
1 0.219316050410271
1.05128205128205 0.938154518604279
1.10897435897436 0.851568818092346
1.16987179487179 0.985646069049835
1.23076923076923 0.971210122108459
1.29807692307692 0.935286819934845
1.36858974358974 0.968973815441132
1.44230769230769 0.961067795753479
1.51923076923077 0.960545539855957
1.6025641025641 0.888456761837006
1.68910256410256 0.970799624919891
1.77884615384615 0.953144073486328
1.875 0.943865597248077
1.9775641025641 0.926710784435272
2.08333333333333 0.947180867195129
2.19551282051282 0.965405404567719
2.31410256410256 0.865653216838837
2.43910256410256 0.855773746967316
2.57051282051282 0.852051556110382
2.70833333333333 0.869077324867249
2.8525641025641 0.934063136577606
3.00641025641026 0.867492854595184
3.16987179487179 0.933247208595276
3.33974358974359 0.88973331451416
3.51923076923077 0.930832862854004
3.70833333333333 0.950731217861176
3.91025641025641 0.953973948955536
4.11858974358974 0.915761768817902
4.34294871794872 0.826594293117523
4.57692307692308 0.87071293592453
4.82371794871795 0.900009453296661
5.08333333333333 0.861703515052795
5.35576923076923 0.86574786901474
5.64423076923077 0.954870998859406
5.94871794871795 0.852251827716827
6.26923076923077 0.844365119934082
6.60576923076923 0.910618960857391
6.96153846153846 0.865170776844025
7.33653846153846 0.89348566532135
7.73397435897436 0.903079450130463
8.15064102564103 0.870280265808105
8.58974358974359 0.867247760295868
9.05128205128205 0.859774589538574
9.53846153846154 0.907321870326996
10.0512820512821 0.875263631343842
10.5929487179487 0.874847590923309
11.1634615384615 0.815462052822113
11.7660256410256 0.80604612827301
12.400641025641 0.798103988170624
13.0673076923077 0.816095769405365
13.7724358974359 0.814946949481964
14.5128205128205 0.843983471393585
15.2948717948718 0.77983021736145
16.1185897435897 0.782485067844391
16.9871794871795 0.756135642528534
17.900641025641 0.796581566333771
18.8653846153846 0.719915866851807
19.8814102564103 0.709145724773407
20.9519230769231 0.717011749744415
22.0801282051282 0.723943769931793
23.2692307692308 0.673079788684845
24.5224358974359 0.68275111913681
25.8429487179487 0.654770255088806
27.2371794871795 0.678841233253479
28.7019230769231 0.641286432743073
30.25 0.659969568252563
31.8782051282051 0.700843274593353
33.5961538461538 0.594046533107758
35.4038461538462 0.632168769836426
37.3108974358974 0.625037550926208
39.3205128205128 0.538616597652435
41.4391025641026 0.613926470279694
43.6698717948718 0.490127176046371
46.0224358974359 0.610361397266388
48.5 0.515202045440674
51.1121794871795 0.496977657079697
53.8653846153846 0.529698312282562
56.7660256410256 0.513827502727509
59.8237179487179 0.506919205188751
63.0448717948718 0.406936079263687
66.4391025641026 0.451145082712173
70.0192307692308 0.455984443426132
73.7884615384615 0.381773471832275
77.7628205128205 0.381287902593613
81.9519230769231 0.417009979486465
86.3653846153846 0.444061487913132
91.0160256410256 0.380745738744736
95.9166666666667 0.344195753335953
101.083333333333 0.34714937210083
106.525641025641 0.32200875878334
112.262820512821 0.372582525014877
118.310897435897 0.313045233488083
124.682692307692 0.35558745265007
131.397435897436 0.353220194578171
138.474358974359 0.31633797287941
145.929487179487 0.389221966266632
153.788461538462 0.300811111927032
162.070512820513 0.36615926027298
170.801282051282 0.335833847522736
180 0.316135853528976
};
\end{axis}

\end{tikzpicture}

      \tikzexternaldisable
    \end{minipage}\hfill
    \begin{minipage}{0.50\linewidth}
      \centering
      % defines the pgfplots style "eigspacedefault"
\pgfkeys{/pgfplots/eigspacedefault/.style={
    width=1.0\linewidth,
    height=0.6\linewidth,
    every axis plot/.append style={line width = 1.5pt},
    tick pos = left,
    ylabel near ticks,
    xlabel near ticks,
    xtick align = inside,
    ytick align = inside,
    legend cell align = left,
    legend columns = 4,
    legend pos = south east,
    legend style = {
      fill opacity = 1,
      text opacity = 1,
      font = \footnotesize,
      at={(1, 1.025)},
      anchor=south east,
      column sep=0.25cm,
    },
    legend image post style={scale=2.5},
    xticklabel style = {font = \footnotesize},
    xlabel style = {font = \footnotesize},
    axis line style = {black},
    yticklabel style = {font = \footnotesize},
    ylabel style = {font = \footnotesize},
    title style = {font = \footnotesize},
    grid = major,
    grid style = {dashed}
  }
}

\pgfkeys{/pgfplots/eigspacedefaultapp/.style={
    eigspacedefault,
    height=0.6\linewidth,
    legend columns = 2,
  }
}

\pgfkeys{/pgfplots/eigspacenolegend/.style={
    legend image post style = {scale=0},
    legend style = {
      fill opacity = 0,
      draw opacity = 0,
      text opacity = 0,
      font = \footnotesize,
      at={(1, 1.025)},
      anchor=south east,
      column sep=0.25cm,
    },
  }
}
%%% Local Variables:
%%% mode: latex
%%% TeX-master: "../../thesis"
%%% End:

      \pgfkeys{/pgfplots/zmystyle/.style={
          eigspacedefaultapp,
        }}
      \tikzexternalenable
      % This file was created by tikzplotlib v0.9.7.
\begin{tikzpicture}

\definecolor{color0}{rgb}{0.274509803921569,0.6,0.564705882352941}
\definecolor{color1}{rgb}{0.870588235294118,0.623529411764706,0.0862745098039216}
\definecolor{color2}{rgb}{0.501960784313725,0.184313725490196,0.6}

\begin{axis}[
axis line style={white!10!black},
legend columns=2,
legend style={fill opacity=0.8, draw opacity=1, text opacity=1, at={(0.03,0.03)}, anchor=south west, draw=white!80!black},
log basis x={10},
tick pos=left,
xlabel={epoch (log scale)},
xmajorgrids,
xmin=0.771323165184619, xmax=233.365219825747,
xmode=log,
ylabel={overlap},
ymajorgrids,
ymin=-0.05, ymax=1.05,
zmystyle
]
\addplot [, white!10!black, dashed, forget plot]
table {%
0.771323165184619 1
233.365219825748 1
};
\addplot [, white!10!black, dashed, forget plot]
table {%
0.771323165184619 0
233.365219825748 0
};
\addplot [, black, opacity=0.6, mark=*, mark size=0.5, mark options={solid}, only marks]
table {%
1 0.305525034666061
1.05128205128205 0.867510735988617
1.10897435897436 0.85441267490387
1.16987179487179 0.968456208705902
1.23076923076923 0.904888570308685
1.29807692307692 0.907327473163605
1.36858974358974 0.956023156642914
1.44230769230769 0.902581512928009
1.51923076923077 0.862817943096161
1.6025641025641 0.873826682567596
1.68910256410256 0.86531525850296
1.77884615384615 0.8625847697258
1.875 0.878564357757568
1.9775641025641 0.877597033977509
2.08333333333333 0.929476261138916
2.19551282051282 0.884415626525879
2.31410256410256 0.841836392879486
2.43910256410256 0.855202496051788
2.57051282051282 0.869396805763245
2.70833333333333 0.876039922237396
2.8525641025641 0.851332128047943
3.00641025641026 0.843810856342316
3.16987179487179 0.93546849489212
3.33974358974359 0.826354205608368
3.51923076923077 0.843248784542084
3.70833333333333 0.809187114238739
3.91025641025641 0.89957582950592
4.11858974358974 0.865671753883362
4.34294871794872 0.930556893348694
4.57692307692308 0.834340035915375
4.82371794871795 0.86573201417923
5.08333333333333 0.842539966106415
5.35576923076923 0.936816513538361
5.64423076923077 0.865322113037109
5.94871794871795 0.855805099010468
6.26923076923077 0.930491924285889
6.60576923076923 0.864096462726593
6.96153846153846 0.852414309978485
7.33653846153846 0.86074298620224
7.73397435897436 0.916081845760345
8.15064102564103 0.879328906536102
8.58974358974359 0.825703263282776
9.05128205128205 0.816291987895966
9.53846153846154 0.841651737689972
10.0512820512821 0.846462249755859
10.5929487179487 0.80458015203476
11.1634615384615 0.808720409870148
11.7660256410256 0.850522994995117
12.400641025641 0.779815375804901
13.0673076923077 0.834940373897552
13.7724358974359 0.769226729869843
14.5128205128205 0.835479378700256
15.2948717948718 0.738794803619385
16.1185897435897 0.739693462848663
16.9871794871795 0.70992386341095
17.900641025641 0.754636585712433
18.8653846153846 0.737872123718262
19.8814102564103 0.770030200481415
20.9519230769231 0.754261314868927
22.0801282051282 0.73608535528183
23.2692307692308 0.653751134872437
24.5224358974359 0.699576854705811
25.8429487179487 0.713391482830048
27.2371794871795 0.685398578643799
28.7019230769231 0.648679077625275
30.25 0.657928943634033
31.8782051282051 0.5930215716362
33.5961538461538 0.667793691158295
35.4038461538462 0.613419532775879
37.3108974358974 0.520879209041595
39.3205128205128 0.606424748897552
41.4391025641026 0.564800441265106
43.6698717948718 0.585728824138641
46.0224358974359 0.557176291942596
48.5 0.494693279266357
51.1121794871795 0.530408561229706
53.8653846153846 0.55501788854599
56.7660256410256 0.545189678668976
59.8237179487179 0.483681201934814
63.0448717948718 0.490001350641251
66.4391025641026 0.484029203653336
70.0192307692308 0.480315834283829
73.7884615384615 0.550341546535492
77.7628205128205 0.413457959890366
81.9519230769231 0.494478046894073
86.3653846153846 0.43643456697464
91.0160256410256 0.38387593626976
95.9166666666667 0.442472130060196
101.083333333333 0.357962042093277
106.525641025641 0.361237108707428
112.262820512821 0.410045772790909
118.310897435897 0.38868772983551
124.682692307692 0.391101658344269
131.397435897436 0.341840505599976
138.474358974359 0.341467618942261
145.929487179487 0.373020857572556
153.788461538462 0.342051088809967
162.070512820513 0.33715483546257
170.801282051282 0.307459622621536
180 0.353144615888596
};
\addlegendentry{mb 128, exact}
\addplot [, black, opacity=0.6, mark=*, mark size=0.5, mark options={solid}, only marks, forget plot]
table {%
1 0.273963123559952
1.05128205128205 0.906545460224152
1.10897435897436 0.86246919631958
1.16987179487179 0.979024350643158
1.23076923076923 0.957169234752655
1.29807692307692 0.900367200374603
1.36858974358974 0.95029604434967
1.44230769230769 0.936038196086884
1.51923076923077 0.945952415466309
1.6025641025641 0.871630311012268
1.68910256410256 0.945213615894318
1.77884615384615 0.916598260402679
1.875 0.970188438892365
1.9775641025641 0.884454190731049
2.08333333333333 0.92325097322464
2.19551282051282 0.946746468544006
2.31410256410256 0.90296745300293
2.43910256410256 0.883409440517426
2.57051282051282 0.889931380748749
2.70833333333333 0.861435115337372
2.8525641025641 0.831447064876556
3.00641025641026 0.829699993133545
3.16987179487179 0.946545422077179
3.33974358974359 0.874669969081879
3.51923076923077 0.922058284282684
3.70833333333333 0.863543629646301
3.91025641025641 0.928695380687714
4.11858974358974 0.928799271583557
4.34294871794872 0.949428975582123
4.57692307692308 0.954972684383392
4.82371794871795 0.962281048297882
5.08333333333333 0.933227002620697
5.35576923076923 0.9554363489151
5.64423076923077 0.957210540771484
5.94871794871795 0.903505265712738
6.26923076923077 0.90545791387558
6.60576923076923 0.941758930683136
6.96153846153846 0.945200562477112
7.33653846153846 0.944108426570892
7.73397435897436 0.895507335662842
8.15064102564103 0.916976928710938
8.58974358974359 0.898813247680664
9.05128205128205 0.874596238136292
9.53846153846154 0.89363044500351
10.0512820512821 0.899371922016144
10.5929487179487 0.859142124652863
11.1634615384615 0.90508633852005
11.7660256410256 0.914695203304291
12.400641025641 0.884411633014679
13.0673076923077 0.897495448589325
13.7724358974359 0.812052845954895
14.5128205128205 0.856431424617767
15.2948717948718 0.862488567829132
16.1185897435897 0.821696281433105
16.9871794871795 0.851251721382141
17.900641025641 0.841248989105225
18.8653846153846 0.803831696510315
19.8814102564103 0.810796558856964
20.9519230769231 0.820944964885712
22.0801282051282 0.786037623882294
23.2692307692308 0.756000816822052
24.5224358974359 0.730936169624329
25.8429487179487 0.727111995220184
27.2371794871795 0.723686218261719
28.7019230769231 0.690563201904297
30.25 0.652049005031586
31.8782051282051 0.654570281505585
33.5961538461538 0.688075721263885
35.4038461538462 0.691811680793762
37.3108974358974 0.639884412288666
39.3205128205128 0.664538204669952
41.4391025641026 0.60777348279953
43.6698717948718 0.595055043697357
46.0224358974359 0.598678946495056
48.5 0.526540458202362
51.1121794871795 0.531236112117767
53.8653846153846 0.538845241069794
56.7660256410256 0.590235352516174
59.8237179487179 0.501786649227142
63.0448717948718 0.487831026315689
66.4391025641026 0.416343986988068
70.0192307692308 0.487059593200684
73.7884615384615 0.384841561317444
77.7628205128205 0.417500257492065
81.9519230769231 0.435793697834015
86.3653846153846 0.34729591012001
91.0160256410256 0.411235630512238
95.9166666666667 0.375062018632889
101.083333333333 0.408260017633438
106.525641025641 0.345399349927902
112.262820512821 0.3559250831604
118.310897435897 0.346348255872726
124.682692307692 0.332870870828629
131.397435897436 0.306294977664948
138.474358974359 0.305483460426331
145.929487179487 0.376949965953827
153.788461538462 0.360994189977646
162.070512820513 0.289540618658066
170.801282051282 0.315924793481827
180 0.318314701318741
};
\addplot [, black, opacity=0.6, mark=*, mark size=0.5, mark options={solid}, only marks, forget plot]
table {%
1 0.206792756915092
1.05128205128205 0.923220276832581
1.10897435897436 0.829094707965851
1.16987179487179 0.888688504695892
1.23076923076923 0.978302001953125
1.29807692307692 0.952070415019989
1.36858974358974 0.972220599651337
1.44230769230769 0.972162842750549
1.51923076923077 0.960039138793945
1.6025641025641 0.927901864051819
1.68910256410256 0.935822129249573
1.77884615384615 0.87205046415329
1.875 0.968385636806488
1.9775641025641 0.933841168880463
2.08333333333333 0.950798630714417
2.19551282051282 0.965398967266083
2.31410256410256 0.916502416133881
2.43910256410256 0.9345743060112
2.57051282051282 0.915764808654785
2.70833333333333 0.893345296382904
2.8525641025641 0.849240481853485
3.00641025641026 0.919723331928253
3.16987179487179 0.878533184528351
3.33974358974359 0.931654930114746
3.51923076923077 0.868087887763977
3.70833333333333 0.922607600688934
3.91025641025641 0.881661593914032
4.11858974358974 0.925730407238007
4.34294871794872 0.936521828174591
4.57692307692308 0.951252937316895
4.82371794871795 0.94819450378418
5.08333333333333 0.945086300373077
5.35576923076923 0.857471883296967
5.64423076923077 0.950258076190948
5.94871794871795 0.91396427154541
6.26923076923077 0.92147558927536
6.60576923076923 0.938277244567871
6.96153846153846 0.943231225013733
7.33653846153846 0.884757459163666
7.73397435897436 0.905120372772217
8.15064102564103 0.907542824745178
8.58974358974359 0.89387172460556
9.05128205128205 0.868740379810333
9.53846153846154 0.878661155700684
10.0512820512821 0.901529610157013
10.5929487179487 0.82360565662384
11.1634615384615 0.81480085849762
11.7660256410256 0.799965918064117
12.400641025641 0.814549744129181
13.0673076923077 0.872050225734711
13.7724358974359 0.801087319850922
14.5128205128205 0.816910564899445
15.2948717948718 0.781413853168488
16.1185897435897 0.781532943248749
16.9871794871795 0.726291477680206
17.900641025641 0.76269143819809
18.8653846153846 0.774917244911194
19.8814102564103 0.791600406169891
20.9519230769231 0.742591381072998
22.0801282051282 0.758284866809845
23.2692307692308 0.731964349746704
24.5224358974359 0.705916523933411
25.8429487179487 0.742393672466278
27.2371794871795 0.7006756067276
28.7019230769231 0.651995241641998
30.25 0.652998745441437
31.8782051282051 0.645965993404388
33.5961538461538 0.613315343856812
35.4038461538462 0.573036372661591
37.3108974358974 0.594161152839661
39.3205128205128 0.533027112483978
41.4391025641026 0.574728190898895
43.6698717948718 0.575229704380035
46.0224358974359 0.582828521728516
48.5 0.541238963603973
51.1121794871795 0.462941557168961
53.8653846153846 0.535358846187592
56.7660256410256 0.494771093130112
59.8237179487179 0.448495715856552
63.0448717948718 0.439537018537521
66.4391025641026 0.433353424072266
70.0192307692308 0.393433421850204
73.7884615384615 0.437160789966583
77.7628205128205 0.345540583133698
81.9519230769231 0.368734806776047
86.3653846153846 0.415588915348053
91.0160256410256 0.347919195890427
95.9166666666667 0.380706697702408
101.083333333333 0.340928256511688
106.525641025641 0.353994935750961
112.262820512821 0.373788893222809
118.310897435897 0.33899986743927
124.682692307692 0.359279811382294
131.397435897436 0.360484182834625
138.474358974359 0.307782620191574
145.929487179487 0.342215925455093
153.788461538462 0.293697208166122
162.070512820513 0.339891225099564
170.801282051282 0.310110777616501
180 0.315504848957062
};
\addplot [, black, opacity=0.6, mark=*, mark size=0.5, mark options={solid}, only marks, forget plot]
table {%
1 0.297460794448853
1.05128205128205 0.882480442523956
1.10897435897436 0.93700760602951
1.16987179487179 0.97179251909256
1.23076923076923 0.974449336528778
1.29807692307692 0.896788239479065
1.36858974358974 0.964662730693817
1.44230769230769 0.962470471858978
1.51923076923077 0.963107883930206
1.6025641025641 0.918927192687988
1.68910256410256 0.952753365039825
1.77884615384615 0.948655545711517
1.875 0.961739361286163
1.9775641025641 0.896327972412109
2.08333333333333 0.953063189983368
2.19551282051282 0.922832131385803
2.31410256410256 0.850580036640167
2.43910256410256 0.840593934059143
2.57051282051282 0.852829396724701
2.70833333333333 0.862501740455627
2.8525641025641 0.937472343444824
3.00641025641026 0.875166356563568
3.16987179487179 0.935013949871063
3.33974358974359 0.890396773815155
3.51923076923077 0.882086098194122
3.70833333333333 0.858346164226532
3.91025641025641 0.946393609046936
4.11858974358974 0.883803188800812
4.34294871794872 0.862473905086517
4.57692307692308 0.864787399768829
4.82371794871795 0.865656793117523
5.08333333333333 0.851107776165009
5.35576923076923 0.887620151042938
5.64423076923077 0.945471286773682
5.94871794871795 0.864815533161163
6.26923076923077 0.873511731624603
6.60576923076923 0.849579513072968
6.96153846153846 0.868328869342804
7.33653846153846 0.923715770244598
7.73397435897436 0.848408341407776
8.15064102564103 0.891354858875275
8.58974358974359 0.856627106666565
9.05128205128205 0.854905307292938
9.53846153846154 0.880493760108948
10.0512820512821 0.908465385437012
10.5929487179487 0.901594758033752
11.1634615384615 0.901255786418915
11.7660256410256 0.853381276130676
12.400641025641 0.813972771167755
13.0673076923077 0.866974830627441
13.7724358974359 0.831236481666565
14.5128205128205 0.792718529701233
15.2948717948718 0.753373742103577
16.1185897435897 0.815903306007385
16.9871794871795 0.815936505794525
17.900641025641 0.776154458522797
18.8653846153846 0.791139721870422
19.8814102564103 0.70494681596756
20.9519230769231 0.77446460723877
22.0801282051282 0.727085828781128
23.2692307692308 0.734372615814209
24.5224358974359 0.708732783794403
25.8429487179487 0.73063725233078
27.2371794871795 0.721469044685364
28.7019230769231 0.753333568572998
30.25 0.678173005580902
31.8782051282051 0.620563507080078
33.5961538461538 0.603319823741913
35.4038461538462 0.688043117523193
37.3108974358974 0.672308743000031
39.3205128205128 0.62028980255127
41.4391025641026 0.582560658454895
43.6698717948718 0.636888563632965
46.0224358974359 0.564417660236359
48.5 0.576473832130432
51.1121794871795 0.578295826911926
53.8653846153846 0.563542604446411
56.7660256410256 0.504777610301971
59.8237179487179 0.471317201852798
63.0448717948718 0.465050220489502
66.4391025641026 0.478257179260254
70.0192307692308 0.468680053949356
73.7884615384615 0.455231875181198
77.7628205128205 0.445824831724167
81.9519230769231 0.422278493642807
86.3653846153846 0.386949330568314
91.0160256410256 0.404509156942368
95.9166666666667 0.411584466695786
101.083333333333 0.415474504232407
106.525641025641 0.405273526906967
112.262820512821 0.354053020477295
118.310897435897 0.389715611934662
124.682692307692 0.367266356945038
131.397435897436 0.3234843313694
138.474358974359 0.359880536794662
145.929487179487 0.346529930830002
153.788461538462 0.353375673294067
162.070512820513 0.34207546710968
170.801282051282 0.405372351408005
180 0.314199268817902
};
\addplot [, black, opacity=0.6, mark=*, mark size=0.5, mark options={solid}, only marks, forget plot]
table {%
1 0.219316050410271
1.05128205128205 0.938154518604279
1.10897435897436 0.851568818092346
1.16987179487179 0.985646069049835
1.23076923076923 0.971210122108459
1.29807692307692 0.935286819934845
1.36858974358974 0.968973815441132
1.44230769230769 0.961067795753479
1.51923076923077 0.960545539855957
1.6025641025641 0.888456761837006
1.68910256410256 0.970799624919891
1.77884615384615 0.953144073486328
1.875 0.943865597248077
1.9775641025641 0.926710784435272
2.08333333333333 0.947180867195129
2.19551282051282 0.965405404567719
2.31410256410256 0.865653216838837
2.43910256410256 0.855773746967316
2.57051282051282 0.852051556110382
2.70833333333333 0.869077324867249
2.8525641025641 0.934063136577606
3.00641025641026 0.867492854595184
3.16987179487179 0.933247208595276
3.33974358974359 0.88973331451416
3.51923076923077 0.930832862854004
3.70833333333333 0.950731217861176
3.91025641025641 0.953973948955536
4.11858974358974 0.915761768817902
4.34294871794872 0.826594293117523
4.57692307692308 0.87071293592453
4.82371794871795 0.900009453296661
5.08333333333333 0.861703515052795
5.35576923076923 0.86574786901474
5.64423076923077 0.954870998859406
5.94871794871795 0.852251827716827
6.26923076923077 0.844365119934082
6.60576923076923 0.910618960857391
6.96153846153846 0.865170776844025
7.33653846153846 0.89348566532135
7.73397435897436 0.903079450130463
8.15064102564103 0.870280265808105
8.58974358974359 0.867247760295868
9.05128205128205 0.859774589538574
9.53846153846154 0.907321870326996
10.0512820512821 0.875263631343842
10.5929487179487 0.874847590923309
11.1634615384615 0.815462052822113
11.7660256410256 0.80604612827301
12.400641025641 0.798103988170624
13.0673076923077 0.816095769405365
13.7724358974359 0.814946949481964
14.5128205128205 0.843983471393585
15.2948717948718 0.77983021736145
16.1185897435897 0.782485067844391
16.9871794871795 0.756135642528534
17.900641025641 0.796581566333771
18.8653846153846 0.719915866851807
19.8814102564103 0.709145724773407
20.9519230769231 0.717011749744415
22.0801282051282 0.723943769931793
23.2692307692308 0.673079788684845
24.5224358974359 0.68275111913681
25.8429487179487 0.654770255088806
27.2371794871795 0.678841233253479
28.7019230769231 0.641286432743073
30.25 0.659969568252563
31.8782051282051 0.700843274593353
33.5961538461538 0.594046533107758
35.4038461538462 0.632168769836426
37.3108974358974 0.625037550926208
39.3205128205128 0.538616597652435
41.4391025641026 0.613926470279694
43.6698717948718 0.490127176046371
46.0224358974359 0.610361397266388
48.5 0.515202045440674
51.1121794871795 0.496977657079697
53.8653846153846 0.529698312282562
56.7660256410256 0.513827502727509
59.8237179487179 0.506919205188751
63.0448717948718 0.406936079263687
66.4391025641026 0.451145082712173
70.0192307692308 0.455984443426132
73.7884615384615 0.381773471832275
77.7628205128205 0.381287902593613
81.9519230769231 0.417009979486465
86.3653846153846 0.444061487913132
91.0160256410256 0.380745738744736
95.9166666666667 0.344195753335953
101.083333333333 0.34714937210083
106.525641025641 0.32200875878334
112.262820512821 0.372582525014877
118.310897435897 0.313045233488083
124.682692307692 0.35558745265007
131.397435897436 0.353220194578171
138.474358974359 0.31633797287941
145.929487179487 0.389221966266632
153.788461538462 0.300811111927032
162.070512820513 0.36615926027298
170.801282051282 0.335833847522736
180 0.316135853528976
};
\addplot [, color0, opacity=0.6, mark=diamond*, mark size=0.5, mark options={solid}, only marks]
table {%
1 nan
1.05128205128205 0.787296414375305
1.10897435897436 0.341716527938843
1.16987179487179 0.559464871883392
1.23076923076923 0.658697128295898
1.29807692307692 0.637834966182709
1.36858974358974 0.618748843669891
1.44230769230769 0.659821331501007
1.51923076923077 0.674141883850098
1.6025641025641 0.619910538196564
1.68910256410256 0.603294432163239
1.77884615384615 0.700062870979309
1.875 0.727553069591522
1.9775641025641 0.636114239692688
2.08333333333333 0.608714759349823
2.19551282051282 0.72404271364212
2.31410256410256 0.645243406295776
2.43910256410256 0.643086135387421
2.57051282051282 0.626960158348083
2.70833333333333 0.703215539455414
2.8525641025641 0.637810707092285
3.00641025641026 0.669680058956146
3.16987179487179 0.685514390468597
3.33974358974359 0.635305106639862
3.51923076923077 0.688464224338531
3.70833333333333 0.659740388393402
3.91025641025641 0.765679180622101
4.11858974358974 0.667312622070312
4.34294871794872 0.670680165290833
4.57692307692308 0.727368354797363
4.82371794871795 0.77883380651474
5.08333333333333 0.712457120418549
5.35576923076923 0.756517231464386
5.64423076923077 0.761342525482178
5.94871794871795 0.769309222698212
6.26923076923077 0.704644680023193
6.60576923076923 0.73539537191391
6.96153846153846 0.673787713050842
7.33653846153846 0.630295276641846
7.73397435897436 0.644200444221497
8.15064102564103 0.663056075572968
8.58974358974359 0.631548583507538
9.05128205128205 0.634353458881378
9.53846153846154 0.643320322036743
10.0512820512821 0.630949795246124
10.5929487179487 0.597316920757294
11.1634615384615 0.641543984413147
11.7660256410256 0.622210681438446
12.400641025641 0.594637811183929
13.0673076923077 0.607145965099335
13.7724358974359 0.522967576980591
14.5128205128205 0.55982780456543
15.2948717948718 0.501726806163788
16.1185897435897 0.490044742822647
16.9871794871795 0.483906030654907
17.900641025641 0.469096750020981
18.8653846153846 0.461012363433838
19.8814102564103 0.406124413013458
20.9519230769231 0.45389586687088
22.0801282051282 0.45491886138916
23.2692307692308 0.407481491565704
24.5224358974359 0.431717485189438
25.8429487179487 0.397908300161362
27.2371794871795 0.375948697328568
28.7019230769231 0.354626387357712
30.25 0.335210531949997
31.8782051282051 0.36757305264473
33.5961538461538 0.380650132894516
35.4038461538462 0.326958149671555
37.3108974358974 0.289631426334381
39.3205128205128 0.357879489660263
41.4391025641026 0.294454544782639
43.6698717948718 0.366615772247314
46.0224358974359 0.311958014965057
48.5 0.343654841184616
51.1121794871795 0.334962099790573
53.8653846153846 0.31632798910141
56.7660256410256 0.325199514627457
59.8237179487179 0.322878181934357
63.0448717948718 0.317509979009628
66.4391025641026 0.288313120603561
70.0192307692308 0.297987997531891
73.7884615384615 0.290280669927597
77.7628205128205 0.329104155302048
81.9519230769231 0.336437880992889
86.3653846153846 0.285860955715179
91.0160256410256 0.293111622333527
95.9166666666667 0.323321253061295
101.083333333333 0.280409276485443
106.525641025641 0.293899148702621
112.262820512821 0.29670262336731
118.310897435897 0.273356944322586
124.682692307692 0.283791273832321
131.397435897436 0.295236498117447
138.474358974359 0.276421159505844
145.929487179487 0.304669141769409
153.788461538462 0.313095062971115
162.070512820513 0.285759031772614
170.801282051282 0.291515082120895
180 0.288416117429733
};
\addlegendentry{sub 16, exact}
\addplot [, color0, opacity=0.6, mark=diamond*, mark size=0.5, mark options={solid}, only marks, forget plot]
table {%
1 nan
1.05128205128205 0.770072758197784
1.10897435897436 0.791204452514648
1.16987179487179 0.820801734924316
1.23076923076923 0.833134770393372
1.29807692307692 0.805033206939697
1.36858974358974 0.756803035736084
1.44230769230769 0.824234902858734
1.51923076923077 0.788467049598694
1.6025641025641 0.79719465970993
1.68910256410256 0.722570240497589
1.77884615384615 0.68423855304718
1.875 0.773012638092041
1.9775641025641 0.752692699432373
2.08333333333333 0.769795894622803
2.19551282051282 0.778992831707001
2.31410256410256 0.701546967029572
2.43910256410256 0.688456952571869
2.57051282051282 0.772290408611298
2.70833333333333 0.665529072284698
2.8525641025641 0.666313827037811
3.00641025641026 0.697724461555481
3.16987179487179 0.687595665454865
3.33974358974359 0.668646275997162
3.51923076923077 0.655002117156982
3.70833333333333 0.617511749267578
3.91025641025641 0.756911277770996
4.11858974358974 0.719839572906494
4.34294871794872 0.751128017902374
4.57692307692308 0.659290492534637
4.82371794871795 0.696772336959839
5.08333333333333 0.645805954933167
5.35576923076923 0.611137986183167
5.64423076923077 0.69206041097641
5.94871794871795 0.607123196125031
6.26923076923077 0.647566020488739
6.60576923076923 0.621948480606079
6.96153846153846 0.645323216915131
7.33653846153846 0.665020883083344
7.73397435897436 0.605235278606415
8.15064102564103 0.575415074825287
8.58974358974359 0.564880073070526
9.05128205128205 0.586161255836487
9.53846153846154 0.591042935848236
10.0512820512821 0.578662812709808
10.5929487179487 0.619256615638733
11.1634615384615 0.537397086620331
11.7660256410256 0.540294826030731
12.400641025641 0.486752599477768
13.0673076923077 0.553405463695526
13.7724358974359 0.532666325569153
14.5128205128205 0.576127231121063
15.2948717948718 0.43267947435379
16.1185897435897 0.452791541814804
16.9871794871795 0.497976779937744
17.900641025641 0.47123971581459
18.8653846153846 0.498738676309586
19.8814102564103 0.513006329536438
20.9519230769231 0.47820383310318
22.0801282051282 0.503439426422119
23.2692307692308 0.452172487974167
24.5224358974359 0.457614332437515
25.8429487179487 0.422363191843033
27.2371794871795 0.407278537750244
28.7019230769231 0.39550718665123
30.25 0.395401209592819
31.8782051282051 0.400595963001251
33.5961538461538 0.379548311233521
35.4038461538462 0.39244070649147
37.3108974358974 0.33978533744812
39.3205128205128 0.370695441961288
41.4391025641026 0.355889946222305
43.6698717948718 0.32410717010498
46.0224358974359 0.33412292599678
48.5 0.331368833780289
51.1121794871795 0.333503633737564
53.8653846153846 0.326424419879913
56.7660256410256 0.340091705322266
59.8237179487179 0.300802052021027
63.0448717948718 0.275890082120895
66.4391025641026 0.275190442800522
70.0192307692308 0.270380914211273
73.7884615384615 0.277205467224121
77.7628205128205 0.324699014425278
81.9519230769231 0.310707271099091
86.3653846153846 0.308576822280884
91.0160256410256 0.277119368314743
95.9166666666667 0.302345186471939
101.083333333333 0.298391968011856
106.525641025641 0.262573331594467
112.262820512821 0.274595379829407
118.310897435897 0.269538253545761
124.682692307692 0.26643705368042
131.397435897436 0.280833840370178
138.474358974359 0.285434544086456
145.929487179487 0.262690305709839
153.788461538462 0.252647250890732
162.070512820513 0.306808620691299
170.801282051282 0.304257929325104
180 0.267520725727081
};
\addplot [, color0, opacity=0.6, mark=diamond*, mark size=0.5, mark options={solid}, only marks, forget plot]
table {%
1 nan
1.05128205128205 0.737269341945648
1.10897435897436 0.668615281581879
1.16987179487179 0.751535594463348
1.23076923076923 0.880674540996552
1.29807692307692 0.817265689373016
1.36858974358974 0.881254613399506
1.44230769230769 0.835234940052032
1.51923076923077 0.793693840503693
1.6025641025641 0.773774445056915
1.68910256410256 0.77345073223114
1.77884615384615 0.788453221321106
1.875 0.793509483337402
1.9775641025641 0.764506161212921
2.08333333333333 0.775597095489502
2.19551282051282 0.822018563747406
2.31410256410256 0.742298781871796
2.43910256410256 0.727816522121429
2.57051282051282 0.693024158477783
2.70833333333333 0.686312913894653
2.8525641025641 0.644751012325287
3.00641025641026 0.728629410266876
3.16987179487179 0.666903734207153
3.33974358974359 0.637276828289032
3.51923076923077 0.668929874897003
3.70833333333333 0.658315300941467
3.91025641025641 0.714023053646088
4.11858974358974 0.678191006183624
4.34294871794872 0.654067695140839
4.57692307692308 0.655840992927551
4.82371794871795 0.727955520153046
5.08333333333333 0.695827007293701
5.35576923076923 0.636587083339691
5.64423076923077 0.666341423988342
5.94871794871795 0.640020072460175
6.26923076923077 0.660502135753632
6.60576923076923 0.683111131191254
6.96153846153846 0.6177077293396
7.33653846153846 0.673321485519409
7.73397435897436 0.615954577922821
8.15064102564103 0.568098843097687
8.58974358974359 0.544478952884674
9.05128205128205 0.61302775144577
9.53846153846154 0.632744193077087
10.0512820512821 0.563591361045837
10.5929487179487 0.586746633052826
11.1634615384615 0.553620576858521
11.7660256410256 0.50322812795639
12.400641025641 0.521730780601501
13.0673076923077 0.534378707408905
13.7724358974359 0.479341953992844
14.5128205128205 0.533807337284088
15.2948717948718 0.517334640026093
16.1185897435897 0.495530903339386
16.9871794871795 0.529996454715729
17.900641025641 0.439991295337677
18.8653846153846 0.457988262176514
19.8814102564103 0.44716340303421
20.9519230769231 0.462819635868073
22.0801282051282 0.453964293003082
23.2692307692308 0.453451603651047
24.5224358974359 0.42837730050087
25.8429487179487 0.402654558420181
27.2371794871795 0.466851621866226
28.7019230769231 0.390777319669724
30.25 0.445998013019562
31.8782051282051 0.42886421084404
33.5961538461538 0.415134996175766
35.4038461538462 0.428709328174591
37.3108974358974 0.402612119913101
39.3205128205128 0.41131392121315
41.4391025641026 0.376880973577499
43.6698717948718 0.410965442657471
46.0224358974359 0.372139185667038
48.5 0.336851686239243
51.1121794871795 0.369508266448975
53.8653846153846 0.352956026792526
56.7660256410256 0.348189651966095
59.8237179487179 0.347900003194809
63.0448717948718 0.342004179954529
66.4391025641026 0.348270088434219
70.0192307692308 0.364360690116882
73.7884615384615 0.388408869504929
77.7628205128205 0.360755532979965
81.9519230769231 0.368019014596939
86.3653846153846 0.320213854312897
91.0160256410256 0.33487531542778
95.9166666666667 0.33641454577446
101.083333333333 0.331885159015656
106.525641025641 0.323686361312866
112.262820512821 0.30768609046936
118.310897435897 0.350720345973969
124.682692307692 0.335415363311768
131.397435897436 0.322355031967163
138.474358974359 0.3294617831707
145.929487179487 0.305220812559128
153.788461538462 0.308044254779816
162.070512820513 0.325859606266022
170.801282051282 0.330958783626556
180 0.313294619321823
};
\addplot [, color0, opacity=0.6, mark=diamond*, mark size=0.5, mark options={solid}, only marks, forget plot]
table {%
1 nan
1.05128205128205 0.809823632240295
1.10897435897436 0.769214034080505
1.16987179487179 0.810292541980743
1.23076923076923 0.785933971405029
1.29807692307692 0.803749263286591
1.36858974358974 0.774912536144257
1.44230769230769 0.774315416812897
1.51923076923077 0.779450118541718
1.6025641025641 0.739787995815277
1.68910256410256 0.748965382575989
1.77884615384615 0.715537011623383
1.875 0.7314213514328
1.9775641025641 0.664735496044159
2.08333333333333 0.72122448682785
2.19551282051282 0.75113046169281
2.31410256410256 0.710469603538513
2.43910256410256 0.629228293895721
2.57051282051282 0.666616082191467
2.70833333333333 0.678719460964203
2.8525641025641 0.660774350166321
3.00641025641026 0.648268401622772
3.16987179487179 0.728893220424652
3.33974358974359 0.723745703697205
3.51923076923077 0.6705561876297
3.70833333333333 0.713748693466187
3.91025641025641 0.762642323970795
4.11858974358974 0.719080567359924
4.34294871794872 0.696426510810852
4.57692307692308 0.728231906890869
4.82371794871795 0.71932452917099
5.08333333333333 0.663029134273529
5.35576923076923 0.753311455249786
5.64423076923077 0.670499265193939
5.94871794871795 0.499130249023438
6.26923076923077 0.640491425991058
6.60576923076923 0.702082097530365
6.96153846153846 0.648268818855286
7.33653846153846 0.599464535713196
7.73397435897436 0.632912456989288
8.15064102564103 0.628660380840302
8.58974358974359 0.634671151638031
9.05128205128205 0.583495795726776
9.53846153846154 0.52623051404953
10.0512820512821 0.552362561225891
10.5929487179487 0.523848235607147
11.1634615384615 0.527920067310333
11.7660256410256 0.508419036865234
12.400641025641 0.544192314147949
13.0673076923077 0.508500993251801
13.7724358974359 0.538316547870636
14.5128205128205 0.499222010374069
15.2948717948718 0.455796211957932
16.1185897435897 0.453293323516846
16.9871794871795 0.478644371032715
17.900641025641 0.45452031493187
18.8653846153846 0.470702737569809
19.8814102564103 0.489201873540878
20.9519230769231 0.476786226034164
22.0801282051282 0.498237460851669
23.2692307692308 0.487183481454849
24.5224358974359 0.410883873701096
25.8429487179487 0.452044486999512
27.2371794871795 0.459716230630875
28.7019230769231 0.445826590061188
30.25 0.404901117086411
31.8782051282051 0.405025392770767
33.5961538461538 0.38564133644104
35.4038461538462 0.360064387321472
37.3108974358974 0.300049513578415
39.3205128205128 0.365083903074265
41.4391025641026 0.416089534759521
43.6698717948718 0.366489499807358
46.0224358974359 0.361625105142593
48.5 0.363093286752701
51.1121794871795 0.424341678619385
53.8653846153846 0.341458082199097
56.7660256410256 0.267859786748886
59.8237179487179 0.315234512090683
63.0448717948718 0.337251275777817
66.4391025641026 0.319413423538208
70.0192307692308 0.317251175642014
73.7884615384615 0.39035364985466
77.7628205128205 0.331262409687042
81.9519230769231 0.329881072044373
86.3653846153846 0.314151287078857
91.0160256410256 0.281196892261505
95.9166666666667 0.31789368391037
101.083333333333 0.332597345113754
106.525641025641 0.288028478622437
112.262820512821 0.334649175405502
118.310897435897 0.322483211755753
124.682692307692 0.338749378919601
131.397435897436 0.351239174604416
138.474358974359 0.296865850687027
145.929487179487 0.31964236497879
153.788461538462 0.29699894785881
162.070512820513 0.350085556507111
170.801282051282 0.294000804424286
180 0.288917928934097
};
\addplot [, color0, opacity=0.6, mark=diamond*, mark size=0.5, mark options={solid}, only marks, forget plot]
table {%
1 0.717557072639465
1.05128205128205 0.703657329082489
1.10897435897436 0.813060581684113
1.16987179487179 0.687048256397247
1.23076923076923 0.621567726135254
1.29807692307692 0.699363172054291
1.36858974358974 0.729743838310242
1.44230769230769 0.701013028621674
1.51923076923077 0.715092658996582
1.6025641025641 0.671732902526855
1.68910256410256 0.705423057079315
1.77884615384615 0.639312446117401
1.875 0.736301064491272
1.9775641025641 0.647394955158234
2.08333333333333 0.723684430122375
2.19551282051282 0.727430939674377
2.31410256410256 0.661649346351624
2.43910256410256 0.690695762634277
2.57051282051282 0.664168775081635
2.70833333333333 0.623188436031342
2.8525641025641 0.65765506029129
3.00641025641026 0.682897865772247
3.16987179487179 0.717012524604797
3.33974358974359 0.713735044002533
3.51923076923077 0.652738749980927
3.70833333333333 0.626223206520081
3.91025641025641 0.688353598117828
4.11858974358974 0.658860683441162
4.34294871794872 0.695876896381378
4.57692307692308 0.734759509563446
4.82371794871795 0.67738151550293
5.08333333333333 0.651400446891785
5.35576923076923 0.679867148399353
5.64423076923077 0.690179944038391
5.94871794871795 0.701943099498749
6.26923076923077 0.62031489610672
6.60576923076923 0.606793403625488
6.96153846153846 0.621835708618164
7.33653846153846 0.592129826545715
7.73397435897436 0.648011565208435
8.15064102564103 0.604087769985199
8.58974358974359 0.520224034786224
9.05128205128205 0.550056397914886
9.53846153846154 0.553486347198486
10.0512820512821 0.567884862422943
10.5929487179487 0.510864198207855
11.1634615384615 0.520150363445282
11.7660256410256 0.543611168861389
12.400641025641 0.51467365026474
13.0673076923077 0.509392499923706
13.7724358974359 0.443770796060562
14.5128205128205 0.505192279815674
15.2948717948718 0.416723936796188
16.1185897435897 0.463872641324997
16.9871794871795 0.429506033658981
17.900641025641 0.491415947675705
18.8653846153846 0.496568113565445
19.8814102564103 0.456797450780869
20.9519230769231 0.398679912090302
22.0801282051282 0.453567326068878
23.2692307692308 0.392470896244049
24.5224358974359 0.458097904920578
25.8429487179487 0.442948967218399
27.2371794871795 0.424284607172012
28.7019230769231 0.397468745708466
30.25 0.364050090312958
31.8782051282051 0.365829765796661
33.5961538461538 0.396704524755478
35.4038461538462 0.374801874160767
37.3108974358974 0.349387139081955
39.3205128205128 0.411450952291489
41.4391025641026 0.350032061338425
43.6698717948718 0.389573603868484
46.0224358974359 0.366480439901352
48.5 0.313062220811844
51.1121794871795 0.328229516744614
53.8653846153846 0.379509925842285
56.7660256410256 0.355902016162872
59.8237179487179 0.321677595376968
63.0448717948718 0.333009749650955
66.4391025641026 0.314825266599655
70.0192307692308 0.340825647115707
73.7884615384615 0.288131535053253
77.7628205128205 0.350397914648056
81.9519230769231 0.329314440488815
86.3653846153846 0.328113496303558
91.0160256410256 0.306049674749374
95.9166666666667 0.315015017986298
101.083333333333 0.30150979757309
106.525641025641 0.326097458600998
112.262820512821 0.318707793951035
118.310897435897 0.328610450029373
124.682692307692 0.278510183095932
131.397435897436 0.302635192871094
138.474358974359 0.295163720846176
145.929487179487 0.298985809087753
153.788461538462 0.302934050559998
162.070512820513 0.299456030130386
170.801282051282 0.3259536921978
180 0.286000102758408
};
\addplot [, color1, opacity=0.6, mark=square*, mark size=0.5, mark options={solid}, only marks]
table {%
1 nan
1.05128205128205 0.621541857719421
1.10897435897436 0.775745511054993
1.16987179487179 0.78773444890976
1.23076923076923 0.880942046642303
1.29807692307692 0.770613789558411
1.36858974358974 0.80517452955246
1.44230769230769 0.802651226520538
1.51923076923077 0.830777764320374
1.6025641025641 0.751787304878235
1.68910256410256 0.752033650875092
1.77884615384615 0.775136530399323
1.875 0.791503071784973
1.9775641025641 0.709574222564697
2.08333333333333 0.766482174396515
2.19551282051282 0.81777012348175
2.31410256410256 0.718554317951202
2.43910256410256 0.843722641468048
2.57051282051282 0.744109332561493
2.70833333333333 0.713825643062592
2.8525641025641 0.698464334011078
3.00641025641026 0.713587939739227
3.16987179487179 0.741166293621063
3.33974358974359 0.712662160396576
3.51923076923077 0.746036946773529
3.70833333333333 0.755732715129852
3.91025641025641 0.798372745513916
4.11858974358974 0.721931874752045
4.34294871794872 0.759517252445221
4.57692307692308 0.805898010730743
4.82371794871795 0.842684924602509
5.08333333333333 0.783035337924957
5.35576923076923 0.751816928386688
5.64423076923077 0.799467504024506
5.94871794871795 0.776795327663422
6.26923076923077 0.747207820415497
6.60576923076923 0.669801354408264
6.96153846153846 0.744377613067627
7.33653846153846 0.701013743877411
7.73397435897436 0.677369117736816
8.15064102564103 0.656846225261688
8.58974358974359 0.626492738723755
9.05128205128205 0.675735890865326
9.53846153846154 0.645003795623779
10.0512820512821 0.669503808021545
10.5929487179487 0.603923976421356
11.1634615384615 0.574891567230225
11.7660256410256 0.631573617458344
12.400641025641 0.561207294464111
13.0673076923077 0.592244267463684
13.7724358974359 0.493695169687271
14.5128205128205 0.483800411224365
15.2948717948718 0.468765467405319
16.1185897435897 0.53763210773468
16.9871794871795 0.464495271444321
17.900641025641 0.505405902862549
18.8653846153846 0.514490127563477
19.8814102564103 0.484131902456284
20.9519230769231 0.441310614347458
22.0801282051282 0.406543463468552
23.2692307692308 0.469471544027328
24.5224358974359 0.44245645403862
25.8429487179487 0.464362442493439
27.2371794871795 0.430447190999985
28.7019230769231 0.429820746183395
30.25 0.38801583647728
31.8782051282051 0.477259933948517
33.5961538461538 0.399388045072556
35.4038461538462 0.389070093631744
37.3108974358974 0.352362960577011
39.3205128205128 0.354051500558853
41.4391025641026 0.318654477596283
43.6698717948718 0.381455630064011
46.0224358974359 0.35791802406311
48.5 0.303042739629745
51.1121794871795 0.344236731529236
53.8653846153846 0.328806936740875
56.7660256410256 0.347741484642029
59.8237179487179 0.33527597784996
63.0448717948718 0.320937305688858
66.4391025641026 0.29811030626297
70.0192307692308 0.277653902769089
73.7884615384615 0.344263911247253
77.7628205128205 0.299950301647186
81.9519230769231 0.30636277794838
86.3653846153846 0.273136109113693
91.0160256410256 0.299319565296173
95.9166666666667 0.313381195068359
101.083333333333 0.309535264968872
106.525641025641 0.308433920145035
112.262820512821 0.310621380805969
118.310897435897 0.264092445373535
124.682692307692 0.277664870023727
131.397435897436 0.257152646780014
138.474358974359 0.277708023786545
145.929487179487 0.288312047719955
153.788461538462 0.298931092023849
162.070512820513 0.305384397506714
170.801282051282 0.236110672354698
180 0.300639927387238
};
\addlegendentry{mb 128, mc 1}
\addplot [, color1, opacity=0.6, mark=square*, mark size=0.5, mark options={solid}, only marks, forget plot]
table {%
1 nan
1.05128205128205 0.650911748409271
1.10897435897436 0.705987155437469
1.16987179487179 0.840046226978302
1.23076923076923 0.855182588100433
1.29807692307692 0.803165853023529
1.36858974358974 0.848217010498047
1.44230769230769 0.813073933124542
1.51923076923077 0.817683041095734
1.6025641025641 0.756056606769562
1.68910256410256 0.747743606567383
1.77884615384615 0.773739635944366
1.875 0.807223439216614
1.9775641025641 0.775766789913177
2.08333333333333 0.840393722057343
2.19551282051282 0.779726386070251
2.31410256410256 0.763619720935822
2.43910256410256 0.779689311981201
2.57051282051282 0.780206620693207
2.70833333333333 0.674983978271484
2.8525641025641 0.7392298579216
3.00641025641026 0.795236051082611
3.16987179487179 0.654142796993256
3.33974358974359 0.810782611370087
3.51923076923077 0.755384206771851
3.70833333333333 0.753488659858704
3.91025641025641 0.76416677236557
4.11858974358974 0.778519093990326
4.34294871794872 0.776477038860321
4.57692307692308 0.762921035289764
4.82371794871795 0.764510333538055
5.08333333333333 0.687806308269501
5.35576923076923 0.640260219573975
5.64423076923077 0.762246429920197
5.94871794871795 0.725310802459717
6.26923076923077 0.734080255031586
6.60576923076923 0.701933324337006
6.96153846153846 0.764155685901642
7.33653846153846 0.742861449718475
7.73397435897436 0.74128669500351
8.15064102564103 0.662566840648651
8.58974358974359 0.666951358318329
9.05128205128205 0.670024335384369
9.53846153846154 0.61127120256424
10.0512820512821 0.671471059322357
10.5929487179487 0.632997334003448
11.1634615384615 0.640892148017883
11.7660256410256 0.62718266248703
12.400641025641 0.614910662174225
13.0673076923077 0.608419299125671
13.7724358974359 0.612344086170197
14.5128205128205 0.597926557064056
15.2948717948718 0.622275054454803
16.1185897435897 0.519202649593353
16.9871794871795 0.563720405101776
17.900641025641 0.567735373973846
18.8653846153846 0.527771472930908
19.8814102564103 0.560007274150848
20.9519230769231 0.496904283761978
22.0801282051282 0.487139314413071
23.2692307692308 0.547139942646027
24.5224358974359 0.517373859882355
25.8429487179487 0.508027970790863
27.2371794871795 0.475695610046387
28.7019230769231 0.445380300283432
30.25 0.428761303424835
31.8782051282051 0.430388271808624
33.5961538461538 0.450563430786133
35.4038461538462 0.458761900663376
37.3108974358974 0.424930721521378
39.3205128205128 0.388629794120789
41.4391025641026 0.428522884845734
43.6698717948718 0.406435579061508
46.0224358974359 0.435052484273911
48.5 0.375604003667831
51.1121794871795 0.344951301813126
53.8653846153846 0.405167669057846
56.7660256410256 0.393583387136459
59.8237179487179 0.368425577878952
63.0448717948718 0.407325357198715
66.4391025641026 0.358487337827682
70.0192307692308 0.353394776582718
73.7884615384615 0.353904694318771
77.7628205128205 0.369127571582794
81.9519230769231 0.353337943553925
86.3653846153846 0.311327427625656
91.0160256410256 0.285257667303085
95.9166666666667 0.309154957532883
101.083333333333 0.338400453329086
106.525641025641 0.349733620882034
112.262820512821 0.330964237451553
118.310897435897 0.338943630456924
124.682692307692 0.33026534318924
131.397435897436 0.274519056081772
138.474358974359 0.323454141616821
145.929487179487 0.335375547409058
153.788461538462 0.306756287813187
162.070512820513 0.357546985149384
170.801282051282 0.31031459569931
180 0.283918470144272
};
\addplot [, color1, opacity=0.6, mark=square*, mark size=0.5, mark options={solid}, only marks, forget plot]
table {%
1 nan
1.05128205128205 0.715058505535126
1.10897435897436 0.73765105009079
1.16987179487179 0.795297265052795
1.23076923076923 0.877149105072021
1.29807692307692 0.890435218811035
1.36858974358974 0.827289402484894
1.44230769230769 0.851753354072571
1.51923076923077 0.878588497638702
1.6025641025641 0.779712796211243
1.68910256410256 0.764125943183899
1.77884615384615 0.761916100978851
1.875 0.850559413433075
1.9775641025641 0.793905675411224
2.08333333333333 0.793603897094727
2.19551282051282 0.823017299175262
2.31410256410256 0.79414576292038
2.43910256410256 0.772831559181213
2.57051282051282 0.746726155281067
2.70833333333333 0.781720280647278
2.8525641025641 0.708444237709045
3.00641025641026 0.72966867685318
3.16987179487179 0.76403820514679
3.33974358974359 0.771784484386444
3.51923076923077 0.753558456897736
3.70833333333333 0.743211388587952
3.91025641025641 0.788610279560089
4.11858974358974 0.764199256896973
4.34294871794872 0.713898837566376
4.57692307692308 0.722317695617676
4.82371794871795 0.80248749256134
5.08333333333333 0.746250450611115
5.35576923076923 0.727075219154358
5.64423076923077 0.760246753692627
5.94871794871795 0.73393839597702
6.26923076923077 0.718207776546478
6.60576923076923 0.766253411769867
6.96153846153846 0.72587251663208
7.33653846153846 0.693492829799652
7.73397435897436 0.745961010456085
8.15064102564103 0.711142778396606
8.58974358974359 0.707009494304657
9.05128205128205 0.699326813220978
9.53846153846154 0.62743067741394
10.0512820512821 0.674797892570496
10.5929487179487 0.602892637252808
11.1634615384615 0.598228633403778
11.7660256410256 0.642693221569061
12.400641025641 0.619509696960449
13.0673076923077 0.614080607891083
13.7724358974359 0.549847066402435
14.5128205128205 0.534785687923431
15.2948717948718 0.610028028488159
16.1185897435897 0.561874568462372
16.9871794871795 0.596505761146545
17.900641025641 0.549331605434418
18.8653846153846 0.568385779857635
19.8814102564103 0.480713814496994
20.9519230769231 0.517721951007843
22.0801282051282 0.444172382354736
23.2692307692308 0.469442576169968
24.5224358974359 0.477766513824463
25.8429487179487 0.498295038938522
27.2371794871795 0.428196251392365
28.7019230769231 0.50215357542038
30.25 0.429853588342667
31.8782051282051 0.427860587835312
33.5961538461538 0.489433586597443
35.4038461538462 0.459582179784775
37.3108974358974 0.396907180547714
39.3205128205128 0.389359921216965
41.4391025641026 0.428816705942154
43.6698717948718 0.412720680236816
46.0224358974359 0.357857912778854
48.5 0.353063881397247
51.1121794871795 0.377741098403931
53.8653846153846 0.38740810751915
56.7660256410256 0.335617870092392
59.8237179487179 0.330778986215591
63.0448717948718 0.358933687210083
66.4391025641026 0.359967797994614
70.0192307692308 0.327378034591675
73.7884615384615 0.344171226024628
77.7628205128205 0.327998846769333
81.9519230769231 0.370962530374527
86.3653846153846 0.351374357938766
91.0160256410256 0.362572580575943
95.9166666666667 0.345485419034958
101.083333333333 0.3576999604702
106.525641025641 0.303516149520874
112.262820512821 0.33893620967865
118.310897435897 0.311138033866882
124.682692307692 0.337819963693619
131.397435897436 0.314479023218155
138.474358974359 0.327120572328568
145.929487179487 0.342254400253296
153.788461538462 0.34183731675148
162.070512820513 0.309044301509857
170.801282051282 0.313460856676102
180 0.312767326831818
};
\addplot [, color1, opacity=0.6, mark=square*, mark size=0.5, mark options={solid}, only marks, forget plot]
table {%
1 nan
1.05128205128205 0.678979992866516
1.10897435897436 0.598061382770538
1.16987179487179 0.750160157680511
1.23076923076923 0.829689979553223
1.29807692307692 0.8694087266922
1.36858974358974 0.908191323280334
1.44230769230769 0.803253829479218
1.51923076923077 0.861486256122589
1.6025641025641 0.81286758184433
1.68910256410256 0.810449421405792
1.77884615384615 0.796546638011932
1.875 0.809664368629456
1.9775641025641 0.757766425609589
2.08333333333333 0.814541280269623
2.19551282051282 0.750497043132782
2.31410256410256 0.801543831825256
2.43910256410256 0.8270303606987
2.57051282051282 0.750586450099945
2.70833333333333 0.705085337162018
2.8525641025641 0.757934749126434
3.00641025641026 0.788193762302399
3.16987179487179 0.735179781913757
3.33974358974359 0.734420657157898
3.51923076923077 0.750696659088135
3.70833333333333 0.720175683498383
3.91025641025641 0.704468190670013
4.11858974358974 0.739460170269012
4.34294871794872 0.812451779842377
4.57692307692308 0.777057647705078
4.82371794871795 0.729142725467682
5.08333333333333 0.722603499889374
5.35576923076923 0.791086077690125
5.64423076923077 0.830015957355499
5.94871794871795 0.725557446479797
6.26923076923077 0.709764659404755
6.60576923076923 0.795534908771515
6.96153846153846 0.78650438785553
7.33653846153846 0.65797370672226
7.73397435897436 0.723525106906891
8.15064102564103 0.723716855049133
8.58974358974359 0.732065260410309
9.05128205128205 0.669852256774902
9.53846153846154 0.627962231636047
10.0512820512821 0.681299984455109
10.5929487179487 0.586448788642883
11.1634615384615 0.614243924617767
11.7660256410256 0.579944491386414
12.400641025641 0.616940498352051
13.0673076923077 0.630346238613129
13.7724358974359 0.557247340679169
14.5128205128205 0.541055977344513
15.2948717948718 0.52092981338501
16.1185897435897 0.554227769374847
16.9871794871795 0.546736001968384
17.900641025641 0.508904278278351
18.8653846153846 0.571966409683228
19.8814102564103 0.428120464086533
20.9519230769231 0.501048982143402
22.0801282051282 0.44447135925293
23.2692307692308 0.438653767108917
24.5224358974359 0.417769193649292
25.8429487179487 0.453697055578232
27.2371794871795 0.418508261442184
28.7019230769231 0.466319531202316
30.25 0.346658766269684
31.8782051282051 0.405190676450729
33.5961538461538 0.373851865530014
35.4038461538462 0.387342244386673
37.3108974358974 0.314457714557648
39.3205128205128 0.350546091794968
41.4391025641026 0.356291025876999
43.6698717948718 0.35986265540123
46.0224358974359 0.32627734541893
48.5 0.331675916910172
51.1121794871795 0.319852441549301
53.8653846153846 0.428568094968796
56.7660256410256 0.313597649335861
59.8237179487179 0.330168843269348
63.0448717948718 0.32099586725235
66.4391025641026 0.319193422794342
70.0192307692308 0.35311159491539
73.7884615384615 0.361970871686935
77.7628205128205 0.318715304136276
81.9519230769231 0.336603492498398
86.3653846153846 0.313748896121979
91.0160256410256 0.300077110528946
95.9166666666667 0.271962016820908
101.083333333333 0.346042573451996
106.525641025641 0.33043560385704
112.262820512821 0.330071210861206
118.310897435897 0.306172788143158
124.682692307692 0.312175720930099
131.397435897436 0.288712352514267
138.474358974359 0.303651303052902
145.929487179487 0.309968560934067
153.788461538462 0.287993341684341
162.070512820513 0.302125513553619
170.801282051282 0.287991315126419
180 0.293312311172485
};
\addplot [, color1, opacity=0.6, mark=square*, mark size=0.5, mark options={solid}, only marks, forget plot]
table {%
1 nan
1.05128205128205 0.702526926994324
1.10897435897436 0.69260036945343
1.16987179487179 0.840897679328918
1.23076923076923 0.889897346496582
1.29807692307692 0.738979339599609
1.36858974358974 0.888575196266174
1.44230769230769 0.822510659694672
1.51923076923077 0.786208093166351
1.6025641025641 0.746097028255463
1.68910256410256 0.743976056575775
1.77884615384615 0.809284508228302
1.875 0.851681709289551
1.9775641025641 0.803438007831573
2.08333333333333 0.783310949802399
2.19551282051282 0.804171562194824
2.31410256410256 0.780920028686523
2.43910256410256 0.79528135061264
2.57051282051282 0.803177773952484
2.70833333333333 0.686826407909393
2.8525641025641 0.748228669166565
3.00641025641026 0.702254772186279
3.16987179487179 0.721778988838196
3.33974358974359 0.714083671569824
3.51923076923077 0.741500794887543
3.70833333333333 0.728003978729248
3.91025641025641 0.78536981344223
4.11858974358974 0.770494401454926
4.34294871794872 0.763288974761963
4.57692307692308 0.814742028713226
4.82371794871795 0.854504287242889
5.08333333333333 0.796543896198273
5.35576923076923 0.848677098751068
5.64423076923077 0.782149434089661
5.94871794871795 0.728753507137299
6.26923076923077 0.747188746929169
6.60576923076923 0.762465476989746
6.96153846153846 0.740066528320312
7.33653846153846 0.711297929286957
7.73397435897436 0.658814251422882
8.15064102564103 0.686265647411346
8.58974358974359 0.596772849559784
9.05128205128205 0.619234025478363
9.53846153846154 0.654058635234833
10.0512820512821 0.635652124881744
10.5929487179487 0.606460511684418
11.1634615384615 0.593633055686951
11.7660256410256 0.619638741016388
12.400641025641 0.541979849338531
13.0673076923077 0.648710906505585
13.7724358974359 0.5505411028862
14.5128205128205 0.589084565639496
15.2948717948718 0.563029229640961
16.1185897435897 0.610403478145599
16.9871794871795 0.567685544490814
17.900641025641 0.518065810203552
18.8653846153846 0.47593766450882
19.8814102564103 0.523380696773529
20.9519230769231 0.448251456022263
22.0801282051282 0.505264282226562
23.2692307692308 0.509782433509827
24.5224358974359 0.484249323606491
25.8429487179487 0.46286803483963
27.2371794871795 0.48913112282753
28.7019230769231 0.402111351490021
30.25 0.4533711373806
31.8782051282051 0.452061414718628
33.5961538461538 0.395415484905243
35.4038461538462 0.389756262302399
37.3108974358974 0.405727684497833
39.3205128205128 0.346476972103119
41.4391025641026 0.383149117231369
43.6698717948718 0.330205917358398
46.0224358974359 0.336203724145889
48.5 0.345316559076309
51.1121794871795 0.377455592155457
53.8653846153846 0.330737888813019
56.7660256410256 0.35355281829834
59.8237179487179 0.345182597637177
63.0448717948718 0.291344165802002
66.4391025641026 0.33289235830307
70.0192307692308 0.371543645858765
73.7884615384615 0.311225414276123
77.7628205128205 0.296523094177246
81.9519230769231 0.312307924032211
86.3653846153846 0.313464820384979
91.0160256410256 0.318951457738876
95.9166666666667 0.311832576990128
101.083333333333 0.372258871793747
106.525641025641 0.285718441009521
112.262820512821 0.302677124738693
118.310897435897 0.346609473228455
124.682692307692 0.314901679754257
131.397435897436 0.262364357709885
138.474358974359 0.26799550652504
145.929487179487 0.311973631381989
153.788461538462 0.358365952968597
162.070512820513 0.303820937871933
170.801282051282 0.354918330907822
180 0.346484422683716
};
\addplot [, color2, opacity=0.6, mark=triangle*, mark size=0.5, mark options={solid,rotate=180}, only marks]
table {%
1 nan
1.05128205128205 0.512235581874847
1.10897435897436 0.328204065561295
1.16987179487179 0.56669294834137
1.23076923076923 0.676874041557312
1.29807692307692 0.551375031471252
1.36858974358974 0.618783056735992
1.44230769230769 0.540830254554749
1.51923076923077 0.563102722167969
1.6025641025641 0.592476546764374
1.68910256410256 0.514214336872101
1.77884615384615 0.556136310100555
1.875 0.622578144073486
1.9775641025641 0.4980328977108
2.08333333333333 0.508540451526642
2.19551282051282 0.5592041015625
2.31410256410256 0.523995220661163
2.43910256410256 0.528251945972443
2.57051282051282 0.505438029766083
2.70833333333333 0.491891473531723
2.8525641025641 0.530031800270081
3.00641025641026 0.545445144176483
3.16987179487179 0.577899396419525
3.33974358974359 0.548975169658661
3.51923076923077 0.51062798500061
3.70833333333333 0.476559728384018
3.91025641025641 0.520724952220917
4.11858974358974 0.528429090976715
4.34294871794872 0.559863448143005
4.57692307692308 0.551920831203461
4.82371794871795 0.516715586185455
5.08333333333333 0.557731866836548
5.35576923076923 0.511766791343689
5.64423076923077 0.5156210064888
5.94871794871795 0.504034221172333
6.26923076923077 0.510116994380951
6.60576923076923 0.508917629718781
6.96153846153846 0.528914391994476
7.33653846153846 0.570712029933929
7.73397435897436 0.521202385425568
8.15064102564103 0.522550761699677
8.58974358974359 0.517526566982269
9.05128205128205 0.496837943792343
9.53846153846154 0.489141762256622
10.0512820512821 0.530968427658081
10.5929487179487 0.490425258874893
11.1634615384615 0.507598876953125
11.7660256410256 0.486316114664078
12.400641025641 0.49485382437706
13.0673076923077 0.492659240961075
13.7724358974359 0.439174950122833
14.5128205128205 0.437863737344742
15.2948717948718 0.435047447681427
16.1185897435897 0.437795728445053
16.9871794871795 0.487945526838303
17.900641025641 0.434185326099396
18.8653846153846 0.40723380446434
19.8814102564103 0.414389342069626
20.9519230769231 0.425236374139786
22.0801282051282 0.401522725820541
23.2692307692308 0.393123179674149
24.5224358974359 0.405467331409454
25.8429487179487 0.409673899412155
27.2371794871795 0.42350697517395
28.7019230769231 0.432354122400284
30.25 0.386073499917984
31.8782051282051 0.380727708339691
33.5961538461538 0.376226723194122
35.4038461538462 0.395212858915329
37.3108974358974 0.348456531763077
39.3205128205128 0.374437987804413
41.4391025641026 0.369054883718491
43.6698717948718 0.35007444024086
46.0224358974359 0.353916704654694
48.5 0.316228806972504
51.1121794871795 0.338546127080917
53.8653846153846 0.341284096240997
56.7660256410256 0.321070730686188
59.8237179487179 0.332392990589142
63.0448717948718 0.33942911028862
66.4391025641026 0.326149195432663
70.0192307692308 0.346087455749512
73.7884615384615 0.335220128297806
77.7628205128205 0.312150895595551
81.9519230769231 0.342995792627335
86.3653846153846 0.328423470258713
91.0160256410256 0.308387339115143
95.9166666666667 0.329517006874084
101.083333333333 0.333743184804916
106.525641025641 0.337723702192307
112.262820512821 0.301016360521317
118.310897435897 0.309274703264236
124.682692307692 0.303356856107712
131.397435897436 0.281184673309326
138.474358974359 0.302013903856277
145.929487179487 0.337984681129456
153.788461538462 0.358477741479874
162.070512820513 0.324386417865753
170.801282051282 0.320874094963074
180 0.293065756559372
};
\addlegendentry{sub 16, mc 1}
\addplot [, color2, opacity=0.6, mark=triangle*, mark size=0.5, mark options={solid,rotate=180}, only marks, forget plot]
table {%
1 nan
1.05128205128205 0.474460333585739
1.10897435897436 0.434003204107285
1.16987179487179 0.517538487911224
1.23076923076923 0.617085039615631
1.29807692307692 0.570273160934448
1.36858974358974 0.564754009246826
1.44230769230769 0.602635562419891
1.51923076923077 0.609652698040009
1.6025641025641 0.519782960414886
1.68910256410256 0.583593308925629
1.77884615384615 0.447401851415634
1.875 0.588682353496552
1.9775641025641 0.577390015125275
2.08333333333333 0.602071702480316
2.19551282051282 0.547460496425629
2.31410256410256 0.489056587219238
2.43910256410256 0.50332772731781
2.57051282051282 0.562836945056915
2.70833333333333 0.446725845336914
2.8525641025641 0.469514280557632
3.00641025641026 0.61840945482254
3.16987179487179 0.593361973762512
3.33974358974359 0.539616048336029
3.51923076923077 0.491264581680298
3.70833333333333 0.52112090587616
3.91025641025641 0.515533447265625
4.11858974358974 0.476986855268478
4.34294871794872 0.461282938718796
4.57692307692308 0.508595407009125
4.82371794871795 0.493188291788101
5.08333333333333 0.52402126789093
5.35576923076923 0.450193017721176
5.64423076923077 0.503819108009338
5.94871794871795 0.467210978269577
6.26923076923077 0.451473474502563
6.60576923076923 0.513232707977295
6.96153846153846 0.461560010910034
7.33653846153846 0.457257241010666
7.73397435897436 0.504045486450195
8.15064102564103 0.535777747631073
8.58974358974359 0.447187423706055
9.05128205128205 0.436969727277756
9.53846153846154 0.39885887503624
10.0512820512821 0.490199774503708
10.5929487179487 0.466228485107422
11.1634615384615 0.441903203725815
11.7660256410256 0.442770302295685
12.400641025641 0.401748865842819
13.0673076923077 0.450219929218292
13.7724358974359 0.459008902311325
14.5128205128205 0.380776077508926
15.2948717948718 0.406945139169693
16.1185897435897 0.381748825311661
16.9871794871795 0.389110118150711
17.900641025641 0.397396385669708
18.8653846153846 0.41127997636795
19.8814102564103 0.35405021905899
20.9519230769231 0.412860304117203
22.0801282051282 0.331152766942978
23.2692307692308 0.404214590787888
24.5224358974359 0.3753981590271
25.8429487179487 0.378652811050415
27.2371794871795 0.399572223424911
28.7019230769231 0.373655408620834
30.25 0.374074935913086
31.8782051282051 0.353140860795975
33.5961538461538 0.382995128631592
35.4038461538462 0.311019420623779
37.3108974358974 0.320950835943222
39.3205128205128 0.325096994638443
41.4391025641026 0.337791979312897
43.6698717948718 0.363008230924606
46.0224358974359 0.331035256385803
48.5 0.293660402297974
51.1121794871795 0.364647775888443
53.8653846153846 0.323269575834274
56.7660256410256 0.349598258733749
59.8237179487179 0.313090324401855
63.0448717948718 0.319872856140137
66.4391025641026 0.343360990285873
70.0192307692308 0.307066887617111
73.7884615384615 0.31560879945755
77.7628205128205 0.325261443853378
81.9519230769231 0.317315369844437
86.3653846153846 0.30568191409111
91.0160256410256 0.294435501098633
95.9166666666667 0.358574569225311
101.083333333333 0.32953092455864
106.525641025641 0.299241036176682
112.262820512821 0.280990689992905
118.310897435897 0.314328968524933
124.682692307692 0.324493318796158
131.397435897436 0.318858474493027
138.474358974359 0.300254106521606
145.929487179487 0.330261528491974
153.788461538462 0.343896239995956
162.070512820513 0.324778467416763
170.801282051282 0.286484152078629
180 0.321863949298859
};
\addplot [, color2, opacity=0.6, mark=triangle*, mark size=0.5, mark options={solid,rotate=180}, only marks, forget plot]
table {%
1 nan
1.05128205128205 0.480017751455307
1.10897435897436 0.418718904256821
1.16987179487179 0.449393600225449
1.23076923076923 0.571604788303375
1.29807692307692 0.542007625102997
1.36858974358974 0.585287868976593
1.44230769230769 0.571768701076508
1.51923076923077 0.621403813362122
1.6025641025641 0.519449889659882
1.68910256410256 0.573084950447083
1.77884615384615 0.578590214252472
1.875 0.52923047542572
1.9775641025641 0.563537061214447
2.08333333333333 0.429549038410187
2.19551282051282 0.49914875626564
2.31410256410256 0.582490682601929
2.43910256410256 0.551305413246155
2.57051282051282 0.454692423343658
2.70833333333333 0.469948142766953
2.8525641025641 0.523184716701508
3.00641025641026 0.49006462097168
3.16987179487179 0.488831907510757
3.33974358974359 0.579599797725677
3.51923076923077 0.489268124103546
3.70833333333333 0.526870429515839
3.91025641025641 0.561821401119232
4.11858974358974 0.440863907337189
4.34294871794872 0.523884892463684
4.57692307692308 0.483162254095078
4.82371794871795 0.537889540195465
5.08333333333333 0.543849647045135
5.35576923076923 0.461601406335831
5.64423076923077 0.501249969005585
5.94871794871795 0.517584383487701
6.26923076923077 0.441641390323639
6.60576923076923 0.520270764827728
6.96153846153846 0.500085771083832
7.33653846153846 0.425243675708771
7.73397435897436 0.473490059375763
8.15064102564103 0.468789488077164
8.58974358974359 0.449180603027344
9.05128205128205 0.450269043445587
9.53846153846154 0.449345022439957
10.0512820512821 0.431020885705948
10.5929487179487 0.452156126499176
11.1634615384615 0.400523006916046
11.7660256410256 0.403245985507965
12.400641025641 0.421239137649536
13.0673076923077 0.453531175851822
13.7724358974359 0.357198297977448
14.5128205128205 0.349831491708755
15.2948717948718 0.382864892482758
16.1185897435897 0.400588899850845
16.9871794871795 0.403753757476807
17.900641025641 0.353908449411392
18.8653846153846 0.384092897176743
19.8814102564103 0.371576637029648
20.9519230769231 0.356100678443909
22.0801282051282 0.33666929602623
23.2692307692308 0.391628831624985
24.5224358974359 0.342908680438995
25.8429487179487 0.365526735782623
27.2371794871795 0.356520920991898
28.7019230769231 0.344386458396912
30.25 0.35532483458519
31.8782051282051 0.345143884420395
33.5961538461538 0.305817753076553
35.4038461538462 0.307193249464035
37.3108974358974 0.303843826055527
39.3205128205128 0.302421897649765
41.4391025641026 0.360422253608704
43.6698717948718 0.302796572446823
46.0224358974359 0.283685028553009
48.5 0.323235094547272
51.1121794871795 0.313028335571289
53.8653846153846 0.286403000354767
56.7660256410256 0.310298264026642
59.8237179487179 0.305169552564621
63.0448717948718 0.292225986719131
66.4391025641026 0.279966026544571
70.0192307692308 0.243739753961563
73.7884615384615 0.287752121686935
77.7628205128205 0.296166300773621
81.9519230769231 0.25813016295433
86.3653846153846 0.272258222103119
91.0160256410256 0.253672271966934
95.9166666666667 0.276905924081802
101.083333333333 0.275909721851349
106.525641025641 0.281722486019135
112.262820512821 0.282397478818893
118.310897435897 0.241657644510269
124.682692307692 0.290190905332565
131.397435897436 0.290309965610504
138.474358974359 0.280518770217896
145.929487179487 0.28484895825386
153.788461538462 0.279284328222275
162.070512820513 0.266681492328644
170.801282051282 0.296066582202911
180 0.277500718832016
};
\addplot [, color2, opacity=0.6, mark=triangle*, mark size=0.5, mark options={solid,rotate=180}, only marks, forget plot]
table {%
1 nan
1.05128205128205 0.454888910055161
1.10897435897436 0.371243298053741
1.16987179487179 0.571331202983856
1.23076923076923 0.66328901052475
1.29807692307692 0.571244657039642
1.36858974358974 0.567432880401611
1.44230769230769 0.596572041511536
1.51923076923077 0.586832940578461
1.6025641025641 0.573674738407135
1.68910256410256 0.547989547252655
1.77884615384615 0.552707970142365
1.875 0.561032712459564
1.9775641025641 0.536881148815155
2.08333333333333 0.539679110050201
2.19551282051282 0.53624564409256
2.31410256410256 0.527253866195679
2.43910256410256 0.488604545593262
2.57051282051282 0.574143409729004
2.70833333333333 0.493412971496582
2.8525641025641 0.528299331665039
3.00641025641026 0.546260893344879
3.16987179487179 0.56116259098053
3.33974358974359 0.501424014568329
3.51923076923077 0.550004482269287
3.70833333333333 0.522311806678772
3.91025641025641 0.525552272796631
4.11858974358974 0.545074999332428
4.34294871794872 0.525904297828674
4.57692307692308 0.531585991382599
4.82371794871795 0.540511727333069
5.08333333333333 0.472627550363541
5.35576923076923 0.454450905323029
5.64423076923077 0.480747312307358
5.94871794871795 0.444818407297134
6.26923076923077 0.436434119939804
6.60576923076923 0.478800535202026
6.96153846153846 0.452978819608688
7.33653846153846 0.473083317279816
7.73397435897436 0.46679949760437
8.15064102564103 0.443266957998276
8.58974358974359 0.453537702560425
9.05128205128205 0.361685276031494
9.53846153846154 0.416784584522247
10.0512820512821 0.42499190568924
10.5929487179487 0.440760284662247
11.1634615384615 0.405608177185059
11.7660256410256 0.364872008562088
12.400641025641 0.395486414432526
13.0673076923077 0.379319578409195
13.7724358974359 0.410384565591812
14.5128205128205 0.374533265829086
15.2948717948718 0.360618472099304
16.1185897435897 0.359120666980743
16.9871794871795 0.355149477720261
17.900641025641 0.336675077676773
18.8653846153846 0.362640559673309
19.8814102564103 0.377529054880142
20.9519230769231 0.389113038778305
22.0801282051282 0.358796775341034
23.2692307692308 0.302441209554672
24.5224358974359 0.353474825620651
25.8429487179487 0.34322664141655
27.2371794871795 0.349585711956024
28.7019230769231 0.324121445417404
30.25 0.307719528675079
31.8782051282051 0.337610453367233
33.5961538461538 0.301304906606674
35.4038461538462 0.344807922840118
37.3108974358974 0.299825519323349
39.3205128205128 0.305433720350266
41.4391025641026 0.354649156332016
43.6698717948718 0.316947847604752
46.0224358974359 0.295400828123093
48.5 0.306247442960739
51.1121794871795 0.301986455917358
53.8653846153846 0.286987274885178
56.7660256410256 0.298087686300278
59.8237179487179 0.305031418800354
63.0448717948718 0.30724310874939
66.4391025641026 0.288002967834473
70.0192307692308 0.294104874134064
73.7884615384615 0.302816957235336
77.7628205128205 0.330823391675949
81.9519230769231 0.262787878513336
86.3653846153846 0.290551751852036
91.0160256410256 0.288363069295883
95.9166666666667 0.279569059610367
101.083333333333 0.251280128955841
106.525641025641 0.307267338037491
112.262820512821 0.327542662620544
118.310897435897 0.263447195291519
124.682692307692 0.277644962072372
131.397435897436 0.252197027206421
138.474358974359 0.274540573358536
145.929487179487 0.289973139762878
153.788461538462 0.279979199171066
162.070512820513 0.299552738666534
170.801282051282 0.259907305240631
180 0.272861808538437
};
\addplot [, color2, opacity=0.6, mark=triangle*, mark size=0.5, mark options={solid,rotate=180}, only marks, forget plot]
table {%
1 nan
1.05128205128205 0.437121599912643
1.10897435897436 0.417015552520752
1.16987179487179 0.599098920822144
1.23076923076923 0.773574650287628
1.29807692307692 0.605981171131134
1.36858974358974 0.632821440696716
1.44230769230769 0.625268578529358
1.51923076923077 0.617282629013062
1.6025641025641 0.555473506450653
1.68910256410256 0.508972823619843
1.77884615384615 0.536830604076385
1.875 0.598303496837616
1.9775641025641 0.584812879562378
2.08333333333333 0.556249618530273
2.19551282051282 0.503579795360565
2.31410256410256 0.546276152133942
2.43910256410256 0.553644299507141
2.57051282051282 0.496064513921738
2.70833333333333 0.515953958034515
2.8525641025641 0.45500835776329
3.00641025641026 0.4748175740242
3.16987179487179 0.5217325091362
3.33974358974359 0.51848977804184
3.51923076923077 0.503091931343079
3.70833333333333 0.529949128627777
3.91025641025641 0.519807994365692
4.11858974358974 0.484880745410919
4.34294871794872 0.530423760414124
4.57692307692308 0.462502390146255
4.82371794871795 0.53246396780014
5.08333333333333 0.548243463039398
5.35576923076923 0.475721031427383
5.64423076923077 0.521773397922516
5.94871794871795 0.491677135229111
6.26923076923077 0.506716907024384
6.60576923076923 0.474774748086929
6.96153846153846 0.435458481311798
7.33653846153846 0.481690019369125
7.73397435897436 0.52952253818512
8.15064102564103 0.482205241918564
8.58974358974359 0.450421541929245
9.05128205128205 0.432707875967026
9.53846153846154 0.476216226816177
10.0512820512821 0.46561986207962
10.5929487179487 0.455663591623306
11.1634615384615 0.435157507658005
11.7660256410256 0.430080235004425
12.400641025641 0.42613497376442
13.0673076923077 0.426486790180206
13.7724358974359 0.406038522720337
14.5128205128205 0.423387140035629
15.2948717948718 0.412570476531982
16.1185897435897 0.44126358628273
16.9871794871795 0.426257222890854
17.900641025641 0.39659121632576
18.8653846153846 0.441071957349777
19.8814102564103 0.373128175735474
20.9519230769231 0.411204814910889
22.0801282051282 0.401075571775436
23.2692307692308 0.347843110561371
24.5224358974359 0.384564310312271
25.8429487179487 0.340942472219467
27.2371794871795 0.393309265375137
28.7019230769231 0.344469428062439
30.25 0.336054354906082
31.8782051282051 0.340553820133209
33.5961538461538 0.355754286050797
35.4038461538462 0.375609397888184
37.3108974358974 0.35467728972435
39.3205128205128 0.354572743177414
41.4391025641026 0.34597173333168
43.6698717948718 0.346892207860947
46.0224358974359 0.270318865776062
48.5 0.3020920753479
51.1121794871795 0.350495427846909
53.8653846153846 0.334720224142075
56.7660256410256 0.315347284078598
59.8237179487179 0.312505781650543
63.0448717948718 0.318619340658188
66.4391025641026 0.309555262327194
70.0192307692308 0.28867495059967
73.7884615384615 0.317645162343979
77.7628205128205 0.30613848567009
81.9519230769231 0.30616757273674
86.3653846153846 0.288762629032135
91.0160256410256 0.306071192026138
95.9166666666667 0.304188430309296
101.083333333333 0.301222890615463
106.525641025641 0.329778522253036
112.262820512821 0.307115077972412
118.310897435897 0.288012266159058
124.682692307692 0.310747653245926
131.397435897436 0.28313159942627
138.474358974359 0.289871990680695
145.929487179487 0.270382404327393
153.788461538462 0.2870973944664
162.070512820513 0.279968827962875
170.801282051282 0.274062007665634
180 0.257966130971909
};
\end{axis}

\end{tikzpicture}

      \tikzexternaldisable
    \end{minipage}
  \end{subfigure}

  \begin{subfigure}[t]{\linewidth}
    \centering
    \caption{\cifarten \resnetthirtytwo \adam}
    \begin{minipage}{0.50\linewidth}
      \centering
      % defines the pgfplots style "eigspacedefault"
\pgfkeys{/pgfplots/eigspacedefault/.style={
    width=1.0\linewidth,
    height=0.6\linewidth,
    every axis plot/.append style={line width = 1.5pt},
    tick pos = left,
    ylabel near ticks,
    xlabel near ticks,
    xtick align = inside,
    ytick align = inside,
    legend cell align = left,
    legend columns = 4,
    legend pos = south east,
    legend style = {
      fill opacity = 1,
      text opacity = 1,
      font = \footnotesize,
      at={(1, 1.025)},
      anchor=south east,
      column sep=0.25cm,
    },
    legend image post style={scale=2.5},
    xticklabel style = {font = \footnotesize},
    xlabel style = {font = \footnotesize},
    axis line style = {black},
    yticklabel style = {font = \footnotesize},
    ylabel style = {font = \footnotesize},
    title style = {font = \footnotesize},
    grid = major,
    grid style = {dashed}
  }
}

\pgfkeys{/pgfplots/eigspacedefaultapp/.style={
    eigspacedefault,
    height=0.6\linewidth,
    legend columns = 2,
  }
}

\pgfkeys{/pgfplots/eigspacenolegend/.style={
    legend image post style = {scale=0},
    legend style = {
      fill opacity = 0,
      draw opacity = 0,
      text opacity = 0,
      font = \footnotesize,
      at={(1, 1.025)},
      anchor=south east,
      column sep=0.25cm,
    },
  }
}
%%% Local Variables:
%%% mode: latex
%%% TeX-master: "../../thesis"
%%% End:

      \pgfkeys{/pgfplots/zmystyle/.style={
          eigspacedefaultapp,
          eigspacenolegend,
        }}
      \tikzexternalenable
      \vspace{-6ex}
      % This file was created by tikzplotlib v0.9.7.
\begin{tikzpicture}

\definecolor{color0}{rgb}{0.501960784313725,0.184313725490196,0.6}
\definecolor{color1}{rgb}{0.870588235294118,0.623529411764706,0.0862745098039216}
\definecolor{color2}{rgb}{0.274509803921569,0.6,0.564705882352941}

\begin{axis}[
axis line style={white!10!black},
legend columns=2,
legend style={fill opacity=0.8, draw opacity=1, text opacity=1, at={(0.03,0.03)}, anchor=south west, draw=white!80!black},
log basis x={10},
tick pos=left,
xlabel={epoch (log scale)},
xmajorgrids,
xmin=0.794328234724281, xmax=125.892541179417,
xmode=log,
ylabel={overlap},
ymajorgrids,
ymin=-0.05, ymax=1.05,
zmystyle
]
\addplot [, white!10!black, dashed, forget plot]
table {%
0.794328234724281 1
125.892541179417 1
};
\addplot [, white!10!black, dashed, forget plot]
table {%
0.794328234724281 0
125.892541179417 0
};
\addplot [, color0, opacity=0.6, mark=triangle*, mark size=0.5, mark options={solid,rotate=180}, only marks]
table {%
1 0.592307209968567
1.04487179487179 0.546835482120514
1.09615384615385 0.603555679321289
1.1474358974359 0.598651230335236
1.20192307692308 0.552099347114563
1.25961538461538 0.594812452793121
1.32051282051282 0.464624881744385
1.38461538461538 0.541784167289734
1.44871794871795 0.517004311084747
1.51923076923077 0.463415771722794
1.58974358974359 0.486117750406265
1.66666666666667 0.472998708486557
1.74679487179487 0.482929140329361
1.83012820512821 0.441577523946762
1.91666666666667 0.463936388492584
2.00641025641026 0.419153779745102
2.1025641025641 0.427711725234985
2.20512820512821 0.448628157377243
2.30769230769231 0.416244477033615
2.41987179487179 0.384994357824326
2.53525641025641 0.404327005147934
2.65384615384615 0.341067999601364
2.78205128205128 0.399928867816925
2.91346153846154 0.367521464824677
3.05128205128205 0.351166844367981
3.19871794871795 0.348866283893585
3.34935897435897 0.357875496149063
3.50961538461538 0.353610426187515
3.67628205128205 0.347389757633209
3.8525641025641 0.350051134824753
4.03525641025641 0.319423377513885
4.2275641025641 0.337003320455551
4.42948717948718 0.309750378131866
4.64102564102564 0.311457186937332
4.86217948717949 0.30822429060936
5.09294871794872 0.302430063486099
5.33653846153846 0.287228554487228
5.58974358974359 0.295301586389542
5.85576923076923 0.288435041904449
6.13461538461539 0.297879159450531
6.42628205128205 0.291455805301666
6.73397435897436 0.293273776769638
7.05448717948718 0.276805967092514
7.38782051282051 0.282050400972366
7.74038461538461 0.27923846244812
8.10897435897436 0.257671684026718
8.49679487179487 0.271702021360397
8.90064102564103 0.259701102972031
9.32371794871795 0.26940530538559
9.76923076923077 0.274723589420319
10.2339743589744 0.271373957395554
10.7211538461538 0.277475923299789
11.2307692307692 0.263631880283356
11.7660256410256 0.277493000030518
12.3269230769231 0.2483199685812
12.9134615384615 0.251137882471085
13.5288461538462 0.220538377761841
14.1730769230769 0.236719653010368
14.849358974359 0.218988224864006
15.5544871794872 0.253796190023422
16.2948717948718 0.249460130929947
17.0705128205128 0.250696420669556
17.8846153846154 0.259507358074188
18.7371794871795 0.210837155580521
19.6282051282051 0.206601023674011
20.5641025641026 0.220209822058678
21.5416666666667 0.224931001663208
22.5673076923077 0.211947128176689
23.6442307692308 0.215814828872681
24.7692307692308 0.204044133424759
25.9487179487179 0.207341432571411
27.1826923076923 0.192573472857475
28.4775641025641 0.221555098891258
29.8333333333333 0.205034300684929
31.2564102564103 0.181921154260635
32.7435897435897 0.189766556024551
34.3044871794872 0.175309851765633
35.9358974358974 0.187651917338371
37.6474358974359 0.191715285181999
39.4391025641026 0.170455917716026
41.3173076923077 0.186214789748192
43.2852564102564 0.183850541710854
45.3461538461538 0.179818972945213
47.5064102564103 0.176553323864937
49.7692307692308 0.166018590331078
52.1378205128205 0.191725924611092
54.6217948717949 0.164068296551704
57.2211538461538 0.161347240209579
59.9455128205128 0.159445092082024
62.8012820512821 0.156176343560219
65.7916666666667 0.179518550634384
68.9230769230769 0.171437412500381
72.2051282051282 0.159951567649841
75.6442307692308 0.145247757434845
79.2467948717949 0.145979598164558
83.0192307692308 0.142506837844849
86.974358974359 0.167011812329292
91.1153846153846 0.160471752285957
95.4519230769231 0.118968799710274
100 0.127207607030869
};
\addlegendentry{mb 2, exact}
\addplot [, color0, opacity=0.6, mark=triangle*, mark size=0.5, mark options={solid,rotate=180}, only marks, forget plot]
table {%
1 0.585482776165009
1.04487179487179 0.642820537090302
1.09615384615385 0.737456917762756
1.1474358974359 0.703723013401031
1.20192307692308 0.659144341945648
1.25961538461538 0.624452710151672
1.32051282051282 0.512674272060394
1.38461538461538 0.58567488193512
1.44871794871795 0.573139607906342
1.51923076923077 0.542164325714111
1.58974358974359 0.590914249420166
1.66666666666667 0.535554945468903
1.74679487179487 0.548782050609589
1.83012820512821 0.508130073547363
1.91666666666667 0.509879767894745
2.00641025641026 0.499742567539215
2.1025641025641 0.498545736074448
2.20512820512821 0.517250716686249
2.30769230769231 0.490232080221176
2.41987179487179 0.453849047422409
2.53525641025641 0.486332803964615
2.65384615384615 0.456932753324509
2.78205128205128 0.523271560668945
2.91346153846154 0.473270416259766
3.05128205128205 0.450023949146271
3.19871794871795 0.438610702753067
3.34935897435897 0.440023511648178
3.50961538461538 0.449272453784943
3.67628205128205 0.410219579935074
3.8525641025641 0.422479212284088
4.03525641025641 0.429505825042725
4.2275641025641 0.428436905145645
4.42948717948718 0.425813645124435
4.64102564102564 0.396275013685226
4.86217948717949 0.417201995849609
5.09294871794872 0.414662599563599
5.33653846153846 0.44944429397583
5.58974358974359 0.415950924158096
5.85576923076923 0.393380492925644
6.13461538461539 0.404739588499069
6.42628205128205 0.434069454669952
6.73397435897436 0.431607455015182
7.05448717948718 0.432514011859894
7.38782051282051 0.447400808334351
7.74038461538461 0.415784686803818
8.10897435897436 0.423454433679581
8.49679487179487 0.395981788635254
8.90064102564103 0.439824670553207
9.32371794871795 0.396825343370438
9.76923076923077 0.418368339538574
10.2339743589744 0.429487377405167
10.7211538461538 0.405466556549072
11.2307692307692 0.403895378112793
11.7660256410256 0.389219135046005
12.3269230769231 0.412729173898697
12.9134615384615 0.396542400121689
13.5288461538462 0.377960354089737
14.1730769230769 0.365768760442734
14.849358974359 0.36883607506752
15.5544871794872 0.377174288034439
16.2948717948718 0.376384526491165
17.0705128205128 0.375020951032639
17.8846153846154 0.375096052885056
18.7371794871795 0.368546962738037
19.6282051282051 0.347338914871216
20.5641025641026 0.353253185749054
21.5416666666667 0.31588938832283
22.5673076923077 0.345776796340942
23.6442307692308 0.356596559286118
24.7692307692308 0.315961897373199
25.9487179487179 0.320177555084229
27.1826923076923 0.345033973455429
28.4775641025641 0.329159826040268
29.8333333333333 0.307424038648605
31.2564102564103 0.315818428993225
32.7435897435897 0.34919410943985
34.3044871794872 0.347409009933472
35.9358974358974 0.289627909660339
37.6474358974359 0.320884436368942
39.4391025641026 0.358908504247665
41.3173076923077 0.371110260486603
43.2852564102564 0.345608711242676
45.3461538461538 0.346660017967224
47.5064102564103 0.333024233579636
49.7692307692308 0.339248389005661
52.1378205128205 0.34049266576767
54.6217948717949 0.268183052539825
57.2211538461538 0.351125538349152
59.9455128205128 0.334892570972443
62.8012820512821 0.326612502336502
65.7916666666667 0.290396451950073
68.9230769230769 0.292425721883774
72.2051282051282 0.29774335026741
75.6442307692308 0.294426590204239
79.2467948717949 0.294831722974777
83.0192307692308 0.317448437213898
86.974358974359 0.308747172355652
91.1153846153846 0.284468680620193
95.4519230769231 0.277747005224228
100 0.272795349359512
};
\addplot [, color0, opacity=0.6, mark=triangle*, mark size=0.5, mark options={solid,rotate=180}, only marks, forget plot]
table {%
1 0.502753794193268
1.04487179487179 0.489509582519531
1.09615384615385 0.533672332763672
1.1474358974359 0.497904062271118
1.20192307692308 0.495550841093063
1.25961538461538 0.488884180784225
1.32051282051282 0.442734003067017
1.38461538461538 0.463622242212296
1.44871794871795 0.401593655347824
1.51923076923077 0.388241201639175
1.58974358974359 0.422877222299576
1.66666666666667 0.40520578622818
1.74679487179487 0.388937205076218
1.83012820512821 0.374259740114212
1.91666666666667 0.351854026317596
2.00641025641026 0.354837030172348
2.1025641025641 0.36358967423439
2.20512820512821 0.312477201223373
2.30769230769231 0.303010433912277
2.41987179487179 0.317272245883942
2.53525641025641 0.326904624700546
2.65384615384615 0.306135922670364
2.78205128205128 0.306288063526154
2.91346153846154 0.317791074514389
3.05128205128205 0.300774574279785
3.19871794871795 0.297990441322327
3.34935897435897 0.284591346979141
3.50961538461538 0.272943556308746
3.67628205128205 0.266404122114182
3.8525641025641 0.267289847135544
4.03525641025641 0.253162771463394
4.2275641025641 0.241636142134666
4.42948717948718 0.267525851726532
4.64102564102564 0.232459455728531
4.86217948717949 0.253876715898514
5.09294871794872 0.259975969791412
5.33653846153846 0.217476710677147
5.58974358974359 0.24880413711071
5.85576923076923 0.24929404258728
6.13461538461539 0.254732519388199
6.42628205128205 0.236306115984917
6.73397435897436 0.253792762756348
7.05448717948718 0.230030134320259
7.38782051282051 0.246845826506615
7.74038461538461 0.247952803969383
8.10897435897436 0.251498281955719
8.49679487179487 0.228025481104851
8.90064102564103 0.254550039768219
9.32371794871795 0.235579594969749
9.76923076923077 0.234713315963745
10.2339743589744 0.225211337208748
10.7211538461538 0.232245117425919
11.2307692307692 0.216675281524658
11.7660256410256 0.247531533241272
12.3269230769231 0.238133653998375
12.9134615384615 0.21694914996624
13.5288461538462 0.217873960733414
14.1730769230769 0.206582620739937
14.849358974359 0.225143864750862
15.5544871794872 0.240500882267952
16.2948717948718 0.233887299895287
17.0705128205128 0.261169612407684
17.8846153846154 0.174442604184151
18.7371794871795 0.207801461219788
19.6282051282051 0.240677699446678
20.5641025641026 0.21805964410305
21.5416666666667 0.237472414970398
22.5673076923077 0.233217343688011
23.6442307692308 0.238422617316246
24.7692307692308 0.226135089993477
25.9487179487179 0.229801654815674
27.1826923076923 0.216780588030815
28.4775641025641 0.221086263656616
29.8333333333333 0.2468171864748
31.2564102564103 0.260253548622131
32.7435897435897 0.252157956361771
34.3044871794872 0.19179430603981
35.9358974358974 0.203749522566795
37.6474358974359 0.197102665901184
39.4391025641026 0.225069284439087
41.3173076923077 0.212511539459229
43.2852564102564 0.213074713945389
45.3461538461538 0.209655001759529
47.5064102564103 0.20054192841053
49.7692307692308 0.225561618804932
52.1378205128205 0.194253131747246
54.6217948717949 0.198165774345398
57.2211538461538 0.212529584765434
59.9455128205128 0.227554872632027
62.8012820512821 0.229551315307617
65.7916666666667 0.211195811629295
68.9230769230769 0.20320038497448
72.2051282051282 0.220923289656639
75.6442307692308 0.184507206082344
79.2467948717949 0.216113522648811
83.0192307692308 0.212736949324608
86.974358974359 0.235083058476448
91.1153846153846 0.252482503652573
95.4519230769231 0.239584118127823
100 0.171486154198647
};
\addplot [, color0, opacity=0.6, mark=triangle*, mark size=0.5, mark options={solid,rotate=180}, only marks, forget plot]
table {%
1 0.57866096496582
1.04487179487179 0.676907181739807
1.09615384615385 0.66103857755661
1.1474358974359 0.613086640834808
1.20192307692308 0.646507441997528
1.25961538461538 0.600063323974609
1.32051282051282 0.530238389968872
1.38461538461538 0.594143986701965
1.44871794871795 0.527702271938324
1.51923076923077 0.467513561248779
1.58974358974359 0.519423186779022
1.66666666666667 0.478883475065231
1.74679487179487 0.516321837902069
1.83012820512821 0.454561084508896
1.91666666666667 0.463756650686264
2.00641025641026 0.458988279104233
2.1025641025641 0.441328257322311
2.20512820512821 0.441762298345566
2.30769230769231 0.410736948251724
2.41987179487179 0.42197060585022
2.53525641025641 0.4247986972332
2.65384615384615 0.407556265592575
2.78205128205128 0.408221930265427
2.91346153846154 0.398326724767685
3.05128205128205 0.385015189647675
3.19871794871795 0.403363198041916
3.34935897435897 0.416102796792984
3.50961538461538 0.387670904397964
3.67628205128205 0.397609114646912
3.8525641025641 0.395324558019638
4.03525641025641 0.397075176239014
4.2275641025641 0.369644612073898
4.42948717948718 0.382799923419952
4.64102564102564 0.377935349941254
4.86217948717949 0.383544087409973
5.09294871794872 0.38995549082756
5.33653846153846 0.383177846670151
5.58974358974359 0.375000923871994
5.85576923076923 0.358095198869705
6.13461538461539 0.372890800237656
6.42628205128205 0.387697488069534
6.73397435897436 0.360351651906967
7.05448717948718 0.369450837373734
7.38782051282051 0.314811795949936
7.74038461538461 0.325424253940582
8.10897435897436 0.327002912759781
8.49679487179487 0.312997251749039
8.90064102564103 0.329159170389175
9.32371794871795 0.31680753827095
9.76923076923077 0.32023224234581
10.2339743589744 0.334200203418732
10.7211538461538 0.294420212507248
11.2307692307692 0.30793172121048
11.7660256410256 0.282094895839691
12.3269230769231 0.300957351922989
12.9134615384615 0.269127458333969
13.5288461538462 0.270617634057999
14.1730769230769 0.243981033563614
14.849358974359 0.252620220184326
15.5544871794872 0.25038480758667
16.2948717948718 0.256727129220963
17.0705128205128 0.263109624385834
17.8846153846154 0.276291131973267
18.7371794871795 0.223415330052376
19.6282051282051 0.244529083371162
20.5641025641026 0.262406677007675
21.5416666666667 0.290581375360489
22.5673076923077 0.295093774795532
23.6442307692308 0.221500709652901
24.7692307692308 0.240476086735725
25.9487179487179 0.237801656126976
27.1826923076923 0.243841007351875
28.4775641025641 0.258962541818619
29.8333333333333 0.23039798438549
31.2564102564103 0.200932174921036
32.7435897435897 0.225850105285645
34.3044871794872 0.20632591843605
35.9358974358974 0.228188320994377
37.6474358974359 0.220425948500633
39.4391025641026 0.200400859117508
41.3173076923077 0.222618132829666
43.2852564102564 0.184313341975212
45.3461538461538 0.181449458003044
47.5064102564103 0.20966449379921
49.7692307692308 0.201858326792717
52.1378205128205 0.19481460750103
54.6217948717949 0.203439623117447
57.2211538461538 0.191560849547386
59.9455128205128 0.191939070820808
62.8012820512821 0.188076540827751
65.7916666666667 0.167691454291344
68.9230769230769 0.19236995279789
72.2051282051282 0.171185702085495
75.6442307692308 0.166320368647575
79.2467948717949 0.171476975083351
83.0192307692308 0.167338952422142
86.974358974359 0.167468458414078
91.1153846153846 0.183997794985771
95.4519230769231 0.1688362210989
100 0.181777387857437
};
\addplot [, color0, opacity=0.6, mark=triangle*, mark size=0.5, mark options={solid,rotate=180}, only marks, forget plot]
table {%
1 0.573790490627289
1.04487179487179 0.631611168384552
1.09615384615385 0.674712359905243
1.1474358974359 0.662199676036835
1.20192307692308 0.661509990692139
1.25961538461538 0.618332922458649
1.32051282051282 0.542199194431305
1.38461538461538 0.626749217510223
1.44871794871795 0.56579601764679
1.51923076923077 0.491115391254425
1.58974358974359 0.55174320936203
1.66666666666667 0.508000075817108
1.74679487179487 0.532376170158386
1.83012820512821 0.477628946304321
1.91666666666667 0.480161011219025
2.00641025641026 0.474416941404343
2.1025641025641 0.46269017457962
2.20512820512821 0.461393177509308
2.30769230769231 0.420006722211838
2.41987179487179 0.466852813959122
2.53525641025641 0.462906748056412
2.65384615384615 0.434481918811798
2.78205128205128 0.434114694595337
2.91346153846154 0.440793335437775
3.05128205128205 0.387094587087631
3.19871794871795 0.411156088113785
3.34935897435897 0.406082630157471
3.50961538461538 0.387847155332565
3.67628205128205 0.357632160186768
3.8525641025641 0.343068480491638
4.03525641025641 0.3526291847229
4.2275641025641 0.33409383893013
4.42948717948718 0.33807572722435
4.64102564102564 0.299622297286987
4.86217948717949 0.334631592035294
5.09294871794872 0.332187294960022
5.33653846153846 0.313861310482025
5.58974358974359 0.31063038110733
5.85576923076923 0.30160865187645
6.13461538461539 0.353870958089828
6.42628205128205 0.288804739713669
6.73397435897436 0.348492741584778
7.05448717948718 0.296383887529373
7.38782051282051 0.3315050303936
7.74038461538461 0.329625159502029
8.10897435897436 0.282636851072311
8.49679487179487 0.336550265550613
8.90064102564103 0.301706373691559
9.32371794871795 0.344898223876953
9.76923076923077 0.326054573059082
10.2339743589744 0.267272859811783
10.7211538461538 0.304627150297165
11.2307692307692 0.312139838933945
11.7660256410256 0.297453254461288
12.3269230769231 0.264882892370224
12.9134615384615 0.267457962036133
13.5288461538462 0.29574978351593
14.1730769230769 0.261932522058487
14.849358974359 0.277858376502991
15.5544871794872 0.247788816690445
16.2948717948718 0.251994550228119
17.0705128205128 0.260666489601135
17.8846153846154 0.249128490686417
18.7371794871795 0.237147927284241
19.6282051282051 0.255518108606339
20.5641025641026 0.234200105071068
21.5416666666667 0.237606689333916
22.5673076923077 0.240628153085709
23.6442307692308 0.245363473892212
24.7692307692308 0.240520715713501
25.9487179487179 0.196251600980759
27.1826923076923 0.227656751871109
28.4775641025641 0.2418542355299
29.8333333333333 0.240934997797012
31.2564102564103 0.226873978972435
32.7435897435897 0.246357545256615
34.3044871794872 0.198945805430412
35.9358974358974 0.211172565817833
37.6474358974359 0.191067695617676
39.4391025641026 0.200824543833733
41.3173076923077 0.208135411143303
43.2852564102564 0.17915952205658
45.3461538461538 0.195072337985039
47.5064102564103 0.221192955970764
49.7692307692308 0.220792457461357
52.1378205128205 0.184967175126076
54.6217948717949 0.205503970384598
57.2211538461538 0.197166845202446
59.9455128205128 0.180766612291336
62.8012820512821 0.199647411704063
65.7916666666667 0.211226090788841
68.9230769230769 0.210257574915886
72.2051282051282 0.19671343266964
75.6442307692308 0.20431461930275
79.2467948717949 0.196685194969177
83.0192307692308 0.208932682871819
86.974358974359 0.205271273851395
91.1153846153846 0.188923090696335
95.4519230769231 0.185675904154778
100 0.171028003096581
};
\addplot [, color1, opacity=0.6, mark=square*, mark size=0.5, mark options={solid}, only marks]
table {%
1 0.775229752063751
1.04487179487179 0.834898114204407
1.09615384615385 0.861268818378448
1.1474358974359 0.802887916564941
1.20192307692308 0.786221385002136
1.25961538461538 0.688303172588348
1.32051282051282 0.674230217933655
1.38461538461538 0.697616338729858
1.44871794871795 0.689489006996155
1.51923076923077 0.6154465675354
1.58974358974359 0.674067914485931
1.66666666666667 0.593099772930145
1.74679487179487 0.593142926692963
1.83012820512821 0.585586965084076
1.91666666666667 0.580171287059784
2.00641025641026 0.566816091537476
2.1025641025641 0.596588909626007
2.20512820512821 0.538447856903076
2.30769230769231 0.52722692489624
2.41987179487179 0.543999195098877
2.53525641025641 0.54193514585495
2.65384615384615 0.516180813312531
2.78205128205128 0.515158653259277
2.91346153846154 0.521232545375824
3.05128205128205 0.51141893863678
3.19871794871795 0.51218169927597
3.34935897435897 0.514228701591492
3.50961538461538 0.539941251277924
3.67628205128205 0.509822010993958
3.8525641025641 0.536152243614197
4.03525641025641 0.502993583679199
4.2275641025641 0.498270511627197
4.42948717948718 0.468910366296768
4.64102564102564 0.478029251098633
4.86217948717949 0.487383127212524
5.09294871794872 0.472910732030869
5.33653846153846 0.456267118453979
5.58974358974359 0.441177576780319
5.85576923076923 0.468663185834885
6.13461538461539 0.487639337778091
6.42628205128205 0.461875528097153
6.73397435897436 0.451742231845856
7.05448717948718 0.440451920032501
7.38782051282051 0.459627538919449
7.74038461538461 0.447512924671173
8.10897435897436 0.482622921466827
8.49679487179487 0.472595304250717
8.90064102564103 0.439602762460709
9.32371794871795 0.424852281808853
9.76923076923077 0.430189818143845
10.2339743589744 0.393048822879791
10.7211538461538 0.405276209115982
11.2307692307692 0.427704960107803
11.7660256410256 0.410948187112808
12.3269230769231 0.411720752716064
12.9134615384615 0.460963070392609
13.5288461538462 0.402852147817612
14.1730769230769 0.452136486768723
14.849358974359 0.390006691217422
15.5544871794872 0.375693142414093
16.2948717948718 0.393276125192642
17.0705128205128 0.374880760908127
17.8846153846154 0.458240807056427
18.7371794871795 0.427198141813278
19.6282051282051 0.416810512542725
20.5641025641026 0.405216991901398
21.5416666666667 0.389627307653427
22.5673076923077 0.387713372707367
23.6442307692308 0.414250046014786
24.7692307692308 0.40398508310318
25.9487179487179 0.367044627666473
27.1826923076923 0.404527515172958
28.4775641025641 0.427076488733292
29.8333333333333 0.406216681003571
31.2564102564103 0.356632679700851
32.7435897435897 0.333788931369781
34.3044871794872 0.337432056665421
35.9358974358974 0.400461912155151
37.6474358974359 0.368151664733887
39.4391025641026 0.343468755483627
41.3173076923077 0.328131169080734
43.2852564102564 0.315718233585358
45.3461538461538 0.329084128141403
47.5064102564103 0.324349790811539
49.7692307692308 0.334141701459885
52.1378205128205 0.271560162305832
54.6217948717949 0.321716129779816
57.2211538461538 0.289410918951035
59.9455128205128 0.337118864059448
62.8012820512821 0.334038347005844
65.7916666666667 0.314336270093918
68.9230769230769 0.294853419065475
72.2051282051282 0.276419997215271
75.6442307692308 0.272727727890015
79.2467948717949 0.217798635363579
83.0192307692308 0.307780057191849
86.974358974359 0.273792952299118
91.1153846153846 0.233823254704475
95.4519230769231 0.281128346920013
100 0.274847507476807
};
\addlegendentry{mb 8, exact}
\addplot [, color1, opacity=0.6, mark=square*, mark size=0.5, mark options={solid}, only marks, forget plot]
table {%
1 0.817290484905243
1.04487179487179 0.833622395992279
1.09615384615385 0.802972435951233
1.1474358974359 0.793247818946838
1.20192307692308 0.798309087753296
1.25961538461538 0.727238595485687
1.32051282051282 0.677721679210663
1.38461538461538 0.723292052745819
1.44871794871795 0.711740672588348
1.51923076923077 0.705733954906464
1.58974358974359 0.735188901424408
1.66666666666667 0.697189748287201
1.74679487179487 0.674855649471283
1.83012820512821 0.632772326469421
1.91666666666667 0.649274170398712
2.00641025641026 0.645285308361053
2.1025641025641 0.623648703098297
2.20512820512821 0.61297744512558
2.30769230769231 0.572667181491852
2.41987179487179 0.597496509552002
2.53525641025641 0.569752812385559
2.65384615384615 0.530937790870667
2.78205128205128 0.591774582862854
2.91346153846154 0.528895735740662
3.05128205128205 0.540111541748047
3.19871794871795 0.538449764251709
3.34935897435897 0.5384281873703
3.50961538461538 0.519609093666077
3.67628205128205 0.54844331741333
3.8525641025641 0.492975771427155
4.03525641025641 0.5104119181633
4.2275641025641 0.522556245326996
4.42948717948718 0.465496361255646
4.64102564102564 0.481006145477295
4.86217948717949 0.507571160793304
5.09294871794872 0.449863642454147
5.33653846153846 0.505663394927979
5.58974358974359 0.487150967121124
5.85576923076923 0.447920173406601
6.13461538461539 0.440423458814621
6.42628205128205 0.492852210998535
6.73397435897436 0.466593712568283
7.05448717948718 0.493542015552521
7.38782051282051 0.425025224685669
7.74038461538461 0.440595239400864
8.10897435897436 0.440321981906891
8.49679487179487 0.423727095127106
8.90064102564103 0.469213098287582
9.32371794871795 0.435114115476608
9.76923076923077 0.413553237915039
10.2339743589744 0.440394133329391
10.7211538461538 0.422849744558334
11.2307692307692 0.407037258148193
11.7660256410256 0.4073785841465
12.3269230769231 0.38017874956131
12.9134615384615 0.376680374145508
13.5288461538462 0.344733774662018
14.1730769230769 0.376605361700058
14.849358974359 0.379657685756683
15.5544871794872 0.359738916158676
16.2948717948718 0.381080955266953
17.0705128205128 0.385384529829025
17.8846153846154 0.383078187704086
18.7371794871795 0.372798174619675
19.6282051282051 0.35928612947464
20.5641025641026 0.358325332403183
21.5416666666667 0.327907085418701
22.5673076923077 0.363713979721069
23.6442307692308 0.356115251779556
24.7692307692308 0.29966613650322
25.9487179487179 0.320610523223877
27.1826923076923 0.328413993120193
28.4775641025641 0.29500937461853
29.8333333333333 0.320685625076294
31.2564102564103 0.278470396995544
32.7435897435897 0.314703196287155
34.3044871794872 0.300110310316086
35.9358974358974 0.270798057317734
37.6474358974359 0.273097932338715
39.4391025641026 0.277230083942413
41.3173076923077 0.285015106201172
43.2852564102564 0.289374470710754
45.3461538461538 0.268408000469208
47.5064102564103 0.247292473912239
49.7692307692308 0.270607203245163
52.1378205128205 0.248081400990486
54.6217948717949 0.249355658888817
57.2211538461538 0.260912716388702
59.9455128205128 0.261591762304306
62.8012820512821 0.26254940032959
65.7916666666667 0.234893187880516
68.9230769230769 0.254209667444229
72.2051282051282 0.272329688072205
75.6442307692308 0.240091368556023
79.2467948717949 0.255794554948807
83.0192307692308 0.218131348490715
86.974358974359 0.229501590132713
91.1153846153846 0.23914460837841
95.4519230769231 0.249630644917488
100 0.230527922511101
};
\addplot [, color1, opacity=0.6, mark=square*, mark size=0.5, mark options={solid}, only marks, forget plot]
table {%
1 0.790875136852264
1.04487179487179 0.859308183193207
1.09615384615385 0.795816123485565
1.1474358974359 0.766025424003601
1.20192307692308 0.75420469045639
1.25961538461538 0.753512561321259
1.32051282051282 0.652438580989838
1.38461538461538 0.698777377605438
1.44871794871795 0.679062485694885
1.51923076923077 0.654459297657013
1.58974358974359 0.689158380031586
1.66666666666667 0.65529078245163
1.74679487179487 0.649127423763275
1.83012820512821 0.603583931922913
1.91666666666667 0.603108108043671
2.00641025641026 0.606627821922302
2.1025641025641 0.572260797023773
2.20512820512821 0.624123275279999
2.30769230769231 0.584132373332977
2.41987179487179 0.576422989368439
2.53525641025641 0.607409358024597
2.65384615384615 0.590995490550995
2.78205128205128 0.635689198970795
2.91346153846154 0.604482591152191
3.05128205128205 0.594979763031006
3.19871794871795 0.574485301971436
3.34935897435897 0.533998608589172
3.50961538461538 0.550468862056732
3.67628205128205 0.540239632129669
3.8525641025641 0.545129477977753
4.03525641025641 0.542207896709442
4.2275641025641 0.552607238292694
4.42948717948718 0.521578967571259
4.64102564102564 0.511578917503357
4.86217948717949 0.489605247974396
5.09294871794872 0.521907091140747
5.33653846153846 0.483219534158707
5.58974358974359 0.461021959781647
5.85576923076923 0.452543824911118
6.13461538461539 0.507509231567383
6.42628205128205 0.424778521060944
6.73397435897436 0.47260782122612
7.05448717948718 0.452501684427261
7.38782051282051 0.440578371286392
7.74038461538461 0.413476675748825
8.10897435897436 0.409554570913315
8.49679487179487 0.428780764341354
8.90064102564103 0.435083538293839
9.32371794871795 0.431083589792252
9.76923076923077 0.405536860227585
10.2339743589744 0.419180363416672
10.7211538461538 0.38334658741951
11.2307692307692 0.396318465471268
11.7660256410256 0.379234999418259
12.3269230769231 0.39835250377655
12.9134615384615 0.3829625248909
13.5288461538462 0.336646050214767
14.1730769230769 0.36872336268425
14.849358974359 0.357752025127411
15.5544871794872 0.366139948368073
16.2948717948718 0.356333762407303
17.0705128205128 0.306655675172806
17.8846153846154 0.321937918663025
18.7371794871795 0.3731988966465
19.6282051282051 0.347998142242432
20.5641025641026 0.351135164499283
21.5416666666667 0.342731863260269
22.5673076923077 0.340925723314285
23.6442307692308 0.358181715011597
24.7692307692308 0.33562096953392
25.9487179487179 0.30395969748497
27.1826923076923 0.347082197666168
28.4775641025641 0.278556674718857
29.8333333333333 0.319155544042587
31.2564102564103 0.324093341827393
32.7435897435897 0.286205381155014
34.3044871794872 0.280888348817825
35.9358974358974 0.30543065071106
37.6474358974359 0.287291467189789
39.4391025641026 0.270487934350967
41.3173076923077 0.276360034942627
43.2852564102564 0.244526296854019
45.3461538461538 0.250813335180283
47.5064102564103 0.243494674563408
49.7692307692308 0.288633614778519
52.1378205128205 0.245680764317513
54.6217948717949 0.243079587817192
57.2211538461538 0.24867208302021
59.9455128205128 0.255564987659454
62.8012820512821 0.231819987297058
65.7916666666667 0.25693079829216
68.9230769230769 0.255542725324631
72.2051282051282 0.238452419638634
75.6442307692308 0.228410348296165
79.2467948717949 0.143796756863594
83.0192307692308 0.228780224919319
86.974358974359 0.21432788670063
91.1153846153846 0.160383209586143
95.4519230769231 0.183212637901306
100 0.217475175857544
};
\addplot [, color1, opacity=0.6, mark=square*, mark size=0.5, mark options={solid}, only marks, forget plot]
table {%
1 0.84156322479248
1.04487179487179 0.810401141643524
1.09615384615385 0.849722683429718
1.1474358974359 0.869491994380951
1.20192307692308 0.863043606281281
1.25961538461538 0.772471606731415
1.32051282051282 0.752154290676117
1.38461538461538 0.718194425106049
1.44871794871795 0.787082135677338
1.51923076923077 0.745417356491089
1.58974358974359 0.700117409229279
1.66666666666667 0.718453586101532
1.74679487179487 0.637394607067108
1.83012820512821 0.682915329933167
1.91666666666667 0.64766663312912
2.00641025641026 0.651719093322754
2.1025641025641 0.629267394542694
2.20512820512821 0.580933511257172
2.30769230769231 0.603518664836884
2.41987179487179 0.572541236877441
2.53525641025641 0.611217439174652
2.65384615384615 0.610289216041565
2.78205128205128 0.583103001117706
2.91346153846154 0.582615852355957
3.05128205128205 0.553732395172119
3.19871794871795 0.544501900672913
3.34935897435897 0.524566113948822
3.50961538461538 0.512038052082062
3.67628205128205 0.496919542551041
3.8525641025641 0.520865440368652
4.03525641025641 0.537538349628448
4.2275641025641 0.473673671483994
4.42948717948718 0.487593948841095
4.64102564102564 0.453553736209869
4.86217948717949 0.434724718332291
5.09294871794872 0.487040668725967
5.33653846153846 0.428161293268204
5.58974358974359 0.497292727231979
5.85576923076923 0.44395238161087
6.13461538461539 0.43573209643364
6.42628205128205 0.454729288816452
6.73397435897436 0.467222779989243
7.05448717948718 0.482168763875961
7.38782051282051 0.425057500600815
7.74038461538461 0.469218701124191
8.10897435897436 0.466367244720459
8.49679487179487 0.454442977905273
8.90064102564103 0.45413276553154
9.32371794871795 0.425470739603043
9.76923076923077 0.407605648040771
10.2339743589744 0.384866714477539
10.7211538461538 0.433651059865952
11.2307692307692 0.415202707052231
11.7660256410256 0.375048875808716
12.3269230769231 0.441868305206299
12.9134615384615 0.419702529907227
13.5288461538462 0.462206423282623
14.1730769230769 0.431682169437408
14.849358974359 0.428505718708038
15.5544871794872 0.393610090017319
16.2948717948718 0.380222767591476
17.0705128205128 0.343656182289124
17.8846153846154 0.393765419721603
18.7371794871795 0.37301430106163
19.6282051282051 0.328802078962326
20.5641025641026 0.37728476524353
21.5416666666667 0.357637286186218
22.5673076923077 0.342895656824112
23.6442307692308 0.388205260038376
24.7692307692308 0.368149280548096
25.9487179487179 0.38114133477211
27.1826923076923 0.343823879957199
28.4775641025641 0.337090462446213
29.8333333333333 0.361041992902756
31.2564102564103 0.32559609413147
32.7435897435897 0.383866786956787
34.3044871794872 0.319305419921875
35.9358974358974 0.358463227748871
37.6474358974359 0.356724947690964
39.4391025641026 0.345072656869888
41.3173076923077 0.303710848093033
43.2852564102564 0.341316372156143
45.3461538461538 0.291898727416992
47.5064102564103 0.378540724515915
49.7692307692308 0.331479996442795
52.1378205128205 0.296035289764404
54.6217948717949 0.390150994062424
57.2211538461538 0.328662484884262
59.9455128205128 0.318418085575104
62.8012820512821 0.378795385360718
65.7916666666667 0.307457029819489
68.9230769230769 0.308516889810562
72.2051282051282 0.341237306594849
75.6442307692308 0.368962168693542
79.2467948717949 0.273258566856384
83.0192307692308 0.316354423761368
86.974358974359 0.312498897314072
91.1153846153846 0.285168647766113
95.4519230769231 0.29633504152298
100 0.244280815124512
};
\addplot [, color1, opacity=0.6, mark=square*, mark size=0.5, mark options={solid}, only marks, forget plot]
table {%
1 0.832199692726135
1.04487179487179 0.852664947509766
1.09615384615385 0.852144837379456
1.1474358974359 0.806996762752533
1.20192307692308 0.804085254669189
1.25961538461538 0.753994345664978
1.32051282051282 0.731126725673676
1.38461538461538 0.739906311035156
1.44871794871795 0.736095309257507
1.51923076923077 0.713423430919647
1.58974358974359 0.698448181152344
1.66666666666667 0.695625603199005
1.74679487179487 0.719115674495697
1.83012820512821 0.661738216876984
1.91666666666667 0.650789558887482
2.00641025641026 0.649781823158264
2.1025641025641 0.567904591560364
2.20512820512821 0.54837840795517
2.30769230769231 0.568767249584198
2.41987179487179 0.579926133155823
2.53525641025641 0.567223846912384
2.65384615384615 0.564363956451416
2.78205128205128 0.52381557226181
2.91346153846154 0.518091201782227
3.05128205128205 0.534168660640717
3.19871794871795 0.555109143257141
3.34935897435897 0.55482017993927
3.50961538461538 0.544473648071289
3.67628205128205 0.502694189548492
3.8525641025641 0.543829321861267
4.03525641025641 0.522872149944305
4.2275641025641 0.530649602413177
4.42948717948718 0.536355674266815
4.64102564102564 0.498820304870605
4.86217948717949 0.512147724628448
5.09294871794872 0.517255783081055
5.33653846153846 0.509665310382843
5.58974358974359 0.487054973840714
5.85576923076923 0.450135856866837
6.13461538461539 0.457638174295425
6.42628205128205 0.463554292917252
6.73397435897436 0.484735459089279
7.05448717948718 0.486353009939194
7.38782051282051 0.452537357807159
7.74038461538461 0.465028017759323
8.10897435897436 0.468230575323105
8.49679487179487 0.425179094076157
8.90064102564103 0.459640234708786
9.32371794871795 0.429241806268692
9.76923076923077 0.428711801767349
10.2339743589744 0.444723606109619
10.7211538461538 0.431533426046371
11.2307692307692 0.420778423547745
11.7660256410256 0.4031982421875
12.3269230769231 0.414184719324112
12.9134615384615 0.402811378240585
13.5288461538462 0.37811404466629
14.1730769230769 0.41451159119606
14.849358974359 0.394625872373581
15.5544871794872 0.391471922397614
16.2948717948718 0.417096823453903
17.0705128205128 0.411496847867966
17.8846153846154 0.389127194881439
18.7371794871795 0.376384884119034
19.6282051282051 0.378056198358536
20.5641025641026 0.396567046642303
21.5416666666667 0.39749550819397
22.5673076923077 0.402093976736069
23.6442307692308 0.3363878428936
24.7692307692308 0.384241670370102
25.9487179487179 0.384370416402817
27.1826923076923 0.321665376424789
28.4775641025641 0.301871031522751
29.8333333333333 0.304227203130722
31.2564102564103 0.33698371052742
32.7435897435897 0.357223361730576
34.3044871794872 0.372363537549973
35.9358974358974 0.273480325937271
37.6474358974359 0.307365745306015
39.4391025641026 0.27471661567688
41.3173076923077 0.28671795129776
43.2852564102564 0.33003157377243
45.3461538461538 0.291920512914658
47.5064102564103 0.304850727319717
49.7692307692308 0.283096015453339
52.1378205128205 0.306422799825668
54.6217948717949 0.270999163389206
57.2211538461538 0.332549631595612
59.9455128205128 0.300238400697708
62.8012820512821 0.312691807746887
65.7916666666667 0.248575523495674
68.9230769230769 0.318888962268829
72.2051282051282 0.277192234992981
75.6442307692308 0.262075334787369
79.2467948717949 0.291235148906708
83.0192307692308 0.306561678647995
86.974358974359 0.242272689938545
91.1153846153846 0.256353706121445
95.4519230769231 0.242180705070496
100 0.247388124465942
};
\addplot [, color2, opacity=0.6, mark=diamond*, mark size=0.5, mark options={solid}, only marks]
table {%
1 0.923596322536469
1.04487179487179 0.948826730251312
1.09615384615385 0.932184875011444
1.1474358974359 0.853067398071289
1.20192307692308 0.936140954494476
1.25961538461538 0.867386043071747
1.32051282051282 0.824814796447754
1.38461538461538 0.788733124732971
1.44871794871795 0.797811985015869
1.51923076923077 0.821383893489838
1.58974358974359 0.780341804027557
1.66666666666667 0.785732805728912
1.74679487179487 0.753855526447296
1.83012820512821 0.810369491577148
1.91666666666667 0.787722826004028
2.00641025641026 0.81292450428009
2.1025641025641 0.745684564113617
2.20512820512821 0.750877201557159
2.30769230769231 0.775173842906952
2.41987179487179 0.742673218250275
2.53525641025641 0.811068713665009
2.65384615384615 0.743331730365753
2.78205128205128 0.735200881958008
2.91346153846154 0.73801326751709
3.05128205128205 0.738528966903687
3.19871794871795 0.696457982063293
3.34935897435897 0.740277945995331
3.50961538461538 0.66485995054245
3.67628205128205 0.685679614543915
3.8525641025641 0.676301598548889
4.03525641025641 0.673131108283997
4.2275641025641 0.632294178009033
4.42948717948718 0.65690678358078
4.64102564102564 0.689555644989014
4.86217948717949 0.682468056678772
5.09294871794872 0.68321430683136
5.33653846153846 0.618815362453461
5.58974358974359 0.637872517108917
5.85576923076923 0.618439793586731
6.13461538461539 0.633284091949463
6.42628205128205 0.616554379463196
6.73397435897436 0.604224383831024
7.05448717948718 0.610110282897949
7.38782051282051 0.613191545009613
7.74038461538461 0.567644536495209
8.10897435897436 0.586249351501465
8.49679487179487 0.608028709888458
8.90064102564103 0.594076156616211
9.32371794871795 0.572559714317322
9.76923076923077 0.592096626758575
10.2339743589744 0.627251267433167
10.7211538461538 0.550368964672089
11.2307692307692 0.549485325813293
11.7660256410256 0.56054276227951
12.3269230769231 0.516850769519806
12.9134615384615 0.565336644649506
13.5288461538462 0.511038780212402
14.1730769230769 0.535507321357727
14.849358974359 0.536588847637177
15.5544871794872 0.569409012794495
16.2948717948718 0.52327960729599
17.0705128205128 0.527316629886627
17.8846153846154 0.528558790683746
18.7371794871795 0.475571364164352
19.6282051282051 0.503554701805115
20.5641025641026 0.463877171278
21.5416666666667 0.512375950813293
22.5673076923077 0.434387028217316
23.6442307692308 0.498825639486313
24.7692307692308 0.430453985929489
25.9487179487179 0.500141620635986
27.1826923076923 0.471800416707993
28.4775641025641 0.452335804700851
29.8333333333333 0.442075252532959
31.2564102564103 0.446137249469757
32.7435897435897 0.411787122488022
34.3044871794872 0.389358043670654
35.9358974358974 0.360162705183029
37.6474358974359 0.401142418384552
39.4391025641026 0.37937119603157
41.3173076923077 0.370106518268585
43.2852564102564 0.329326212406158
45.3461538461538 0.315330415964127
47.5064102564103 0.353149741888046
49.7692307692308 0.33413153886795
52.1378205128205 0.312396258115768
54.6217948717949 0.247092172503471
57.2211538461538 0.290685564279556
59.9455128205128 0.278045952320099
62.8012820512821 0.277869075536728
65.7916666666667 0.264151334762573
68.9230769230769 0.234184190630913
72.2051282051282 0.282983273267746
75.6442307692308 0.218035265803337
79.2467948717949 0.210621744394302
83.0192307692308 0.243516847491264
86.974358974359 0.241593942046165
91.1153846153846 0.291359573602676
95.4519230769231 0.193163186311722
100 0.243242546916008
};
\addlegendentry{mb 32, exact}
\addplot [, color2, opacity=0.6, mark=diamond*, mark size=0.5, mark options={solid}, only marks, forget plot]
table {%
1 0.909110009670258
1.04487179487179 0.91324371099472
1.09615384615385 0.943406522274017
1.1474358974359 0.93501341342926
1.20192307692308 0.935328900814056
1.25961538461538 0.864865005016327
1.32051282051282 0.847835958003998
1.38461538461538 0.897116124629974
1.44871794871795 0.832647442817688
1.51923076923077 0.82162743806839
1.58974358974359 0.830810964107513
1.66666666666667 0.84141606092453
1.74679487179487 0.810281097888947
1.83012820512821 0.787113606929779
1.91666666666667 0.795083463191986
2.00641025641026 0.743732988834381
2.1025641025641 0.737479329109192
2.20512820512821 0.733865201473236
2.30769230769231 0.753193378448486
2.41987179487179 0.735922515392303
2.53525641025641 0.756545960903168
2.65384615384615 0.702017903327942
2.78205128205128 0.761334836483002
2.91346153846154 0.670202374458313
3.05128205128205 0.733997464179993
3.19871794871795 0.677244544029236
3.34935897435897 0.711730778217316
3.50961538461538 0.695183753967285
3.67628205128205 0.651515185832977
3.8525641025641 0.668325126171112
4.03525641025641 0.637773990631104
4.2275641025641 0.663382589817047
4.42948717948718 0.69854748249054
4.64102564102564 0.604002714157104
4.86217948717949 0.624508023262024
5.09294871794872 0.62965977191925
5.33653846153846 0.671830236911774
5.58974358974359 0.664434373378754
5.85576923076923 0.605644404888153
6.13461538461539 0.628252685070038
6.42628205128205 0.629720866680145
6.73397435897436 0.616782307624817
7.05448717948718 0.645887315273285
7.38782051282051 0.622330844402313
7.74038461538461 0.589641928672791
8.10897435897436 0.63943886756897
8.49679487179487 0.636010468006134
8.90064102564103 0.615337193012238
9.32371794871795 0.560231328010559
9.76923076923077 0.567024648189545
10.2339743589744 0.606737196445465
10.7211538461538 0.577746570110321
11.2307692307692 0.513975083827972
11.7660256410256 0.605417191982269
12.3269230769231 0.573237717151642
12.9134615384615 0.560134053230286
13.5288461538462 0.520385026931763
14.1730769230769 0.552833735942841
14.849358974359 0.567299485206604
15.5544871794872 0.49006849527359
16.2948717948718 0.533996999263763
17.0705128205128 0.490483850240707
17.8846153846154 0.514454483985901
18.7371794871795 0.510523855686188
19.6282051282051 0.498424053192139
20.5641025641026 0.481738150119781
21.5416666666667 0.445556789636612
22.5673076923077 0.511438846588135
23.6442307692308 0.442458629608154
24.7692307692308 0.40256068110466
25.9487179487179 0.444501399993896
27.1826923076923 0.435781449079514
28.4775641025641 0.39306303858757
29.8333333333333 0.427590429782867
31.2564102564103 0.379556506872177
32.7435897435897 0.347934484481812
34.3044871794872 0.380713552236557
35.9358974358974 0.375784069299698
37.6474358974359 0.343354463577271
39.4391025641026 0.350786298513412
41.3173076923077 0.318407446146011
43.2852564102564 0.314587205648422
45.3461538461538 0.315156042575836
47.5064102564103 0.318936109542847
49.7692307692308 0.345795124769211
52.1378205128205 0.350533217191696
54.6217948717949 0.308702081441879
57.2211538461538 0.327336460351944
59.9455128205128 0.277222514152527
62.8012820512821 0.255620628595352
65.7916666666667 0.279867708683014
68.9230769230769 0.259991317987442
72.2051282051282 0.28376841545105
75.6442307692308 0.24960994720459
79.2467948717949 0.236764907836914
83.0192307692308 0.246803596615791
86.974358974359 0.263487964868546
91.1153846153846 0.19944603741169
95.4519230769231 0.222612023353577
100 0.236471846699715
};
\addplot [, color2, opacity=0.6, mark=diamond*, mark size=0.5, mark options={solid}, only marks, forget plot]
table {%
1 0.939186036586761
1.04487179487179 0.960581600666046
1.09615384615385 0.972133636474609
1.1474358974359 0.963629066944122
1.20192307692308 0.957696855068207
1.25961538461538 0.861648201942444
1.32051282051282 0.898684680461884
1.38461538461538 0.847900331020355
1.44871794871795 0.830069243907928
1.51923076923077 0.823318660259247
1.58974358974359 0.810707211494446
1.66666666666667 0.789223611354828
1.74679487179487 0.822940051555634
1.83012820512821 0.84147834777832
1.91666666666667 0.817030608654022
2.00641025641026 0.832604885101318
2.1025641025641 0.814369142055511
2.20512820512821 0.789141356945038
2.30769230769231 0.77628093957901
2.41987179487179 0.766906261444092
2.53525641025641 0.762664496898651
2.65384615384615 0.78702586889267
2.78205128205128 0.782090127468109
2.91346153846154 0.780976057052612
3.05128205128205 0.789249062538147
3.19871794871795 0.751678049564362
3.34935897435897 0.771072387695312
3.50961538461538 0.707693934440613
3.67628205128205 0.704369187355042
3.8525641025641 0.711309731006622
4.03525641025641 0.686074912548065
4.2275641025641 0.702359020709991
4.42948717948718 0.637596547603607
4.64102564102564 0.673157870769501
4.86217948717949 0.703161180019379
5.09294871794872 0.648456513881683
5.33653846153846 0.704696714878082
5.58974358974359 0.703649640083313
5.85576923076923 0.646322250366211
6.13461538461539 0.67479807138443
6.42628205128205 0.632371842861176
6.73397435897436 0.619209110736847
7.05448717948718 0.655877709388733
7.38782051282051 0.64228618144989
7.74038461538461 0.578345894813538
8.10897435897436 0.59954845905304
8.49679487179487 0.644607424736023
8.90064102564103 0.59172511100769
9.32371794871795 0.565201222896576
9.76923076923077 0.557239532470703
10.2339743589744 0.540659189224243
10.7211538461538 0.555424869060516
11.2307692307692 0.527645409107208
11.7660256410256 0.558537006378174
12.3269230769231 0.54282009601593
12.9134615384615 0.533755958080292
13.5288461538462 0.452355116605759
14.1730769230769 0.521063446998596
14.849358974359 0.49442520737648
15.5544871794872 0.451494455337524
16.2948717948718 0.41840335726738
17.0705128205128 0.406205862760544
17.8846153846154 0.435362428426743
18.7371794871795 0.387539386749268
19.6282051282051 0.432558447122574
20.5641025641026 0.43563437461853
21.5416666666667 0.418601661920547
22.5673076923077 0.348040133714676
23.6442307692308 0.329881995916367
24.7692307692308 0.318592876195908
25.9487179487179 0.356861919164658
27.1826923076923 0.345322757959366
28.4775641025641 0.360304206609726
29.8333333333333 0.318399399518967
31.2564102564103 0.356953412294388
32.7435897435897 0.332024395465851
34.3044871794872 0.284701973199844
35.9358974358974 0.286424994468689
37.6474358974359 0.285911500453949
39.4391025641026 0.247767686843872
41.3173076923077 0.246083840727806
43.2852564102564 0.267797708511353
45.3461538461538 0.262297958135605
47.5064102564103 0.267339199781418
49.7692307692308 0.231055855751038
52.1378205128205 0.206860780715942
54.6217948717949 0.199464365839958
57.2211538461538 0.234666749835014
59.9455128205128 0.248019799590111
62.8012820512821 0.200420722365379
65.7916666666667 0.265529900789261
68.9230769230769 0.218236789107323
72.2051282051282 0.224767878651619
75.6442307692308 0.226058706641197
79.2467948717949 0.212494567036629
83.0192307692308 0.195730149745941
86.974358974359 0.23101818561554
91.1153846153846 0.23189289867878
95.4519230769231 0.199117541313171
100 0.165166720747948
};
\addplot [, color2, opacity=0.6, mark=diamond*, mark size=0.5, mark options={solid}, only marks, forget plot]
table {%
1 0.870489716529846
1.04487179487179 0.935364246368408
1.09615384615385 0.964053153991699
1.1474358974359 0.958510041236877
1.20192307692308 0.958301961421967
1.25961538461538 0.921665489673615
1.32051282051282 0.842723548412323
1.38461538461538 0.913231790065765
1.44871794871795 0.833271026611328
1.51923076923077 0.800750732421875
1.58974358974359 0.863539516925812
1.66666666666667 0.829654812812805
1.74679487179487 0.841280102729797
1.83012820512821 0.81115859746933
1.91666666666667 0.819491803646088
2.00641025641026 0.816375911235809
2.1025641025641 0.774799764156342
2.20512820512821 0.786855638027191
2.30769230769231 0.791038334369659
2.41987179487179 0.758108615875244
2.53525641025641 0.782817363739014
2.65384615384615 0.724452912807465
2.78205128205128 0.807171642780304
2.91346153846154 0.728748023509979
3.05128205128205 0.78945118188858
3.19871794871795 0.789476811885834
3.34935897435897 0.764855682849884
3.50961538461538 0.739502370357513
3.67628205128205 0.758156895637512
3.8525641025641 0.74662834405899
4.03525641025641 0.731626212596893
4.2275641025641 0.736231744289398
4.42948717948718 0.723129510879517
4.64102564102564 0.663209617137909
4.86217948717949 0.680518090724945
5.09294871794872 0.68487948179245
5.33653846153846 0.673256456851959
5.58974358974359 0.651246070861816
5.85576923076923 0.609593570232391
6.13461538461539 0.638026893138885
6.42628205128205 0.618495583534241
6.73397435897436 0.605496168136597
7.05448717948718 0.65783703327179
7.38782051282051 0.637362122535706
7.74038461538461 0.6588214635849
8.10897435897436 0.595795571804047
8.49679487179487 0.584287583827972
8.90064102564103 0.588076889514923
9.32371794871795 0.551468312740326
9.76923076923077 0.537117481231689
10.2339743589744 0.554873049259186
10.7211538461538 0.56600433588028
11.2307692307692 0.554200947284698
11.7660256410256 0.517959713935852
12.3269230769231 0.556491076946259
12.9134615384615 0.541067600250244
13.5288461538462 0.46036496758461
14.1730769230769 0.496356099843979
14.849358974359 0.480073422193527
15.5544871794872 0.495301306247711
16.2948717948718 0.457664549350739
17.0705128205128 0.450729936361313
17.8846153846154 0.485267251729965
18.7371794871795 0.480254650115967
19.6282051282051 0.440441608428955
20.5641025641026 0.424300670623779
21.5416666666667 0.42279776930809
22.5673076923077 0.396065056324005
23.6442307692308 0.365687519311905
24.7692307692308 0.345124691724777
25.9487179487179 0.369284152984619
27.1826923076923 0.342686921358109
28.4775641025641 0.388363242149353
29.8333333333333 0.340485632419586
31.2564102564103 0.334116131067276
32.7435897435897 0.329119861125946
34.3044871794872 0.32552632689476
35.9358974358974 0.341098845005035
37.6474358974359 0.310205519199371
39.4391025641026 0.3306964635849
41.3173076923077 0.291908174753189
43.2852564102564 0.283678859472275
45.3461538461538 0.294953346252441
47.5064102564103 0.299306273460388
49.7692307692308 0.326312065124512
52.1378205128205 0.33295202255249
54.6217948717949 0.288879603147507
57.2211538461538 0.294006496667862
59.9455128205128 0.284581869840622
62.8012820512821 0.300698757171631
65.7916666666667 0.268146961927414
68.9230769230769 0.255880564451218
72.2051282051282 0.325813591480255
75.6442307692308 0.239172413945198
79.2467948717949 0.258913040161133
83.0192307692308 0.327995985746384
86.974358974359 0.229960232973099
91.1153846153846 0.259236067533493
95.4519230769231 0.218886658549309
100 0.256976455450058
};
\addplot [, color2, opacity=0.6, mark=diamond*, mark size=0.5, mark options={solid}, only marks, forget plot]
table {%
1 0.903891086578369
1.04487179487179 0.960523068904877
1.09615384615385 0.969154298305511
1.1474358974359 0.957531869411469
1.20192307692308 0.953636348247528
1.25961538461538 0.869651794433594
1.32051282051282 0.843073189258575
1.38461538461538 0.833102822303772
1.44871794871795 0.837513089179993
1.51923076923077 0.826095223426819
1.58974358974359 0.819530427455902
1.66666666666667 0.767179429531097
1.74679487179487 0.804807126522064
1.83012820512821 0.767197787761688
1.91666666666667 0.766448795795441
2.00641025641026 0.801736176013947
2.1025641025641 0.784634232521057
2.20512820512821 0.76101940870285
2.30769230769231 0.744242191314697
2.41987179487179 0.756465435028076
2.53525641025641 0.707584023475647
2.65384615384615 0.761402130126953
2.78205128205128 0.7372727394104
2.91346153846154 0.719739139080048
3.05128205128205 0.712562024593353
3.19871794871795 0.719883501529694
3.34935897435897 0.70027220249176
3.50961538461538 0.683774828910828
3.67628205128205 0.684413969516754
3.8525641025641 0.663357138633728
4.03525641025641 0.633002936840057
4.2275641025641 0.66645336151123
4.42948717948718 0.633135497570038
4.64102564102564 0.599463939666748
4.86217948717949 0.609501779079437
5.09294871794872 0.591087162494659
5.33653846153846 0.599561393260956
5.58974358974359 0.591695606708527
5.85576923076923 0.604640960693359
6.13461538461539 0.602146565914154
6.42628205128205 0.570812225341797
6.73397435897436 0.566800713539124
7.05448717948718 0.616502225399017
7.38782051282051 0.576175391674042
7.74038461538461 0.531761050224304
8.10897435897436 0.534079074859619
8.49679487179487 0.578649699687958
8.90064102564103 0.562284648418427
9.32371794871795 0.572964429855347
9.76923076923077 0.516609668731689
10.2339743589744 0.509173214435577
10.7211538461538 0.49501821398735
11.2307692307692 0.480398893356323
11.7660256410256 0.504102349281311
12.3269230769231 0.493724882602692
12.9134615384615 0.497609227895737
13.5288461538462 0.45828253030777
14.1730769230769 0.460320293903351
14.849358974359 0.452550798654556
15.5544871794872 0.456444352865219
16.2948717948718 0.475094527006149
17.0705128205128 0.431394189596176
17.8846153846154 0.497371882200241
18.7371794871795 0.438674211502075
19.6282051282051 0.424494653940201
20.5641025641026 0.466628313064575
21.5416666666667 0.443159580230713
22.5673076923077 0.413116216659546
23.6442307692308 0.421346008777618
24.7692307692308 0.429418653249741
25.9487179487179 0.363405764102936
27.1826923076923 0.387297719717026
28.4775641025641 0.393820881843567
29.8333333333333 0.338872283697128
31.2564102564103 0.369581669569016
32.7435897435897 0.355920374393463
34.3044871794872 0.341797351837158
35.9358974358974 0.353797197341919
37.6474358974359 0.304833739995956
39.4391025641026 0.309753388166428
41.3173076923077 0.263724654912949
43.2852564102564 0.312895238399506
45.3461538461538 0.270923256874084
47.5064102564103 0.291931629180908
49.7692307692308 0.337716162204742
52.1378205128205 0.29282534122467
54.6217948717949 0.256777554750443
57.2211538461538 0.259248822927475
59.9455128205128 0.27093893289566
62.8012820512821 0.309323042631149
65.7916666666667 0.267243236303329
68.9230769230769 0.257831424474716
72.2051282051282 0.243019253015518
75.6442307692308 0.273396700620651
79.2467948717949 0.257131069898605
83.0192307692308 0.246834471821785
86.974358974359 0.213358119130135
91.1153846153846 0.294928878545761
95.4519230769231 0.231307178735733
100 0.25795790553093
};
\addplot [, black, opacity=0.6, mark=*, mark size=0.5, mark options={solid}, only marks]
table {%
1 0.980172097682953
1.04487179487179 0.991716206073761
1.09615384615385 0.992607712745667
1.1474358974359 0.989956080913544
1.20192307692308 0.990800857543945
1.25961538461538 0.907791912555695
1.32051282051282 0.961326599121094
1.38461538461538 0.88730925321579
1.44871794871795 0.947700977325439
1.51923076923077 0.937219440937042
1.58974358974359 0.949622809886932
1.66666666666667 0.968756198883057
1.74679487179487 0.887838780879974
1.83012820512821 0.916164219379425
1.91666666666667 0.947528004646301
2.00641025641026 0.922192215919495
2.1025641025641 0.928771018981934
2.20512820512821 0.897602498531342
2.30769230769231 0.870130360126495
2.41987179487179 0.878030240535736
2.53525641025641 0.866652131080627
2.65384615384615 0.909153580665588
2.78205128205128 0.937457025051117
2.91346153846154 0.9390869140625
3.05128205128205 0.866176545619965
3.19871794871795 0.88209742307663
3.34935897435897 0.900237679481506
3.50961538461538 0.905459046363831
3.67628205128205 0.9186971783638
3.8525641025641 0.922338128089905
4.03525641025641 0.842959880828857
4.2275641025641 0.896037757396698
4.42948717948718 0.919768989086151
4.64102564102564 0.837957322597504
4.86217948717949 0.866225063800812
5.09294871794872 0.874295353889465
5.33653846153846 0.857604801654816
5.58974358974359 0.839182078838348
5.85576923076923 0.831863701343536
6.13461538461539 0.827021539211273
6.42628205128205 0.851464450359344
6.73397435897436 0.853148102760315
7.05448717948718 0.827787399291992
7.38782051282051 0.85309374332428
7.74038461538461 0.840215861797333
8.10897435897436 0.842476665973663
8.49679487179487 0.846230506896973
8.90064102564103 0.825722634792328
9.32371794871795 0.853772282600403
9.76923076923077 0.857580304145813
10.2339743589744 0.806088864803314
10.7211538461538 0.82514476776123
11.2307692307692 0.816884994506836
11.7660256410256 0.813710987567902
12.3269230769231 0.778618633747101
12.9134615384615 0.794024109840393
13.5288461538462 0.733671605587006
14.1730769230769 0.75925225019455
14.849358974359 0.770133435726166
15.5544871794872 0.770629107952118
16.2948717948718 0.775415062904358
17.0705128205128 0.771032631397247
17.8846153846154 0.748296916484833
18.7371794871795 0.683089137077332
19.6282051282051 0.702462375164032
20.5641025641026 0.673430383205414
21.5416666666667 0.739798545837402
22.5673076923077 0.656615376472473
23.6442307692308 0.712085545063019
24.7692307692308 0.661113440990448
25.9487179487179 0.686876118183136
27.1826923076923 0.673646330833435
28.4775641025641 0.661878705024719
29.8333333333333 0.658248126506805
31.2564102564103 0.601502895355225
32.7435897435897 0.635036945343018
34.3044871794872 0.598907768726349
35.9358974358974 0.649761617183685
37.6474358974359 0.624386310577393
39.4391025641026 0.583890557289124
41.3173076923077 0.602699160575867
43.2852564102564 0.562650620937347
45.3461538461538 0.520900547504425
47.5064102564103 0.571963310241699
49.7692307692308 0.508267223834991
52.1378205128205 0.49919381737709
54.6217948717949 0.480009287595749
57.2211538461538 0.501744508743286
59.9455128205128 0.439036667346954
62.8012820512821 0.425910860300064
65.7916666666667 0.406243652105331
68.9230769230769 0.426879316568375
72.2051282051282 0.436816990375519
75.6442307692308 0.393632560968399
79.2467948717949 0.367149382829666
83.0192307692308 0.35765129327774
86.974358974359 0.372109144926071
91.1153846153846 0.358631610870361
95.4519230769231 0.363734215497971
100 0.373095273971558
};
\addlegendentry{mb 128, exact}
\addplot [, black, opacity=0.6, mark=*, mark size=0.5, mark options={solid}, only marks, forget plot]
table {%
1 0.980207860469818
1.04487179487179 0.986687839031219
1.09615384615385 0.99109822511673
1.1474358974359 0.988957226276398
1.20192307692308 0.989828884601593
1.25961538461538 0.981170475482941
1.32051282051282 0.944415390491486
1.38461538461538 0.97414493560791
1.44871794871795 0.977206408977509
1.51923076923077 0.970506191253662
1.58974358974359 0.97232574224472
1.66666666666667 0.960947215557098
1.74679487179487 0.94966596364975
1.83012820512821 0.947515308856964
1.91666666666667 0.953044831752777
2.00641025641026 0.931983470916748
2.1025641025641 0.938220322132111
2.20512820512821 0.923636555671692
2.30769230769231 0.946711838245392
2.41987179487179 0.919902265071869
2.53525641025641 0.92106819152832
2.65384615384615 0.876817166805267
2.78205128205128 0.888949573040009
2.91346153846154 0.922625064849854
3.05128205128205 0.911627113819122
3.19871794871795 0.863408267498016
3.34935897435897 0.921932399272919
3.50961538461538 0.911691844463348
3.67628205128205 0.869503676891327
3.8525641025641 0.902216255664825
4.03525641025641 0.885700643062592
4.2275641025641 0.849653422832489
4.42948717948718 0.834793746471405
4.64102564102564 0.833202004432678
4.86217948717949 0.89985865354538
5.09294871794872 0.843555629253387
5.33653846153846 0.880906879901886
5.58974358974359 0.910552024841309
5.85576923076923 0.877913773059845
6.13461538461539 0.821998298168182
6.42628205128205 0.865674018859863
6.73397435897436 0.856091499328613
7.05448717948718 0.854511678218842
7.38782051282051 0.867451846599579
7.74038461538461 0.811384618282318
8.10897435897436 0.835431277751923
8.49679487179487 0.874100983142853
8.90064102564103 0.797595322132111
9.32371794871795 0.824461162090302
9.76923076923077 0.82661360502243
10.2339743589744 0.82679957151413
10.7211538461538 0.822871804237366
11.2307692307692 0.773120164871216
11.7660256410256 0.766072332859039
12.3269230769231 0.809707462787628
12.9134615384615 0.796929776668549
13.5288461538462 0.747281968593597
14.1730769230769 0.740639626979828
14.849358974359 0.753473103046417
15.5544871794872 0.73205977678299
16.2948717948718 0.766981542110443
17.0705128205128 0.712304592132568
17.8846153846154 0.791382789611816
18.7371794871795 0.703667104244232
19.6282051282051 0.689324736595154
20.5641025641026 0.644148707389832
21.5416666666667 0.634638488292694
22.5673076923077 0.67344206571579
23.6442307692308 0.653737664222717
24.7692307692308 0.664027214050293
25.9487179487179 0.686160087585449
27.1826923076923 0.632227838039398
28.4775641025641 0.638557136058807
29.8333333333333 0.659567296504974
31.2564102564103 0.581326365470886
32.7435897435897 0.600731611251831
34.3044871794872 0.56149685382843
35.9358974358974 0.593187272548676
37.6474358974359 0.560176849365234
39.4391025641026 0.536633670330048
41.3173076923077 0.543563067913055
43.2852564102564 0.568399846553802
45.3461538461538 0.481604427099228
47.5064102564103 0.538267314434052
49.7692307692308 0.473466485738754
52.1378205128205 0.518011033535004
54.6217948717949 0.448529452085495
57.2211538461538 0.464230448007584
59.9455128205128 0.472301870584488
62.8012820512821 0.444409996271133
65.7916666666667 0.427443742752075
68.9230769230769 0.413966983556747
72.2051282051282 0.409690469503403
75.6442307692308 0.459181159734726
79.2467948717949 0.353957533836365
83.0192307692308 0.398236364126205
86.974358974359 0.31950780749321
91.1153846153846 0.304738879203796
95.4519230769231 0.313452303409576
100 0.317403376102448
};
\addplot [, black, opacity=0.6, mark=*, mark size=0.5, mark options={solid}, only marks, forget plot]
table {%
1 0.983946800231934
1.04487179487179 0.988971710205078
1.09615384615385 0.992758393287659
1.1474358974359 0.990560352802277
1.20192307692308 0.989403367042542
1.25961538461538 0.973469197750092
1.32051282051282 0.899483203887939
1.38461538461538 0.972841441631317
1.44871794871795 0.971960842609406
1.51923076923077 0.968037068843842
1.58974358974359 0.967146873474121
1.66666666666667 0.912840843200684
1.74679487179487 0.959828197956085
1.83012820512821 0.939577400684357
1.91666666666667 0.936375319957733
2.00641025641026 0.937602162361145
2.1025641025641 0.946089446544647
2.20512820512821 0.927560746669769
2.30769230769231 0.925248324871063
2.41987179487179 0.926818192005157
2.53525641025641 0.936004757881165
2.65384615384615 0.900654315948486
2.78205128205128 0.908032417297363
2.91346153846154 0.909767746925354
3.05128205128205 0.936206161975861
3.19871794871795 0.888727605342865
3.34935897435897 0.905566990375519
3.50961538461538 0.903665363788605
3.67628205128205 0.893167674541473
3.8525641025641 0.904410183429718
4.03525641025641 0.899736523628235
4.2275641025641 0.855421245098114
4.42948717948718 0.84848165512085
4.64102564102564 0.849940001964569
4.86217948717949 0.838516414165497
5.09294871794872 0.821752548217773
5.33653846153846 0.874599099159241
5.58974358974359 0.886141002178192
5.85576923076923 0.879460513591766
6.13461538461539 0.883392512798309
6.42628205128205 0.852098405361176
6.73397435897436 0.825584411621094
7.05448717948718 0.805654466152191
7.38782051282051 0.85651171207428
7.74038461538461 0.826787769794464
8.10897435897436 0.813583970069885
8.49679487179487 0.853635966777802
8.90064102564103 0.787913739681244
9.32371794871795 0.808611512184143
9.76923076923077 0.77556037902832
10.2339743589744 0.760887801647186
10.7211538461538 0.785118699073792
11.2307692307692 0.791158854961395
11.7660256410256 0.775297701358795
12.3269230769231 0.764155805110931
12.9134615384615 0.810503602027893
13.5288461538462 0.77249950170517
14.1730769230769 0.756299197673798
14.849358974359 0.724047720432281
15.5544871794872 0.702045857906342
16.2948717948718 0.755908310413361
17.0705128205128 0.690865457057953
17.8846153846154 0.739099204540253
18.7371794871795 0.696106493473053
19.6282051282051 0.656948268413544
20.5641025641026 0.730032682418823
21.5416666666667 0.692141950130463
22.5673076923077 0.69680780172348
23.6442307692308 0.670672595500946
24.7692307692308 0.637449860572815
25.9487179487179 0.644334495067596
27.1826923076923 0.635448336601257
28.4775641025641 0.646008908748627
29.8333333333333 0.639594495296478
31.2564102564103 0.57536393404007
32.7435897435897 0.579100787639618
34.3044871794872 0.598610639572144
35.9358974358974 0.641373157501221
37.6474358974359 0.558124184608459
39.4391025641026 0.518050730228424
41.3173076923077 0.534603893756866
43.2852564102564 0.511987686157227
45.3461538461538 0.50470495223999
47.5064102564103 0.483739674091339
49.7692307692308 0.471539467573166
52.1378205128205 0.392315179109573
54.6217948717949 0.452086359262466
57.2211538461538 0.38785719871521
59.9455128205128 0.432793229818344
62.8012820512821 0.401318460702896
65.7916666666667 0.384503036737442
68.9230769230769 0.244176179170609
72.2051282051282 0.297735422849655
75.6442307692308 0.372241079807281
79.2467948717949 0.284189194440842
83.0192307692308 0.352907985448837
86.974358974359 0.332353800535202
91.1153846153846 0.307710289955139
95.4519230769231 0.33690333366394
100 0.282698422670364
};
\addplot [, black, opacity=0.6, mark=*, mark size=0.5, mark options={solid}, only marks, forget plot]
table {%
1 0.982988655567169
1.04487179487179 0.992447853088379
1.09615384615385 0.993656158447266
1.1474358974359 0.990851044654846
1.20192307692308 0.989253461360931
1.25961538461538 0.947678983211517
1.32051282051282 0.947290420532227
1.38461538461538 0.93720954656601
1.44871794871795 0.948299825191498
1.51923076923077 0.963698089122772
1.58974358974359 0.967891335487366
1.66666666666667 0.898256957530975
1.74679487179487 0.940048694610596
1.83012820512821 0.948034584522247
1.91666666666667 0.927294194698334
2.00641025641026 0.900033593177795
2.1025641025641 0.879429519176483
2.20512820512821 0.871008992195129
2.30769230769231 0.928948998451233
2.41987179487179 0.922504425048828
2.53525641025641 0.897674560546875
2.65384615384615 0.855482757091522
2.78205128205128 0.916048645973206
2.91346153846154 0.892417430877686
3.05128205128205 0.895148456096649
3.19871794871795 0.88008588552475
3.34935897435897 0.883312880992889
3.50961538461538 0.869988441467285
3.67628205128205 0.875664055347443
3.8525641025641 0.895642876625061
4.03525641025641 0.853973388671875
4.2275641025641 0.823393523693085
4.42948717948718 0.86317777633667
4.64102564102564 0.853497803211212
4.86217948717949 0.854592263698578
5.09294871794872 0.832616984844208
5.33653846153846 0.821280896663666
5.58974358974359 0.801458775997162
5.85576923076923 0.821565270423889
6.13461538461539 0.808993637561798
6.42628205128205 0.83081191778183
6.73397435897436 0.827543199062347
7.05448717948718 0.815712153911591
7.38782051282051 0.772370278835297
7.74038461538461 0.788699567317963
8.10897435897436 0.818086802959442
8.49679487179487 0.793797016143799
8.90064102564103 0.739077568054199
9.32371794871795 0.76023668050766
9.76923076923077 0.78579705953598
10.2339743589744 0.788220643997192
10.7211538461538 0.775658845901489
11.2307692307692 0.789422690868378
11.7660256410256 0.799331724643707
12.3269230769231 0.750704288482666
12.9134615384615 0.782992780208588
13.5288461538462 0.727857768535614
14.1730769230769 0.753467559814453
14.849358974359 0.715115964412689
15.5544871794872 0.71807450056076
16.2948717948718 0.719560086727142
17.0705128205128 0.712568938732147
17.8846153846154 0.773623049259186
18.7371794871795 0.725335478782654
19.6282051282051 0.712160050868988
20.5641025641026 0.708537459373474
21.5416666666667 0.71966141462326
22.5673076923077 0.670502722263336
23.6442307692308 0.675801575183868
24.7692307692308 0.722430825233459
25.9487179487179 0.691539704799652
27.1826923076923 0.679106056690216
28.4775641025641 0.644481837749481
29.8333333333333 0.676703870296478
31.2564102564103 0.677815973758698
32.7435897435897 0.662394165992737
34.3044871794872 0.585131764411926
35.9358974358974 0.593388736248016
37.6474358974359 0.601539433002472
39.4391025641026 0.586460292339325
41.3173076923077 0.527765691280365
43.2852564102564 0.563214004039764
45.3461538461538 0.503712356090546
47.5064102564103 0.531544148921967
49.7692307692308 0.531743943691254
52.1378205128205 0.483150869607925
54.6217948717949 0.520426452159882
57.2211538461538 0.445996850728989
59.9455128205128 0.488213866949081
62.8012820512821 0.411276251077652
65.7916666666667 0.443762362003326
68.9230769230769 0.447615712881088
72.2051282051282 0.410937458276749
75.6442307692308 0.370191097259521
79.2467948717949 0.346092462539673
83.0192307692308 0.437259286642075
86.974358974359 0.347940772771835
91.1153846153846 0.33807760477066
95.4519230769231 0.378379315137863
100 0.385885238647461
};
\addplot [, black, opacity=0.6, mark=*, mark size=0.5, mark options={solid}, only marks, forget plot]
table {%
1 0.98374742269516
1.04487179487179 0.99119645357132
1.09615384615385 0.993699729442596
1.1474358974359 0.99211448431015
1.20192307692308 0.990588963031769
1.25961538461538 0.986877083778381
1.32051282051282 0.97865241765976
1.38461538461538 0.972180366516113
1.44871794871795 0.95185375213623
1.51923076923077 0.970642864704132
1.58974358974359 0.96730625629425
1.66666666666667 0.977256953716278
1.74679487179487 0.942020893096924
1.83012820512821 0.953889489173889
1.91666666666667 0.960329711437225
2.00641025641026 0.972863674163818
2.1025641025641 0.954428672790527
2.20512820512821 0.941209018230438
2.30769230769231 0.938596367835999
2.41987179487179 0.945420861244202
2.53525641025641 0.92853319644928
2.65384615384615 0.896694481372833
2.78205128205128 0.949042141437531
2.91346153846154 0.932797908782959
3.05128205128205 0.931452929973602
3.19871794871795 0.939801871776581
3.34935897435897 0.919247806072235
3.50961538461538 0.930974185466766
3.67628205128205 0.929368615150452
3.8525641025641 0.92060798406601
4.03525641025641 0.896500587463379
4.2275641025641 0.896802723407745
4.42948717948718 0.908545315265656
4.64102564102564 0.859995663166046
4.86217948717949 0.881321370601654
5.09294871794872 0.890502572059631
5.33653846153846 0.845744609832764
5.58974358974359 0.849791526794434
5.85576923076923 0.823348045349121
6.13461538461539 0.846448540687561
6.42628205128205 0.854026317596436
6.73397435897436 0.871932685375214
7.05448717948718 0.865126073360443
7.38782051282051 0.870404899120331
7.74038461538461 0.837932050228119
8.10897435897436 0.855006158351898
8.49679487179487 0.83854866027832
8.90064102564103 0.777332901954651
9.32371794871795 0.803152740001678
9.76923076923077 0.85256427526474
10.2339743589744 0.77681827545166
10.7211538461538 0.762219071388245
11.2307692307692 0.827032208442688
11.7660256410256 0.762560188770294
12.3269230769231 0.791698157787323
12.9134615384615 0.779714524745941
13.5288461538462 0.720747232437134
14.1730769230769 0.766886711120605
14.849358974359 0.783985257148743
15.5544871794872 0.681840837001801
16.2948717948718 0.746723890304565
17.0705128205128 0.732045769691467
17.8846153846154 0.719694912433624
18.7371794871795 0.711341559886932
19.6282051282051 0.666857838630676
20.5641025641026 0.634184539318085
21.5416666666667 0.637069165706635
22.5673076923077 0.70036906003952
23.6442307692308 0.615962445735931
24.7692307692308 0.640998721122742
25.9487179487179 0.668058335781097
27.1826923076923 0.653937339782715
28.4775641025641 0.580042839050293
29.8333333333333 0.623236477375031
31.2564102564103 0.546408355236053
32.7435897435897 0.630167126655579
34.3044871794872 0.565100312232971
35.9358974358974 0.522231757640839
37.6474358974359 0.509883403778076
39.4391025641026 0.487043231725693
41.3173076923077 0.491316556930542
43.2852564102564 0.464536666870117
45.3461538461538 0.469104677438736
47.5064102564103 0.46718493103981
49.7692307692308 0.463799208402634
52.1378205128205 0.467945963144302
54.6217948717949 0.437769860029221
57.2211538461538 0.466181248426437
59.9455128205128 0.439740240573883
62.8012820512821 0.377387911081314
65.7916666666667 0.388364851474762
68.9230769230769 0.362410694360733
72.2051282051282 0.351156383752823
75.6442307692308 0.339453905820847
79.2467948717949 0.348567098379135
83.0192307692308 0.412407159805298
86.974358974359 0.329574853181839
91.1153846153846 0.317077040672302
95.4519230769231 0.365579962730408
100 0.345400899648666
};
\end{axis}

\end{tikzpicture}

      \tikzexternaldisable
    \end{minipage}\hfill
    \begin{minipage}{0.50\linewidth}
      \centering
      % defines the pgfplots style "eigspacedefault"
\pgfkeys{/pgfplots/eigspacedefault/.style={
    width=1.0\linewidth,
    height=0.6\linewidth,
    every axis plot/.append style={line width = 1.5pt},
    tick pos = left,
    ylabel near ticks,
    xlabel near ticks,
    xtick align = inside,
    ytick align = inside,
    legend cell align = left,
    legend columns = 4,
    legend pos = south east,
    legend style = {
      fill opacity = 1,
      text opacity = 1,
      font = \footnotesize,
      at={(1, 1.025)},
      anchor=south east,
      column sep=0.25cm,
    },
    legend image post style={scale=2.5},
    xticklabel style = {font = \footnotesize},
    xlabel style = {font = \footnotesize},
    axis line style = {black},
    yticklabel style = {font = \footnotesize},
    ylabel style = {font = \footnotesize},
    title style = {font = \footnotesize},
    grid = major,
    grid style = {dashed}
  }
}

\pgfkeys{/pgfplots/eigspacedefaultapp/.style={
    eigspacedefault,
    height=0.6\linewidth,
    legend columns = 2,
  }
}

\pgfkeys{/pgfplots/eigspacenolegend/.style={
    legend image post style = {scale=0},
    legend style = {
      fill opacity = 0,
      draw opacity = 0,
      text opacity = 0,
      font = \footnotesize,
      at={(1, 1.025)},
      anchor=south east,
      column sep=0.25cm,
    },
  }
}
%%% Local Variables:
%%% mode: latex
%%% TeX-master: "../../thesis"
%%% End:

      \pgfkeys{/pgfplots/zmystyle/.style={
          eigspacedefaultapp,
          eigspacenolegend,
        }}
      \tikzexternalenable
      \vspace{-6ex}
      % This file was created by tikzplotlib v0.9.7.
\begin{tikzpicture}

\definecolor{color0}{rgb}{0.274509803921569,0.6,0.564705882352941}
\definecolor{color1}{rgb}{0.870588235294118,0.623529411764706,0.0862745098039216}
\definecolor{color2}{rgb}{0.501960784313725,0.184313725490196,0.6}

\begin{axis}[
axis line style={white!10!black},
legend columns=2,
legend style={fill opacity=0.8, draw opacity=1, text opacity=1, at={(0.03,0.03)}, anchor=south west, draw=white!80!black},
log basis x={10},
tick pos=left,
xlabel={epoch (log scale)},
xmajorgrids,
xmin=0.794328234724281, xmax=125.892541179417,
xmode=log,
ylabel={overlap},
ymajorgrids,
ymin=-0.05, ymax=1.05,
zmystyle
]
\addplot [, white!10!black, dashed, forget plot]
table {%
0.794328234724281 1
125.892541179417 1
};
\addplot [, white!10!black, dashed, forget plot]
table {%
0.794328234724281 0
125.892541179417 0
};
\addplot [, black, opacity=0.6, mark=*, mark size=0.5, mark options={solid}, only marks]
table {%
1 0.980172097682953
1.04487179487179 0.991716206073761
1.09615384615385 0.992607712745667
1.1474358974359 0.989956080913544
1.20192307692308 0.990800857543945
1.25961538461538 0.907791912555695
1.32051282051282 0.961326599121094
1.38461538461538 0.88730925321579
1.44871794871795 0.947700977325439
1.51923076923077 0.937219440937042
1.58974358974359 0.949622809886932
1.66666666666667 0.968756198883057
1.74679487179487 0.887838780879974
1.83012820512821 0.916164219379425
1.91666666666667 0.947528004646301
2.00641025641026 0.922192215919495
2.1025641025641 0.928771018981934
2.20512820512821 0.897602498531342
2.30769230769231 0.870130360126495
2.41987179487179 0.878030240535736
2.53525641025641 0.866652131080627
2.65384615384615 0.909153580665588
2.78205128205128 0.937457025051117
2.91346153846154 0.9390869140625
3.05128205128205 0.866176545619965
3.19871794871795 0.88209742307663
3.34935897435897 0.900237679481506
3.50961538461538 0.905459046363831
3.67628205128205 0.9186971783638
3.8525641025641 0.922338128089905
4.03525641025641 0.842959880828857
4.2275641025641 0.896037757396698
4.42948717948718 0.919768989086151
4.64102564102564 0.837957322597504
4.86217948717949 0.866225063800812
5.09294871794872 0.874295353889465
5.33653846153846 0.857604801654816
5.58974358974359 0.839182078838348
5.85576923076923 0.831863701343536
6.13461538461539 0.827021539211273
6.42628205128205 0.851464450359344
6.73397435897436 0.853148102760315
7.05448717948718 0.827787399291992
7.38782051282051 0.85309374332428
7.74038461538461 0.840215861797333
8.10897435897436 0.842476665973663
8.49679487179487 0.846230506896973
8.90064102564103 0.825722634792328
9.32371794871795 0.853772282600403
9.76923076923077 0.857580304145813
10.2339743589744 0.806088864803314
10.7211538461538 0.82514476776123
11.2307692307692 0.816884994506836
11.7660256410256 0.813710987567902
12.3269230769231 0.778618633747101
12.9134615384615 0.794024109840393
13.5288461538462 0.733671605587006
14.1730769230769 0.75925225019455
14.849358974359 0.770133435726166
15.5544871794872 0.770629107952118
16.2948717948718 0.775415062904358
17.0705128205128 0.771032631397247
17.8846153846154 0.748296916484833
18.7371794871795 0.683089137077332
19.6282051282051 0.702462375164032
20.5641025641026 0.673430383205414
21.5416666666667 0.739798545837402
22.5673076923077 0.656615376472473
23.6442307692308 0.712085545063019
24.7692307692308 0.661113440990448
25.9487179487179 0.686876118183136
27.1826923076923 0.673646330833435
28.4775641025641 0.661878705024719
29.8333333333333 0.658248126506805
31.2564102564103 0.601502895355225
32.7435897435897 0.635036945343018
34.3044871794872 0.598907768726349
35.9358974358974 0.649761617183685
37.6474358974359 0.624386310577393
39.4391025641026 0.583890557289124
41.3173076923077 0.602699160575867
43.2852564102564 0.562650620937347
45.3461538461538 0.520900547504425
47.5064102564103 0.571963310241699
49.7692307692308 0.508267223834991
52.1378205128205 0.49919381737709
54.6217948717949 0.480009287595749
57.2211538461538 0.501744508743286
59.9455128205128 0.439036667346954
62.8012820512821 0.425910860300064
65.7916666666667 0.406243652105331
68.9230769230769 0.426879316568375
72.2051282051282 0.436816990375519
75.6442307692308 0.393632560968399
79.2467948717949 0.367149382829666
83.0192307692308 0.35765129327774
86.974358974359 0.372109144926071
91.1153846153846 0.358631610870361
95.4519230769231 0.363734215497971
100 0.373095273971558
};
\addlegendentry{mb 128, exact}
\addplot [, black, opacity=0.6, mark=*, mark size=0.5, mark options={solid}, only marks, forget plot]
table {%
1 0.980207860469818
1.04487179487179 0.986687839031219
1.09615384615385 0.99109822511673
1.1474358974359 0.988957226276398
1.20192307692308 0.989828884601593
1.25961538461538 0.981170475482941
1.32051282051282 0.944415390491486
1.38461538461538 0.97414493560791
1.44871794871795 0.977206408977509
1.51923076923077 0.970506191253662
1.58974358974359 0.97232574224472
1.66666666666667 0.960947215557098
1.74679487179487 0.94966596364975
1.83012820512821 0.947515308856964
1.91666666666667 0.953044831752777
2.00641025641026 0.931983470916748
2.1025641025641 0.938220322132111
2.20512820512821 0.923636555671692
2.30769230769231 0.946711838245392
2.41987179487179 0.919902265071869
2.53525641025641 0.92106819152832
2.65384615384615 0.876817166805267
2.78205128205128 0.888949573040009
2.91346153846154 0.922625064849854
3.05128205128205 0.911627113819122
3.19871794871795 0.863408267498016
3.34935897435897 0.921932399272919
3.50961538461538 0.911691844463348
3.67628205128205 0.869503676891327
3.8525641025641 0.902216255664825
4.03525641025641 0.885700643062592
4.2275641025641 0.849653422832489
4.42948717948718 0.834793746471405
4.64102564102564 0.833202004432678
4.86217948717949 0.89985865354538
5.09294871794872 0.843555629253387
5.33653846153846 0.880906879901886
5.58974358974359 0.910552024841309
5.85576923076923 0.877913773059845
6.13461538461539 0.821998298168182
6.42628205128205 0.865674018859863
6.73397435897436 0.856091499328613
7.05448717948718 0.854511678218842
7.38782051282051 0.867451846599579
7.74038461538461 0.811384618282318
8.10897435897436 0.835431277751923
8.49679487179487 0.874100983142853
8.90064102564103 0.797595322132111
9.32371794871795 0.824461162090302
9.76923076923077 0.82661360502243
10.2339743589744 0.82679957151413
10.7211538461538 0.822871804237366
11.2307692307692 0.773120164871216
11.7660256410256 0.766072332859039
12.3269230769231 0.809707462787628
12.9134615384615 0.796929776668549
13.5288461538462 0.747281968593597
14.1730769230769 0.740639626979828
14.849358974359 0.753473103046417
15.5544871794872 0.73205977678299
16.2948717948718 0.766981542110443
17.0705128205128 0.712304592132568
17.8846153846154 0.791382789611816
18.7371794871795 0.703667104244232
19.6282051282051 0.689324736595154
20.5641025641026 0.644148707389832
21.5416666666667 0.634638488292694
22.5673076923077 0.67344206571579
23.6442307692308 0.653737664222717
24.7692307692308 0.664027214050293
25.9487179487179 0.686160087585449
27.1826923076923 0.632227838039398
28.4775641025641 0.638557136058807
29.8333333333333 0.659567296504974
31.2564102564103 0.581326365470886
32.7435897435897 0.600731611251831
34.3044871794872 0.56149685382843
35.9358974358974 0.593187272548676
37.6474358974359 0.560176849365234
39.4391025641026 0.536633670330048
41.3173076923077 0.543563067913055
43.2852564102564 0.568399846553802
45.3461538461538 0.481604427099228
47.5064102564103 0.538267314434052
49.7692307692308 0.473466485738754
52.1378205128205 0.518011033535004
54.6217948717949 0.448529452085495
57.2211538461538 0.464230448007584
59.9455128205128 0.472301870584488
62.8012820512821 0.444409996271133
65.7916666666667 0.427443742752075
68.9230769230769 0.413966983556747
72.2051282051282 0.409690469503403
75.6442307692308 0.459181159734726
79.2467948717949 0.353957533836365
83.0192307692308 0.398236364126205
86.974358974359 0.31950780749321
91.1153846153846 0.304738879203796
95.4519230769231 0.313452303409576
100 0.317403376102448
};
\addplot [, black, opacity=0.6, mark=*, mark size=0.5, mark options={solid}, only marks, forget plot]
table {%
1 0.983946800231934
1.04487179487179 0.988971710205078
1.09615384615385 0.992758393287659
1.1474358974359 0.990560352802277
1.20192307692308 0.989403367042542
1.25961538461538 0.973469197750092
1.32051282051282 0.899483203887939
1.38461538461538 0.972841441631317
1.44871794871795 0.971960842609406
1.51923076923077 0.968037068843842
1.58974358974359 0.967146873474121
1.66666666666667 0.912840843200684
1.74679487179487 0.959828197956085
1.83012820512821 0.939577400684357
1.91666666666667 0.936375319957733
2.00641025641026 0.937602162361145
2.1025641025641 0.946089446544647
2.20512820512821 0.927560746669769
2.30769230769231 0.925248324871063
2.41987179487179 0.926818192005157
2.53525641025641 0.936004757881165
2.65384615384615 0.900654315948486
2.78205128205128 0.908032417297363
2.91346153846154 0.909767746925354
3.05128205128205 0.936206161975861
3.19871794871795 0.888727605342865
3.34935897435897 0.905566990375519
3.50961538461538 0.903665363788605
3.67628205128205 0.893167674541473
3.8525641025641 0.904410183429718
4.03525641025641 0.899736523628235
4.2275641025641 0.855421245098114
4.42948717948718 0.84848165512085
4.64102564102564 0.849940001964569
4.86217948717949 0.838516414165497
5.09294871794872 0.821752548217773
5.33653846153846 0.874599099159241
5.58974358974359 0.886141002178192
5.85576923076923 0.879460513591766
6.13461538461539 0.883392512798309
6.42628205128205 0.852098405361176
6.73397435897436 0.825584411621094
7.05448717948718 0.805654466152191
7.38782051282051 0.85651171207428
7.74038461538461 0.826787769794464
8.10897435897436 0.813583970069885
8.49679487179487 0.853635966777802
8.90064102564103 0.787913739681244
9.32371794871795 0.808611512184143
9.76923076923077 0.77556037902832
10.2339743589744 0.760887801647186
10.7211538461538 0.785118699073792
11.2307692307692 0.791158854961395
11.7660256410256 0.775297701358795
12.3269230769231 0.764155805110931
12.9134615384615 0.810503602027893
13.5288461538462 0.77249950170517
14.1730769230769 0.756299197673798
14.849358974359 0.724047720432281
15.5544871794872 0.702045857906342
16.2948717948718 0.755908310413361
17.0705128205128 0.690865457057953
17.8846153846154 0.739099204540253
18.7371794871795 0.696106493473053
19.6282051282051 0.656948268413544
20.5641025641026 0.730032682418823
21.5416666666667 0.692141950130463
22.5673076923077 0.69680780172348
23.6442307692308 0.670672595500946
24.7692307692308 0.637449860572815
25.9487179487179 0.644334495067596
27.1826923076923 0.635448336601257
28.4775641025641 0.646008908748627
29.8333333333333 0.639594495296478
31.2564102564103 0.57536393404007
32.7435897435897 0.579100787639618
34.3044871794872 0.598610639572144
35.9358974358974 0.641373157501221
37.6474358974359 0.558124184608459
39.4391025641026 0.518050730228424
41.3173076923077 0.534603893756866
43.2852564102564 0.511987686157227
45.3461538461538 0.50470495223999
47.5064102564103 0.483739674091339
49.7692307692308 0.471539467573166
52.1378205128205 0.392315179109573
54.6217948717949 0.452086359262466
57.2211538461538 0.38785719871521
59.9455128205128 0.432793229818344
62.8012820512821 0.401318460702896
65.7916666666667 0.384503036737442
68.9230769230769 0.244176179170609
72.2051282051282 0.297735422849655
75.6442307692308 0.372241079807281
79.2467948717949 0.284189194440842
83.0192307692308 0.352907985448837
86.974358974359 0.332353800535202
91.1153846153846 0.307710289955139
95.4519230769231 0.33690333366394
100 0.282698422670364
};
\addplot [, black, opacity=0.6, mark=*, mark size=0.5, mark options={solid}, only marks, forget plot]
table {%
1 0.982988655567169
1.04487179487179 0.992447853088379
1.09615384615385 0.993656158447266
1.1474358974359 0.990851044654846
1.20192307692308 0.989253461360931
1.25961538461538 0.947678983211517
1.32051282051282 0.947290420532227
1.38461538461538 0.93720954656601
1.44871794871795 0.948299825191498
1.51923076923077 0.963698089122772
1.58974358974359 0.967891335487366
1.66666666666667 0.898256957530975
1.74679487179487 0.940048694610596
1.83012820512821 0.948034584522247
1.91666666666667 0.927294194698334
2.00641025641026 0.900033593177795
2.1025641025641 0.879429519176483
2.20512820512821 0.871008992195129
2.30769230769231 0.928948998451233
2.41987179487179 0.922504425048828
2.53525641025641 0.897674560546875
2.65384615384615 0.855482757091522
2.78205128205128 0.916048645973206
2.91346153846154 0.892417430877686
3.05128205128205 0.895148456096649
3.19871794871795 0.88008588552475
3.34935897435897 0.883312880992889
3.50961538461538 0.869988441467285
3.67628205128205 0.875664055347443
3.8525641025641 0.895642876625061
4.03525641025641 0.853973388671875
4.2275641025641 0.823393523693085
4.42948717948718 0.86317777633667
4.64102564102564 0.853497803211212
4.86217948717949 0.854592263698578
5.09294871794872 0.832616984844208
5.33653846153846 0.821280896663666
5.58974358974359 0.801458775997162
5.85576923076923 0.821565270423889
6.13461538461539 0.808993637561798
6.42628205128205 0.83081191778183
6.73397435897436 0.827543199062347
7.05448717948718 0.815712153911591
7.38782051282051 0.772370278835297
7.74038461538461 0.788699567317963
8.10897435897436 0.818086802959442
8.49679487179487 0.793797016143799
8.90064102564103 0.739077568054199
9.32371794871795 0.76023668050766
9.76923076923077 0.78579705953598
10.2339743589744 0.788220643997192
10.7211538461538 0.775658845901489
11.2307692307692 0.789422690868378
11.7660256410256 0.799331724643707
12.3269230769231 0.750704288482666
12.9134615384615 0.782992780208588
13.5288461538462 0.727857768535614
14.1730769230769 0.753467559814453
14.849358974359 0.715115964412689
15.5544871794872 0.71807450056076
16.2948717948718 0.719560086727142
17.0705128205128 0.712568938732147
17.8846153846154 0.773623049259186
18.7371794871795 0.725335478782654
19.6282051282051 0.712160050868988
20.5641025641026 0.708537459373474
21.5416666666667 0.71966141462326
22.5673076923077 0.670502722263336
23.6442307692308 0.675801575183868
24.7692307692308 0.722430825233459
25.9487179487179 0.691539704799652
27.1826923076923 0.679106056690216
28.4775641025641 0.644481837749481
29.8333333333333 0.676703870296478
31.2564102564103 0.677815973758698
32.7435897435897 0.662394165992737
34.3044871794872 0.585131764411926
35.9358974358974 0.593388736248016
37.6474358974359 0.601539433002472
39.4391025641026 0.586460292339325
41.3173076923077 0.527765691280365
43.2852564102564 0.563214004039764
45.3461538461538 0.503712356090546
47.5064102564103 0.531544148921967
49.7692307692308 0.531743943691254
52.1378205128205 0.483150869607925
54.6217948717949 0.520426452159882
57.2211538461538 0.445996850728989
59.9455128205128 0.488213866949081
62.8012820512821 0.411276251077652
65.7916666666667 0.443762362003326
68.9230769230769 0.447615712881088
72.2051282051282 0.410937458276749
75.6442307692308 0.370191097259521
79.2467948717949 0.346092462539673
83.0192307692308 0.437259286642075
86.974358974359 0.347940772771835
91.1153846153846 0.33807760477066
95.4519230769231 0.378379315137863
100 0.385885238647461
};
\addplot [, black, opacity=0.6, mark=*, mark size=0.5, mark options={solid}, only marks, forget plot]
table {%
1 0.98374742269516
1.04487179487179 0.99119645357132
1.09615384615385 0.993699729442596
1.1474358974359 0.99211448431015
1.20192307692308 0.990588963031769
1.25961538461538 0.986877083778381
1.32051282051282 0.97865241765976
1.38461538461538 0.972180366516113
1.44871794871795 0.95185375213623
1.51923076923077 0.970642864704132
1.58974358974359 0.96730625629425
1.66666666666667 0.977256953716278
1.74679487179487 0.942020893096924
1.83012820512821 0.953889489173889
1.91666666666667 0.960329711437225
2.00641025641026 0.972863674163818
2.1025641025641 0.954428672790527
2.20512820512821 0.941209018230438
2.30769230769231 0.938596367835999
2.41987179487179 0.945420861244202
2.53525641025641 0.92853319644928
2.65384615384615 0.896694481372833
2.78205128205128 0.949042141437531
2.91346153846154 0.932797908782959
3.05128205128205 0.931452929973602
3.19871794871795 0.939801871776581
3.34935897435897 0.919247806072235
3.50961538461538 0.930974185466766
3.67628205128205 0.929368615150452
3.8525641025641 0.92060798406601
4.03525641025641 0.896500587463379
4.2275641025641 0.896802723407745
4.42948717948718 0.908545315265656
4.64102564102564 0.859995663166046
4.86217948717949 0.881321370601654
5.09294871794872 0.890502572059631
5.33653846153846 0.845744609832764
5.58974358974359 0.849791526794434
5.85576923076923 0.823348045349121
6.13461538461539 0.846448540687561
6.42628205128205 0.854026317596436
6.73397435897436 0.871932685375214
7.05448717948718 0.865126073360443
7.38782051282051 0.870404899120331
7.74038461538461 0.837932050228119
8.10897435897436 0.855006158351898
8.49679487179487 0.83854866027832
8.90064102564103 0.777332901954651
9.32371794871795 0.803152740001678
9.76923076923077 0.85256427526474
10.2339743589744 0.77681827545166
10.7211538461538 0.762219071388245
11.2307692307692 0.827032208442688
11.7660256410256 0.762560188770294
12.3269230769231 0.791698157787323
12.9134615384615 0.779714524745941
13.5288461538462 0.720747232437134
14.1730769230769 0.766886711120605
14.849358974359 0.783985257148743
15.5544871794872 0.681840837001801
16.2948717948718 0.746723890304565
17.0705128205128 0.732045769691467
17.8846153846154 0.719694912433624
18.7371794871795 0.711341559886932
19.6282051282051 0.666857838630676
20.5641025641026 0.634184539318085
21.5416666666667 0.637069165706635
22.5673076923077 0.70036906003952
23.6442307692308 0.615962445735931
24.7692307692308 0.640998721122742
25.9487179487179 0.668058335781097
27.1826923076923 0.653937339782715
28.4775641025641 0.580042839050293
29.8333333333333 0.623236477375031
31.2564102564103 0.546408355236053
32.7435897435897 0.630167126655579
34.3044871794872 0.565100312232971
35.9358974358974 0.522231757640839
37.6474358974359 0.509883403778076
39.4391025641026 0.487043231725693
41.3173076923077 0.491316556930542
43.2852564102564 0.464536666870117
45.3461538461538 0.469104677438736
47.5064102564103 0.46718493103981
49.7692307692308 0.463799208402634
52.1378205128205 0.467945963144302
54.6217948717949 0.437769860029221
57.2211538461538 0.466181248426437
59.9455128205128 0.439740240573883
62.8012820512821 0.377387911081314
65.7916666666667 0.388364851474762
68.9230769230769 0.362410694360733
72.2051282051282 0.351156383752823
75.6442307692308 0.339453905820847
79.2467948717949 0.348567098379135
83.0192307692308 0.412407159805298
86.974358974359 0.329574853181839
91.1153846153846 0.317077040672302
95.4519230769231 0.365579962730408
100 0.345400899648666
};
\addplot [, color0, opacity=0.6, mark=diamond*, mark size=0.5, mark options={solid}, only marks]
table {%
1 0.860992848873138
1.04487179487179 0.892344951629639
1.09615384615385 0.823918163776398
1.1474358974359 0.7441645860672
1.20192307692308 0.801132202148438
1.25961538461538 0.738307893276215
1.32051282051282 0.737305164337158
1.38461538461538 0.725374221801758
1.44871794871795 0.722569942474365
1.51923076923077 0.74552184343338
1.58974358974359 0.746620118618011
1.66666666666667 0.730974614620209
1.74679487179487 0.730906009674072
1.83012820512821 0.710865914821625
1.91666666666667 0.713170826435089
2.00641025641026 0.682346761226654
2.1025641025641 0.661310136318207
2.20512820512821 0.673846244812012
2.30769230769231 0.63198322057724
2.41987179487179 0.653586387634277
2.53525641025641 0.639173626899719
2.65384615384615 0.661026537418365
2.78205128205128 0.621147930622101
2.91346153846154 0.581974685192108
3.05128205128205 0.609411656856537
3.19871794871795 0.637974083423615
3.34935897435897 0.579538762569427
3.50961538461538 0.593585312366486
3.67628205128205 0.597751557826996
3.8525641025641 0.562667667865753
4.03525641025641 0.553296685218811
4.2275641025641 0.549773633480072
4.42948717948718 0.55252343416214
4.64102564102564 0.539897382259369
4.86217948717949 0.516569912433624
5.09294871794872 0.514274299144745
5.33653846153846 0.557376563549042
5.58974358974359 0.485806524753571
5.85576923076923 0.518746972084045
6.13461538461539 0.483038514852524
6.42628205128205 0.512336313724518
6.73397435897436 0.511140167713165
7.05448717948718 0.479569166898727
7.38782051282051 0.479413419961929
7.74038461538461 0.488739967346191
8.10897435897436 0.464967638254166
8.49679487179487 0.489871591329575
8.90064102564103 0.471107870340347
9.32371794871795 0.447679758071899
9.76923076923077 0.435437977313995
10.2339743589744 0.452548325061798
10.7211538461538 0.415754705667496
11.2307692307692 0.413579136133194
11.7660256410256 0.417902559041977
12.3269230769231 0.413538545370102
12.9134615384615 0.398396730422974
13.5288461538462 0.377598613500595
14.1730769230769 0.406299114227295
14.849358974359 0.372663199901581
15.5544871794872 0.363748759031296
16.2948717948718 0.364039808511734
17.0705128205128 0.365611404180527
17.8846153846154 0.345699548721313
18.7371794871795 0.302672922611237
19.6282051282051 0.3581922352314
20.5641025641026 0.336457967758179
21.5416666666667 0.348222225904465
22.5673076923077 0.361511439085007
23.6442307692308 0.312443792819977
24.7692307692308 0.340276718139648
25.9487179487179 0.351905554533005
27.1826923076923 0.323430836200714
28.4775641025641 0.333444774150848
29.8333333333333 0.304251849651337
31.2564102564103 0.347670465707779
32.7435897435897 0.320370823144913
34.3044871794872 0.31298753619194
35.9358974358974 0.313371866941452
37.6474358974359 0.344379812479019
39.4391025641026 0.305190354585648
41.3173076923077 0.339858621358871
43.2852564102564 0.283608019351959
45.3461538461538 0.270850449800491
47.5064102564103 0.331485599279404
49.7692307692308 0.302991956472397
52.1378205128205 0.281433582305908
54.6217948717949 0.315260171890259
57.2211538461538 0.262642502784729
59.9455128205128 0.290677845478058
62.8012820512821 0.313325792551041
65.7916666666667 0.266252160072327
68.9230769230769 0.30434376001358
72.2051282051282 0.229493170976639
75.6442307692308 0.249063447117805
79.2467948717949 0.269422829151154
83.0192307692308 0.310287296772003
86.974358974359 0.246017098426819
91.1153846153846 0.28729447722435
95.4519230769231 0.244613796472549
100 0.265447944402695
};
\addlegendentry{sub 16, exact}
\addplot [, color0, opacity=0.6, mark=diamond*, mark size=0.5, mark options={solid}, only marks, forget plot]
table {%
1 0.860444664955139
1.04487179487179 0.928392887115479
1.09615384615385 0.931609630584717
1.1474358974359 0.915550053119659
1.20192307692308 0.923747658729553
1.25961538461538 0.81793624162674
1.32051282051282 0.789078176021576
1.38461538461538 0.815545260906219
1.44871794871795 0.801478862762451
1.51923076923077 0.736773133277893
1.58974358974359 0.80506831407547
1.66666666666667 0.749329090118408
1.74679487179487 0.767842710018158
1.83012820512821 0.691979050636292
1.91666666666667 0.704763889312744
2.00641025641026 0.679268062114716
2.1025641025641 0.691880881786346
2.20512820512821 0.680962264537811
2.30769230769231 0.646706104278564
2.41987179487179 0.629142642021179
2.53525641025641 0.663328766822815
2.65384615384615 0.614461362361908
2.78205128205128 0.638552904129028
2.91346153846154 0.636721014976501
3.05128205128205 0.624222695827484
3.19871794871795 0.606026291847229
3.34935897435897 0.6509108543396
3.50961538461538 0.593663334846497
3.67628205128205 0.571529567241669
3.8525641025641 0.589094936847687
4.03525641025641 0.571939468383789
4.2275641025641 0.538677930831909
4.42948717948718 0.577403485774994
4.64102564102564 0.487995713949203
4.86217948717949 0.51754355430603
5.09294871794872 0.489810705184937
5.33653846153846 0.491236686706543
5.58974358974359 0.513736546039581
5.85576923076923 0.446437567472458
6.13461538461539 0.478004068136215
6.42628205128205 0.477623760700226
6.73397435897436 0.467316448688507
7.05448717948718 0.502695381641388
7.38782051282051 0.4743971824646
7.74038461538461 0.491034656763077
8.10897435897436 0.446462541818619
8.49679487179487 0.457213789224625
8.90064102564103 0.388940781354904
9.32371794871795 0.442613422870636
9.76923076923077 0.430276721715927
10.2339743589744 0.416165560483932
10.7211538461538 0.407509654760361
11.2307692307692 0.402035623788834
11.7660256410256 0.411157608032227
12.3269230769231 0.395867139101028
12.9134615384615 0.390715450048447
13.5288461538462 0.373276740312576
14.1730769230769 0.381951957941055
14.849358974359 0.407582819461823
15.5544871794872 0.383623600006104
16.2948717948718 0.373151868581772
17.0705128205128 0.374163955450058
17.8846153846154 0.40241551399231
18.7371794871795 0.347761392593384
19.6282051282051 0.304075241088867
20.5641025641026 0.352734059095383
21.5416666666667 0.375276327133179
22.5673076923077 0.392354637384415
23.6442307692308 0.355653643608093
24.7692307692308 0.340053766965866
25.9487179487179 0.312725871801376
27.1826923076923 0.309933662414551
28.4775641025641 0.279562562704086
29.8333333333333 0.306813716888428
31.2564102564103 0.276081442832947
32.7435897435897 0.251543939113617
34.3044871794872 0.289635092020035
35.9358974358974 0.267010658979416
37.6474358974359 0.244917392730713
39.4391025641026 0.239098504185677
41.3173076923077 0.265054553747177
43.2852564102564 0.26198336482048
45.3461538461538 0.233927875757217
47.5064102564103 0.244653224945068
49.7692307692308 0.247437506914139
52.1378205128205 0.253242939710617
54.6217948717949 0.234389111399651
57.2211538461538 0.179333835840225
59.9455128205128 0.215434908866882
62.8012820512821 0.223949149250984
65.7916666666667 0.220515176653862
68.9230769230769 0.229365661740303
72.2051282051282 0.201969012618065
75.6442307692308 0.212255865335464
79.2467948717949 0.224710181355476
83.0192307692308 0.217072889208794
86.974358974359 0.245852038264275
91.1153846153846 0.189456537365913
95.4519230769231 0.211614564061165
100 0.207913503050804
};
\addplot [, color0, opacity=0.6, mark=diamond*, mark size=0.5, mark options={solid}, only marks, forget plot]
table {%
1 0.890892803668976
1.04487179487179 0.938475251197815
1.09615384615385 0.95172780752182
1.1474358974359 0.93952351808548
1.20192307692308 0.938625454902649
1.25961538461538 0.88451874256134
1.32051282051282 0.817189395427704
1.38461538461538 0.863233745098114
1.44871794871795 0.837721049785614
1.51923076923077 0.792255580425262
1.58974358974359 0.797558665275574
1.66666666666667 0.805702209472656
1.74679487179487 0.788794875144958
1.83012820512821 0.752457559108734
1.91666666666667 0.777341246604919
2.00641025641026 0.757799386978149
2.1025641025641 0.718825995922089
2.20512820512821 0.733715057373047
2.30769230769231 0.730856597423553
2.41987179487179 0.689408898353577
2.53525641025641 0.708274304866791
2.65384615384615 0.698091328144073
2.78205128205128 0.700066685676575
2.91346153846154 0.646445035934448
3.05128205128205 0.668342113494873
3.19871794871795 0.617294311523438
3.34935897435897 0.661685228347778
3.50961538461538 0.642695307731628
3.67628205128205 0.62440550327301
3.8525641025641 0.63926762342453
4.03525641025641 0.638611495494843
4.2275641025641 0.617121458053589
4.42948717948718 0.606868386268616
4.64102564102564 0.586089789867401
4.86217948717949 0.612089693546295
5.09294871794872 0.58807498216629
5.33653846153846 0.5828777551651
5.58974358974359 0.575221717357635
5.85576923076923 0.589239537715912
6.13461538461539 0.5655717253685
6.42628205128205 0.521285951137543
6.73397435897436 0.592981994152069
7.05448717948718 0.572408974170685
7.38782051282051 0.549133121967316
7.74038461538461 0.548787772655487
8.10897435897436 0.515187859535217
8.49679487179487 0.572144150733948
8.90064102564103 0.501758754253387
9.32371794871795 0.474455803632736
9.76923076923077 0.497018337249756
10.2339743589744 0.469075977802277
10.7211538461538 0.526267051696777
11.2307692307692 0.506640434265137
11.7660256410256 0.493066042661667
12.3269230769231 0.470579475164413
12.9134615384615 0.515702426433563
13.5288461538462 0.451410204172134
14.1730769230769 0.428697645664215
14.849358974359 0.426507532596588
15.5544871794872 0.372526377439499
16.2948717948718 0.364423036575317
17.0705128205128 0.445634365081787
17.8846153846154 0.423056095838547
18.7371794871795 0.340795248746872
19.6282051282051 0.355673044919968
20.5641025641026 0.332795381546021
21.5416666666667 0.350276231765747
22.5673076923077 0.318451255559921
23.6442307692308 0.299259573221207
24.7692307692308 0.321578323841095
25.9487179487179 0.308305889368057
27.1826923076923 0.297343611717224
28.4775641025641 0.285895824432373
29.8333333333333 0.315602153539658
31.2564102564103 0.292469888925552
32.7435897435897 0.285817533731461
34.3044871794872 0.335625618696213
35.9358974358974 0.257306694984436
37.6474358974359 0.341187298297882
39.4391025641026 0.365381836891174
41.3173076923077 0.270805239677429
43.2852564102564 0.28551658987999
45.3461538461538 0.278515815734863
47.5064102564103 0.268705695867538
49.7692307692308 0.279886484146118
52.1378205128205 0.268429487943649
54.6217948717949 0.292847514152527
57.2211538461538 0.299233168363571
59.9455128205128 0.216808080673218
62.8012820512821 0.261133283376694
65.7916666666667 0.274537444114685
68.9230769230769 0.233089923858643
72.2051282051282 0.225452408194542
75.6442307692308 0.228130534291267
79.2467948717949 0.1875329464674
83.0192307692308 0.252193838357925
86.974358974359 0.218955546617508
91.1153846153846 0.241251900792122
95.4519230769231 0.170218393206596
100 0.220372557640076
};
\addplot [, color0, opacity=0.6, mark=diamond*, mark size=0.5, mark options={solid}, only marks, forget plot]
table {%
1 0.87019681930542
1.04487179487179 0.913722038269043
1.09615384615385 0.941634953022003
1.1474358974359 0.92918872833252
1.20192307692308 0.927006185054779
1.25961538461538 0.823446452617645
1.32051282051282 0.824841320514679
1.38461538461538 0.824603855609894
1.44871794871795 0.819951832294464
1.51923076923077 0.784321963787079
1.58974358974359 0.775236785411835
1.66666666666667 0.76844722032547
1.74679487179487 0.773922026157379
1.83012820512821 0.779536068439484
1.91666666666667 0.786202371120453
2.00641025641026 0.778388142585754
2.1025641025641 0.719281673431396
2.20512820512821 0.691725730895996
2.30769230769231 0.712027370929718
2.41987179487179 0.667527794837952
2.53525641025641 0.692149877548218
2.65384615384615 0.66925185918808
2.78205128205128 0.657330214977264
2.91346153846154 0.631011605262756
3.05128205128205 0.652633965015411
3.19871794871795 0.668539881706238
3.34935897435897 0.637573897838593
3.50961538461538 0.625498354434967
3.67628205128205 0.616036534309387
3.8525641025641 0.630297303199768
4.03525641025641 0.63679975271225
4.2275641025641 0.57340544462204
4.42948717948718 0.593968331813812
4.64102564102564 0.577471017837524
4.86217948717949 0.595125615596771
5.09294871794872 0.583095490932465
5.33653846153846 0.576386094093323
5.58974358974359 0.576989352703094
5.85576923076923 0.468805551528931
6.13461538461539 0.540302455425262
6.42628205128205 0.513910233974457
6.73397435897436 0.541025936603546
7.05448717948718 0.548005402088165
7.38782051282051 0.53130030632019
7.74038461538461 0.5748530626297
8.10897435897436 0.508989453315735
8.49679487179487 0.529834270477295
8.90064102564103 0.509609639644623
9.32371794871795 0.522071719169617
9.76923076923077 0.462595671415329
10.2339743589744 0.479950875043869
10.7211538461538 0.491844952106476
11.2307692307692 0.472049087285995
11.7660256410256 0.457933723926544
12.3269230769231 0.394485533237457
12.9134615384615 0.428000599145889
13.5288461538462 0.430046081542969
14.1730769230769 0.387380212545395
14.849358974359 0.403182804584503
15.5544871794872 0.431074440479279
16.2948717948718 0.412648499011993
17.0705128205128 0.390059471130371
17.8846153846154 0.416656076908112
18.7371794871795 0.385417103767395
19.6282051282051 0.375000089406967
20.5641025641026 0.370560646057129
21.5416666666667 0.382299244403839
22.5673076923077 0.397766441106796
23.6442307692308 0.40088739991188
24.7692307692308 0.358211189508438
25.9487179487179 0.419089883565903
27.1826923076923 0.364112466573715
28.4775641025641 0.353068977594376
29.8333333333333 0.35141795873642
31.2564102564103 0.340419977903366
32.7435897435897 0.333000183105469
34.3044871794872 0.369581431150436
35.9358974358974 0.306281536817551
37.6474358974359 0.278300285339355
39.4391025641026 0.295454651117325
41.3173076923077 0.296001881361008
43.2852564102564 0.280763477087021
45.3461538461538 0.301892399787903
47.5064102564103 0.294060915708542
49.7692307692308 0.299918740987778
52.1378205128205 0.286176532506943
54.6217948717949 0.29368194937706
57.2211538461538 0.276437371969223
59.9455128205128 0.262390345335007
62.8012820512821 0.297131538391113
65.7916666666667 0.256762385368347
68.9230769230769 0.220945030450821
72.2051282051282 0.269754260778427
75.6442307692308 0.276794761419296
79.2467948717949 0.275366395711899
83.0192307692308 0.325062245130539
86.974358974359 0.281228631734848
91.1153846153846 0.264606863260269
95.4519230769231 0.262997061014175
100 0.302665621042252
};
\addplot [, color0, opacity=0.6, mark=diamond*, mark size=0.5, mark options={solid}, only marks, forget plot]
table {%
1 0.889948487281799
1.04487179487179 0.93640673160553
1.09615384615385 0.94278758764267
1.1474358974359 0.924996018409729
1.20192307692308 0.928551197052002
1.25961538461538 0.804385125637054
1.32051282051282 0.781125485897064
1.38461538461538 0.784711062908173
1.44871794871795 0.815053164958954
1.51923076923077 0.750312030315399
1.58974358974359 0.76828521490097
1.66666666666667 0.738130748271942
1.74679487179487 0.759201645851135
1.83012820512821 0.706417858600616
1.91666666666667 0.726467311382294
2.00641025641026 0.70868045091629
2.1025641025641 0.687180161476135
2.20512820512821 0.684219002723694
2.30769230769231 0.67307710647583
2.41987179487179 0.607285141944885
2.53525641025641 0.592164158821106
2.65384615384615 0.608381450176239
2.78205128205128 0.659370422363281
2.91346153846154 0.620877206325531
3.05128205128205 0.609286785125732
3.19871794871795 0.61617386341095
3.34935897435897 0.541272103786469
3.50961538461538 0.575032651424408
3.67628205128205 0.559536039829254
3.8525641025641 0.567671239376068
4.03525641025641 0.52724426984787
4.2275641025641 0.540580749511719
4.42948717948718 0.50359833240509
4.64102564102564 0.466407686471939
4.86217948717949 0.524787068367004
5.09294871794872 0.449942797422409
5.33653846153846 0.438492119312286
5.58974358974359 0.448543637990952
5.85576923076923 0.429678648710251
6.13461538461539 0.420867294073105
6.42628205128205 0.443538278341293
6.73397435897436 0.40187931060791
7.05448717948718 0.442580133676529
7.38782051282051 0.404153645038605
7.74038461538461 0.397739857435226
8.10897435897436 0.366039544343948
8.49679487179487 0.403776556253433
8.90064102564103 0.349451959133148
9.32371794871795 0.368320673704147
9.76923076923077 0.372489392757416
10.2339743589744 0.380041271448135
10.7211538461538 0.335601091384888
11.2307692307692 0.348117619752884
11.7660256410256 0.360536903142929
12.3269230769231 0.337315946817398
12.9134615384615 0.330901563167572
13.5288461538462 0.318796157836914
14.1730769230769 0.317127853631973
14.849358974359 0.32819128036499
15.5544871794872 0.281036913394928
16.2948717948718 0.309944033622742
17.0705128205128 0.303658068180084
17.8846153846154 0.324016183614731
18.7371794871795 0.291385173797607
19.6282051282051 0.272760778665543
20.5641025641026 0.237963438034058
21.5416666666667 0.234408140182495
22.5673076923077 0.283042728900909
23.6442307692308 0.278632551431656
24.7692307692308 0.237166866660118
25.9487179487179 0.289711445569992
27.1826923076923 0.272768378257751
28.4775641025641 0.286693304777145
29.8333333333333 0.257146835327148
31.2564102564103 0.266930729150772
32.7435897435897 0.275360673666
34.3044871794872 0.204768419265747
35.9358974358974 0.235359773039818
37.6474358974359 0.190605014562607
39.4391025641026 0.243993237614632
41.3173076923077 0.236428543925285
43.2852564102564 0.204455927014351
45.3461538461538 0.217014417052269
47.5064102564103 0.21175654232502
49.7692307692308 0.208007797598839
52.1378205128205 0.183520570397377
54.6217948717949 0.213924080133438
57.2211538461538 0.221110582351685
59.9455128205128 0.2117590457201
62.8012820512821 0.189344838261604
65.7916666666667 0.179524168372154
68.9230769230769 0.169765502214432
72.2051282051282 0.201682046055794
75.6442307692308 0.173765867948532
79.2467948717949 0.217665299773216
83.0192307692308 0.22142219543457
86.974358974359 0.187772154808044
91.1153846153846 0.144345209002495
95.4519230769231 0.186378717422485
100 0.184396415948868
};
\addplot [, color1, opacity=0.6, mark=square*, mark size=0.5, mark options={solid}, only marks]
table {%
1 0.831718385219574
1.04487179487179 0.91532689332962
1.09615384615385 0.934202969074249
1.1474358974359 0.924113094806671
1.20192307692308 0.860385537147522
1.25961538461538 0.832666993141174
1.32051282051282 0.872560799121857
1.38461538461538 0.813702881336212
1.44871794871795 0.778506696224213
1.51923076923077 0.820350825786591
1.58974358974359 0.811107099056244
1.66666666666667 0.830908596515656
1.74679487179487 0.828177928924561
1.83012820512821 0.821011245250702
1.91666666666667 0.80619353055954
2.00641025641026 0.741983592510223
2.1025641025641 0.797460556030273
2.20512820512821 0.767449796199799
2.30769230769231 0.743414342403412
2.41987179487179 0.781626343727112
2.53525641025641 0.749984741210938
2.65384615384615 0.761724174022675
2.78205128205128 0.741706311702728
2.91346153846154 0.733337759971619
3.05128205128205 0.72812956571579
3.19871794871795 0.708706378936768
3.34935897435897 0.744928240776062
3.50961538461538 0.748154282569885
3.67628205128205 0.719311058521271
3.8525641025641 0.713043034076691
4.03525641025641 0.657571732997894
4.2275641025641 0.658721148967743
4.42948717948718 0.649153232574463
4.64102564102564 0.680747509002686
4.86217948717949 0.703750550746918
5.09294871794872 0.580794930458069
5.33653846153846 0.671773493289948
5.58974358974359 0.642318844795227
5.85576923076923 0.690593600273132
6.13461538461539 0.619023621082306
6.42628205128205 0.593171119689941
6.73397435897436 0.685479462146759
7.05448717948718 0.588415324687958
7.38782051282051 0.626609086990356
7.74038461538461 0.627931535243988
8.10897435897436 0.655384242534637
8.49679487179487 0.585829138755798
8.90064102564103 0.626469135284424
9.32371794871795 0.598152101039886
9.76923076923077 0.616691589355469
10.2339743589744 0.561780393123627
10.7211538461538 0.578710794448853
11.2307692307692 0.539949774742126
11.7660256410256 0.579552173614502
12.3269230769231 0.545426964759827
12.9134615384615 0.557892441749573
13.5288461538462 0.533496499061584
14.1730769230769 0.531644523143768
14.849358974359 0.510430037975311
15.5544871794872 0.545254349708557
16.2948717948718 0.521359086036682
17.0705128205128 0.505151093006134
17.8846153846154 0.551330506801605
18.7371794871795 0.398896545171738
19.6282051282051 0.499796390533447
20.5641025641026 0.449761211872101
21.5416666666667 0.474911898374557
22.5673076923077 0.459392130374908
23.6442307692308 0.46713000535965
24.7692307692308 0.404979914426804
25.9487179487179 0.459385007619858
27.1826923076923 0.451853483915329
28.4775641025641 0.457371920347214
29.8333333333333 0.425272911787033
31.2564102564103 0.395643532276154
32.7435897435897 0.407275408506393
34.3044871794872 0.379059761762619
35.9358974358974 0.408226251602173
37.6474358974359 0.382270961999893
39.4391025641026 0.322337061166763
41.3173076923077 0.377113163471222
43.2852564102564 0.318112105131149
45.3461538461538 0.322911113500595
47.5064102564103 0.325559467077255
49.7692307692308 0.258837223052979
52.1378205128205 0.323755979537964
54.6217948717949 0.264705806970596
57.2211538461538 0.272887051105499
59.9455128205128 0.282916009426117
62.8012820512821 0.278243690729141
65.7916666666667 0.29192391037941
68.9230769230769 0.280155420303345
72.2051282051282 0.224564895033836
75.6442307692308 0.257812678813934
79.2467948717949 0.242182806134224
83.0192307692308 0.278226792812347
86.974358974359 0.280973523855209
91.1153846153846 0.284133583307266
95.4519230769231 0.269515365362167
100 0.256322622299194
};
\addlegendentry{mb 128, mc 1}
\addplot [, color1, opacity=0.6, mark=square*, mark size=0.5, mark options={solid}, only marks, forget plot]
table {%
1 0.858656346797943
1.04487179487179 0.907752156257629
1.09615384615385 0.929432094097137
1.1474358974359 0.913423001766205
1.20192307692308 0.865697205066681
1.25961538461538 0.816353619098663
1.32051282051282 0.832290947437286
1.38461538461538 0.763093769550323
1.44871794871795 0.821062028408051
1.51923076923077 0.812784373760223
1.58974358974359 0.787710130214691
1.66666666666667 0.837903499603271
1.74679487179487 0.794794917106628
1.83012820512821 0.809977352619171
1.91666666666667 0.807985603809357
2.00641025641026 0.778876721858978
2.1025641025641 0.764621734619141
2.20512820512821 0.756167888641357
2.30769230769231 0.755573213100433
2.41987179487179 0.774213790893555
2.53525641025641 0.72566556930542
2.65384615384615 0.738050878047943
2.78205128205128 0.785201966762543
2.91346153846154 0.780164539813995
3.05128205128205 0.74937915802002
3.19871794871795 0.740462124347687
3.34935897435897 0.759673714637756
3.50961538461538 0.720031023025513
3.67628205128205 0.681498229503632
3.8525641025641 0.795181393623352
4.03525641025641 0.74979555606842
4.2275641025641 0.723177433013916
4.42948717948718 0.66851794719696
4.64102564102564 0.67188024520874
4.86217948717949 0.661979377269745
5.09294871794872 0.688683807849884
5.33653846153846 0.614040791988373
5.58974358974359 0.633781909942627
5.85576923076923 0.639364182949066
6.13461538461539 0.625432193279266
6.42628205128205 0.573632836341858
6.73397435897436 0.611762166023254
7.05448717948718 0.62417459487915
7.38782051282051 0.598569333553314
7.74038461538461 0.594687402248383
8.10897435897436 0.577013194561005
8.49679487179487 0.574857294559479
8.90064102564103 0.573448836803436
9.32371794871795 0.581501483917236
9.76923076923077 0.604185163974762
10.2339743589744 0.578490972518921
10.7211538461538 0.538881719112396
11.2307692307692 0.545246362686157
11.7660256410256 0.499398916959763
12.3269230769231 0.486677408218384
12.9134615384615 0.523888051509857
13.5288461538462 0.487390339374542
14.1730769230769 0.510754525661469
14.849358974359 0.462380319833755
15.5544871794872 0.458656698465347
16.2948717948718 0.502867341041565
17.0705128205128 0.480082273483276
17.8846153846154 0.502073705196381
18.7371794871795 0.424837559461594
19.6282051282051 0.391698509454727
20.5641025641026 0.442054957151413
21.5416666666667 0.400834530591965
22.5673076923077 0.417273253202438
23.6442307692308 0.383448511362076
24.7692307692308 0.345826327800751
25.9487179487179 0.427371174097061
27.1826923076923 0.372452944517136
28.4775641025641 0.371306329965591
29.8333333333333 0.385769486427307
31.2564102564103 0.361891120672226
32.7435897435897 0.362316250801086
34.3044871794872 0.384469091892242
35.9358974358974 0.339487075805664
37.6474358974359 0.352418392896652
39.4391025641026 0.317721992731094
41.3173076923077 0.339120447635651
43.2852564102564 0.250126928091049
45.3461538461538 0.311809539794922
47.5064102564103 0.375011265277863
49.7692307692308 0.263584464788437
52.1378205128205 0.314800947904587
54.6217948717949 0.296242415904999
57.2211538461538 0.24090464413166
59.9455128205128 0.284472644329071
62.8012820512821 0.281670391559601
65.7916666666667 0.294350832700729
68.9230769230769 0.309636890888214
72.2051282051282 0.276949107646942
75.6442307692308 0.261805027723312
79.2467948717949 0.293683022260666
83.0192307692308 0.286657452583313
86.974358974359 0.284863740205765
91.1153846153846 0.267405927181244
95.4519230769231 0.255504190921783
100 0.265985369682312
};
\addplot [, color1, opacity=0.6, mark=square*, mark size=0.5, mark options={solid}, only marks, forget plot]
table {%
1 0.828397452831268
1.04487179487179 0.907137215137482
1.09615384615385 0.931383907794952
1.1474358974359 0.907084107398987
1.20192307692308 0.932597577571869
1.25961538461538 0.832961857318878
1.32051282051282 0.818427681922913
1.38461538461538 0.798737645149231
1.44871794871795 0.834369778633118
1.51923076923077 0.805143654346466
1.58974358974359 0.742678821086884
1.66666666666667 0.785011470317841
1.74679487179487 0.770569443702698
1.83012820512821 0.753225982189178
1.91666666666667 0.749503016471863
2.00641025641026 0.755612075328827
2.1025641025641 0.732974231243134
2.20512820512821 0.750865161418915
2.30769230769231 0.730873346328735
2.41987179487179 0.719896912574768
2.53525641025641 0.742928266525269
2.65384615384615 0.725754082202911
2.78205128205128 0.728431224822998
2.91346153846154 0.749420821666718
3.05128205128205 0.764041244983673
3.19871794871795 0.725082635879517
3.34935897435897 0.73115473985672
3.50961538461538 0.703763246536255
3.67628205128205 0.675777554512024
3.8525641025641 0.732781112194061
4.03525641025641 0.703109920024872
4.2275641025641 0.661949813365936
4.42948717948718 0.636121928691864
4.64102564102564 0.668044209480286
4.86217948717949 0.627338528633118
5.09294871794872 0.628443896770477
5.33653846153846 0.695911824703217
5.58974358974359 0.656872391700745
5.85576923076923 0.66290694475174
6.13461538461539 0.642579317092896
6.42628205128205 0.623920381069183
6.73397435897436 0.636985838413239
7.05448717948718 0.670020818710327
7.38782051282051 0.601765811443329
7.74038461538461 0.621480703353882
8.10897435897436 0.627401769161224
8.49679487179487 0.685251712799072
8.90064102564103 0.603479981422424
9.32371794871795 0.589451968669891
9.76923076923077 0.505423724651337
10.2339743589744 0.585445582866669
10.7211538461538 0.615051209926605
11.2307692307692 0.531665444374084
11.7660256410256 0.560044288635254
12.3269230769231 0.559454083442688
12.9134615384615 0.536922097206116
13.5288461538462 0.585442125797272
14.1730769230769 0.538672566413879
14.849358974359 0.537347137928009
15.5544871794872 0.512918889522552
16.2948717948718 0.503639221191406
17.0705128205128 0.491594612598419
17.8846153846154 0.50579035282135
18.7371794871795 0.446386426687241
19.6282051282051 0.428650110960007
20.5641025641026 0.412720650434494
21.5416666666667 0.487561225891113
22.5673076923077 0.4489386677742
23.6442307692308 0.410760313272476
24.7692307692308 0.439415991306305
25.9487179487179 0.398690909147263
27.1826923076923 0.428749412298203
28.4775641025641 0.436763972043991
29.8333333333333 0.445739030838013
31.2564102564103 0.411928236484528
32.7435897435897 0.393013119697571
34.3044871794872 0.402959018945694
35.9358974358974 0.379105180501938
37.6474358974359 0.386829018592834
39.4391025641026 0.325447529554367
41.3173076923077 0.401644378900528
43.2852564102564 0.293905735015869
45.3461538461538 0.281503587961197
47.5064102564103 0.327936083078384
49.7692307692308 0.319226652383804
52.1378205128205 0.255844175815582
54.6217948717949 0.262244254350662
57.2211538461538 0.278933852910995
59.9455128205128 0.269355744123459
62.8012820512821 0.267186403274536
65.7916666666667 0.329450696706772
68.9230769230769 0.288878917694092
72.2051282051282 0.218272119760513
75.6442307692308 0.240003615617752
79.2467948717949 0.273418992757797
83.0192307692308 0.260252803564072
86.974358974359 0.223961398005486
91.1153846153846 0.27972999215126
95.4519230769231 0.254800707101822
100 0.247597649693489
};
\addplot [, color1, opacity=0.6, mark=square*, mark size=0.5, mark options={solid}, only marks, forget plot]
table {%
1 0.817034423351288
1.04487179487179 0.909705758094788
1.09615384615385 0.928344249725342
1.1474358974359 0.927618026733398
1.20192307692308 0.936348915100098
1.25961538461538 0.893659591674805
1.32051282051282 0.852953732013702
1.38461538461538 0.872898876667023
1.44871794871795 0.792310833930969
1.51923076923077 0.795319199562073
1.58974358974359 0.884269654750824
1.66666666666667 0.869260787963867
1.74679487179487 0.871471405029297
1.83012820512821 0.840520977973938
1.91666666666667 0.855638921260834
2.00641025641026 0.778341591358185
2.1025641025641 0.80114883184433
2.20512820512821 0.772222697734833
2.30769230769231 0.786143124103546
2.41987179487179 0.745745360851288
2.53525641025641 0.785748183727264
2.65384615384615 0.7447469830513
2.78205128205128 0.743236243724823
2.91346153846154 0.741190612316132
3.05128205128205 0.763479053974152
3.19871794871795 0.762459337711334
3.34935897435897 0.737563967704773
3.50961538461538 0.676618993282318
3.67628205128205 0.70228499174118
3.8525641025641 0.7459756731987
4.03525641025641 0.729778110980988
4.2275641025641 0.755058526992798
4.42948717948718 0.687171041965485
4.64102564102564 0.68555873632431
4.86217948717949 0.692193210124969
5.09294871794872 0.668319880962372
5.33653846153846 0.694530844688416
5.58974358974359 0.719569325447083
5.85576923076923 0.603907465934753
6.13461538461539 0.638790309429169
6.42628205128205 0.614583313465118
6.73397435897436 0.583330273628235
7.05448717948718 0.592248618602753
7.38782051282051 0.647828876972198
7.74038461538461 0.646674335002899
8.10897435897436 0.571582555770874
8.49679487179487 0.600022375583649
8.90064102564103 0.612543284893036
9.32371794871795 0.59609979391098
9.76923076923077 0.556222438812256
10.2339743589744 0.653578281402588
10.7211538461538 0.583583772182465
11.2307692307692 0.528942465782166
11.7660256410256 0.532447338104248
12.3269230769231 0.561110496520996
12.9134615384615 0.529454827308655
13.5288461538462 0.513769328594208
14.1730769230769 0.459704875946045
14.849358974359 0.481569766998291
15.5544871794872 0.526020348072052
16.2948717948718 0.477293103933334
17.0705128205128 0.458888620138168
17.8846153846154 0.483338177204132
18.7371794871795 0.487065851688385
19.6282051282051 0.48175048828125
20.5641025641026 0.463883638381958
21.5416666666667 0.49774631857872
22.5673076923077 0.415245026350021
23.6442307692308 0.415585517883301
24.7692307692308 0.392017811536789
25.9487179487179 0.454217731952667
27.1826923076923 0.458766847848892
28.4775641025641 0.367083847522736
29.8333333333333 0.407296001911163
31.2564102564103 0.417179673910141
32.7435897435897 0.362832278013229
34.3044871794872 0.333935111761093
35.9358974358974 0.324279189109802
37.6474358974359 0.309039562940598
39.4391025641026 0.324372977018356
41.3173076923077 0.333293884992599
43.2852564102564 0.28301340341568
45.3461538461538 0.317429780960083
47.5064102564103 0.357645660638809
49.7692307692308 0.297974050045013
52.1378205128205 0.241303682327271
54.6217948717949 0.312122374773026
57.2211538461538 0.282743990421295
59.9455128205128 0.283277601003647
62.8012820512821 0.247865840792656
65.7916666666667 0.26517516374588
68.9230769230769 0.27281329035759
72.2051282051282 0.253391712903976
75.6442307692308 0.252779692411423
79.2467948717949 0.238938376307487
83.0192307692308 0.268889456987381
86.974358974359 0.231116443872452
91.1153846153846 0.221858263015747
95.4519230769231 0.218844279646873
100 0.293239623308182
};
\addplot [, color1, opacity=0.6, mark=square*, mark size=0.5, mark options={solid}, only marks, forget plot]
table {%
1 0.827813267707825
1.04487179487179 0.846263885498047
1.09615384615385 0.925744235515594
1.1474358974359 0.910347163677216
1.20192307692308 0.938778102397919
1.25961538461538 0.797892451286316
1.32051282051282 0.812102913856506
1.38461538461538 0.814592480659485
1.44871794871795 0.860499203205109
1.51923076923077 0.834038555622101
1.58974358974359 0.853531301021576
1.66666666666667 0.834654271602631
1.74679487179487 0.848230183124542
1.83012820512821 0.78334367275238
1.91666666666667 0.790080189704895
2.00641025641026 0.834882915019989
2.1025641025641 0.763410806655884
2.20512820512821 0.815198123455048
2.30769230769231 0.798012912273407
2.41987179487179 0.746797442436218
2.53525641025641 0.769881069660187
2.65384615384615 0.724181592464447
2.78205128205128 0.776508569717407
2.91346153846154 0.761766135692596
3.05128205128205 0.751246869564056
3.19871794871795 0.714191257953644
3.34935897435897 0.731342732906342
3.50961538461538 0.699134528636932
3.67628205128205 0.655514001846313
3.8525641025641 0.738500773906708
4.03525641025641 0.757274568080902
4.2275641025641 0.771870136260986
4.42948717948718 0.742603302001953
4.64102564102564 0.691445350646973
4.86217948717949 0.699981510639191
5.09294871794872 0.729809939861298
5.33653846153846 0.698283731937408
5.58974358974359 0.681201934814453
5.85576923076923 0.688197731971741
6.13461538461539 0.654345154762268
6.42628205128205 0.678068518638611
6.73397435897436 0.631806790828705
7.05448717948718 0.677231311798096
7.38782051282051 0.644726872444153
7.74038461538461 0.618898332118988
8.10897435897436 0.635389745235443
8.49679487179487 0.62433522939682
8.90064102564103 0.682459652423859
9.32371794871795 0.587493538856506
9.76923076923077 0.593628764152527
10.2339743589744 0.592686474323273
10.7211538461538 0.537352859973907
11.2307692307692 0.571306884288788
11.7660256410256 0.602908313274384
12.3269230769231 0.577520549297333
12.9134615384615 0.578780472278595
13.5288461538462 0.541737139225006
14.1730769230769 0.521243035793304
14.849358974359 0.524567246437073
15.5544871794872 0.497612565755844
16.2948717948718 0.485525608062744
17.0705128205128 0.532859444618225
17.8846153846154 0.546761870384216
18.7371794871795 0.478527992963791
19.6282051282051 0.475883334875107
20.5641025641026 0.449024260044098
21.5416666666667 0.452493965625763
22.5673076923077 0.469335079193115
23.6442307692308 0.434750378131866
24.7692307692308 0.439308255910873
25.9487179487179 0.505074977874756
27.1826923076923 0.415506422519684
28.4775641025641 0.434371948242188
29.8333333333333 0.416286915540695
31.2564102564103 0.408123970031738
32.7435897435897 0.404401868581772
34.3044871794872 0.430812984704971
35.9358974358974 0.367878824472427
37.6474358974359 0.37580081820488
39.4391025641026 0.348804384469986
41.3173076923077 0.354967355728149
43.2852564102564 0.390192002058029
45.3461538461538 0.401204496622086
47.5064102564103 0.356531530618668
49.7692307692308 0.292953729629517
52.1378205128205 0.351745277643204
54.6217948717949 0.314463436603546
57.2211538461538 0.338084548711777
59.9455128205128 0.333658665418625
62.8012820512821 0.314126044511795
65.7916666666667 0.329912334680557
68.9230769230769 0.318909138441086
72.2051282051282 0.293680816888809
75.6442307692308 0.327805042266846
79.2467948717949 0.305316567420959
83.0192307692308 0.290589779615402
86.974358974359 0.252898305654526
91.1153846153846 0.297418087720871
95.4519230769231 0.26307338476181
100 0.256137877702713
};
\addplot [, color2, opacity=0.6, mark=triangle*, mark size=0.5, mark options={solid,rotate=180}, only marks]
table {%
1 0.403328388929367
1.04487179487179 0.500678479671478
1.09615384615385 0.531769931316376
1.1474358974359 0.580257713794708
1.20192307692308 0.610610008239746
1.25961538461538 0.491812318563461
1.32051282051282 0.455650806427002
1.38461538461538 0.508278131484985
1.44871794871795 0.49448749423027
1.51923076923077 0.509385526180267
1.58974358974359 0.554943978786469
1.66666666666667 0.475136965513229
1.74679487179487 0.524940133094788
1.83012820512821 0.446522623300552
1.91666666666667 0.507020592689514
2.00641025641026 0.498102575540543
2.1025641025641 0.428548812866211
2.20512820512821 0.45367032289505
2.30769230769231 0.44624862074852
2.41987179487179 0.410327345132828
2.53525641025641 0.493331730365753
2.65384615384615 0.412990570068359
2.78205128205128 0.447286337614059
2.91346153846154 0.448118031024933
3.05128205128205 0.46475550532341
3.19871794871795 0.422908514738083
3.34935897435897 0.45371276140213
3.50961538461538 0.465672463178635
3.67628205128205 0.38970759510994
3.8525641025641 0.408283233642578
4.03525641025641 0.376380890607834
4.2275641025641 0.375652551651001
4.42948717948718 0.378371775150299
4.64102564102564 0.391542166471481
4.86217948717949 0.383591413497925
5.09294871794872 0.353204280138016
5.33653846153846 0.385457992553711
5.58974358974359 0.365813255310059
5.85576923076923 0.34700021147728
6.13461538461539 0.361018687486649
6.42628205128205 0.390840113162994
6.73397435897436 0.335453361272812
7.05448717948718 0.324993103742599
7.38782051282051 0.334690302610397
7.74038461538461 0.325295120477676
8.10897435897436 0.368836730718613
8.49679487179487 0.311197131872177
8.90064102564103 0.390583276748657
9.32371794871795 0.322690486907959
9.76923076923077 0.325783461332321
10.2339743589744 0.348309844732285
10.7211538461538 0.317012161016464
11.2307692307692 0.345407873392105
11.7660256410256 0.323500156402588
12.3269230769231 0.321108788251877
12.9134615384615 0.309399098157883
13.5288461538462 0.300587087869644
14.1730769230769 0.314572960138321
14.849358974359 0.301639646291733
15.5544871794872 0.310845583677292
16.2948717948718 0.331483691930771
17.0705128205128 0.276376515626907
17.8846153846154 0.309700936079025
18.7371794871795 0.32034358382225
19.6282051282051 0.309875458478928
20.5641025641026 0.283959567546844
21.5416666666667 0.300912231206894
22.5673076923077 0.311990708112717
23.6442307692308 0.292921394109726
24.7692307692308 0.298220723867416
25.9487179487179 0.271746724843979
27.1826923076923 0.264651209115982
28.4775641025641 0.288961231708527
29.8333333333333 0.327028185129166
31.2564102564103 0.256611078977585
32.7435897435897 0.283818274736404
34.3044871794872 0.294581741094589
35.9358974358974 0.216543480753899
37.6474358974359 0.272851318120956
39.4391025641026 0.25249195098877
41.3173076923077 0.26741087436676
43.2852564102564 0.230801627039909
45.3461538461538 0.271301597356796
47.5064102564103 0.28218212723732
49.7692307692308 0.251521974802017
52.1378205128205 0.268582940101624
54.6217948717949 0.264646768569946
57.2211538461538 0.287348657846451
59.9455128205128 0.286088019609451
62.8012820512821 0.290025353431702
65.7916666666667 0.31400129199028
68.9230769230769 0.274712771177292
72.2051282051282 0.235672384500504
75.6442307692308 0.268685907125473
79.2467948717949 0.251919090747833
83.0192307692308 0.277584820985794
86.974358974359 0.253834784030914
91.1153846153846 0.217347726225853
95.4519230769231 0.275533050298691
100 0.261717021465302
};
\addlegendentry{sub 16, mc 1}
\addplot [, color2, opacity=0.6, mark=triangle*, mark size=0.5, mark options={solid,rotate=180}, only marks, forget plot]
table {%
1 0.407858848571777
1.04487179487179 0.502846360206604
1.09615384615385 0.58879154920578
1.1474358974359 0.562698841094971
1.20192307692308 0.610564887523651
1.25961538461538 0.606741726398468
1.32051282051282 0.523517429828644
1.38461538461538 0.571849465370178
1.44871794871795 0.525750577449799
1.51923076923077 0.570542454719543
1.58974358974359 0.579596042633057
1.66666666666667 0.533606171607971
1.74679487179487 0.529407441616058
1.83012820512821 0.591767549514771
1.91666666666667 0.566542744636536
2.00641025641026 0.564357101917267
2.1025641025641 0.494915813207626
2.20512820512821 0.519237458705902
2.30769230769231 0.501540124416351
2.41987179487179 0.486162006855011
2.53525641025641 0.548281788825989
2.65384615384615 0.495616108179092
2.78205128205128 0.521672368049622
2.91346153846154 0.487012833356857
3.05128205128205 0.50938481092453
3.19871794871795 0.509578108787537
3.34935897435897 0.485174089670181
3.50961538461538 0.469097435474396
3.67628205128205 0.466290056705475
3.8525641025641 0.450110912322998
4.03525641025641 0.496870517730713
4.2275641025641 0.464010715484619
4.42948717948718 0.444722175598145
4.64102564102564 0.424870729446411
4.86217948717949 0.434516102075577
5.09294871794872 0.360240131616592
5.33653846153846 0.43590459227562
5.58974358974359 0.416973441839218
5.85576923076923 0.418000906705856
6.13461538461539 0.397315680980682
6.42628205128205 0.388605982065201
6.73397435897436 0.384747713804245
7.05448717948718 0.396933168172836
7.38782051282051 0.380245625972748
7.74038461538461 0.316427290439606
8.10897435897436 0.37564691901207
8.49679487179487 0.337792128324509
8.90064102564103 0.376842498779297
9.32371794871795 0.363871157169342
9.76923076923077 0.343962907791138
10.2339743589744 0.407250612974167
10.7211538461538 0.311388432979584
11.2307692307692 0.325923323631287
11.7660256410256 0.336207300424576
12.3269230769231 0.374246150255203
12.9134615384615 0.324291884899139
13.5288461538462 0.341023206710815
14.1730769230769 0.330274105072021
14.849358974359 0.356853902339935
15.5544871794872 0.336841344833374
16.2948717948718 0.349818557500839
17.0705128205128 0.323360830545425
17.8846153846154 0.3030846118927
18.7371794871795 0.340853661298752
19.6282051282051 0.310323685407639
20.5641025641026 0.305444151163101
21.5416666666667 0.326009839773178
22.5673076923077 0.312898606061935
23.6442307692308 0.32131415605545
24.7692307692308 0.301416844129562
25.9487179487179 0.29797551035881
27.1826923076923 0.289734095335007
28.4775641025641 0.269061654806137
29.8333333333333 0.30263552069664
31.2564102564103 0.305902302265167
32.7435897435897 0.296522498130798
34.3044871794872 0.302342742681503
35.9358974358974 0.264042764902115
37.6474358974359 0.236106038093567
39.4391025641026 0.257474452257156
41.3173076923077 0.247835382819176
43.2852564102564 0.276715010404587
45.3461538461538 0.254599392414093
47.5064102564103 0.237850531935692
49.7692307692308 0.243172645568848
52.1378205128205 0.281688302755356
54.6217948717949 0.238183185458183
57.2211538461538 0.230761811137199
59.9455128205128 0.232109233736992
62.8012820512821 0.25855153799057
65.7916666666667 0.213054686784744
68.9230769230769 0.195758521556854
72.2051282051282 0.217609032988548
75.6442307692308 0.217123076319695
79.2467948717949 0.226714059710503
83.0192307692308 0.218353554606438
86.974358974359 0.229764610528946
91.1153846153846 0.190627917647362
95.4519230769231 0.180565983057022
100 0.213041737675667
};
\addplot [, color2, opacity=0.6, mark=triangle*, mark size=0.5, mark options={solid,rotate=180}, only marks, forget plot]
table {%
1 0.452566713094711
1.04487179487179 0.555111587047577
1.09615384615385 0.548881769180298
1.1474358974359 0.602046430110931
1.20192307692308 0.62760728597641
1.25961538461538 0.590451717376709
1.32051282051282 0.566168010234833
1.38461538461538 0.591725409030914
1.44871794871795 0.519594788551331
1.51923076923077 0.490444272756577
1.58974358974359 0.510723352432251
1.66666666666667 0.505447387695312
1.74679487179487 0.553880035877228
1.83012820512821 0.483377367258072
1.91666666666667 0.502935409545898
2.00641025641026 0.477682322263718
2.1025641025641 0.485909849405289
2.20512820512821 0.503089547157288
2.30769230769231 0.417315691709518
2.41987179487179 0.444171905517578
2.53525641025641 0.471772402524948
2.65384615384615 0.450422286987305
2.78205128205128 0.40151110291481
2.91346153846154 0.469295978546143
3.05128205128205 0.41649055480957
3.19871794871795 0.401228278875351
3.34935897435897 0.41104793548584
3.50961538461538 0.436288833618164
3.67628205128205 0.412791818380356
3.8525641025641 0.41339522600174
4.03525641025641 0.397449404001236
4.2275641025641 0.378023236989975
4.42948717948718 0.397342711687088
4.64102564102564 0.387521356344223
4.86217948717949 0.397853910923004
5.09294871794872 0.371277958154678
5.33653846153846 0.341769278049469
5.58974358974359 0.347661733627319
5.85576923076923 0.341797977685928
6.13461538461539 0.348016947507858
6.42628205128205 0.332917064428329
6.73397435897436 0.325384825468063
7.05448717948718 0.322986215353012
7.38782051282051 0.304545789957047
7.74038461538461 0.299807906150818
8.10897435897436 0.338314086198807
8.49679487179487 0.295009762048721
8.90064102564103 0.318485409021378
9.32371794871795 0.321156591176987
9.76923076923077 0.3239766061306
10.2339743589744 0.300059795379639
10.7211538461538 0.318225890398026
11.2307692307692 0.314446806907654
11.7660256410256 0.32953754067421
12.3269230769231 0.308620184659958
12.9134615384615 0.290620476007462
13.5288461538462 0.341040581464767
14.1730769230769 0.318509638309479
14.849358974359 0.293774664402008
15.5544871794872 0.31644532084465
16.2948717948718 0.322812646627426
17.0705128205128 0.323996514081955
17.8846153846154 0.248896673321724
18.7371794871795 0.284689337015152
19.6282051282051 0.266220986843109
20.5641025641026 0.299442619085312
21.5416666666667 0.299358010292053
22.5673076923077 0.311369359493256
23.6442307692308 0.279814869165421
24.7692307692308 0.271275460720062
25.9487179487179 0.258940100669861
27.1826923076923 0.279079526662827
28.4775641025641 0.293164074420929
29.8333333333333 0.279764801263809
31.2564102564103 0.272250711917877
32.7435897435897 0.307057350873947
34.3044871794872 0.257985562086105
35.9358974358974 0.259776264429092
37.6474358974359 0.274165600538254
39.4391025641026 0.29090029001236
41.3173076923077 0.284646093845367
43.2852564102564 0.269764959812164
45.3461538461538 0.273502916097641
47.5064102564103 0.255349904298782
49.7692307692308 0.252426445484161
52.1378205128205 0.271670788526535
54.6217948717949 0.251283258199692
57.2211538461538 0.275805801153183
59.9455128205128 0.246348097920418
62.8012820512821 0.255823075771332
65.7916666666667 0.220743849873543
68.9230769230769 0.223265796899796
72.2051282051282 0.22325225174427
75.6442307692308 0.240950584411621
79.2467948717949 0.22792737185955
83.0192307692308 0.246221944689751
86.974358974359 0.249203354120255
91.1153846153846 0.246047884225845
95.4519230769231 0.23921899497509
100 0.228757500648499
};
\addplot [, color2, opacity=0.6, mark=triangle*, mark size=0.5, mark options={solid,rotate=180}, only marks, forget plot]
table {%
1 0.476896584033966
1.04487179487179 0.573431849479675
1.09615384615385 0.636700332164764
1.1474358974359 0.609522044658661
1.20192307692308 0.64972972869873
1.25961538461538 0.606738030910492
1.32051282051282 0.553682088851929
1.38461538461538 0.580105006694794
1.44871794871795 0.570102632045746
1.51923076923077 0.516376554965973
1.58974358974359 0.510970175266266
1.66666666666667 0.502802550792694
1.74679487179487 0.53408020734787
1.83012820512821 0.467223465442657
1.91666666666667 0.498631209135056
2.00641025641026 0.414588451385498
2.1025641025641 0.464940786361694
2.20512820512821 0.481362104415894
2.30769230769231 0.456477969884872
2.41987179487179 0.503044545650482
2.53525641025641 0.467787325382233
2.65384615384615 0.436957120895386
2.78205128205128 0.465717226266861
2.91346153846154 0.440336138010025
3.05128205128205 0.460307896137238
3.19871794871795 0.385144352912903
3.34935897435897 0.443659693002701
3.50961538461538 0.39060378074646
3.67628205128205 0.448398679494858
3.8525641025641 0.414163589477539
4.03525641025641 0.427004337310791
4.2275641025641 0.396601498126984
4.42948717948718 0.43778458237648
4.64102564102564 0.391023099422455
4.86217948717949 0.393404215574265
5.09294871794872 0.423762142658234
5.33653846153846 0.363830745220184
5.58974358974359 0.388390690088272
5.85576923076923 0.402399629354477
6.13461538461539 0.33602187037468
6.42628205128205 0.390635848045349
6.73397435897436 0.407109260559082
7.05448717948718 0.391096740961075
7.38782051282051 0.374573141336441
7.74038461538461 0.415051311254501
8.10897435897436 0.383230149745941
8.49679487179487 0.358995825052261
8.90064102564103 0.378960222005844
9.32371794871795 0.382364094257355
9.76923076923077 0.364135712385178
10.2339743589744 0.344790995121002
10.7211538461538 0.360283225774765
11.2307692307692 0.394431531429291
11.7660256410256 0.345769733190536
12.3269230769231 0.35903987288475
12.9134615384615 0.380283683538437
13.5288461538462 0.336706936359406
14.1730769230769 0.32210049033165
14.849358974359 0.29973977804184
15.5544871794872 0.337621837854385
16.2948717948718 0.345727771520615
17.0705128205128 0.335712283849716
17.8846153846154 0.322233289480209
18.7371794871795 0.339669078588486
19.6282051282051 0.326494604349136
20.5641025641026 0.323687881231308
21.5416666666667 0.297789484262466
22.5673076923077 0.305610239505768
23.6442307692308 0.316064804792404
24.7692307692308 0.298534125089645
25.9487179487179 0.276820361614227
27.1826923076923 0.297135472297668
28.4775641025641 0.276917517185211
29.8333333333333 0.294411659240723
31.2564102564103 0.288006722927094
32.7435897435897 0.282210916280746
34.3044871794872 0.261523872613907
35.9358974358974 0.267753154039383
37.6474358974359 0.26441764831543
39.4391025641026 0.234028458595276
41.3173076923077 0.275474399328232
43.2852564102564 0.278287082910538
45.3461538461538 0.278962224721909
47.5064102564103 0.283633708953857
49.7692307692308 0.276561349630356
52.1378205128205 0.223923355340958
54.6217948717949 0.234238669276237
57.2211538461538 0.237907275557518
59.9455128205128 0.225008398294449
62.8012820512821 0.229689478874207
65.7916666666667 0.218459561467171
68.9230769230769 0.202610850334167
72.2051282051282 0.236436530947685
75.6442307692308 0.249004647135735
79.2467948717949 0.220484301447868
83.0192307692308 0.235340639948845
86.974358974359 0.24115352332592
91.1153846153846 0.214632660150528
95.4519230769231 0.210192039608955
100 0.286387622356415
};
\addplot [, color2, opacity=0.6, mark=triangle*, mark size=0.5, mark options={solid,rotate=180}, only marks, forget plot]
table {%
1 0.423201769590378
1.04487179487179 0.481868743896484
1.09615384615385 0.591870307922363
1.1474358974359 0.623361527919769
1.20192307692308 0.615780830383301
1.25961538461538 0.604755222797394
1.32051282051282 0.561985433101654
1.38461538461538 0.571284711360931
1.44871794871795 0.600662410259247
1.51923076923077 0.537410914897919
1.58974358974359 0.501823306083679
1.66666666666667 0.565169513225555
1.74679487179487 0.495580047369003
1.83012820512821 0.541629791259766
1.91666666666667 0.543665051460266
2.00641025641026 0.520067453384399
2.1025641025641 0.510983169078827
2.20512820512821 0.477101892232895
2.30769230769231 0.487212270498276
2.41987179487179 0.488206595182419
2.53525641025641 0.493708699941635
2.65384615384615 0.45949187874794
2.78205128205128 0.493779003620148
2.91346153846154 0.457040309906006
3.05128205128205 0.458070248365402
3.19871794871795 0.476939588785172
3.34935897435897 0.468901008367538
3.50961538461538 0.416797399520874
3.67628205128205 0.454002946615219
3.8525641025641 0.434823721647263
4.03525641025641 0.421017080545425
4.2275641025641 0.479255020618439
4.42948717948718 0.434119611978531
4.64102564102564 0.411223500967026
4.86217948717949 0.382410943508148
5.09294871794872 0.392111033201218
5.33653846153846 0.392803162336349
5.58974358974359 0.385303318500519
5.85576923076923 0.407132118940353
6.13461538461539 0.382713794708252
6.42628205128205 0.387118309736252
6.73397435897436 0.437141954898834
7.05448717948718 0.388004630804062
7.38782051282051 0.378326267004013
7.74038461538461 0.378219991922379
8.10897435897436 0.369479656219482
8.49679487179487 0.366816014051437
8.90064102564103 0.330578058958054
9.32371794871795 0.308064609766006
9.76923076923077 0.34880992770195
10.2339743589744 0.342554658651352
10.7211538461538 0.347850680351257
11.2307692307692 0.298551321029663
11.7660256410256 0.329081565141678
12.3269230769231 0.342234939336777
12.9134615384615 0.301361531019211
13.5288461538462 0.367499828338623
14.1730769230769 0.353368163108826
14.849358974359 0.333645015954971
15.5544871794872 0.297293245792389
16.2948717948718 0.304090827703476
17.0705128205128 0.359236419200897
17.8846153846154 0.311598837375641
18.7371794871795 0.302559196949005
19.6282051282051 0.302731931209564
20.5641025641026 0.323415726423264
21.5416666666667 0.282303541898727
22.5673076923077 0.317699640989304
23.6442307692308 0.322174191474915
24.7692307692308 0.30498930811882
25.9487179487179 0.277491241693497
27.1826923076923 0.278138041496277
28.4775641025641 0.279551595449448
29.8333333333333 0.275996893644333
31.2564102564103 0.278350114822388
32.7435897435897 0.287115514278412
34.3044871794872 0.275540202856064
35.9358974358974 0.273132294416428
37.6474358974359 0.252303838729858
39.4391025641026 0.258446902036667
41.3173076923077 0.233400583267212
43.2852564102564 0.291865080595016
45.3461538461538 0.239434525370598
47.5064102564103 0.287410646677017
49.7692307692308 0.282996237277985
52.1378205128205 0.260377079248428
54.6217948717949 0.255564272403717
57.2211538461538 0.251687854528427
59.9455128205128 0.271311551332474
62.8012820512821 0.255502074956894
65.7916666666667 0.241249233484268
68.9230769230769 0.267258614301682
72.2051282051282 0.23657500743866
75.6442307692308 0.244346842169762
79.2467948717949 0.252802848815918
83.0192307692308 0.228952839970589
86.974358974359 0.297624260187149
91.1153846153846 0.227431699633598
95.4519230769231 0.249187469482422
100 0.241004854440689
};
\end{axis}

\end{tikzpicture}

      \tikzexternaldisable
    \end{minipage}
  \end{subfigure}

  \begin{subfigure}[t]{\linewidth}
    \centering
    \caption{\cifarhun \allcnnc \sgd}
    \begin{minipage}{0.50\linewidth}
      \centering
      % defines the pgfplots style "eigspacedefault"
\pgfkeys{/pgfplots/eigspacedefault/.style={
    width=1.0\linewidth,
    height=0.6\linewidth,
    every axis plot/.append style={line width = 1.5pt},
    tick pos = left,
    ylabel near ticks,
    xlabel near ticks,
    xtick align = inside,
    ytick align = inside,
    legend cell align = left,
    legend columns = 4,
    legend pos = south east,
    legend style = {
      fill opacity = 1,
      text opacity = 1,
      font = \footnotesize,
      at={(1, 1.025)},
      anchor=south east,
      column sep=0.25cm,
    },
    legend image post style={scale=2.5},
    xticklabel style = {font = \footnotesize},
    xlabel style = {font = \footnotesize},
    axis line style = {black},
    yticklabel style = {font = \footnotesize},
    ylabel style = {font = \footnotesize},
    title style = {font = \footnotesize},
    grid = major,
    grid style = {dashed}
  }
}

\pgfkeys{/pgfplots/eigspacedefaultapp/.style={
    eigspacedefault,
    height=0.6\linewidth,
    legend columns = 2,
  }
}

\pgfkeys{/pgfplots/eigspacenolegend/.style={
    legend image post style = {scale=0},
    legend style = {
      fill opacity = 0,
      draw opacity = 0,
      text opacity = 0,
      font = \footnotesize,
      at={(1, 1.025)},
      anchor=south east,
      column sep=0.25cm,
    },
  }
}
%%% Local Variables:
%%% mode: latex
%%% TeX-master: "../../thesis"
%%% End:

      \pgfkeys{/pgfplots/zmystyle/.style={
          eigspacedefaultapp,
          eigspacenolegend,
        }}
      \tikzexternalenable
      \vspace{-6ex}
      % This file was created by tikzplotlib v0.9.7.
\begin{tikzpicture}

\definecolor{color0}{rgb}{0.501960784313725,0.184313725490196,0.6}
\definecolor{color1}{rgb}{0.870588235294118,0.623529411764706,0.0862745098039216}
\definecolor{color2}{rgb}{0.274509803921569,0.6,0.564705882352941}

\begin{axis}[
axis line style={white!10!black},
legend columns=2,
legend style={fill opacity=0.8, draw opacity=1, text opacity=1, draw=white!80!black},
log basis x={10},
tick pos=left,
xlabel={epoch (log scale)},
xmajorgrids,
xmin=0.746099240306814, xmax=469.106495613199,
xmode=log,
ylabel={overlap},
ymajorgrids,
ymin=-0.05, ymax=1.05,
zmystyle
]
\addplot [, white!10!black, dashed, forget plot]
table {%
0.746099240306814 1
469.106495613199 1
};
\addplot [, white!10!black, dashed, forget plot]
table {%
0.746099240306814 0
469.106495613199 0
};
\addplot [, color0, opacity=0.6, mark=triangle*, mark size=0.5, mark options={solid,rotate=180}, only marks]
table {%
1 0.791732788085938
1.05769230769231 0.765409827232361
1.12179487179487 0.785296618938446
1.19230769230769 0.756134629249573
1.26282051282051 0.758359134197235
1.33974358974359 0.697048544883728
1.42307692307692 0.526818037033081
1.51282051282051 0.478270679712296
1.6025641025641 0.461277276277542
1.69871794871795 0.403579026460648
1.80128205128205 0.517794966697693
1.91666666666667 0.432361632585526
2.03205128205128 0.439486920833588
2.15384615384615 0.412330955266953
2.28846153846154 0.454176634550095
2.42307692307692 0.337712824344635
2.57692307692308 0.386521339416504
2.73076923076923 0.445201396942139
2.8974358974359 0.311065912246704
3.07692307692308 0.367929369211197
3.26282051282051 0.298572868108749
3.46153846153846 0.335516840219498
3.67307692307692 0.305664867162704
3.8974358974359 0.348087072372437
4.13461538461539 0.35147699713707
4.38461538461539 0.28137531876564
4.65384615384615 0.344588160514832
4.93589743589744 0.361682772636414
5.23717948717949 0.342574059963226
5.55769230769231 0.325968772172928
5.8974358974359 0.298941314220428
6.25641025641026 0.272649735212326
6.64102564102564 0.293054759502411
7.04487179487179 0.297763139009476
7.47435897435897 0.3223617374897
7.92948717948718 0.28100112080574
8.41025641025641 0.267537981271744
8.92307692307692 0.289072066545486
9.46794871794872 0.316077798604965
10.0448717948718 0.328793793916702
10.6602564102564 0.277209043502808
11.3076923076923 0.304397463798523
12 0.284644782543182
12.7307692307692 0.30106320977211
13.5064102564103 0.270345449447632
14.3333333333333 0.291975408792496
15.2051282051282 0.268277078866959
16.1346153846154 0.275267064571381
17.1153846153846 0.277081191539764
18.1602564102564 0.26803982257843
19.2692307692308 0.229230493307114
20.4423076923077 0.232438802719116
21.6858974358974 0.248474434018135
23.0128205128205 0.235620990395546
24.4102564102564 0.240279600024223
25.9038461538462 0.253866285085678
27.4807692307692 0.222585335373878
29.1538461538462 0.213374346494675
30.9358974358974 0.205386102199554
32.8205128205128 0.217085316777229
34.8205128205128 0.20955216884613
36.9423076923077 0.210395783185959
39.1923076923077 0.200063779950142
41.5833333333333 0.195949897170067
44.1153846153846 0.187356010079384
46.8076923076923 0.174161717295647
49.6602564102564 0.171335503458977
52.6858974358974 0.180296450853348
55.8974358974359 0.179611653089523
59.3076923076923 0.163145691156387
62.9230769230769 0.155659526586533
66.7564102564103 0.168629243969917
70.8269230769231 0.167902544140816
75.1474358974359 0.15610434114933
79.7307692307692 0.158299133181572
84.5897435897436 0.159784987568855
89.7435897435897 0.152892604470253
95.2179487179487 0.152101278305054
101.019230769231 0.148219391703606
107.179487179487 0.144742161035538
113.711538461538 0.149590417742729
120.641025641026 0.15397085249424
127.99358974359 0.146772399544716
135.801282051282 0.134215995669365
144.076923076923 0.155297428369522
152.858974358974 0.142413288354874
162.179487179487 0.135797113180161
172.064102564103 0.13882440328598
182.551282051282 0.135342970490456
193.679487179487 0.132246136665344
205.487179487179 0.137527495622635
218.012820512821 0.144990190863609
231.301282051282 0.130989104509354
245.403846153846 0.144517004489899
260.358974358974 0.122604690492153
276.230769230769 0.129503011703491
293.070512820513 0.121406599879265
310.935897435897 0.123357549309731
329.884615384615 0.13338503241539
350 0.119821205735207
};
\addlegendentry{mb 2, exact}
\addplot [, color0, opacity=0.6, mark=triangle*, mark size=0.5, mark options={solid,rotate=180}, only marks, forget plot]
table {%
1 0.750176668167114
1.05769230769231 0.710568964481354
1.12179487179487 0.70837265253067
1.19230769230769 0.677587926387787
1.26282051282051 0.675881326198578
1.33974358974359 0.649144172668457
1.42307692307692 0.584457278251648
1.51282051282051 0.600467920303345
1.6025641025641 0.420357912778854
1.69871794871795 0.435182571411133
1.80128205128205 0.374062269926071
1.91666666666667 0.430514961481094
2.03205128205128 0.405913472175598
2.15384615384615 0.410684168338776
2.28846153846154 0.349807500839233
2.42307692307692 0.374206393957138
2.57692307692308 0.363726109266281
2.73076923076923 0.233577743172646
2.8974358974359 0.383067458868027
3.07692307692308 0.34719854593277
3.26282051282051 0.352805554866791
3.46153846153846 0.343576461076736
3.67307692307692 0.324868530035019
3.8974358974359 0.319949299097061
4.13461538461539 0.389006465673447
4.38461538461539 0.329734325408936
4.65384615384615 0.292149722576141
4.93589743589744 0.395158439874649
5.23717948717949 0.373132288455963
5.55769230769231 0.365065783262253
5.8974358974359 0.367268204689026
6.25641025641026 0.341423869132996
6.64102564102564 0.284749895334244
7.04487179487179 0.311431407928467
7.47435897435897 0.322428971529007
7.92948717948718 0.277573823928833
8.41025641025641 0.270712643861771
8.92307692307692 0.303095996379852
9.46794871794872 0.300169825553894
10.0448717948718 0.246563136577606
10.6602564102564 0.242211475968361
11.3076923076923 0.252604275941849
12 0.258344292640686
12.7307692307692 0.234025493264198
13.5064102564103 0.249202653765678
14.3333333333333 0.271692186594009
15.2051282051282 0.223167642951012
16.1346153846154 0.248646914958954
17.1153846153846 0.232241988182068
18.1602564102564 0.24584773182869
19.2692307692308 0.223080217838287
20.4423076923077 0.26360484957695
21.6858974358974 0.230618998408318
23.0128205128205 0.231551960110664
24.4102564102564 0.218163028359413
25.9038461538462 0.240254431962967
27.4807692307692 0.220911085605621
29.1538461538462 0.202140420675278
30.9358974358974 0.197273209691048
32.8205128205128 0.206445455551147
34.8205128205128 0.209624633193016
36.9423076923077 0.201879188418388
39.1923076923077 0.196634784340858
41.5833333333333 0.206257209181786
44.1153846153846 0.193442150950432
46.8076923076923 0.172898143529892
49.6602564102564 0.17237700521946
52.6858974358974 0.154973313212395
55.8974358974359 0.155588001012802
59.3076923076923 0.159353241324425
62.9230769230769 0.160975143313408
66.7564102564103 0.160667270421982
70.8269230769231 0.156277894973755
75.1474358974359 0.152237474918365
79.7307692307692 0.156330719590187
84.5897435897436 0.140867456793785
89.7435897435897 0.145389065146446
95.2179487179487 0.135496333241463
101.019230769231 0.131429478526115
107.179487179487 0.138174518942833
113.711538461538 0.122848756611347
120.641025641026 0.135878965258598
127.99358974359 0.129013389348984
135.801282051282 0.128687649965286
144.076923076923 0.142291307449341
152.858974358974 0.130329713225365
162.179487179487 0.12238597869873
172.064102564103 0.123537488281727
182.551282051282 0.12601925432682
193.679487179487 0.112018294632435
205.487179487179 0.130763322114944
218.012820512821 0.116397671401501
231.301282051282 0.111032150685787
245.403846153846 0.126775160431862
260.358974358974 0.127058461308479
276.230769230769 0.113328121602535
293.070512820513 0.12220311909914
310.935897435897 0.128774374723434
329.884615384615 0.10888759046793
350 0.113005004823208
};
\addplot [, color0, opacity=0.6, mark=triangle*, mark size=0.5, mark options={solid,rotate=180}, only marks, forget plot]
table {%
1 0.841996490955353
1.05769230769231 0.812447011470795
1.12179487179487 0.863546788692474
1.19230769230769 0.878388524055481
1.26282051282051 0.888404965400696
1.33974358974359 0.881206929683685
1.42307692307692 0.868398904800415
1.51282051282051 0.638259291648865
1.6025641025641 0.529887437820435
1.69871794871795 0.449452042579651
1.80128205128205 0.396126866340637
1.91666666666667 0.444342941045761
2.03205128205128 0.384877681732178
2.15384615384615 0.466228783130646
2.28846153846154 0.478984981775284
2.42307692307692 0.409530103206635
2.57692307692308 0.42579984664917
2.73076923076923 0.44150710105896
2.8974358974359 0.343522876501083
3.07692307692308 0.328540563583374
3.26282051282051 0.346107482910156
3.46153846153846 0.365377426147461
3.67307692307692 0.293000608682632
3.8974358974359 0.298807233572006
4.13461538461539 0.291616439819336
4.38461538461539 0.304445683956146
4.65384615384615 0.3483567237854
4.93589743589744 0.209390699863434
5.23717948717949 0.253983944654465
5.55769230769231 0.263789296150208
5.8974358974359 0.207014620304108
6.25641025641026 0.319169342517853
6.64102564102564 0.269270420074463
7.04487179487179 0.25434997677803
7.47435897435897 0.274325221776962
7.92948717948718 0.2246253490448
8.41025641025641 0.214188501238823
8.92307692307692 0.197275936603546
9.46794871794872 0.202341184020042
10.0448717948718 0.216360583901405
10.6602564102564 0.185465469956398
11.3076923076923 0.19378936290741
12 0.220641851425171
12.7307692307692 0.232704311609268
13.5064102564103 0.202947482466698
14.3333333333333 0.20728799700737
15.2051282051282 0.200175523757935
16.1346153846154 0.228618785738945
17.1153846153846 0.234763860702515
18.1602564102564 0.19681154191494
19.2692307692308 0.189309269189835
20.4423076923077 0.186625629663467
21.6858974358974 0.20204184949398
23.0128205128205 0.19907945394516
24.4102564102564 0.196858555078506
25.9038461538462 0.171804964542389
27.4807692307692 0.171051904559135
29.1538461538462 0.162411227822304
30.9358974358974 0.175815120339394
32.8205128205128 0.195122465491295
34.8205128205128 0.166956052184105
36.9423076923077 0.163832813501358
39.1923076923077 0.165063813328743
41.5833333333333 0.159911453723907
44.1153846153846 0.170931771397591
46.8076923076923 0.151370763778687
49.6602564102564 0.15037202835083
52.6858974358974 0.151189491152763
55.8974358974359 0.141155660152435
59.3076923076923 0.141427919268608
62.9230769230769 0.137003690004349
66.7564102564103 0.142555609345436
70.8269230769231 0.126294136047363
75.1474358974359 0.135909542441368
79.7307692307692 0.136314272880554
84.5897435897436 0.125006780028343
89.7435897435897 0.123755112290382
95.2179487179487 0.125108137726784
101.019230769231 0.126937612891197
107.179487179487 0.133277550339699
113.711538461538 0.114755056798458
120.641025641026 0.119972795248032
127.99358974359 0.12761315703392
135.801282051282 0.122911587357521
144.076923076923 0.124151267111301
152.858974358974 0.119577616453171
162.179487179487 0.112329766154289
172.064102564103 0.11980228126049
182.551282051282 0.109970577061176
193.679487179487 0.114232800900936
205.487179487179 0.13013382256031
218.012820512821 0.118041299283504
231.301282051282 0.112963929772377
245.403846153846 0.121422737836838
260.358974358974 0.114657685160637
276.230769230769 0.106259383261204
293.070512820513 0.131240844726562
310.935897435897 0.118931517004967
329.884615384615 0.10843151062727
350 0.119194835424423
};
\addplot [, color0, opacity=0.6, mark=triangle*, mark size=0.5, mark options={solid,rotate=180}, only marks, forget plot]
table {%
1 0.729294419288635
1.05769230769231 0.684965491294861
1.12179487179487 0.68757289648056
1.19230769230769 0.675778567790985
1.26282051282051 0.680043458938599
1.33974358974359 0.688807606697083
1.42307692307692 0.697566986083984
1.51282051282051 0.494359374046326
1.6025641025641 0.453307956457138
1.69871794871795 0.440082013607025
1.80128205128205 0.493452668190002
1.91666666666667 0.486060410737991
2.03205128205128 0.489408582448959
2.15384615384615 0.500741243362427
2.28846153846154 0.450641691684723
2.42307692307692 0.444231033325195
2.57692307692308 0.382009774446487
2.73076923076923 0.398858189582825
2.8974358974359 0.338062465190887
3.07692307692308 0.298795968294144
3.26282051282051 0.258138567209244
3.46153846153846 0.330007553100586
3.67307692307692 0.258504420518875
3.8974358974359 0.253192752599716
4.13461538461539 0.307462334632874
4.38461538461539 0.234228655695915
4.65384615384615 0.290793687105179
4.93589743589744 0.286259353160858
5.23717948717949 0.313071370124817
5.55769230769231 0.237897753715515
5.8974358974359 0.255184769630432
6.25641025641026 0.232578620314598
6.64102564102564 0.234007939696312
7.04487179487179 0.256751358509064
7.47435897435897 0.303339213132858
7.92948717948718 0.279918849468231
8.41025641025641 0.239910125732422
8.92307692307692 0.265078604221344
9.46794871794872 0.290270775556564
10.0448717948718 0.276907026767731
10.6602564102564 0.269116669893265
11.3076923076923 0.222758904099464
12 0.22127003967762
12.7307692307692 0.226383209228516
13.5064102564103 0.216591656208038
14.3333333333333 0.244613215327263
15.2051282051282 0.218849942088127
16.1346153846154 0.209253236651421
17.1153846153846 0.220394268631935
18.1602564102564 0.230195224285126
19.2692307692308 0.204856559634209
20.4423076923077 0.206362679600716
21.6858974358974 0.210426092147827
23.0128205128205 0.21301257610321
24.4102564102564 0.19277323782444
25.9038461538462 0.200661957263947
27.4807692307692 0.181834101676941
29.1538461538462 0.192440032958984
30.9358974358974 0.181573957204819
32.8205128205128 0.191689297556877
34.8205128205128 0.159648567438126
36.9423076923077 0.1907599568367
39.1923076923077 0.174641668796539
41.5833333333333 0.17724272608757
44.1153846153846 0.162722051143646
46.8076923076923 0.153056710958481
49.6602564102564 0.142903178930283
52.6858974358974 0.157376110553741
55.8974358974359 0.148728981614113
59.3076923076923 0.133554205298424
62.9230769230769 0.127524748444557
66.7564102564103 0.152608245611191
70.8269230769231 0.148725643754005
75.1474358974359 0.127086192369461
79.7307692307692 0.119430750608444
84.5897435897436 0.137880459427834
89.7435897435897 0.140163436532021
95.2179487179487 0.133765786886215
101.019230769231 0.122233629226685
107.179487179487 0.145636364817619
113.711538461538 0.125491753220558
120.641025641026 0.121353544294834
127.99358974359 0.115429416298866
135.801282051282 0.103527314960957
144.076923076923 0.116682790219784
152.858974358974 0.104458436369896
162.179487179487 0.125805586576462
172.064102564103 0.0970917269587517
182.551282051282 0.121892601251602
193.679487179487 0.107520885765553
205.487179487179 0.133015438914299
218.012820512821 0.128814786672592
231.301282051282 0.106131754815578
245.403846153846 0.118802554905415
260.358974358974 0.156643003225327
276.230769230769 0.107456170022488
293.070512820513 0.127920612692833
310.935897435897 0.124383457005024
329.884615384615 0.105284251272678
350 0.108771987259388
};
\addplot [, color0, opacity=0.6, mark=triangle*, mark size=0.5, mark options={solid,rotate=180}, only marks, forget plot]
table {%
1 0.764247119426727
1.05769230769231 0.7348473072052
1.12179487179487 0.734258532524109
1.19230769230769 0.69929701089859
1.26282051282051 0.720068037509918
1.33974358974359 0.674318969249725
1.42307692307692 0.567017912864685
1.51282051282051 0.600959837436676
1.6025641025641 0.57817006111145
1.69871794871795 0.54201203584671
1.80128205128205 0.625092089176178
1.91666666666667 0.52514123916626
2.03205128205128 0.567103505134583
2.15384615384615 0.532314717769623
2.28846153846154 0.518073499202728
2.42307692307692 0.50467985868454
2.57692307692308 0.481521606445312
2.73076923076923 0.467195868492126
2.8974358974359 0.44583448767662
3.07692307692308 0.40271207690239
3.26282051282051 0.366946548223495
3.46153846153846 0.403023362159729
3.67307692307692 0.382703304290771
3.8974358974359 0.373981893062592
4.13461538461539 0.381218940019608
4.38461538461539 0.32429713010788
4.65384615384615 0.338891476392746
4.93589743589744 0.419111788272858
5.23717948717949 0.402944028377533
5.55769230769231 0.35328197479248
5.8974358974359 0.347149699926376
6.25641025641026 0.340856552124023
6.64102564102564 0.318451464176178
7.04487179487179 0.316605895757675
7.47435897435897 0.361397117376328
7.92948717948718 0.35025829076767
8.41025641025641 0.303041070699692
8.92307692307692 0.349312543869019
9.46794871794872 0.360333859920502
10.0448717948718 0.355277687311172
10.6602564102564 0.328618317842484
11.3076923076923 0.317378550767899
12 0.315894663333893
12.7307692307692 0.327141970396042
13.5064102564103 0.280912011861801
14.3333333333333 0.310359448194504
15.2051282051282 0.293037384748459
16.1346153846154 0.300409764051437
17.1153846153846 0.316252708435059
18.1602564102564 0.291019469499588
19.2692307692308 0.27197939157486
20.4423076923077 0.274129539728165
21.6858974358974 0.272359490394592
23.0128205128205 0.268627792596817
24.4102564102564 0.269264280796051
25.9038461538462 0.271905690431595
27.4807692307692 0.242468535900116
29.1538461538462 0.221639230847359
30.9358974358974 0.233358561992645
32.8205128205128 0.226131319999695
34.8205128205128 0.215400099754333
36.9423076923077 0.221809521317482
39.1923076923077 0.22572486102581
41.5833333333333 0.210478469729424
44.1153846153846 0.194761008024216
46.8076923076923 0.195926740765572
49.6602564102564 0.180807664990425
52.6858974358974 0.175872460007668
55.8974358974359 0.185876503586769
59.3076923076923 0.163679763674736
62.9230769230769 0.162954971194267
66.7564102564103 0.164993613958359
70.8269230769231 0.1704271286726
75.1474358974359 0.14290052652359
79.7307692307692 0.150410667061806
84.5897435897436 0.151391357183456
89.7435897435897 0.138884261250496
95.2179487179487 0.138739630579948
101.019230769231 0.141275629401207
107.179487179487 0.142683386802673
113.711538461538 0.135256573557854
120.641025641026 0.142736420035362
127.99358974359 0.128566637635231
135.801282051282 0.12341833114624
144.076923076923 0.134741440415382
152.858974358974 0.132130607962608
162.179487179487 0.133351698517799
172.064102564103 0.1257683634758
182.551282051282 0.140646800398827
193.679487179487 0.120016902685165
205.487179487179 0.134399116039276
218.012820512821 0.13110139966011
231.301282051282 0.122370697557926
245.403846153846 0.128804981708527
260.358974358974 0.120616093277931
276.230769230769 0.127030998468399
293.070512820513 0.129939705133438
310.935897435897 0.126311510801315
329.884615384615 0.128426849842072
350 0.129538461565971
};
\addplot [, color1, opacity=0.6, mark=square*, mark size=0.5, mark options={solid}, only marks]
table {%
1 0.826257824897766
1.05769230769231 0.81615537405014
1.12179487179487 0.838696956634521
1.19230769230769 0.847310602664948
1.26282051282051 0.87627911567688
1.33974358974359 0.929170489311218
1.42307692307692 0.932943403720856
1.51282051282051 0.86830872297287
1.6025641025641 0.847339510917664
1.69871794871795 0.800785899162292
1.80128205128205 0.78944057226181
1.91666666666667 0.776248157024384
2.03205128205128 0.777317941188812
2.15384615384615 0.790808022022247
2.28846153846154 0.676606595516205
2.42307692307692 0.64393538236618
2.57692307692308 0.604737401008606
2.73076923076923 0.556989312171936
2.8974358974359 0.575447082519531
3.07692307692308 0.510622441768646
3.26282051282051 0.53592848777771
3.46153846153846 0.524407923221588
3.67307692307692 0.521623969078064
3.8974358974359 0.488293677568436
4.13461538461539 0.536375284194946
4.38461538461539 0.46405902504921
4.65384615384615 0.4847352206707
4.93589743589744 0.498333126306534
5.23717948717949 0.438215047121048
5.55769230769231 0.470254093408585
5.8974358974359 0.457055956125259
6.25641025641026 0.478402614593506
6.64102564102564 0.422178745269775
7.04487179487179 0.474831789731979
7.47435897435897 0.462420254945755
7.92948717948718 0.438222050666809
8.41025641025641 0.404840618371964
8.92307692307692 0.411138474941254
9.46794871794872 0.435670047998428
10.0448717948718 0.432304710149765
10.6602564102564 0.448863297700882
11.3076923076923 0.400128126144409
12 0.423079907894135
12.7307692307692 0.379280477762222
13.5064102564103 0.36703634262085
14.3333333333333 0.388284623622894
15.2051282051282 0.361192017793655
16.1346153846154 0.357839375734329
17.1153846153846 0.351929932832718
18.1602564102564 0.350987166166306
19.2692307692308 0.339143961668015
20.4423076923077 0.328970640897751
21.6858974358974 0.316131442785263
23.0128205128205 0.324496865272522
24.4102564102564 0.305449575185776
25.9038461538462 0.303669214248657
27.4807692307692 0.282682925462723
29.1538461538462 0.278768599033356
30.9358974358974 0.272744059562683
32.8205128205128 0.27568382024765
34.8205128205128 0.265415996313095
36.9423076923077 0.257872492074966
39.1923076923077 0.245646446943283
41.5833333333333 0.241412103176117
44.1153846153846 0.239516943693161
46.8076923076923 0.232863262295723
49.6602564102564 0.225365057587624
52.6858974358974 0.206094920635223
55.8974358974359 0.217962607741356
59.3076923076923 0.199789389967918
62.9230769230769 0.21735030412674
66.7564102564103 0.214262843132019
70.8269230769231 0.215653792023659
75.1474358974359 0.196440726518631
79.7307692307692 0.191430255770683
84.5897435897436 0.185608059167862
89.7435897435897 0.185607969760895
95.2179487179487 0.18092155456543
101.019230769231 0.177552372217178
107.179487179487 0.16897189617157
113.711538461538 0.169628813862801
120.641025641026 0.178713262081146
127.99358974359 0.171615600585938
135.801282051282 0.178740337491035
144.076923076923 0.193704605102539
152.858974358974 0.159829303622246
162.179487179487 0.178689062595367
172.064102564103 0.170107632875443
182.551282051282 0.16723895072937
193.679487179487 0.169360876083374
205.487179487179 0.17177352309227
218.012820512821 0.148355066776276
231.301282051282 0.162023916840553
245.403846153846 0.167919993400574
260.358974358974 0.166715890169144
276.230769230769 0.154170408844948
293.070512820513 0.170840978622437
310.935897435897 0.167491644620895
329.884615384615 0.155857592821121
350 0.132294043898582
};
\addlegendentry{mb 8, exact}
\addplot [, color1, opacity=0.6, mark=square*, mark size=0.5, mark options={solid}, only marks, forget plot]
table {%
1 0.83707332611084
1.05769230769231 0.828525602817535
1.12179487179487 0.844192922115326
1.19230769230769 0.85131424665451
1.26282051282051 0.857158124446869
1.33974358974359 0.874734461307526
1.42307692307692 0.883916139602661
1.51282051282051 0.871286630630493
1.6025641025641 0.824827194213867
1.69871794871795 0.862464427947998
1.80128205128205 0.834381222724915
1.91666666666667 0.788374602794647
2.03205128205128 0.795728981494904
2.15384615384615 0.767927467823029
2.28846153846154 0.763548731803894
2.42307692307692 0.722810626029968
2.57692307692308 0.678178906440735
2.73076923076923 0.715118944644928
2.8974358974359 0.686241924762726
3.07692307692308 0.696500241756439
3.26282051282051 0.678122937679291
3.46153846153846 0.689086973667145
3.67307692307692 0.639672636985779
3.8974358974359 0.622434198856354
4.13461538461539 0.667950093746185
4.38461538461539 0.611477792263031
4.65384615384615 0.602124571800232
4.93589743589744 0.62827605009079
5.23717948717949 0.588814735412598
5.55769230769231 0.571596920490265
5.8974358974359 0.589247941970825
6.25641025641026 0.556927084922791
6.64102564102564 0.491542220115662
7.04487179487179 0.51828545331955
7.47435897435897 0.511918485164642
7.92948717948718 0.482115119695663
8.41025641025641 0.438143759965897
8.92307692307692 0.46990230679512
9.46794871794872 0.488480716943741
10.0448717948718 0.467779040336609
10.6602564102564 0.44493380188942
11.3076923076923 0.396722376346588
12 0.417552262544632
12.7307692307692 0.405910700559616
13.5064102564103 0.385403960943222
14.3333333333333 0.404126346111298
15.2051282051282 0.378702998161316
16.1346153846154 0.367580413818359
17.1153846153846 0.36620706319809
18.1602564102564 0.353755414485931
19.2692307692308 0.321898251771927
20.4423076923077 0.343031406402588
21.6858974358974 0.319122642278671
23.0128205128205 0.327638417482376
24.4102564102564 0.305770576000214
25.9038461538462 0.307262063026428
27.4807692307692 0.284895241260529
29.1538461538462 0.285371482372284
30.9358974358974 0.27467668056488
32.8205128205128 0.26735657453537
34.8205128205128 0.278193116188049
36.9423076923077 0.276705056428909
39.1923076923077 0.263609111309052
41.5833333333333 0.258418560028076
44.1153846153846 0.251733273267746
46.8076923076923 0.237613514065742
49.6602564102564 0.221151381731033
52.6858974358974 0.226649895310402
55.8974358974359 0.231584832072258
59.3076923076923 0.221832886338234
62.9230769230769 0.216576755046844
66.7564102564103 0.212352022528648
70.8269230769231 0.220881879329681
75.1474358974359 0.209802210330963
79.7307692307692 0.19820012152195
84.5897435897436 0.186065018177032
89.7435897435897 0.187745466828346
95.2179487179487 0.200396135449409
101.019230769231 0.197164058685303
107.179487179487 0.187634125351906
113.711538461538 0.181326746940613
120.641025641026 0.180714949965477
127.99358974359 0.174303010106087
135.801282051282 0.194749385118484
144.076923076923 0.194793865084648
152.858974358974 0.18276534974575
162.179487179487 0.183307334780693
172.064102564103 0.194632634520531
182.551282051282 0.17491851747036
193.679487179487 0.159817755222321
205.487179487179 0.191907778382301
218.012820512821 0.160218507051468
231.301282051282 0.167851939797401
245.403846153846 0.174369767308235
260.358974358974 0.178396984934807
276.230769230769 0.163826674222946
293.070512820513 0.177297860383987
310.935897435897 0.176927715539932
329.884615384615 0.180696442723274
350 0.163719102740288
};
\addplot [, color1, opacity=0.6, mark=square*, mark size=0.5, mark options={solid}, only marks, forget plot]
table {%
1 0.901654958724976
1.05769230769231 0.885625898838043
1.12179487179487 0.913147270679474
1.19230769230769 0.920085072517395
1.26282051282051 0.92897093296051
1.33974358974359 0.939274549484253
1.42307692307692 0.939430356025696
1.51282051282051 0.869231879711151
1.6025641025641 0.843161344528198
1.69871794871795 0.731403768062592
1.80128205128205 0.728235006332397
1.91666666666667 0.712364971637726
2.03205128205128 0.68870484828949
2.15384615384615 0.72391676902771
2.28846153846154 0.707644939422607
2.42307692307692 0.61799031496048
2.57692307692308 0.600627481937408
2.73076923076923 0.671766817569733
2.8974358974359 0.625289857387543
3.07692307692308 0.619248926639557
3.26282051282051 0.610220432281494
3.46153846153846 0.646152913570404
3.67307692307692 0.562099277973175
3.8974358974359 0.618728220462799
4.13461538461539 0.629637658596039
4.38461538461539 0.570919573307037
4.65384615384615 0.641226351261139
4.93589743589744 0.622930645942688
5.23717948717949 0.651973843574524
5.55769230769231 0.593798637390137
5.8974358974359 0.593031287193298
6.25641025641026 0.612176239490509
6.64102564102564 0.559253931045532
7.04487179487179 0.516795873641968
7.47435897435897 0.536433815956116
7.92948717948718 0.511284291744232
8.41025641025641 0.460240930318832
8.92307692307692 0.478715896606445
9.46794871794872 0.511123061180115
10.0448717948718 0.450832843780518
10.6602564102564 0.45312961935997
11.3076923076923 0.393179774284363
12 0.402316212654114
12.7307692307692 0.393972128629684
13.5064102564103 0.362162619829178
14.3333333333333 0.369239568710327
15.2051282051282 0.334583133459091
16.1346153846154 0.331758826971054
17.1153846153846 0.328421026468277
18.1602564102564 0.311536997556686
19.2692307692308 0.300818175077438
20.4423076923077 0.297650784254074
21.6858974358974 0.292589664459229
23.0128205128205 0.287288784980774
24.4102564102564 0.274494171142578
25.9038461538462 0.285333126783371
27.4807692307692 0.264195322990417
29.1538461538462 0.245819807052612
30.9358974358974 0.243872746825218
32.8205128205128 0.250516921281815
34.8205128205128 0.25389164686203
36.9423076923077 0.244614973664284
39.1923076923077 0.232756793498993
41.5833333333333 0.227467104792595
44.1153846153846 0.218789517879486
46.8076923076923 0.207923889160156
49.6602564102564 0.202998608350754
52.6858974358974 0.206772267818451
55.8974358974359 0.207921177148819
59.3076923076923 0.19189926981926
62.9230769230769 0.200461313128471
66.7564102564103 0.200284242630005
70.8269230769231 0.187103226780891
75.1474358974359 0.174088954925537
79.7307692307692 0.190467223525047
84.5897435897436 0.180889695882797
89.7435897435897 0.157835364341736
95.2179487179487 0.164956286549568
101.019230769231 0.171970054507256
107.179487179487 0.165900260210037
113.711538461538 0.168534561991692
120.641025641026 0.158742427825928
127.99358974359 0.163976594805717
135.801282051282 0.153387039899826
144.076923076923 0.153875946998596
152.858974358974 0.150484412908554
162.179487179487 0.154045820236206
172.064102564103 0.154068917036057
182.551282051282 0.157973796129227
193.679487179487 0.148649796843529
205.487179487179 0.152493178844452
218.012820512821 0.148997440934181
231.301282051282 0.145644351840019
245.403846153846 0.153753012418747
260.358974358974 0.149159386754036
276.230769230769 0.13942976295948
293.070512820513 0.147379159927368
310.935897435897 0.143662378191948
329.884615384615 0.140376791357994
350 0.143178686499596
};
\addplot [, color1, opacity=0.6, mark=square*, mark size=0.5, mark options={solid}, only marks, forget plot]
table {%
1 0.905726730823517
1.05769230769231 0.901780843734741
1.12179487179487 0.917018175125122
1.19230769230769 0.907346487045288
1.26282051282051 0.896318435668945
1.33974358974359 0.881994962692261
1.42307692307692 0.837151825428009
1.51282051282051 0.844390213489532
1.6025641025641 0.740124642848969
1.69871794871795 0.761476933956146
1.80128205128205 0.732832014560699
1.91666666666667 0.695253133773804
2.03205128205128 0.670494139194489
2.15384615384615 0.674267113208771
2.28846153846154 0.715534090995789
2.42307692307692 0.666480243206024
2.57692307692308 0.606193363666534
2.73076923076923 0.680170297622681
2.8974358974359 0.660685360431671
3.07692307692308 0.689992129802704
3.26282051282051 0.667281031608582
3.46153846153846 0.688047111034393
3.67307692307692 0.618017315864563
3.8974358974359 0.634268224239349
4.13461538461539 0.650869131088257
4.38461538461539 0.590750992298126
4.65384615384615 0.612287163734436
4.93589743589744 0.613946497440338
5.23717948717949 0.623297810554504
5.55769230769231 0.5895015001297
5.8974358974359 0.55329167842865
6.25641025641026 0.575298428535461
6.64102564102564 0.54318755865097
7.04487179487179 0.509299337863922
7.47435897435897 0.506684839725494
7.92948717948718 0.48899832367897
8.41025641025641 0.444532990455627
8.92307692307692 0.44579541683197
9.46794871794872 0.462139129638672
10.0448717948718 0.440410673618317
10.6602564102564 0.422998875379562
11.3076923076923 0.405236124992371
12 0.386991500854492
12.7307692307692 0.377220451831818
13.5064102564103 0.357557505369186
14.3333333333333 0.351019352674484
15.2051282051282 0.339099198579788
16.1346153846154 0.336545407772064
17.1153846153846 0.326288253068924
18.1602564102564 0.316090852022171
19.2692307692308 0.299073547124863
20.4423076923077 0.284730017185211
21.6858974358974 0.311986356973648
23.0128205128205 0.292280554771423
24.4102564102564 0.282726764678955
25.9038461538462 0.287112832069397
27.4807692307692 0.260706126689911
29.1538461538462 0.266904592514038
30.9358974358974 0.264168232679367
32.8205128205128 0.263894528150558
34.8205128205128 0.259998172521591
36.9423076923077 0.26892164349556
39.1923076923077 0.243801757693291
41.5833333333333 0.246940091252327
44.1153846153846 0.242637172341347
46.8076923076923 0.231994807720184
49.6602564102564 0.221041098237038
52.6858974358974 0.219667389988899
55.8974358974359 0.211789578199387
59.3076923076923 0.218019410967827
62.9230769230769 0.217432975769043
66.7564102564103 0.212504327297211
70.8269230769231 0.219117343425751
75.1474358974359 0.201111793518066
79.7307692307692 0.199809849262238
84.5897435897436 0.190897017717361
89.7435897435897 0.189757168292999
95.2179487179487 0.196953967213631
101.019230769231 0.187993273139
107.179487179487 0.185502216219902
113.711538461538 0.174838498234749
120.641025641026 0.189050674438477
127.99358974359 0.175166353583336
135.801282051282 0.187329784035683
144.076923076923 0.181408986449242
152.858974358974 0.173323526978493
162.179487179487 0.185369834303856
172.064102564103 0.185566931962967
182.551282051282 0.165965482592583
193.679487179487 0.166245117783546
205.487179487179 0.189491853117943
218.012820512821 0.169754669070244
231.301282051282 0.146876499056816
245.403846153846 0.169072031974792
260.358974358974 0.162103191018105
276.230769230769 0.150783360004425
293.070512820513 0.149913683533669
310.935897435897 0.169294282793999
329.884615384615 0.16040313243866
350 0.144597247242928
};
\addplot [, color1, opacity=0.6, mark=square*, mark size=0.5, mark options={solid}, only marks, forget plot]
table {%
1 0.872075617313385
1.05769230769231 0.870210886001587
1.12179487179487 0.874342560768127
1.19230769230769 0.864269971847534
1.26282051282051 0.864239573478699
1.33974358974359 0.848511576652527
1.42307692307692 0.807054281234741
1.51282051282051 0.814929306507111
1.6025641025641 0.727291405200958
1.69871794871795 0.727852523326874
1.80128205128205 0.78208714723587
1.91666666666667 0.753648042678833
2.03205128205128 0.774862587451935
2.15384615384615 0.708811163902283
2.28846153846154 0.680289745330811
2.42307692307692 0.628676474094391
2.57692307692308 0.615075588226318
2.73076923076923 0.664217829704285
2.8974358974359 0.624742269515991
3.07692307692308 0.570719182491302
3.26282051282051 0.568798184394836
3.46153846153846 0.590399503707886
3.67307692307692 0.574808299541473
3.8974358974359 0.580270290374756
4.13461538461539 0.532823622226715
4.38461538461539 0.509966790676117
4.65384615384615 0.533484876155853
4.93589743589744 0.568180739879608
5.23717948717949 0.53875344991684
5.55769230769231 0.508096396923065
5.8974358974359 0.527517974376678
6.25641025641026 0.517975449562073
6.64102564102564 0.486270844936371
7.04487179487179 0.459133833646774
7.47435897435897 0.510524570941925
7.92948717948718 0.47858515381813
8.41025641025641 0.453653544187546
8.92307692307692 0.441356033086777
9.46794871794872 0.433909744024277
10.0448717948718 0.451208055019379
10.6602564102564 0.460223138332367
11.3076923076923 0.413667440414429
12 0.419343411922455
12.7307692307692 0.394958347082138
13.5064102564103 0.357521116733551
14.3333333333333 0.39045238494873
15.2051282051282 0.364071220159531
16.1346153846154 0.346015453338623
17.1153846153846 0.358793675899506
18.1602564102564 0.346105515956879
19.2692307692308 0.315193921327591
20.4423076923077 0.307340502738953
21.6858974358974 0.301022261381149
23.0128205128205 0.294561862945557
24.4102564102564 0.296223938465118
25.9038461538462 0.28548076748848
27.4807692307692 0.271098583936691
29.1538461538462 0.259377092123032
30.9358974358974 0.252476543188095
32.8205128205128 0.258589088916779
34.8205128205128 0.249224275350571
36.9423076923077 0.242902755737305
39.1923076923077 0.235408410429955
41.5833333333333 0.23461540043354
44.1153846153846 0.224891185760498
46.8076923076923 0.209940060973167
49.6602564102564 0.204498827457428
52.6858974358974 0.206268534064293
55.8974358974359 0.201454848051071
59.3076923076923 0.199054777622223
62.9230769230769 0.201070249080658
66.7564102564103 0.182482466101646
70.8269230769231 0.182601392269135
75.1474358974359 0.194788619875908
79.7307692307692 0.174929156899452
84.5897435897436 0.159132033586502
89.7435897435897 0.1697818338871
95.2179487179487 0.156913012266159
101.019230769231 0.161422312259674
107.179487179487 0.175281003117561
113.711538461538 0.169483408331871
120.641025641026 0.167330592870712
127.99358974359 0.167158275842667
135.801282051282 0.15289531648159
144.076923076923 0.159689053893089
152.858974358974 0.152811244130135
162.179487179487 0.144601538777351
172.064102564103 0.162533789873123
182.551282051282 0.1522586196661
193.679487179487 0.154987916350365
205.487179487179 0.150681048631668
218.012820512821 0.150590762495995
231.301282051282 0.158257678151131
245.403846153846 0.161679610610008
260.358974358974 0.146874368190765
276.230769230769 0.151522114872932
293.070512820513 0.158786624670029
310.935897435897 0.146680146455765
329.884615384615 0.152834355831146
350 0.137296974658966
};
\addplot [, color2, opacity=0.6, mark=diamond*, mark size=0.5, mark options={solid}, only marks]
table {%
1 0.958740532398224
1.05769230769231 0.957960844039917
1.12179487179487 0.969886600971222
1.19230769230769 0.969629049301147
1.26282051282051 0.959771692752838
1.33974358974359 0.965772390365601
1.42307692307692 0.95246696472168
1.51282051282051 0.956778228282928
1.6025641025641 0.955083310604095
1.69871794871795 0.934446454048157
1.80128205128205 0.928161382675171
1.91666666666667 0.920785367488861
2.03205128205128 0.911284863948822
2.15384615384615 0.923168480396271
2.28846153846154 0.918965876102448
2.42307692307692 0.909889578819275
2.57692307692308 0.87801867723465
2.73076923076923 0.858970463275909
2.8974358974359 0.859918832778931
3.07692307692308 0.857446193695068
3.26282051282051 0.853565335273743
3.46153846153846 0.837840735912323
3.67307692307692 0.847639381885529
3.8974358974359 0.812911033630371
4.13461538461539 0.830991268157959
4.38461538461539 0.804028272628784
4.65384615384615 0.788401901721954
4.93589743589744 0.812549948692322
5.23717948717949 0.792973458766937
5.55769230769231 0.77584046125412
5.8974358974359 0.784420371055603
6.25641025641026 0.768277704715729
6.64102564102564 0.734220325946808
7.04487179487179 0.746353030204773
7.47435897435897 0.736869931221008
7.92948717948718 0.696171045303345
8.41025641025641 0.677179038524628
8.92307692307692 0.665870547294617
9.46794871794872 0.648605465888977
10.0448717948718 0.625814735889435
10.6602564102564 0.612020194530487
11.3076923076923 0.571928262710571
12 0.565243303775787
12.7307692307692 0.550774931907654
13.5064102564103 0.52460515499115
14.3333333333333 0.526878952980042
15.2051282051282 0.487041383981705
16.1346153846154 0.470037937164307
17.1153846153846 0.469595938920975
18.1602564102564 0.455542296171188
19.2692307692308 0.444737762212753
20.4423076923077 0.424640506505966
21.6858974358974 0.410325229167938
23.0128205128205 0.414948254823685
24.4102564102564 0.384088128805161
25.9038461538462 0.383041262626648
27.4807692307692 0.366610378026962
29.1538461538462 0.354704916477203
30.9358974358974 0.354157716035843
32.8205128205128 0.341494083404541
34.8205128205128 0.332605123519897
36.9423076923077 0.318306058645248
39.1923076923077 0.314307272434235
41.5833333333333 0.301935940980911
44.1153846153846 0.292566508054733
46.8076923076923 0.295503884553909
49.6602564102564 0.271244525909424
52.6858974358974 0.275856971740723
55.8974358974359 0.26583793759346
59.3076923076923 0.259915560483932
62.9230769230769 0.257468044757843
66.7564102564103 0.258391827344894
70.8269230769231 0.254494279623032
75.1474358974359 0.248070403933525
79.7307692307692 0.245457276701927
84.5897435897436 0.230459555983543
89.7435897435897 0.227346390485764
95.2179487179487 0.22085765004158
101.019230769231 0.219558849930763
107.179487179487 0.230380207300186
113.711538461538 0.215252608060837
120.641025641026 0.219602808356285
127.99358974359 0.213432759046555
135.801282051282 0.217174887657166
144.076923076923 0.22421333193779
152.858974358974 0.221350744366646
162.179487179487 0.213250041007996
172.064102564103 0.209605321288109
182.551282051282 0.203882366418839
193.679487179487 0.209196522831917
205.487179487179 0.198052808642387
218.012820512821 0.20618149638176
231.301282051282 0.198604315519333
245.403846153846 0.209965243935585
260.358974358974 0.19192798435688
276.230769230769 0.192276313900948
293.070512820513 0.188002586364746
310.935897435897 0.201529383659363
329.884615384615 0.201391980051994
350 0.193713426589966
};
\addlegendentry{mb 32, exact}
\addplot [, color2, opacity=0.6, mark=diamond*, mark size=0.5, mark options={solid}, only marks, forget plot]
table {%
1 0.928076982498169
1.05769230769231 0.921423494815826
1.12179487179487 0.933827817440033
1.19230769230769 0.94065260887146
1.26282051282051 0.9408038854599
1.33974358974359 0.975385963916779
1.42307692307692 0.980409979820251
1.51282051282051 0.942822158336639
1.6025641025641 0.936015605926514
1.69871794871795 0.926747858524323
1.80128205128205 0.919594943523407
1.91666666666667 0.914653658866882
2.03205128205128 0.923152089118958
2.15384615384615 0.916812598705292
2.28846153846154 0.895771741867065
2.42307692307692 0.900333404541016
2.57692307692308 0.895005881786346
2.73076923076923 0.860407769680023
2.8974358974359 0.890775263309479
3.07692307692308 0.871246635913849
3.26282051282051 0.828527331352234
3.46153846153846 0.8442622423172
3.67307692307692 0.838485419750214
3.8974358974359 0.814291834831238
4.13461538461539 0.810585737228394
4.38461538461539 0.783009648323059
4.65384615384615 0.797901213169098
4.93589743589744 0.806826889514923
5.23717948717949 0.816460490226746
5.55769230769231 0.768466770648956
5.8974358974359 0.768082559108734
6.25641025641026 0.759775459766388
6.64102564102564 0.746167600154877
7.04487179487179 0.717500686645508
7.47435897435897 0.734722375869751
7.92948717948718 0.695993959903717
8.41025641025641 0.664258778095245
8.92307692307692 0.650263249874115
9.46794871794872 0.659684062004089
10.0448717948718 0.63140481710434
10.6602564102564 0.609898269176483
11.3076923076923 0.582761466503143
12 0.571419239044189
12.7307692307692 0.547248959541321
13.5064102564103 0.528351962566376
14.3333333333333 0.517994523048401
15.2051282051282 0.492573827505112
16.1346153846154 0.478319048881531
17.1153846153846 0.46678164601326
18.1602564102564 0.450553148984909
19.2692307692308 0.428739756345749
20.4423076923077 0.42446905374527
21.6858974358974 0.405989706516266
23.0128205128205 0.396386295557022
24.4102564102564 0.385483235120773
25.9038461538462 0.381696224212646
27.4807692307692 0.365975052118301
29.1538461538462 0.367674559354782
30.9358974358974 0.349917083978653
32.8205128205128 0.3456891477108
34.8205128205128 0.333186894655228
36.9423076923077 0.338122248649597
39.1923076923077 0.32894441485405
41.5833333333333 0.309143513441086
44.1153846153846 0.300344377756119
46.8076923076923 0.306044578552246
49.6602564102564 0.288070857524872
52.6858974358974 0.289959967136383
55.8974358974359 0.274813622236252
59.3076923076923 0.266938626766205
62.9230769230769 0.268501400947571
66.7564102564103 0.267278403043747
70.8269230769231 0.25799286365509
75.1474358974359 0.240558460354805
79.7307692307692 0.257645159959793
84.5897435897436 0.248015955090523
89.7435897435897 0.23422746360302
95.2179487179487 0.229973286390305
101.019230769231 0.224706947803497
107.179487179487 0.235950276255608
113.711538461538 0.214408293366432
120.641025641026 0.230885460972786
127.99358974359 0.21308109164238
135.801282051282 0.221675470471382
144.076923076923 0.226335749030113
152.858974358974 0.227410241961479
162.179487179487 0.213751688599586
172.064102564103 0.223280444741249
182.551282051282 0.204619064927101
193.679487179487 0.204957455396652
205.487179487179 0.213681295514107
218.012820512821 0.193643108010292
231.301282051282 0.206665679812431
245.403846153846 0.220243036746979
260.358974358974 0.192472293972969
276.230769230769 0.194303318858147
293.070512820513 0.19511978328228
310.935897435897 0.203643679618835
329.884615384615 0.203341662883759
350 0.183957323431969
};
\addplot [, color2, opacity=0.6, mark=diamond*, mark size=0.5, mark options={solid}, only marks, forget plot]
table {%
1 0.924734771251678
1.05769230769231 0.929189562797546
1.12179487179487 0.93914657831192
1.19230769230769 0.939278721809387
1.26282051282051 0.930522620677948
1.33974358974359 0.946722984313965
1.42307692307692 0.940815508365631
1.51282051282051 0.899995565414429
1.6025641025641 0.850165367126465
1.69871794871795 0.835371434688568
1.80128205128205 0.823898315429688
1.91666666666667 0.827328264713287
2.03205128205128 0.818470656871796
2.15384615384615 0.800453186035156
2.28846153846154 0.776821136474609
2.42307692307692 0.723036766052246
2.57692307692308 0.794639587402344
2.73076923076923 0.785894215106964
2.8974358974359 0.739506542682648
3.07692307692308 0.733889281749725
3.26282051282051 0.697848498821259
3.46153846153846 0.77585107088089
3.67307692307692 0.716930270195007
3.8974358974359 0.738352119922638
4.13461538461539 0.740175664424896
4.38461538461539 0.669004201889038
4.65384615384615 0.723458111286163
4.93589743589744 0.726921379566193
5.23717948717949 0.679857611656189
5.55769230769231 0.677678763866425
5.8974358974359 0.667704939842224
6.25641025641026 0.64570564031601
6.64102564102564 0.667117238044739
7.04487179487179 0.650012195110321
7.47435897435897 0.665595531463623
7.92948717948718 0.65048885345459
8.41025641025641 0.588063359260559
8.92307692307692 0.591038465499878
9.46794871794872 0.595911204814911
10.0448717948718 0.586964428424835
10.6602564102564 0.585464954376221
11.3076923076923 0.548656523227692
12 0.549184799194336
12.7307692307692 0.539573609828949
13.5064102564103 0.515588223934174
14.3333333333333 0.514450132846832
15.2051282051282 0.486786693334579
16.1346153846154 0.480277091264725
17.1153846153846 0.45027494430542
18.1602564102564 0.446929037570953
19.2692307692308 0.428293436765671
20.4423076923077 0.416857719421387
21.6858974358974 0.408630132675171
23.0128205128205 0.401803493499756
24.4102564102564 0.394508421421051
25.9038461538462 0.38070473074913
27.4807692307692 0.371781229972839
29.1538461538462 0.361351311206818
30.9358974358974 0.354943692684174
32.8205128205128 0.334377348423004
34.8205128205128 0.33829864859581
36.9423076923077 0.324125438928604
39.1923076923077 0.308633923530579
41.5833333333333 0.3177250623703
44.1153846153846 0.295964688062668
46.8076923076923 0.291080951690674
49.6602564102564 0.268130362033844
52.6858974358974 0.276152402162552
55.8974358974359 0.275618821382523
59.3076923076923 0.261385887861252
62.9230769230769 0.259633749723434
66.7564102564103 0.26077601313591
70.8269230769231 0.2527736723423
75.1474358974359 0.248950645327568
79.7307692307692 0.242367446422577
84.5897435897436 0.232360795140266
89.7435897435897 0.223584935069084
95.2179487179487 0.226964339613914
101.019230769231 0.229025229811668
107.179487179487 0.22955359518528
113.711538461538 0.231052085757256
120.641025641026 0.236114799976349
127.99358974359 0.22369946539402
135.801282051282 0.210655704140663
144.076923076923 0.21586662530899
152.858974358974 0.2142553627491
162.179487179487 0.212092459201813
172.064102564103 0.213163331151009
182.551282051282 0.200809061527252
193.679487179487 0.202483966946602
205.487179487179 0.221096575260162
218.012820512821 0.207335501909256
231.301282051282 0.197326943278313
245.403846153846 0.218048647046089
260.358974358974 0.196242645382881
276.230769230769 0.192530930042267
293.070512820513 0.191071093082428
310.935897435897 0.212534353137016
329.884615384615 0.202694550156593
350 0.197711139917374
};
\addplot [, color2, opacity=0.6, mark=diamond*, mark size=0.5, mark options={solid}, only marks, forget plot]
table {%
1 0.922390401363373
1.05769230769231 0.918475091457367
1.12179487179487 0.937286198139191
1.19230769230769 0.934059858322144
1.26282051282051 0.936414480209351
1.33974358974359 0.938175618648529
1.42307692307692 0.927669286727905
1.51282051282051 0.89090359210968
1.6025641025641 0.876926720142365
1.69871794871795 0.887462854385376
1.80128205128205 0.856290280818939
1.91666666666667 0.823980391025543
2.03205128205128 0.828077554702759
2.15384615384615 0.844301283359528
2.28846153846154 0.799807012081146
2.42307692307692 0.815638840198517
2.57692307692308 0.772512495517731
2.73076923076923 0.77386349439621
2.8974358974359 0.792642652988434
3.07692307692308 0.770532071590424
3.26282051282051 0.750882387161255
3.46153846153846 0.758179664611816
3.67307692307692 0.750505864620209
3.8974358974359 0.719296097755432
4.13461538461539 0.747840583324432
4.38461538461539 0.708653688430786
4.65384615384615 0.713885128498077
4.93589743589744 0.757978439331055
5.23717948717949 0.75123655796051
5.55769230769231 0.719912469387054
5.8974358974359 0.68965095281601
6.25641025641026 0.720750272274017
6.64102564102564 0.664137840270996
7.04487179487179 0.646975219249725
7.47435897435897 0.684446692466736
7.92948717948718 0.654565691947937
8.41025641025641 0.602857410907745
8.92307692307692 0.625370919704437
9.46794871794872 0.621292114257812
10.0448717948718 0.606795966625214
10.6602564102564 0.585344672203064
11.3076923076923 0.555417358875275
12 0.535142064094543
12.7307692307692 0.525905728340149
13.5064102564103 0.508110582828522
14.3333333333333 0.507404446601868
15.2051282051282 0.47355592250824
16.1346153846154 0.460442185401917
17.1153846153846 0.453128188848495
18.1602564102564 0.447207391262054
19.2692307692308 0.419102340936661
20.4423076923077 0.428554892539978
21.6858974358974 0.404138416051865
23.0128205128205 0.403330981731415
24.4102564102564 0.392592161893845
25.9038461538462 0.375977456569672
27.4807692307692 0.370479583740234
29.1538461538462 0.361727595329285
30.9358974358974 0.358311831951141
32.8205128205128 0.346536368131638
34.8205128205128 0.330462753772736
36.9423076923077 0.326254040002823
39.1923076923077 0.313528180122375
41.5833333333333 0.320139229297638
44.1153846153846 0.304629117250443
46.8076923076923 0.305407166481018
49.6602564102564 0.287006258964539
52.6858974358974 0.287801742553711
55.8974358974359 0.272196292877197
59.3076923076923 0.256637006998062
62.9230769230769 0.266904979944229
66.7564102564103 0.263692080974579
70.8269230769231 0.266323268413544
75.1474358974359 0.253909081220627
79.7307692307692 0.261747002601624
84.5897435897436 0.233980968594551
89.7435897435897 0.230812430381775
95.2179487179487 0.226860076189041
101.019230769231 0.236718475818634
107.179487179487 0.225004881620407
113.711538461538 0.211820214986801
120.641025641026 0.231803730130196
127.99358974359 0.218456000089645
135.801282051282 0.226013258099556
144.076923076923 0.222635000944138
152.858974358974 0.220748871564865
162.179487179487 0.212863758206367
172.064102564103 0.184295624494553
182.551282051282 0.200685098767281
193.679487179487 0.1981131285429
205.487179487179 0.210242852568626
218.012820512821 0.196124523878098
231.301282051282 0.199326351284981
245.403846153846 0.210449174046516
260.358974358974 0.190186575055122
276.230769230769 0.184068948030472
293.070512820513 0.189748644828796
310.935897435897 0.195758923888206
329.884615384615 0.198957040905952
350 0.18407641351223
};
\addplot [, color2, opacity=0.6, mark=diamond*, mark size=0.5, mark options={solid}, only marks, forget plot]
table {%
1 0.952925264835358
1.05769230769231 0.950379610061646
1.12179487179487 0.963039696216583
1.19230769230769 0.958707988262177
1.26282051282051 0.96318906545639
1.33974358974359 0.972240626811981
1.42307692307692 0.97203516960144
1.51282051282051 0.905118405818939
1.6025641025641 0.897236168384552
1.69871794871795 0.896579623222351
1.80128205128205 0.872727334499359
1.91666666666667 0.902436971664429
2.03205128205128 0.907840549945831
2.15384615384615 0.928442656993866
2.28846153846154 0.903498649597168
2.42307692307692 0.884651899337769
2.57692307692308 0.881089627742767
2.73076923076923 0.842694938182831
2.8974358974359 0.883152902126312
3.07692307692308 0.861822426319122
3.26282051282051 0.821399211883545
3.46153846153846 0.839963972568512
3.67307692307692 0.835405886173248
3.8974358974359 0.819187164306641
4.13461538461539 0.831318616867065
4.38461538461539 0.788109123706818
4.65384615384615 0.80332350730896
4.93589743589744 0.823441445827484
5.23717948717949 0.816209495067596
5.55769230769231 0.764937877655029
5.8974358974359 0.744358837604523
6.25641025641026 0.751301050186157
6.64102564102564 0.72545337677002
7.04487179487179 0.728115499019623
7.47435897435897 0.707328021526337
7.92948717948718 0.69020813703537
8.41025641025641 0.652195870876312
8.92307692307692 0.628446340560913
9.46794871794872 0.626954734325409
10.0448717948718 0.612270891666412
10.6602564102564 0.609319686889648
11.3076923076923 0.568204462528229
12 0.55857914686203
12.7307692307692 0.543052971363068
13.5064102564103 0.520628809928894
14.3333333333333 0.513640642166138
15.2051282051282 0.489234566688538
16.1346153846154 0.466807991266251
17.1153846153846 0.452858507633209
18.1602564102564 0.463645964860916
19.2692307692308 0.445450633764267
20.4423076923077 0.425617545843124
21.6858974358974 0.409767955541611
23.0128205128205 0.409733504056931
24.4102564102564 0.387439727783203
25.9038461538462 0.381030797958374
27.4807692307692 0.389577209949493
29.1538461538462 0.362588405609131
30.9358974358974 0.344708770513535
32.8205128205128 0.342593848705292
34.8205128205128 0.344805151224136
36.9423076923077 0.342003166675568
39.1923076923077 0.334688305854797
41.5833333333333 0.316735446453094
44.1153846153846 0.315196692943573
46.8076923076923 0.295460551977158
49.6602564102564 0.284201830625534
52.6858974358974 0.288486510515213
55.8974358974359 0.276005566120148
59.3076923076923 0.27626621723175
62.9230769230769 0.269196540117264
66.7564102564103 0.271945863962173
70.8269230769231 0.271693855524063
75.1474358974359 0.259626775979996
79.7307692307692 0.262505203485489
84.5897435897436 0.24746036529541
89.7435897435897 0.242875665426254
95.2179487179487 0.241382211446762
101.019230769231 0.245289668440819
107.179487179487 0.240549653768539
113.711538461538 0.237864911556244
120.641025641026 0.233640268445015
127.99358974359 0.228239208459854
135.801282051282 0.222788617014885
144.076923076923 0.236681953072548
152.858974358974 0.223878011107445
162.179487179487 0.226288974285126
172.064102564103 0.213757932186127
182.551282051282 0.215066179633141
193.679487179487 0.21755613386631
205.487179487179 0.218965634703636
218.012820512821 0.205340832471848
231.301282051282 0.206724405288696
245.403846153846 0.22197799384594
260.358974358974 0.217971250414848
276.230769230769 0.203976690769196
293.070512820513 0.210615813732147
310.935897435897 0.215168312191963
329.884615384615 0.210220068693161
350 0.193155437707901
};
\addplot [, black, opacity=0.6, mark=*, mark size=0.5, mark options={solid}, only marks]
table {%
1 0.977723062038422
1.05769230769231 0.979054987430573
1.12179487179487 0.985264897346497
1.19230769230769 0.978571832180023
1.26282051282051 0.981565058231354
1.33974358974359 0.989220082759857
1.42307692307692 0.987241506576538
1.51282051282051 0.983149230480194
1.6025641025641 0.978557586669922
1.69871794871795 0.9748894572258
1.80128205128205 0.966607809066772
1.91666666666667 0.976728200912476
2.03205128205128 0.973711550235748
2.15384615384615 0.96799236536026
2.28846153846154 0.964619398117065
2.42307692307692 0.961774706840515
2.57692307692308 0.959895551204681
2.73076923076923 0.931321382522583
2.8974358974359 0.948131918907166
3.07692307692308 0.946124732494354
3.26282051282051 0.924159824848175
3.46153846153846 0.933367609977722
3.67307692307692 0.930918097496033
3.8974358974359 0.906047940254211
4.13461538461539 0.933397948741913
4.38461538461539 0.908103466033936
4.65384615384615 0.914104759693146
4.93589743589744 0.91682505607605
5.23717948717949 0.909251868724823
5.55769230769231 0.893991589546204
5.8974358974359 0.884307205677032
6.25641025641026 0.883995175361633
6.64102564102564 0.870899796485901
7.04487179487179 0.873602271080017
7.47435897435897 0.859926879405975
7.92948717948718 0.851292550563812
8.41025641025641 0.826033592224121
8.92307692307692 0.809674024581909
9.46794871794872 0.802682459354401
10.0448717948718 0.775369882583618
10.6602564102564 0.762922048568726
11.3076923076923 0.73332691192627
12 0.712755024433136
12.7307692307692 0.7106773853302
13.5064102564103 0.677334666252136
14.3333333333333 0.666222274303436
15.2051282051282 0.659609735012054
16.1346153846154 0.641543507575989
17.1153846153846 0.62599766254425
18.1602564102564 0.611256182193756
19.2692307692308 0.59850150346756
20.4423076923077 0.580182135105133
21.6858974358974 0.564831972122192
23.0128205128205 0.552014231681824
24.4102564102564 0.54677677154541
25.9038461538462 0.524325311183929
27.4807692307692 0.532438218593597
29.1538461538462 0.516314744949341
30.9358974358974 0.500522911548615
32.8205128205128 0.482573807239532
34.8205128205128 0.47178789973259
36.9423076923077 0.458847403526306
39.1923076923077 0.450161963701248
41.5833333333333 0.450517177581787
44.1153846153846 0.429062783718109
46.8076923076923 0.422146588563919
49.6602564102564 0.412428021430969
52.6858974358974 0.395786464214325
55.8974358974359 0.397876501083374
59.3076923076923 0.37336927652359
62.9230769230769 0.379122376441956
66.7564102564103 0.35991033911705
70.8269230769231 0.361950904130936
75.1474358974359 0.346306294202805
79.7307692307692 0.337672114372253
84.5897435897436 0.323739618062973
89.7435897435897 0.312207967042923
95.2179487179487 0.314184576272964
101.019230769231 0.315516233444214
107.179487179487 0.298435926437378
113.711538461538 0.29811355471611
120.641025641026 0.293935835361481
127.99358974359 0.285696685314178
135.801282051282 0.277951121330261
144.076923076923 0.299682140350342
152.858974358974 0.27744334936142
162.179487179487 0.277605146169662
172.064102564103 0.281594663858414
182.551282051282 0.266644477844238
193.679487179487 0.273602783679962
205.487179487179 0.282504260540009
218.012820512821 0.253155469894409
231.301282051282 0.273877143859863
245.403846153846 0.289760053157806
260.358974358974 0.264607161283493
276.230769230769 0.255048602819443
293.070512820513 0.265161991119385
310.935897435897 0.254896283149719
329.884615384615 0.25929781794548
350 0.24486568570137
};
\addlegendentry{mb 128, exact}
\addplot [, black, opacity=0.6, mark=*, mark size=0.5, mark options={solid}, only marks, forget plot]
table {%
1 0.986895263195038
1.05769230769231 0.980787634849548
1.12179487179487 0.985461950302124
1.19230769230769 0.991671502590179
1.26282051282051 0.995383679866791
1.33974358974359 0.995356857776642
1.42307692307692 0.996258497238159
1.51282051282051 0.989187598228455
1.6025641025641 0.989968836307526
1.69871794871795 0.98010641336441
1.80128205128205 0.971567988395691
1.91666666666667 0.980160355567932
2.03205128205128 0.97682797908783
2.15384615384615 0.969399690628052
2.28846153846154 0.96156370639801
2.42307692307692 0.956043362617493
2.57692307692308 0.947030007839203
2.73076923076923 0.952257513999939
2.8974358974359 0.939471900463104
3.07692307692308 0.936847865581512
3.26282051282051 0.929396569728851
3.46153846153846 0.945724368095398
3.67307692307692 0.927689850330353
3.8974358974359 0.92981094121933
4.13461538461539 0.941705465316772
4.38461538461539 0.906059086322784
4.65384615384615 0.886296153068542
4.93589743589744 0.919317007064819
5.23717948717949 0.909274399280548
5.55769230769231 0.891047596931458
5.8974358974359 0.875378847122192
6.25641025641026 0.877204239368439
6.64102564102564 0.850964188575745
7.04487179487179 0.845690906047821
7.47435897435897 0.847758591175079
7.92948717948718 0.830593228340149
8.41025641025641 0.796346008777618
8.92307692307692 0.801896512508392
9.46794871794872 0.798394322395325
10.0448717948718 0.777367055416107
10.6602564102564 0.756431877613068
11.3076923076923 0.731467723846436
12 0.725441813468933
12.7307692307692 0.712042212486267
13.5064102564103 0.675029873847961
14.3333333333333 0.672647058963776
15.2051282051282 0.675873160362244
16.1346153846154 0.654091775417328
17.1153846153846 0.636921882629395
18.1602564102564 0.62453305721283
19.2692307692308 0.615143001079559
20.4423076923077 0.592768251895905
21.6858974358974 0.555790841579437
23.0128205128205 0.562988936901093
24.4102564102564 0.54248982667923
25.9038461538462 0.530127346515656
27.4807692307692 0.52151745557785
29.1538461538462 0.511283934116364
30.9358974358974 0.472534507513046
32.8205128205128 0.47830519080162
34.8205128205128 0.477901667356491
36.9423076923077 0.464605212211609
39.1923076923077 0.450261682271957
41.5833333333333 0.437666684389114
44.1153846153846 0.421621948480606
46.8076923076923 0.414727747440338
49.6602564102564 0.407199054956436
52.6858974358974 0.392490386962891
55.8974358974359 0.389752984046936
59.3076923076923 0.377608835697174
62.9230769230769 0.37678787112236
66.7564102564103 0.351413398981094
70.8269230769231 0.362528920173645
75.1474358974359 0.336033344268799
79.7307692307692 0.344733536243439
84.5897435897436 0.329869747161865
89.7435897435897 0.307305425405502
95.2179487179487 0.314863651990891
101.019230769231 0.318781226873398
107.179487179487 0.305703341960907
113.711538461538 0.307104140520096
120.641025641026 0.298313707113266
127.99358974359 0.2934889793396
135.801282051282 0.290980607271194
144.076923076923 0.295665860176086
152.858974358974 0.289781093597412
162.179487179487 0.290582776069641
172.064102564103 0.285847723484039
182.551282051282 0.271805763244629
193.679487179487 0.277707159519196
205.487179487179 0.296357899904251
218.012820512821 0.267042249441147
231.301282051282 0.260891109704971
245.403846153846 0.28995606303215
260.358974358974 0.259319484233856
276.230769230769 0.247401759028435
293.070512820513 0.268549770116806
310.935897435897 0.267071485519409
329.884615384615 0.268940776586533
350 0.254630863666534
};
\addplot [, black, opacity=0.6, mark=*, mark size=0.5, mark options={solid}, only marks, forget plot]
table {%
1 0.989810764789581
1.05769230769231 0.988471150398254
1.12179487179487 0.993696570396423
1.19230769230769 0.991568446159363
1.26282051282051 0.989658772945404
1.33974358974359 0.994469583034515
1.42307692307692 0.994873046875
1.51282051282051 0.981999516487122
1.6025641025641 0.97782027721405
1.69871794871795 0.950825691223145
1.80128205128205 0.94676661491394
1.91666666666667 0.966289281845093
2.03205128205128 0.947389423847198
2.15384615384615 0.943173050880432
2.28846153846154 0.939484834671021
2.42307692307692 0.935541212558746
2.57692307692308 0.913724660873413
2.73076923076923 0.912989616394043
2.8974358974359 0.912814259529114
3.07692307692308 0.922054409980774
3.26282051282051 0.883259117603302
3.46153846153846 0.909272313117981
3.67307692307692 0.90310537815094
3.8974358974359 0.898523211479187
4.13461538461539 0.89713728427887
4.38461538461539 0.875198900699615
4.65384615384615 0.8771613240242
4.93589743589744 0.901280164718628
5.23717948717949 0.898040771484375
5.55769230769231 0.871551036834717
5.8974358974359 0.868200063705444
6.25641025641026 0.8705033659935
6.64102564102564 0.855795979499817
7.04487179487179 0.846056878566742
7.47435897435897 0.861203134059906
7.92948717948718 0.82034707069397
8.41025641025641 0.804534256458282
8.92307692307692 0.789363920688629
9.46794871794872 0.812005281448364
10.0448717948718 0.783037722110748
10.6602564102564 0.772508680820465
11.3076923076923 0.730777204036713
12 0.723770558834076
12.7307692307692 0.710988879203796
13.5064102564103 0.699017226696014
14.3333333333333 0.679319143295288
15.2051282051282 0.667685866355896
16.1346153846154 0.64519476890564
17.1153846153846 0.61491471529007
18.1602564102564 0.609516561031342
19.2692307692308 0.609775364398956
20.4423076923077 0.582055509090424
21.6858974358974 0.55249285697937
23.0128205128205 0.555131196975708
24.4102564102564 0.544568479061127
25.9038461538462 0.534587323665619
27.4807692307692 0.520486116409302
29.1538461538462 0.50546795129776
30.9358974358974 0.485412120819092
32.8205128205128 0.4737608730793
34.8205128205128 0.471971601247787
36.9423076923077 0.471086144447327
39.1923076923077 0.451848834753036
41.5833333333333 0.43164137005806
44.1153846153846 0.423379212617874
46.8076923076923 0.42055556178093
49.6602564102564 0.412801057100296
52.6858974358974 0.402280032634735
55.8974358974359 0.381641149520874
59.3076923076923 0.37458735704422
62.9230769230769 0.377476722002029
66.7564102564103 0.365892171859741
70.8269230769231 0.365151822566986
75.1474358974359 0.343942672014236
79.7307692307692 0.364530503749847
84.5897435897436 0.321894973516464
89.7435897435897 0.310723483562469
95.2179487179487 0.308152675628662
101.019230769231 0.310962468385696
107.179487179487 0.319169253110886
113.711538461538 0.307224750518799
120.641025641026 0.313114106655121
127.99358974359 0.298181712627411
135.801282051282 0.288338422775269
144.076923076923 0.310233771800995
152.858974358974 0.280428826808929
162.179487179487 0.291744619607925
172.064102564103 0.282480090856552
182.551282051282 0.251580029726028
193.679487179487 0.268506616353989
205.487179487179 0.27697017788887
218.012820512821 0.276775896549225
231.301282051282 0.266894072294235
245.403846153846 0.281022578477859
260.358974358974 0.266442745923996
276.230769230769 0.251469820737839
293.070512820513 0.253527164459229
310.935897435897 0.275888025760651
329.884615384615 0.266400367021561
350 0.26028373837471
};
\addplot [, black, opacity=0.6, mark=*, mark size=0.5, mark options={solid}, only marks, forget plot]
table {%
1 0.98018878698349
1.05769230769231 0.979297637939453
1.12179487179487 0.990327954292297
1.19230769230769 0.990190386772156
1.26282051282051 0.979087889194489
1.33974358974359 0.987639188766479
1.42307692307692 0.986421942710876
1.51282051282051 0.988199234008789
1.6025641025641 0.988146662712097
1.69871794871795 0.97368973493576
1.80128205128205 0.96921980381012
1.91666666666667 0.97775137424469
2.03205128205128 0.97203665971756
2.15384615384615 0.975655674934387
2.28846153846154 0.960072219371796
2.42307692307692 0.958180367946625
2.57692307692308 0.957045197486877
2.73076923076923 0.946048080921173
2.8974358974359 0.934350490570068
3.07692307692308 0.94107311964035
3.26282051282051 0.917692720890045
3.46153846153846 0.925226330757141
3.67307692307692 0.920620560646057
3.8974358974359 0.908173799514771
4.13461538461539 0.913211286067963
4.38461538461539 0.896114945411682
4.65384615384615 0.913611173629761
4.93589743589744 0.88570636510849
5.23717948717949 0.912522733211517
5.55769230769231 0.887013673782349
5.8974358974359 0.882735788822174
6.25641025641026 0.888155519962311
6.64102564102564 0.858199119567871
7.04487179487179 0.850301206111908
7.47435897435897 0.85232800245285
7.92948717948718 0.837811231613159
8.41025641025641 0.79827094078064
8.92307692307692 0.813076674938202
9.46794871794872 0.812685370445251
10.0448717948718 0.79168975353241
10.6602564102564 0.77166622877121
11.3076923076923 0.748108983039856
12 0.756922662258148
12.7307692307692 0.73523360490799
13.5064102564103 0.697845876216888
14.3333333333333 0.695247530937195
15.2051282051282 0.691827058792114
16.1346153846154 0.659954905509949
17.1153846153846 0.637606084346771
18.1602564102564 0.626903355121613
19.2692307692308 0.611139118671417
20.4423076923077 0.596262276172638
21.6858974358974 0.56735759973526
23.0128205128205 0.55872106552124
24.4102564102564 0.563576519489288
25.9038461538462 0.540385246276855
27.4807692307692 0.532660186290741
29.1538461538462 0.512855768203735
30.9358974358974 0.511415302753448
32.8205128205128 0.486744821071625
34.8205128205128 0.496074587106705
36.9423076923077 0.491412729024887
39.1923076923077 0.46189683675766
41.5833333333333 0.450743168592453
44.1153846153846 0.438691288232803
46.8076923076923 0.430243343114853
49.6602564102564 0.417863875627518
52.6858974358974 0.418734014034271
55.8974358974359 0.399037927389145
59.3076923076923 0.379649102687836
62.9230769230769 0.380256116390228
66.7564102564103 0.369411200284958
70.8269230769231 0.372477054595947
75.1474358974359 0.34940168261528
79.7307692307692 0.362079828977585
84.5897435897436 0.328754127025604
89.7435897435897 0.328092277050018
95.2179487179487 0.321638017892838
101.019230769231 0.32082125544548
107.179487179487 0.309922963380814
113.711538461538 0.304620981216431
120.641025641026 0.319176018238068
127.99358974359 0.30242046713829
135.801282051282 0.299052149057388
144.076923076923 0.317478746175766
152.858974358974 0.296295762062073
162.179487179487 0.292951464653015
172.064102564103 0.294750928878784
182.551282051282 0.270549535751343
193.679487179487 0.285436391830444
205.487179487179 0.295127511024475
218.012820512821 0.267993271350861
231.301282051282 0.279267370700836
245.403846153846 0.282593756914139
260.358974358974 0.265568226575851
276.230769230769 0.264495581388474
293.070512820513 0.27492019534111
310.935897435897 0.279478132724762
329.884615384615 0.304643422365189
350 0.265547186136246
};
\addplot [, black, opacity=0.6, mark=*, mark size=0.5, mark options={solid}, only marks, forget plot]
table {%
1 0.973753869533539
1.05769230769231 0.97322815656662
1.12179487179487 0.983072459697723
1.19230769230769 0.97915130853653
1.26282051282051 0.985690414905548
1.33974358974359 0.989498734474182
1.42307692307692 0.989751994609833
1.51282051282051 0.981911420822144
1.6025641025641 0.973996162414551
1.69871794871795 0.978213787078857
1.80128205128205 0.969274580478668
1.91666666666667 0.967932105064392
2.03205128205128 0.980430424213409
2.15384615384615 0.967014133930206
2.28846153846154 0.95588618516922
2.42307692307692 0.955709993839264
2.57692307692308 0.945317804813385
2.73076923076923 0.935469806194305
2.8974358974359 0.94493842124939
3.07692307692308 0.94318675994873
3.26282051282051 0.935243487358093
3.46153846153846 0.938679456710815
3.67307692307692 0.918027341365814
3.8974358974359 0.92744243144989
4.13461538461539 0.926182985305786
4.38461538461539 0.908518970012665
4.65384615384615 0.908053278923035
4.93589743589744 0.888128399848938
5.23717948717949 0.898898422718048
5.55769230769231 0.886095881462097
5.8974358974359 0.886300325393677
6.25641025641026 0.878016352653503
6.64102564102564 0.866545975208282
7.04487179487179 0.86702436208725
7.47435897435897 0.861173689365387
7.92948717948718 0.839564204216003
8.41025641025641 0.817166268825531
8.92307692307692 0.805991172790527
9.46794871794872 0.818654000759125
10.0448717948718 0.794586002826691
10.6602564102564 0.791329264640808
11.3076923076923 0.748766601085663
12 0.737976968288422
12.7307692307692 0.723856210708618
13.5064102564103 0.698262870311737
14.3333333333333 0.685054183006287
15.2051282051282 0.68641859292984
16.1346153846154 0.652000546455383
17.1153846153846 0.634006202220917
18.1602564102564 0.631703495979309
19.2692307692308 0.605153739452362
20.4423076923077 0.593919992446899
21.6858974358974 0.56074196100235
23.0128205128205 0.553378820419312
24.4102564102564 0.544493138790131
25.9038461538462 0.53988242149353
27.4807692307692 0.524265229701996
29.1538461538462 0.504412651062012
30.9358974358974 0.501520276069641
32.8205128205128 0.491924613714218
34.8205128205128 0.476371258497238
36.9423076923077 0.46665033698082
39.1923076923077 0.446092903614044
41.5833333333333 0.450221657752991
44.1153846153846 0.419289171695709
46.8076923076923 0.415226817131042
49.6602564102564 0.401206821203232
52.6858974358974 0.382014006376266
55.8974358974359 0.394410997629166
59.3076923076923 0.363665223121643
62.9230769230769 0.362125068902969
66.7564102564103 0.36014986038208
70.8269230769231 0.360298007726669
75.1474358974359 0.331577956676483
79.7307692307692 0.344969004392624
84.5897435897436 0.328490436077118
89.7435897435897 0.323612362146378
95.2179487179487 0.318572223186493
101.019230769231 0.317898690700531
107.179487179487 0.303832083940506
113.711538461538 0.30374538898468
120.641025641026 0.301226437091827
127.99358974359 0.288552403450012
135.801282051282 0.307082146406174
144.076923076923 0.305051684379578
152.858974358974 0.285187929868698
162.179487179487 0.278125047683716
172.064102564103 0.299753844738007
182.551282051282 0.267872631549835
193.679487179487 0.263909816741943
205.487179487179 0.293269544839859
218.012820512821 0.261766344308853
231.301282051282 0.263620346784592
245.403846153846 0.277827829122543
260.358974358974 0.268119752407074
276.230769230769 0.245374888181686
293.070512820513 0.247209012508392
310.935897435897 0.281961977481842
329.884615384615 0.276907503604889
350 0.257376223802567
};
\end{axis}

\end{tikzpicture}

      \tikzexternaldisable
    \end{minipage}\hfill
    \begin{minipage}{0.50\linewidth}
      \centering
      % defines the pgfplots style "eigspacedefault"
\pgfkeys{/pgfplots/eigspacedefault/.style={
    width=1.0\linewidth,
    height=0.6\linewidth,
    every axis plot/.append style={line width = 1.5pt},
    tick pos = left,
    ylabel near ticks,
    xlabel near ticks,
    xtick align = inside,
    ytick align = inside,
    legend cell align = left,
    legend columns = 4,
    legend pos = south east,
    legend style = {
      fill opacity = 1,
      text opacity = 1,
      font = \footnotesize,
      at={(1, 1.025)},
      anchor=south east,
      column sep=0.25cm,
    },
    legend image post style={scale=2.5},
    xticklabel style = {font = \footnotesize},
    xlabel style = {font = \footnotesize},
    axis line style = {black},
    yticklabel style = {font = \footnotesize},
    ylabel style = {font = \footnotesize},
    title style = {font = \footnotesize},
    grid = major,
    grid style = {dashed}
  }
}

\pgfkeys{/pgfplots/eigspacedefaultapp/.style={
    eigspacedefault,
    height=0.6\linewidth,
    legend columns = 2,
  }
}

\pgfkeys{/pgfplots/eigspacenolegend/.style={
    legend image post style = {scale=0},
    legend style = {
      fill opacity = 0,
      draw opacity = 0,
      text opacity = 0,
      font = \footnotesize,
      at={(1, 1.025)},
      anchor=south east,
      column sep=0.25cm,
    },
  }
}
%%% Local Variables:
%%% mode: latex
%%% TeX-master: "../../thesis"
%%% End:

      \pgfkeys{/pgfplots/zmystyle/.style={
          eigspacedefaultapp,
          eigspacenolegend,
        }}
      \tikzexternalenable
      \vspace{-6ex}
      % This file was created by tikzplotlib v0.9.7.
\begin{tikzpicture}

\definecolor{color0}{rgb}{0.274509803921569,0.6,0.564705882352941}
\definecolor{color1}{rgb}{0.870588235294118,0.623529411764706,0.0862745098039216}
\definecolor{color2}{rgb}{0.501960784313725,0.184313725490196,0.6}

\begin{axis}[
axis line style={white!10!black},
legend columns=2,
legend style={fill opacity=0.8, draw opacity=1, text opacity=1, at={(0.03,0.03)}, anchor=south west, draw=white!80!black},
log basis x={10},
tick pos=left,
xlabel={epoch (log scale)},
xmajorgrids,
xmin=0.746099240306814, xmax=469.106495613199,
xmode=log,
ylabel={overlap},
ymajorgrids,
ymin=-0.05, ymax=1.05,
zmystyle
]
\addplot [, white!10!black, dashed, forget plot]
table {%
0.746099240306814 1
469.106495613199 1
};
\addplot [, white!10!black, dashed, forget plot]
table {%
0.746099240306814 0
469.106495613199 0
};
\addplot [, black, opacity=0.6, mark=*, mark size=0.5, mark options={solid}, only marks]
table {%
1 0.977723062038422
1.05769230769231 0.979054987430573
1.12179487179487 0.985264897346497
1.19230769230769 0.978571832180023
1.26282051282051 0.981565058231354
1.33974358974359 0.989220082759857
1.42307692307692 0.987241506576538
1.51282051282051 0.983149230480194
1.6025641025641 0.978557586669922
1.69871794871795 0.9748894572258
1.80128205128205 0.966607809066772
1.91666666666667 0.976728200912476
2.03205128205128 0.973711550235748
2.15384615384615 0.96799236536026
2.28846153846154 0.964619398117065
2.42307692307692 0.961774706840515
2.57692307692308 0.959895551204681
2.73076923076923 0.931321382522583
2.8974358974359 0.948131918907166
3.07692307692308 0.946124732494354
3.26282051282051 0.924159824848175
3.46153846153846 0.933367609977722
3.67307692307692 0.930918097496033
3.8974358974359 0.906047940254211
4.13461538461539 0.933397948741913
4.38461538461539 0.908103466033936
4.65384615384615 0.914104759693146
4.93589743589744 0.91682505607605
5.23717948717949 0.909251868724823
5.55769230769231 0.893991589546204
5.8974358974359 0.884307205677032
6.25641025641026 0.883995175361633
6.64102564102564 0.870899796485901
7.04487179487179 0.873602271080017
7.47435897435897 0.859926879405975
7.92948717948718 0.851292550563812
8.41025641025641 0.826033592224121
8.92307692307692 0.809674024581909
9.46794871794872 0.802682459354401
10.0448717948718 0.775369882583618
10.6602564102564 0.762922048568726
11.3076923076923 0.73332691192627
12 0.712755024433136
12.7307692307692 0.7106773853302
13.5064102564103 0.677334666252136
14.3333333333333 0.666222274303436
15.2051282051282 0.659609735012054
16.1346153846154 0.641543507575989
17.1153846153846 0.62599766254425
18.1602564102564 0.611256182193756
19.2692307692308 0.59850150346756
20.4423076923077 0.580182135105133
21.6858974358974 0.564831972122192
23.0128205128205 0.552014231681824
24.4102564102564 0.54677677154541
25.9038461538462 0.524325311183929
27.4807692307692 0.532438218593597
29.1538461538462 0.516314744949341
30.9358974358974 0.500522911548615
32.8205128205128 0.482573807239532
34.8205128205128 0.47178789973259
36.9423076923077 0.458847403526306
39.1923076923077 0.450161963701248
41.5833333333333 0.450517177581787
44.1153846153846 0.429062783718109
46.8076923076923 0.422146588563919
49.6602564102564 0.412428021430969
52.6858974358974 0.395786464214325
55.8974358974359 0.397876501083374
59.3076923076923 0.37336927652359
62.9230769230769 0.379122376441956
66.7564102564103 0.35991033911705
70.8269230769231 0.361950904130936
75.1474358974359 0.346306294202805
79.7307692307692 0.337672114372253
84.5897435897436 0.323739618062973
89.7435897435897 0.312207967042923
95.2179487179487 0.314184576272964
101.019230769231 0.315516233444214
107.179487179487 0.298435926437378
113.711538461538 0.29811355471611
120.641025641026 0.293935835361481
127.99358974359 0.285696685314178
135.801282051282 0.277951121330261
144.076923076923 0.299682140350342
152.858974358974 0.27744334936142
162.179487179487 0.277605146169662
172.064102564103 0.281594663858414
182.551282051282 0.266644477844238
193.679487179487 0.273602783679962
205.487179487179 0.282504260540009
218.012820512821 0.253155469894409
231.301282051282 0.273877143859863
245.403846153846 0.289760053157806
260.358974358974 0.264607161283493
276.230769230769 0.255048602819443
293.070512820513 0.265161991119385
310.935897435897 0.254896283149719
329.884615384615 0.25929781794548
350 0.24486568570137
};
\addlegendentry{mb 128, exact}
\addplot [, black, opacity=0.6, mark=*, mark size=0.5, mark options={solid}, only marks, forget plot]
table {%
1 0.986895263195038
1.05769230769231 0.980787634849548
1.12179487179487 0.985461950302124
1.19230769230769 0.991671502590179
1.26282051282051 0.995383679866791
1.33974358974359 0.995356857776642
1.42307692307692 0.996258497238159
1.51282051282051 0.989187598228455
1.6025641025641 0.989968836307526
1.69871794871795 0.98010641336441
1.80128205128205 0.971567988395691
1.91666666666667 0.980160355567932
2.03205128205128 0.97682797908783
2.15384615384615 0.969399690628052
2.28846153846154 0.96156370639801
2.42307692307692 0.956043362617493
2.57692307692308 0.947030007839203
2.73076923076923 0.952257513999939
2.8974358974359 0.939471900463104
3.07692307692308 0.936847865581512
3.26282051282051 0.929396569728851
3.46153846153846 0.945724368095398
3.67307692307692 0.927689850330353
3.8974358974359 0.92981094121933
4.13461538461539 0.941705465316772
4.38461538461539 0.906059086322784
4.65384615384615 0.886296153068542
4.93589743589744 0.919317007064819
5.23717948717949 0.909274399280548
5.55769230769231 0.891047596931458
5.8974358974359 0.875378847122192
6.25641025641026 0.877204239368439
6.64102564102564 0.850964188575745
7.04487179487179 0.845690906047821
7.47435897435897 0.847758591175079
7.92948717948718 0.830593228340149
8.41025641025641 0.796346008777618
8.92307692307692 0.801896512508392
9.46794871794872 0.798394322395325
10.0448717948718 0.777367055416107
10.6602564102564 0.756431877613068
11.3076923076923 0.731467723846436
12 0.725441813468933
12.7307692307692 0.712042212486267
13.5064102564103 0.675029873847961
14.3333333333333 0.672647058963776
15.2051282051282 0.675873160362244
16.1346153846154 0.654091775417328
17.1153846153846 0.636921882629395
18.1602564102564 0.62453305721283
19.2692307692308 0.615143001079559
20.4423076923077 0.592768251895905
21.6858974358974 0.555790841579437
23.0128205128205 0.562988936901093
24.4102564102564 0.54248982667923
25.9038461538462 0.530127346515656
27.4807692307692 0.52151745557785
29.1538461538462 0.511283934116364
30.9358974358974 0.472534507513046
32.8205128205128 0.47830519080162
34.8205128205128 0.477901667356491
36.9423076923077 0.464605212211609
39.1923076923077 0.450261682271957
41.5833333333333 0.437666684389114
44.1153846153846 0.421621948480606
46.8076923076923 0.414727747440338
49.6602564102564 0.407199054956436
52.6858974358974 0.392490386962891
55.8974358974359 0.389752984046936
59.3076923076923 0.377608835697174
62.9230769230769 0.37678787112236
66.7564102564103 0.351413398981094
70.8269230769231 0.362528920173645
75.1474358974359 0.336033344268799
79.7307692307692 0.344733536243439
84.5897435897436 0.329869747161865
89.7435897435897 0.307305425405502
95.2179487179487 0.314863651990891
101.019230769231 0.318781226873398
107.179487179487 0.305703341960907
113.711538461538 0.307104140520096
120.641025641026 0.298313707113266
127.99358974359 0.2934889793396
135.801282051282 0.290980607271194
144.076923076923 0.295665860176086
152.858974358974 0.289781093597412
162.179487179487 0.290582776069641
172.064102564103 0.285847723484039
182.551282051282 0.271805763244629
193.679487179487 0.277707159519196
205.487179487179 0.296357899904251
218.012820512821 0.267042249441147
231.301282051282 0.260891109704971
245.403846153846 0.28995606303215
260.358974358974 0.259319484233856
276.230769230769 0.247401759028435
293.070512820513 0.268549770116806
310.935897435897 0.267071485519409
329.884615384615 0.268940776586533
350 0.254630863666534
};
\addplot [, black, opacity=0.6, mark=*, mark size=0.5, mark options={solid}, only marks, forget plot]
table {%
1 0.989810764789581
1.05769230769231 0.988471150398254
1.12179487179487 0.993696570396423
1.19230769230769 0.991568446159363
1.26282051282051 0.989658772945404
1.33974358974359 0.994469583034515
1.42307692307692 0.994873046875
1.51282051282051 0.981999516487122
1.6025641025641 0.97782027721405
1.69871794871795 0.950825691223145
1.80128205128205 0.94676661491394
1.91666666666667 0.966289281845093
2.03205128205128 0.947389423847198
2.15384615384615 0.943173050880432
2.28846153846154 0.939484834671021
2.42307692307692 0.935541212558746
2.57692307692308 0.913724660873413
2.73076923076923 0.912989616394043
2.8974358974359 0.912814259529114
3.07692307692308 0.922054409980774
3.26282051282051 0.883259117603302
3.46153846153846 0.909272313117981
3.67307692307692 0.90310537815094
3.8974358974359 0.898523211479187
4.13461538461539 0.89713728427887
4.38461538461539 0.875198900699615
4.65384615384615 0.8771613240242
4.93589743589744 0.901280164718628
5.23717948717949 0.898040771484375
5.55769230769231 0.871551036834717
5.8974358974359 0.868200063705444
6.25641025641026 0.8705033659935
6.64102564102564 0.855795979499817
7.04487179487179 0.846056878566742
7.47435897435897 0.861203134059906
7.92948717948718 0.82034707069397
8.41025641025641 0.804534256458282
8.92307692307692 0.789363920688629
9.46794871794872 0.812005281448364
10.0448717948718 0.783037722110748
10.6602564102564 0.772508680820465
11.3076923076923 0.730777204036713
12 0.723770558834076
12.7307692307692 0.710988879203796
13.5064102564103 0.699017226696014
14.3333333333333 0.679319143295288
15.2051282051282 0.667685866355896
16.1346153846154 0.64519476890564
17.1153846153846 0.61491471529007
18.1602564102564 0.609516561031342
19.2692307692308 0.609775364398956
20.4423076923077 0.582055509090424
21.6858974358974 0.55249285697937
23.0128205128205 0.555131196975708
24.4102564102564 0.544568479061127
25.9038461538462 0.534587323665619
27.4807692307692 0.520486116409302
29.1538461538462 0.50546795129776
30.9358974358974 0.485412120819092
32.8205128205128 0.4737608730793
34.8205128205128 0.471971601247787
36.9423076923077 0.471086144447327
39.1923076923077 0.451848834753036
41.5833333333333 0.43164137005806
44.1153846153846 0.423379212617874
46.8076923076923 0.42055556178093
49.6602564102564 0.412801057100296
52.6858974358974 0.402280032634735
55.8974358974359 0.381641149520874
59.3076923076923 0.37458735704422
62.9230769230769 0.377476722002029
66.7564102564103 0.365892171859741
70.8269230769231 0.365151822566986
75.1474358974359 0.343942672014236
79.7307692307692 0.364530503749847
84.5897435897436 0.321894973516464
89.7435897435897 0.310723483562469
95.2179487179487 0.308152675628662
101.019230769231 0.310962468385696
107.179487179487 0.319169253110886
113.711538461538 0.307224750518799
120.641025641026 0.313114106655121
127.99358974359 0.298181712627411
135.801282051282 0.288338422775269
144.076923076923 0.310233771800995
152.858974358974 0.280428826808929
162.179487179487 0.291744619607925
172.064102564103 0.282480090856552
182.551282051282 0.251580029726028
193.679487179487 0.268506616353989
205.487179487179 0.27697017788887
218.012820512821 0.276775896549225
231.301282051282 0.266894072294235
245.403846153846 0.281022578477859
260.358974358974 0.266442745923996
276.230769230769 0.251469820737839
293.070512820513 0.253527164459229
310.935897435897 0.275888025760651
329.884615384615 0.266400367021561
350 0.26028373837471
};
\addplot [, black, opacity=0.6, mark=*, mark size=0.5, mark options={solid}, only marks, forget plot]
table {%
1 0.98018878698349
1.05769230769231 0.979297637939453
1.12179487179487 0.990327954292297
1.19230769230769 0.990190386772156
1.26282051282051 0.979087889194489
1.33974358974359 0.987639188766479
1.42307692307692 0.986421942710876
1.51282051282051 0.988199234008789
1.6025641025641 0.988146662712097
1.69871794871795 0.97368973493576
1.80128205128205 0.96921980381012
1.91666666666667 0.97775137424469
2.03205128205128 0.97203665971756
2.15384615384615 0.975655674934387
2.28846153846154 0.960072219371796
2.42307692307692 0.958180367946625
2.57692307692308 0.957045197486877
2.73076923076923 0.946048080921173
2.8974358974359 0.934350490570068
3.07692307692308 0.94107311964035
3.26282051282051 0.917692720890045
3.46153846153846 0.925226330757141
3.67307692307692 0.920620560646057
3.8974358974359 0.908173799514771
4.13461538461539 0.913211286067963
4.38461538461539 0.896114945411682
4.65384615384615 0.913611173629761
4.93589743589744 0.88570636510849
5.23717948717949 0.912522733211517
5.55769230769231 0.887013673782349
5.8974358974359 0.882735788822174
6.25641025641026 0.888155519962311
6.64102564102564 0.858199119567871
7.04487179487179 0.850301206111908
7.47435897435897 0.85232800245285
7.92948717948718 0.837811231613159
8.41025641025641 0.79827094078064
8.92307692307692 0.813076674938202
9.46794871794872 0.812685370445251
10.0448717948718 0.79168975353241
10.6602564102564 0.77166622877121
11.3076923076923 0.748108983039856
12 0.756922662258148
12.7307692307692 0.73523360490799
13.5064102564103 0.697845876216888
14.3333333333333 0.695247530937195
15.2051282051282 0.691827058792114
16.1346153846154 0.659954905509949
17.1153846153846 0.637606084346771
18.1602564102564 0.626903355121613
19.2692307692308 0.611139118671417
20.4423076923077 0.596262276172638
21.6858974358974 0.56735759973526
23.0128205128205 0.55872106552124
24.4102564102564 0.563576519489288
25.9038461538462 0.540385246276855
27.4807692307692 0.532660186290741
29.1538461538462 0.512855768203735
30.9358974358974 0.511415302753448
32.8205128205128 0.486744821071625
34.8205128205128 0.496074587106705
36.9423076923077 0.491412729024887
39.1923076923077 0.46189683675766
41.5833333333333 0.450743168592453
44.1153846153846 0.438691288232803
46.8076923076923 0.430243343114853
49.6602564102564 0.417863875627518
52.6858974358974 0.418734014034271
55.8974358974359 0.399037927389145
59.3076923076923 0.379649102687836
62.9230769230769 0.380256116390228
66.7564102564103 0.369411200284958
70.8269230769231 0.372477054595947
75.1474358974359 0.34940168261528
79.7307692307692 0.362079828977585
84.5897435897436 0.328754127025604
89.7435897435897 0.328092277050018
95.2179487179487 0.321638017892838
101.019230769231 0.32082125544548
107.179487179487 0.309922963380814
113.711538461538 0.304620981216431
120.641025641026 0.319176018238068
127.99358974359 0.30242046713829
135.801282051282 0.299052149057388
144.076923076923 0.317478746175766
152.858974358974 0.296295762062073
162.179487179487 0.292951464653015
172.064102564103 0.294750928878784
182.551282051282 0.270549535751343
193.679487179487 0.285436391830444
205.487179487179 0.295127511024475
218.012820512821 0.267993271350861
231.301282051282 0.279267370700836
245.403846153846 0.282593756914139
260.358974358974 0.265568226575851
276.230769230769 0.264495581388474
293.070512820513 0.27492019534111
310.935897435897 0.279478132724762
329.884615384615 0.304643422365189
350 0.265547186136246
};
\addplot [, black, opacity=0.6, mark=*, mark size=0.5, mark options={solid}, only marks, forget plot]
table {%
1 0.973753869533539
1.05769230769231 0.97322815656662
1.12179487179487 0.983072459697723
1.19230769230769 0.97915130853653
1.26282051282051 0.985690414905548
1.33974358974359 0.989498734474182
1.42307692307692 0.989751994609833
1.51282051282051 0.981911420822144
1.6025641025641 0.973996162414551
1.69871794871795 0.978213787078857
1.80128205128205 0.969274580478668
1.91666666666667 0.967932105064392
2.03205128205128 0.980430424213409
2.15384615384615 0.967014133930206
2.28846153846154 0.95588618516922
2.42307692307692 0.955709993839264
2.57692307692308 0.945317804813385
2.73076923076923 0.935469806194305
2.8974358974359 0.94493842124939
3.07692307692308 0.94318675994873
3.26282051282051 0.935243487358093
3.46153846153846 0.938679456710815
3.67307692307692 0.918027341365814
3.8974358974359 0.92744243144989
4.13461538461539 0.926182985305786
4.38461538461539 0.908518970012665
4.65384615384615 0.908053278923035
4.93589743589744 0.888128399848938
5.23717948717949 0.898898422718048
5.55769230769231 0.886095881462097
5.8974358974359 0.886300325393677
6.25641025641026 0.878016352653503
6.64102564102564 0.866545975208282
7.04487179487179 0.86702436208725
7.47435897435897 0.861173689365387
7.92948717948718 0.839564204216003
8.41025641025641 0.817166268825531
8.92307692307692 0.805991172790527
9.46794871794872 0.818654000759125
10.0448717948718 0.794586002826691
10.6602564102564 0.791329264640808
11.3076923076923 0.748766601085663
12 0.737976968288422
12.7307692307692 0.723856210708618
13.5064102564103 0.698262870311737
14.3333333333333 0.685054183006287
15.2051282051282 0.68641859292984
16.1346153846154 0.652000546455383
17.1153846153846 0.634006202220917
18.1602564102564 0.631703495979309
19.2692307692308 0.605153739452362
20.4423076923077 0.593919992446899
21.6858974358974 0.56074196100235
23.0128205128205 0.553378820419312
24.4102564102564 0.544493138790131
25.9038461538462 0.53988242149353
27.4807692307692 0.524265229701996
29.1538461538462 0.504412651062012
30.9358974358974 0.501520276069641
32.8205128205128 0.491924613714218
34.8205128205128 0.476371258497238
36.9423076923077 0.46665033698082
39.1923076923077 0.446092903614044
41.5833333333333 0.450221657752991
44.1153846153846 0.419289171695709
46.8076923076923 0.415226817131042
49.6602564102564 0.401206821203232
52.6858974358974 0.382014006376266
55.8974358974359 0.394410997629166
59.3076923076923 0.363665223121643
62.9230769230769 0.362125068902969
66.7564102564103 0.36014986038208
70.8269230769231 0.360298007726669
75.1474358974359 0.331577956676483
79.7307692307692 0.344969004392624
84.5897435897436 0.328490436077118
89.7435897435897 0.323612362146378
95.2179487179487 0.318572223186493
101.019230769231 0.317898690700531
107.179487179487 0.303832083940506
113.711538461538 0.30374538898468
120.641025641026 0.301226437091827
127.99358974359 0.288552403450012
135.801282051282 0.307082146406174
144.076923076923 0.305051684379578
152.858974358974 0.285187929868698
162.179487179487 0.278125047683716
172.064102564103 0.299753844738007
182.551282051282 0.267872631549835
193.679487179487 0.263909816741943
205.487179487179 0.293269544839859
218.012820512821 0.261766344308853
231.301282051282 0.263620346784592
245.403846153846 0.277827829122543
260.358974358974 0.268119752407074
276.230769230769 0.245374888181686
293.070512820513 0.247209012508392
310.935897435897 0.281961977481842
329.884615384615 0.276907503604889
350 0.257376223802567
};
\addplot [, color0, opacity=0.6, mark=diamond*, mark size=0.5, mark options={solid}, only marks]
table {%
1 0.875189483165741
1.05769230769231 0.8740394115448
1.12179487179487 0.883608520030975
1.19230769230769 0.891330063343048
1.26282051282051 0.90313047170639
1.33974358974359 0.917201220989227
1.42307692307692 0.922449946403503
1.51282051282051 0.870025217533112
1.6025641025641 0.854385077953339
1.69871794871795 0.830901443958282
1.80128205128205 0.805828154087067
1.91666666666667 0.798629283905029
2.03205128205128 0.793399930000305
2.15384615384615 0.817229211330414
2.28846153846154 0.760928928852081
2.42307692307692 0.718404293060303
2.57692307692308 0.698277473449707
2.73076923076923 0.721617162227631
2.8974358974359 0.710832834243774
3.07692307692308 0.678299069404602
3.26282051282051 0.653535723686218
3.46153846153846 0.685653507709503
3.67307692307692 0.668963849544525
3.8974358974359 0.685673534870148
4.13461538461539 0.64773041009903
4.38461538461539 0.602159440517426
4.65384615384615 0.640514731407166
4.93589743589744 0.675305485725403
5.23717948717949 0.607063353061676
5.55769230769231 0.598080813884735
5.8974358974359 0.601458430290222
6.25641025641026 0.565017104148865
6.64102564102564 0.586391866207123
7.04487179487179 0.582023620605469
7.47435897435897 0.588295757770538
7.92948717948718 0.569146811962128
8.41025641025641 0.532387614250183
8.92307692307692 0.518216133117676
9.46794871794872 0.561469316482544
10.0448717948718 0.526633441448212
10.6602564102564 0.540346562862396
11.3076923076923 0.491753339767456
12 0.482996046543121
12.7307692307692 0.459570288658142
13.5064102564103 0.426471740007401
14.3333333333333 0.47346094250679
15.2051282051282 0.439480125904083
16.1346153846154 0.43012011051178
17.1153846153846 0.42142242193222
18.1602564102564 0.399709641933441
19.2692307692308 0.390595763921738
20.4423076923077 0.384106755256653
21.6858974358974 0.388998776674271
23.0128205128205 0.376472920179367
24.4102564102564 0.357863068580627
25.9038461538462 0.377156138420105
27.4807692307692 0.350927650928497
29.1538461538462 0.338443607091904
30.9358974358974 0.323206633329391
32.8205128205128 0.325027853250504
34.8205128205128 0.322214961051941
36.9423076923077 0.320475578308105
39.1923076923077 0.300307720899582
41.5833333333333 0.291318863630295
44.1153846153846 0.285845816135406
46.8076923076923 0.264387607574463
49.6602564102564 0.264296591281891
52.6858974358974 0.266782939434052
55.8974358974359 0.248620480298996
59.3076923076923 0.242508113384247
62.9230769230769 0.243498221039772
66.7564102564103 0.247922822833061
70.8269230769231 0.244471430778503
75.1474358974359 0.235793635249138
79.7307692307692 0.22210681438446
84.5897435897436 0.215969011187553
89.7435897435897 0.213412091135979
95.2179487179487 0.225021183490753
101.019230769231 0.210704639554024
107.179487179487 0.216441050171852
113.711538461538 0.198670998215675
120.641025641026 0.206334829330444
127.99358974359 0.196934252977371
135.801282051282 0.199281841516495
144.076923076923 0.208159148693085
152.858974358974 0.197201415896416
162.179487179487 0.191014289855957
172.064102564103 0.195009633898735
182.551282051282 0.192917630076408
193.679487179487 0.186529323458672
205.487179487179 0.190160572528839
218.012820512821 0.17552562057972
231.301282051282 0.176354274153709
245.403846153846 0.196731567382812
260.358974358974 0.186116114258766
276.230769230769 0.172914952039719
293.070512820513 0.190101310610771
310.935897435897 0.181567192077637
329.884615384615 0.177476063370705
350 0.161953806877136
};
\addlegendentry{sub 16, exact}
\addplot [, color0, opacity=0.6, mark=diamond*, mark size=0.5, mark options={solid}, only marks, forget plot]
table {%
1 0.927157402038574
1.05769230769231 0.913491725921631
1.12179487179487 0.936347961425781
1.19230769230769 0.937316477298737
1.26282051282051 0.955624222755432
1.33974358974359 0.961221277713776
1.42307692307692 0.960431218147278
1.51282051282051 0.862184822559357
1.6025641025641 0.795210242271423
1.69871794871795 0.759148240089417
1.80128205128205 0.795528531074524
1.91666666666667 0.841739416122437
2.03205128205128 0.828876793384552
2.15384615384615 0.870138049125671
2.28846153846154 0.826793372631073
2.42307692307692 0.845210492610931
2.57692307692308 0.792996346950531
2.73076923076923 0.762033224105835
2.8974358974359 0.787913203239441
3.07692307692308 0.78369015455246
3.26282051282051 0.728266596794128
3.46153846153846 0.7672980427742
3.67307692307692 0.750728487968445
3.8974358974359 0.738189518451691
4.13461538461539 0.744510769844055
4.38461538461539 0.730700075626373
4.65384615384615 0.703212857246399
4.93589743589744 0.726396024227142
5.23717948717949 0.74069619178772
5.55769230769231 0.669283032417297
5.8974358974359 0.665505826473236
6.25641025641026 0.667175352573395
6.64102564102564 0.635149121284485
7.04487179487179 0.610286712646484
7.47435897435897 0.618981838226318
7.92948717948718 0.58574765920639
8.41025641025641 0.545603632926941
8.92307692307692 0.579907715320587
9.46794871794872 0.584928631782532
10.0448717948718 0.547972023487091
10.6602564102564 0.510992288589478
11.3076923076923 0.499764859676361
12 0.471774727106094
12.7307692307692 0.455731123685837
13.5064102564103 0.458067208528519
14.3333333333333 0.441439658403397
15.2051282051282 0.432060241699219
16.1346153846154 0.426171720027924
17.1153846153846 0.41768217086792
18.1602564102564 0.404353469610214
19.2692307692308 0.370139926671982
20.4423076923077 0.398771017789841
21.6858974358974 0.364968091249466
23.0128205128205 0.370050847530365
24.4102564102564 0.365127414464951
25.9038461538462 0.364946097135544
27.4807692307692 0.35405445098877
29.1538461538462 0.318911075592041
30.9358974358974 0.328347623348236
32.8205128205128 0.311085820198059
34.8205128205128 0.314388304948807
36.9423076923077 0.312134385108948
39.1923076923077 0.30512472987175
41.5833333333333 0.301414966583252
44.1153846153846 0.284112423658371
46.8076923076923 0.282346159219742
49.6602564102564 0.267519652843475
52.6858974358974 0.268865972757339
55.8974358974359 0.27155601978302
59.3076923076923 0.256206691265106
62.9230769230769 0.247465580701828
66.7564102564103 0.250571817159653
70.8269230769231 0.260919362306595
75.1474358974359 0.260214596986771
79.7307692307692 0.258260250091553
84.5897435897436 0.253165036439896
89.7435897435897 0.229546919465065
95.2179487179487 0.235294297337532
101.019230769231 0.233752399682999
107.179487179487 0.220280110836029
113.711538461538 0.214908182621002
120.641025641026 0.223695859313011
127.99358974359 0.223214089870453
135.801282051282 0.223116427659988
144.076923076923 0.234466657042503
152.858974358974 0.208353415131569
162.179487179487 0.22870796918869
172.064102564103 0.221314564347267
182.551282051282 0.200341373682022
193.679487179487 0.194317758083344
205.487179487179 0.222353875637054
218.012820512821 0.212038531899452
231.301282051282 0.196982741355896
245.403846153846 0.21039192378521
260.358974358974 0.192075401544571
276.230769230769 0.186554938554764
293.070512820513 0.206832587718964
310.935897435897 0.221240267157555
329.884615384615 0.201918259263039
350 0.181838035583496
};
\addplot [, color0, opacity=0.6, mark=diamond*, mark size=0.5, mark options={solid}, only marks, forget plot]
table {%
1 0.907960653305054
1.05769230769231 0.899477660655975
1.12179487179487 0.921608507633209
1.19230769230769 0.926802217960358
1.26282051282051 0.946551620960236
1.33974358974359 0.946784794330597
1.42307692307692 0.926008224487305
1.51282051282051 0.8985316157341
1.6025641025641 0.886330604553223
1.69871794871795 0.851521134376526
1.80128205128205 0.843134164810181
1.91666666666667 0.84803706407547
2.03205128205128 0.828599631786346
2.15384615384615 0.792546808719635
2.28846153846154 0.817457735538483
2.42307692307692 0.698288381099701
2.57692307692308 0.72036999464035
2.73076923076923 0.739227056503296
2.8974358974359 0.656362354755402
3.07692307692308 0.698777318000793
3.26282051282051 0.69790381193161
3.46153846153846 0.71742045879364
3.67307692307692 0.622631430625916
3.8974358974359 0.679356038570404
4.13461538461539 0.676908791065216
4.38461538461539 0.636787414550781
4.65384615384615 0.641245245933533
4.93589743589744 0.693260788917542
5.23717948717949 0.649294257164001
5.55769230769231 0.618285953998566
5.8974358974359 0.627655208110809
6.25641025641026 0.594342768192291
6.64102564102564 0.568973243236542
7.04487179487179 0.571904540061951
7.47435897435897 0.558385252952576
7.92948717948718 0.550283372402191
8.41025641025641 0.500765800476074
8.92307692307692 0.514584362506866
9.46794871794872 0.543490648269653
10.0448717948718 0.532554864883423
10.6602564102564 0.528592646121979
11.3076923076923 0.497530549764633
12 0.486192762851715
12.7307692307692 0.475432813167572
13.5064102564103 0.444287300109863
14.3333333333333 0.451669424772263
15.2051282051282 0.442490071058273
16.1346153846154 0.415438681840897
17.1153846153846 0.406571418046951
18.1602564102564 0.393296271562576
19.2692307692308 0.379446864128113
20.4423076923077 0.37020280957222
21.6858974358974 0.358220934867859
23.0128205128205 0.351902842521667
24.4102564102564 0.336272120475769
25.9038461538462 0.344561904668808
27.4807692307692 0.339056819677353
29.1538461538462 0.314291894435883
30.9358974358974 0.301766961812973
32.8205128205128 0.300488889217377
34.8205128205128 0.296777963638306
36.9423076923077 0.287078201770782
39.1923076923077 0.292871534824371
41.5833333333333 0.286548763513565
44.1153846153846 0.268847793340683
46.8076923076923 0.267096430063248
49.6602564102564 0.257296562194824
52.6858974358974 0.253237217664719
55.8974358974359 0.254436999559402
59.3076923076923 0.244765624403954
62.9230769230769 0.252775579690933
66.7564102564103 0.245845288038254
70.8269230769231 0.237398564815521
75.1474358974359 0.233931541442871
79.7307692307692 0.228437185287476
84.5897435897436 0.232585564255714
89.7435897435897 0.214326962828636
95.2179487179487 0.229344412684441
101.019230769231 0.208151519298553
107.179487179487 0.2105762809515
113.711538461538 0.222135230898857
120.641025641026 0.225293129682541
127.99358974359 0.200980424880981
135.801282051282 0.218251869082451
144.076923076923 0.211352795362473
152.858974358974 0.206671580672264
162.179487179487 0.19957123696804
172.064102564103 0.213250041007996
182.551282051282 0.19020488858223
193.679487179487 0.190417900681496
205.487179487179 0.196878999471664
218.012820512821 0.187478482723236
231.301282051282 0.180483818054199
245.403846153846 0.208609342575073
260.358974358974 0.20469257235527
276.230769230769 0.178083419799805
293.070512820513 0.175313740968704
310.935897435897 0.196062654256821
329.884615384615 0.195131883025169
350 0.177484929561615
};
\addplot [, color0, opacity=0.6, mark=diamond*, mark size=0.5, mark options={solid}, only marks, forget plot]
table {%
1 0.886715829372406
1.05769230769231 0.873136401176453
1.12179487179487 0.890723645687103
1.19230769230769 0.900638580322266
1.26282051282051 0.918478965759277
1.33974358974359 0.94061541557312
1.42307692307692 0.941164553165436
1.51282051282051 0.899572610855103
1.6025641025641 0.844488561153412
1.69871794871795 0.852157413959503
1.80128205128205 0.846371293067932
1.91666666666667 0.836470305919647
2.03205128205128 0.842281460762024
2.15384615384615 0.838032841682434
2.28846153846154 0.79779839515686
2.42307692307692 0.794649958610535
2.57692307692308 0.774590313434601
2.73076923076923 0.775602340698242
2.8974358974359 0.780587196350098
3.07692307692308 0.748438775539398
3.26282051282051 0.74237859249115
3.46153846153846 0.762369990348816
3.67307692307692 0.733459889888763
3.8974358974359 0.723633229732513
4.13461538461539 0.726911067962646
4.38461538461539 0.68025529384613
4.65384615384615 0.713484048843384
4.93589743589744 0.729710042476654
5.23717948717949 0.715787470340729
5.55769230769231 0.669338822364807
5.8974358974359 0.659125208854675
6.25641025641026 0.652501821517944
6.64102564102564 0.625300049781799
7.04487179487179 0.642605364322662
7.47435897435897 0.63705313205719
7.92948717948718 0.609722554683685
8.41025641025641 0.553276240825653
8.92307692307692 0.5812166929245
9.46794871794872 0.563161253929138
10.0448717948718 0.547271728515625
10.6602564102564 0.543911933898926
11.3076923076923 0.487612366676331
12 0.493218690156937
12.7307692307692 0.46988445520401
13.5064102564103 0.4525026679039
14.3333333333333 0.460589587688446
15.2051282051282 0.438872098922729
16.1346153846154 0.413459450006485
17.1153846153846 0.41434845328331
18.1602564102564 0.401778161525726
19.2692307692308 0.387885421514511
20.4423076923077 0.386802464723587
21.6858974358974 0.370950222015381
23.0128205128205 0.361233860254288
24.4102564102564 0.341898798942566
25.9038461538462 0.337038934230804
27.4807692307692 0.317129045724869
29.1538461538462 0.319417536258698
30.9358974358974 0.303591936826706
32.8205128205128 0.302766144275665
34.8205128205128 0.291976243257523
36.9423076923077 0.295648962259293
39.1923076923077 0.286425054073334
41.5833333333333 0.265327453613281
44.1153846153846 0.265426516532898
46.8076923076923 0.259623438119888
49.6602564102564 0.248611062765121
52.6858974358974 0.244203448295593
55.8974358974359 0.24769851565361
59.3076923076923 0.224947199225426
62.9230769230769 0.235183328390121
66.7564102564103 0.22653865814209
70.8269230769231 0.229149162769318
75.1474358974359 0.22237491607666
79.7307692307692 0.211294829845428
84.5897435897436 0.200461566448212
89.7435897435897 0.197572439908981
95.2179487179487 0.20318940281868
101.019230769231 0.196598246693611
107.179487179487 0.195848762989044
113.711538461538 0.19074435532093
120.641025641026 0.195224702358246
127.99358974359 0.190983772277832
135.801282051282 0.183329001069069
144.076923076923 0.18398979306221
152.858974358974 0.17451137304306
162.179487179487 0.18167182803154
172.064102564103 0.174668550491333
182.551282051282 0.169318199157715
193.679487179487 0.17231111228466
205.487179487179 0.177924394607544
218.012820512821 0.177364930510521
231.301282051282 0.168255254626274
245.403846153846 0.176309391856194
260.358974358974 0.159644454717636
276.230769230769 0.159723922610283
293.070512820513 0.183653563261032
310.935897435897 0.162593230605125
329.884615384615 0.166135415434837
350 0.154997020959854
};
\addplot [, color0, opacity=0.6, mark=diamond*, mark size=0.5, mark options={solid}, only marks, forget plot]
table {%
1 0.920768201351166
1.05769230769231 0.922607600688934
1.12179487179487 0.939803123474121
1.19230769230769 0.941757023334503
1.26282051282051 0.958697617053986
1.33974358974359 0.968611419200897
1.42307692307692 0.963573515415192
1.51282051282051 0.904239475727081
1.6025641025641 0.85851263999939
1.69871794871795 0.854909479618073
1.80128205128205 0.823170900344849
1.91666666666667 0.86471563577652
2.03205128205128 0.858860075473785
2.15384615384615 0.883299231529236
2.28846153846154 0.850973427295685
2.42307692307692 0.84367847442627
2.57692307692308 0.802963256835938
2.73076923076923 0.761950075626373
2.8974358974359 0.77672803401947
3.07692307692308 0.783218085765839
3.26282051282051 0.745612978935242
3.46153846153846 0.754541993141174
3.67307692307692 0.751965343952179
3.8974358974359 0.712158262729645
4.13461538461539 0.733751833438873
4.38461538461539 0.725594937801361
4.65384615384615 0.742225766181946
4.93589743589744 0.730198800563812
5.23717948717949 0.727936863899231
5.55769230769231 0.687447667121887
5.8974358974359 0.693859219551086
6.25641025641026 0.686065971851349
6.64102564102564 0.641756892204285
7.04487179487179 0.629996478557587
7.47435897435897 0.633586645126343
7.92948717948718 0.586866736412048
8.41025641025641 0.529873430728912
8.92307692307692 0.533731460571289
9.46794871794872 0.557930767536163
10.0448717948718 0.522641360759735
10.6602564102564 0.506107330322266
11.3076923076923 0.473597824573517
12 0.483898460865021
12.7307692307692 0.460118025541306
13.5064102564103 0.425599277019501
14.3333333333333 0.429172575473785
15.2051282051282 0.419709831476212
16.1346153846154 0.405384123325348
17.1153846153846 0.391377627849579
18.1602564102564 0.404565423727036
19.2692307692308 0.377562940120697
20.4423076923077 0.379576176404953
21.6858974358974 0.358943313360214
23.0128205128205 0.362985134124756
24.4102564102564 0.34936660528183
25.9038461538462 0.350693345069885
27.4807692307692 0.324539303779602
29.1538461538462 0.312141209840775
30.9358974358974 0.312078595161438
32.8205128205128 0.305971145629883
34.8205128205128 0.309769839048386
36.9423076923077 0.307235270738602
39.1923076923077 0.282705008983612
41.5833333333333 0.28606328368187
44.1153846153846 0.275733053684235
46.8076923076923 0.273362219333649
49.6602564102564 0.266403883695602
52.6858974358974 0.26519176363945
55.8974358974359 0.25615867972374
59.3076923076923 0.236261576414108
62.9230769230769 0.244299620389938
66.7564102564103 0.234643548727036
70.8269230769231 0.234910413622856
75.1474358974359 0.237130239605904
79.7307692307692 0.243150979280472
84.5897435897436 0.212984174489975
89.7435897435897 0.205339461565018
95.2179487179487 0.205185905098915
101.019230769231 0.212897941470146
107.179487179487 0.231069713830948
113.711538461538 0.208236575126648
120.641025641026 0.214450910687447
127.99358974359 0.198757112026215
135.801282051282 0.196229353547096
144.076923076923 0.213941648602486
152.858974358974 0.211059734225273
162.179487179487 0.192705765366554
172.064102564103 0.205281212925911
182.551282051282 0.187998190522194
193.679487179487 0.187856763601303
205.487179487179 0.198505416512489
218.012820512821 0.193879425525665
231.301282051282 0.18184794485569
245.403846153846 0.197203084826469
260.358974358974 0.194577783346176
276.230769230769 0.164607003331184
293.070512820513 0.190710142254829
310.935897435897 0.197510749101639
329.884615384615 0.18644605576992
350 0.166118174791336
};
\addplot [, color1, opacity=0.6, mark=square*, mark size=0.5, mark options={solid}, only marks]
table {%
1 0.919934391975403
1.05769230769231 0.917576909065247
1.12179487179487 0.928869307041168
1.19230769230769 0.925379633903503
1.26282051282051 0.927200019359589
1.33974358974359 0.931416988372803
1.42307692307692 0.920888185501099
1.51282051282051 0.885994851589203
1.6025641025641 0.84444534778595
1.69871794871795 0.846072316169739
1.80128205128205 0.842436492443085
1.91666666666667 0.835174560546875
2.03205128205128 0.835184931755066
2.15384615384615 0.841755568981171
2.28846153846154 0.847946763038635
2.42307692307692 0.807844400405884
2.57692307692308 0.81243622303009
2.73076923076923 0.819994032382965
2.8974358974359 0.821331143379211
3.07692307692308 0.815215289592743
3.26282051282051 0.799257397651672
3.46153846153846 0.81323766708374
3.67307692307692 0.785248219966888
3.8974358974359 0.786533117294312
4.13461538461539 0.821253955364227
4.38461538461539 0.795643746852875
4.65384615384615 0.794733703136444
4.93589743589744 0.827913701534271
5.23717948717949 0.806143045425415
5.55769230769231 0.786491096019745
5.8974358974359 0.772709965705872
6.25641025641026 0.757712185382843
6.64102564102564 0.743919253349304
7.04487179487179 0.749263048171997
7.47435897435897 0.754460453987122
7.92948717948718 0.741026699542999
8.41025641025641 0.711199343204498
8.92307692307692 0.716759979724884
9.46794871794872 0.718541860580444
10.0448717948718 0.680694162845612
10.6602564102564 0.686083912849426
11.3076923076923 0.646480858325958
12 0.6402188539505
12.7307692307692 0.654795050621033
13.5064102564103 0.616154968738556
14.3333333333333 0.628556668758392
15.2051282051282 0.604594886302948
16.1346153846154 0.570105373859406
17.1153846153846 0.555931627750397
18.1602564102564 0.553333699703217
19.2692307692308 0.55579000711441
20.4423076923077 0.536518394947052
21.6858974358974 0.523126661777496
23.0128205128205 0.523099184036255
24.4102564102564 0.493800520896912
25.9038461538462 0.492122054100037
27.4807692307692 0.481238722801208
29.1538461538462 0.472695797681808
30.9358974358974 0.440902650356293
32.8205128205128 0.438124239444733
34.8205128205128 0.427687913179398
36.9423076923077 0.422461092472076
39.1923076923077 0.408383399248123
41.5833333333333 0.388841778039932
44.1153846153846 0.383251368999481
46.8076923076923 0.384861201047897
49.6602564102564 0.36222442984581
52.6858974358974 0.360994726419449
55.8974358974359 0.335586071014404
59.3076923076923 0.343161076307297
62.9230769230769 0.349091440439224
66.7564102564103 0.319235742092133
70.8269230769231 0.334862172603607
75.1474358974359 0.314811199903488
79.7307692307692 0.311720162630081
84.5897435897436 0.291425615549088
89.7435897435897 0.280010998249054
95.2179487179487 0.286818385124207
101.019230769231 0.288742631673813
107.179487179487 0.283499896526337
113.711538461538 0.272032648324966
120.641025641026 0.276150584220886
127.99358974359 0.263138025999069
135.801282051282 0.270960599184036
144.076923076923 0.261260092258453
152.858974358974 0.239454448223114
162.179487179487 0.246945410966873
172.064102564103 0.256189674139023
182.551282051282 0.241357460618019
193.679487179487 0.242683500051498
205.487179487179 0.254702895879745
218.012820512821 0.222556561231613
231.301282051282 0.245189890265465
245.403846153846 0.25360631942749
260.358974358974 0.238104775547981
276.230769230769 0.223765105009079
293.070512820513 0.236946076154709
310.935897435897 0.243561580777168
329.884615384615 0.24982276558876
350 0.238862991333008
};
\addlegendentry{mb 128, mc 10}
\addplot [, color1, opacity=0.6, mark=square*, mark size=0.5, mark options={solid}, only marks, forget plot]
table {%
1 0.918156087398529
1.05769230769231 0.914106667041779
1.12179487179487 0.922685980796814
1.19230769230769 0.918078422546387
1.26282051282051 0.928219377994537
1.33974358974359 0.915538311004639
1.42307692307692 0.924085974693298
1.51282051282051 0.892690420150757
1.6025641025641 0.849537253379822
1.69871794871795 0.852365732192993
1.80128205128205 0.845792233943939
1.91666666666667 0.842653810977936
2.03205128205128 0.835087537765503
2.15384615384615 0.844984710216522
2.28846153846154 0.836247682571411
2.42307692307692 0.828506290912628
2.57692307692308 0.829505920410156
2.73076923076923 0.805950045585632
2.8974358974359 0.795123517513275
3.07692307692308 0.816497921943665
3.26282051282051 0.794873178005219
3.46153846153846 0.814696788787842
3.67307692307692 0.803788721561432
3.8974358974359 0.775910913944244
4.13461538461539 0.795919954776764
4.38461538461539 0.783231496810913
4.65384615384615 0.782178461551666
4.93589743589744 0.77957022190094
5.23717948717949 0.785630643367767
5.55769230769231 0.75599193572998
5.8974358974359 0.750757098197937
6.25641025641026 0.749039590358734
6.64102564102564 0.745561182498932
7.04487179487179 0.726299285888672
7.47435897435897 0.735160291194916
7.92948717948718 0.720340430736542
8.41025641025641 0.680202484130859
8.92307692307692 0.693466007709503
9.46794871794872 0.69743537902832
10.0448717948718 0.673444032669067
10.6602564102564 0.671055614948273
11.3076923076923 0.636883437633514
12 0.623424530029297
12.7307692307692 0.629108846187592
13.5064102564103 0.591179668903351
14.3333333333333 0.602578818798065
15.2051282051282 0.585139095783234
16.1346153846154 0.569035768508911
17.1153846153846 0.548498749732971
18.1602564102564 0.546054303646088
19.2692307692308 0.533577263355255
20.4423076923077 0.503501892089844
21.6858974358974 0.491269171237946
23.0128205128205 0.491815328598022
24.4102564102564 0.463805913925171
25.9038461538462 0.456184864044189
27.4807692307692 0.462245464324951
29.1538461538462 0.433746188879013
30.9358974358974 0.427037507295609
32.8205128205128 0.404274314641953
34.8205128205128 0.417607098817825
36.9423076923077 0.397934943437576
39.1923076923077 0.4102982878685
41.5833333333333 0.386406779289246
44.1153846153846 0.373251080513
46.8076923076923 0.370953857898712
49.6602564102564 0.35742324590683
52.6858974358974 0.352539122104645
55.8974358974359 0.335681229829788
59.3076923076923 0.323281228542328
62.9230769230769 0.341340243816376
66.7564102564103 0.317993193864822
70.8269230769231 0.312428206205368
75.1474358974359 0.28869953751564
79.7307692307692 0.297276437282562
84.5897435897436 0.292517304420471
89.7435897435897 0.27042743563652
95.2179487179487 0.283863633871078
101.019230769231 0.291310846805573
107.179487179487 0.267091602087021
113.711538461538 0.268317699432373
120.641025641026 0.254977256059647
127.99358974359 0.269574135541916
135.801282051282 0.28043058514595
144.076923076923 0.280195862054825
152.858974358974 0.26008802652359
162.179487179487 0.25755450129509
172.064102564103 0.255536943674088
182.551282051282 0.238462254405022
193.679487179487 0.233800783753395
205.487179487179 0.248536556959152
218.012820512821 0.237881198525429
231.301282051282 0.231696307659149
245.403846153846 0.258141666650772
260.358974358974 0.235410019755363
276.230769230769 0.235418274998665
293.070512820513 0.237288847565651
310.935897435897 0.245346680283546
329.884615384615 0.2623450756073
350 0.229948058724403
};
\addplot [, color1, opacity=0.6, mark=square*, mark size=0.5, mark options={solid}, only marks, forget plot]
table {%
1 0.917679131031036
1.05769230769231 0.907364189624786
1.12179487179487 0.913465857505798
1.19230769230769 0.920037448406219
1.26282051282051 0.916538834571838
1.33974358974359 0.918070435523987
1.42307692307692 0.925293564796448
1.51282051282051 0.892566025257111
1.6025641025641 0.878292798995972
1.69871794871795 0.835309147834778
1.80128205128205 0.829295337200165
1.91666666666667 0.843868136405945
2.03205128205128 0.819988667964935
2.15384615384615 0.838625371456146
2.28846153846154 0.817925691604614
2.42307692307692 0.799832165241241
2.57692307692308 0.811834394931793
2.73076923076923 0.793222963809967
2.8974358974359 0.811550438404083
3.07692307692308 0.819061160087585
3.26282051282051 0.794136643409729
3.46153846153846 0.837053656578064
3.67307692307692 0.81229293346405
3.8974358974359 0.805356442928314
4.13461538461539 0.804150998592377
4.38461538461539 0.802068948745728
4.65384615384615 0.793264627456665
4.93589743589744 0.803297102451324
5.23717948717949 0.781928718090057
5.55769230769231 0.768967926502228
5.8974358974359 0.757806062698364
6.25641025641026 0.777746081352234
6.64102564102564 0.754313170909882
7.04487179487179 0.738031625747681
7.47435897435897 0.739158689975739
7.92948717948718 0.726167976856232
8.41025641025641 0.692220985889435
8.92307692307692 0.705377280712128
9.46794871794872 0.716328501701355
10.0448717948718 0.685757875442505
10.6602564102564 0.686263620853424
11.3076923076923 0.643999636173248
12 0.631025969982147
12.7307692307692 0.632628262042999
13.5064102564103 0.61070328950882
14.3333333333333 0.600338041782379
15.2051282051282 0.592919707298279
16.1346153846154 0.562014997005463
17.1153846153846 0.531083643436432
18.1602564102564 0.557679116725922
19.2692307692308 0.526433944702148
20.4423076923077 0.506186187267303
21.6858974358974 0.491547465324402
23.0128205128205 0.486137241125107
24.4102564102564 0.475768893957138
25.9038461538462 0.461359620094299
27.4807692307692 0.457593232393265
29.1538461538462 0.431699812412262
30.9358974358974 0.421564370393753
32.8205128205128 0.409952998161316
34.8205128205128 0.400724589824677
36.9423076923077 0.392070144414902
39.1923076923077 0.381137847900391
41.5833333333333 0.361247926950455
44.1153846153846 0.36671194434166
46.8076923076923 0.349532544612885
49.6602564102564 0.344471365213394
52.6858974358974 0.348674237728119
55.8974358974359 0.327865481376648
59.3076923076923 0.325096607208252
62.9230769230769 0.329475998878479
66.7564102564103 0.319259107112885
70.8269230769231 0.312655985355377
75.1474358974359 0.289460241794586
79.7307692307692 0.308453112840652
84.5897435897436 0.275089472532272
89.7435897435897 0.26858589053154
95.2179487179487 0.278079152107239
101.019230769231 0.272106766700745
107.179487179487 0.279338002204895
113.711538461538 0.256693720817566
120.641025641026 0.267661809921265
127.99358974359 0.255953669548035
135.801282051282 0.247966602444649
144.076923076923 0.269341200590134
152.858974358974 0.253644853830338
162.179487179487 0.239846915006638
172.064102564103 0.252344816923141
182.551282051282 0.240830183029175
193.679487179487 0.224154844880104
205.487179487179 0.235819235444069
218.012820512821 0.22089059650898
231.301282051282 0.226144060492516
245.403846153846 0.249578848481178
260.358974358974 0.234048381447792
276.230769230769 0.221018821001053
293.070512820513 0.229782521724701
310.935897435897 0.23613540828228
329.884615384615 0.238630905747414
350 0.212963759899139
};
\addplot [, color1, opacity=0.6, mark=square*, mark size=0.5, mark options={solid}, only marks, forget plot]
table {%
1 0.898513495922089
1.05769230769231 0.891846835613251
1.12179487179487 0.905149519443512
1.19230769230769 0.902307569980621
1.26282051282051 0.900908052921295
1.33974358974359 0.919408142566681
1.42307692307692 0.910706460475922
1.51282051282051 0.884168207645416
1.6025641025641 0.845344066619873
1.69871794871795 0.856781125068665
1.80128205128205 0.837858438491821
1.91666666666667 0.821271359920502
2.03205128205128 0.8269984126091
2.15384615384615 0.833570837974548
2.28846153846154 0.830426335334778
2.42307692307692 0.804027080535889
2.57692307692308 0.81917142868042
2.73076923076923 0.771492898464203
2.8974358974359 0.812482416629791
3.07692307692308 0.800785601139069
3.26282051282051 0.808541595935822
3.46153846153846 0.81196928024292
3.67307692307692 0.817614138126373
3.8974358974359 0.778629720211029
4.13461538461539 0.796374797821045
4.38461538461539 0.785302400588989
4.65384615384615 0.784361839294434
4.93589743589744 0.782527685165405
5.23717948717949 0.763091087341309
5.55769230769231 0.750265657901764
5.8974358974359 0.730601131916046
6.25641025641026 0.752892434597015
6.64102564102564 0.751801788806915
7.04487179487179 0.717436730861664
7.47435897435897 0.739365816116333
7.92948717948718 0.725097954273224
8.41025641025641 0.705753207206726
8.92307692307692 0.670825481414795
9.46794871794872 0.692617833614349
10.0448717948718 0.684685111045837
10.6602564102564 0.685184001922607
11.3076923076923 0.65087753534317
12 0.648466229438782
12.7307692307692 0.623319625854492
13.5064102564103 0.603066861629486
14.3333333333333 0.606551229953766
15.2051282051282 0.591168224811554
16.1346153846154 0.568820118904114
17.1153846153846 0.548101782798767
18.1602564102564 0.546739161014557
19.2692307692308 0.536815941333771
20.4423076923077 0.528983950614929
21.6858974358974 0.496153920888901
23.0128205128205 0.495914101600647
24.4102564102564 0.490186274051666
25.9038461538462 0.464105278253555
27.4807692307692 0.468995034694672
29.1538461538462 0.444546192884445
30.9358974358974 0.433934926986694
32.8205128205128 0.433704018592834
34.8205128205128 0.426890939474106
36.9423076923077 0.422900319099426
39.1923076923077 0.39982008934021
41.5833333333333 0.404714345932007
44.1153846153846 0.384748995304108
46.8076923076923 0.388424813747406
49.6602564102564 0.357273042201996
52.6858974358974 0.356424808502197
55.8974358974359 0.335248410701752
59.3076923076923 0.334797203540802
62.9230769230769 0.354834020137787
66.7564102564103 0.324988484382629
70.8269230769231 0.316874980926514
75.1474358974359 0.326118797063828
79.7307692307692 0.302791088819504
84.5897435897436 0.295210063457489
89.7435897435897 0.295911997556686
95.2179487179487 0.29665020108223
101.019230769231 0.27708637714386
107.179487179487 0.286052912473679
113.711538461538 0.265448898077011
120.641025641026 0.270707249641418
127.99358974359 0.258154898881912
135.801282051282 0.26082119345665
144.076923076923 0.273348689079285
152.858974358974 0.259836047887802
162.179487179487 0.256753385066986
172.064102564103 0.252526223659515
182.551282051282 0.254011750221252
193.679487179487 0.252078801393509
205.487179487179 0.258284360170364
218.012820512821 0.229505434632301
231.301282051282 0.234797433018684
245.403846153846 0.253953725099564
260.358974358974 0.261161983013153
276.230769230769 0.216769650578499
293.070512820513 0.251977980136871
310.935897435897 0.234314113855362
329.884615384615 0.2544125020504
350 0.223572179675102
};
\addplot [, color1, opacity=0.6, mark=square*, mark size=0.5, mark options={solid}, only marks, forget plot]
table {%
1 0.913337051868439
1.05769230769231 0.904207229614258
1.12179487179487 0.91230708360672
1.19230769230769 0.907361507415771
1.26282051282051 0.914642751216888
1.33974358974359 0.914365947246552
1.42307692307692 0.922064542770386
1.51282051282051 0.88023716211319
1.6025641025641 0.851478695869446
1.69871794871795 0.838766157627106
1.80128205128205 0.836102247238159
1.91666666666667 0.831934928894043
2.03205128205128 0.852659285068512
2.15384615384615 0.855472147464752
2.28846153846154 0.847407519817352
2.42307692307692 0.804819703102112
2.57692307692308 0.817116975784302
2.73076923076923 0.792172968387604
2.8974358974359 0.849844396114349
3.07692307692308 0.83525139093399
3.26282051282051 0.811218976974487
3.46153846153846 0.845361948013306
3.67307692307692 0.808929741382599
3.8974358974359 0.803146481513977
4.13461538461539 0.819624006748199
4.38461538461539 0.796879589557648
4.65384615384615 0.780103921890259
4.93589743589744 0.799456536769867
5.23717948717949 0.788799703121185
5.55769230769231 0.784285664558411
5.8974358974359 0.775838315486908
6.25641025641026 0.763877868652344
6.64102564102564 0.75449138879776
7.04487179487179 0.754133522510529
7.47435897435897 0.753728687763214
7.92948717948718 0.747219324111938
8.41025641025641 0.708585262298584
8.92307692307692 0.7040194272995
9.46794871794872 0.717277050018311
10.0448717948718 0.692801177501678
10.6602564102564 0.676034688949585
11.3076923076923 0.641839563846588
12 0.636106371879578
12.7307692307692 0.626689732074738
13.5064102564103 0.620883464813232
14.3333333333333 0.594672977924347
15.2051282051282 0.593406498432159
16.1346153846154 0.573159396648407
17.1153846153846 0.545417070388794
18.1602564102564 0.552330911159515
19.2692307692308 0.527698934078217
20.4423076923077 0.513913810253143
21.6858974358974 0.486647963523865
23.0128205128205 0.49402990937233
24.4102564102564 0.488130420446396
25.9038461538462 0.463467001914978
27.4807692307692 0.463670134544373
29.1538461538462 0.44649663567543
30.9358974358974 0.440016478300095
32.8205128205128 0.420943886041641
34.8205128205128 0.436612129211426
36.9423076923077 0.423575133085251
39.1923076923077 0.398955911397934
41.5833333333333 0.41399610042572
44.1153846153846 0.391564697027206
46.8076923076923 0.370131224393845
49.6602564102564 0.373759716749191
52.6858974358974 0.371295243501663
55.8974358974359 0.356450378894806
59.3076923076923 0.334638983011246
62.9230769230769 0.347963511943817
66.7564102564103 0.33438429236412
70.8269230769231 0.344747602939606
75.1474358974359 0.309907078742981
79.7307692307692 0.325311303138733
84.5897435897436 0.293626964092255
89.7435897435897 0.292491942644119
95.2179487179487 0.297810256481171
101.019230769231 0.290963649749756
107.179487179487 0.2832872569561
113.711538461538 0.287733197212219
120.641025641026 0.290511041879654
127.99358974359 0.279544144868851
135.801282051282 0.279766052961349
144.076923076923 0.292543947696686
152.858974358974 0.266946822404861
162.179487179487 0.282336711883545
172.064102564103 0.263512521982193
182.551282051282 0.245828464627266
193.679487179487 0.251392960548401
205.487179487179 0.271721810102463
218.012820512821 0.246079668402672
231.301282051282 0.247929781675339
245.403846153846 0.276833474636078
260.358974358974 0.254432439804077
276.230769230769 0.240465208888054
293.070512820513 0.2447290122509
310.935897435897 0.256819844245911
329.884615384615 0.244172275066376
350 0.229433298110962
};
\addplot [, color2, opacity=0.6, mark=triangle*, mark size=0.5, mark options={solid,rotate=180}, only marks]
table {%
1 0.695104598999023
1.05769230769231 0.671399474143982
1.12179487179487 0.65598326921463
1.19230769230769 0.639806568622589
1.26282051282051 0.633371472358704
1.33974358974359 0.62629234790802
1.42307692307692 0.588725626468658
1.51282051282051 0.604473531246185
1.6025641025641 0.56170791387558
1.69871794871795 0.573318600654602
1.80128205128205 0.570551574230194
1.91666666666667 0.558525025844574
2.03205128205128 0.544627130031586
2.15384615384615 0.584422528743744
2.28846153846154 0.531260251998901
2.42307692307692 0.517985880374908
2.57692307692308 0.526535391807556
2.73076923076923 0.565951883792877
2.8974358974359 0.537366926670074
3.07692307692308 0.544354438781738
3.26282051282051 0.495747089385986
3.46153846153846 0.515186131000519
3.67307692307692 0.512856662273407
3.8974358974359 0.53894966840744
4.13461538461539 0.495803505182266
4.38461538461539 0.509238183498383
4.65384615384615 0.514962434768677
4.93589743589744 0.538620829582214
5.23717948717949 0.517927706241608
5.55769230769231 0.48777762055397
5.8974358974359 0.484768599271774
6.25641025641026 0.507547080516815
6.64102564102564 0.478386282920837
7.04487179487179 0.456555515527725
7.47435897435897 0.462804317474365
7.92948717948718 0.453769445419312
8.41025641025641 0.440952301025391
8.92307692307692 0.428733587265015
9.46794871794872 0.431283831596375
10.0448717948718 0.420501321554184
10.6602564102564 0.399262607097626
11.3076923076923 0.395660698413849
12 0.365399897098541
12.7307692307692 0.406874150037766
13.5064102564103 0.357180714607239
14.3333333333333 0.354351222515106
15.2051282051282 0.356085151433945
16.1346153846154 0.304543048143387
17.1153846153846 0.337512582540512
18.1602564102564 0.360034674406052
19.2692307692308 0.332597643136978
20.4423076923077 0.339191824197769
21.6858974358974 0.359243988990784
23.0128205128205 0.313125908374786
24.4102564102564 0.301694631576538
25.9038461538462 0.326949387788773
27.4807692307692 0.307119637727737
29.1538461538462 0.289833337068558
30.9358974358974 0.294480323791504
32.8205128205128 0.290850311517715
34.8205128205128 0.296463966369629
36.9423076923077 0.293320953845978
39.1923076923077 0.266292363405228
41.5833333333333 0.275690525770187
44.1153846153846 0.266473561525345
46.8076923076923 0.28559273481369
49.6602564102564 0.24273493885994
52.6858974358974 0.250895500183105
55.8974358974359 0.247608363628387
59.3076923076923 0.24154494702816
62.9230769230769 0.229303374886513
66.7564102564103 0.23811636865139
70.8269230769231 0.236089363694191
75.1474358974359 0.240311443805695
79.7307692307692 0.249126434326172
84.5897435897436 0.216810628771782
89.7435897435897 0.219897374510765
95.2179487179487 0.236613348126411
101.019230769231 0.221875220537186
107.179487179487 0.213050037622452
113.711538461538 0.2028848528862
120.641025641026 0.238238751888275
127.99358974359 0.222929790616035
135.801282051282 0.215837761759758
144.076923076923 0.182321578264236
152.858974358974 0.200850799679756
162.179487179487 0.205385610461235
172.064102564103 0.208077102899551
182.551282051282 0.180099710822105
193.679487179487 0.225108966231346
205.487179487179 0.203237116336823
218.012820512821 0.195582076907158
231.301282051282 0.183931455016136
245.403846153846 0.182815134525299
260.358974358974 0.201471552252769
276.230769230769 0.170618936419487
293.070512820513 0.21938981115818
310.935897435897 0.169149324297905
329.884615384615 0.173979565501213
350 0.160088390111923
};
\addlegendentry{sub 16, mc 10}
\addplot [, color2, opacity=0.6, mark=triangle*, mark size=0.5, mark options={solid,rotate=180}, only marks, forget plot]
table {%
1 0.709465146064758
1.05769230769231 0.65797221660614
1.12179487179487 0.649605929851532
1.19230769230769 0.644875943660736
1.26282051282051 0.618848085403442
1.33974358974359 0.644179046154022
1.42307692307692 0.64870023727417
1.51282051282051 0.598121047019958
1.6025641025641 0.605211675167084
1.69871794871795 0.584377110004425
1.80128205128205 0.566447854042053
1.91666666666667 0.59898167848587
2.03205128205128 0.606732249259949
2.15384615384615 0.583361327648163
2.28846153846154 0.571518898010254
2.42307692307692 0.568749070167542
2.57692307692308 0.58251678943634
2.73076923076923 0.511624038219452
2.8974358974359 0.526629567146301
3.07692307692308 0.537382245063782
3.26282051282051 0.516832292079926
3.46153846153846 0.50773411989212
3.67307692307692 0.514347434043884
3.8974358974359 0.464302062988281
4.13461538461539 0.484553754329681
4.38461538461539 0.475434720516205
4.65384615384615 0.45551735162735
4.93589743589744 0.483919322490692
5.23717948717949 0.482275605201721
5.55769230769231 0.448161453008652
5.8974358974359 0.446744710206985
6.25641025641026 0.479088991880417
6.64102564102564 0.438984453678131
7.04487179487179 0.458349823951721
7.47435897435897 0.422762662172318
7.92948717948718 0.420683979988098
8.41025641025641 0.401850283145905
8.92307692307692 0.403592675924301
9.46794871794872 0.402406752109528
10.0448717948718 0.396066725254059
10.6602564102564 0.399795293807983
11.3076923076923 0.374495148658752
12 0.348899781703949
12.7307692307692 0.376204818487167
13.5064102564103 0.337981700897217
14.3333333333333 0.337478429079056
15.2051282051282 0.337336659431458
16.1346153846154 0.340938955545425
17.1153846153846 0.356627255678177
18.1602564102564 0.347324788570404
19.2692307692308 0.326910138130188
20.4423076923077 0.325686872005463
21.6858974358974 0.317219376564026
23.0128205128205 0.325748652219772
24.4102564102564 0.29592952132225
25.9038461538462 0.292153000831604
27.4807692307692 0.299564152956009
29.1538461538462 0.276195496320724
30.9358974358974 0.283918589353561
32.8205128205128 0.308986127376556
34.8205128205128 0.276111304759979
36.9423076923077 0.272115975618362
39.1923076923077 0.31098484992981
41.5833333333333 0.266177713871002
44.1153846153846 0.251867055892944
46.8076923076923 0.269552052021027
49.6602564102564 0.271015465259552
52.6858974358974 0.263023555278778
55.8974358974359 0.252819180488586
59.3076923076923 0.251890599727631
62.9230769230769 0.249323204159737
66.7564102564103 0.276172697544098
70.8269230769231 0.235407754778862
75.1474358974359 0.259342163801193
79.7307692307692 0.225234746932983
84.5897435897436 0.23322482407093
89.7435897435897 0.2171391248703
95.2179487179487 0.219441175460815
101.019230769231 0.229918092489243
107.179487179487 0.220792084932327
113.711538461538 0.211908832192421
120.641025641026 0.206357374787331
127.99358974359 0.174309462308884
135.801282051282 0.212006375193596
144.076923076923 0.221162483096123
152.858974358974 0.183653101325035
162.179487179487 0.224471196532249
172.064102564103 0.242487043142319
182.551282051282 0.202970385551453
193.679487179487 0.198559865355492
205.487179487179 0.231784448027611
218.012820512821 0.212175473570824
231.301282051282 0.211002960801125
245.403846153846 0.231372311711311
260.358974358974 0.214177146553993
276.230769230769 0.204365804791451
293.070512820513 0.21369606256485
310.935897435897 0.21522319316864
329.884615384615 0.246545135974884
350 0.228799283504486
};
\addplot [, color2, opacity=0.6, mark=triangle*, mark size=0.5, mark options={solid,rotate=180}, only marks, forget plot]
table {%
1 0.710859060287476
1.05769230769231 0.68465793132782
1.12179487179487 0.645713806152344
1.19230769230769 0.664057910442352
1.26282051282051 0.640120923519135
1.33974358974359 0.65608686208725
1.42307692307692 0.635411262512207
1.51282051282051 0.593621134757996
1.6025641025641 0.579883992671967
1.69871794871795 0.597195446491241
1.80128205128205 0.586952030658722
1.91666666666667 0.588058888912201
2.03205128205128 0.611877679824829
2.15384615384615 0.595243871212006
2.28846153846154 0.538580596446991
2.42307692307692 0.579856932163239
2.57692307692308 0.549726843833923
2.73076923076923 0.496955096721649
2.8974358974359 0.532823145389557
3.07692307692308 0.541922271251678
3.26282051282051 0.512376546859741
3.46153846153846 0.508139073848724
3.67307692307692 0.535699546337128
3.8974358974359 0.492870777845383
4.13461538461539 0.528167486190796
4.38461538461539 0.510218858718872
4.65384615384615 0.511157214641571
4.93589743589744 0.517184793949127
5.23717948717949 0.496850967407227
5.55769230769231 0.472786396741867
5.8974358974359 0.477531641721725
6.25641025641026 0.498432606458664
6.64102564102564 0.460486829280853
7.04487179487179 0.438479781150818
7.47435897435897 0.46664947271347
7.92948717948718 0.441045969724655
8.41025641025641 0.421119689941406
8.92307692307692 0.422891765832901
9.46794871794872 0.42658007144928
10.0448717948718 0.432987064123154
10.6602564102564 0.413601756095886
11.3076923076923 0.383596956729889
12 0.377312123775482
12.7307692307692 0.402666449546814
13.5064102564103 0.377176970243454
14.3333333333333 0.368500173091888
15.2051282051282 0.382433325052261
16.1346153846154 0.35497859120369
17.1153846153846 0.382538676261902
18.1602564102564 0.340335696935654
19.2692307692308 0.342615813016891
20.4423076923077 0.332889437675476
21.6858974358974 0.335856556892395
23.0128205128205 0.324691116809845
24.4102564102564 0.360475450754166
25.9038461538462 0.328890651464462
27.4807692307692 0.319437861442566
29.1538461538462 0.313668519258499
30.9358974358974 0.309333980083466
32.8205128205128 0.314044982194901
34.8205128205128 0.334709823131561
36.9423076923077 0.335591197013855
39.1923076923077 0.264169543981552
41.5833333333333 0.295094668865204
44.1153846153846 0.306091904640198
46.8076923076923 0.317763596773148
49.6602564102564 0.255792409181595
52.6858974358974 0.304184287786484
55.8974358974359 0.284889817237854
59.3076923076923 0.239944070577621
62.9230769230769 0.266788065433502
66.7564102564103 0.265516012907028
70.8269230769231 0.259475290775299
75.1474358974359 0.250136256217957
79.7307692307692 0.26816725730896
84.5897435897436 0.243269801139832
89.7435897435897 0.263031780719757
95.2179487179487 0.257965892553329
101.019230769231 0.231171473860741
107.179487179487 0.244554206728935
113.711538461538 0.220664024353027
120.641025641026 0.228769481182098
127.99358974359 0.219357118010521
135.801282051282 0.225902855396271
144.076923076923 0.216986805200577
152.858974358974 0.234073057770729
162.179487179487 0.219689428806305
172.064102564103 0.215743243694305
182.551282051282 0.211942851543427
193.679487179487 0.236804306507111
205.487179487179 0.194442957639694
218.012820512821 0.203221127390862
231.301282051282 0.220468834042549
245.403846153846 0.212179586291313
260.358974358974 0.205360293388367
276.230769230769 0.201681137084961
293.070512820513 0.221929892897606
310.935897435897 0.195198476314545
329.884615384615 0.175350770354271
350 0.209788396954536
};
\addplot [, color2, opacity=0.6, mark=triangle*, mark size=0.5, mark options={solid,rotate=180}, only marks, forget plot]
table {%
1 0.709929645061493
1.05769230769231 0.661476492881775
1.12179487179487 0.654055953025818
1.19230769230769 0.658202946186066
1.26282051282051 0.658425271511078
1.33974358974359 0.645660758018494
1.42307692307692 0.611770331859589
1.51282051282051 0.581015467643738
1.6025641025641 0.560784220695496
1.69871794871795 0.54262501001358
1.80128205128205 0.530614674091339
1.91666666666667 0.523192048072815
2.03205128205128 0.523999929428101
2.15384615384615 0.522952079772949
2.28846153846154 0.550732016563416
2.42307692307692 0.548570036888123
2.57692307692308 0.525221765041351
2.73076923076923 0.475638002157211
2.8974358974359 0.527211129665375
3.07692307692308 0.523084998130798
3.26282051282051 0.495727598667145
3.46153846153846 0.489909648895264
3.67307692307692 0.498298704624176
3.8974358974359 0.503126919269562
4.13461538461539 0.490450590848923
4.38461538461539 0.500726401805878
4.65384615384615 0.467392325401306
4.93589743589744 0.49252450466156
5.23717948717949 0.483961045742035
5.55769230769231 0.455408781766891
5.8974358974359 0.453444361686707
6.25641025641026 0.475347578525543
6.64102564102564 0.437987953424454
7.04487179487179 0.43772566318512
7.47435897435897 0.437475264072418
7.92948717948718 0.40141886472702
8.41025641025641 0.388427883386612
8.92307692307692 0.3845134973526
9.46794871794872 0.384101331233978
10.0448717948718 0.382831871509552
10.6602564102564 0.377270013093948
11.3076923076923 0.37665468454361
12 0.363342046737671
12.7307692307692 0.378310650587082
13.5064102564103 0.340169429779053
14.3333333333333 0.385502576828003
15.2051282051282 0.347375720739365
16.1346153846154 0.358537554740906
17.1153846153846 0.373846590518951
18.1602564102564 0.36324667930603
19.2692307692308 0.337150514125824
20.4423076923077 0.372345954179764
21.6858974358974 0.336289882659912
23.0128205128205 0.348206549882889
24.4102564102564 0.331802546977997
25.9038461538462 0.310964703559875
27.4807692307692 0.319952011108398
29.1538461538462 0.302917093038559
30.9358974358974 0.301790803670883
32.8205128205128 0.29354789853096
34.8205128205128 0.285156220197678
36.9423076923077 0.277139246463776
39.1923076923077 0.291673868894577
41.5833333333333 0.288128763437271
44.1153846153846 0.306794971227646
46.8076923076923 0.284113496541977
49.6602564102564 0.26854282617569
52.6858974358974 0.362858831882477
55.8974358974359 0.265324801206589
59.3076923076923 0.292509615421295
62.9230769230769 0.313730865716934
66.7564102564103 0.298955857753754
70.8269230769231 0.276224851608276
75.1474358974359 0.282352387905121
79.7307692307692 0.313784062862396
84.5897435897436 0.2672458589077
89.7435897435897 0.262320011854172
95.2179487179487 0.278077930212021
101.019230769231 0.24647168815136
107.179487179487 0.24945692718029
113.711538461538 0.302964389324188
120.641025641026 0.298262029886246
127.99358974359 0.258493989706039
135.801282051282 0.319558531045914
144.076923076923 0.278159588575363
152.858974358974 0.304880946874619
162.179487179487 0.287475377321243
172.064102564103 0.379067331552505
182.551282051282 0.209097251296043
193.679487179487 0.269838690757751
205.487179487179 0.240936383605003
218.012820512821 0.237175360321999
231.301282051282 0.262874156236649
245.403846153846 0.265143364667892
260.358974358974 0.303285390138626
276.230769230769 0.277356743812561
293.070512820513 0.288594424724579
310.935897435897 0.283373206853867
329.884615384615 0.262528717517853
350 0.313852161169052
};
\addplot [, color2, opacity=0.6, mark=triangle*, mark size=0.5, mark options={solid,rotate=180}, only marks, forget plot]
table {%
1 0.713838040828705
1.05769230769231 0.680650770664215
1.12179487179487 0.660059034824371
1.19230769230769 0.63089382648468
1.26282051282051 0.625465393066406
1.33974358974359 0.62207818031311
1.42307692307692 0.611583530902863
1.51282051282051 0.551476240158081
1.6025641025641 0.586498498916626
1.69871794871795 0.558214783668518
1.80128205128205 0.560494542121887
1.91666666666667 0.546748161315918
2.03205128205128 0.568022608757019
2.15384615384615 0.569947183132172
2.28846153846154 0.551776647567749
2.42307692307692 0.535740315914154
2.57692307692308 0.544866681098938
2.73076923076923 0.541088044643402
2.8974358974359 0.517265439033508
3.07692307692308 0.537785649299622
3.26282051282051 0.504805982112885
3.46153846153846 0.533848643302917
3.67307692307692 0.500144958496094
3.8974358974359 0.509709239006042
4.13461538461539 0.491724222898483
4.38461538461539 0.469246119260788
4.65384615384615 0.482795774936676
4.93589743589744 0.494996786117554
5.23717948717949 0.461327284574509
5.55769230769231 0.458588093519211
5.8974358974359 0.469387352466583
6.25641025641026 0.449335843324661
6.64102564102564 0.466501444578171
7.04487179487179 0.440632462501526
7.47435897435897 0.442272543907166
7.92948717948718 0.417618215084076
8.41025641025641 0.420079082250595
8.92307692307692 0.404069811105728
9.46794871794872 0.422534823417664
10.0448717948718 0.397647172212601
10.6602564102564 0.402311384677887
11.3076923076923 0.37046280503273
12 0.360604465007782
12.7307692307692 0.371036559343338
13.5064102564103 0.372358322143555
14.3333333333333 0.357747495174408
15.2051282051282 0.350447416305542
16.1346153846154 0.347291469573975
17.1153846153846 0.34366962313652
18.1602564102564 0.344366878271103
19.2692307692308 0.340786457061768
20.4423076923077 0.303503453731537
21.6858974358974 0.348332405090332
23.0128205128205 0.367438346147537
24.4102564102564 0.334063559770584
25.9038461538462 0.333479225635529
27.4807692307692 0.326611697673798
29.1538461538462 0.303031235933304
30.9358974358974 0.281131893396378
32.8205128205128 0.29285055398941
34.8205128205128 0.306966423988342
36.9423076923077 0.292563766241074
39.1923076923077 0.286253303289413
41.5833333333333 0.302805125713348
44.1153846153846 0.300446480512619
46.8076923076923 0.279358893632889
49.6602564102564 0.300411641597748
52.6858974358974 0.308160126209259
55.8974358974359 0.266606509685516
59.3076923076923 0.25622832775116
62.9230769230769 0.253314882516861
66.7564102564103 0.255644023418427
70.8269230769231 0.248242005705833
75.1474358974359 0.264386713504791
79.7307692307692 0.253403842449188
84.5897435897436 0.240413710474968
89.7435897435897 0.227411225438118
95.2179487179487 0.239878609776497
101.019230769231 0.260700851678848
107.179487179487 0.218208283185959
113.711538461538 0.207608059048653
120.641025641026 0.217206060886383
127.99358974359 0.312494426965714
135.801282051282 0.217053279280663
144.076923076923 0.210543170571327
152.858974358974 0.205354019999504
162.179487179487 0.228276699781418
172.064102564103 0.206248015165329
182.551282051282 0.23309962451458
193.679487179487 0.200682565569878
205.487179487179 0.25390699505806
218.012820512821 0.16775581240654
231.301282051282 0.232962399721146
245.403846153846 0.169281646609306
260.358974358974 0.228300020098686
276.230769230769 0.202328622341156
293.070512820513 0.22395883500576
310.935897435897 0.228832885622978
329.884615384615 0.171160280704498
350 0.186985209584236
};
\end{axis}

\end{tikzpicture}

      \tikzexternaldisable
    \end{minipage}
  \end{subfigure}

  \begin{subfigure}[t]{\linewidth}
    \centering
    \caption{\cifarhun \allcnnc \adam}
    \begin{minipage}{0.50\linewidth}
      \centering
      % defines the pgfplots style "eigspacedefault"
\pgfkeys{/pgfplots/eigspacedefault/.style={
    width=1.0\linewidth,
    height=0.6\linewidth,
    every axis plot/.append style={line width = 1.5pt},
    tick pos = left,
    ylabel near ticks,
    xlabel near ticks,
    xtick align = inside,
    ytick align = inside,
    legend cell align = left,
    legend columns = 4,
    legend pos = south east,
    legend style = {
      fill opacity = 1,
      text opacity = 1,
      font = \footnotesize,
      at={(1, 1.025)},
      anchor=south east,
      column sep=0.25cm,
    },
    legend image post style={scale=2.5},
    xticklabel style = {font = \footnotesize},
    xlabel style = {font = \footnotesize},
    axis line style = {black},
    yticklabel style = {font = \footnotesize},
    ylabel style = {font = \footnotesize},
    title style = {font = \footnotesize},
    grid = major,
    grid style = {dashed}
  }
}

\pgfkeys{/pgfplots/eigspacedefaultapp/.style={
    eigspacedefault,
    height=0.6\linewidth,
    legend columns = 2,
  }
}

\pgfkeys{/pgfplots/eigspacenolegend/.style={
    legend image post style = {scale=0},
    legend style = {
      fill opacity = 0,
      draw opacity = 0,
      text opacity = 0,
      font = \footnotesize,
      at={(1, 1.025)},
      anchor=south east,
      column sep=0.25cm,
    },
  }
}
%%% Local Variables:
%%% mode: latex
%%% TeX-master: "../../thesis"
%%% End:

      \pgfkeys{/pgfplots/zmystyle/.style={
          eigspacedefaultapp,
          eigspacenolegend,
        }}
      \tikzexternalenable
      \vspace{-6ex}
      % This file was created by tikzplotlib v0.9.7.
\begin{tikzpicture}

\definecolor{color0}{rgb}{0.501960784313725,0.184313725490196,0.6}
\definecolor{color1}{rgb}{0.870588235294118,0.623529411764706,0.0862745098039216}
\definecolor{color2}{rgb}{0.274509803921569,0.6,0.564705882352941}

\begin{axis}[
axis line style={white!10!black},
legend columns=2,
legend style={fill opacity=0.8, draw opacity=1, text opacity=1, at={(0.03,0.03)}, anchor=south west, draw=white!80!black},
log basis x={10},
tick pos=left,
xlabel={epoch (log scale)},
xmajorgrids,
xmin=0.746099240306814, xmax=469.106495613199,
xmode=log,
ylabel={overlap},
ymajorgrids,
ymin=-0.05, ymax=1.05,
zmystyle
]
\addplot [, white!10!black, dashed, forget plot]
table {%
0.746099240306814 1
469.106495613199 1
};
\addplot [, white!10!black, dashed, forget plot]
table {%
0.746099240306814 0
469.106495613199 0
};
\addplot [, color0, opacity=0.6, mark=triangle*, mark size=0.5, mark options={solid,rotate=180}, only marks]
table {%
1 0.793246686458588
1.05769230769231 0.774845719337463
1.12179487179487 0.742143750190735
1.19230769230769 0.644991159439087
1.26282051282051 0.470882445573807
1.33974358974359 0.488690048456192
1.42307692307692 0.529799282550812
1.51282051282051 0.551525354385376
1.6025641025641 0.527334332466125
1.69871794871795 0.505793154239655
1.80128205128205 0.522373855113983
1.91666666666667 0.475527763366699
2.03205128205128 0.50280350446701
2.15384615384615 0.498953491449356
2.28846153846154 0.461020350456238
2.42307692307692 0.508001327514648
2.57692307692308 0.42589396238327
2.73076923076923 0.447598487138748
2.8974358974359 0.413882970809937
3.07692307692308 0.461887031793594
3.26282051282051 0.403804123401642
3.46153846153846 0.475040644407272
3.67307692307692 0.419210106134415
3.8974358974359 0.431470483541489
4.13461538461539 0.503166019916534
4.38461538461539 0.482395708560944
4.65384615384615 0.487207621335983
4.93589743589744 0.49531278014183
5.23717948717949 0.508805096149445
5.55769230769231 0.461981803178787
5.8974358974359 0.425445556640625
6.25641025641026 0.440809398889542
6.64102564102564 0.444982945919037
7.04487179487179 0.436324149370193
7.47435897435897 0.428252369165421
7.92948717948718 0.411004602909088
8.41025641025641 0.442070603370667
8.92307692307692 0.397738933563232
9.46794871794872 0.418749213218689
10.0448717948718 0.433534979820251
10.6602564102564 0.402150720357895
11.3076923076923 0.428758233785629
12 0.39135816693306
12.7307692307692 0.423500597476959
13.5064102564103 0.386882245540619
14.3333333333333 0.404055774211884
15.2051282051282 0.390650928020477
16.1346153846154 0.40841743350029
17.1153846153846 0.386057287454605
18.1602564102564 0.394187420606613
19.2692307692308 0.388472884893417
20.4423076923077 0.385715693235397
21.6858974358974 0.381951212882996
23.0128205128205 0.385925143957138
24.4102564102564 0.372194707393646
25.9038461538462 0.353976011276245
27.4807692307692 0.349448651075363
29.1538461538462 0.365646243095398
30.9358974358974 0.355080485343933
32.8205128205128 0.378538280725479
34.8205128205128 0.372231096029282
36.9423076923077 0.342481911182404
39.1923076923077 0.315003514289856
41.5833333333333 0.332304418087006
44.1153846153846 0.324362635612488
46.8076923076923 0.339172750711441
49.6602564102564 0.316260933876038
52.6858974358974 0.316843926906586
55.8974358974359 0.320232003927231
59.3076923076923 0.322043418884277
62.9230769230769 0.309828191995621
66.7564102564103 0.319123357534409
70.8269230769231 0.294459611177444
75.1474358974359 0.297058254480362
79.7307692307692 0.30745866894722
84.5897435897436 0.295030653476715
89.7435897435897 0.3005730509758
95.2179487179487 0.283224016427994
101.019230769231 0.284030616283417
107.179487179487 0.281940460205078
113.711538461538 0.285442799329758
120.641025641026 0.277311980724335
127.99358974359 0.293025135993958
135.801282051282 0.27681091427803
144.076923076923 0.27960392832756
152.858974358974 0.287288665771484
162.179487179487 0.289877474308014
172.064102564103 0.289868205785751
182.551282051282 0.273409366607666
193.679487179487 0.280507624149323
205.487179487179 0.273161351680756
218.012820512821 0.281634330749512
231.301282051282 0.286675095558167
245.403846153846 0.275004357099533
260.358974358974 0.28181666135788
276.230769230769 0.283496201038361
293.070512820513 0.2775499522686
310.935897435897 0.286068618297577
329.884615384615 0.28189891576767
350 0.282122135162354
};
\addlegendentry{mb 2, exact}
\addplot [, color0, opacity=0.6, mark=triangle*, mark size=0.5, mark options={solid,rotate=180}, only marks, forget plot]
table {%
1 0.755359947681427
1.05769230769231 0.758731365203857
1.12179487179487 0.754085302352905
1.19230769230769 0.655419170856476
1.26282051282051 0.551905512809753
1.33974358974359 0.543245673179626
1.42307692307692 0.553290903568268
1.51282051282051 0.518876969814301
1.6025641025641 0.440645128488541
1.69871794871795 0.493505954742432
1.80128205128205 0.423752784729004
1.91666666666667 0.48847159743309
2.03205128205128 0.498514503240585
2.15384615384615 0.487112790346146
2.28846153846154 0.474547803401947
2.42307692307692 0.500971257686615
2.57692307692308 0.499632239341736
2.73076923076923 0.469125807285309
2.8974358974359 0.473997831344604
3.07692307692308 0.438296943902969
3.26282051282051 0.418061405420303
3.46153846153846 0.439655870199203
3.67307692307692 0.491532415151596
3.8974358974359 0.414289921522141
4.13461538461539 0.445367813110352
4.38461538461539 0.448873788118362
4.65384615384615 0.415966033935547
4.93589743589744 0.447782725095749
5.23717948717949 0.41165816783905
5.55769230769231 0.398233473300934
5.8974358974359 0.372105121612549
6.25641025641026 0.39750275015831
6.64102564102564 0.407548516988754
7.04487179487179 0.422376781702042
7.47435897435897 0.383938431739807
7.92948717948718 0.379307478666306
8.41025641025641 0.396584004163742
8.92307692307692 0.386242508888245
9.46794871794872 0.428839266300201
10.0448717948718 0.42895033955574
10.6602564102564 0.389748185873032
11.3076923076923 0.400299191474915
12 0.382566839456558
12.7307692307692 0.408149033784866
13.5064102564103 0.381190150976181
14.3333333333333 0.397528231143951
15.2051282051282 0.378678798675537
16.1346153846154 0.392534166574478
17.1153846153846 0.382490068674088
18.1602564102564 0.373769611120224
19.2692307692308 0.366878658533096
20.4423076923077 0.368917405605316
21.6858974358974 0.375721663236618
23.0128205128205 0.373187392950058
24.4102564102564 0.36210173368454
25.9038461538462 0.349851757287979
27.4807692307692 0.337980568408966
29.1538461538462 0.342924147844315
30.9358974358974 0.335139393806458
32.8205128205128 0.355039805173874
34.8205128205128 0.347726404666901
36.9423076923077 0.330625832080841
39.1923076923077 0.310145169496536
41.5833333333333 0.318939954042435
44.1153846153846 0.316066801548004
46.8076923076923 0.307588309049606
49.6602564102564 0.308419585227966
52.6858974358974 0.303726196289062
55.8974358974359 0.305620789527893
59.3076923076923 0.309611529111862
62.9230769230769 0.294347524642944
66.7564102564103 0.294322222471237
70.8269230769231 0.276831358671188
75.1474358974359 0.280568897724152
79.7307692307692 0.289939194917679
84.5897435897436 0.277535408735275
89.7435897435897 0.276948034763336
95.2179487179487 0.284639567136765
101.019230769231 0.279077231884003
107.179487179487 0.273152619600296
113.711538461538 0.274838417768478
120.641025641026 0.26915967464447
127.99358974359 0.272559314966202
135.801282051282 0.267493218183517
144.076923076923 0.268727421760559
152.858974358974 0.260871201753616
162.179487179487 0.272146552801132
172.064102564103 0.265245229005814
182.551282051282 0.263369619846344
193.679487179487 0.275171488523483
205.487179487179 0.271213293075562
218.012820512821 0.266526937484741
231.301282051282 0.271692305803299
245.403846153846 0.267114967107773
260.358974358974 0.26280489563942
276.230769230769 0.265317261219025
293.070512820513 0.266160696744919
310.935897435897 0.264097660779953
329.884615384615 0.264874517917633
350 0.261476159095764
};
\addplot [, color0, opacity=0.6, mark=triangle*, mark size=0.5, mark options={solid,rotate=180}, only marks, forget plot]
table {%
1 0.767154514789581
1.05769230769231 0.666435241699219
1.12179487179487 0.73258513212204
1.19230769230769 0.733433127403259
1.26282051282051 0.697629392147064
1.33974358974359 0.668901681900024
1.42307692307692 0.697328925132751
1.51282051282051 0.642585039138794
1.6025641025641 0.563968062400818
1.69871794871795 0.618226706981659
1.80128205128205 0.455004036426544
1.91666666666667 0.492829203605652
2.03205128205128 0.525875747203827
2.15384615384615 0.477933526039124
2.28846153846154 0.493941396474838
2.42307692307692 0.52580600976944
2.57692307692308 0.510558724403381
2.73076923076923 0.509945511817932
2.8974358974359 0.495130211114883
3.07692307692308 0.425825476646423
3.26282051282051 0.467749685049057
3.46153846153846 0.492549538612366
3.67307692307692 0.48531112074852
3.8974358974359 0.424923837184906
4.13461538461539 0.493934035301208
4.38461538461539 0.45100000500679
4.65384615384615 0.413902997970581
4.93589743589744 0.430559873580933
5.23717948717949 0.436805903911591
5.55769230769231 0.355286926031113
5.8974358974359 0.352439314126968
6.25641025641026 0.332711905241013
6.64102564102564 0.382865905761719
7.04487179487179 0.374063670635223
7.47435897435897 0.380889505147934
7.92948717948718 0.354876697063446
8.41025641025641 0.370484083890915
8.92307692307692 0.337274551391602
9.46794871794872 0.377379029989243
10.0448717948718 0.382929384708405
10.6602564102564 0.397717207670212
11.3076923076923 0.36634749174118
12 0.346944570541382
12.7307692307692 0.363356709480286
13.5064102564103 0.348041564226151
14.3333333333333 0.363394051790237
15.2051282051282 0.343345642089844
16.1346153846154 0.373401552438736
17.1153846153846 0.360141217708588
18.1602564102564 0.368906557559967
19.2692307692308 0.348874926567078
20.4423076923077 0.337938100099564
21.6858974358974 0.349375665187836
23.0128205128205 0.332844346761703
24.4102564102564 0.342226505279541
25.9038461538462 0.329669117927551
27.4807692307692 0.321041703224182
29.1538461538462 0.327884197235107
30.9358974358974 0.310375273227692
32.8205128205128 0.328599870204926
34.8205128205128 0.332038640975952
36.9423076923077 0.308958947658539
39.1923076923077 0.289573103189468
41.5833333333333 0.291812568902969
44.1153846153846 0.287944138050079
46.8076923076923 0.295949548482895
49.6602564102564 0.294827520847321
52.6858974358974 0.280673682689667
55.8974358974359 0.300203859806061
59.3076923076923 0.298245131969452
62.9230769230769 0.286756634712219
66.7564102564103 0.289148181676865
70.8269230769231 0.273697823286057
75.1474358974359 0.283032923936844
79.7307692307692 0.290672421455383
84.5897435897436 0.284023940563202
89.7435897435897 0.271042376756668
95.2179487179487 0.277518063783646
101.019230769231 0.273054391145706
107.179487179487 0.262390285730362
113.711538461538 0.270325183868408
120.641025641026 0.256631791591644
127.99358974359 0.269971162080765
135.801282051282 0.243642121553421
144.076923076923 0.25660640001297
152.858974358974 0.253190636634827
162.179487179487 0.267460078001022
172.064102564103 0.261264711618423
182.551282051282 0.256422847509384
193.679487179487 0.260583996772766
205.487179487179 0.264443725347519
218.012820512821 0.274806082248688
231.301282051282 0.278864681720734
245.403846153846 0.279251426458359
260.358974358974 0.277719587087631
276.230769230769 0.275957256555557
293.070512820513 0.263682872056961
310.935897435897 0.280012547969818
329.884615384615 0.281758308410645
350 0.264529019594193
};
\addplot [, color0, opacity=0.6, mark=triangle*, mark size=0.5, mark options={solid,rotate=180}, only marks, forget plot]
table {%
1 0.716560363769531
1.05769230769231 0.685537874698639
1.12179487179487 0.758944630622864
1.19230769230769 0.645853698253632
1.26282051282051 0.629837036132812
1.33974358974359 0.619229137897491
1.42307692307692 0.51331353187561
1.51282051282051 0.51598995923996
1.6025641025641 0.582072198390961
1.69871794871795 0.589608073234558
1.80128205128205 0.5658820271492
1.91666666666667 0.556092381477356
2.03205128205128 0.601901113986969
2.15384615384615 0.564995050430298
2.28846153846154 0.568283438682556
2.42307692307692 0.549529731273651
2.57692307692308 0.587241411209106
2.73076923076923 0.522071242332458
2.8974358974359 0.540183782577515
3.07692307692308 0.500568091869354
3.26282051282051 0.571507930755615
3.46153846153846 0.572633981704712
3.67307692307692 0.534929811954498
3.8974358974359 0.517212450504303
4.13461538461539 0.52415406703949
4.38461538461539 0.505014836788177
4.65384615384615 0.486618787050247
4.93589743589744 0.466440737247467
5.23717948717949 0.462377160787582
5.55769230769231 0.453974783420563
5.8974358974359 0.402289927005768
6.25641025641026 0.396276503801346
6.64102564102564 0.405550688505173
7.04487179487179 0.403522938489914
7.47435897435897 0.382002174854279
7.92948717948718 0.380125671625137
8.41025641025641 0.395869880914688
8.92307692307692 0.385646313428879
9.46794871794872 0.380195677280426
10.0448717948718 0.383531719446182
10.6602564102564 0.393270879983902
11.3076923076923 0.384594559669495
12 0.363411009311676
12.7307692307692 0.390275478363037
13.5064102564103 0.354672431945801
14.3333333333333 0.381057739257812
15.2051282051282 0.365705102682114
16.1346153846154 0.380258709192276
17.1153846153846 0.381630629301071
18.1602564102564 0.376073181629181
19.2692307692308 0.37703862786293
20.4423076923077 0.352657318115234
21.6858974358974 0.369902193546295
23.0128205128205 0.360136866569519
24.4102564102564 0.37635001540184
25.9038461538462 0.355403751134872
27.4807692307692 0.355525881052017
29.1538461538462 0.360312134027481
30.9358974358974 0.35003337264061
32.8205128205128 0.342603832483292
34.8205128205128 0.338067770004272
36.9423076923077 0.330541759729385
39.1923076923077 0.33058750629425
41.5833333333333 0.31242847442627
44.1153846153846 0.321057111024857
46.8076923076923 0.33111160993576
49.6602564102564 0.291487872600555
52.6858974358974 0.304245114326477
55.8974358974359 0.319876253604889
59.3076923076923 0.300223290920258
62.9230769230769 0.311548382043839
66.7564102564103 0.301384508609772
70.8269230769231 0.304608076810837
75.1474358974359 0.294742107391357
79.7307692307692 0.295718848705292
84.5897435897436 0.303636014461517
89.7435897435897 0.297366827726364
95.2179487179487 0.28708279132843
101.019230769231 0.290144354104996
107.179487179487 0.289747059345245
113.711538461538 0.281350582838058
120.641025641026 0.286394566297531
127.99358974359 0.278971940279007
135.801282051282 0.275151580572128
144.076923076923 0.280127316713333
152.858974358974 0.260329455137253
162.179487179487 0.265271157026291
172.064102564103 0.268063634634018
182.551282051282 0.260706543922424
193.679487179487 0.268586337566376
205.487179487179 0.258010029792786
218.012820512821 0.260864734649658
231.301282051282 0.264122545719147
245.403846153846 0.270500719547272
260.358974358974 0.263084590435028
276.230769230769 0.268430709838867
293.070512820513 0.270681768655777
310.935897435897 0.277792155742645
329.884615384615 0.274905234575272
350 0.264933347702026
};
\addplot [, color0, opacity=0.6, mark=triangle*, mark size=0.5, mark options={solid,rotate=180}, only marks, forget plot]
table {%
1 0.727912247180939
1.05769230769231 0.723761677742004
1.12179487179487 0.758196711540222
1.19230769230769 0.621961951255798
1.26282051282051 0.657626390457153
1.33974358974359 0.616607666015625
1.42307692307692 0.548345386981964
1.51282051282051 0.620282471179962
1.6025641025641 0.639028131961823
1.69871794871795 0.567456066608429
1.80128205128205 0.644322514533997
1.91666666666667 0.579512000083923
2.03205128205128 0.573110818862915
2.15384615384615 0.548118352890015
2.28846153846154 0.540154457092285
2.42307692307692 0.46214759349823
2.57692307692308 0.508308708667755
2.73076923076923 0.489877611398697
2.8974358974359 0.425765246152878
3.07692307692308 0.490445882081985
3.26282051282051 0.50288200378418
3.46153846153846 0.51367712020874
3.67307692307692 0.53587394952774
3.8974358974359 0.501436471939087
4.13461538461539 0.532074451446533
4.38461538461539 0.490831732749939
4.65384615384615 0.517195582389832
4.93589743589744 0.497772127389908
5.23717948717949 0.484767556190491
5.55769230769231 0.470230311155319
5.8974358974359 0.426253646612167
6.25641025641026 0.464414954185486
6.64102564102564 0.467041850090027
7.04487179487179 0.467384397983551
7.47435897435897 0.446205973625183
7.92948717948718 0.414973795413971
8.41025641025641 0.424027770757675
8.92307692307692 0.371405363082886
9.46794871794872 0.360112130641937
10.0448717948718 0.385181307792664
10.6602564102564 0.364351570606232
11.3076923076923 0.346366196870804
12 0.321441531181335
12.7307692307692 0.308536648750305
13.5064102564103 0.295022428035736
14.3333333333333 0.289687275886536
15.2051282051282 0.267816245555878
16.1346153846154 0.274396061897278
17.1153846153846 0.270310640335083
18.1602564102564 0.291421979665756
19.2692307692308 0.27652695775032
20.4423076923077 0.256981372833252
21.6858974358974 0.26189523935318
23.0128205128205 0.269029319286346
24.4102564102564 0.277190059423447
25.9038461538462 0.280331790447235
27.4807692307692 0.26530459523201
29.1538461538462 0.272745728492737
30.9358974358974 0.270849317312241
32.8205128205128 0.245808094739914
34.8205128205128 0.258771777153015
36.9423076923077 0.248169988393784
39.1923076923077 0.268304884433746
41.5833333333333 0.265237778425217
44.1153846153846 0.255126237869263
46.8076923076923 0.24665130674839
49.6602564102564 0.252057522535324
52.6858974358974 0.259929567575455
55.8974358974359 0.254450023174286
59.3076923076923 0.265275090932846
62.9230769230769 0.246299773454666
66.7564102564103 0.25032913684845
70.8269230769231 0.258096665143967
75.1474358974359 0.249968186020851
79.7307692307692 0.264387667179108
84.5897435897436 0.255920112133026
89.7435897435897 0.25663486123085
95.2179487179487 0.245269119739532
101.019230769231 0.237830430269241
107.179487179487 0.251990646123886
113.711538461538 0.239118725061417
120.641025641026 0.231614321470261
127.99358974359 0.25298273563385
135.801282051282 0.233489796519279
144.076923076923 0.246873319149017
152.858974358974 0.251733630895615
162.179487179487 0.249728694558144
172.064102564103 0.253989279270172
182.551282051282 0.258929252624512
193.679487179487 0.251402050256729
205.487179487179 0.247225522994995
218.012820512821 0.247686222195625
231.301282051282 0.2449861317873
245.403846153846 0.249676018953323
260.358974358974 0.248075172305107
276.230769230769 0.250444024801254
293.070512820513 0.257044017314911
310.935897435897 0.259179890155792
329.884615384615 0.25171285867691
350 0.255068391561508
};
\addplot [, color1, opacity=0.6, mark=square*, mark size=0.5, mark options={solid}, only marks]
table {%
1 0.867826342582703
1.05769230769231 0.870680212974548
1.12179487179487 0.908377945423126
1.19230769230769 0.839245676994324
1.26282051282051 0.817536532878876
1.33974358974359 0.79462593793869
1.42307692307692 0.681016981601715
1.51282051282051 0.680646657943726
1.6025641025641 0.675846874713898
1.69871794871795 0.741382896900177
1.80128205128205 0.710706055164337
1.91666666666667 0.670921325683594
2.03205128205128 0.721932351589203
2.15384615384615 0.654193699359894
2.28846153846154 0.705449402332306
2.42307692307692 0.726293325424194
2.57692307692308 0.702725887298584
2.73076923076923 0.666114807128906
2.8974358974359 0.69686758518219
3.07692307692308 0.617198169231415
3.26282051282051 0.663582980632782
3.46153846153846 0.652130544185638
3.67307692307692 0.640122830867767
3.8974358974359 0.595792949199677
4.13461538461539 0.62706333398819
4.38461538461539 0.610784411430359
4.65384615384615 0.592592000961304
4.93589743589744 0.585455715656281
5.23717948717949 0.523314893245697
5.55769230769231 0.576977670192719
5.8974358974359 0.570084035396576
6.25641025641026 0.558309018611908
6.64102564102564 0.545837640762329
7.04487179487179 0.567672073841095
7.47435897435897 0.539479672908783
7.92948717948718 0.550464987754822
8.41025641025641 0.55353832244873
8.92307692307692 0.542004466056824
9.46794871794872 0.533555388450623
10.0448717948718 0.53227710723877
10.6602564102564 0.525602400302887
11.3076923076923 0.504887878894806
12 0.502330601215363
12.7307692307692 0.511539876461029
13.5064102564103 0.497761815786362
14.3333333333333 0.495380461215973
15.2051282051282 0.476374298334122
16.1346153846154 0.485484272241592
17.1153846153846 0.478236347436905
18.1602564102564 0.464453041553497
19.2692307692308 0.456325829029083
20.4423076923077 0.462257653474808
21.6858974358974 0.47023668885231
23.0128205128205 0.452386081218719
24.4102564102564 0.444444239139557
25.9038461538462 0.435731381177902
27.4807692307692 0.442624777555466
29.1538461538462 0.439605236053467
30.9358974358974 0.415408074855804
32.8205128205128 0.434436947107315
34.8205128205128 0.422098487615585
36.9423076923077 0.414060652256012
39.1923076923077 0.411868274211884
41.5833333333333 0.402589708566666
44.1153846153846 0.417713791131973
46.8076923076923 0.402784407138824
49.6602564102564 0.39198637008667
52.6858974358974 0.404604852199554
55.8974358974359 0.39944851398468
59.3076923076923 0.388396888971329
62.9230769230769 0.381343156099319
66.7564102564103 0.38196387887001
70.8269230769231 0.39131972193718
75.1474358974359 0.379100114107132
79.7307692307692 0.384473711252213
84.5897435897436 0.381387174129486
89.7435897435897 0.390856206417084
95.2179487179487 0.379694133996964
101.019230769231 0.36076346039772
107.179487179487 0.364801675081253
113.711538461538 0.36585333943367
120.641025641026 0.368166506290436
127.99358974359 0.355139166116714
135.801282051282 0.361519932746887
144.076923076923 0.353466659784317
152.858974358974 0.352580100297928
162.179487179487 0.360991656780243
172.064102564103 0.350637823343277
182.551282051282 0.353546977043152
193.679487179487 0.357731848955154
205.487179487179 0.36268362402916
218.012820512821 0.343117356300354
231.301282051282 0.365451663732529
245.403846153846 0.359901487827301
260.358974358974 0.358305662870407
276.230769230769 0.361052393913269
293.070512820513 0.374967485666275
310.935897435897 0.361714631319046
329.884615384615 0.357857227325439
350 0.363104045391083
};
\addlegendentry{mb 8, exact}
\addplot [, color1, opacity=0.6, mark=square*, mark size=0.5, mark options={solid}, only marks, forget plot]
table {%
1 0.877033233642578
1.05769230769231 0.886822938919067
1.12179487179487 0.900850057601929
1.19230769230769 0.865303993225098
1.26282051282051 0.771969139575958
1.33974358974359 0.773715317249298
1.42307692307692 0.822133719921112
1.51282051282051 0.831314980983734
1.6025641025641 0.748504221439362
1.69871794871795 0.728114604949951
1.80128205128205 0.718515753746033
1.91666666666667 0.733468174934387
2.03205128205128 0.702340364456177
2.15384615384615 0.653134226799011
2.28846153846154 0.645872354507446
2.42307692307692 0.660398781299591
2.57692307692308 0.660195469856262
2.73076923076923 0.648078143596649
2.8974358974359 0.624230921268463
3.07692307692308 0.647242665290833
3.26282051282051 0.618540048599243
3.46153846153846 0.65432220697403
3.67307692307692 0.615051627159119
3.8974358974359 0.610474050045013
4.13461538461539 0.670550048351288
4.38461538461539 0.630305230617523
4.65384615384615 0.634339332580566
4.93589743589744 0.627969026565552
5.23717948717949 0.629674196243286
5.55769230769231 0.590091705322266
5.8974358974359 0.543181955814362
6.25641025641026 0.522636771202087
6.64102564102564 0.585013270378113
7.04487179487179 0.521116018295288
7.47435897435897 0.550144195556641
7.92948717948718 0.511033117771149
8.41025641025641 0.518102765083313
8.92307692307692 0.496842950582504
9.46794871794872 0.49970954656601
10.0448717948718 0.508717656135559
10.6602564102564 0.512285053730011
11.3076923076923 0.4809190928936
12 0.439425230026245
12.7307692307692 0.434189110994339
13.5064102564103 0.429218292236328
14.3333333333333 0.428583592176437
15.2051282051282 0.446612954139709
16.1346153846154 0.427907526493073
17.1153846153846 0.409113049507141
18.1602564102564 0.432314366102219
19.2692307692308 0.416774928569794
20.4423076923077 0.396315306425095
21.6858974358974 0.398844361305237
23.0128205128205 0.399576485157013
24.4102564102564 0.41085883975029
25.9038461538462 0.401943206787109
27.4807692307692 0.386264264583588
29.1538461538462 0.396133035421371
30.9358974358974 0.38043949007988
32.8205128205128 0.417403340339661
34.8205128205128 0.406747579574585
36.9423076923077 0.366704255342484
39.1923076923077 0.348740637302399
41.5833333333333 0.368059575557709
44.1153846153846 0.350389540195465
46.8076923076923 0.354720264673233
49.6602564102564 0.351101070642471
52.6858974358974 0.336329489946365
55.8974358974359 0.328103393316269
59.3076923076923 0.349369943141937
62.9230769230769 0.338964194059372
66.7564102564103 0.34757536649704
70.8269230769231 0.335181891918182
75.1474358974359 0.318570822477341
79.7307692307692 0.355343610048294
84.5897435897436 0.332092553377151
89.7435897435897 0.338311851024628
95.2179487179487 0.317845702171326
101.019230769231 0.312469065189362
107.179487179487 0.327023148536682
113.711538461538 0.322750627994537
120.641025641026 0.325309067964554
127.99358974359 0.330462753772736
135.801282051282 0.317111611366272
144.076923076923 0.32227149605751
152.858974358974 0.299916833639145
162.179487179487 0.314618945121765
172.064102564103 0.298182010650635
182.551282051282 0.316247582435608
193.679487179487 0.316939562559128
205.487179487179 0.310619503259659
218.012820512821 0.316008031368256
231.301282051282 0.312373459339142
245.403846153846 0.30505096912384
260.358974358974 0.317182928323746
276.230769230769 0.330335468053818
293.070512820513 0.32904914021492
310.935897435897 0.325127482414246
329.884615384615 0.335046947002411
350 0.306783199310303
};
\addplot [, color1, opacity=0.6, mark=square*, mark size=0.5, mark options={solid}, only marks, forget plot]
table {%
1 0.837624669075012
1.05769230769231 0.861359715461731
1.12179487179487 0.877668261528015
1.19230769230769 0.904847681522369
1.26282051282051 0.808125793933868
1.33974358974359 0.759432196617126
1.42307692307692 0.731429040431976
1.51282051282051 0.774981915950775
1.6025641025641 0.795464634895325
1.69871794871795 0.774860799312592
1.80128205128205 0.763229191303253
1.91666666666667 0.770067751407623
2.03205128205128 0.772400319576263
2.15384615384615 0.742692708969116
2.28846153846154 0.73603755235672
2.42307692307692 0.700455784797668
2.57692307692308 0.72574245929718
2.73076923076923 0.680208563804626
2.8974358974359 0.68520849943161
3.07692307692308 0.660072386264801
3.26282051282051 0.64326423406601
3.46153846153846 0.688507676124573
3.67307692307692 0.652909517288208
3.8974358974359 0.625031054019928
4.13461538461539 0.706974446773529
4.38461538461539 0.655564725399017
4.65384615384615 0.66609913110733
4.93589743589744 0.643384993076324
5.23717948717949 0.667938351631165
5.55769230769231 0.630628407001495
5.8974358974359 0.57565301656723
6.25641025641026 0.578224956989288
6.64102564102564 0.597883224487305
7.04487179487179 0.575717926025391
7.47435897435897 0.585530579090118
7.92948717948718 0.54997456073761
8.41025641025641 0.587992489337921
8.92307692307692 0.533065617084503
9.46794871794872 0.545669257640839
10.0448717948718 0.558410882949829
10.6602564102564 0.541958808898926
11.3076923076923 0.528051853179932
12 0.508415758609772
12.7307692307692 0.519729554653168
13.5064102564103 0.517505884170532
14.3333333333333 0.511726319789886
15.2051282051282 0.495287925004959
16.1346153846154 0.491559207439423
17.1153846153846 0.490969836711884
18.1602564102564 0.480875313282013
19.2692307692308 0.479986399412155
20.4423076923077 0.459598898887634
21.6858974358974 0.460064232349396
23.0128205128205 0.456446141004562
24.4102564102564 0.467631131410599
25.9038461538462 0.45915812253952
27.4807692307692 0.447922974824905
29.1538461538462 0.449643015861511
30.9358974358974 0.435065150260925
32.8205128205128 0.442735970020294
34.8205128205128 0.441641211509705
36.9423076923077 0.410493910312653
39.1923076923077 0.40312597155571
41.5833333333333 0.385808259248734
44.1153846153846 0.38593801856041
46.8076923076923 0.401337027549744
49.6602564102564 0.374367922544479
52.6858974358974 0.376787483692169
55.8974358974359 0.388384997844696
59.3076923076923 0.375318259000778
62.9230769230769 0.377186268568039
66.7564102564103 0.393567502498627
70.8269230769231 0.392686456441879
75.1474358974359 0.379237115383148
79.7307692307692 0.389675438404083
84.5897435897436 0.364555358886719
89.7435897435897 0.382005751132965
95.2179487179487 0.371748447418213
101.019230769231 0.363150030374527
107.179487179487 0.358473628759384
113.711538461538 0.36386451125145
120.641025641026 0.370725154876709
127.99358974359 0.371555209159851
135.801282051282 0.366646438837051
144.076923076923 0.369421124458313
152.858974358974 0.357012510299683
162.179487179487 0.366013467311859
172.064102564103 0.359062254428864
182.551282051282 0.365966260433197
193.679487179487 0.36337599158287
205.487179487179 0.358292311429977
218.012820512821 0.352423459291458
231.301282051282 0.342393934726715
245.403846153846 0.34603363275528
260.358974358974 0.359891057014465
276.230769230769 0.353584289550781
293.070512820513 0.345352232456207
310.935897435897 0.356914967298508
329.884615384615 0.35246679186821
350 0.351590871810913
};
\addplot [, color1, opacity=0.6, mark=square*, mark size=0.5, mark options={solid}, only marks, forget plot]
table {%
1 0.875446736812592
1.05769230769231 0.896196722984314
1.12179487179487 0.903291761875153
1.19230769230769 0.8692826628685
1.26282051282051 0.783959925174713
1.33974358974359 0.775725483894348
1.42307692307692 0.724047064781189
1.51282051282051 0.767565011978149
1.6025641025641 0.780515789985657
1.69871794871795 0.757099747657776
1.80128205128205 0.726081550121307
1.91666666666667 0.719274759292603
2.03205128205128 0.721284449100494
2.15384615384615 0.721999406814575
2.28846153846154 0.71784907579422
2.42307692307692 0.696724355220795
2.57692307692308 0.693620443344116
2.73076923076923 0.64294707775116
2.8974358974359 0.67568176984787
3.07692307692308 0.639168739318848
3.26282051282051 0.650091528892517
3.46153846153846 0.644347429275513
3.67307692307692 0.600160121917725
3.8974358974359 0.604834794998169
4.13461538461539 0.643069267272949
4.38461538461539 0.581520199775696
4.65384615384615 0.597866654396057
4.93589743589744 0.583842992782593
5.23717948717949 0.584210932254791
5.55769230769231 0.552587509155273
5.8974358974359 0.506261110305786
6.25641025641026 0.520394265651703
6.64102564102564 0.537773966789246
7.04487179487179 0.520617306232452
7.47435897435897 0.540018141269684
7.92948717948718 0.503455877304077
8.41025641025641 0.514251112937927
8.92307692307692 0.48879611492157
9.46794871794872 0.497729331254959
10.0448717948718 0.523116052150726
10.6602564102564 0.484785884618759
11.3076923076923 0.49144035577774
12 0.474914342164993
12.7307692307692 0.46324098110199
13.5064102564103 0.476446509361267
14.3333333333333 0.46183505654335
15.2051282051282 0.453023701906204
16.1346153846154 0.46035560965538
17.1153846153846 0.459127008914948
18.1602564102564 0.441047549247742
19.2692307692308 0.451670557260513
20.4423076923077 0.43457442522049
21.6858974358974 0.427996546030045
23.0128205128205 0.432394027709961
24.4102564102564 0.424982249736786
25.9038461538462 0.431481659412384
27.4807692307692 0.407999724149704
29.1538461538462 0.429335445165634
30.9358974358974 0.403846472501755
32.8205128205128 0.422821402549744
34.8205128205128 0.408061891794205
36.9423076923077 0.399457842111588
39.1923076923077 0.400705367326736
41.5833333333333 0.380907475948334
44.1153846153846 0.376779437065125
46.8076923076923 0.359467446804047
49.6602564102564 0.380327701568604
52.6858974358974 0.367314964532852
55.8974358974359 0.369947612285614
59.3076923076923 0.370582491159439
62.9230769230769 0.37256121635437
66.7564102564103 0.356985151767731
70.8269230769231 0.364017367362976
75.1474358974359 0.353927135467529
79.7307692307692 0.367917776107788
84.5897435897436 0.360410541296005
89.7435897435897 0.356604009866714
95.2179487179487 0.36050209403038
101.019230769231 0.358089596033096
107.179487179487 0.35248926281929
113.711538461538 0.353375554084778
120.641025641026 0.362716823816299
127.99358974359 0.355164170265198
135.801282051282 0.356171786785126
144.076923076923 0.353104621171951
152.858974358974 0.346524953842163
162.179487179487 0.362506091594696
172.064102564103 0.346590429544449
182.551282051282 0.353742808103561
193.679487179487 0.357135683298111
205.487179487179 0.357069909572601
218.012820512821 0.34849289059639
231.301282051282 0.348036646842957
245.403846153846 0.364650726318359
260.358974358974 0.342706739902496
276.230769230769 0.353538274765015
293.070512820513 0.349445283412933
310.935897435897 0.368811339139938
329.884615384615 0.35683599114418
350 0.345701068639755
};
\addplot [, color1, opacity=0.6, mark=square*, mark size=0.5, mark options={solid}, only marks, forget plot]
table {%
1 0.80811870098114
1.05769230769231 0.873037219047546
1.12179487179487 0.880240738391876
1.19230769230769 0.889500558376312
1.26282051282051 0.820010662078857
1.33974358974359 0.806582987308502
1.42307692307692 0.80079710483551
1.51282051282051 0.774086594581604
1.6025641025641 0.737964987754822
1.69871794871795 0.730465590953827
1.80128205128205 0.727491915225983
1.91666666666667 0.718303263187408
2.03205128205128 0.760468423366547
2.15384615384615 0.712354898452759
2.28846153846154 0.758355379104614
2.42307692307692 0.699099540710449
2.57692307692308 0.740717172622681
2.73076923076923 0.697207629680634
2.8974358974359 0.711261868476868
3.07692307692308 0.681191623210907
3.26282051282051 0.684488892555237
3.46153846153846 0.707621872425079
3.67307692307692 0.697548985481262
3.8974358974359 0.636819124221802
4.13461538461539 0.679191410541534
4.38461538461539 0.65766853094101
4.65384615384615 0.627270460128784
4.93589743589744 0.644034802913666
5.23717948717949 0.621904969215393
5.55769230769231 0.625465154647827
5.8974358974359 0.573417782783508
6.25641025641026 0.590110898017883
6.64102564102564 0.595173716545105
7.04487179487179 0.598508477210999
7.47435897435897 0.596423804759979
7.92948717948718 0.585047841072083
8.41025641025641 0.557641208171844
8.92307692307692 0.555726230144501
9.46794871794872 0.548427700996399
10.0448717948718 0.56479424238205
10.6602564102564 0.550039768218994
11.3076923076923 0.523974180221558
12 0.496235311031342
12.7307692307692 0.517973124980927
13.5064102564103 0.497068017721176
14.3333333333333 0.486099302768707
15.2051282051282 0.461683452129364
16.1346153846154 0.474702507257462
17.1153846153846 0.476893454790115
18.1602564102564 0.48224538564682
19.2692307692308 0.468663245439529
20.4423076923077 0.431084007024765
21.6858974358974 0.428247451782227
23.0128205128205 0.432371437549591
24.4102564102564 0.431298196315765
25.9038461538462 0.416121602058411
27.4807692307692 0.425631523132324
29.1538461538462 0.434808403253555
30.9358974358974 0.390754818916321
32.8205128205128 0.398065745830536
34.8205128205128 0.398777574300766
36.9423076923077 0.421386182308197
39.1923076923077 0.411867886781693
41.5833333333333 0.390919357538223
44.1153846153846 0.402580320835114
46.8076923076923 0.403248965740204
49.6602564102564 0.384721517562866
52.6858974358974 0.401494741439819
55.8974358974359 0.411366403102875
59.3076923076923 0.410230904817581
62.9230769230769 0.405577331781387
66.7564102564103 0.411673724651337
70.8269230769231 0.404240339994431
75.1474358974359 0.381450653076172
79.7307692307692 0.388042330741882
84.5897435897436 0.375384896993637
89.7435897435897 0.397134512662888
95.2179487179487 0.385438352823257
101.019230769231 0.369958639144897
107.179487179487 0.379385530948639
113.711538461538 0.382118612527847
120.641025641026 0.371915280818939
127.99358974359 0.386652737855911
135.801282051282 0.377703607082367
144.076923076923 0.373304635286331
152.858974358974 0.36449459195137
162.179487179487 0.378649026155472
172.064102564103 0.385395586490631
182.551282051282 0.376416236162186
193.679487179487 0.396370470523834
205.487179487179 0.379022240638733
218.012820512821 0.367390275001526
231.301282051282 0.358088880777359
245.403846153846 0.378076165914536
260.358974358974 0.378307610750198
276.230769230769 0.374477654695511
293.070512820513 0.393286854028702
310.935897435897 0.371841728687286
329.884615384615 0.386513143777847
350 0.378349483013153
};
\addplot [, color2, opacity=0.6, mark=diamond*, mark size=0.5, mark options={solid}, only marks]
table {%
1 0.957247197628021
1.05769230769231 0.958641469478607
1.12179487179487 0.971569061279297
1.19230769230769 0.95826780796051
1.26282051282051 0.925794661045074
1.33974358974359 0.932329833507538
1.42307692307692 0.952133774757385
1.51282051282051 0.950578153133392
1.6025641025641 0.922703444957733
1.69871794871795 0.92592865228653
1.80128205128205 0.915040671825409
1.91666666666667 0.913286745548248
2.03205128205128 0.918747246265411
2.15384615384615 0.916126310825348
2.28846153846154 0.909493446350098
2.42307692307692 0.897784411907196
2.57692307692308 0.916556835174561
2.73076923076923 0.894977986812592
2.8974358974359 0.889941930770874
3.07692307692308 0.879608750343323
3.26282051282051 0.862856507301331
3.46153846153846 0.863696098327637
3.67307692307692 0.853933334350586
3.8974358974359 0.82973575592041
4.13461538461539 0.820061922073364
4.38461538461539 0.803379952907562
4.65384615384615 0.81152355670929
4.93589743589744 0.794380605220795
5.23717948717949 0.77792090177536
5.55769230769231 0.762539207935333
5.8974358974359 0.720024466514587
6.25641025641026 0.734093844890594
6.64102564102564 0.73324066400528
7.04487179487179 0.691756665706635
7.47435897435897 0.684718549251556
7.92948717948718 0.657052159309387
8.41025641025641 0.688404083251953
8.92307692307692 0.668163537979126
9.46794871794872 0.646072268486023
10.0448717948718 0.648435473442078
10.6602564102564 0.648274779319763
11.3076923076923 0.636222898960114
12 0.595574796199799
12.7307692307692 0.622713387012482
13.5064102564103 0.603226661682129
14.3333333333333 0.617555618286133
15.2051282051282 0.582461476325989
16.1346153846154 0.598486602306366
17.1153846153846 0.57739794254303
18.1602564102564 0.584509432315826
19.2692307692308 0.567200899124146
20.4423076923077 0.553277373313904
21.6858974358974 0.557111978530884
23.0128205128205 0.566354691982269
24.4102564102564 0.549098789691925
25.9038461538462 0.530691742897034
27.4807692307692 0.545854985713959
29.1538461538462 0.530861735343933
30.9358974358974 0.551056146621704
32.8205128205128 0.549709141254425
34.8205128205128 0.534467756748199
36.9423076923077 0.522005438804626
39.1923076923077 0.511753737926483
41.5833333333333 0.497836977243423
44.1153846153846 0.508043766021729
46.8076923076923 0.505572855472565
49.6602564102564 0.493503481149673
52.6858974358974 0.496477395296097
55.8974358974359 0.492968410253525
59.3076923076923 0.483520805835724
62.9230769230769 0.485338479280472
66.7564102564103 0.481070011854172
70.8269230769231 0.490495443344116
75.1474358974359 0.477241545915604
79.7307692307692 0.489980101585388
84.5897435897436 0.475008606910706
89.7435897435897 0.461418002843857
95.2179487179487 0.466919004917145
101.019230769231 0.462794631719589
107.179487179487 0.458638817071915
113.711538461538 0.458325982093811
120.641025641026 0.455720663070679
127.99358974359 0.45380225777626
135.801282051282 0.443631649017334
144.076923076923 0.44856783747673
152.858974358974 0.441827148199081
162.179487179487 0.448420882225037
172.064102564103 0.445194154977798
182.551282051282 0.430723130702972
193.679487179487 0.446466356515884
205.487179487179 0.440953522920609
218.012820512821 0.446818619966507
231.301282051282 0.441900700330734
245.403846153846 0.436202943325043
260.358974358974 0.454324632883072
276.230769230769 0.447250425815582
293.070512820513 0.436865925788879
310.935897435897 0.441131263971329
329.884615384615 0.446096181869507
350 0.435405343770981
};
\addlegendentry{mb 32, exact}
\addplot [, color2, opacity=0.6, mark=diamond*, mark size=0.5, mark options={solid}, only marks, forget plot]
table {%
1 0.948967635631561
1.05769230769231 0.945995151996613
1.12179487179487 0.961746394634247
1.19230769230769 0.942927777767181
1.26282051282051 0.893275439739227
1.33974358974359 0.8944331407547
1.42307692307692 0.892670750617981
1.51282051282051 0.876900851726532
1.6025641025641 0.865440905094147
1.69871794871795 0.846083045005798
1.80128205128205 0.832872748374939
1.91666666666667 0.853930950164795
2.03205128205128 0.866246342658997
2.15384615384615 0.810130000114441
2.28846153846154 0.842374384403229
2.42307692307692 0.837665617465973
2.57692307692308 0.850787281990051
2.73076923076923 0.829951286315918
2.8974358974359 0.830089151859283
3.07692307692308 0.802658081054688
3.26282051282051 0.810402154922485
3.46153846153846 0.812469601631165
3.67307692307692 0.814545929431915
3.8974358974359 0.797027707099915
4.13461538461539 0.818503797054291
4.38461538461539 0.794745147228241
4.65384615384615 0.799766540527344
4.93589743589744 0.778900444507599
5.23717948717949 0.794259905815125
5.55769230769231 0.769133687019348
5.8974358974359 0.710263133049011
6.25641025641026 0.710600852966309
6.64102564102564 0.74707305431366
7.04487179487179 0.718452453613281
7.47435897435897 0.7019362449646
7.92948717948718 0.67710280418396
8.41025641025641 0.689832985401154
8.92307692307692 0.653680384159088
9.46794871794872 0.671520531177521
10.0448717948718 0.676909029483795
10.6602564102564 0.648759245872498
11.3076923076923 0.641396760940552
12 0.631149113178253
12.7307692307692 0.623247683048248
13.5064102564103 0.599981307983398
14.3333333333333 0.604866087436676
15.2051282051282 0.588473737239838
16.1346153846154 0.597627580165863
17.1153846153846 0.570838868618011
18.1602564102564 0.576996982097626
19.2692307692308 0.566210746765137
20.4423076923077 0.554212033748627
21.6858974358974 0.542788982391357
23.0128205128205 0.553005635738373
24.4102564102564 0.565857410430908
25.9038461538462 0.546138882637024
27.4807692307692 0.551911592483521
29.1538461538462 0.541604399681091
30.9358974358974 0.536496698856354
32.8205128205128 0.53823447227478
34.8205128205128 0.521817684173584
36.9423076923077 0.52770721912384
39.1923076923077 0.514692008495331
41.5833333333333 0.503147721290588
44.1153846153846 0.50717031955719
46.8076923076923 0.515039801597595
49.6602564102564 0.474790245294571
52.6858974358974 0.479489237070084
55.8974358974359 0.484399020671844
59.3076923076923 0.466029971837997
62.9230769230769 0.463033974170685
66.7564102564103 0.449972987174988
70.8269230769231 0.471254646778107
75.1474358974359 0.464938014745712
79.7307692307692 0.456620067358017
84.5897435897436 0.439804404973984
89.7435897435897 0.428655743598938
95.2179487179487 0.439142018556595
101.019230769231 0.444456398487091
107.179487179487 0.432361721992493
113.711538461538 0.434628516435623
120.641025641026 0.432763636112213
127.99358974359 0.444527745246887
135.801282051282 0.412983477115631
144.076923076923 0.410042107105255
152.858974358974 0.434408873319626
162.179487179487 0.425993502140045
172.064102564103 0.411725848913193
182.551282051282 0.417330473661423
193.679487179487 0.424827694892883
205.487179487179 0.431436538696289
218.012820512821 0.429216325283051
231.301282051282 0.405819475650787
245.403846153846 0.424686372280121
260.358974358974 0.411213457584381
276.230769230769 0.419592648744583
293.070512820513 0.418808549642563
310.935897435897 0.420681178569794
329.884615384615 0.418580323457718
350 0.429652690887451
};
\addplot [, color2, opacity=0.6, mark=diamond*, mark size=0.5, mark options={solid}, only marks, forget plot]
table {%
1 0.943355679512024
1.05769230769231 0.947238087654114
1.12179487179487 0.965274930000305
1.19230769230769 0.952735424041748
1.26282051282051 0.924333035945892
1.33974358974359 0.8998903632164
1.42307692307692 0.882811844348907
1.51282051282051 0.872307419776917
1.6025641025641 0.902672350406647
1.69871794871795 0.894188642501831
1.80128205128205 0.875541508197784
1.91666666666667 0.886324167251587
2.03205128205128 0.914751172065735
2.15384615384615 0.888162195682526
2.28846153846154 0.888289630413055
2.42307692307692 0.875966906547546
2.57692307692308 0.901877820491791
2.73076923076923 0.872124791145325
2.8974358974359 0.862595856189728
3.07692307692308 0.848294794559479
3.26282051282051 0.854764223098755
3.46153846153846 0.837902069091797
3.67307692307692 0.843992948532104
3.8974358974359 0.836562633514404
4.13461538461539 0.817032277584076
4.38461538461539 0.804579019546509
4.65384615384615 0.80912184715271
4.93589743589744 0.802739262580872
5.23717948717949 0.798364877700806
5.55769230769231 0.794657588005066
5.8974358974359 0.739554405212402
6.25641025641026 0.748320996761322
6.64102564102564 0.757598996162415
7.04487179487179 0.733508288860321
7.47435897435897 0.728272140026093
7.92948717948718 0.721473217010498
8.41025641025641 0.722909808158875
8.92307692307692 0.676788449287415
9.46794871794872 0.690992951393127
10.0448717948718 0.699740886688232
10.6602564102564 0.67982280254364
11.3076923076923 0.66518497467041
12 0.631672859191895
12.7307692307692 0.662704467773438
13.5064102564103 0.628847539424896
14.3333333333333 0.617759764194489
15.2051282051282 0.605762004852295
16.1346153846154 0.607848286628723
17.1153846153846 0.606723070144653
18.1602564102564 0.614624381065369
19.2692307692308 0.583165287971497
20.4423076923077 0.578153610229492
21.6858974358974 0.57830536365509
23.0128205128205 0.56661057472229
24.4102564102564 0.573403239250183
25.9038461538462 0.569792687892914
27.4807692307692 0.564299941062927
29.1538461538462 0.556440591812134
30.9358974358974 0.564683079719543
32.8205128205128 0.554552018642426
34.8205128205128 0.561120808124542
36.9423076923077 0.547329425811768
39.1923076923077 0.540001332759857
41.5833333333333 0.541527032852173
44.1153846153846 0.524877607822418
46.8076923076923 0.523096144199371
49.6602564102564 0.505370318889618
52.6858974358974 0.514018893241882
55.8974358974359 0.512808442115784
59.3076923076923 0.496724605560303
62.9230769230769 0.513305246829987
66.7564102564103 0.495147913694382
70.8269230769231 0.493164747953415
75.1474358974359 0.482689648866653
79.7307692307692 0.49640029668808
84.5897435897436 0.48292949795723
89.7435897435897 0.466151118278503
95.2179487179487 0.462518334388733
101.019230769231 0.472751647233963
107.179487179487 0.460770964622498
113.711538461538 0.464716702699661
120.641025641026 0.467133790254593
127.99358974359 0.456517845392227
135.801282051282 0.449936479330063
144.076923076923 0.453638821840286
152.858974358974 0.460504055023193
162.179487179487 0.430441647768021
172.064102564103 0.451584935188293
182.551282051282 0.450013101100922
193.679487179487 0.4354048371315
205.487179487179 0.435002744197845
218.012820512821 0.435994982719421
231.301282051282 0.439974427223206
245.403846153846 0.439825356006622
260.358974358974 0.438822150230408
276.230769230769 0.444935947656631
293.070512820513 0.449500560760498
310.935897435897 0.451501995325089
329.884615384615 0.428129881620407
350 0.419978320598602
};
\addplot [, color2, opacity=0.6, mark=diamond*, mark size=0.5, mark options={solid}, only marks, forget plot]
table {%
1 0.955471158027649
1.05769230769231 0.969539821147919
1.12179487179487 0.976379990577698
1.19230769230769 0.975469946861267
1.26282051282051 0.951690971851349
1.33974358974359 0.942441821098328
1.42307692307692 0.931137204170227
1.51282051282051 0.918174862861633
1.6025641025641 0.897397458553314
1.69871794871795 0.899433553218842
1.80128205128205 0.874198496341705
1.91666666666667 0.850633323192596
2.03205128205128 0.890804171562195
2.15384615384615 0.8834068775177
2.28846153846154 0.885733783245087
2.42307692307692 0.86816680431366
2.57692307692308 0.863661229610443
2.73076923076923 0.868464469909668
2.8974358974359 0.841334044933319
3.07692307692308 0.849199414253235
3.26282051282051 0.833263456821442
3.46153846153846 0.828640639781952
3.67307692307692 0.831869006156921
3.8974358974359 0.802628695964813
4.13461538461539 0.795897960662842
4.38461538461539 0.77028089761734
4.65384615384615 0.767873644828796
4.93589743589744 0.763194262981415
5.23717948717949 0.759284019470215
5.55769230769231 0.752625167369843
5.8974358974359 0.726304173469543
6.25641025641026 0.720137774944305
6.64102564102564 0.726788461208344
7.04487179487179 0.705230951309204
7.47435897435897 0.688909351825714
7.92948717948718 0.68192994594574
8.41025641025641 0.673000991344452
8.92307692307692 0.645807266235352
9.46794871794872 0.658748030662537
10.0448717948718 0.665299832820892
10.6602564102564 0.64666211605072
11.3076923076923 0.645703732967377
12 0.628106296062469
12.7307692307692 0.625202834606171
13.5064102564103 0.60641086101532
14.3333333333333 0.608493387699127
15.2051282051282 0.593195736408234
16.1346153846154 0.581811487674713
17.1153846153846 0.589970827102661
18.1602564102564 0.57656329870224
19.2692307692308 0.559966504573822
20.4423076923077 0.549919188022614
21.6858974358974 0.543627679347992
23.0128205128205 0.549831330776215
24.4102564102564 0.542087137699127
25.9038461538462 0.537644743919373
27.4807692307692 0.533233225345612
29.1538461538462 0.530562996864319
30.9358974358974 0.526451349258423
32.8205128205128 0.520977318286896
34.8205128205128 0.512134373188019
36.9423076923077 0.513892531394958
39.1923076923077 0.521334648132324
41.5833333333333 0.507445573806763
44.1153846153846 0.506744503974915
46.8076923076923 0.512688457965851
49.6602564102564 0.488736480474472
52.6858974358974 0.498850226402283
55.8974358974359 0.486457824707031
59.3076923076923 0.484321057796478
62.9230769230769 0.485508531332016
66.7564102564103 0.484507411718369
70.8269230769231 0.477752357721329
75.1474358974359 0.469966202974319
79.7307692307692 0.464920192956924
84.5897435897436 0.466722518205643
89.7435897435897 0.454683989286423
95.2179487179487 0.462904959917068
101.019230769231 0.447167158126831
107.179487179487 0.447170794010162
113.711538461538 0.450070112943649
120.641025641026 0.455027639865875
127.99358974359 0.445994168519974
135.801282051282 0.435298532247543
144.076923076923 0.444675177335739
152.858974358974 0.446683883666992
162.179487179487 0.441957086324692
172.064102564103 0.446005403995514
182.551282051282 0.439605563879013
193.679487179487 0.448561459779739
205.487179487179 0.436361312866211
218.012820512821 0.442899644374847
231.301282051282 0.435806572437286
245.403846153846 0.460106611251831
260.358974358974 0.453348457813263
276.230769230769 0.450364291667938
293.070512820513 0.445328384637833
310.935897435897 0.466595530509949
329.884615384615 0.4608553647995
350 0.443741977214813
};
\addplot [, color2, opacity=0.6, mark=diamond*, mark size=0.5, mark options={solid}, only marks, forget plot]
table {%
1 0.953535199165344
1.05769230769231 0.946636497974396
1.12179487179487 0.918208241462708
1.19230769230769 0.908063769340515
1.26282051282051 0.914648413658142
1.33974358974359 0.901265799999237
1.42307692307692 0.901818215847015
1.51282051282051 0.917825043201447
1.6025641025641 0.88136488199234
1.69871794871795 0.855189979076385
1.80128205128205 0.850403726100922
1.91666666666667 0.795339465141296
2.03205128205128 0.812125325202942
2.15384615384615 0.804018259048462
2.28846153846154 0.810548067092896
2.42307692307692 0.776688992977142
2.57692307692308 0.811104416847229
2.73076923076923 0.813039541244507
2.8974358974359 0.785120606422424
3.07692307692308 0.786777079105377
3.26282051282051 0.801649451255798
3.46153846153846 0.809235334396362
3.67307692307692 0.803907930850983
3.8974358974359 0.784770011901855
4.13461538461539 0.809520602226257
4.38461538461539 0.758967578411102
4.65384615384615 0.777616322040558
4.93589743589744 0.779416084289551
5.23717948717949 0.783508658409119
5.55769230769231 0.730127692222595
5.8974358974359 0.712226092815399
6.25641025641026 0.72767186164856
6.64102564102564 0.71076375246048
7.04487179487179 0.71121084690094
7.47435897435897 0.694158434867859
7.92948717948718 0.658790051937103
8.41025641025641 0.677513539791107
8.92307692307692 0.661059021949768
9.46794871794872 0.681021571159363
10.0448717948718 0.669405400753021
10.6602564102564 0.667059302330017
11.3076923076923 0.633362233638763
12 0.620015621185303
12.7307692307692 0.624139368534088
13.5064102564103 0.619312405586243
14.3333333333333 0.591809272766113
15.2051282051282 0.595150649547577
16.1346153846154 0.590304672718048
17.1153846153846 0.576550543308258
18.1602564102564 0.56677120923996
19.2692307692308 0.55789452791214
20.4423076923077 0.546545445919037
21.6858974358974 0.539927363395691
23.0128205128205 0.546612441539764
24.4102564102564 0.542432308197021
25.9038461538462 0.536222577095032
27.4807692307692 0.54590505361557
29.1538461538462 0.541542589664459
30.9358974358974 0.516621828079224
32.8205128205128 0.541372716426849
34.8205128205128 0.530688345432281
36.9423076923077 0.527234673500061
39.1923076923077 0.511830270290375
41.5833333333333 0.504635095596313
44.1153846153846 0.50698709487915
46.8076923076923 0.522443056106567
49.6602564102564 0.481420040130615
52.6858974358974 0.497375220060349
55.8974358974359 0.503079414367676
59.3076923076923 0.491553246974945
62.9230769230769 0.481639236211777
66.7564102564103 0.481461733579636
70.8269230769231 0.472662419080734
75.1474358974359 0.471924304962158
79.7307692307692 0.479990154504776
84.5897435897436 0.475828856229782
89.7435897435897 0.463929653167725
95.2179487179487 0.472211927175522
101.019230769231 0.471648633480072
107.179487179487 0.45773458480835
113.711538461538 0.468104779720306
120.641025641026 0.463011831045151
127.99358974359 0.455894500017166
135.801282051282 0.447931498289108
144.076923076923 0.453033596277237
152.858974358974 0.4537453353405
162.179487179487 0.441115021705627
172.064102564103 0.447057127952576
182.551282051282 0.444020688533783
193.679487179487 0.45335990190506
205.487179487179 0.438271462917328
218.012820512821 0.442009806632996
231.301282051282 0.431539535522461
245.403846153846 0.444379180669785
260.358974358974 0.449849277734756
276.230769230769 0.434175938367844
293.070512820513 0.428645849227905
310.935897435897 0.440616279840469
329.884615384615 0.435294598340988
350 0.41680908203125
};
\addplot [, black, opacity=0.6, mark=*, mark size=0.5, mark options={solid}, only marks]
table {%
1 0.975527107715607
1.05769230769231 0.975879549980164
1.12179487179487 0.981298506259918
1.19230769230769 0.982291400432587
1.26282051282051 0.978530049324036
1.33974358974359 0.946166694164276
1.42307692307692 0.956062853336334
1.51282051282051 0.965766727924347
1.6025641025641 0.950586199760437
1.69871794871795 0.941674172878265
1.80128205128205 0.948362469673157
1.91666666666667 0.958436846733093
2.03205128205128 0.95274555683136
2.15384615384615 0.95756071805954
2.28846153846154 0.95621246099472
2.42307692307692 0.943868815898895
2.57692307692308 0.947786211967468
2.73076923076923 0.940457284450531
2.8974358974359 0.9387167096138
3.07692307692308 0.935720503330231
3.26282051282051 0.930732607841492
3.46153846153846 0.935683131217957
3.67307692307692 0.922274708747864
3.8974358974359 0.906444251537323
4.13461538461539 0.923512697219849
4.38461538461539 0.909830629825592
4.65384615384615 0.90929913520813
4.93589743589744 0.903362929821014
5.23717948717949 0.890452861785889
5.55769230769231 0.901482343673706
5.8974358974359 0.867946922779083
6.25641025641026 0.871417045593262
6.64102564102564 0.888647139072418
7.04487179487179 0.858071565628052
7.47435897435897 0.844823122024536
7.92948717948718 0.845068573951721
8.41025641025641 0.848812699317932
8.92307692307692 0.842553675174713
9.46794871794872 0.824855804443359
10.0448717948718 0.827118337154388
10.6602564102564 0.796650886535645
11.3076923076923 0.783909440040588
12 0.777671635150909
12.7307692307692 0.79072368144989
13.5064102564103 0.768704414367676
14.3333333333333 0.750085711479187
15.2051282051282 0.738444328308105
16.1346153846154 0.746802031993866
17.1153846153846 0.728871166706085
18.1602564102564 0.703183114528656
19.2692307692308 0.718234360218048
20.4423076923077 0.687351882457733
21.6858974358974 0.681278228759766
23.0128205128205 0.686552882194519
24.4102564102564 0.693485677242279
25.9038461538462 0.672252476215363
27.4807692307692 0.678914904594421
29.1538461538462 0.68155300617218
30.9358974358974 0.665793240070343
32.8205128205128 0.675159156322479
34.8205128205128 0.663778305053711
36.9423076923077 0.642848789691925
39.1923076923077 0.6383296251297
41.5833333333333 0.627930223941803
44.1153846153846 0.629244148731232
46.8076923076923 0.627208113670349
49.6602564102564 0.603394746780396
52.6858974358974 0.61290717124939
55.8974358974359 0.604208409786224
59.3076923076923 0.592757701873779
62.9230769230769 0.599962413311005
66.7564102564103 0.581532061100006
70.8269230769231 0.592190504074097
75.1474358974359 0.560584306716919
79.7307692307692 0.572852551937103
84.5897435897436 0.544415891170502
89.7435897435897 0.532862365245819
95.2179487179487 0.542996346950531
101.019230769231 0.527527093887329
107.179487179487 0.527780294418335
113.711538461538 0.52032059431076
120.641025641026 0.518957257270813
127.99358974359 0.513621747493744
135.801282051282 0.529812097549438
144.076923076923 0.495187819004059
152.858974358974 0.514414191246033
162.179487179487 0.506864964962006
172.064102564103 0.496896624565125
182.551282051282 0.49849408864975
193.679487179487 0.487976759672165
205.487179487179 0.496980279684067
218.012820512821 0.502931535243988
231.301282051282 0.493928521871567
245.403846153846 0.49935507774353
260.358974358974 0.510318458080292
276.230769230769 0.492345869541168
293.070512820513 0.490240544080734
310.935897435897 0.497594207525253
329.884615384615 0.485403895378113
350 0.489300459623337
};
\addlegendentry{mb 128, exact}
\addplot [, black, opacity=0.6, mark=*, mark size=0.5, mark options={solid}, only marks, forget plot]
table {%
1 0.98547375202179
1.05769230769231 0.981072545051575
1.12179487179487 0.987987518310547
1.19230769230769 0.986805498600006
1.26282051282051 0.973555445671082
1.33974358974359 0.972347259521484
1.42307692307692 0.974495828151703
1.51282051282051 0.969120740890503
1.6025641025641 0.957382321357727
1.69871794871795 0.939426422119141
1.80128205128205 0.943614780902863
1.91666666666667 0.929763615131378
2.03205128205128 0.944261610507965
2.15384615384615 0.940847814083099
2.28846153846154 0.940940856933594
2.42307692307692 0.939785301685333
2.57692307692308 0.945573806762695
2.73076923076923 0.924932181835175
2.8974358974359 0.938235580921173
3.07692307692308 0.922984421253204
3.26282051282051 0.910995781421661
3.46153846153846 0.924165487289429
3.67307692307692 0.931661665439606
3.8974358974359 0.906217157840729
4.13461538461539 0.924329936504364
4.38461538461539 0.912600755691528
4.65384615384615 0.88957667350769
4.93589743589744 0.888119637966156
5.23717948717949 0.883500754833221
5.55769230769231 0.889799833297729
5.8974358974359 0.860232889652252
6.25641025641026 0.877396762371063
6.64102564102564 0.852205157279968
7.04487179487179 0.855269730091095
7.47435897435897 0.848165273666382
7.92948717948718 0.835215449333191
8.41025641025641 0.847373187541962
8.92307692307692 0.812402546405792
9.46794871794872 0.801014244556427
10.0448717948718 0.800452411174774
10.6602564102564 0.804883122444153
11.3076923076923 0.774041593074799
12 0.769234597682953
12.7307692307692 0.782009124755859
13.5064102564103 0.741140961647034
14.3333333333333 0.742492437362671
15.2051282051282 0.726064682006836
16.1346153846154 0.725980818271637
17.1153846153846 0.721560418605804
18.1602564102564 0.741487383842468
19.2692307692308 0.718854188919067
20.4423076923077 0.692035555839539
21.6858974358974 0.696824908256531
23.0128205128205 0.707006335258484
24.4102564102564 0.711516559123993
25.9038461538462 0.680367708206177
27.4807692307692 0.703445851802826
29.1538461538462 0.681106686592102
30.9358974358974 0.662260472774506
32.8205128205128 0.659499168395996
34.8205128205128 0.657371401786804
36.9423076923077 0.652992606163025
39.1923076923077 0.655940234661102
41.5833333333333 0.64131772518158
44.1153846153846 0.638317108154297
46.8076923076923 0.630138874053955
49.6602564102564 0.627388834953308
52.6858974358974 0.629440009593964
55.8974358974359 0.612715542316437
59.3076923076923 0.605165004730225
62.9230769230769 0.613393068313599
66.7564102564103 0.61031037569046
70.8269230769231 0.587878942489624
75.1474358974359 0.595539748668671
79.7307692307692 0.586910665035248
84.5897435897436 0.5835240483284
89.7435897435897 0.555045783519745
95.2179487179487 0.550371527671814
101.019230769231 0.556595504283905
107.179487179487 0.548309564590454
113.711538461538 0.548957645893097
120.641025641026 0.539990663528442
127.99358974359 0.544257640838623
135.801282051282 0.533892750740051
144.076923076923 0.53219997882843
152.858974358974 0.527959644794464
162.179487179487 0.545309007167816
172.064102564103 0.525468349456787
182.551282051282 0.541029930114746
193.679487179487 0.531932771205902
205.487179487179 0.512557685375214
218.012820512821 0.514129638671875
231.301282051282 0.497415691614151
245.403846153846 0.502278208732605
260.358974358974 0.498234361410141
276.230769230769 0.506744742393494
293.070512820513 0.506226360797882
310.935897435897 0.513335585594177
329.884615384615 0.501690149307251
350 0.49129256606102
};
\addplot [, black, opacity=0.6, mark=*, mark size=0.5, mark options={solid}, only marks, forget plot]
table {%
1 0.982185363769531
1.05769230769231 0.981845617294312
1.12179487179487 0.990415930747986
1.19230769230769 0.986113667488098
1.26282051282051 0.963619351387024
1.33974358974359 0.969328045845032
1.42307692307692 0.971268832683563
1.51282051282051 0.975955128669739
1.6025641025641 0.957674026489258
1.69871794871795 0.946043431758881
1.80128205128205 0.947996973991394
1.91666666666667 0.949277698993683
2.03205128205128 0.957247734069824
2.15384615384615 0.960746586322784
2.28846153846154 0.945977389812469
2.42307692307692 0.939614057540894
2.57692307692308 0.953782200813293
2.73076923076923 0.95204496383667
2.8974358974359 0.940797686576843
3.07692307692308 0.93306964635849
3.26282051282051 0.939587354660034
3.46153846153846 0.932318806648254
3.67307692307692 0.933220088481903
3.8974358974359 0.923075020313263
4.13461538461539 0.9228236079216
4.38461538461539 0.910681128501892
4.65384615384615 0.903499186038971
4.93589743589744 0.901083648204803
5.23717948717949 0.901457369327545
5.55769230769231 0.896666765213013
5.8974358974359 0.868629276752472
6.25641025641026 0.887513875961304
6.64102564102564 0.873347759246826
7.04487179487179 0.879052877426147
7.47435897435897 0.867180168628693
7.92948717948718 0.84938383102417
8.41025641025641 0.853730916976929
8.92307692307692 0.833457827568054
9.46794871794872 0.830531001091003
10.0448717948718 0.827811479568481
10.6602564102564 0.823979020118713
11.3076923076923 0.800252377986908
12 0.785353362560272
12.7307692307692 0.791669547557831
13.5064102564103 0.766777992248535
14.3333333333333 0.770790755748749
15.2051282051282 0.753612399101257
16.1346153846154 0.769104897975922
17.1153846153846 0.748502969741821
18.1602564102564 0.752066016197205
19.2692307692308 0.740846395492554
20.4423076923077 0.719961524009705
21.6858974358974 0.717281639575958
23.0128205128205 0.725273787975311
24.4102564102564 0.728144228458405
25.9038461538462 0.690377056598663
27.4807692307692 0.704762756824493
29.1538461538462 0.687010943889618
30.9358974358974 0.695802569389343
32.8205128205128 0.702136516571045
34.8205128205128 0.697620987892151
36.9423076923077 0.685297667980194
39.1923076923077 0.659804046154022
41.5833333333333 0.666839301586151
44.1153846153846 0.654598534107208
46.8076923076923 0.647260546684265
49.6602564102564 0.631142258644104
52.6858974358974 0.631148099899292
55.8974358974359 0.630903959274292
59.3076923076923 0.625231862068176
62.9230769230769 0.633013844490051
66.7564102564103 0.61236035823822
70.8269230769231 0.60189950466156
75.1474358974359 0.584398567676544
79.7307692307692 0.59369945526123
84.5897435897436 0.572606801986694
89.7435897435897 0.565208554267883
95.2179487179487 0.57346510887146
101.019230769231 0.549895286560059
107.179487179487 0.554142653942108
113.711538461538 0.559076189994812
120.641025641026 0.545779228210449
127.99358974359 0.556437730789185
135.801282051282 0.527416288852692
144.076923076923 0.544892251491547
152.858974358974 0.529297471046448
162.179487179487 0.539228141307831
172.064102564103 0.527096748352051
182.551282051282 0.514154672622681
193.679487179487 0.517471551895142
205.487179487179 0.505025029182434
218.012820512821 0.531710922718048
231.301282051282 0.517846047878265
245.403846153846 0.516850769519806
260.358974358974 0.496802896261215
276.230769230769 0.504508435726166
293.070512820513 0.490358412265778
310.935897435897 0.519662916660309
329.884615384615 0.498074859380722
350 0.49126210808754
};
\addplot [, black, opacity=0.6, mark=*, mark size=0.5, mark options={solid}, only marks, forget plot]
table {%
1 0.973074793815613
1.05769230769231 0.974036633968353
1.12179487179487 0.974056363105774
1.19230769230769 0.971284151077271
1.26282051282051 0.951432943344116
1.33974358974359 0.95293253660202
1.42307692307692 0.962950706481934
1.51282051282051 0.9568110704422
1.6025641025641 0.924650609493256
1.69871794871795 0.925031006336212
1.80128205128205 0.918965041637421
1.91666666666667 0.919165015220642
2.03205128205128 0.939877986907959
2.15384615384615 0.913984358310699
2.28846153846154 0.93136078119278
2.42307692307692 0.931186258792877
2.57692307692308 0.946491837501526
2.73076923076923 0.92075377702713
2.8974358974359 0.927034735679626
3.07692307692308 0.91768354177475
3.26282051282051 0.922376692295074
3.46153846153846 0.929089963436127
3.67307692307692 0.919421851634979
3.8974358974359 0.911060929298401
4.13461538461539 0.902861893177032
4.38461538461539 0.900286376476288
4.65384615384615 0.909771084785461
4.93589743589744 0.893032371997833
5.23717948717949 0.889559745788574
5.55769230769231 0.887795984745026
5.8974358974359 0.878243386745453
6.25641025641026 0.860329747200012
6.64102564102564 0.867230892181396
7.04487179487179 0.849577248096466
7.47435897435897 0.848664045333862
7.92948717948718 0.83221822977066
8.41025641025641 0.8178591132164
8.92307692307692 0.821105778217316
9.46794871794872 0.826702833175659
10.0448717948718 0.81689465045929
10.6602564102564 0.822579562664032
11.3076923076923 0.786180853843689
12 0.765198647975922
12.7307692307692 0.783441424369812
13.5064102564103 0.771246492862701
14.3333333333333 0.771606028079987
15.2051282051282 0.752801954746246
16.1346153846154 0.764906585216522
17.1153846153846 0.750824630260468
18.1602564102564 0.754004716873169
19.2692307692308 0.748698711395264
20.4423076923077 0.715615570545197
21.6858974358974 0.707802593708038
23.0128205128205 0.704050898551941
24.4102564102564 0.707008183002472
25.9038461538462 0.696439504623413
27.4807692307692 0.702544450759888
29.1538461538462 0.69283539056778
30.9358974358974 0.696598172187805
32.8205128205128 0.671212911605835
34.8205128205128 0.685897588729858
36.9423076923077 0.664141118526459
39.1923076923077 0.65936553478241
41.5833333333333 0.648882567882538
44.1153846153846 0.639372527599335
46.8076923076923 0.62833309173584
49.6602564102564 0.614473044872284
52.6858974358974 0.621728360652924
55.8974358974359 0.614195466041565
59.3076923076923 0.623970746994019
62.9230769230769 0.618123412132263
66.7564102564103 0.612947583198547
70.8269230769231 0.596266329288483
75.1474358974359 0.589711904525757
79.7307692307692 0.578270316123962
84.5897435897436 0.564283132553101
89.7435897435897 0.557705223560333
95.2179487179487 0.566841125488281
101.019230769231 0.54389101266861
107.179487179487 0.556021630764008
113.711538461538 0.554065048694611
120.641025641026 0.547782599925995
127.99358974359 0.539099514484406
135.801282051282 0.521797478199005
144.076923076923 0.527629673480988
152.858974358974 0.520894110202789
162.179487179487 0.535523056983948
172.064102564103 0.514408230781555
182.551282051282 0.506534993648529
193.679487179487 0.524326682090759
205.487179487179 0.495340406894684
218.012820512821 0.512826442718506
231.301282051282 0.492131114006042
245.403846153846 0.50511234998703
260.358974358974 0.507764875888824
276.230769230769 0.495755195617676
293.070512820513 0.50640469789505
310.935897435897 0.500794231891632
329.884615384615 0.501163244247437
350 0.48217448592186
};
\addplot [, black, opacity=0.6, mark=*, mark size=0.5, mark options={solid}, only marks, forget plot]
table {%
1 0.981819450855255
1.05769230769231 0.986995697021484
1.12179487179487 0.992421269416809
1.19230769230769 0.981902122497559
1.26282051282051 0.966894567012787
1.33974358974359 0.965654730796814
1.42307692307692 0.96755576133728
1.51282051282051 0.963389039039612
1.6025641025641 0.95392644405365
1.69871794871795 0.944283068180084
1.80128205128205 0.946841418743134
1.91666666666667 0.945733070373535
2.03205128205128 0.947330892086029
2.15384615384615 0.940836906433105
2.28846153846154 0.946786046028137
2.42307692307692 0.913325309753418
2.57692307692308 0.941345930099487
2.73076923076923 0.928341031074524
2.8974358974359 0.92276006937027
3.07692307692308 0.909208834171295
3.26282051282051 0.927972257137299
3.46153846153846 0.91912567615509
3.67307692307692 0.913002133369446
3.8974358974359 0.908333241939545
4.13461538461539 0.911212921142578
4.38461538461539 0.898937940597534
4.65384615384615 0.910302877426147
4.93589743589744 0.902023375034332
5.23717948717949 0.888281226158142
5.55769230769231 0.886549055576324
5.8974358974359 0.879764258861542
6.25641025641026 0.869912445545197
6.64102564102564 0.871969878673553
7.04487179487179 0.867350280284882
7.47435897435897 0.86539626121521
7.92948717948718 0.854749917984009
8.41025641025641 0.85090833902359
8.92307692307692 0.83866935968399
9.46794871794872 0.821592330932617
10.0448717948718 0.826583862304688
10.6602564102564 0.813197672367096
11.3076923076923 0.808744788169861
12 0.777637183666229
12.7307692307692 0.79894232749939
13.5064102564103 0.796946704387665
14.3333333333333 0.780136525630951
15.2051282051282 0.759592473506927
16.1346153846154 0.773946642875671
17.1153846153846 0.762903451919556
18.1602564102564 0.763047218322754
19.2692307692308 0.741581559181213
20.4423076923077 0.738186776638031
21.6858974358974 0.723470449447632
23.0128205128205 0.73141747713089
24.4102564102564 0.732576906681061
25.9038461538462 0.703784942626953
27.4807692307692 0.705468714237213
29.1538461538462 0.70492148399353
30.9358974358974 0.701606452465057
32.8205128205128 0.704365074634552
34.8205128205128 0.688567399978638
36.9423076923077 0.66524749994278
39.1923076923077 0.657791137695312
41.5833333333333 0.663558542728424
44.1153846153846 0.644113779067993
46.8076923076923 0.665998756885529
49.6602564102564 0.630981028079987
52.6858974358974 0.636042535305023
55.8974358974359 0.640016317367554
59.3076923076923 0.613934218883514
62.9230769230769 0.629107892513275
66.7564102564103 0.598999798297882
70.8269230769231 0.622009634971619
75.1474358974359 0.594530999660492
79.7307692307692 0.601461708545685
84.5897435897436 0.588247537612915
89.7435897435897 0.5713831782341
95.2179487179487 0.578068315982819
101.019230769231 0.573698818683624
107.179487179487 0.551535069942474
113.711538461538 0.552440881729126
120.641025641026 0.540886044502258
127.99358974359 0.545330822467804
135.801282051282 0.534851789474487
144.076923076923 0.535168528556824
152.858974358974 0.540027737617493
162.179487179487 0.520861566066742
172.064102564103 0.508939146995544
182.551282051282 0.512246787548065
193.679487179487 0.510369122028351
205.487179487179 0.506465375423431
218.012820512821 0.517984926700592
231.301282051282 0.517126619815826
245.403846153846 0.509512901306152
260.358974358974 0.50203275680542
276.230769230769 0.49760177731514
293.070512820513 0.509005784988403
310.935897435897 0.50272136926651
329.884615384615 0.496519833803177
350 0.488369286060333
};
\end{axis}

\end{tikzpicture}

      \tikzexternaldisable
    \end{minipage}\hfill
    \begin{minipage}{0.50\linewidth}
      \centering
      % defines the pgfplots style "eigspacedefault"
\pgfkeys{/pgfplots/eigspacedefault/.style={
    width=1.0\linewidth,
    height=0.6\linewidth,
    every axis plot/.append style={line width = 1.5pt},
    tick pos = left,
    ylabel near ticks,
    xlabel near ticks,
    xtick align = inside,
    ytick align = inside,
    legend cell align = left,
    legend columns = 4,
    legend pos = south east,
    legend style = {
      fill opacity = 1,
      text opacity = 1,
      font = \footnotesize,
      at={(1, 1.025)},
      anchor=south east,
      column sep=0.25cm,
    },
    legend image post style={scale=2.5},
    xticklabel style = {font = \footnotesize},
    xlabel style = {font = \footnotesize},
    axis line style = {black},
    yticklabel style = {font = \footnotesize},
    ylabel style = {font = \footnotesize},
    title style = {font = \footnotesize},
    grid = major,
    grid style = {dashed}
  }
}

\pgfkeys{/pgfplots/eigspacedefaultapp/.style={
    eigspacedefault,
    height=0.6\linewidth,
    legend columns = 2,
  }
}

\pgfkeys{/pgfplots/eigspacenolegend/.style={
    legend image post style = {scale=0},
    legend style = {
      fill opacity = 0,
      draw opacity = 0,
      text opacity = 0,
      font = \footnotesize,
      at={(1, 1.025)},
      anchor=south east,
      column sep=0.25cm,
    },
  }
}
%%% Local Variables:
%%% mode: latex
%%% TeX-master: "../../thesis"
%%% End:

      \pgfkeys{/pgfplots/zmystyle/.style={
          eigspacedefaultapp,
          eigspacenolegend,
        }}
      \tikzexternalenable
      \vspace{-6ex}
      % This file was created by tikzplotlib v0.9.7.
\begin{tikzpicture}

\definecolor{color0}{rgb}{0.274509803921569,0.6,0.564705882352941}
\definecolor{color1}{rgb}{0.870588235294118,0.623529411764706,0.0862745098039216}
\definecolor{color2}{rgb}{0.501960784313725,0.184313725490196,0.6}

\begin{axis}[
axis line style={white!10!black},
legend columns=2,
legend style={fill opacity=0.8, draw opacity=1, text opacity=1, at={(0.03,0.03)}, anchor=south west, draw=white!80!black},
log basis x={10},
tick pos=left,
xlabel={epoch (log scale)},
xmajorgrids,
xmin=0.746099240306814, xmax=469.106495613199,
xmode=log,
ylabel={overlap},
ymajorgrids,
ymin=-0.05, ymax=1.05,
zmystyle
]
\addplot [, white!10!black, dashed, forget plot]
table {%
0.746099240306814 1
469.106495613199 1
};
\addplot [, white!10!black, dashed, forget plot]
table {%
0.746099240306814 0
469.106495613199 0
};
\addplot [, black, opacity=0.6, mark=*, mark size=0.5, mark options={solid}, only marks]
table {%
1 0.975527107715607
1.05769230769231 0.975879549980164
1.12179487179487 0.981298506259918
1.19230769230769 0.982291400432587
1.26282051282051 0.978530049324036
1.33974358974359 0.946166694164276
1.42307692307692 0.956062853336334
1.51282051282051 0.965766727924347
1.6025641025641 0.950586199760437
1.69871794871795 0.941674172878265
1.80128205128205 0.948362469673157
1.91666666666667 0.958436846733093
2.03205128205128 0.95274555683136
2.15384615384615 0.95756071805954
2.28846153846154 0.95621246099472
2.42307692307692 0.943868815898895
2.57692307692308 0.947786211967468
2.73076923076923 0.940457284450531
2.8974358974359 0.9387167096138
3.07692307692308 0.935720503330231
3.26282051282051 0.930732607841492
3.46153846153846 0.935683131217957
3.67307692307692 0.922274708747864
3.8974358974359 0.906444251537323
4.13461538461539 0.923512697219849
4.38461538461539 0.909830629825592
4.65384615384615 0.90929913520813
4.93589743589744 0.903362929821014
5.23717948717949 0.890452861785889
5.55769230769231 0.901482343673706
5.8974358974359 0.867946922779083
6.25641025641026 0.871417045593262
6.64102564102564 0.888647139072418
7.04487179487179 0.858071565628052
7.47435897435897 0.844823122024536
7.92948717948718 0.845068573951721
8.41025641025641 0.848812699317932
8.92307692307692 0.842553675174713
9.46794871794872 0.824855804443359
10.0448717948718 0.827118337154388
10.6602564102564 0.796650886535645
11.3076923076923 0.783909440040588
12 0.777671635150909
12.7307692307692 0.79072368144989
13.5064102564103 0.768704414367676
14.3333333333333 0.750085711479187
15.2051282051282 0.738444328308105
16.1346153846154 0.746802031993866
17.1153846153846 0.728871166706085
18.1602564102564 0.703183114528656
19.2692307692308 0.718234360218048
20.4423076923077 0.687351882457733
21.6858974358974 0.681278228759766
23.0128205128205 0.686552882194519
24.4102564102564 0.693485677242279
25.9038461538462 0.672252476215363
27.4807692307692 0.678914904594421
29.1538461538462 0.68155300617218
30.9358974358974 0.665793240070343
32.8205128205128 0.675159156322479
34.8205128205128 0.663778305053711
36.9423076923077 0.642848789691925
39.1923076923077 0.6383296251297
41.5833333333333 0.627930223941803
44.1153846153846 0.629244148731232
46.8076923076923 0.627208113670349
49.6602564102564 0.603394746780396
52.6858974358974 0.61290717124939
55.8974358974359 0.604208409786224
59.3076923076923 0.592757701873779
62.9230769230769 0.599962413311005
66.7564102564103 0.581532061100006
70.8269230769231 0.592190504074097
75.1474358974359 0.560584306716919
79.7307692307692 0.572852551937103
84.5897435897436 0.544415891170502
89.7435897435897 0.532862365245819
95.2179487179487 0.542996346950531
101.019230769231 0.527527093887329
107.179487179487 0.527780294418335
113.711538461538 0.52032059431076
120.641025641026 0.518957257270813
127.99358974359 0.513621747493744
135.801282051282 0.529812097549438
144.076923076923 0.495187819004059
152.858974358974 0.514414191246033
162.179487179487 0.506864964962006
172.064102564103 0.496896624565125
182.551282051282 0.49849408864975
193.679487179487 0.487976759672165
205.487179487179 0.496980279684067
218.012820512821 0.502931535243988
231.301282051282 0.493928521871567
245.403846153846 0.49935507774353
260.358974358974 0.510318458080292
276.230769230769 0.492345869541168
293.070512820513 0.490240544080734
310.935897435897 0.497594207525253
329.884615384615 0.485403895378113
350 0.489300459623337
};
\addlegendentry{mb 128, exact}
\addplot [, black, opacity=0.6, mark=*, mark size=0.5, mark options={solid}, only marks, forget plot]
table {%
1 0.98547375202179
1.05769230769231 0.981072545051575
1.12179487179487 0.987987518310547
1.19230769230769 0.986805498600006
1.26282051282051 0.973555445671082
1.33974358974359 0.972347259521484
1.42307692307692 0.974495828151703
1.51282051282051 0.969120740890503
1.6025641025641 0.957382321357727
1.69871794871795 0.939426422119141
1.80128205128205 0.943614780902863
1.91666666666667 0.929763615131378
2.03205128205128 0.944261610507965
2.15384615384615 0.940847814083099
2.28846153846154 0.940940856933594
2.42307692307692 0.939785301685333
2.57692307692308 0.945573806762695
2.73076923076923 0.924932181835175
2.8974358974359 0.938235580921173
3.07692307692308 0.922984421253204
3.26282051282051 0.910995781421661
3.46153846153846 0.924165487289429
3.67307692307692 0.931661665439606
3.8974358974359 0.906217157840729
4.13461538461539 0.924329936504364
4.38461538461539 0.912600755691528
4.65384615384615 0.88957667350769
4.93589743589744 0.888119637966156
5.23717948717949 0.883500754833221
5.55769230769231 0.889799833297729
5.8974358974359 0.860232889652252
6.25641025641026 0.877396762371063
6.64102564102564 0.852205157279968
7.04487179487179 0.855269730091095
7.47435897435897 0.848165273666382
7.92948717948718 0.835215449333191
8.41025641025641 0.847373187541962
8.92307692307692 0.812402546405792
9.46794871794872 0.801014244556427
10.0448717948718 0.800452411174774
10.6602564102564 0.804883122444153
11.3076923076923 0.774041593074799
12 0.769234597682953
12.7307692307692 0.782009124755859
13.5064102564103 0.741140961647034
14.3333333333333 0.742492437362671
15.2051282051282 0.726064682006836
16.1346153846154 0.725980818271637
17.1153846153846 0.721560418605804
18.1602564102564 0.741487383842468
19.2692307692308 0.718854188919067
20.4423076923077 0.692035555839539
21.6858974358974 0.696824908256531
23.0128205128205 0.707006335258484
24.4102564102564 0.711516559123993
25.9038461538462 0.680367708206177
27.4807692307692 0.703445851802826
29.1538461538462 0.681106686592102
30.9358974358974 0.662260472774506
32.8205128205128 0.659499168395996
34.8205128205128 0.657371401786804
36.9423076923077 0.652992606163025
39.1923076923077 0.655940234661102
41.5833333333333 0.64131772518158
44.1153846153846 0.638317108154297
46.8076923076923 0.630138874053955
49.6602564102564 0.627388834953308
52.6858974358974 0.629440009593964
55.8974358974359 0.612715542316437
59.3076923076923 0.605165004730225
62.9230769230769 0.613393068313599
66.7564102564103 0.61031037569046
70.8269230769231 0.587878942489624
75.1474358974359 0.595539748668671
79.7307692307692 0.586910665035248
84.5897435897436 0.5835240483284
89.7435897435897 0.555045783519745
95.2179487179487 0.550371527671814
101.019230769231 0.556595504283905
107.179487179487 0.548309564590454
113.711538461538 0.548957645893097
120.641025641026 0.539990663528442
127.99358974359 0.544257640838623
135.801282051282 0.533892750740051
144.076923076923 0.53219997882843
152.858974358974 0.527959644794464
162.179487179487 0.545309007167816
172.064102564103 0.525468349456787
182.551282051282 0.541029930114746
193.679487179487 0.531932771205902
205.487179487179 0.512557685375214
218.012820512821 0.514129638671875
231.301282051282 0.497415691614151
245.403846153846 0.502278208732605
260.358974358974 0.498234361410141
276.230769230769 0.506744742393494
293.070512820513 0.506226360797882
310.935897435897 0.513335585594177
329.884615384615 0.501690149307251
350 0.49129256606102
};
\addplot [, black, opacity=0.6, mark=*, mark size=0.5, mark options={solid}, only marks, forget plot]
table {%
1 0.982185363769531
1.05769230769231 0.981845617294312
1.12179487179487 0.990415930747986
1.19230769230769 0.986113667488098
1.26282051282051 0.963619351387024
1.33974358974359 0.969328045845032
1.42307692307692 0.971268832683563
1.51282051282051 0.975955128669739
1.6025641025641 0.957674026489258
1.69871794871795 0.946043431758881
1.80128205128205 0.947996973991394
1.91666666666667 0.949277698993683
2.03205128205128 0.957247734069824
2.15384615384615 0.960746586322784
2.28846153846154 0.945977389812469
2.42307692307692 0.939614057540894
2.57692307692308 0.953782200813293
2.73076923076923 0.95204496383667
2.8974358974359 0.940797686576843
3.07692307692308 0.93306964635849
3.26282051282051 0.939587354660034
3.46153846153846 0.932318806648254
3.67307692307692 0.933220088481903
3.8974358974359 0.923075020313263
4.13461538461539 0.9228236079216
4.38461538461539 0.910681128501892
4.65384615384615 0.903499186038971
4.93589743589744 0.901083648204803
5.23717948717949 0.901457369327545
5.55769230769231 0.896666765213013
5.8974358974359 0.868629276752472
6.25641025641026 0.887513875961304
6.64102564102564 0.873347759246826
7.04487179487179 0.879052877426147
7.47435897435897 0.867180168628693
7.92948717948718 0.84938383102417
8.41025641025641 0.853730916976929
8.92307692307692 0.833457827568054
9.46794871794872 0.830531001091003
10.0448717948718 0.827811479568481
10.6602564102564 0.823979020118713
11.3076923076923 0.800252377986908
12 0.785353362560272
12.7307692307692 0.791669547557831
13.5064102564103 0.766777992248535
14.3333333333333 0.770790755748749
15.2051282051282 0.753612399101257
16.1346153846154 0.769104897975922
17.1153846153846 0.748502969741821
18.1602564102564 0.752066016197205
19.2692307692308 0.740846395492554
20.4423076923077 0.719961524009705
21.6858974358974 0.717281639575958
23.0128205128205 0.725273787975311
24.4102564102564 0.728144228458405
25.9038461538462 0.690377056598663
27.4807692307692 0.704762756824493
29.1538461538462 0.687010943889618
30.9358974358974 0.695802569389343
32.8205128205128 0.702136516571045
34.8205128205128 0.697620987892151
36.9423076923077 0.685297667980194
39.1923076923077 0.659804046154022
41.5833333333333 0.666839301586151
44.1153846153846 0.654598534107208
46.8076923076923 0.647260546684265
49.6602564102564 0.631142258644104
52.6858974358974 0.631148099899292
55.8974358974359 0.630903959274292
59.3076923076923 0.625231862068176
62.9230769230769 0.633013844490051
66.7564102564103 0.61236035823822
70.8269230769231 0.60189950466156
75.1474358974359 0.584398567676544
79.7307692307692 0.59369945526123
84.5897435897436 0.572606801986694
89.7435897435897 0.565208554267883
95.2179487179487 0.57346510887146
101.019230769231 0.549895286560059
107.179487179487 0.554142653942108
113.711538461538 0.559076189994812
120.641025641026 0.545779228210449
127.99358974359 0.556437730789185
135.801282051282 0.527416288852692
144.076923076923 0.544892251491547
152.858974358974 0.529297471046448
162.179487179487 0.539228141307831
172.064102564103 0.527096748352051
182.551282051282 0.514154672622681
193.679487179487 0.517471551895142
205.487179487179 0.505025029182434
218.012820512821 0.531710922718048
231.301282051282 0.517846047878265
245.403846153846 0.516850769519806
260.358974358974 0.496802896261215
276.230769230769 0.504508435726166
293.070512820513 0.490358412265778
310.935897435897 0.519662916660309
329.884615384615 0.498074859380722
350 0.49126210808754
};
\addplot [, black, opacity=0.6, mark=*, mark size=0.5, mark options={solid}, only marks, forget plot]
table {%
1 0.973074793815613
1.05769230769231 0.974036633968353
1.12179487179487 0.974056363105774
1.19230769230769 0.971284151077271
1.26282051282051 0.951432943344116
1.33974358974359 0.95293253660202
1.42307692307692 0.962950706481934
1.51282051282051 0.9568110704422
1.6025641025641 0.924650609493256
1.69871794871795 0.925031006336212
1.80128205128205 0.918965041637421
1.91666666666667 0.919165015220642
2.03205128205128 0.939877986907959
2.15384615384615 0.913984358310699
2.28846153846154 0.93136078119278
2.42307692307692 0.931186258792877
2.57692307692308 0.946491837501526
2.73076923076923 0.92075377702713
2.8974358974359 0.927034735679626
3.07692307692308 0.91768354177475
3.26282051282051 0.922376692295074
3.46153846153846 0.929089963436127
3.67307692307692 0.919421851634979
3.8974358974359 0.911060929298401
4.13461538461539 0.902861893177032
4.38461538461539 0.900286376476288
4.65384615384615 0.909771084785461
4.93589743589744 0.893032371997833
5.23717948717949 0.889559745788574
5.55769230769231 0.887795984745026
5.8974358974359 0.878243386745453
6.25641025641026 0.860329747200012
6.64102564102564 0.867230892181396
7.04487179487179 0.849577248096466
7.47435897435897 0.848664045333862
7.92948717948718 0.83221822977066
8.41025641025641 0.8178591132164
8.92307692307692 0.821105778217316
9.46794871794872 0.826702833175659
10.0448717948718 0.81689465045929
10.6602564102564 0.822579562664032
11.3076923076923 0.786180853843689
12 0.765198647975922
12.7307692307692 0.783441424369812
13.5064102564103 0.771246492862701
14.3333333333333 0.771606028079987
15.2051282051282 0.752801954746246
16.1346153846154 0.764906585216522
17.1153846153846 0.750824630260468
18.1602564102564 0.754004716873169
19.2692307692308 0.748698711395264
20.4423076923077 0.715615570545197
21.6858974358974 0.707802593708038
23.0128205128205 0.704050898551941
24.4102564102564 0.707008183002472
25.9038461538462 0.696439504623413
27.4807692307692 0.702544450759888
29.1538461538462 0.69283539056778
30.9358974358974 0.696598172187805
32.8205128205128 0.671212911605835
34.8205128205128 0.685897588729858
36.9423076923077 0.664141118526459
39.1923076923077 0.65936553478241
41.5833333333333 0.648882567882538
44.1153846153846 0.639372527599335
46.8076923076923 0.62833309173584
49.6602564102564 0.614473044872284
52.6858974358974 0.621728360652924
55.8974358974359 0.614195466041565
59.3076923076923 0.623970746994019
62.9230769230769 0.618123412132263
66.7564102564103 0.612947583198547
70.8269230769231 0.596266329288483
75.1474358974359 0.589711904525757
79.7307692307692 0.578270316123962
84.5897435897436 0.564283132553101
89.7435897435897 0.557705223560333
95.2179487179487 0.566841125488281
101.019230769231 0.54389101266861
107.179487179487 0.556021630764008
113.711538461538 0.554065048694611
120.641025641026 0.547782599925995
127.99358974359 0.539099514484406
135.801282051282 0.521797478199005
144.076923076923 0.527629673480988
152.858974358974 0.520894110202789
162.179487179487 0.535523056983948
172.064102564103 0.514408230781555
182.551282051282 0.506534993648529
193.679487179487 0.524326682090759
205.487179487179 0.495340406894684
218.012820512821 0.512826442718506
231.301282051282 0.492131114006042
245.403846153846 0.50511234998703
260.358974358974 0.507764875888824
276.230769230769 0.495755195617676
293.070512820513 0.50640469789505
310.935897435897 0.500794231891632
329.884615384615 0.501163244247437
350 0.48217448592186
};
\addplot [, black, opacity=0.6, mark=*, mark size=0.5, mark options={solid}, only marks, forget plot]
table {%
1 0.981819450855255
1.05769230769231 0.986995697021484
1.12179487179487 0.992421269416809
1.19230769230769 0.981902122497559
1.26282051282051 0.966894567012787
1.33974358974359 0.965654730796814
1.42307692307692 0.96755576133728
1.51282051282051 0.963389039039612
1.6025641025641 0.95392644405365
1.69871794871795 0.944283068180084
1.80128205128205 0.946841418743134
1.91666666666667 0.945733070373535
2.03205128205128 0.947330892086029
2.15384615384615 0.940836906433105
2.28846153846154 0.946786046028137
2.42307692307692 0.913325309753418
2.57692307692308 0.941345930099487
2.73076923076923 0.928341031074524
2.8974358974359 0.92276006937027
3.07692307692308 0.909208834171295
3.26282051282051 0.927972257137299
3.46153846153846 0.91912567615509
3.67307692307692 0.913002133369446
3.8974358974359 0.908333241939545
4.13461538461539 0.911212921142578
4.38461538461539 0.898937940597534
4.65384615384615 0.910302877426147
4.93589743589744 0.902023375034332
5.23717948717949 0.888281226158142
5.55769230769231 0.886549055576324
5.8974358974359 0.879764258861542
6.25641025641026 0.869912445545197
6.64102564102564 0.871969878673553
7.04487179487179 0.867350280284882
7.47435897435897 0.86539626121521
7.92948717948718 0.854749917984009
8.41025641025641 0.85090833902359
8.92307692307692 0.83866935968399
9.46794871794872 0.821592330932617
10.0448717948718 0.826583862304688
10.6602564102564 0.813197672367096
11.3076923076923 0.808744788169861
12 0.777637183666229
12.7307692307692 0.79894232749939
13.5064102564103 0.796946704387665
14.3333333333333 0.780136525630951
15.2051282051282 0.759592473506927
16.1346153846154 0.773946642875671
17.1153846153846 0.762903451919556
18.1602564102564 0.763047218322754
19.2692307692308 0.741581559181213
20.4423076923077 0.738186776638031
21.6858974358974 0.723470449447632
23.0128205128205 0.73141747713089
24.4102564102564 0.732576906681061
25.9038461538462 0.703784942626953
27.4807692307692 0.705468714237213
29.1538461538462 0.70492148399353
30.9358974358974 0.701606452465057
32.8205128205128 0.704365074634552
34.8205128205128 0.688567399978638
36.9423076923077 0.66524749994278
39.1923076923077 0.657791137695312
41.5833333333333 0.663558542728424
44.1153846153846 0.644113779067993
46.8076923076923 0.665998756885529
49.6602564102564 0.630981028079987
52.6858974358974 0.636042535305023
55.8974358974359 0.640016317367554
59.3076923076923 0.613934218883514
62.9230769230769 0.629107892513275
66.7564102564103 0.598999798297882
70.8269230769231 0.622009634971619
75.1474358974359 0.594530999660492
79.7307692307692 0.601461708545685
84.5897435897436 0.588247537612915
89.7435897435897 0.5713831782341
95.2179487179487 0.578068315982819
101.019230769231 0.573698818683624
107.179487179487 0.551535069942474
113.711538461538 0.552440881729126
120.641025641026 0.540886044502258
127.99358974359 0.545330822467804
135.801282051282 0.534851789474487
144.076923076923 0.535168528556824
152.858974358974 0.540027737617493
162.179487179487 0.520861566066742
172.064102564103 0.508939146995544
182.551282051282 0.512246787548065
193.679487179487 0.510369122028351
205.487179487179 0.506465375423431
218.012820512821 0.517984926700592
231.301282051282 0.517126619815826
245.403846153846 0.509512901306152
260.358974358974 0.50203275680542
276.230769230769 0.49760177731514
293.070512820513 0.509005784988403
310.935897435897 0.50272136926651
329.884615384615 0.496519833803177
350 0.488369286060333
};
\addplot [, color0, opacity=0.6, mark=diamond*, mark size=0.5, mark options={solid}, only marks]
table {%
1 0.936439633369446
1.05769230769231 0.932086408138275
1.12179487179487 0.939946055412292
1.19230769230769 0.914643287658691
1.26282051282051 0.901255548000336
1.33974358974359 0.882337629795074
1.42307692307692 0.893976092338562
1.51282051282051 0.875895738601685
1.6025641025641 0.870437920093536
1.69871794871795 0.865155339241028
1.80128205128205 0.84481292963028
1.91666666666667 0.785620272159576
2.03205128205128 0.815882802009583
2.15384615384615 0.795432865619659
2.28846153846154 0.801658511161804
2.42307692307692 0.77410900592804
2.57692307692308 0.804407775402069
2.73076923076923 0.795560657978058
2.8974358974359 0.784322023391724
3.07692307692308 0.7731112241745
3.26282051282051 0.796203136444092
3.46153846153846 0.794804573059082
3.67307692307692 0.788516223430634
3.8974358974359 0.764145195484161
4.13461538461539 0.774282276630402
4.38461538461539 0.733729660511017
4.65384615384615 0.747815251350403
4.93589743589744 0.734493494033813
5.23717948717949 0.722838580608368
5.55769230769231 0.700708627700806
5.8974358974359 0.668090343475342
6.25641025641026 0.650139689445496
6.64102564102564 0.672174990177155
7.04487179487179 0.649082064628601
7.47435897435897 0.644184410572052
7.92948717948718 0.590073585510254
8.41025641025641 0.631867587566376
8.92307692307692 0.596216082572937
9.46794871794872 0.596657633781433
10.0448717948718 0.602220058441162
10.6602564102564 0.576139748096466
11.3076923076923 0.565365314483643
12 0.537430584430695
12.7307692307692 0.551621437072754
13.5064102564103 0.528318464756012
14.3333333333333 0.532430648803711
15.2051282051282 0.51909726858139
16.1346153846154 0.522771716117859
17.1153846153846 0.502735793590546
18.1602564102564 0.508953213691711
19.2692307692308 0.502262115478516
20.4423076923077 0.473986953496933
21.6858974358974 0.457163006067276
23.0128205128205 0.466552197933197
24.4102564102564 0.475241243839264
25.9038461538462 0.480052947998047
27.4807692307692 0.473011314868927
29.1538461538462 0.472847700119019
30.9358974358974 0.475900173187256
32.8205128205128 0.479537069797516
34.8205128205128 0.471518993377686
36.9423076923077 0.480231463909149
39.1923076923077 0.455073446035385
41.5833333333333 0.453890830278397
44.1153846153846 0.43617507815361
46.8076923076923 0.442894846200943
49.6602564102564 0.426432549953461
52.6858974358974 0.428810715675354
55.8974358974359 0.431519985198975
59.3076923076923 0.43310222029686
62.9230769230769 0.423543781042099
66.7564102564103 0.437787085771561
70.8269230769231 0.431640297174454
75.1474358974359 0.432536542415619
79.7307692307692 0.437518328428268
84.5897435897436 0.432222992181778
89.7435897435897 0.409190326929092
95.2179487179487 0.408750146627426
101.019230769231 0.422462612390518
107.179487179487 0.414987027645111
113.711538461538 0.413513511419296
120.641025641026 0.418418198823929
127.99358974359 0.422802239656448
135.801282051282 0.410862386226654
144.076923076923 0.402502477169037
152.858974358974 0.428563743829727
162.179487179487 0.431398034095764
172.064102564103 0.410982847213745
182.551282051282 0.414090096950531
193.679487179487 0.42648634314537
205.487179487179 0.415347725152969
218.012820512821 0.433764934539795
231.301282051282 0.42604723572731
245.403846153846 0.419025719165802
260.358974358974 0.422412127256393
276.230769230769 0.422326892614365
293.070512820513 0.413734585046768
310.935897435897 0.412371426820755
329.884615384615 0.41583976149559
350 0.415099203586578
};
\addlegendentry{sub 16, exact}
\addplot [, color0, opacity=0.6, mark=diamond*, mark size=0.5, mark options={solid}, only marks, forget plot]
table {%
1 0.930098652839661
1.05769230769231 0.936324000358582
1.12179487179487 0.887235999107361
1.19230769230769 0.896987438201904
1.26282051282051 0.836709856987
1.33974358974359 0.809770882129669
1.42307692307692 0.776368677616119
1.51282051282051 0.785330474376678
1.6025641025641 0.808314025402069
1.69871794871795 0.801929712295532
1.80128205128205 0.753406941890717
1.91666666666667 0.685248851776123
2.03205128205128 0.747293710708618
2.15384615384615 0.702381372451782
2.28846153846154 0.738197863101959
2.42307692307692 0.734760820865631
2.57692307692308 0.736591875553131
2.73076923076923 0.698021233081818
2.8974358974359 0.72564697265625
3.07692307692308 0.679457545280457
3.26282051282051 0.737400352954865
3.46153846153846 0.711812496185303
3.67307692307692 0.689024746417999
3.8974358974359 0.703938126564026
4.13461538461539 0.735087633132935
4.38461538461539 0.687119603157043
4.65384615384615 0.691500782966614
4.93589743589744 0.69467830657959
5.23717948717949 0.709119856357574
5.55769230769231 0.702312886714935
5.8974358974359 0.658713459968567
6.25641025641026 0.664749264717102
6.64102564102564 0.657972931861877
7.04487179487179 0.648160934448242
7.47435897435897 0.625959515571594
7.92948717948718 0.619797229766846
8.41025641025641 0.628389179706573
8.92307692307692 0.581790089607239
9.46794871794872 0.595971763134003
10.0448717948718 0.577891707420349
10.6602564102564 0.601671576499939
11.3076923076923 0.566995561122894
12 0.560415267944336
12.7307692307692 0.566740393638611
13.5064102564103 0.550153195858002
14.3333333333333 0.557715594768524
15.2051282051282 0.53915673494339
16.1346153846154 0.559297025203705
17.1153846153846 0.524544358253479
18.1602564102564 0.534090518951416
19.2692307692308 0.518975079059601
20.4423076923077 0.512084424495697
21.6858974358974 0.506290256977081
23.0128205128205 0.512994110584259
24.4102564102564 0.515943109989166
25.9038461538462 0.510537385940552
27.4807692307692 0.511310040950775
29.1538461538462 0.513623118400574
30.9358974358974 0.504810631275177
32.8205128205128 0.514741599559784
34.8205128205128 0.51353508234024
36.9423076923077 0.503508925437927
39.1923076923077 0.493344157934189
41.5833333333333 0.492171168327332
44.1153846153846 0.477276146411896
46.8076923076923 0.48408779501915
49.6602564102564 0.455945372581482
52.6858974358974 0.46320167183876
55.8974358974359 0.475885391235352
59.3076923076923 0.464786171913147
62.9230769230769 0.465249747037888
66.7564102564103 0.460118323564529
70.8269230769231 0.451480805873871
75.1474358974359 0.435274571180344
79.7307692307692 0.449397802352905
84.5897435897436 0.43225622177124
89.7435897435897 0.447993755340576
95.2179487179487 0.432117372751236
101.019230769231 0.427645862102509
107.179487179487 0.429674416780472
113.711538461538 0.433178395032883
120.641025641026 0.424209505319595
127.99358974359 0.420053780078888
135.801282051282 0.417565912008286
144.076923076923 0.415380477905273
152.858974358974 0.42665159702301
162.179487179487 0.423745095729828
172.064102564103 0.414129853248596
182.551282051282 0.416862279176712
193.679487179487 0.412725895643234
205.487179487179 0.418720781803131
218.012820512821 0.423996329307556
231.301282051282 0.417504727840424
245.403846153846 0.411237984895706
260.358974358974 0.41784393787384
276.230769230769 0.431709408760071
293.070512820513 0.415151715278625
310.935897435897 0.425715744495392
329.884615384615 0.416638255119324
350 0.390501290559769
};
\addplot [, color0, opacity=0.6, mark=diamond*, mark size=0.5, mark options={solid}, only marks, forget plot]
table {%
1 0.933270871639252
1.05769230769231 0.940491437911987
1.12179487179487 0.942421078681946
1.19230769230769 0.95472377538681
1.26282051282051 0.918484389781952
1.33974358974359 0.893985450267792
1.42307692307692 0.892791092395782
1.51282051282051 0.89630001783371
1.6025641025641 0.870285034179688
1.69871794871795 0.86382007598877
1.80128205128205 0.857481360435486
1.91666666666667 0.826464831829071
2.03205128205128 0.819771230220795
2.15384615384615 0.82402515411377
2.28846153846154 0.8212970495224
2.42307692307692 0.808269917964935
2.57692307692308 0.798818111419678
2.73076923076923 0.80361133813858
2.8974358974359 0.777217626571655
3.07692307692308 0.785517692565918
3.26282051282051 0.763309836387634
3.46153846153846 0.782493889331818
3.67307692307692 0.758544206619263
3.8974358974359 0.762992382049561
4.13461538461539 0.773100554943085
4.38461538461539 0.739786803722382
4.65384615384615 0.740129053592682
4.93589743589744 0.731232881546021
5.23717948717949 0.746581256389618
5.55769230769231 0.723346412181854
5.8974358974359 0.682069540023804
6.25641025641026 0.678088903427124
6.64102564102564 0.674465835094452
7.04487179487179 0.666225433349609
7.47435897435897 0.64240437746048
7.92948717948718 0.642122030258179
8.41025641025641 0.650824368000031
8.92307692307692 0.620442509651184
9.46794871794872 0.626511991024017
10.0448717948718 0.629148364067078
10.6602564102564 0.611198246479034
11.3076923076923 0.590630412101746
12 0.546530306339264
12.7307692307692 0.558150470256805
13.5064102564103 0.535409212112427
14.3333333333333 0.53939962387085
15.2051282051282 0.523659706115723
16.1346153846154 0.549286782741547
17.1153846153846 0.513083219528198
18.1602564102564 0.524329721927643
19.2692307692308 0.513449847698212
20.4423076923077 0.516359984874725
21.6858974358974 0.512772619724274
23.0128205128205 0.507116436958313
24.4102564102564 0.512115597724915
25.9038461538462 0.489657938480377
27.4807692307692 0.495036065578461
29.1538461538462 0.493038386106491
30.9358974358974 0.483463704586029
32.8205128205128 0.4917331635952
34.8205128205128 0.487339586019516
36.9423076923077 0.485363602638245
39.1923076923077 0.484793245792389
41.5833333333333 0.477769941091537
44.1153846153846 0.474489510059357
46.8076923076923 0.478943318128586
49.6602564102564 0.453219205141068
52.6858974358974 0.457178801298141
55.8974358974359 0.450193017721176
59.3076923076923 0.461207419633865
62.9230769230769 0.453179150819778
66.7564102564103 0.446081966161728
70.8269230769231 0.443074405193329
75.1474358974359 0.427557349205017
79.7307692307692 0.43535315990448
84.5897435897436 0.425668179988861
89.7435897435897 0.411206096410751
95.2179487179487 0.398852974176407
101.019230769231 0.414361834526062
107.179487179487 0.409717172384262
113.711538461538 0.407701075077057
120.641025641026 0.403611749410629
127.99358974359 0.391916990280151
135.801282051282 0.387197345495224
144.076923076923 0.390724301338196
152.858974358974 0.388925611972809
162.179487179487 0.398891448974609
172.064102564103 0.38312965631485
182.551282051282 0.378791451454163
193.679487179487 0.386349260807037
205.487179487179 0.389277577400208
218.012820512821 0.400312423706055
231.301282051282 0.384331375360489
245.403846153846 0.412204384803772
260.358974358974 0.395265489816666
276.230769230769 0.397576183080673
293.070512820513 0.385695189237595
310.935897435897 0.403286188840866
329.884615384615 0.409073084592819
350 0.37557378411293
};
\addplot [, color0, opacity=0.6, mark=diamond*, mark size=0.5, mark options={solid}, only marks, forget plot]
table {%
1 0.931726038455963
1.05769230769231 0.930187225341797
1.12179487179487 0.933761417865753
1.19230769230769 0.932210683822632
1.26282051282051 0.888843357563019
1.33974358974359 0.842708349227905
1.42307692307692 0.856487572193146
1.51282051282051 0.849077582359314
1.6025641025641 0.863771200180054
1.69871794871795 0.843335390090942
1.80128205128205 0.811417281627655
1.91666666666667 0.829980850219727
2.03205128205128 0.838548362255096
2.15384615384615 0.824245452880859
2.28846153846154 0.817582607269287
2.42307692307692 0.788572549819946
2.57692307692308 0.806861996650696
2.73076923076923 0.771566152572632
2.8974358974359 0.793925702571869
3.07692307692308 0.763812839984894
3.26282051282051 0.77752548456192
3.46153846153846 0.781073272228241
3.67307692307692 0.768491625785828
3.8974358974359 0.771456182003021
4.13461538461539 0.796904444694519
4.38461538461539 0.754819571971893
4.65384615384615 0.759666740894318
4.93589743589744 0.760250508785248
5.23717948717949 0.753797292709351
5.55769230769231 0.729410588741302
5.8974358974359 0.678593277931213
6.25641025641026 0.662622809410095
6.64102564102564 0.682383418083191
7.04487179487179 0.664328575134277
7.47435897435897 0.647923290729523
7.92948717948718 0.640129864215851
8.41025641025641 0.630919694900513
8.92307692307692 0.597106754779816
9.46794871794872 0.596174120903015
10.0448717948718 0.623570621013641
10.6602564102564 0.600597381591797
11.3076923076923 0.576222240924835
12 0.552050173282623
12.7307692307692 0.55936735868454
13.5064102564103 0.550142347812653
14.3333333333333 0.52959793806076
15.2051282051282 0.512343406677246
16.1346153846154 0.545985043048859
17.1153846153846 0.501706659793854
18.1602564102564 0.531832277774811
19.2692307692308 0.513427138328552
20.4423076923077 0.496354669332504
21.6858974358974 0.480106800794601
23.0128205128205 0.497616648674011
24.4102564102564 0.495485633611679
25.9038461538462 0.472858279943466
27.4807692307692 0.486423552036285
29.1538461538462 0.483402937650681
30.9358974358974 0.491280645132065
32.8205128205128 0.481512665748596
34.8205128205128 0.48005285859108
36.9423076923077 0.472878873348236
39.1923076923077 0.466998308897018
41.5833333333333 0.452332377433777
44.1153846153846 0.453185111284256
46.8076923076923 0.448763430118561
49.6602564102564 0.427755206823349
52.6858974358974 0.440765231847763
55.8974358974359 0.448018431663513
59.3076923076923 0.437930107116699
62.9230769230769 0.432350188493729
66.7564102564103 0.426109045743942
70.8269230769231 0.439452707767487
75.1474358974359 0.428783029317856
79.7307692307692 0.426903456449509
84.5897435897436 0.429814785718918
89.7435897435897 0.413943082094193
95.2179487179487 0.420913308858871
101.019230769231 0.413435786962509
107.179487179487 0.415014714002609
113.711538461538 0.406839847564697
120.641025641026 0.402666628360748
127.99358974359 0.424334019422531
135.801282051282 0.411405235528946
144.076923076923 0.409592092037201
152.858974358974 0.395472377538681
162.179487179487 0.400523662567139
172.064102564103 0.401677161455154
182.551282051282 0.413583785295486
193.679487179487 0.40368789434433
205.487179487179 0.407499223947525
218.012820512821 0.406911611557007
231.301282051282 0.410262525081635
245.403846153846 0.40980726480484
260.358974358974 0.427005082368851
276.230769230769 0.406012296676636
293.070512820513 0.40661159157753
310.935897435897 0.415760099887848
329.884615384615 0.413130015134811
350 0.398159980773926
};
\addplot [, color0, opacity=0.6, mark=diamond*, mark size=0.5, mark options={solid}, only marks, forget plot]
table {%
1 0.914267122745514
1.05769230769231 0.911040067672729
1.12179487179487 0.907613933086395
1.19230769230769 0.852823317050934
1.26282051282051 0.820928931236267
1.33974358974359 0.812134087085724
1.42307692307692 0.742417752742767
1.51282051282051 0.775601923465729
1.6025641025641 0.797860562801361
1.69871794871795 0.744879364967346
1.80128205128205 0.76965194940567
1.91666666666667 0.700527608394623
2.03205128205128 0.715976238250732
2.15384615384615 0.712589085102081
2.28846153846154 0.692628920078278
2.42307692307692 0.662370204925537
2.57692307692308 0.70674329996109
2.73076923076923 0.694352686405182
2.8974358974359 0.69632887840271
3.07692307692308 0.661267995834351
3.26282051282051 0.750168919563293
3.46153846153846 0.699730336666107
3.67307692307692 0.69275176525116
3.8974358974359 0.713401675224304
4.13461538461539 0.718294501304626
4.38461538461539 0.677066326141357
4.65384615384615 0.682547807693481
4.93589743589744 0.675775587558746
5.23717948717949 0.695060849189758
5.55769230769231 0.663176476955414
5.8974358974359 0.621763050556183
6.25641025641026 0.631386876106262
6.64102564102564 0.6263267993927
7.04487179487179 0.606400728225708
7.47435897435897 0.596426844596863
7.92948717948718 0.592089831829071
8.41025641025641 0.6106898188591
8.92307692307692 0.575954139232635
9.46794871794872 0.572589933872223
10.0448717948718 0.584258019924164
10.6602564102564 0.555274784564972
11.3076923076923 0.550122916698456
12 0.52741265296936
12.7307692307692 0.529035747051239
13.5064102564103 0.518456280231476
14.3333333333333 0.514563500881195
15.2051282051282 0.514461636543274
16.1346153846154 0.508955717086792
17.1153846153846 0.482096701860428
18.1602564102564 0.50201803445816
19.2692307692308 0.473786294460297
20.4423076923077 0.457755386829376
21.6858974358974 0.471911519765854
23.0128205128205 0.474655598402023
24.4102564102564 0.504735708236694
25.9038461538462 0.474529713392258
27.4807692307692 0.476856529712677
29.1538461538462 0.489820063114166
30.9358974358974 0.480429917573929
32.8205128205128 0.492304176092148
34.8205128205128 0.494919449090958
36.9423076923077 0.476417064666748
39.1923076923077 0.471622139215469
41.5833333333333 0.469870656728745
44.1153846153846 0.457305043935776
46.8076923076923 0.482921600341797
49.6602564102564 0.45060396194458
52.6858974358974 0.452161252498627
55.8974358974359 0.456999659538269
59.3076923076923 0.453168749809265
62.9230769230769 0.432332456111908
66.7564102564103 0.468690246343613
70.8269230769231 0.453002274036407
75.1474358974359 0.411635041236877
79.7307692307692 0.458475589752197
84.5897435897436 0.451189339160919
89.7435897435897 0.434138208627701
95.2179487179487 0.415565758943558
101.019230769231 0.41633141040802
107.179487179487 0.42101177573204
113.711538461538 0.418516963720322
120.641025641026 0.400375217199326
127.99358974359 0.421439737081528
135.801282051282 0.41895255446434
144.076923076923 0.41204559803009
152.858974358974 0.423721492290497
162.179487179487 0.436360627412796
172.064102564103 0.426213204860687
182.551282051282 0.414214700460434
193.679487179487 0.426547795534134
205.487179487179 0.428483009338379
218.012820512821 0.449408173561096
231.301282051282 0.415514558553696
245.403846153846 0.434568554162979
260.358974358974 0.424976944923401
276.230769230769 0.421851456165314
293.070512820513 0.428046643733978
310.935897435897 0.427186965942383
329.884615384615 0.421970903873444
350 0.4096300303936
};
\addplot [, color1, opacity=0.6, mark=square*, mark size=0.5, mark options={solid}, only marks]
table {%
1 0.895026087760925
1.05769230769231 0.91273021697998
1.12179487179487 0.913562476634979
1.19230769230769 0.85784786939621
1.26282051282051 0.880396723747253
1.33974358974359 0.85784125328064
1.42307692307692 0.852949678897858
1.51282051282051 0.815407872200012
1.6025641025641 0.827289044857025
1.69871794871795 0.81375652551651
1.80128205128205 0.818469047546387
1.91666666666667 0.798443257808685
2.03205128205128 0.830052316188812
2.15384615384615 0.79740047454834
2.28846153846154 0.828336000442505
2.42307692307692 0.799253880977631
2.57692307692308 0.82762748003006
2.73076923076923 0.809667348861694
2.8974358974359 0.793619394302368
3.07692307692308 0.802735030651093
3.26282051282051 0.830536782741547
3.46153846153846 0.845715820789337
3.67307692307692 0.814646303653717
3.8974358974359 0.785737812519073
4.13461538461539 0.816376030445099
4.38461538461539 0.807143092155457
4.65384615384615 0.809456825256348
4.93589743589744 0.813360750675201
5.23717948717949 0.794027090072632
5.55769230769231 0.797111630439758
5.8974358974359 0.757156431674957
6.25641025641026 0.76405930519104
6.64102564102564 0.768119513988495
7.04487179487179 0.783468544483185
7.47435897435897 0.783824145793915
7.92948717948718 0.757048189640045
8.41025641025641 0.762384474277496
8.92307692307692 0.74178946018219
9.46794871794872 0.737581491470337
10.0448717948718 0.7448690533638
10.6602564102564 0.735121190547943
11.3076923076923 0.704544484615326
12 0.706261575222015
12.7307692307692 0.732634544372559
13.5064102564103 0.69481748342514
14.3333333333333 0.692395448684692
15.2051282051282 0.676822483539581
16.1346153846154 0.69724041223526
17.1153846153846 0.68621689081192
18.1602564102564 0.688799440860748
19.2692307692308 0.688109219074249
20.4423076923077 0.646568417549133
21.6858974358974 0.647376775741577
23.0128205128205 0.653742969036102
24.4102564102564 0.642067730426788
25.9038461538462 0.629220128059387
27.4807692307692 0.655843317508698
29.1538461538462 0.6407710313797
30.9358974358974 0.64071387052536
32.8205128205128 0.642535984516144
34.8205128205128 0.628225088119507
36.9423076923077 0.613034665584564
39.1923076923077 0.602855026721954
41.5833333333333 0.608140349388123
44.1153846153846 0.597893953323364
46.8076923076923 0.601031184196472
49.6602564102564 0.583495497703552
52.6858974358974 0.582461178302765
55.8974358974359 0.58353716135025
59.3076923076923 0.567424118518829
62.9230769230769 0.56151807308197
66.7564102564103 0.569585621356964
70.8269230769231 0.548322916030884
75.1474358974359 0.552358090877533
79.7307692307692 0.552101373672485
84.5897435897436 0.538802325725555
89.7435897435897 0.525000154972076
95.2179487179487 0.519688963890076
101.019230769231 0.519348561763763
107.179487179487 0.498552620410919
113.711538461538 0.501473844051361
120.641025641026 0.49714720249176
127.99358974359 0.502254068851471
135.801282051282 0.49982488155365
144.076923076923 0.494211733341217
152.858974358974 0.490295618772507
162.179487179487 0.482559263706207
172.064102564103 0.478329002857208
182.551282051282 0.478344947099686
193.679487179487 0.479232549667358
205.487179487179 0.473385602235794
218.012820512821 0.469956815242767
231.301282051282 0.45597243309021
245.403846153846 0.448147654533386
260.358974358974 0.465117752552032
276.230769230769 0.456770539283752
293.070512820513 0.475770950317383
310.935897435897 0.456342995166779
329.884615384615 0.466485619544983
350 0.444045096635818
};
\addlegendentry{mb 128, mc 10}
\addplot [, color1, opacity=0.6, mark=square*, mark size=0.5, mark options={solid}, only marks, forget plot]
table {%
1 0.910532355308533
1.05769230769231 0.927267730236053
1.12179487179487 0.915995478630066
1.19230769230769 0.891276001930237
1.26282051282051 0.887530326843262
1.33974358974359 0.874638497829437
1.42307692307692 0.852442920207977
1.51282051282051 0.851866662502289
1.6025641025641 0.835761547088623
1.69871794871795 0.845355272293091
1.80128205128205 0.839844703674316
1.91666666666667 0.815570831298828
2.03205128205128 0.832287132740021
2.15384615384615 0.859672546386719
2.28846153846154 0.841521739959717
2.42307692307692 0.834565579891205
2.57692307692308 0.820834279060364
2.73076923076923 0.813644409179688
2.8974358974359 0.827554285526276
3.07692307692308 0.83391934633255
3.26282051282051 0.814892649650574
3.46153846153846 0.828999102115631
3.67307692307692 0.807915687561035
3.8974358974359 0.804605841636658
4.13461538461539 0.826688647270203
4.38461538461539 0.799359440803528
4.65384615384615 0.795830130577087
4.93589743589744 0.818508267402649
5.23717948717949 0.813117444515228
5.55769230769231 0.81680691242218
5.8974358974359 0.766204833984375
6.25641025641026 0.774137079715729
6.64102564102564 0.789131283760071
7.04487179487179 0.79110449552536
7.47435897435897 0.773920297622681
7.92948717948718 0.765023708343506
8.41025641025641 0.7683464884758
8.92307692307692 0.729079484939575
9.46794871794872 0.748536348342896
10.0448717948718 0.734066903591156
10.6602564102564 0.717568516731262
11.3076923076923 0.710588216781616
12 0.685728967189789
12.7307692307692 0.711672425270081
13.5064102564103 0.694351255893707
14.3333333333333 0.681260526180267
15.2051282051282 0.671515464782715
16.1346153846154 0.668837130069733
17.1153846153846 0.651925027370453
18.1602564102564 0.665541589260101
19.2692307692308 0.637967944145203
20.4423076923077 0.615665853023529
21.6858974358974 0.617234706878662
23.0128205128205 0.620338797569275
24.4102564102564 0.625524401664734
25.9038461538462 0.595243871212006
27.4807692307692 0.621575832366943
29.1538461538462 0.606139659881592
30.9358974358974 0.610131919384003
32.8205128205128 0.612762331962585
34.8205128205128 0.592966198921204
36.9423076923077 0.582863748073578
39.1923076923077 0.566697537899017
41.5833333333333 0.559789061546326
44.1153846153846 0.571477174758911
46.8076923076923 0.558198511600494
49.6602564102564 0.542504251003265
52.6858974358974 0.551504731178284
55.8974358974359 0.552933275699615
59.3076923076923 0.551933586597443
62.9230769230769 0.541701912879944
66.7564102564103 0.529432713985443
70.8269230769231 0.523299098014832
75.1474358974359 0.525295913219452
79.7307692307692 0.511102795600891
84.5897435897436 0.503622829914093
89.7435897435897 0.495087236166
95.2179487179487 0.487614125013351
101.019230769231 0.49089315533638
107.179487179487 0.486544013023376
113.711538461538 0.476891160011292
120.641025641026 0.469243139028549
127.99358974359 0.489649564027786
135.801282051282 0.450748443603516
144.076923076923 0.465919822454453
152.858974358974 0.450518190860748
162.179487179487 0.45781198143959
172.064102564103 0.442210555076599
182.551282051282 0.451158434152603
193.679487179487 0.436173111200333
205.487179487179 0.445156991481781
218.012820512821 0.443850010633469
231.301282051282 0.427117377519608
245.403846153846 0.439135134220123
260.358974358974 0.437696516513824
276.230769230769 0.431692481040955
293.070512820513 0.413511425256729
310.935897435897 0.434751123189926
329.884615384615 0.433948129415512
350 0.404979825019836
};
\addplot [, color1, opacity=0.6, mark=square*, mark size=0.5, mark options={solid}, only marks, forget plot]
table {%
1 0.904009222984314
1.05769230769231 0.917912125587463
1.12179487179487 0.90190589427948
1.19230769230769 0.879743158817291
1.26282051282051 0.864506304264069
1.33974358974359 0.865482449531555
1.42307692307692 0.84936261177063
1.51282051282051 0.817354440689087
1.6025641025641 0.827049314975739
1.69871794871795 0.827577710151672
1.80128205128205 0.822985410690308
1.91666666666667 0.820055365562439
2.03205128205128 0.824317634105682
2.15384615384615 0.830839157104492
2.28846153846154 0.828629195690155
2.42307692307692 0.807436347007751
2.57692307692308 0.837042152881622
2.73076923076923 0.842273414134979
2.8974358974359 0.841518700122833
3.07692307692308 0.80930769443512
3.26282051282051 0.824752390384674
3.46153846153846 0.845909535884857
3.67307692307692 0.846459329128265
3.8974358974359 0.826198935508728
4.13461538461539 0.858924388885498
4.38461538461539 0.828780710697174
4.65384615384615 0.789839684963226
4.93589743589744 0.832162916660309
5.23717948717949 0.819885611534119
5.55769230769231 0.809110701084137
5.8974358974359 0.774251699447632
6.25641025641026 0.772099494934082
6.64102564102564 0.788254678249359
7.04487179487179 0.786806643009186
7.47435897435897 0.77839595079422
7.92948717948718 0.763541400432587
8.41025641025641 0.754239320755005
8.92307692307692 0.756094932556152
9.46794871794872 0.750860750675201
10.0448717948718 0.745193302631378
10.6602564102564 0.731208324432373
11.3076923076923 0.730417013168335
12 0.704124867916107
12.7307692307692 0.733039975166321
13.5064102564103 0.715430676937103
14.3333333333333 0.691055715084076
15.2051282051282 0.687590599060059
16.1346153846154 0.681096494197845
17.1153846153846 0.684240877628326
18.1602564102564 0.683987736701965
19.2692307692308 0.668043971061707
20.4423076923077 0.664744794368744
21.6858974358974 0.642054200172424
23.0128205128205 0.656174540519714
24.4102564102564 0.641381084918976
25.9038461538462 0.631656467914581
27.4807692307692 0.652118504047394
29.1538461538462 0.623218536376953
30.9358974358974 0.6224485039711
32.8205128205128 0.623175799846649
34.8205128205128 0.624390244483948
36.9423076923077 0.597179234027863
39.1923076923077 0.595929682254791
41.5833333333333 0.607441127300262
44.1153846153846 0.569210112094879
46.8076923076923 0.5822873711586
49.6602564102564 0.567029476165771
52.6858974358974 0.577822506427765
55.8974358974359 0.575224578380585
59.3076923076923 0.554940044879913
62.9230769230769 0.546550214290619
66.7564102564103 0.532273530960083
70.8269230769231 0.546018362045288
75.1474358974359 0.53265380859375
79.7307692307692 0.529966473579407
84.5897435897436 0.531863749027252
89.7435897435897 0.519340813159943
95.2179487179487 0.502964437007904
101.019230769231 0.50765597820282
107.179487179487 0.494180977344513
113.711538461538 0.490999162197113
120.641025641026 0.483168751001358
127.99358974359 0.482464134693146
135.801282051282 0.482428133487701
144.076923076923 0.470675110816956
152.858974358974 0.476894587278366
162.179487179487 0.463007420301437
172.064102564103 0.448869109153748
182.551282051282 0.447983205318451
193.679487179487 0.471532464027405
205.487179487179 0.453176915645599
218.012820512821 0.442729949951172
231.301282051282 0.450680881738663
245.403846153846 0.437385201454163
260.358974358974 0.456855714321136
276.230769230769 0.433828413486481
293.070512820513 0.425296455621719
310.935897435897 0.453116476535797
329.884615384615 0.435546725988388
350 0.416505336761475
};
\addplot [, color1, opacity=0.6, mark=square*, mark size=0.5, mark options={solid}, only marks, forget plot]
table {%
1 0.901552259922028
1.05769230769231 0.930428445339203
1.12179487179487 0.916645228862762
1.19230769230769 0.889305830001831
1.26282051282051 0.894463002681732
1.33974358974359 0.858455955982208
1.42307692307692 0.862920343875885
1.51282051282051 0.834447622299194
1.6025641025641 0.841048717498779
1.69871794871795 0.841306030750275
1.80128205128205 0.816614806652069
1.91666666666667 0.828755915164948
2.03205128205128 0.867031157016754
2.15384615384615 0.827190399169922
2.28846153846154 0.840454697608948
2.42307692307692 0.812182366847992
2.57692307692308 0.82395339012146
2.73076923076923 0.826059877872467
2.8974358974359 0.824369013309479
3.07692307692308 0.803335845470428
3.26282051282051 0.810802280902863
3.46153846153846 0.839359998703003
3.67307692307692 0.821480870246887
3.8974358974359 0.777844309806824
4.13461538461539 0.819152355194092
4.38461538461539 0.804844796657562
4.65384615384615 0.796742975711823
4.93589743589744 0.80405193567276
5.23717948717949 0.782207787036896
5.55769230769231 0.776097178459167
5.8974358974359 0.771194577217102
6.25641025641026 0.743569016456604
6.64102564102564 0.779579162597656
7.04487179487179 0.765991508960724
7.47435897435897 0.757098734378815
7.92948717948718 0.750011682510376
8.41025641025641 0.750681757926941
8.92307692307692 0.732090592384338
9.46794871794872 0.717319309711456
10.0448717948718 0.74774956703186
10.6602564102564 0.727437257766724
11.3076923076923 0.707348167896271
12 0.694304049015045
12.7307692307692 0.72152191400528
13.5064102564103 0.68577915430069
14.3333333333333 0.674016535282135
15.2051282051282 0.676907539367676
16.1346153846154 0.668241500854492
17.1153846153846 0.675901472568512
18.1602564102564 0.662297785282135
19.2692307692308 0.664410531520844
20.4423076923077 0.636400759220123
21.6858974358974 0.638454914093018
23.0128205128205 0.647033452987671
24.4102564102564 0.633752286434174
25.9038461538462 0.591892242431641
27.4807692307692 0.620823681354523
29.1538461538462 0.61301052570343
30.9358974358974 0.607185363769531
32.8205128205128 0.600000262260437
34.8205128205128 0.587838053703308
36.9423076923077 0.588951230049133
39.1923076923077 0.569410026073456
41.5833333333333 0.574178278446198
44.1153846153846 0.565673172473907
46.8076923076923 0.553898453712463
49.6602564102564 0.530152142047882
52.6858974358974 0.543918609619141
55.8974358974359 0.542346775531769
59.3076923076923 0.526454150676727
62.9230769230769 0.520126342773438
66.7564102564103 0.5178342461586
70.8269230769231 0.499129474163055
75.1474358974359 0.497554689645767
79.7307692307692 0.484347701072693
84.5897435897436 0.491989850997925
89.7435897435897 0.47470611333847
95.2179487179487 0.475317299365997
101.019230769231 0.455498188734055
107.179487179487 0.456057846546173
113.711538461538 0.459951400756836
120.641025641026 0.459063112735748
127.99358974359 0.46025824546814
135.801282051282 0.448154747486115
144.076923076923 0.442786633968353
152.858974358974 0.4533431828022
162.179487179487 0.435755461454391
172.064102564103 0.437334269285202
182.551282051282 0.435009986162186
193.679487179487 0.447338551282883
205.487179487179 0.414462268352509
218.012820512821 0.422449707984924
231.301282051282 0.412587851285934
245.403846153846 0.422448426485062
260.358974358974 0.41622719168663
276.230769230769 0.413670122623444
293.070512820513 0.421852946281433
310.935897435897 0.425639450550079
329.884615384615 0.419991701841354
350 0.410764276981354
};
\addplot [, color1, opacity=0.6, mark=square*, mark size=0.5, mark options={solid}, only marks, forget plot]
table {%
1 0.910223960876465
1.05769230769231 0.92803955078125
1.12179487179487 0.926936149597168
1.19230769230769 0.877564370632172
1.26282051282051 0.865333259105682
1.33974358974359 0.839495301246643
1.42307692307692 0.850455224514008
1.51282051282051 0.84604811668396
1.6025641025641 0.852501511573792
1.69871794871795 0.814870119094849
1.80128205128205 0.811839878559113
1.91666666666667 0.838808596134186
2.03205128205128 0.852265536785126
2.15384615384615 0.84391051530838
2.28846153846154 0.832561850547791
2.42307692307692 0.83478844165802
2.57692307692308 0.820452272891998
2.73076923076923 0.829350411891937
2.8974358974359 0.833479046821594
3.07692307692308 0.815111517906189
3.26282051282051 0.834924697875977
3.46153846153846 0.822761654853821
3.67307692307692 0.82017993927002
3.8974358974359 0.811099231243134
4.13461538461539 0.835246384143829
4.38461538461539 0.810250997543335
4.65384615384615 0.842010021209717
4.93589743589744 0.830913245677948
5.23717948717949 0.824224054813385
5.55769230769231 0.80413281917572
5.8974358974359 0.781443893909454
6.25641025641026 0.783243238925934
6.64102564102564 0.788551449775696
7.04487179487179 0.782995223999023
7.47435897435897 0.782214164733887
7.92948717948718 0.773711740970612
8.41025641025641 0.754446089267731
8.92307692307692 0.749343574047089
9.46794871794872 0.758246421813965
10.0448717948718 0.740626335144043
10.6602564102564 0.75166928768158
11.3076923076923 0.72123771905899
12 0.698492586612701
12.7307692307692 0.719740986824036
13.5064102564103 0.707544982433319
14.3333333333333 0.673251092433929
15.2051282051282 0.68961375951767
16.1346153846154 0.695536196231842
17.1153846153846 0.668698132038116
18.1602564102564 0.673464179039001
19.2692307692308 0.667788684368134
20.4423076923077 0.651221811771393
21.6858974358974 0.63510262966156
23.0128205128205 0.642892897129059
24.4102564102564 0.642683565616608
25.9038461538462 0.633985459804535
27.4807692307692 0.637018799781799
29.1538461538462 0.629485368728638
30.9358974358974 0.625230073928833
32.8205128205128 0.615431666374207
34.8205128205128 0.622273027896881
36.9423076923077 0.61098837852478
39.1923076923077 0.598002076148987
41.5833333333333 0.577151477336884
44.1153846153846 0.571718335151672
46.8076923076923 0.582086622714996
49.6602564102564 0.57509571313858
52.6858974358974 0.577062129974365
55.8974358974359 0.565696120262146
59.3076923076923 0.542023658752441
62.9230769230769 0.559756755828857
66.7564102564103 0.537396132946014
70.8269230769231 0.534067928791046
75.1474358974359 0.529752969741821
79.7307692307692 0.535503208637238
84.5897435897436 0.508808493614197
89.7435897435897 0.507130920886993
95.2179487179487 0.494038105010986
101.019230769231 0.490043848752975
107.179487179487 0.512650609016418
113.711538461538 0.478834360837936
120.641025641026 0.487378865480423
127.99358974359 0.485248863697052
135.801282051282 0.466144055128098
144.076923076923 0.473202496767044
152.858974358974 0.473405689001083
162.179487179487 0.440136641263962
172.064102564103 0.451612085103989
182.551282051282 0.470903813838959
193.679487179487 0.459606349468231
205.487179487179 0.459317296743393
218.012820512821 0.448282927274704
231.301282051282 0.45581778883934
245.403846153846 0.440941154956818
260.358974358974 0.433373808860779
276.230769230769 0.426142185926437
293.070512820513 0.458342224359512
310.935897435897 0.436917871236801
329.884615384615 0.441342204809189
350 0.428972959518433
};
\addplot [, color2, opacity=0.6, mark=triangle*, mark size=0.5, mark options={solid,rotate=180}, only marks]
table {%
1 0.673772037029266
1.05769230769231 0.670427680015564
1.12179487179487 0.669018864631653
1.19230769230769 0.596098184585571
1.26282051282051 0.578183829784393
1.33974358974359 0.552935242652893
1.42307692307692 0.576664865016937
1.51282051282051 0.540373623371124
1.6025641025641 0.561157524585724
1.69871794871795 0.598748564720154
1.80128205128205 0.58704686164856
1.91666666666667 0.600325524806976
2.03205128205128 0.604187548160553
2.15384615384615 0.582744956016541
2.28846153846154 0.577111661434174
2.42307692307692 0.583044469356537
2.57692307692308 0.54921954870224
2.73076923076923 0.517734289169312
2.8974358974359 0.538412094116211
3.07692307692308 0.530360400676727
3.26282051282051 0.518475115299225
3.46153846153846 0.538646280765533
3.67307692307692 0.523139655590057
3.8974358974359 0.526966273784637
4.13461538461539 0.511755406856537
4.38461538461539 0.534623086452484
4.65384615384615 0.506045460700989
4.93589743589744 0.524599254131317
5.23717948717949 0.505771219730377
5.55769230769231 0.505630612373352
5.8974358974359 0.48131650686264
6.25641025641026 0.465922832489014
6.64102564102564 0.498295247554779
7.04487179487179 0.467209070920944
7.47435897435897 0.479375213384628
7.92948717948718 0.459269732236862
8.41025641025641 0.471644669771194
8.92307692307692 0.44985967874527
9.46794871794872 0.446293771266937
10.0448717948718 0.460056662559509
10.6602564102564 0.43929448723793
11.3076923076923 0.421927094459534
12 0.445138335227966
12.7307692307692 0.426779538393021
13.5064102564103 0.425626486539841
14.3333333333333 0.44759538769722
15.2051282051282 0.424184948205948
16.1346153846154 0.444124847650528
17.1153846153846 0.435624539852142
18.1602564102564 0.436645835638046
19.2692307692308 0.414198517799377
20.4423076923077 0.416374742984772
21.6858974358974 0.425188362598419
23.0128205128205 0.426174879074097
24.4102564102564 0.41788986325264
25.9038461538462 0.430553048849106
27.4807692307692 0.426262229681015
29.1538461538462 0.431886672973633
30.9358974358974 0.437909156084061
32.8205128205128 0.429050445556641
34.8205128205128 0.414119243621826
36.9423076923077 0.431285321712494
39.1923076923077 0.421520352363586
41.5833333333333 0.397159159183502
44.1153846153846 0.406749367713928
46.8076923076923 0.394541651010513
49.6602564102564 0.3994020819664
52.6858974358974 0.416355133056641
55.8974358974359 0.398399978876114
59.3076923076923 0.426340699195862
62.9230769230769 0.409695386886597
66.7564102564103 0.41942772269249
70.8269230769231 0.401327192783356
75.1474358974359 0.446282267570496
79.7307692307692 0.398736566305161
84.5897435897436 0.394280105829239
89.7435897435897 0.40281417965889
95.2179487179487 0.44737297296524
101.019230769231 0.385272175073624
107.179487179487 0.395274430513382
113.711538461538 0.385266333818436
120.641025641026 0.460104703903198
127.99358974359 0.404287546873093
135.801282051282 0.379720479249954
144.076923076923 0.438384234905243
152.858974358974 0.352953255176544
162.179487179487 0.442953407764435
172.064102564103 0.401690274477005
182.551282051282 0.406183987855911
193.679487179487 0.374524146318436
205.487179487179 0.428814381361008
218.012820512821 0.444134026765823
231.301282051282 0.432431787252426
245.403846153846 0.45915213227272
260.358974358974 0.467318952083588
276.230769230769 0.437745124101639
293.070512820513 0.504992365837097
310.935897435897 0.540200591087341
329.884615384615 0.4277723133564
350 0.456889301538467
};
\addlegendentry{sub 16, mc 10}
\addplot [, color2, opacity=0.6, mark=triangle*, mark size=0.5, mark options={solid,rotate=180}, only marks, forget plot]
table {%
1 0.695609271526337
1.05769230769231 0.672130107879639
1.12179487179487 0.675661742687225
1.19230769230769 0.636198580265045
1.26282051282051 0.563960790634155
1.33974358974359 0.536933958530426
1.42307692307692 0.595403432846069
1.51282051282051 0.569731533527374
1.6025641025641 0.583801627159119
1.69871794871795 0.622017860412598
1.80128205128205 0.606498837471008
1.91666666666667 0.596440136432648
2.03205128205128 0.591050386428833
2.15384615384615 0.612375915050507
2.28846153846154 0.574909269809723
2.42307692307692 0.562505424022675
2.57692307692308 0.567353367805481
2.73076923076923 0.553649723529816
2.8974358974359 0.553119480609894
3.07692307692308 0.559594869613647
3.26282051282051 0.569319605827332
3.46153846153846 0.542696356773376
3.67307692307692 0.54191130399704
3.8974358974359 0.535565197467804
4.13461538461539 0.566004872322083
4.38461538461539 0.533773481845856
4.65384615384615 0.518930554389954
4.93589743589744 0.51916515827179
5.23717948717949 0.51658296585083
5.55769230769231 0.514919102191925
5.8974358974359 0.496425658464432
6.25641025641026 0.49590528011322
6.64102564102564 0.482799530029297
7.04487179487179 0.482264250516891
7.47435897435897 0.483423441648483
7.92948717948718 0.475480496883392
8.41025641025641 0.476391673088074
8.92307692307692 0.481682419776917
9.46794871794872 0.444243162870407
10.0448717948718 0.476368010044098
10.6602564102564 0.458612859249115
11.3076923076923 0.464119255542755
12 0.463186621665955
12.7307692307692 0.440640777349472
13.5064102564103 0.449529498815536
14.3333333333333 0.458343297243118
15.2051282051282 0.439853578805923
16.1346153846154 0.458953499794006
17.1153846153846 0.462312906980515
18.1602564102564 0.47301572561264
19.2692307692308 0.489200204610825
20.4423076923077 0.457235634326935
21.6858974358974 0.448245406150818
23.0128205128205 0.426530480384827
24.4102564102564 0.420814663171768
25.9038461538462 0.467424988746643
27.4807692307692 0.421521604061127
29.1538461538462 0.453004747629166
30.9358974358974 0.462682396173477
32.8205128205128 0.434674143791199
34.8205128205128 0.446554750204086
36.9423076923077 0.440309017896652
39.1923076923077 0.424336045980453
41.5833333333333 0.442124217748642
44.1153846153846 0.467358767986298
46.8076923076923 0.453441888093948
49.6602564102564 0.454780995845795
52.6858974358974 0.464075088500977
55.8974358974359 0.413880467414856
59.3076923076923 0.443905860185623
62.9230769230769 0.433857411146164
66.7564102564103 0.408841848373413
70.8269230769231 0.447876036167145
75.1474358974359 0.464158862829208
79.7307692307692 0.433071672916412
84.5897435897436 0.408720374107361
89.7435897435897 0.397412866353989
95.2179487179487 0.417119055986404
101.019230769231 0.418856352567673
107.179487179487 0.431296646595001
113.711538461538 0.39644980430603
120.641025641026 0.435819298028946
127.99358974359 0.38034588098526
135.801282051282 0.402549892663956
144.076923076923 0.390762746334076
152.858974358974 0.379458844661713
162.179487179487 0.463647753000259
172.064102564103 0.422785609960556
182.551282051282 0.473481744527817
193.679487179487 0.402032166719437
205.487179487179 0.418710350990295
218.012820512821 0.429531931877136
231.301282051282 0.405523657798767
245.403846153846 0.444546580314636
260.358974358974 0.467879623174667
276.230769230769 0.48124822974205
293.070512820513 0.42882838845253
310.935897435897 0.445739150047302
329.884615384615 0.467519521713257
350 0.456232607364655
};
\addplot [, color2, opacity=0.6, mark=triangle*, mark size=0.5, mark options={solid,rotate=180}, only marks, forget plot]
table {%
1 0.698930501937866
1.05769230769231 0.689102470874786
1.12179487179487 0.67279189825058
1.19230769230769 0.630267143249512
1.26282051282051 0.595996975898743
1.33974358974359 0.560366809368134
1.42307692307692 0.549517095088959
1.51282051282051 0.576935410499573
1.6025641025641 0.619231045246124
1.69871794871795 0.603411495685577
1.80128205128205 0.549823760986328
1.91666666666667 0.504163563251495
2.03205128205128 0.568620502948761
2.15384615384615 0.572974562644958
2.28846153846154 0.560099422931671
2.42307692307692 0.52781480550766
2.57692307692308 0.546152353286743
2.73076923076923 0.493750095367432
2.8974358974359 0.544755518436432
3.07692307692308 0.517202496528625
3.26282051282051 0.525796055793762
3.46153846153846 0.552193999290466
3.67307692307692 0.543900191783905
3.8974358974359 0.570367097854614
4.13461538461539 0.556070446968079
4.38461538461539 0.506485641002655
4.65384615384615 0.539070904254913
4.93589743589744 0.523009181022644
5.23717948717949 0.509635150432587
5.55769230769231 0.491362988948822
5.8974358974359 0.475119680166245
6.25641025641026 0.487640827894211
6.64102564102564 0.495582431554794
7.04487179487179 0.465195924043655
7.47435897435897 0.464538246393204
7.92948717948718 0.480458527803421
8.41025641025641 0.452559262514114
8.92307692307692 0.451365947723389
9.46794871794872 0.461935937404633
10.0448717948718 0.463208794593811
10.6602564102564 0.469166696071625
11.3076923076923 0.433526605367661
12 0.439141273498535
12.7307692307692 0.464939951896667
13.5064102564103 0.422924220561981
14.3333333333333 0.432341367006302
15.2051282051282 0.423585206270218
16.1346153846154 0.459090560674667
17.1153846153846 0.412669122219086
18.1602564102564 0.417983740568161
19.2692307692308 0.389219015836716
20.4423076923077 0.36629444360733
21.6858974358974 0.446053922176361
23.0128205128205 0.412205725908279
24.4102564102564 0.430034935474396
25.9038461538462 0.392484813928604
27.4807692307692 0.414143443107605
29.1538461538462 0.432073503732681
30.9358974358974 0.403451949357986
32.8205128205128 0.43868887424469
34.8205128205128 0.396438449621201
36.9423076923077 0.424502670764923
39.1923076923077 0.423555672168732
41.5833333333333 0.385935813188553
44.1153846153846 0.397460699081421
46.8076923076923 0.43653815984726
49.6602564102564 0.420291423797607
52.6858974358974 0.421494334936142
55.8974358974359 0.387515634298325
59.3076923076923 0.393099218606949
62.9230769230769 0.402090966701508
66.7564102564103 0.417899191379547
70.8269230769231 0.414456397294998
75.1474358974359 0.392827302217484
79.7307692307692 0.386216282844543
84.5897435897436 0.438449084758759
89.7435897435897 0.413785696029663
95.2179487179487 0.421494513750076
101.019230769231 0.400592446327209
107.179487179487 0.374116092920303
113.711538461538 0.421774595975876
120.641025641026 0.407871395349503
127.99358974359 0.439810246229172
135.801282051282 0.378792285919189
144.076923076923 0.457066714763641
152.858974358974 0.431208550930023
162.179487179487 0.389964044094086
172.064102564103 0.444655299186707
182.551282051282 0.410665512084961
193.679487179487 0.42384946346283
205.487179487179 0.424982905387878
218.012820512821 0.420777887105942
231.301282051282 0.436174154281616
245.403846153846 0.448300570249557
260.358974358974 0.448703289031982
276.230769230769 0.470759242773056
293.070512820513 0.381782382726669
310.935897435897 0.423959255218506
329.884615384615 0.499769061803818
350 0.433991551399231
};
\addplot [, color2, opacity=0.6, mark=triangle*, mark size=0.5, mark options={solid,rotate=180}, only marks, forget plot]
table {%
1 0.670502662658691
1.05769230769231 0.679176688194275
1.12179487179487 0.680385887622833
1.19230769230769 0.606757581233978
1.26282051282051 0.569440901279449
1.33974358974359 0.559368491172791
1.42307692307692 0.610257387161255
1.51282051282051 0.55231124162674
1.6025641025641 0.576520919799805
1.69871794871795 0.613166034221649
1.80128205128205 0.567158102989197
1.91666666666667 0.611656844615936
2.03205128205128 0.576766133308411
2.15384615384615 0.55899441242218
2.28846153846154 0.53368091583252
2.42307692307692 0.52361935377121
2.57692307692308 0.534695267677307
2.73076923076923 0.526253283023834
2.8974358974359 0.524928569793701
3.07692307692308 0.498880952596664
3.26282051282051 0.533529698848724
3.46153846153846 0.507850050926208
3.67307692307692 0.498744279146194
3.8974358974359 0.483945369720459
4.13461538461539 0.506835043430328
4.38461538461539 0.495109915733337
4.65384615384615 0.492871850728989
4.93589743589744 0.51047295331955
5.23717948717949 0.492855787277222
5.55769230769231 0.502688467502594
5.8974358974359 0.470352649688721
6.25641025641026 0.454829096794128
6.64102564102564 0.478559672832489
7.04487179487179 0.447901576757431
7.47435897435897 0.443135529756546
7.92948717948718 0.447617471218109
8.41025641025641 0.440342903137207
8.92307692307692 0.440627664327621
9.46794871794872 0.44295409321785
10.0448717948718 0.425924748182297
10.6602564102564 0.432428419589996
11.3076923076923 0.442138344049454
12 0.414854884147644
12.7307692307692 0.411547273397446
13.5064102564103 0.385312765836716
14.3333333333333 0.457138657569885
15.2051282051282 0.399558484554291
16.1346153846154 0.41151374578476
17.1153846153846 0.415000975131989
18.1602564102564 0.397609442472458
19.2692307692308 0.429339677095413
20.4423076923077 0.414062827825546
21.6858974358974 0.422547221183777
23.0128205128205 0.401458352804184
24.4102564102564 0.387716591358185
25.9038461538462 0.392489016056061
27.4807692307692 0.375306308269501
29.1538461538462 0.386103481054306
30.9358974358974 0.372717320919037
32.8205128205128 0.389598309993744
34.8205128205128 0.391054004430771
36.9423076923077 0.392924189567566
39.1923076923077 0.388837605714798
41.5833333333333 0.362038880586624
44.1153846153846 0.371655941009521
46.8076923076923 0.370505601167679
49.6602564102564 0.360549300909042
52.6858974358974 0.368824541568756
55.8974358974359 0.356364637613297
59.3076923076923 0.365491092205048
62.9230769230769 0.360013723373413
66.7564102564103 0.385095804929733
70.8269230769231 0.445562213659286
75.1474358974359 0.410032391548157
79.7307692307692 0.355321377515793
84.5897435897436 0.342453002929688
89.7435897435897 0.377739489078522
95.2179487179487 0.408435434103012
101.019230769231 0.397172123193741
107.179487179487 0.386269867420197
113.711538461538 0.3708675801754
120.641025641026 0.340962409973145
127.99358974359 0.437822461128235
135.801282051282 0.370002210140228
144.076923076923 0.426039427518845
152.858974358974 0.414626151323318
162.179487179487 0.40787273645401
172.064102564103 0.414591819047928
182.551282051282 0.420837461948395
193.679487179487 0.433156818151474
205.487179487179 0.48987689614296
218.012820512821 0.432803601026535
231.301282051282 0.398980587720871
245.403846153846 0.451790601015091
260.358974358974 0.462767004966736
276.230769230769 0.424896001815796
293.070512820513 0.438621044158936
310.935897435897 0.487403482198715
329.884615384615 0.464087843894958
350 0.434723913669586
};
\addplot [, color2, opacity=0.6, mark=triangle*, mark size=0.5, mark options={solid,rotate=180}, only marks, forget plot]
table {%
1 0.683347463607788
1.05769230769231 0.697467863559723
1.12179487179487 0.678545653820038
1.19230769230769 0.640260219573975
1.26282051282051 0.583135902881622
1.33974358974359 0.57809990644455
1.42307692307692 0.58257532119751
1.51282051282051 0.56199723482132
1.6025641025641 0.569940745830536
1.69871794871795 0.546406328678131
1.80128205128205 0.545152366161346
1.91666666666667 0.542364656925201
2.03205128205128 0.541296362876892
2.15384615384615 0.581005692481995
2.28846153846154 0.515447020530701
2.42307692307692 0.519599795341492
2.57692307692308 0.518295705318451
2.73076923076923 0.518819332122803
2.8974358974359 0.53775018453598
3.07692307692308 0.537723958492279
3.26282051282051 0.55036199092865
3.46153846153846 0.528587758541107
3.67307692307692 0.510694205760956
3.8974358974359 0.536922454833984
4.13461538461539 0.550985336303711
4.38461538461539 0.497310310602188
4.65384615384615 0.543574154376984
4.93589743589744 0.517009139060974
5.23717948717949 0.522975564002991
5.55769230769231 0.527230739593506
5.8974358974359 0.512312352657318
6.25641025641026 0.450761944055557
6.64102564102564 0.468281388282776
7.04487179487179 0.463147848844528
7.47435897435897 0.451735019683838
7.92948717948718 0.466011345386505
8.41025641025641 0.475084573030472
8.92307692307692 0.460682719945908
9.46794871794872 0.452530443668365
10.0448717948718 0.449616551399231
10.6602564102564 0.430477887392044
11.3076923076923 0.42636850476265
12 0.43417552113533
12.7307692307692 0.430974334478378
13.5064102564103 0.402008205652237
14.3333333333333 0.421458661556244
15.2051282051282 0.415281593799591
16.1346153846154 0.426760703325272
17.1153846153846 0.423732936382294
18.1602564102564 0.395532071590424
19.2692307692308 0.383279651403427
20.4423076923077 0.413489758968353
21.6858974358974 0.433670789003372
23.0128205128205 0.402082622051239
24.4102564102564 0.40100309252739
25.9038461538462 0.395782619714737
27.4807692307692 0.402439653873444
29.1538461538462 0.388124376535416
30.9358974358974 0.385205060243607
32.8205128205128 0.41234427690506
34.8205128205128 0.399581134319305
36.9423076923077 0.390341550111771
39.1923076923077 0.37975400686264
41.5833333333333 0.393030852079391
44.1153846153846 0.365261226892471
46.8076923076923 0.363742589950562
49.6602564102564 0.419954895973206
52.6858974358974 0.374307543039322
55.8974358974359 0.380428731441498
59.3076923076923 0.383376926183701
62.9230769230769 0.380914837121964
66.7564102564103 0.431321710348129
70.8269230769231 0.35054150223732
75.1474358974359 0.369643241167068
79.7307692307692 0.418015599250793
84.5897435897436 0.374144583940506
89.7435897435897 0.35032045841217
95.2179487179487 0.401550590991974
101.019230769231 0.398025125265121
107.179487179487 0.383770972490311
113.711538461538 0.368026167154312
120.641025641026 0.374317318201065
127.99358974359 0.37986159324646
135.801282051282 0.372506022453308
144.076923076923 0.389126271009445
152.858974358974 0.446982264518738
162.179487179487 0.478278577327728
172.064102564103 0.4248286485672
182.551282051282 0.37523627281189
193.679487179487 0.447574198246002
205.487179487179 0.445245653390884
218.012820512821 0.370043635368347
231.301282051282 0.468013375997543
245.403846153846 0.461858212947845
260.358974358974 0.500148236751556
276.230769230769 0.462519317865372
293.070512820513 0.429401934146881
310.935897435897 0.459376066923141
329.884615384615 0.49083885550499
350 0.454435646533966
};
\end{axis}

\end{tikzpicture}

      \tikzexternaldisable
    \end{minipage}
  \end{subfigure}
  \caption{\textbf{\bfvivit{} versus full-batch \ggn (2).} Overlap between the
    top-$C$ eigenspaces of different \ggn approximations and the full-batch \ggn
    during training for all test problems. Each approximation is evaluated on
    $5$ different mini-batches.} \label{vivit::fig:vivit_vs_full_batch_ggn_2}
\end{figure*}

%%% Local Variables:
%%% mode: latex
%%% TeX-master: "../../thesis"
%%% End:


% ViViT versus mini-batch GGN
%
%========== approximations versus mini-batch GGN
\subsubsection{Procedure (2)}
Since \vivit{}'s \ggn approximations using curvature sub-sampling and/or the MC
approximation (the cases \textbf{mb, mc} as well as \textbf{sub, exact} and
\textbf{sub, mc} in \Cref{vivit::tab:cases_full_batch}) are based on the
\textit{mini}-batch \ggn{}, we cannot expect them to perform better than this
baseline. We thus repeat the analysis from above but use the mini-batch \ggn
with batch-size $N=128$ as ground truth instead of the full-batch \ggn. Of
course, the mini-batch reference top-$C$ eigenspace is always evaluated on the
same mini-batch as the approximation.

\subsubsection{Results (2)}

\Cref{vivit::fig:vivit_vs_mini_batch_ggn} shows the results. Over large parts of
the optimization (note the log scale for the epoch-axis), the \mc approximation
seems to be better suited than curvature sub-sampling (which has comparable
computational cost). For the \cifarhun \allcnnc problem, the \mc approximation
stands out particularly early from the other approximations and consistently
yields higher overlaps with the mini-batch \ggn.

\newcommand{\plotEigspaceVivitvsMB}[3]{
  % defines the pgfplots style "eigspacedefault"
\pgfkeys{/pgfplots/eigspacedefault/.style={
    width=1.0\linewidth,
    height=0.6\linewidth,
    every axis plot/.append style={line width = 1.5pt},
    tick pos = left,
    ylabel near ticks,
    xlabel near ticks,
    xtick align = inside,
    ytick align = inside,
    legend cell align = left,
    legend columns = 4,
    legend pos = south east,
    legend style = {
      fill opacity = 1,
      text opacity = 1,
      font = \footnotesize,
      at={(1, 1.025)},
      anchor=south east,
      column sep=0.25cm,
    },
    legend image post style={scale=2.5},
    xticklabel style = {font = \footnotesize},
    xlabel style = {font = \footnotesize},
    axis line style = {black},
    yticklabel style = {font = \footnotesize},
    ylabel style = {font = \footnotesize},
    title style = {font = \footnotesize},
    grid = major,
    grid style = {dashed}
  }
}

\pgfkeys{/pgfplots/eigspacedefaultapp/.style={
    eigspacedefault,
    height=0.6\linewidth,
    legend columns = 2,
  }
}

\pgfkeys{/pgfplots/eigspacenolegend/.style={
    legend image post style = {scale=0},
    legend style = {
      fill opacity = 0,
      draw opacity = 0,
      text opacity = 0,
      font = \footnotesize,
      at={(1, 1.025)},
      anchor=south east,
      column sep=0.25cm,
    },
  }
}
%%% Local Variables:
%%% mode: latex
%%% TeX-master: "../../thesis"
%%% End:

  \pgfkeys{/pgfplots/zmystyle/.style={
      eigspacedefault
    }}
  \input{../../fig/exp13_full_batch_monitoring/results/plots/eigspace_vivit_vs_mb/#1_#2_#3_plot}
}

\begin{figure}[p]
\centering
\begin{minipage}{0.50\textwidth}
\centering
\textbf{\fmnist \twoctwod \sgd}\\[1mm]
\plotEigspaceVivitvsMB{fmnist}{2c2d}{sgd}
% \includegraphics[scale=1.0]{fig/exp13_plots/eigspace_vivit_vs_mb/fmnist_2c2d_sgd_plot.pdf}
\end{minipage}\hfill
\begin{minipage}{0.50\textwidth}
\centering
\textbf{\fmnist \twoctwod \adam}\\[1mm]
\plotEigspaceVivitvsMB{fmnist}{2c2d}{adam}
% \includegraphics[scale=1.0]{fig/exp13_plots/eigspace_vivit_vs_mb/fmnist_2c2d_adam_plot.pdf}
\end{minipage}

\begin{minipage}{0.50\textwidth}
\centering
\textbf{\cifarten \threecthreed \sgd}\\[1mm]
\plotEigspaceVivitvsMB{cifar10}{3c3d}{sgd}
% \includegraphics[scale=1.0]{fig/exp13_plots/eigspace_vivit_vs_mb/cifar10_3c3d_sgd_plot.pdf}
\end{minipage}\hfill
\begin{minipage}{0.50\textwidth}
\centering
\textbf{\cifarten \threecthreed \adam}\\[1mm]
\plotEigspaceVivitvsMB{cifar10}{3c3d}{adam}
% \includegraphics[scale=1.0]{fig/exp13_plots/eigspace_vivit_vs_mb/cifar10_3c3d_adam_plot.pdf}
\end{minipage}

\begin{minipage}{0.50\textwidth}
\centering
\textbf{\cifarten \resnetthirtytwo \sgd}\\[1mm]
\plotEigspaceVivitvsMB{cifar10}{resnet32}{sgd}
% \includegraphics[scale=1.0]{fig/exp13_plots/eigspace_vivit_vs_mb/cifar10_resnet32_sgd_plot.pdf}
\end{minipage}\hfill
\begin{minipage}{0.50\textwidth}
\centering
\textbf{\cifarten \resnetthirtytwo \adam}\\[1mm]
\plotEigspaceVivitvsMB{cifar10}{resnet32}{adam}
% \includegraphics[scale=1.0]{fig/exp13_plots/eigspace_vivit_vs_mb/cifar10_resnet32_adam_plot.pdf}
\end{minipage}

\vspace{3mm}

\begin{minipage}{0.50\textwidth}
\centering
\textbf{\cifarhun \allcnnc \sgd}\\[1mm]
\plotEigspaceVivitvsMB{cifar100}{allcnnc}{sgd}
% \includegraphics[scale=1.0]{fig/exp13_plots/eigspace_vivit_vs_mb/cifar100_allcnnc_sgd_plot.pdf}
\end{minipage}\hfill
\begin{minipage}{0.50\textwidth}
\centering
\textbf{\cifarhun \allcnnc \adam}\\[1mm]
\plotEigspaceVivitvsMB{cifar100}{allcnnc}{adam}
% \includegraphics[scale=1.0]{fig/exp13_plots/eigspace_vivit_vs_mb/cifar100_allcnnc_adam_plot.pdf}
\end{minipage}

\caption{\textbf{\bfvivit{} vs. mini-batch \ggn{}:}
Overlap between the top-$C$ eigenspaces of different \ggn approximations and the mini-batch \ggn during training for all test problems.
Each approximation is evaluated on $5$ different mini-batches.}
\label{fig:vivit_vs_mini_batch_ggn}
\end{figure}

%%% Local Variables:
%%% mode: latex
%%% TeX-master: "../main"
%%% End:


%%% Local Variables:
%%% mode: latex
%%% TeX-master: "../thesis"
%%% End:
