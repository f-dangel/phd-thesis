The previous section focused on structure in the empirical risk---like its sum
structure, and interpretations of risk-based learning for specific loss
functions---without assumptions about the model $f_{\vtheta}$. This section
introduces structure in the model. While DNNs are generally highly
over-parameterized, they usually rely on relatively simple components and
construction principles.

This manuscript considers sequential feedforward neural networks that consist of
layers. They comprise ``classic'' architectures like multi-layer perceptrons
(MLPs), convolutional neural networks (CNNs), and established architectures like
VGG~\cite{simonyan2015deep}, and ResNets~\cite{he2016deep}. Most of the
discussion in this text also applies to other architectures, but would require
heavier notation.

\subsection{Layer-wise Notation}

\tikzexternalenable
\begin{figure*}[!t]
  \centering \resizebox{\linewidth}{!}{ {\footnotesize
      % basic setting of a fully-connected neural network with data flow for
% forward pass

\begin{tikzpicture}
  % first two layers
  \node (in1)
  [inner sep=0]
  {\tikz \drawMessagesWithArrows{$\vz^{(0)}$}{ }{ }{\hNodeDistance};};
  \node (layer1)
  [anchor=south west, inner sep=0]
  at (in1.south east)
  {\tikz \drawModuleWithParams{$f^{(1)}_{\vtheta^{(1)}}$}{16}{$\vtheta^{(1)}$}{ }{ };};
  \node (out1)
  [inner sep=0, anchor=south west]
  at (layer1.south east)
  {\tikz \drawMessagesWithArrows{$\vz^{(1)}$}{ }{ }{\hNodeDistance};};
  \node (layer2)
  [inner sep=0pt, anchor=south west]
  at (out1.south east)
  {\tikz \drawModuleWithParams{$f^{(2)}_{\vtheta^{(2)}}$}{16}{$\vtheta^{(2)}$}{ }{ };};

  % dots with messages
  \node (in2)
  [inner sep=0, anchor=south west]
  at (layer2.south east)
  {\tikz \drawMessagesWithArrows{$\vz^{(2)}$}{ }{ }{\hNodeDistance};};
  \node (dots)
  [xshift=0.75ex, inner sep=0pt, anchor=west]
  at (in2.east)
  {$\dots$};

  \node (inLast)
  [xshift=0.75ex, inner sep=0pt, anchor=west]
  at (dots.east)
  {\tikz \drawMessagesWithArrows{$\vz^{(L-1)}$}{ }{ }{\hNodeDistance};};

  \node (layerLast)
  [anchor=south west, inner sep=0]
  at (inLast.south east)
  {\tikz \drawModuleWithParams{$f^{(L)}_{\vtheta^{(L)}}$}{16}{$\vtheta^{(L)}$}{ }{ };};
  \node (outLast)
  [inner sep=0, anchor=south west]
  at (layerLast.south east)
  {\tikz \drawMessagesWithArrows{$\vz^{(L)}$}{ }{ }{\hNodeDistance};};
\end{tikzpicture}

%%% Local Variables:
%%% mode: latex
%%% TeX-master: "../../thesis"
%%% End:
}}
  \caption{\textbf{Forward pass of a sequential feedforward neural network
      (\Cref{eq:background::neuralNetwork}).} The computational graph indicates
    the data flow and dependencies of intermediate
    variables.}\label{fig:background::neuralNetwork}
\end{figure*}
\tikzexternaldisable

Sequential feedforward neural networks of depth $L$ consist of \emph{modules},
or \emph{layers}, $f^{(l)}_{\vtheta^{(l)}}, l = 1, \ldots, L$, stacked on top of
each other such that
\begin{subequations}\label{eq:background::neuralNetworkAndModule}
  \begin{align}
    \label{eq:background::neuralNetwork}
    f_{\vtheta}
    =
    f^{(L)}_{\vtheta^{(L)}}
    \circ
    f^{(L-1)}_{\vtheta^{(L-1)}}
    \circ
    \ldots
    \circ
    f^{(1)}_{\vtheta^{(1)}}
  \end{align}
  They map input features $\vx =: \vz^{(0)}$ to predictions $f_{\vtheta}(\vx) =:
  \vz^{(L)}$ via a sequence of intermediate hidden features $\vz^{(1)}, \dots,
  \vz^{(L-1)}$. In a forward pass, a module $f^{(l)}_{\vtheta^{(l)}}$ receives the
  parental output $\vz^{(l-1)} \in \sR^{h^{(l-1)}}$ and applies an operation with
  (potentially empty) parameters $\vtheta^{(l)} \in \sR^{d^{(l)}}$,
  \begin{align}
    \label{eq:background::Module}
    \vz^{(l)}
    =
    f^{(l)}_{\vtheta^{(l)}}(\vz^{(l-1)})\,.
  \end{align}
\end{subequations}
The
output features $\vz^{(l)} \in \sR^{h^{(l)}}$ serve as input to the next layer
$l+1$. This builds up dependencies in form of the computational graph shown in
\Cref{fig:background::neuralNetwork} that maps the leaf nodes $\vz^{(0)}$ and
$\vtheta^{(1)}, \dots, \vtheta^{(L)}$ to the prediction $\vz^{(L)}$. The neural
network parameters are often treated as a single vector $\vtheta \in \sR^{D}$
which results from layer-wise concatenation,%
\marginnote{%
  \begin{definition}[\textbf{Tensor flattening}]\label{def:background::Flattening}
    Let $\tA \in \sR^{n_1 \times n_2 \times \dots, \times n_m}$ denote a tensor
    of rank $m$ with dimensions $n_1, n_2, \dots, n_{m}$.
    The flattened tensor $\vec(\tA) \in \sR^{n_1 n_2 \cdots n_m}$ is a vector
    that concatenates $\tA$'s elements in a
    first-index-varies-fastest fashion,
    \begin{equation}\label{eq:background::Flattening}
      \vec(\tA)
      =
      \begin{pmatrix}
        \etens{A}_{1,1,1,\dots,1}
        \\
        \etens{A}_{2,1,1,\dots,1}
        \\
        \vdots
        \\
        \etens{A}_{n_1,1,1,\dots,1}
        \\
        \etens{A}_{1,2,1,\dots,1}
        \\
        \vdots
        \\
        \etens{A}_{n_1,2,1,\dots,1}
        \\
        \vdots
        \\
        \etens{A}_{n_1, n_2, n_3, \dots, n_m}
      \end{pmatrix}\,.
    \end{equation}
    The matrix case $m=2$ corresponds to column-stacking. Flattening a vector
    $\va$ leaves it unaffected, \ie $\vec(\va) = \va$.
  \end{definition}}%
\begin{align}
  \label{eq:background::neuralNetworkParameters}
  \vtheta
  =
  \begin{pmatrix}
    \vtheta^{(1)}
    \\
    \vtheta^{(2)}
    \\
    \vdots
    \\
    \vtheta^{(L)}
  \end{pmatrix}\,.
\end{align}
To simplify the presentation, \Cref{eq:background::Module} assumes vector-shaped
quantities. However, many neural networks process higher-dimensional data like images,
represented by tensors. Sometimes, the tensor structure is convenient to work
with. One can convert between the tensor and vector view without loss of
generality by introducing conventions for tensor flattening
(\Cref{def:background::Flattening}) and reshaping
(\Cref{def:background::TensorReshape})%
\marginnote{%
  \begin{definition}[\textbf{Vector reshaping}]\label{def:background::TensorReshape}
    Let $\va\in \sR^{n_1 n_2 \cdots n_m}$ be a vector. Reshaping that vector
    into a rank-$m$ tensor of shape $S = (n_1, n_2, \dots, n_m)$ happens by
    filling $\va$'s elements into the tensor in a first-index-varies-fastest
    fashion,
    \begin{subequations}\label{eq:background::TensorReshape}
      \begin{align}
        \tA &= \reshape_{S}(\va)
      \end{align}
      with elements
      \begin{align}
        \begin{split}
          \etA_{1, 1, 1, \dots, 1} &= \eva_1\,,
          \\
          \etA_{2, 1, 1, \dots, 1} &= \eva_2\,,
          \\
                                   &\vdots
          \\
          \etA_{n_1, 1, 1, \dots, 1} &= \eva_{n_1}\,,
          \\
          \etA_{1, 2, 1, \dots, 1} &= \eva_{n_1 + 1}\,,
          \\
                                   &\vdots
          \\
          \etA_{n_1, 2, 1, \dots, 1} &= \eva_{2 n_1}\,,
          \\
                                   &\vdots
          \\
          \hspace{-2ex}\etA_{n_1, n_2, n_3, \dots, n_m} &= \eva_{n_1 n_2 n_3 \cdots n_m}\,.
        \end{split}
      \end{align}
    \end{subequations}
    The matrix case $m=2$ corresponds to column-filling. With tensor flattening
    (\Cref{def:background::Flattening}) this allows to define tensor reshaping:
    A tensor $\tB_1$ of shape $S_1$ is rearranged into any tensor $\tB_2$ of
    compatible shape $S_2$ by first flattening, then reshaping it, \ie $\tB_2 :=
    \reshape_{S_2}(\tB_{1}) := \reshape_{S_2}(\vec \tB_{1})$.
  \end{definition}
  }%
%
; after all, multi-dimensional arrays are represented in a vector format in
memory.

However, there exist different flattening conventions. Implementations often
favor row-major ordering. This manuscript uses (the more common in literature)
column-major order as it allows for elegant generalizations of derivative
concepts for multi-variate functions, like the Jacobian
(\Cref{hbp::def:generalizedJacobian}), the Hessian
(\Cref{hbp::def:generalizedHessian}), and their chain rules
(\Cref{hbp::the:chainRuleJacobians,hbp::the:chainRuleHessians}). To translate
analytical results into implementations, it is crucial to be aware of these
differing conventions.

\subsection{Modularity \& Common
  Operations}\label{sec:background::CommonOperations}

An important strength of deep learning is its \emph{modularity}. ML libraries
provide a large number of operations, or modules, that can be combined in almost
arbitrary ways through function composition, like in
\Cref{eq:background::neuralNetwork}. Training the resulting models with
first-order methods remains simple because their gradient can be automatically
computed via AD (\Cref{sec:background::GradientBackpropagation}). New operations
can easily be added because its implementation is decoupled to the modular
level.

% What is a module?
Modules are vaguely defined. Often, multiple operations that form a logical
processing unit in a neural network are grouped into a single module, \eg an MLP
layer combines affine transformation and elementwise activation (see below). In
extreme cases, even an entire neural network can be considered a single module
that can be used in other neural networks; \eg the neural network in
\Cref{fig:background::ModelLossFunctionSplit} resembles a single layer in
\Cref{fig:background::neuralNetwork} and could act as one layer in a larger
network.

% Which perspective are we choosing?
For theoretical analyses, it is preferable to consider units with a small number
of operations as modules. This, however, is inconvenient for constructing large
architectures, where many operations are grouped into higher-level units. This
text adapts a rather fine-grained view on modules that is close to their
implementation in ML libraries like \pytorch.

A common categorization for modules distinguishes trainable functions with
parameters, and parameter-free operations. \Cref{tab:background::forward} lists
the forward passes of common operations that will be illustrated in the
following presentation of network architectures. To distringuish more clearly
between input and output of an operation, the notation uses the symbols $\vx,
\vz$ for module input and output instead of $\vz^{(l-1)}, \vz^{(l)}$, and
neglects the layer superscript for the parameters, writing $\vtheta$ instead of
$\vtheta^{(l)}$.

\begin{table}[!t]
  \caption{\textbf{Forward pass for common modules used in feedforward
      networks.} Input and output are denoted $\vx, \vz$ rather than $\vz^{(l)},
    \vz^{(l+1)}$ to avoid clutter. $\mI$ is the identity matrix. Bold upper-case
    symbols ($\mW, \mX, \mZ, \dots$) denote matrices and bold upper-case sans
    serif symbols ($\tW, \tX, \tZ, \dots$) denote tensors. See
    \Cref{hbp::sec:examples_fcnn,hbp::sec:examples_loss,hbp::sec:examples_cnn}
    for details, and
    \Cref{tab:background::Jacobians,hbp::table:backpropEquations} for extended
    versions of this table for the backward, and Hessian backward, pass.}
  \label{tab:background::forward}
  \centering
  \begin{footnotesize}
    \begin{tabular}{ll}
      \toprule
      \textbf{OPERATION} & \textbf{FORWARD}
      \\
      \midrule
      % matrix-vector multiplication
      Matrix-vector multiplication & $\vz(\vx, \mW) = \mW\vx$
      \\
      % matrix-matrix multiplication
      Matrix-matrix multiplication & $\mZ(\mX, \mW) = \mW\mX$
      \\
      % addition
      Addition & $\vz(\vx, \vb) = \vx + \vb$
      \\
      % elementwise activation
      Elementwise activation & $\vz(\vx) = \vphi(\vx)$\,,\ \text{s.t.} $z_i(\vx) = \phi(x_i)$
      \\
      \midrule
      % residual unit/skip-connection
      Skip-connection & $\vz(\vx, \vtheta) = \vx + \vs(\vx, \vtheta)$
      \\
      \midrule
      % reshape/view operation
      Reshape/view & $\tZ(\tX)= \mathrm{reshape}(\tX)$
      \\
      % extraction operator
      Index select/map $\pi$ & $\vz(\vx) = \mPi \vx\, ,$ $\emPi_{j,\pi(j)} = 1\,, $
      \\
      % convolution
      Convolution & $\tZ(\tX, \tW) = \tX \star \tW$\,,
      \\
                         & $\mZ(\mW, \llbracket\tX\rrbracket) = \mW \llbracket \tX \rrbracket$\,,
      \\
      \midrule
      % square loss
      Square loss & $\ell(\vf, \vy) = \nicefrac{1}{C} (\vy-\vf)^\top (\vy - \vf)$ \\
      % cross-entropy loss
      Softmax cross-entropy & $\ell(\vf, y) = - \onehot(y)^\top \log\left[\vp(\vf)\right]$
      \\
      \bottomrule
    \end{tabular}
  \end{footnotesize}
\end{table}

\subsubsection{Deep Linear Networks \& Multi-layer Perceptrons (MLPs)}

\emph{Linear layers} process inputs $\vx$ by affine transformation, \ie multiplication
with a weight matrix $\mW$, followed by addition of a bias vector $\vb$,
\begin{align}\label{eq:background::LinearLayer}
  \vz = \mW \vx + \vb
  \qquad
  \text{where}
  \qquad
  \vtheta =
  \begin{pmatrix}
    {(\vec \mW)}^{\top} & {\vb}^{\top}
  \end{pmatrix}^{\top}\,.
\end{align}
They are also referred to as \emph{fully-connected} layers, because each output
$z_i$ depends on all inputs $\vx$ through $\mW_{i,:}$ and $\evb_i$.

\emph{Deep linear networks} %
\marginnote{%
  \begin{example}[\textbf{Deep linear
      network}]\label{ex:background::deepLinearNetwork}
    In notation of \Cref{eq:background::neuralNetworkAndModule}, a deep linear
    network of depth $L$ reads
    \begin{align*}
      \vz^{(l)}
      &=
        \mW^{(l)} \vz^{(\ell-1)} + \vb^{(l)}
        \qquad
        \shortintertext{where}
        \qquad
        \vtheta^{(l)}
      &=
        \begin{pmatrix}
          {\left(\vec \mW^{(l)}\right)}^{\top} & {\vb^{(l)}}^{\top}
        \end{pmatrix}^{\top}\,,
      \\
      l &= 1, \dots, L\,.
    \end{align*}
  \end{example}
}%
(\Cref{ex:background::deepLinearNetwork}) consist of
only linear layers and are of interest for analytical
studies~\cite[\eg][]{saxe2014exact,mulayoff2020unique,bernacchia2018exact} as
they are somewhat tractable. They describe a linear feature map, \ie a linear
function \wrt the input $\vz^{(0)}$, that is non-linear in the parameters.
Therefore, such networks are as expressive as a single linear layer, but highly
overparameterized.

A common technique to turn a deep linear network into a non-linear feature map
is to interlace affine transformations with non-linear
activations~\cite{rosenblatt1958perceptron}. An \emph{activation layer} $\vphi$
acts elementwise on its input, \ie applies the same function $\phi$ to each
input element,
\begin{align*}
  \vz = \vphi(\vx)\qquad \text{such that}\qquad \evz_i = \phi(\evx_i)\,.
\end{align*}
There exist many activations (ReLU, sigmoid, $\tanh$, \etc \cite[Chapter
6]{goodfellow2016deep}), and recent works proposing new choices (\eg squared
ReLU~\citep{so2021searching}).

\emph{Multi-layer perceptrons} %
\marginnote{%
  \begin{example}[\textbf{Multi-layer perceptron (MLP)}]\label{ex:background::MLP}
    In terms of \Cref{eq:background::neuralNetworkAndModule}, an MLP of depth
    $L$ reads
    \begin{align*}
      \vz^{(l)}
      &=
        \vphi^{(l)}\left( \mW^{(l)} \vz^{(\ell-1)} + \vb^{(l)}\right)\,,
        \shortintertext{where}
        \vtheta^{(l)}
      &=
        \begin{pmatrix}
          {\left(\vec \mW^{(l)}\right)}^{\top} & {\vb^{(l)}}^{\top}
        \end{pmatrix}^{\top}\,,
      \\
      l &= 1, \dots, L\,,
      \\
      \vphi^{(L)} &= \mathrm{id}
    \end{align*}
    and $\mathrm{id}$ denotes the identity.
  \end{example}
}%
(MLPs, \Cref{ex:background::MLP}) combine affine transformation and activation
in a layer. Activation functions $\vphi^{(l)}$ may vary between layers, but are
often chosen identically, with no activation in the last layer. One way to
interpret this design is that the mapping $\vz^{(0)} \mapsto \vz^{(L-1)}$ acts
as non-linear feature transformation, and the last layer $\vz^{(L-1)} \mapsto
\vz^{(L)}$ as linear classifier for the learned features.

\subsubsection{Convolutional Neural Networks (CNNs)}

CNNs represent an important neural network architecture revolution and were the
first class of deep neural network to beat ``classical'' methods on the
\imagenet
competition~\cite{deng2009imagenet,krizhevsky2012imagenet,russakovsky2015imagenet}.
Broadly speaking, such networks contain convolutional layers with trainable
parameters.

\paragraph{Convolutional layers:} Convolutions process multi-channel input
features such as images and are parameterized by a kernel that can be imagined
as a filter for patterns like edges, corners, \etc During the convolution
operation, the kernel slides over the input features and produces an output
element by contraction with the overlapping elements of the image. In most
cases, each output channel is also shifted by a bias parameter which will be
neglected in this presentation for simplicity (detailed discussion in
\Cref{hbp::sec:examples_cnn}, example in \Cref{hbp::equ:convolutionExample}).
Because the kernel moves over the image, it can detect patterns irrespective of
their position. The process can be adjusted with various hyperparameters, such
as stride, padding, groups, dilation
(see~\cite{dumoulin2016ConvolutionArithmeticGuide} for a visual guide).

In contrast to the linear layer (\Cref{eq:background::LinearLayer}) where each
output is connected to all inputs via independent rows of the weight matrix, the
parameters in the kernel are shared across all outputs. Therefore, convolutions
usually require less parameters than fully-connected layers.

Nevertheless, both layers are related because convolution is a linear operation
and can therefore be regarded as matrix multiplication. Due to the weight
sharing, this matrix is structured by the kernel elements\sidenote{\Eg, in the
  one-dimensional case, the convolution of two vectors can be computed by
  expanding one into a Toeplitz matrix, and multiplying that onto the second
  vector~\cite{wiki2022toeplitz}.}. Alternatively, one can stack patches---input
elements that overlap with the kernel at each stage---into columns of a matrix,
which yields the unfolded input, denoted by $\llbracket \tX \rrbracket$ in
\Cref{tab:background::forward}. Then, convolution is a matrix-matrix product
between a matrix reshape of the kernel and the unfolded
input~\cite{chellapilla2006HighPerformanceCNN}
(see~\Cref{hbp::subfig:convolutionIllustration3} for an illustration).

\paragraph{Padding \& pooling layers:} Convolutions are often combined with
other modules. \emph{Padding layers} add pixels around the outer dimensions of
an image, which helps to reduce information loss at the image boundaries during
convolution. \emph{Pooling layers} down-sample images by summarizing patches of
pixels and reduce the number of hidden features. Similar to convolution, pooling
considers patches of an input image. Two common summary operations are
per-channel averaging and taking the per-channel maximum. They give rise to
maximum and average pooling.

One can interpret padding and pooling as scatter operations, realized by
multiplication with a sparse, binary matrix $\mPi$ (compare
\Cref{tab:background::forward}:
\begin{itemize}
\item For padding, $\mPi$ does not dependent on the input, but only its shape
  and the hyperparameters. A row is empty if its index corresponds to the padded
  area, and otherwise contains a one at the element's index to be copied from
  the input.

\item For maximum pooling, $\mPi$ depends on the input. Each row contains a one
  at the index of the element with maximum value in the patch.

\item For average pooling, $\mPi$ does not dependent on the input, but only its
  shape and the hyperparameters. Each row contains the inverse patch size at
  indices of the elements in the current patch.
\end{itemize}

\subsubsection{Residual Networks (ResNets)}

The inclusion of skip (or residual) connections~\cite{he2016deep} represents
another revolution in the design of CNNs, and enabled training of deeper
architectures, with 100 or even 1000 layers. This lead to improved performance
of such CNNs on tasks like
\imagenet~\cite{deng2009imagenet,russakovsky2015imagenet}. Skip connections
branch off a hidden feature and feed it back after the residual block $\vs(\vx,
\vtheta)$,
\begin{align*}
  \vz = \vx + \vs(\vx, \vtheta)\,.
\end{align*}
This can be seen as learning a small modification $\vs(\vx, \vtheta)$---a
residual function---for $\vx$; hence the name \emph{residual connection}.

\subsubsection{Closing Remarks \& Sources of Non-linearity}

This short overview of common neural network layers and architectures is, of
course, incomplete. Other famous layers include
dropout~\cite{srivastava2014dropout},
recursive~\cite{hochreither1997lstm,cho2014properties,elman1990finding},
normalization~\cite{ioffe2015batch,wu2019group}, attention
layers~\cite{vaswani2017attention}, \etc

An interesting observation about the operations in
\Cref{tab:background::forward} is that most of them are linear (linear,
convolution, padding, and average pooling layers) or piece-wise linear (maximum
pooling and ReLU activation layer) \wrt both their input and parameters. This
implies that their second- and higher-order derivatives vanish. Non-linearity is
often only introduced by activation layers (elementwise) and loss functions
(after the model's forward pass). While the properties of these components are
not inherited by the entire neural network, they give rise to structure in a
network's higher-order derivatives, see \eg \Cref{chap:hbp}.


%%% Local Variables:
%%% mode: latex
%%% TeX-master: "../thesis"
%%% End:
