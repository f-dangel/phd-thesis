\subsection{Performance of Two-dimensional Histograms}\label{cockpit::app:histograms}

Both one- and two-dimensional histograms require $|\sB| \times D$ elements be
accessed, and hence perform similarly. However, we observed different behavior
on GPU and decided to omit the two-dimensional histogram's run time in the main
text. As explained here, this performance lack is not fundamental, but a
shortcoming of the GPU implementation. \pytorch provides built-in functionality
for computing one-dimensional histograms at the time of writing, but is not yet
featuring multi-dimensional histograms. We experimented with three
implementations:
\begin{itemize}
\item \textbf{\pytorch (third party):} A third party
  implementation\footnote{Permission granted by the authors of
    \href{cockpit::https://github.com/miranov25/RootInteractive/blob/7019e4c2b9f291551aeeb8677a969cfcfde690d1/RootInteractive/Tools/Histograms/histogramdd_pytorch.py}{\texttt{github.com/miranov25/.../histogramdd\_pytorch.py}}.}
  under review for being integrated into \pytorch\footnote{See
    \url{https://github.com/pytorch/pytorch/pull/44485}.}. It relies on
  \texttt{torch.bincount}, which uses \inlinecode{atomicAdd}s that represent a
  bottleneck for histograms where most counts are contained in one
  bin.\footnote{See
    \mbox{\url{https://discuss.pytorch.org/t/torch-bincount-1000x-slower-on-cuda/42654}}}
  This occurs often for over-parameterized deep models, as most of the gradient
  elements are zero.

\item \textbf{\pytorch (\cockpittitle):} Our implementation uses a workaround,
  computes bin indices and scatters the counts into their associated bins with
  \inlinecode{torch.Tensor.put\_}. This circumvents \inlinecode{atomicAdd}s, but
  has poor memory locality.

\item \textbf{\numpy:} The single-threaded \inlinecode{numpy.histogram2d} serves
  as baseline, but does not run on GPUs.
\end{itemize}


% pgfplots style "histogrambenchmarkdefault"
\pgfkeys{/pgfplots/histogrambenchmarkdefault/.style={
    enlarge x limits=-0.05,
    width=1.0\linewidth,
    height=0.7\linewidth,
    every axis plot/.append style={line width = 1.5pt},
    tick pos = left,
    ylabel near ticks,
    xlabel near ticks,
    xtick align = inside,
    ytick align = inside,
    legend cell align = left,
    legend columns = 1,
    % legend pos = south east,
    legend style = {
      fill opacity = 0.7,
      text opacity = 1,
      font = \footnotesize,
    },
    xticklabel style = {font = \footnotesize, inner xsep = -5ex},
    xlabel style = {font = \footnotesize},
    axis line style = {black},
    yticklabel style = {font = \footnotesize, inner ysep = -4ex},
    ylabel style = {font = \footnotesize},
    title style = {font = \footnotesize, inner ysep = -3ex},
    grid = major,
    grid style = {dashed}
  }
}

\begin{figure*}
  \begin{subfigure}[t]{0.49\linewidth}
    \pgfkeys{/pgfplots/zmystyle/.style={histogrambenchmarkdefault,
        xlabel={Histogram Balance $b$}
      }}
    \caption{GPU}
    \tikzexternalenable
    % This file was created by tikzplotlib v0.9.3.
\begin{tikzpicture}

\definecolor{color0}{rgb}{0.12156862745098,0.466666666666667,0.705882352941177}
\definecolor{color1}{rgb}{1,0.498039215686275,0.0549019607843137}

\begin{axis}[
axis line style={white!80!black},
legend style={fill opacity=0.8, draw opacity=1, text opacity=1, draw=white!80!black},
tick pos=left,
xlabel={Histogram Balance},
xmin=-0.039, xmax=1.039,
ylabel={Run Time [s]},
ymin=0.0647793941403482, ymax=344.025767453251,
ymode=log,
zmystyle
]
\addplot [, color0, dashed]
table {%
0.01 4.54533004760742
0.0615789473684211 2.08099224567413
0.113157894736842 0.744264936447144
0.164736842105263 0.445815587043762
0.216315789473684 0.335660696029663
0.267894736842105 0.282471227645874
0.319473684210526 0.253187036514282
0.371052631578947 0.232630848884583
0.422631578947368 0.219761872291565
0.474210526315789 0.210704207420349
0.52578947368421 0.206429862976074
0.577368421052632 0.218602991104126
0.628947368421053 0.214805579185486
0.680526315789474 0.20776104927063
0.732105263157895 0.20455629825592
0.783684210526316 0.20727813243866
0.835263157894737 0.206504940986633
0.886842105263158 0.203398704528809
0.9 0.20402295589447
0.938421052631579 0.204293870925903
0.99 0.204171013832092
};
\addlegendentry{\textsc{PyTorch} (\textsc{Cockpit})}
\addplot [, color1, dashed]
table {%
0.01 232.951647591591
0.0615789473684211 67.0944833517075
0.113157894736842 12.7715488433838
0.164736842105263 5.53443562984467
0.216315789473684 2.58965411186218
0.267894736842105 1.4656699180603
0.319473684210526 0.907050752639771
0.371052631578947 0.633884406089783
0.422631578947368 0.44161102771759
0.474210526315789 0.332058143615723
0.52578947368421 0.243267822265625
0.577368421052632 0.193850588798523
0.628947368421053 0.160495829582214
0.680526315789474 0.137686920166016
0.732105263157895 0.118721318244934
0.783684210526316 0.105425000190735
0.835263157894737 0.106805515289307
0.886842105263158 0.108340263366699
0.9 0.105959153175354
0.938421052631579 0.103169775009155
0.99 0.0956669807434082
};
\addlegendentry{\textsc{PyTorch} (third party)}
\end{axis}

\end{tikzpicture}

    \tikzexternaldisable
    \label{cockpit::fig:app-histogram2d-benchmark-gpu}
  \end{subfigure}
  \hfill
  \begin{subfigure}[t]{0.49\linewidth}
    \pgfkeys{/pgfplots/zmystyle/.style={histogrambenchmarkdefault,
        xlabel={Histogram Balance $b$}
      }}
    \caption{CPU}
    \tikzexternalenable
    % This file was created by tikzplotlib v0.9.3.
\begin{tikzpicture}

\definecolor{color0}{rgb}{0.12156862745098,0.466666666666667,0.705882352941177}
\definecolor{color1}{rgb}{1,0.498039215686275,0.0549019607843137}
\definecolor{color2}{rgb}{0.172549019607843,0.627450980392157,0.172549019607843}

\begin{axis}[
axis line style={white!80!black},
legend style={fill opacity=0.8, draw opacity=1, text opacity=1, at={(0.91,0.5)}, anchor=east, draw=white!80!black},
tick pos=left,
xlabel={Histogram Balance},
xmin=-0.039, xmax=1.039,
ylabel={Run Time [s]},
ymin=0.919387552518064, ymax=8.28341609861724,
ymode=log,
zmystyle
]
\addplot [, color0, dashed]
table {%
0.01 1.04204399585724
0.0615789473684211 1.03900332450867
0.113157894736842 1.07859117984772
0.164736842105263 1.06876609325409
0.216315789473684 1.04880547523499
0.267894736842105 1.04529702663422
0.319473684210526 1.05313646793365
0.371052631578947 1.06466472148895
0.422631578947368 1.06266655921936
0.474210526315789 1.05389556884766
0.52578947368421 1.06146202087402
0.577368421052632 1.04773416519165
0.628947368421053 1.06101129055023
0.680526315789474 1.04648265838623
0.732105263157895 1.0483683347702
0.783684210526316 1.04395830631256
0.835263157894737 1.04497628211975
0.886842105263158 1.04231572151184
0.9 1.04766194820404
0.938421052631579 1.04687905311584
0.99 1.0437038898468
};
\addlegendentry{\textsc{PyTorch} (\textsc{Cockpit})}
\addplot [, color1, dashed]
table {%
0.01 1.02336490154266
0.0615789473684211 1.02175369262695
0.113157894736842 1.02237765789032
0.164736842105263 1.02118575572968
0.216315789473684 1.01996810436249
0.267894736842105 1.02348549365997
0.319473684210526 1.0248743057251
0.371052631578947 1.02083847522736
0.422631578947368 1.01863925457001
0.474210526315789 1.02599215507507
0.52578947368421 1.0236154794693
0.577368421052632 1.02260437011719
0.628947368421053 1.03915128707886
0.680526315789474 1.03184995651245
0.732105263157895 1.0255656003952
0.783684210526316 1.0171329498291
0.835263157894737 1.02414612770081
0.886842105263158 1.02574753761292
0.9 1.01600201129913
0.938421052631579 1.01842267513275
0.99 1.02955875396729
};
\addlegendentry{\textsc{PyTorch} (third party)}
\addplot [, color2, dashed]
table {%
0.01 3.44840505123138
0.0615789473684211 3.95587799549103
0.113157894736842 4.67446537017822
0.164736842105263 4.98263554573059
0.216315789473684 5.3127799987793
0.267894736842105 5.58865821361542
0.319473684210526 5.89873046875
0.371052631578947 6.09700152873993
0.422631578947368 6.21813678741455
0.474210526315789 6.48546359539032
0.52578947368421 6.69456965923309
0.577368421052632 6.80199146270752
0.628947368421053 6.89959270954132
0.680526315789474 7.07993612289429
0.732105263157895 7.23664236068726
0.783684210526316 7.27940900325775
0.835263157894737 7.2930465221405
0.886842105263158 7.35840466022491
0.938421052631579 7.44740259647369
0.99 7.4957230091095
};
\addlegendentry{\textsc{NumPy} (single thread)}
\end{axis}

\end{tikzpicture}

    \tikzexternaldisable
    \label{cockpit::fig:app-histogram2d-benchmark-cpu}
  \end{subfigure}
  \caption{\textbf{Performance of two-dimensional histogram GPU implementations
      depends on the data.} (\subref{cockpit::fig:app-histogram2d-benchmark-gpu}) Run
    time for two different GPU implementations with histograms of different
    imbalance. \cockpit's implementation outperforms the third party solution by
    more than one order of magnitude in the deep learning regime ($b \ll 1$).
    (\subref{cockpit::fig:app-histogram2d-benchmark-cpu}) On CPU, performance is robust
    to histogram balance. The run time difference between \numpy and \pytorch is
    due to multi-threading. Data has the same size as \deepobs's \cifarten
    \threecthreed problem ($D =895,210, |\sB| = 128$). Curves represent averages
    over 10 independent runs. Error bars are omitted to improve legibility.}
  \label{cockpit::fig:app-histogram2d-benchmark}
\end{figure*}

To demonstrate the strong performance dependence on the data, we generate data
from a uniform distribution over $[0, b]\times[0, b]$, where $b \in (0, 1)$
parametrizes the histogram's balance, and compute two-dimensional histograms on
$[0,1]\times [0, 1]$. \Cref{cockpit::fig:app-histogram2d-benchmark-gpu} shows a
clear increase in run time of both GPU implementations for more imbalanced
histograms. Note that even though our implementation outperforms the third party
by more than one order of magnitude in the deep neural network regime ($b \ll
1$), it is still considerably slower than a one-dimensional histogram (see
\Cref{cockpit::fig:app_benchmark_instruments_cuda} (c)), and even slower on GPU
than on CPU (\Cref{cockpit::fig:app-histogram2d-benchmark} (b)). As expected,
the CPU implementations do not significantly depend on the data
(\Cref{cockpit::fig:app-histogram2d-benchmark-cpu}). The performance difference
between \pytorch and \numpy is likely due to multi-threading versus
single-threading.

Although a carefully engineered histogram GPU implementation is currently not
available, we think it will reduce the computational overhead to that of a
one-dimensional histogram in future releases.

%%% Local Variables:
%%% mode: latex
%%% TeX-master: "../thesis"
%%% End:
