\paragraph{\KFAC versus \KFLR:}
As the \KFLR of \citet{botev2017practical} is orders of magnitude more expensive
to compute than the \KFAC of \citet{martens2015optimizing} on \CIFAR{100}, it
was not included in the main plot. This is not an implementation error; it
follows from the definition of those methods. To approximate the \GGN,
$\mG(\vtheta) = \nicefrac{1}{N} \sum_n [\jac_{\vtheta} \vf_n]^\top
\nabla_{\vf_n}^2 \: \ell_n \: [\jac_{\vtheta} \vf_n]$, \KFAC uses a rank-1
approximation for each of the inner Hessian $\nabla_{\vf_n}^2 \: \ell_n \approx
\vs_n\vs_n^\top$, and needs to propagate a \emph{vector} through the computation
graph for each sample. \KFLR uses the complete inner Hessian instead. For
\CIFAR{100}, the network has 100 output nodes---one for each class---and the
inner Hessians are $[100 \times 100]$ matrices. \KFLR needs to propagate a
\emph{matrix} through the computation graph for each sample, which is
$100\times$ more expensive as shown in \Cref{backpack::fig:apx-bench}.

\begin{figure*}[!htbp]
  \centering
  \pgfkeys{/pgfplots/BackPACKBenchBarplot/.style={
    width=1.02\linewidth,
    height=0.5\linewidth,
    grid=major,
    grid style = {dashed},
    every axis plot/.append style={line width = 1.5pt},
    tick pos = left,
    xtick align = outside,
    ytick align = inside,
    xmajorticks = true,
    ymajorticks = true,
    ylabel near ticks,
    xlabel near ticks,
    xticklabel style = {font = \footnotesize},
    xlabel style = {font = \footnotesize},
    axis line style = {black},
    yticklabel style = {font = \footnotesize},
    ylabel style = {font = \footnotesize},
    title style = {font = \footnotesize, inner ysep = 0ex},
    grid = major,
    grid style = {dashed},
    legend cell align = left,
    legend style = {
      fill opacity = 0.7,
      text opacity = 1,
      font = \footnotesize,
      legend columns = 4,
      legend cell align={left},
      legend image post style={opacity = 1, yscale = 1.1, line width=-1pt},
      column sep=0.25cm,
    },
  }
}

%%% Local Variables:
%%% mode: latex
%%% TeX-master: "../../thesis"
%%% End:

  \pgfkeys{/pgfplots/zmystyle/.style={
      BackPACKBenchBarplot,
      title={\cifarhun on \allcnnc, $B=16$},
    }
  }
  \tikzexternalenable
  % This file was created by tikzplotlib v0.9.7.
\begin{tikzpicture}

\begin{axis}[
tick align=outside,
tick pos=left,
title={CIFAR100 on All-CNN-C, B=16},
x grid style={white!69.0196078431373!black},
xmin=-0.5, xmax=8.5,
xtick style={color=black},
xtick={0,1,2,3,4,5,6,7,8},
xticklabels={
Grad (ref.),2nd Moment,
Batch L2,DiagGGN-MC,
KFAC,Indiv. Grad,
Variance,KFLR,
DiagGGN},
y grid style={white!69.0196078431373!black},
ylabel={Time [ms]},
ymajorgrids,
ymin=0, ymax=396.199168879476,
ytick style={color=black},
zmystyle
]
\draw[draw=none,fill=white!80!black] (axis cs:-0.25,0) rectangle (axis cs:0.25,10.6404150000117);
\draw[draw=none,fill=white!80!black] (axis cs:0.75,0) rectangle (axis cs:1.25,15.614299000049);
\draw[draw=none,fill=white!80!black] (axis cs:1.75,0) rectangle (axis cs:2.25,20.5042349999189);
\draw[draw=none,fill=white!80!black] (axis cs:2.75,0) rectangle (axis cs:3.25,16.9321719999971);
\draw[draw=none,fill=white!80!black] (axis cs:3.75,0) rectangle (axis cs:4.25,21.9831315000079);
\draw[draw=none,fill=white!80!black] (axis cs:4.75,0) rectangle (axis cs:5.25,29.4275924999852);
\draw[draw=none,fill=white!80!black] (axis cs:5.75,0) rectangle (axis cs:6.25,33.0175950000466);
\draw[draw=none,fill=white!80!black] (axis cs:6.75,0) rectangle (axis cs:7.25,245.259722000014);
\draw[draw=none,fill=white!80!black] (axis cs:7.75,0) rectangle (axis cs:8.25,374.177160499983);
\path [draw=black]
(axis cs:0,10.5087382099845)
--(axis cs:0,10.768611370035);

\path [draw=black]
(axis cs:1,13.7685239999882)
--(axis cs:1,17.4527360000866);

\path [draw=black]
(axis cs:2,20.4103945925041)
--(axis cs:2,20.5949782223442);

\path [draw=black]
(axis cs:3,16.7874805750074)
--(axis cs:3,17.0743782749895);

\path [draw=black]
(axis cs:4,21.8817708700126)
--(axis cs:4,22.0835303899983);

\path [draw=black]
(axis cs:5,29.2194357674305)
--(axis cs:5,29.6297776975223);

\path [draw=black]
(axis cs:6,32.7443037674664)
--(axis cs:6,33.2814666976208);

\path [draw=black]
(axis cs:7,242.111124464981)
--(axis cs:7,248.373806605052);

\path [draw=black]
(axis cs:8,370.971561069995)
--(axis cs:8,377.332541789978);

\addplot [semithick, black, mark=-, mark size=4, mark options={solid}, only marks]
table {%
0 10.5087382099845
1 13.7685239999882
2 20.4103945925041
3 16.7874805750074
4 21.8817708700126
5 29.2194357674305
6 32.7443037674664
7 242.111124464981
8 370.971561069995
};
\addplot [semithick, black, mark=-, mark size=4, mark options={solid}, only marks]
table {%
0 10.768611370035
1 17.4527360000866
2 20.5949782223442
3 17.0743782749895
4 22.0835303899983
5 29.6297776975223
6 33.2814666976208
7 248.373806605052
8 377.332541789978
};
\end{axis}

\end{tikzpicture}

  \tikzexternaldisable
  \vspace{-2ex}
  \caption{ \textbf{\KFLR and \DiagGGN are more expensive to run on large
      networks.} The gradient takes less than 20\,ms to compute, but \KFLR and
    \DiagGGN are approximately $100\times$ more expensive. }
  \label{backpack::fig:apx-bench}
\end{figure*}

\begin{figure}[!htbp]
  \vspace{0ex}
  \centering
  \pgfkeys{/pgfplots/BackPACKBenchBarplot/.style={
    width=1.02\linewidth,
    height=0.5\linewidth,
    grid=major,
    grid style = {dashed},
    every axis plot/.append style={line width = 1.5pt},
    tick pos = left,
    xtick align = outside,
    ytick align = inside,
    xmajorticks = true,
    ymajorticks = true,
    ylabel near ticks,
    xlabel near ticks,
    xticklabel style = {font = \footnotesize},
    xlabel style = {font = \footnotesize},
    axis line style = {black},
    yticklabel style = {font = \footnotesize},
    ylabel style = {font = \footnotesize},
    title style = {font = \footnotesize, inner ysep = 0ex},
    grid = major,
    grid style = {dashed},
    legend cell align = left,
    legend style = {
      fill opacity = 0.7,
      text opacity = 1,
      font = \footnotesize,
      legend columns = 4,
      legend cell align={left},
      legend image post style={opacity = 1, yscale = 1.1, line width=-1pt},
      column sep=0.25cm,
    },
  }
}

%%% Local Variables:
%%% mode: latex
%%% TeX-master: "../../thesis"
%%% End:

  \pgfkeys{/pgfplots/zmystyle/.style={
      BackPACKBenchBarplot,
      x grid style = {opacity = 0},
      log origin=infty,
      legend pos = north west,
      title = {\threecthreed with one sigmoid on \cifarten},
    }
  }
  \tikzexternalenable
  % This file was created by tikzplotlib v0.9.7.
\begin{tikzpicture}

\begin{axis}[
title={CIFAR10 on 3C3D with one sigmoid},
xtick={0,1,2},
xticklabels={Grad (ref.),Diag GGN,Diag Hessian},
y grid style={white!69.0196078431373!black},
ylabel={Time [ms]},
ymajorgrids,
ymin=3.99888415189711, ymax=773.295775089618,
ymode=log,
zmystyle
]
\draw[draw=none,fill=white!20!black] (axis cs:-0.4,0) rectangle (axis cs:-0.2,5.19836299997678);
\addlegendimage{ybar,ybar legend,draw=none,fill=white!20!black};
\addlegendentry{Batch = 4}

\draw[draw=none,fill=white!20!black] (axis cs:0.6,0) rectangle (axis cs:0.8,9.20453499998075);
\draw[draw=none,fill=white!20!black] (axis cs:1.6,0) rectangle (axis cs:1.8,70.5987415000209);
\draw[draw=none,fill=white!40!black] (axis cs:-0.2,0) rectangle (axis cs:0,6.37467400002834);
\addlegendimage{ybar,ybar legend,draw=none,fill=white!40!black};
\addlegendentry{8}

\draw[draw=none,fill=white!40!black] (axis cs:0.8,0) rectangle (axis cs:1,9.94495199995526);
\draw[draw=none,fill=white!40!black] (axis cs:1.8,0) rectangle (axis cs:2,144.046141499985);
\draw[draw=none,fill=white!60!black] (axis cs:-1.38777878078145e-17,0) rectangle (axis cs:0.2,8.76269899998761);
\addlegendimage{ybar,ybar legend,draw=none,fill=white!60!black};
\addlegendentry{16}

\draw[draw=none,fill=white!60!black] (axis cs:1,0) rectangle (axis cs:1.2,13.868441999989);
\draw[draw=none,fill=white!60!black] (axis cs:2,0) rectangle (axis cs:2.2,290.797718500016);
\draw[draw=none,fill=white!80!black] (axis cs:0.2,0) rectangle (axis cs:0.4,13.6071079999738);
\addlegendimage{ybar,ybar legend,draw=none,fill=white!80!black};
\addlegendentry{32}

\draw[draw=none,fill=white!80!black] (axis cs:1.2,0) rectangle (axis cs:1.4,22.2690985000327);
\draw[draw=none,fill=white!80!black] (axis cs:2.2,0) rectangle (axis cs:2.4,588.104477500053);
\path [draw=black]
(axis cs:-0.3,5.08003149998785)
--(axis cs:-0.3,5.30290307998712);

\path [draw=black]
(axis cs:0.7,8.38988006499164)
--(axis cs:0.7,10.0149758049491);

\path [draw=black]
(axis cs:1.7,69.3483818374409)
--(axis cs:1.7,71.8489714875898);

\path [draw=black]
(axis cs:-0.1,6.27996298500591)
--(axis cs:-0.1,6.45955552502585);

\path [draw=black]
(axis cs:0.9,9.86659800000382)
--(axis cs:0.9,10.0217379999441);

\path [draw=black]
(axis cs:1.9,141.949676097554)
--(axis cs:1.9,146.125926707425);

\path [draw=black]
(axis cs:0.1,8.61271881245642)
--(axis cs:0.1,8.90320556249407);

\path [draw=black]
(axis cs:1.1,13.7110157599773)
--(axis cs:1.1,14.0249727199762);

\path [draw=black]
(axis cs:2.1,283.103387470028)
--(axis cs:2.1,298.482772590013);

\path [draw=black]
(axis cs:0.3,13.4293697050612)
--(axis cs:0.3,13.7667368849156);

\path [draw=black]
(axis cs:1.3,22.0949469275433)
--(axis cs:1.3,22.4351402175375);

\path [draw=black]
(axis cs:2.3,567.014260947444)
--(axis cs:2.3,608.720678157656);

\addplot [semithick, black, mark=-, mark size=4, mark options={solid}, only marks]
table {%
-0.3 5.08003149998785
0.7 8.38988006499164
1.7 69.3483818374409
};
\addplot [semithick, black, mark=-, mark size=4, mark options={solid}, only marks]
table {%
-0.3 5.30290307998712
0.7 10.0149758049491
1.7 71.8489714875898
};
\addplot [semithick, black, mark=-, mark size=4, mark options={solid}, only marks]
table {%
-0.1 6.27996298500591
0.9 9.86659800000382
1.9 141.949676097554
};
\addplot [semithick, black, mark=-, mark size=4, mark options={solid}, only marks]
table {%
-0.1 6.45955552502585
0.9 10.0217379999441
1.9 146.125926707425
};
\addplot [semithick, black, mark=-, mark size=4, mark options={solid}, only marks]
table {%
0.1 8.61271881245642
1.1 13.7110157599773
2.1 283.103387470028
};
\addplot [semithick, black, mark=-, mark size=4, mark options={solid}, only marks]
table {%
0.1 8.90320556249407
1.1 14.0249727199762
2.1 298.482772590013
};
\addplot [semithick, black, mark=-, mark size=4, mark options={solid}, only marks]
table {%
0.3 13.4293697050612
1.3 22.0949469275433
2.3 567.014260947444
};
\addplot [semithick, black, mark=-, mark size=4, mark options={solid}, only marks]
table {%
0.3 13.7667368849156
1.3 22.4351402175375
2.3 608.720678157656
};
\end{axis}

\end{tikzpicture}

  \tikzexternaldisable
  \caption{ \textbf{Diagonal of the Hessian versus the \GGN.} If the network
    contains a single sigmoid activation function, the diagonal of the Hessian
    is an order of magnitude more computationally intensive than the diagonal of
    the \GGN. }
  \label{backpack::fig:apx-diagh}
\end{figure}

\paragraph{\GGN diagonal versus Hessian diagonal:}
Most nets used in deep learning use ReLU activations. ReLU functions have no
\emph{curvature} as they are piecewise linear. Because of this, the diagonal of
the \GGN is equivalent to the diagonal of the Hessian \citep{martens2014new}.
However, for networks that use non piecewise linear activation functions like
sigmoids or tanh, computing the Hessian diagonal can be much more expensive than
the \GGN diagonal. To illustrate this, we modify the smaller net used in our
benchmarks to include a single sigmoid activation function before the last
classification layer. The results in \Cref{backpack::fig:apx-diagh} show that
the Hessian diagonal computation is already an order of magnitude more expensive
than for the \GGN.


%%% Local Variables:
%%% mode: latex
%%% TeX-master: "../thesis"
%%% End:
