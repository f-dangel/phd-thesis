For the practical use of the \vivit{} concept, it is essential that (i) the
computations are efficient and (ii) that we gain an understanding of how
sub-sampling noise and the approximations introduced in
\Cref{vivit::sec:approximations} alter the structural properties of the \ggn{}. In the
following, we therefore empirically investigate \vivit{}'s scalability and
approximation properties in the context of deep learning. The insights from this
analysis substantiate \vivit{}'s value as a monitoring tool for deep learning
optimization.

\subsubsection{Experimental Setting}

Architectures include three deep CNNs from \deepobs \cite{schneider2019deepobs}
(\twoctwod on \fmnist{}, \threecthreed on \cifarten and \allcnnc on \cifarhun),
as well as ResNets from \citet{he2016deep} on \cifarten based on
\citet{idelbayev2018proper}---all architectures use cross-entropy loss. Based on
the approximations presented in \Cref{vivit::sec:approximations}, we distinguish
the following cases:
\begin{itemize}
\item \textbf{mb, exact:} Exact \ggn with all mini-batch samples. Backpropagates
  $NC$ vectors.
\item \textbf{mb, mc:} \mc-approximated \ggn with all mini-batch samples.
  Backpropagates $N M$ vectors with $M$ the number of \mc{}-samples.
\item \textbf{sub, exact:} Exact \ggn on a subset of mini-batch samples
  ($\floor{\nicefrac{N}{8}}$ as in \cite{zhang2017blockdiagonal}).
  Backpropagates $\floor{\nicefrac{N}{8}} C$ vectors.
\item \textbf{sub, mc:} \mc-approximated \ggn on a subset of mini-batch samples.
  Backpropagates $\floor{\nicefrac{N}{8}} M$ vectors with $M$ the number of
  \mc{}-samples.
\end{itemize}

\subsection{Scalability}\label{vivit::subsec:scalability}
We now complement the theoretical computational complexity analysis from
\Cref{vivit::sec:method-complexity} with empirical studies. Results were generated on a
workstation with an Intel Core i7-8700K CPU (32\,GB) and one NVIDIA GeForce RTX
2080 Ti GPU (11\,GB). We use $M=1$ in the following.
% Furthermore, we set the number of \mc
% samples to $M=1$ in the following.


\begin{figure}[tb]
  \begin{subfigure}[t]{0.35\linewidth}
    \centering
    \caption{}
    \label{subfig:performance-cifar10-3c3d-cuda_main_1}

    \vspace{-\baselineskip}
    \begin{normalsize}
      $N_{\text{crit}}$ (eigenvalues)
    \end{normalsize}
    \vspace{0.15\baselineskip}

    \begin{normalsize}
      \begin{tabular}{lll}
    \toprule
    $_{\text{\tiny{\ggn}}}$$^{\text{\tiny{Data}}}$ & mb & sub \\
    \midrule
    exact & 208
              & 727 \\
    mc   & 1055
              & 1816 \\
    \bottomrule
\end{tabular}
    \end{normalsize}

    \vspace{\baselineskip}

    \begin{normalsize}
      $N_{\text{crit}}$ (top eigenpair)
    \end{normalsize}
    \vspace{0.15\baselineskip}

    \begin{normalsize}
      \begin{tabular}{lll}
    \toprule
    $_{\text{\tiny{\ggn}}}$$^{\text{\tiny{Data}}}$ & mb & sub \\
    \midrule
    exact & 208
              & 727 \\
    mc   & 1055
              & 1816 \\
    \bottomrule
\end{tabular}
    \end{normalsize}

    \vspace{3.6\baselineskip}

  \end{subfigure}
  \begin{subfigure}[t]{0.64\linewidth}
    \centering
    \caption{}
    \label{subfig:performance-cifar10-3c3d-cuda_main_2}

    \vspace{-1.5\baselineskip}

    % load "performancedefault" style
    % defines the pgfplots style "performancedefault"
\pgfkeys{/pgfplots/performancedefault/.style={
    width=1.04\linewidth,
    height=\goldenRatioInv*1.04\linewidth,
    every axis plot/.append style={line width = 1.2pt},
    every axis plot post/.append style={
      mark size=2, mark options={opacity=0.9, solid, line width = 1pt}
    },
    tick pos = left,
    xmajorticks = true,
    ymajorticks = true,
    ylabel near ticks,
    xlabel near ticks,
    xtick align = inside,
    ytick align = inside,
    legend cell align = left,
    legend columns = 3,
    % legend pos = north east,
    legend style = {
      fill opacity = 0.9,
      text opacity = 1,
      font = \small,
      at={(1, 1.025)},
      anchor=south east,
    },
    xticklabel style = {font = \small, inner xsep = 0ex},
    xlabel style = {font = \small},
    axis line style = {black},
    yticklabel style = {font = \small, inner ysep = 0ex},
    ylabel style = {font = \small, inner ysep = 0ex},
    title style = {font = \small, inner ysep = 0ex, yshift = -0.75ex},
    grid = major,
    grid style = {dashed},
    title = {},
  }
}
%%% Local Variables:
%%% mode: latex
%%% TeX-master: "../main"
%%% End:

    % customize "zmystyle" as you wish
    \pgfkeys{/pgfplots/zmystyle/.style={performancedefault, height=0.5\linewidth}}
    % This file was created by tikzplotlib v0.9.7.
\begin{tikzpicture}

\definecolor{color0}{rgb}{0.937254901960784,0.231372549019608,0.172549019607843}
\definecolor{color1}{rgb}{0.274509803921569,0.6,0.564705882352941}
\definecolor{color2}{rgb}{0.870588235294118,0.623529411764706,0.0862745098039216}
\definecolor{color3}{rgb}{0.501960784313725,0.184313725490196,0.6}

\begin{axis}[
axis line style={white!80!black},
legend style={fill opacity=0.8, draw opacity=1, text opacity=1, at={(0.03,0.97)}, anchor=north west, draw=white!80!black},
tick pos=left,
title={cifar10\_3c3d, N=128, cuda, one\_group},
xlabel={top eigenpairs (\(\displaystyle k\))},
xmin=0.55, xmax=10.45,
ylabel={time [s]},
ymin=0.00910443848057122, ymax=1.89239674986061,
ymode=log,
zmystyle
]
\addplot [, color0, dashed, mark=pentagon*, mark size=3, mark options={solid}]
table {%
1 0.10646684000676
2 0.263332082999113
3 0.41296657500061
4 0.562371585998335
5 0.961040356996818
6 1.08828064100089
7 1.20301098700293
8 1.29626815899974
9 1.36217651501647
10 1.48477111599641
};
\addlegendentry{power iteration}
\addplot [, black, dashed, mark=*, mark size=3, mark options={solid}]
table {%
1 0.185383651005395
2 0.186310245997447
3 0.186981072998606
4 0.187526319001336
5 0.18765962299949
6 0.18803448399558
7 0.187778580999293
8 0.186061766995408
9 0.189352801004134
10 0.188033979997272
};
\addlegendentry{mb, exact}
\addplot [, color1, dashed, mark=diamond*, mark size=3, mark options={solid}]
table {%
1 0.0260587410011794
2 0.0255732080040616
3 0.0259397419940797
4 0.0259047899962752
5 0.0255934300002991
6 0.0258599759981735
7 0.0261660430041957
8 0.02605828599917
9 0.0260880819987506
10 0.0260066910050227
};
\addlegendentry{sub, exact}
\addplot [, color2, dashed, mark=square*, mark size=3, mark options={solid}]
table {%
1 0.0172249570023268
2 0.0170504369962146
3 0.017168151003716
4 0.0170781389970216
5 0.0171006899981876
6 0.0171182029953343
7 0.0170811919961125
8 0.0169726210006047
9 0.017144368001027
10 0.0171162970000296
};
\addlegendentry{mb, mc}
\addplot [, color3, dashed, mark=triangle*, mark size=3, mark options={solid,rotate=180}]
table {%
1 0.0127098220036714
2 0.0126403089961968
3 0.0126599400027771
4 0.0123849169976893
5 0.011915481001779
6 0.0116139689998818
7 0.0116039499989711
8 0.0117499150001095
9 0.0118982629937818
10 0.0121629770001164
};
\addlegendentry{sub, mc}
\end{axis}

\end{tikzpicture}

  \end{subfigure}

  \vspace{-7ex}
  \caption{\textbf{GPU memory and run time performance:}
  Performance measurements for the
  \threecthreed architecture ($D = 895,\!210$)
  on \cifarten ($C=10$).
  \textbf{(a)} Critical batch sizes $N_{\text{crit}}$
    for computing eigenvalues and the top eigenpair.
    \textbf{(b)} Run time comparison with a power iteration for extracting
    the $k$ leading eigenpairs using a batch of size $N=128$.
  }
  \label{fig:performance-cifar10-3c3d-cuda_main}
\end{figure}

%%% Local Variables:
%%% mode: latex
%%% TeX-master: "../main"
%%% End:



\subsubsection{Memory Performance}

% Describe computation
We consider two tasks:
\begin{enumerate}
\item \textbf{Computing eigenvalues:} The nontrivial eigenvalues
  $\{\lambda_{k}\,|\, (\lambda_{k}, \vetilde_{k}) \in \tilde{\sS}_+\}$ are
  obtained by forming and eigen-decomposing the Gram matrix $\mGtilde$, allowing
  stage-wise discarding of $\mV$ (see
  \Cref{vivit::sec:computing-full-ggn-eigenspectrum,vivit::sec:method-complexity}).
  \label{vivit::item:task-eigenvalues}

\item \textbf{Computing the top eigenpair:} For $(\lambda_{1}, \ve_{1})$, we
  compute the Gram matrix spectrum $\tilde{\sS}_{+}$, extract its top eigenpair
  $(\lambda_{1}, \vetilde_{1})$, and transform it into parameter space by
  \Cref{vivit::eq:ggn-eigenvectors}, \ie $(\lambda_{1}, \ve_{1} =
  \nicefrac{1}{\sqrt{\lambda_{1}}} \mV \vetilde_{1} )$. This requires more
  memory than task~\ref{vivit::item:task-eigenvalues} as $\mV$ must be stored.
  \label{vivit::item:task-eigenvectors}
\end{enumerate}
As a comprehensive memory performance measure, we use the largest batch size
before our system runs out of memory---we call this the \emph{critical batch size}
$N_{\text{crit}}$.

% Describe and explain results
\Cref{vivit::subfig:performance-cifar10-3c3d-cuda_main1} tabularizes the critical batch
sizes on GPU for the \threecthreed architecture on \cifarten. As expected,
computing eigenpairs requires more memory and leads to consistently smaller
critical batch sizes in comparison to computing only eigenvalues. Yet, they all
exceed the traditional batch size used for training ($N=128$, see
\cite{schneider2019deepobs}), even when using the exact \ggn. With \vivit{}'s
approximations, the memory overhead can be reduced to significantly increase the
applicable batch size.

We report similar results for more architectures, a block-diagonal approximation
(as in \citet{zhang2017blockdiagonal}), and on CPU in
\Cref{vivit::sec:performance-experiments}, where we also benchmark a third
task---computing damped Newton steps.

% Describe procedure
\subsubsection{Run Time Performance}

Next, we consider computing the $k$ leading eigenvectors and eigenvalues of a
matrix. A power iteration that computes eigenpairs iteratively via matrix-vector
products serves as a reference. For a fixed value of $k$, we repeat both
approaches $20$ times and report the shortest time.

% Describe computation
For the power iteration, we adapt the implementation from the \pyhessian library
\cite{yao2020pyhessian} and replace its Hessian-vector product by a matrix-free
\ggn-vector product \cite{schraudolph2002fast} through \pytorch's AD. We use the
same default hyperparameters for the termination criterion.
%
Similar to task~\ref{vivit::item:task-eigenvalues}, our method obtains the top-$k$
eigenpairs\sidenote{In contrast to the power iteration that is restricted to
  dominating eigenpairs, our approach allows choosing arbitrary eigenpairs.}
% $\{(\lambda_{1}, \ve_{1}), (\lambda_{2}, \ve_{2}), \ldots, (\lambda_{k},\ve_{k})\}$
by computing $\tilde{\sS}_{+}$, extracting its leading eigenpairs
% $\{(\lambda_{1}, \vetilde_{1}), (\lambda_{2},
% \vetilde_{2}), \ldots, (\lambda_{k}, \vetilde_{k})\}$,
and transforming the eigenvectors $\vetilde_{1}, \vetilde_{2}, \ldots,
\vetilde_{k}$ into parameter space by application of $\mV$ (see
\Cref{vivit::eq:ggn-eigenvectors}).

\begin{figure}
  \centering
  % defines the pgfplots style "eigspacedefault"
\pgfkeys{/pgfplots/eigspacedefault/.style={
    width=1.0\linewidth,
    height=0.6\linewidth,
    every axis plot/.append style={line width = 1.5pt},
    tick pos = left,
    ylabel near ticks,
    xlabel near ticks,
    xtick align = inside,
    ytick align = inside,
    legend cell align = left,
    legend columns = 4,
    legend pos = south east,
    legend style = {
      fill opacity = 1,
      text opacity = 1,
      font = \footnotesize,
      at={(1, 1.025)},
      anchor=south east,
      column sep=0.25cm,
    },
    legend image post style={scale=2.5},
    xticklabel style = {font = \footnotesize},
    xlabel style = {font = \footnotesize},
    axis line style = {black},
    yticklabel style = {font = \footnotesize},
    ylabel style = {font = \footnotesize},
    title style = {font = \footnotesize},
    grid = major,
    grid style = {dashed}
  }
}

\pgfkeys{/pgfplots/eigspacedefaultapp/.style={
    eigspacedefault,
    height=0.6\linewidth,
    legend columns = 2,
  }
}

\pgfkeys{/pgfplots/eigspacenolegend/.style={
    legend image post style = {scale=0},
    legend style = {
      fill opacity = 0,
      draw opacity = 0,
      text opacity = 0,
      font = \footnotesize,
      at={(1, 1.025)},
      anchor=south east,
      column sep=0.25cm,
    },
  }
}
%%% Local Variables:
%%% mode: latex
%%% TeX-master: "../../thesis"
%%% End:

  \pgfkeys{/pgfplots/zmystyle/.style={
      eigspacedefault
    }}
  \tikzexternalenable
  % This file was created by tikzplotlib v0.9.7.
\begin{tikzpicture}

\definecolor{color0}{rgb}{0.145098039215686,0.490196078431373,0.349019607843137}

\begin{axis}[
axis line style={white!10!black},
log basis x={10},
tick pos=left,
xlabel={epoch (log scale)},
xmajorgrids,
xmin=0.794328234724281, xmax=125.892541179417,
xmode=log,
ylabel={overlap},
ymajorgrids,
ymin=0.842878600955009, ymax=1.00748197138309,
zmystyle
]
\addplot [, white!10!black, dashed]
table {%
0.794328234724281 1
125.892541179417 1
};
\addplot [, color0, mark=*, mark size=0.5, mark options={solid}, only marks]
table {%
1 0.850360572338104
1.04487179487179 0.878133177757263
1.09615384615385 0.91087943315506
1.1474358974359 0.980224907398224
1.20192307692308 0.993440926074982
1.25961538461538 0.988232254981995
1.32051282051282 0.987772941589355
1.38461538461538 0.996818244457245
1.44871794871795 0.992499232292175
1.51923076923077 0.993051826953888
1.58974358974359 0.988804459571838
1.66666666666667 0.993102371692657
1.74679487179487 0.994531810283661
1.83012820512821 0.990985095500946
1.91666666666667 0.989441215991974
2.00641025641026 0.996042609214783
2.1025641025641 0.990116953849792
2.20512820512821 0.996223270893097
2.30769230769231 0.996023297309875
2.41987179487179 0.992021441459656
2.53525641025641 0.994040310382843
2.65384615384615 0.990503132343292
2.78205128205128 0.994559407234192
2.91346153846154 0.994334101676941
3.05128205128205 0.996053695678711
3.19871794871795 0.994520008563995
3.34935897435897 0.997268378734589
3.50961538461538 0.997389912605286
3.67628205128205 0.992188334465027
3.8525641025641 0.993352770805359
4.03525641025641 0.995678246021271
4.2275641025641 0.992665469646454
4.42948717948718 0.99213582277298
4.64102564102564 0.997660756111145
4.86217948717949 0.967168152332306
5.09294871794872 0.993785679340363
5.33653846153846 0.993943214416504
5.58974358974359 0.993542373180389
5.85576923076923 0.991654217243195
6.13461538461539 0.969390213489532
6.42628205128205 0.99034708738327
6.73397435897436 0.990933060646057
7.05448717948718 0.99559211730957
7.38782051282051 0.995241284370422
7.74038461538461 0.990916728973389
8.10897435897436 0.990530490875244
8.49679487179487 0.994798362255096
8.90064102564103 0.997474670410156
9.32371794871795 0.995593726634979
9.76923076923077 0.993783950805664
10.2339743589744 0.992836356163025
10.7211538461538 0.983698666095734
11.2307692307692 0.99790632724762
11.7660256410256 0.995324611663818
12.3269230769231 0.997409164905548
12.9134615384615 0.989908039569855
13.5288461538462 0.99628621339798
14.1730769230769 0.9957315325737
14.849358974359 0.996454894542694
15.5544871794872 0.997333228588104
16.2948717948718 0.994928538799286
17.0705128205128 0.989782631397247
17.8846153846154 0.997422814369202
18.7371794871795 0.997152030467987
19.6282051282051 0.99539452791214
20.5641025641026 0.995177149772644
21.5416666666667 0.993471741676331
22.5673076923077 0.994047284126282
23.6442307692308 0.997011065483093
24.7692307692308 0.99792355298996
25.9487179487179 0.995034217834473
27.1826923076923 0.994511604309082
28.4775641025641 0.996814131736755
29.8333333333333 0.99793940782547
31.2564102564103 0.997101902961731
32.7435897435897 0.995478749275208
34.3044871794872 0.99685537815094
35.9358974358974 0.997273564338684
37.6474358974359 0.994191527366638
39.4391025641026 0.99822723865509
41.3173076923077 0.995255470275879
43.2852564102564 0.996559143066406
45.3461538461538 0.992474675178528
47.5064102564103 0.994227528572083
49.7692307692308 0.998042583465576
52.1378205128205 0.996337890625
54.6217948717949 0.995224297046661
57.2211538461538 0.998927772045135
59.9455128205128 0.99811202287674
62.8012820512821 0.997424483299255
65.7916666666667 0.998414218425751
68.9230769230769 0.9983189702034
72.2051282051282 0.9963658452034
75.6442307692308 0.994187355041504
79.2467948717949 0.998039245605469
83.0192307692308 0.99797123670578
86.974358974359 0.996883273124695
91.1153846153846 0.997422218322754
95.4519230769231 0.995517551898956
100 0.998668372631073
};
\end{axis}

\end{tikzpicture}

  \tikzexternaldisable

  \caption{ \textbf{Full-batch \ggn versus full-batch Hessian:} Overlap between the
    top-$C$ eigenspaces of the full-batch \ggn and full-batch Hessian during
    training of the \threecthreed network on \cifarten with \sgd{}. }
  \label{vivit::fig:approx_GGN_Hessian}
\end{figure}

% Describe and explain results
\Cref{vivit::subfig:performance-cifar10-3c3d-cuda_main2} shows the GPU run time
for the \threecthreed architecture on \cifarten, using a mini-batch of size
$N=128$. Without any approximations to the \ggn, our method already outperforms
the power iteration for $k>1$ and increases \textit{much} slower in run time as
more leading eigenpairs are requested. This means that, relative to the
transformation of each eigenvector from the Gram space into the parameter space
through $\mV$, the run time mainly results from computing $\mV,\mGtilde$, and
eigendecomposing the latter. This is consistent with the computational
complexity of those operations in $NC$ (compare
\Cref{vivit::sec:method-complexity}) and allows for efficient extraction of a
large number of eigenpairs. The run time curves of the approximations confirm
this behavior by featuring the same flat profile. Additionally, they require
significantly less time than the exact mini-batch computation. Results for more
network architectures, a block-diagonal approximation and on CPU are reported in
\Cref{vivit::sec:performance-experiments}.

%%% Local Variables:
%%% mode: latex
%%% TeX-master: "../thesis"
%%% End:


\subsection{Approximation Quality}\label{vivit::subsec:approx_quality}
% Hessian --> GGN
\vivit is based on the Hessian's generalized Gauss-Newton approximation (see
\Cref{vivit::eq:ggn}).
% full-batch GGN --> mini-batch GGN
In practice, the \ggn is only computed on a mini-batch which yields a
statistical estimator for the \textit{full-batch} \ggn (\ie the \ggn evaluated
on the entire training set).
% mini-batch GGN --> approximations
Additionally, we introduce curvature sub-sampling and an \mc approximation (see
\Cref{vivit::sec:approximations}), \ie further approximations that alter the
curvature's structural properties.
% Summary of section
In this section, we compare quantities at different stages within this hierarchy
of approximations. We use the test problems from above and train the networks
with both \sgd and \adam{} (details in \Cref{vivit::sec:training_of_nns}).

% ---------------------------
\subsubsection{\ggn Versus Hessian}

First, we empirically study the relationship between the \ggn and the Hessian in
the deep learning context. To capture \textit{solely} the effect of neglecting
the residual $\mR$ (see \Cref{vivit::eq:ggn}), we consider the noise-free case and
compute $\mH$ and $\mG$ on the entire training set.

We characterize both curvature matrices by their top-$C$ eigenspace: the space
spanned by the eigenvectors to the $C$
% (the number of classes)
largest eigenvalues. This is a $C$-dimensional subspace of the parameter space
$\Theta$, on which the loss function is subject to particularly strong
curvature. The \textit{overlap} between these spaces serves as the comparison
metric.
% Definition of eigenspace
Let $\{ \ve_c^\mU \}_{c=1}^C$ the set of orthonormal eigenvectors to the $C$
largest eigenvalues of some symmetric matrix $\mU$ and $\mathcal{E}^\mU =
\vecspan (\ve_1^\mU, ..., \ve_C^\mU)$.
% the corresponding top-$C$ eigenspace.
% Definition overlap
The projection onto this subspace $\mathcal{E}^\mU$ is given by the projection
matrix $\mP^\mU = (\ve_1^\mU, ..., \ve_C^\mU) (\ve_1^\mU, ..., \ve_C^\mU)^\top$.
As in \citet{gurari2018gradient}, we define the overlap between two top-$C$
eigenspaces $\mathcal{E}^\mU$ and $\mathcal{E}^\mV$ of the matrices $\mU$ and
$\mV$ by \vspace{-2mm}
\begin{equation}
  \text{overlap}(\mathcal{E}^\mU, \mathcal{E}^\mV)
  = \frac{\Tr{}(\mP^\mU \mP^\mV)}
  {\sqrt{\Tr{}(\mP^\mU) \Tr{}(\mP^\mV)}}
  \in [0, 1]\, .
  \label{vivit::eq:overlap_eigenspaces}
\end{equation}
If $\text{overlap}(\mathcal{E}^\mU, \mathcal{E}^\mV) = 0$, then
$\mathcal{E}^\mU$ and $\mathcal{E}^\mV$ are orthogonal to each other; if the
overlap is $1$, the subspaces are identical.

\Cref{vivit::fig:approx_GGN_Hessian} shows the overlap between the full-batch \ggn and
Hessian during training of the \threecthreed network on \cifarten with \sgd{}.
Except for a short phase at the beginning of the optimization procedure (note
the log scale for the epoch-axis), a strong agreement ($\text{overlap} \geq
0.85$) between the top-$C$ eigenspaces is observed. We make similar observations
with the other test problems (see \Cref{vivit::sec:ggn_vs_hessian}), yet to a slightly
lesser extent for \cifarhun{}. Consequently, we identify the \ggn as an
interesting object, since it consistently shares relevant structure with the
Hessian matrix.


% ---------------------------
\subsubsection{Eigenspace Under Noise \& Approximations}

\begin{figure}[t]
  \centering
  % \textbf{\cifarten \threecthreed \sgd}\\[1mm]
  % defines the pgfplots style "eigspacedefault"
\pgfkeys{/pgfplots/eigspacedefault/.style={
    width=1.0\linewidth,
    height=0.6\linewidth,
    every axis plot/.append style={line width = 1.5pt},
    tick pos = left,
    ylabel near ticks,
    xlabel near ticks,
    xtick align = inside,
    ytick align = inside,
    legend cell align = left,
    legend columns = 4,
    legend pos = south east,
    legend style = {
      fill opacity = 1,
      text opacity = 1,
      font = \footnotesize,
      at={(1, 1.025)},
      anchor=south east,
      column sep=0.25cm,
    },
    legend image post style={scale=2.5},
    xticklabel style = {font = \footnotesize},
    xlabel style = {font = \footnotesize},
    axis line style = {black},
    yticklabel style = {font = \footnotesize},
    ylabel style = {font = \footnotesize},
    title style = {font = \footnotesize},
    grid = major,
    grid style = {dashed}
  }
}

\pgfkeys{/pgfplots/eigspacedefaultapp/.style={
    eigspacedefault,
    height=0.6\linewidth,
    legend columns = 2,
  }
}

\pgfkeys{/pgfplots/eigspacenolegend/.style={
    legend image post style = {scale=0},
    legend style = {
      fill opacity = 0,
      draw opacity = 0,
      text opacity = 0,
      font = \footnotesize,
      at={(1, 1.025)},
      anchor=south east,
      column sep=0.25cm,
    },
  }
}
%%% Local Variables:
%%% mode: latex
%%% TeX-master: "../../thesis"
%%% End:

  \pgfkeys{/pgfplots/zmystyle/.style={
      eigspacedefault,
    }}
  \tikzexternalenable
  % This file was created by tikzplotlib v0.9.7.
\begin{tikzpicture}

\definecolor{color0}{rgb}{0.501960784313725,0.184313725490196,0.6}
\definecolor{color1}{rgb}{0.870588235294118,0.623529411764706,0.0862745098039216}
\definecolor{color2}{rgb}{0.274509803921569,0.6,0.564705882352941}

\begin{axis}[
axis line style={white!10!black},
legend columns=2,
legend style={fill opacity=0.8, draw opacity=1, text opacity=1, at={(0.03,0.03)}, anchor=south west, draw=white!80!black},
log basis x={10},
tick pos=left,
xlabel={epoch (log scale)},
xmajorgrids,
xmin=0.794328234724281, xmax=125.892541179417,
xmode=log,
ylabel={overlap},
ymajorgrids,
ymin=-0.05, ymax=1.05,
zmystyle
]
\addplot [, white!10!black, dashed, forget plot]
table {%
0.794328234724281 1
125.892541179417 1
};
\addplot [, white!10!black, dashed, forget plot]
table {%
0.794328234724281 0
125.892541179417 0
};
\addplot [, color0, opacity=0.6, mark=triangle*, mark size=0.5, mark options={solid,rotate=180}, only marks]
table {%
1 0.633321225643158
1.04487179487179 0.633167922496796
1.09615384615385 0.645319879055023
1.1474358974359 0.696935772895813
1.20192307692308 0.735691547393799
1.25961538461538 0.743004858493805
1.32051282051282 0.663159072399139
1.38461538461538 0.578848659992218
1.44871794871795 0.566265523433685
1.51923076923077 0.511307239532471
1.58974358974359 0.553184747695923
1.66666666666667 0.507019340991974
1.74679487179487 0.527087032794952
1.83012820512821 0.489433765411377
1.91666666666667 0.52100533246994
2.00641025641026 0.487207233905792
2.1025641025641 0.516088545322418
2.20512820512821 0.523843228816986
2.30769230769231 0.506053566932678
2.41987179487179 0.543740272521973
2.53525641025641 0.500154674053192
2.65384615384615 0.497678607702255
2.78205128205128 0.518427073955536
2.91346153846154 0.490070313215256
3.05128205128205 0.478386849164963
3.19871794871795 0.484398126602173
3.34935897435897 0.461323797702789
3.50961538461538 0.456510841846466
3.67628205128205 0.400044173002243
3.8525641025641 0.404926776885986
4.03525641025641 0.426442533731461
4.2275641025641 0.415361940860748
4.42948717948718 0.38182258605957
4.64102564102564 0.365217059850693
4.86217948717949 0.365998953580856
5.09294871794872 0.351479113101959
5.33653846153846 0.366690158843994
5.58974358974359 0.331267327070236
5.85576923076923 0.361305773258209
6.13461538461539 0.32453528046608
6.42628205128205 0.343815416097641
6.73397435897436 0.304473489522934
7.05448717948718 0.287028342485428
7.38782051282051 0.304264098405838
7.74038461538461 0.291326016187668
8.10897435897436 0.310010522603989
8.49679487179487 0.287607759237289
8.90064102564103 0.270657986402512
9.32371794871795 0.272140860557556
9.76923076923077 0.291391462087631
10.2339743589744 0.288339495658875
10.7211538461538 0.265323370695114
11.2307692307692 0.268832862377167
11.7660256410256 0.28240692615509
12.3269230769231 0.261321276426315
12.9134615384615 0.211589977145195
13.5288461538462 0.254175752401352
14.1730769230769 0.26838681101799
14.849358974359 0.239705845713615
15.5544871794872 0.268844097852707
16.2948717948718 0.263189524412155
17.0705128205128 0.248800024390221
17.8846153846154 0.234545215964317
18.7371794871795 0.243242412805557
19.6282051282051 0.237384006381035
20.5641025641026 0.232320502400398
21.5416666666667 0.21445694565773
22.5673076923077 0.203809887170792
23.6442307692308 0.211414203047752
24.7692307692308 0.192375466227531
25.9487179487179 0.207051798701286
27.1826923076923 0.184000551700592
28.4775641025641 0.186780527234077
29.8333333333333 0.209896236658096
31.2564102564103 0.193499848246574
32.7435897435897 0.187499046325684
34.3044871794872 0.177010402083397
35.9358974358974 0.178522393107414
37.6474358974359 0.184974029660225
39.4391025641026 0.175946339964867
41.3173076923077 0.168570205569267
43.2852564102564 0.150046691298485
45.3461538461538 0.160229966044426
47.5064102564103 0.15055838227272
49.7692307692308 0.151106879115105
52.1378205128205 0.149630650877953
54.6217948717949 0.159002006053925
57.2211538461538 0.151609182357788
59.9455128205128 0.14301697909832
62.8012820512821 0.140859991312027
65.7916666666667 0.114369906485081
68.9230769230769 0.127198085188866
72.2051282051282 0.117595888674259
75.6442307692308 0.133174225687981
79.2467948717949 0.114469043910503
83.0192307692308 0.127625197172165
86.974358974359 0.120496347546577
91.1153846153846 0.0918874368071556
95.4519230769231 0.114998415112495
100 0.101283550262451
};
\addlegendentry{mb 2, exact}
\addplot [, color0, opacity=0.6, mark=triangle*, mark size=0.5, mark options={solid,rotate=180}, only marks, forget plot]
table {%
1 0.494709342718124
1.04487179487179 0.508072674274445
1.09615384615385 0.472669273614883
1.1474358974359 0.4533970952034
1.20192307692308 0.487139940261841
1.25961538461538 0.536234498023987
1.32051282051282 0.51693856716156
1.38461538461538 0.508000075817108
1.44871794871795 0.532252967357635
1.51923076923077 0.521783828735352
1.58974358974359 0.515431880950928
1.66666666666667 0.483936071395874
1.74679487179487 0.511323273181915
1.83012820512821 0.463768869638443
1.91666666666667 0.501585304737091
2.00641025641026 0.491608619689941
2.1025641025641 0.502568542957306
2.20512820512821 0.48529776930809
2.30769230769231 0.425935506820679
2.41987179487179 0.493673413991928
2.53525641025641 0.466302067041397
2.65384615384615 0.40349555015564
2.78205128205128 0.4183629155159
2.91346153846154 0.443614333868027
3.05128205128205 0.400776952505112
3.19871794871795 0.397828459739685
3.34935897435897 0.40813159942627
3.50961538461538 0.371348828077316
3.67628205128205 0.370048195123672
3.8525641025641 0.338275015354156
4.03525641025641 0.371874868869781
4.2275641025641 0.349921315908432
4.42948717948718 0.33645948767662
4.64102564102564 0.342290461063385
4.86217948717949 0.342065185308456
5.09294871794872 0.343239516019821
5.33653846153846 0.304060429334641
5.58974358974359 0.302288204431534
5.85576923076923 0.348987132310867
6.13461538461539 0.307548046112061
6.42628205128205 0.293838262557983
6.73397435897436 0.298730999231339
7.05448717948718 0.288828045129776
7.38782051282051 0.291887164115906
7.74038461538461 0.271417856216431
8.10897435897436 0.291929572820663
8.49679487179487 0.239717289805412
8.90064102564103 0.256766855716705
9.32371794871795 0.255834400653839
9.76923076923077 0.235459715127945
10.2339743589744 0.249298378825188
10.7211538461538 0.238867193460464
11.2307692307692 0.261288821697235
11.7660256410256 0.239302918314934
12.3269230769231 0.252336323261261
12.9134615384615 0.21277180314064
13.5288461538462 0.241548523306847
14.1730769230769 0.214700922369957
14.849358974359 0.220108270645142
15.5544871794872 0.233450695872307
16.2948717948718 0.220849707722664
17.0705128205128 0.232942014932632
17.8846153846154 0.2052371352911
18.7371794871795 0.208879426121712
19.6282051282051 0.201221585273743
20.5641025641026 0.228594049811363
21.5416666666667 0.201344445347786
22.5673076923077 0.182441547513008
23.6442307692308 0.194797322154045
24.7692307692308 0.189139351248741
25.9487179487179 0.192408084869385
27.1826923076923 0.210514530539513
28.4775641025641 0.197941929101944
29.8333333333333 0.183976471424103
31.2564102564103 0.200855448842049
32.7435897435897 0.202761322259903
34.3044871794872 0.165803536772728
35.9358974358974 0.170795083045959
37.6474358974359 0.16943621635437
39.4391025641026 0.188188686966896
41.3173076923077 0.168842270970345
43.2852564102564 0.170913800597191
45.3461538461538 0.15562430024147
47.5064102564103 0.161235079169273
49.7692307692308 0.146307706832886
52.1378205128205 0.129889413714409
54.6217948717949 0.160507008433342
57.2211538461538 0.143643841147423
59.9455128205128 0.13008388876915
62.8012820512821 0.12173768132925
65.7916666666667 0.131866618990898
68.9230769230769 0.121484279632568
72.2051282051282 0.114791117608547
75.6442307692308 0.132824599742889
79.2467948717949 0.122266843914986
83.0192307692308 0.0980019569396973
86.974358974359 0.0903954803943634
91.1153846153846 0.111445426940918
95.4519230769231 0.121098451316357
100 0.103437542915344
};
\addplot [, color0, opacity=0.6, mark=triangle*, mark size=0.5, mark options={solid,rotate=180}, only marks, forget plot]
table {%
1 0.545524299144745
1.04487179487179 0.531452834606171
1.09615384615385 0.595713555812836
1.1474358974359 0.604683220386505
1.20192307692308 0.612162947654724
1.25961538461538 0.601134896278381
1.32051282051282 0.598043084144592
1.38461538461538 0.58671897649765
1.44871794871795 0.643461883068085
1.51923076923077 0.584388673305511
1.58974358974359 0.603388786315918
1.66666666666667 0.542600691318512
1.74679487179487 0.56792676448822
1.83012820512821 0.542393863201141
1.91666666666667 0.532415986061096
2.00641025641026 0.550632297992706
2.1025641025641 0.503512680530548
2.20512820512821 0.522281169891357
2.30769230769231 0.520580768585205
2.41987179487179 0.526549100875854
2.53525641025641 0.547998547554016
2.65384615384615 0.477871954441071
2.78205128205128 0.502317786216736
2.91346153846154 0.455182939767838
3.05128205128205 0.465587705373764
3.19871794871795 0.455523103475571
3.34935897435897 0.462501853704453
3.50961538461538 0.43492203950882
3.67628205128205 0.40585994720459
3.8525641025641 0.409336149692535
4.03525641025641 0.440798580646515
4.2275641025641 0.402796983718872
4.42948717948718 0.403908252716064
4.64102564102564 0.391839325428009
4.86217948717949 0.360024124383926
5.09294871794872 0.385159879922867
5.33653846153846 0.368329137563705
5.58974358974359 0.352680623531342
5.85576923076923 0.370293498039246
6.13461538461539 0.354883462190628
6.42628205128205 0.348915994167328
6.73397435897436 0.332288801670074
7.05448717948718 0.291275918483734
7.38782051282051 0.276949733495712
7.74038461538461 0.283589214086533
8.10897435897436 0.314339220523834
8.49679487179487 0.275656849145889
8.90064102564103 0.275349378585815
9.32371794871795 0.281154692173004
9.76923076923077 0.274319857358932
10.2339743589744 0.217059180140495
10.7211538461538 0.239347815513611
11.2307692307692 0.265282303094864
11.7660256410256 0.263832896947861
12.3269230769231 0.239591643214226
12.9134615384615 0.207162380218506
13.5288461538462 0.215716153383255
14.1730769230769 0.232497081160545
14.849358974359 0.242441937327385
15.5544871794872 0.202556610107422
16.2948717948718 0.210512146353722
17.0705128205128 0.202129125595093
17.8846153846154 0.192183658480644
18.7371794871795 0.196333274245262
19.6282051282051 0.211474731564522
20.5641025641026 0.203455209732056
21.5416666666667 0.195467472076416
22.5673076923077 0.176268443465233
23.6442307692308 0.173090502619743
24.7692307692308 0.179466798901558
25.9487179487179 0.154363766312599
27.1826923076923 0.176492363214493
28.4775641025641 0.135159716010094
29.8333333333333 0.157441377639771
31.2564102564103 0.138464614748955
32.7435897435897 0.139906913042068
34.3044871794872 0.122646294534206
35.9358974358974 0.145737156271935
37.6474358974359 0.133732527494431
39.4391025641026 0.142999321222305
41.3173076923077 0.132487401366234
43.2852564102564 0.118065379559994
45.3461538461538 0.127987772226334
47.5064102564103 0.138686999678612
49.7692307692308 0.117941595613956
52.1378205128205 0.114847496151924
54.6217948717949 0.124358415603638
57.2211538461538 0.123346768319607
59.9455128205128 0.108971193432808
62.8012820512821 0.0969991162419319
65.7916666666667 0.111605107784271
68.9230769230769 0.0979296118021011
72.2051282051282 0.0991314128041267
75.6442307692308 0.0898933187127113
79.2467948717949 0.0954435020685196
83.0192307692308 0.0925521999597549
86.974358974359 0.104363583028316
91.1153846153846 0.108041144907475
95.4519230769231 0.0872776731848717
100 0.091015987098217
};
\addplot [, color0, opacity=0.6, mark=triangle*, mark size=0.5, mark options={solid,rotate=180}, only marks, forget plot]
table {%
1 0.523422837257385
1.04487179487179 0.517627477645874
1.09615384615385 0.528109967708588
1.1474358974359 0.520947277545929
1.20192307692308 0.494197100400925
1.25961538461538 0.441493839025497
1.32051282051282 0.421509712934494
1.38461538461538 0.397586405277252
1.44871794871795 0.484627097845078
1.51923076923077 0.474800407886505
1.58974358974359 0.428153336048126
1.66666666666667 0.458623945713043
1.74679487179487 0.433956056833267
1.83012820512821 0.451023578643799
1.91666666666667 0.422371596097946
2.00641025641026 0.41279274225235
2.1025641025641 0.418320804834366
2.20512820512821 0.408322244882584
2.30769230769231 0.391086012125015
2.41987179487179 0.413396567106247
2.53525641025641 0.459487289190292
2.65384615384615 0.341690868139267
2.78205128205128 0.389077931642532
2.91346153846154 0.366262018680573
3.05128205128205 0.369984745979309
3.19871794871795 0.345057189464569
3.34935897435897 0.35873094201088
3.50961538461538 0.344358503818512
3.67628205128205 0.333692252635956
3.8525641025641 0.328528970479965
4.03525641025641 0.331492394208908
4.2275641025641 0.313480824232101
4.42948717948718 0.343261927366257
4.64102564102564 0.340480923652649
4.86217948717949 0.342256844043732
5.09294871794872 0.302556663751602
5.33653846153846 0.313608795404434
5.58974358974359 0.302308350801468
5.85576923076923 0.337082266807556
6.13461538461539 0.311828523874283
6.42628205128205 0.271950960159302
6.73397435897436 0.298972755670547
7.05448717948718 0.266834586858749
7.38782051282051 0.288553267717361
7.74038461538461 0.286214828491211
8.10897435897436 0.295535951852798
8.49679487179487 0.248173668980598
8.90064102564103 0.275844842195511
9.32371794871795 0.27056673169136
9.76923076923077 0.255090594291687
10.2339743589744 0.242637678980827
10.7211538461538 0.251264572143555
11.2307692307692 0.256876617670059
11.7660256410256 0.258264362812042
12.3269230769231 0.247272834181786
12.9134615384615 0.20845952630043
13.5288461538462 0.232171177864075
14.1730769230769 0.224469527602196
14.849358974359 0.236231431365013
15.5544871794872 0.220951840281487
16.2948717948718 0.209961369633675
17.0705128205128 0.229115352034569
17.8846153846154 0.200397402048111
18.7371794871795 0.194268450140953
19.6282051282051 0.191540837287903
20.5641025641026 0.20839436352253
21.5416666666667 0.190234184265137
22.5673076923077 0.18439456820488
23.6442307692308 0.164636000990868
24.7692307692308 0.173214182257652
25.9487179487179 0.180703267455101
27.1826923076923 0.171099841594696
28.4775641025641 0.164580777287483
29.8333333333333 0.188525423407555
31.2564102564103 0.168585330247879
32.7435897435897 0.15962065756321
34.3044871794872 0.150746837258339
35.9358974358974 0.145215258002281
37.6474358974359 0.121914483606815
39.4391025641026 0.128806188702583
41.3173076923077 0.136148855090141
43.2852564102564 0.136276051402092
45.3461538461538 0.141158908605576
47.5064102564103 0.114671371877193
49.7692307692308 0.119016073644161
52.1378205128205 0.120105646550655
54.6217948717949 0.116573847830296
57.2211538461538 0.122383452951908
59.9455128205128 0.116962432861328
62.8012820512821 0.105055682361126
65.7916666666667 0.107373334467411
68.9230769230769 0.0942141935229301
72.2051282051282 0.0986168012022972
75.6442307692308 0.0988930389285088
79.2467948717949 0.0927614718675613
83.0192307692308 0.0953556448221207
86.974358974359 0.0847640186548233
91.1153846153846 0.0852355360984802
95.4519230769231 0.0992736965417862
100 0.0802000463008881
};
\addplot [, color0, opacity=0.6, mark=triangle*, mark size=0.5, mark options={solid,rotate=180}, only marks, forget plot]
table {%
1 0.605333626270294
1.04487179487179 0.597685515880585
1.09615384615385 0.667412877082825
1.1474358974359 0.641108334064484
1.20192307692308 0.659764111042023
1.25961538461538 0.584210574626923
1.32051282051282 0.504951179027557
1.38461538461538 0.468591302633286
1.44871794871795 0.491764754056931
1.51923076923077 0.445441067218781
1.58974358974359 0.461829245090485
1.66666666666667 0.396699368953705
1.74679487179487 0.431269645690918
1.83012820512821 0.396298855543137
1.91666666666667 0.418356865644455
2.00641025641026 0.396757781505585
2.1025641025641 0.407334387302399
2.20512820512821 0.437328070402145
2.30769230769231 0.436087816953659
2.41987179487179 0.443477481603622
2.53525641025641 0.421981781721115
2.65384615384615 0.393724977970123
2.78205128205128 0.412965029478073
2.91346153846154 0.396501153707504
3.05128205128205 0.384052693843842
3.19871794871795 0.419008076190948
3.34935897435897 0.374358713626862
3.50961538461538 0.370988756418228
3.67628205128205 0.373596727848053
3.8525641025641 0.365131318569183
4.03525641025641 0.359287589788437
4.2275641025641 0.357491701841354
4.42948717948718 0.33229061961174
4.64102564102564 0.326798528432846
4.86217948717949 0.311623483896255
5.09294871794872 0.388948082923889
5.33653846153846 0.328317314386368
5.58974358974359 0.314519166946411
5.85576923076923 0.343756437301636
6.13461538461539 0.347886890172958
6.42628205128205 0.328506678342819
6.73397435897436 0.301367402076721
7.05448717948718 0.306648939847946
7.38782051282051 0.297799736261368
7.74038461538461 0.297945588827133
8.10897435897436 0.296382278203964
8.49679487179487 0.255705922842026
8.90064102564103 0.276719629764557
9.32371794871795 0.252005964517593
9.76923076923077 0.245079860091209
10.2339743589744 0.280048936605453
10.7211538461538 0.259938299655914
11.2307692307692 0.242932483553886
11.7660256410256 0.231496959924698
12.3269230769231 0.240419551730156
12.9134615384615 0.234377935528755
13.5288461538462 0.236405208706856
14.1730769230769 0.236070021986961
14.849358974359 0.223111420869827
15.5544871794872 0.234789177775383
16.2948717948718 0.24242801964283
17.0705128205128 0.231832429766655
17.8846153846154 0.205523729324341
18.7371794871795 0.229432567954063
19.6282051282051 0.221449330449104
20.5641025641026 0.212581545114517
21.5416666666667 0.19132462143898
22.5673076923077 0.207483127713203
23.6442307692308 0.191727444529533
24.7692307692308 0.185348510742188
25.9487179487179 0.169403895735741
27.1826923076923 0.191393688321114
28.4775641025641 0.181521043181419
29.8333333333333 0.191006451845169
31.2564102564103 0.184154734015465
32.7435897435897 0.176578268408775
34.3044871794872 0.160316348075867
35.9358974358974 0.171448990702629
37.6474358974359 0.186104118824005
39.4391025641026 0.165023446083069
41.3173076923077 0.187238797545433
43.2852564102564 0.156787440180779
45.3461538461538 0.154439687728882
47.5064102564103 0.140450239181519
49.7692307692308 0.145386666059494
52.1378205128205 0.121340773999691
54.6217948717949 0.143825083971024
57.2211538461538 0.130993768572807
59.9455128205128 0.141369804739952
62.8012820512821 0.135775536298752
65.7916666666667 0.1395603120327
68.9230769230769 0.12280859798193
72.2051282051282 0.126340761780739
75.6442307692308 0.122518993914127
79.2467948717949 0.122738234698772
83.0192307692308 0.111788801848888
86.974358974359 0.10277109593153
91.1153846153846 0.114183381199837
95.4519230769231 0.111956752836704
100 0.1044527515769
};
\addplot [, color1, opacity=0.6, mark=square*, mark size=0.5, mark options={solid}, only marks]
table {%
1 0.777247369289398
1.04487179487179 0.770448386669159
1.09615384615385 0.803281962871552
1.1474358974359 0.834520161151886
1.20192307692308 0.872211098670959
1.25961538461538 0.871958553791046
1.32051282051282 0.841903150081635
1.38461538461538 0.8604736328125
1.44871794871795 0.879635751247406
1.51923076923077 0.848080456256866
1.58974358974359 0.836516857147217
1.66666666666667 0.818723857402802
1.74679487179487 0.792382717132568
1.83012820512821 0.787184357643127
1.91666666666667 0.766308486461639
2.00641025641026 0.756807565689087
2.1025641025641 0.750781238079071
2.20512820512821 0.732520520687103
2.30769230769231 0.711338639259338
2.41987179487179 0.687405049800873
2.53525641025641 0.661563098430634
2.65384615384615 0.628782272338867
2.78205128205128 0.613285064697266
2.91346153846154 0.596566617488861
3.05128205128205 0.570539176464081
3.19871794871795 0.536603391170502
3.34935897435897 0.559958875179291
3.50961538461538 0.559679687023163
3.67628205128205 0.488923400640488
3.8525641025641 0.484778791666031
4.03525641025641 0.500897228717804
4.2275641025641 0.45257568359375
4.42948717948718 0.438570111989975
4.64102564102564 0.44877552986145
4.86217948717949 0.429499715566635
5.09294871794872 0.439673900604248
5.33653846153846 0.397564023733139
5.58974358974359 0.383366823196411
5.85576923076923 0.391532182693481
6.13461538461539 0.394979059696198
6.42628205128205 0.397225320339203
6.73397435897436 0.357703119516373
7.05448717948718 0.325975149869919
7.38782051282051 0.341578394174576
7.74038461538461 0.360099494457245
8.10897435897436 0.360215574502945
8.49679487179487 0.299640864133835
8.90064102564103 0.333866506814957
9.32371794871795 0.30286979675293
9.76923076923077 0.301265954971313
10.2339743589744 0.301431447267532
10.7211538461538 0.295453041791916
11.2307692307692 0.278884559869766
11.7660256410256 0.308207601308823
12.3269230769231 0.278986752033234
12.9134615384615 0.26557445526123
13.5288461538462 0.292152255773544
14.1730769230769 0.247988864779472
14.849358974359 0.249606162309647
15.5544871794872 0.279840379953384
16.2948717948718 0.284491121768951
17.0705128205128 0.282015144824982
17.8846153846154 0.265656620264053
18.7371794871795 0.258210599422455
19.6282051282051 0.24965600669384
20.5641025641026 0.257215738296509
21.5416666666667 0.225708723068237
22.5673076923077 0.23987241089344
23.6442307692308 0.212257727980614
24.7692307692308 0.215380147099495
25.9487179487179 0.19194233417511
27.1826923076923 0.230384543538094
28.4775641025641 0.186996221542358
29.8333333333333 0.20021764934063
31.2564102564103 0.173916697502136
32.7435897435897 0.157003805041313
34.3044871794872 0.162724554538727
35.9358974358974 0.199153184890747
37.6474358974359 0.1612488925457
39.4391025641026 0.169204756617546
41.3173076923077 0.169399812817574
43.2852564102564 0.156952604651451
45.3461538461538 0.13326258957386
47.5064102564103 0.157900258898735
49.7692307692308 0.151678755879402
52.1378205128205 0.130881577730179
54.6217948717949 0.172865629196167
57.2211538461538 0.10993454605341
59.9455128205128 0.140342459082603
62.8012820512821 0.111761748790741
65.7916666666667 0.135233223438263
68.9230769230769 0.0985468775033951
72.2051282051282 0.107280112802982
75.6442307692308 0.121876381337643
79.2467948717949 0.0944937095046043
83.0192307692308 0.105866812169552
86.974358974359 0.101951658725739
91.1153846153846 0.102934278547764
95.4519230769231 0.109189748764038
100 0.0980146899819374
};
\addlegendentry{mb 8, exact}
\addplot [, color1, opacity=0.6, mark=square*, mark size=0.5, mark options={solid}, only marks, forget plot]
table {%
1 0.80650120973587
1.04487179487179 0.821713924407959
1.09615384615385 0.846667110919952
1.1474358974359 0.879475891590118
1.20192307692308 0.886012375354767
1.25961538461538 0.885229706764221
1.32051282051282 0.866720676422119
1.38461538461538 0.878308951854706
1.44871794871795 0.889364838600159
1.51923076923077 0.857218861579895
1.58974358974359 0.857651233673096
1.66666666666667 0.771526575088501
1.74679487179487 0.848896622657776
1.83012820512821 0.774302363395691
1.91666666666667 0.749237477779388
2.00641025641026 0.775936841964722
2.1025641025641 0.711972415447235
2.20512820512821 0.707637846469879
2.30769230769231 0.744617640972137
2.41987179487179 0.687649488449097
2.53525641025641 0.663406848907471
2.65384615384615 0.661353290081024
2.78205128205128 0.661773800849915
2.91346153846154 0.627224624156952
3.05128205128205 0.598433494567871
3.19871794871795 0.553872883319855
3.34935897435897 0.579225957393646
3.50961538461538 0.541127860546112
3.67628205128205 0.494524478912354
3.8525641025641 0.500620543956757
4.03525641025641 0.460244804620743
4.2275641025641 0.51121586561203
4.42948717948718 0.480776607990265
4.64102564102564 0.45650252699852
4.86217948717949 0.438486099243164
5.09294871794872 0.417048275470734
5.33653846153846 0.444022655487061
5.58974358974359 0.404520004987717
5.85576923076923 0.399827271699905
6.13461538461539 0.394353955984116
6.42628205128205 0.419809192419052
6.73397435897436 0.371141403913498
7.05448717948718 0.392766296863556
7.38782051282051 0.372695922851562
7.74038461538461 0.362761110067368
8.10897435897436 0.376328468322754
8.49679487179487 0.368702739477158
8.90064102564103 0.364896714687347
9.32371794871795 0.347817301750183
9.76923076923077 0.369773596525192
10.2339743589744 0.32089838385582
10.7211538461538 0.287603288888931
11.2307692307692 0.326994985342026
11.7660256410256 0.324010074138641
12.3269230769231 0.303202301263809
12.9134615384615 0.305205971002579
13.5288461538462 0.289823979139328
14.1730769230769 0.31696605682373
14.849358974359 0.28064638376236
15.5544871794872 0.277780413627625
16.2948717948718 0.289385050535202
17.0705128205128 0.262940227985382
17.8846153846154 0.278089195489883
18.7371794871795 0.27421972155571
19.6282051282051 0.267389208078384
20.5641025641026 0.240693598985672
21.5416666666667 0.258440762758255
22.5673076923077 0.23048098385334
23.6442307692308 0.231325432658195
24.7692307692308 0.248114019632339
25.9487179487179 0.224651649594307
27.1826923076923 0.23150072991848
28.4775641025641 0.212005808949471
29.8333333333333 0.230271726846695
31.2564102564103 0.213032558560371
32.7435897435897 0.208778187632561
34.3044871794872 0.204019069671631
35.9358974358974 0.203365445137024
37.6474358974359 0.215405061841011
39.4391025641026 0.189480572938919
41.3173076923077 0.182646736502647
43.2852564102564 0.163511916995049
45.3461538461538 0.202956065535545
47.5064102564103 0.181064739823341
49.7692307692308 0.181889921426773
52.1378205128205 0.172009155154228
54.6217948717949 0.154074549674988
57.2211538461538 0.157865285873413
59.9455128205128 0.15925757586956
62.8012820512821 0.15394501388073
65.7916666666667 0.146935254335403
68.9230769230769 0.136101558804512
72.2051282051282 0.132477253675461
75.6442307692308 0.129364117980003
79.2467948717949 0.137396246194839
83.0192307692308 0.145724207162857
86.974358974359 0.156431078910828
91.1153846153846 0.152589112520218
95.4519230769231 0.112779669463634
100 0.128511294722557
};
\addplot [, color1, opacity=0.6, mark=square*, mark size=0.5, mark options={solid}, only marks, forget plot]
table {%
1 0.822732448577881
1.04487179487179 0.824281334877014
1.09615384615385 0.862076282501221
1.1474358974359 0.886173248291016
1.20192307692308 0.915951728820801
1.25961538461538 0.907984673976898
1.32051282051282 0.869018733501434
1.38461538461538 0.867047488689423
1.44871794871795 0.884066700935364
1.51923076923077 0.863296031951904
1.58974358974359 0.798871695995331
1.66666666666667 0.79884934425354
1.74679487179487 0.745267391204834
1.83012820512821 0.741233289241791
1.91666666666667 0.721401631832123
2.00641025641026 0.693054139614105
2.1025641025641 0.699892103672028
2.20512820512821 0.660626590251923
2.30769230769231 0.583081126213074
2.41987179487179 0.612829327583313
2.53525641025641 0.631756186485291
2.65384615384615 0.537929356098175
2.78205128205128 0.607859075069427
2.91346153846154 0.529702663421631
3.05128205128205 0.569072782993317
3.19871794871795 0.508898198604584
3.34935897435897 0.53626012802124
3.50961538461538 0.51501053571701
3.67628205128205 0.514332592487335
3.8525641025641 0.479987055063248
4.03525641025641 0.485254853963852
4.2275641025641 0.491164892911911
4.42948717948718 0.499634563922882
4.64102564102564 0.456337213516235
4.86217948717949 0.413573116064072
5.09294871794872 0.432276219129562
5.33653846153846 0.418748944997787
5.58974358974359 0.438474148511887
5.85576923076923 0.397523790597916
6.13461538461539 0.435563653707504
6.42628205128205 0.428228199481964
6.73397435897436 0.400451749563217
7.05448717948718 0.376968204975128
7.38782051282051 0.378808587789536
7.74038461538461 0.367855787277222
8.10897435897436 0.390940517187119
8.49679487179487 0.340040653944016
8.90064102564103 0.346644431352615
9.32371794871795 0.335524648427963
9.76923076923077 0.344602167606354
10.2339743589744 0.313341945409775
10.7211538461538 0.342174410820007
11.2307692307692 0.313525289297104
11.7660256410256 0.317241787910461
12.3269230769231 0.302724123001099
12.9134615384615 0.300145953893661
13.5288461538462 0.268837749958038
14.1730769230769 0.274399518966675
14.849358974359 0.27524733543396
15.5544871794872 0.254760235548019
16.2948717948718 0.244571357965469
17.0705128205128 0.267859518527985
17.8846153846154 0.278651714324951
18.7371794871795 0.249213144183159
19.6282051282051 0.252094268798828
20.5641025641026 0.225551083683968
21.5416666666667 0.237385079264641
22.5673076923077 0.215372905135155
23.6442307692308 0.230469092726707
24.7692307692308 0.219263538718224
25.9487179487179 0.178676664829254
27.1826923076923 0.196067586541176
28.4775641025641 0.192209377884865
29.8333333333333 0.232647761702538
31.2564102564103 0.207955032587051
32.7435897435897 0.191949233412743
34.3044871794872 0.18684609234333
35.9358974358974 0.158464059233665
37.6474358974359 0.162177518010139
39.4391025641026 0.166888028383255
41.3173076923077 0.149999007582664
43.2852564102564 0.153630018234253
45.3461538461538 0.165447860956192
47.5064102564103 0.143311947584152
49.7692307692308 0.147649720311165
52.1378205128205 0.131863236427307
54.6217948717949 0.139704301953316
57.2211538461538 0.115432880818844
59.9455128205128 0.107081584632397
62.8012820512821 0.136034920811653
65.7916666666667 0.129533633589745
68.9230769230769 0.129929646849632
72.2051282051282 0.115943811833858
75.6442307692308 0.107810162007809
79.2467948717949 0.105069078505039
83.0192307692308 0.0990055128931999
86.974358974359 0.0766728147864342
91.1153846153846 0.0973830297589302
95.4519230769231 0.0907079502940178
100 0.0839100256562233
};
\addplot [, color1, opacity=0.6, mark=square*, mark size=0.5, mark options={solid}, only marks, forget plot]
table {%
1 0.832188427448273
1.04487179487179 0.819915950298309
1.09615384615385 0.854742050170898
1.1474358974359 0.878585994243622
1.20192307692308 0.884367763996124
1.25961538461538 0.892147064208984
1.32051282051282 0.795580983161926
1.38461538461538 0.783761441707611
1.44871794871795 0.791585922241211
1.51923076923077 0.739203274250031
1.58974358974359 0.687588512897491
1.66666666666667 0.716758489608765
1.74679487179487 0.734555542469025
1.83012820512821 0.643817782402039
1.91666666666667 0.710172474384308
2.00641025641026 0.667308747768402
2.1025641025641 0.628762304782867
2.20512820512821 0.68874591588974
2.30769230769231 0.664266049861908
2.41987179487179 0.688416302204132
2.53525641025641 0.600337088108063
2.65384615384615 0.615456581115723
2.78205128205128 0.574759602546692
2.91346153846154 0.588202595710754
3.05128205128205 0.546380221843719
3.19871794871795 0.605657815933228
3.34935897435897 0.579697549343109
3.50961538461538 0.543127000331879
3.67628205128205 0.560913503170013
3.8525641025641 0.532085835933685
4.03525641025641 0.561569571495056
4.2275641025641 0.523038327693939
4.42948717948718 0.49079155921936
4.64102564102564 0.497922241687775
4.86217948717949 0.490234583616257
5.09294871794872 0.488036453723907
5.33653846153846 0.465779215097427
5.58974358974359 0.495301306247711
5.85576923076923 0.493101507425308
6.13461538461539 0.439215958118439
6.42628205128205 0.47782489657402
6.73397435897436 0.407924383878708
7.05448717948718 0.444955259561539
7.38782051282051 0.449209421873093
7.74038461538461 0.408889144659042
8.10897435897436 0.377460926771164
8.49679487179487 0.394536972045898
8.90064102564103 0.397503942251205
9.32371794871795 0.363994807004929
9.76923076923077 0.341746091842651
10.2339743589744 0.347708225250244
10.7211538461538 0.352140575647354
11.2307692307692 0.337055265903473
11.7660256410256 0.349001258611679
12.3269230769231 0.318394392728806
12.9134615384615 0.297825664281845
13.5288461538462 0.330868780612946
14.1730769230769 0.294612646102905
14.849358974359 0.298680365085602
15.5544871794872 0.307645857334137
16.2948717948718 0.332239598035812
17.0705128205128 0.326354563236237
17.8846153846154 0.286406844854355
18.7371794871795 0.296731948852539
19.6282051282051 0.301868259906769
20.5641025641026 0.283962309360504
21.5416666666667 0.278897285461426
22.5673076923077 0.268107652664185
23.6442307692308 0.247674375772476
24.7692307692308 0.258186638355255
25.9487179487179 0.259999960660934
27.1826923076923 0.238257318735123
28.4775641025641 0.222829386591911
29.8333333333333 0.265736550092697
31.2564102564103 0.247794985771179
32.7435897435897 0.210168316960335
34.3044871794872 0.223555564880371
35.9358974358974 0.196300163865089
37.6474358974359 0.218035131692886
39.4391025641026 0.21666131913662
41.3173076923077 0.201963424682617
43.2852564102564 0.161626935005188
45.3461538461538 0.178665056824684
47.5064102564103 0.18268945813179
49.7692307692308 0.158858358860016
52.1378205128205 0.18066842854023
54.6217948717949 0.211129054427147
57.2211538461538 0.164845615625381
59.9455128205128 0.155562683939934
62.8012820512821 0.151147231459618
65.7916666666667 0.137063190340996
68.9230769230769 0.130773141980171
72.2051282051282 0.124488286674023
75.6442307692308 0.118389323353767
79.2467948717949 0.128397688269615
83.0192307692308 0.112234406173229
86.974358974359 0.129150375723839
91.1153846153846 0.15471588075161
95.4519230769231 0.1021488904953
100 0.12485633045435
};
\addplot [, color1, opacity=0.6, mark=square*, mark size=0.5, mark options={solid}, only marks, forget plot]
table {%
1 0.729271829128265
1.04487179487179 0.755219519138336
1.09615384615385 0.775568544864655
1.1474358974359 0.810066521167755
1.20192307692308 0.844181180000305
1.25961538461538 0.863144874572754
1.32051282051282 0.842780530452728
1.38461538461538 0.834303319454193
1.44871794871795 0.869764745235443
1.51923076923077 0.832343876361847
1.58974358974359 0.818791806697845
1.66666666666667 0.792705655097961
1.74679487179487 0.779376447200775
1.83012820512821 0.724467575550079
1.91666666666667 0.721819937229156
2.00641025641026 0.766333758831024
2.1025641025641 0.687975108623505
2.20512820512821 0.692751228809357
2.30769230769231 0.653091728687286
2.41987179487179 0.658458530902863
2.53525641025641 0.613855719566345
2.65384615384615 0.578324735164642
2.78205128205128 0.617970168590546
2.91346153846154 0.626808941364288
3.05128205128205 0.577496647834778
3.19871794871795 0.556955456733704
3.34935897435897 0.557727217674255
3.50961538461538 0.558555424213409
3.67628205128205 0.533319890499115
3.8525641025641 0.509102284908295
4.03525641025641 0.497951835393906
4.2275641025641 0.502053439617157
4.42948717948718 0.494830101728439
4.64102564102564 0.445264160633087
4.86217948717949 0.456398814916611
5.09294871794872 0.424147427082062
5.33653846153846 0.439424127340317
5.58974358974359 0.406697273254395
5.85576923076923 0.422736018896103
6.13461538461539 0.437953561544418
6.42628205128205 0.413424491882324
6.73397435897436 0.363414376974106
7.05448717948718 0.369961351156235
7.38782051282051 0.370476692914963
7.74038461538461 0.380969017744064
8.10897435897436 0.404207855463028
8.49679487179487 0.363017469644547
8.90064102564103 0.353272914886475
9.32371794871795 0.325083076953888
9.76923076923077 0.342595815658569
10.2339743589744 0.305660337209702
10.7211538461538 0.34252792596817
11.2307692307692 0.299053519964218
11.7660256410256 0.339979976415634
12.3269230769231 0.327261954545975
12.9134615384615 0.278658360242844
13.5288461538462 0.314245194196701
14.1730769230769 0.306252837181091
14.849358974359 0.299396336078644
15.5544871794872 0.315703809261322
16.2948717948718 0.296998023986816
17.0705128205128 0.288243621587753
17.8846153846154 0.284331321716309
18.7371794871795 0.274262011051178
19.6282051282051 0.284875363111496
20.5641025641026 0.262186765670776
21.5416666666667 0.263358652591705
22.5673076923077 0.251400768756866
23.6442307692308 0.259766846895218
24.7692307692308 0.225276932120323
25.9487179487179 0.259557038545609
27.1826923076923 0.243178889155388
28.4775641025641 0.243529915809631
29.8333333333333 0.221827939152718
31.2564102564103 0.233973607420921
32.7435897435897 0.228265836834908
34.3044871794872 0.212328478693962
35.9358974358974 0.228460907936096
37.6474358974359 0.194003537297249
39.4391025641026 0.206472739577293
41.3173076923077 0.198425635695457
43.2852564102564 0.213042497634888
45.3461538461538 0.187539830803871
47.5064102564103 0.173514053225517
49.7692307692308 0.160965517163277
52.1378205128205 0.157441332936287
54.6217948717949 0.166003942489624
57.2211538461538 0.176121026277542
59.9455128205128 0.163126260042191
62.8012820512821 0.142935663461685
65.7916666666667 0.134364292025566
68.9230769230769 0.143608137965202
72.2051282051282 0.130777627229691
75.6442307692308 0.136319518089294
79.2467948717949 0.141721427440643
83.0192307692308 0.130325794219971
86.974358974359 0.129972085356712
91.1153846153846 0.118830107152462
95.4519230769231 0.125920921564102
100 0.126060917973518
};
\addplot [, color2, opacity=0.6, mark=diamond*, mark size=0.5, mark options={solid}, only marks]
table {%
1 0.932628273963928
1.04487179487179 0.938291192054749
1.09615384615385 0.95775955915451
1.1474358974359 0.968070209026337
1.20192307692308 0.974394500255585
1.25961538461538 0.973689079284668
1.32051282051282 0.966567993164062
1.38461538461538 0.963431000709534
1.44871794871795 0.966016590595245
1.51923076923077 0.94951456785202
1.58974358974359 0.950101315975189
1.66666666666667 0.88587874174118
1.74679487179487 0.940914332866669
1.83012820512821 0.84552401304245
1.91666666666667 0.913096070289612
2.00641025641026 0.922969281673431
2.1025641025641 0.909432411193848
2.20512820512821 0.893574893474579
2.30769230769231 0.884677529335022
2.41987179487179 0.877425014972687
2.53525641025641 0.880024135112762
2.65384615384615 0.889153122901917
2.78205128205128 0.802811801433563
2.91346153846154 0.790535628795624
3.05128205128205 0.766136646270752
3.19871794871795 0.76026064157486
3.34935897435897 0.735775649547577
3.50961538461538 0.750852048397064
3.67628205128205 0.78460294008255
3.8525641025641 0.770893275737762
4.03525641025641 0.747582137584686
4.2275641025641 0.701965272426605
4.42948717948718 0.698621392250061
4.64102564102564 0.728967428207397
4.86217948717949 0.711494028568268
5.09294871794872 0.744661927223206
5.33653846153846 0.720024466514587
5.58974358974359 0.675424695014954
5.85576923076923 0.653375327587128
6.13461538461539 0.675180852413177
6.42628205128205 0.664807617664337
6.73397435897436 0.660577178001404
7.05448717948718 0.672150552272797
7.38782051282051 0.617719173431396
7.74038461538461 0.639324367046356
8.10897435897436 0.584364354610443
8.49679487179487 0.633070945739746
8.90064102564103 0.643986284732819
9.32371794871795 0.544728577136993
9.76923076923077 0.521876454353333
10.2339743589744 0.54596072435379
10.7211538461538 0.558124899864197
11.2307692307692 0.602697789669037
11.7660256410256 0.552951753139496
12.3269230769231 0.537913501262665
12.9134615384615 0.587661921977997
13.5288461538462 0.457116335630417
14.1730769230769 0.534262597560883
14.849358974359 0.506164252758026
15.5544871794872 0.430409163236618
16.2948717948718 0.452758133411407
17.0705128205128 0.471149921417236
17.8846153846154 0.440868765115738
18.7371794871795 0.411999464035034
19.6282051282051 0.399665117263794
20.5641025641026 0.407514631748199
21.5416666666667 0.459341675043106
22.5673076923077 0.387651532888412
23.6442307692308 0.345926523208618
24.7692307692308 0.354522913694382
25.9487179487179 0.352641344070435
27.1826923076923 0.37148904800415
28.4775641025641 0.328543424606323
29.8333333333333 0.344101250171661
31.2564102564103 0.32806059718132
32.7435897435897 0.294717878103256
34.3044871794872 0.263706922531128
35.9358974358974 0.258626490831375
37.6474358974359 0.273335069417953
39.4391025641026 0.310210436582565
41.3173076923077 0.273879289627075
43.2852564102564 0.263814598321915
45.3461538461538 0.228401228785515
47.5064102564103 0.209037065505981
49.7692307692308 0.228399440646172
52.1378205128205 0.222367271780968
54.6217948717949 0.19845524430275
57.2211538461538 0.219551682472229
59.9455128205128 0.188205301761627
62.8012820512821 0.214487314224243
65.7916666666667 0.14883154630661
68.9230769230769 0.188908979296684
72.2051282051282 0.173554807901382
75.6442307692308 0.159559071063995
79.2467948717949 0.14506408572197
83.0192307692308 0.17661364376545
86.974358974359 0.17196224629879
91.1153846153846 0.154570445418358
95.4519230769231 0.127721652388573
100 0.140174746513367
};
\addlegendentry{mb 32, exact}
\addplot [, color2, opacity=0.6, mark=diamond*, mark size=0.5, mark options={solid}, only marks, forget plot]
table {%
1 0.92177402973175
1.04487179487179 0.931659519672394
1.09615384615385 0.952114701271057
1.1474358974359 0.961091816425323
1.20192307692308 0.969603180885315
1.25961538461538 0.963872909545898
1.32051282051282 0.961275100708008
1.38461538461538 0.960747718811035
1.44871794871795 0.967851579189301
1.51923076923077 0.955938518047333
1.58974358974359 0.949824512004852
1.66666666666667 0.940075874328613
1.74679487179487 0.937249958515167
1.83012820512821 0.931238174438477
1.91666666666667 0.930560529232025
2.00641025641026 0.925332248210907
2.1025641025641 0.918525159358978
2.20512820512821 0.919639050960541
2.30769230769231 0.902421474456787
2.41987179487179 0.898367047309875
2.53525641025641 0.895472168922424
2.65384615384615 0.870235085487366
2.78205128205128 0.813703000545502
2.91346153846154 0.848914563655853
3.05128205128205 0.829177856445312
3.19871794871795 0.841319680213928
3.34935897435897 0.795029997825623
3.50961538461538 0.80362743139267
3.67628205128205 0.728380918502808
3.8525641025641 0.759848058223724
4.03525641025641 0.778432786464691
4.2275641025641 0.765615165233612
4.42948717948718 0.716013610363007
4.64102564102564 0.729758441448212
4.86217948717949 0.743451416492462
5.09294871794872 0.713679730892181
5.33653846153846 0.732860624790192
5.58974358974359 0.719551026821136
5.85576923076923 0.673462510108948
6.13461538461539 0.667547583580017
6.42628205128205 0.668962240219116
6.73397435897436 0.641673862934113
7.05448717948718 0.661937892436981
7.38782051282051 0.642151653766632
7.74038461538461 0.648819148540497
8.10897435897436 0.627889037132263
8.49679487179487 0.647981643676758
8.90064102564103 0.651549279689789
9.32371794871795 0.643559575080872
9.76923076923077 0.571582794189453
10.2339743589744 0.562951982021332
10.7211538461538 0.558732450008392
11.2307692307692 0.480571955442429
11.7660256410256 0.538775503635406
12.3269230769231 0.52058619260788
12.9134615384615 0.513228118419647
13.5288461538462 0.502796471118927
14.1730769230769 0.487495422363281
14.849358974359 0.507695853710175
15.5544871794872 0.472815662622452
16.2948717948718 0.481665134429932
17.0705128205128 0.487552642822266
17.8846153846154 0.462441056966782
18.7371794871795 0.410538733005524
19.6282051282051 0.39987388253212
20.5641025641026 0.401975303888321
21.5416666666667 0.37185201048851
22.5673076923077 0.395497411489487
23.6442307692308 0.375535815954208
24.7692307692308 0.337764263153076
25.9487179487179 0.340410709381104
27.1826923076923 0.318518370389938
28.4775641025641 0.35831543803215
29.8333333333333 0.340592473745346
31.2564102564103 0.329118221998215
32.7435897435897 0.291967242956161
34.3044871794872 0.284279346466064
35.9358974358974 0.309867918491364
37.6474358974359 0.296899706125259
39.4391025641026 0.285943508148193
41.3173076923077 0.243596956133842
43.2852564102564 0.233159527182579
45.3461538461538 0.267334848642349
47.5064102564103 0.256620615720749
49.7692307692308 0.24321885406971
52.1378205128205 0.223377659916878
54.6217948717949 0.23184834420681
57.2211538461538 0.172045648097992
59.9455128205128 0.178952977061272
62.8012820512821 0.177083283662796
65.7916666666667 0.179162889719009
68.9230769230769 0.18260882794857
72.2051282051282 0.165032982826233
75.6442307692308 0.154753610491753
79.2467948717949 0.131483241915703
83.0192307692308 0.169182419776917
86.974358974359 0.149082258343697
91.1153846153846 0.129810333251953
95.4519230769231 0.153520584106445
100 0.150085091590881
};
\addplot [, color2, opacity=0.6, mark=diamond*, mark size=0.5, mark options={solid}, only marks, forget plot]
table {%
1 0.902037560939789
1.04487179487179 0.870559334754944
1.09615384615385 0.944737076759338
1.1474358974359 0.951055824756622
1.20192307692308 0.962268054485321
1.25961538461538 0.961669385433197
1.32051282051282 0.946449220180511
1.38461538461538 0.94780695438385
1.44871794871795 0.951630592346191
1.51923076923077 0.933424413204193
1.58974358974359 0.92816162109375
1.66666666666667 0.922088444232941
1.74679487179487 0.917732238769531
1.83012820512821 0.910565555095673
1.91666666666667 0.913563430309296
2.00641025641026 0.892433166503906
2.1025641025641 0.898369014263153
2.20512820512821 0.897662937641144
2.30769230769231 0.859035491943359
2.41987179487179 0.84860897064209
2.53525641025641 0.865632176399231
2.65384615384615 0.836091697216034
2.78205128205128 0.846567571163177
2.91346153846154 0.763507723808289
3.05128205128205 0.801495552062988
3.19871794871795 0.764559388160706
3.34935897435897 0.749117136001587
3.50961538461538 0.781613945960999
3.67628205128205 0.788118302822113
3.8525641025641 0.762565493583679
4.03525641025641 0.734131634235382
4.2275641025641 0.731942594051361
4.42948717948718 0.757953405380249
4.64102564102564 0.696383655071259
4.86217948717949 0.715134859085083
5.09294871794872 0.674463748931885
5.33653846153846 0.686055183410645
5.58974358974359 0.684452950954437
5.85576923076923 0.633180260658264
6.13461538461539 0.637839615345001
6.42628205128205 0.641892850399017
6.73397435897436 0.606902897357941
7.05448717948718 0.576881349086761
7.38782051282051 0.628829598426819
7.74038461538461 0.580028712749481
8.10897435897436 0.56657201051712
8.49679487179487 0.597490787506104
8.90064102564103 0.60525780916214
9.32371794871795 0.599415004253387
9.76923076923077 0.556484639644623
10.2339743589744 0.557806253433228
10.7211538461538 0.50328004360199
11.2307692307692 0.51010799407959
11.7660256410256 0.536683022975922
12.3269230769231 0.532055497169495
12.9134615384615 0.493945330381393
13.5288461538462 0.48908343911171
14.1730769230769 0.438534557819366
14.849358974359 0.494656294584274
15.5544871794872 0.452489197254181
16.2948717948718 0.44525671005249
17.0705128205128 0.46257096529007
17.8846153846154 0.433210283517838
18.7371794871795 0.430849313735962
19.6282051282051 0.375973552465439
20.5641025641026 0.361198425292969
21.5416666666667 0.470883578062057
22.5673076923077 0.392141073942184
23.6442307692308 0.366648644208908
24.7692307692308 0.383997410535812
25.9487179487179 0.378357857465744
27.1826923076923 0.364077657461166
28.4775641025641 0.318030089139938
29.8333333333333 0.319372326135635
31.2564102564103 0.293523639440536
32.7435897435897 0.309991121292114
34.3044871794872 0.328725308179855
35.9358974358974 0.255352228879929
37.6474358974359 0.242363438010216
39.4391025641026 0.222953990101814
41.3173076923077 0.251070499420166
43.2852564102564 0.235842660069466
45.3461538461538 0.268515825271606
47.5064102564103 0.256174296140671
49.7692307692308 0.209148988127708
52.1378205128205 0.212478324770927
54.6217948717949 0.196045160293579
57.2211538461538 0.196020394563675
59.9455128205128 0.186025947332382
62.8012820512821 0.216279983520508
65.7916666666667 0.167298391461372
68.9230769230769 0.171931236982346
72.2051282051282 0.148505434393883
75.6442307692308 0.148019716143608
79.2467948717949 0.155865892767906
83.0192307692308 0.137825831770897
86.974358974359 0.143041059374809
91.1153846153846 0.137221693992615
95.4519230769231 0.136360466480255
100 0.120750807225704
};
\addplot [, color2, opacity=0.6, mark=diamond*, mark size=0.5, mark options={solid}, only marks, forget plot]
table {%
1 0.93377411365509
1.04487179487179 0.933065116405487
1.09615384615385 0.954798221588135
1.1474358974359 0.961155593395233
1.20192307692308 0.964705288410187
1.25961538461538 0.963135242462158
1.32051282051282 0.951974093914032
1.38461538461538 0.95636510848999
1.44871794871795 0.965109348297119
1.51923076923077 0.951481640338898
1.58974358974359 0.950816571712494
1.66666666666667 0.944420754909515
1.74679487179487 0.942672252655029
1.83012820512821 0.94032746553421
1.91666666666667 0.926631569862366
2.00641025641026 0.932477653026581
2.1025641025641 0.903216779232025
2.20512820512821 0.90234911441803
2.30769230769231 0.887151718139648
2.41987179487179 0.88903933763504
2.53525641025641 0.889262974262238
2.65384615384615 0.883346021175385
2.78205128205128 0.862769424915314
2.91346153846154 0.832723259925842
3.05128205128205 0.829456925392151
3.19871794871795 0.828689515590668
3.34935897435897 0.810957849025726
3.50961538461538 0.79102611541748
3.67628205128205 0.783592879772186
3.8525641025641 0.757906556129456
4.03525641025641 0.734292447566986
4.2275641025641 0.773223519325256
4.42948717948718 0.74751877784729
4.64102564102564 0.699628531932831
4.86217948717949 0.703737437725067
5.09294871794872 0.765421867370605
5.33653846153846 0.791190683841705
5.58974358974359 0.734164893627167
5.85576923076923 0.657411515712738
6.13461538461539 0.6817946434021
6.42628205128205 0.637957811355591
6.73397435897436 0.696737170219421
7.05448717948718 0.672457695007324
7.38782051282051 0.642049610614777
7.74038461538461 0.644884288311005
8.10897435897436 0.56655877828598
8.49679487179487 0.645265579223633
8.90064102564103 0.597652673721313
9.32371794871795 0.64038360118866
9.76923076923077 0.55820620059967
10.2339743589744 0.546125471591949
10.7211538461538 0.508544087409973
11.2307692307692 0.440067201852798
11.7660256410256 0.453912973403931
12.3269230769231 0.450243383646011
12.9134615384615 0.466993302106857
13.5288461538462 0.401077032089233
14.1730769230769 0.384100407361984
14.849358974359 0.362000167369843
15.5544871794872 0.36279234290123
16.2948717948718 0.3930604159832
17.0705128205128 0.38768607378006
17.8846153846154 0.367034196853638
18.7371794871795 0.345895737409592
19.6282051282051 0.31819748878479
20.5641025641026 0.334131568670273
21.5416666666667 0.338995784521103
22.5673076923077 0.274976491928101
23.6442307692308 0.336225718259811
24.7692307692308 0.301344275474548
25.9487179487179 0.287796378135681
27.1826923076923 0.322814971208572
28.4775641025641 0.309925615787506
29.8333333333333 0.285752892494202
31.2564102564103 0.251562595367432
32.7435897435897 0.248492434620857
34.3044871794872 0.265980839729309
35.9358974358974 0.258852332830429
37.6474358974359 0.291488826274872
39.4391025641026 0.241829872131348
41.3173076923077 0.205918431282043
43.2852564102564 0.213039681315422
45.3461538461538 0.203236728906631
47.5064102564103 0.220217376947403
49.7692307692308 0.203176781535149
52.1378205128205 0.18151481449604
54.6217948717949 0.156794488430023
57.2211538461538 0.224233508110046
59.9455128205128 0.153899446129799
62.8012820512821 0.169163212180138
65.7916666666667 0.166928142309189
68.9230769230769 0.155261501669884
72.2051282051282 0.168304517865181
75.6442307692308 0.175465226173401
79.2467948717949 0.122561097145081
83.0192307692308 0.149315819144249
86.974358974359 0.164364844560623
91.1153846153846 0.13106207549572
95.4519230769231 0.135219275951385
100 0.159893423318863
};
\addplot [, color2, opacity=0.6, mark=diamond*, mark size=0.5, mark options={solid}, only marks, forget plot]
table {%
1 0.918709754943848
1.04487179487179 0.895618438720703
1.09615384615385 0.944429397583008
1.1474358974359 0.952463150024414
1.20192307692308 0.961049675941467
1.25961538461538 0.955068707466125
1.32051282051282 0.948335349559784
1.38461538461538 0.949211716651917
1.44871794871795 0.960704624652863
1.51923076923077 0.949916064739227
1.58974358974359 0.942548930644989
1.66666666666667 0.931734025478363
1.74679487179487 0.929324924945831
1.83012820512821 0.926349639892578
1.91666666666667 0.903291702270508
2.00641025641026 0.920504868030548
2.1025641025641 0.899483621120453
2.20512820512821 0.889619529247284
2.30769230769231 0.859867691993713
2.41987179487179 0.86351090669632
2.53525641025641 0.836765229701996
2.65384615384615 0.815252482891083
2.78205128205128 0.825437963008881
2.91346153846154 0.783214330673218
3.05128205128205 0.78251326084137
3.19871794871795 0.7662553191185
3.34935897435897 0.777276813983917
3.50961538461538 0.78879177570343
3.67628205128205 0.763471245765686
3.8525641025641 0.772683084011078
4.03525641025641 0.747499287128448
4.2275641025641 0.724232017993927
4.42948717948718 0.745960593223572
4.64102564102564 0.704512298107147
4.86217948717949 0.719797253608704
5.09294871794872 0.705552995204926
5.33653846153846 0.712149500846863
5.58974358974359 0.719823360443115
5.85576923076923 0.658820450305939
6.13461538461539 0.661200225353241
6.42628205128205 0.717771112918854
6.73397435897436 0.652535259723663
7.05448717948718 0.613766133785248
7.38782051282051 0.619120538234711
7.74038461538461 0.658773124217987
8.10897435897436 0.608645617961884
8.49679487179487 0.632529437541962
8.90064102564103 0.608493030071259
9.32371794871795 0.586467504501343
9.76923076923077 0.569081962108612
10.2339743589744 0.560609817504883
10.7211538461538 0.536305010318756
11.2307692307692 0.545849800109863
11.7660256410256 0.552339255809784
12.3269230769231 0.499271959066391
12.9134615384615 0.49711799621582
13.5288461538462 0.452871292829514
14.1730769230769 0.50910872220993
14.849358974359 0.461415857076645
15.5544871794872 0.506064653396606
16.2948717948718 0.466512680053711
17.0705128205128 0.48555326461792
17.8846153846154 0.458046734333038
18.7371794871795 0.397574007511139
19.6282051282051 0.441966742277145
20.5641025641026 0.459091275930405
21.5416666666667 0.408522427082062
22.5673076923077 0.371674627065659
23.6442307692308 0.35003998875618
24.7692307692308 0.37718865275383
25.9487179487179 0.413035213947296
27.1826923076923 0.353285223245621
28.4775641025641 0.368711769580841
29.8333333333333 0.330677002668381
31.2564102564103 0.326293230056763
32.7435897435897 0.304729074239731
34.3044871794872 0.304218143224716
35.9358974358974 0.302728235721588
37.6474358974359 0.315792948007584
39.4391025641026 0.24323458969593
41.3173076923077 0.27931135892868
43.2852564102564 0.269490569829941
45.3461538461538 0.230944752693176
47.5064102564103 0.261680334806442
49.7692307692308 0.231959626078606
52.1378205128205 0.216957196593285
54.6217948717949 0.191049367189407
57.2211538461538 0.19061179459095
59.9455128205128 0.181484803557396
62.8012820512821 0.195404812693596
65.7916666666667 0.173414841294289
68.9230769230769 0.17659604549408
72.2051282051282 0.173661828041077
75.6442307692308 0.156608209013939
79.2467948717949 0.143463164567947
83.0192307692308 0.161770209670067
86.974358974359 0.130946978926659
91.1153846153846 0.129203006625175
95.4519230769231 0.136560201644897
100 0.14145527780056
};
\addplot [, black, opacity=0.6, mark=*, mark size=0.5, mark options={solid}, only marks]
table {%
1 0.981747090816498
1.04487179487179 0.98182338476181
1.09615384615385 0.988020360469818
1.1474358974359 0.990345180034637
1.20192307692308 0.991780400276184
1.25961538461538 0.99025171995163
1.32051282051282 0.988504528999329
1.38461538461538 0.98907083272934
1.44871794871795 0.990697205066681
1.51923076923077 0.988103091716766
1.58974358974359 0.985977172851562
1.66666666666667 0.98432582616806
1.74679487179487 0.983087539672852
1.83012820512821 0.981574535369873
1.91666666666667 0.980182111263275
2.00641025641026 0.979159355163574
2.1025641025641 0.978555619716644
2.20512820512821 0.974648296833038
2.30769230769231 0.973026692867279
2.41987179487179 0.970817983150482
2.53525641025641 0.967321872711182
2.65384615384615 0.957172393798828
2.78205128205128 0.96312552690506
2.91346153846154 0.932680547237396
3.05128205128205 0.902878761291504
3.19871794871795 0.872137367725372
3.34935897435897 0.914182841777802
3.50961538461538 0.912082493305206
3.67628205128205 0.924856305122375
3.8525641025641 0.878294587135315
4.03525641025641 0.910778641700745
4.2275641025641 0.842510998249054
4.42948717948718 0.867378413677216
4.64102564102564 0.819855988025665
4.86217948717949 0.837713837623596
5.09294871794872 0.864742696285248
5.33653846153846 0.874320983886719
5.58974358974359 0.825038135051727
5.85576923076923 0.856293499469757
6.13461538461539 0.864015102386475
6.42628205128205 0.861823260784149
6.73397435897436 0.88582843542099
7.05448717948718 0.859955608844757
7.38782051282051 0.815200984477997
7.74038461538461 0.806585609912872
8.10897435897436 0.842734754085541
8.49679487179487 0.815576016902924
8.90064102564103 0.82606166601181
9.32371794871795 0.858394265174866
9.76923076923077 0.754096686840057
10.2339743589744 0.807255387306213
10.7211538461538 0.784341633319855
11.2307692307692 0.759694039821625
11.7660256410256 0.759757816791534
12.3269230769231 0.785604655742645
12.9134615384615 0.772199630737305
13.5288461538462 0.677051663398743
14.1730769230769 0.699923515319824
14.849358974359 0.711808145046234
15.5544871794872 0.689821422100067
16.2948717948718 0.719122231006622
17.0705128205128 0.755955517292023
17.8846153846154 0.685849487781525
18.7371794871795 0.674093425273895
19.6282051282051 0.663297057151794
20.5641025641026 0.651888072490692
21.5416666666667 0.679920017719269
22.5673076923077 0.673120439052582
23.6442307692308 0.54052209854126
24.7692307692308 0.550203800201416
25.9487179487179 0.58035671710968
27.1826923076923 0.525099635124207
28.4775641025641 0.566456913948059
29.8333333333333 0.538295209407806
31.2564102564103 0.560106933116913
32.7435897435897 0.494608640670776
34.3044871794872 0.478361934423447
35.9358974358974 0.481842517852783
37.6474358974359 0.464411735534668
39.4391025641026 0.383963644504547
41.3173076923077 0.414829462766647
43.2852564102564 0.420306831598282
45.3461538461538 0.395040482282639
47.5064102564103 0.368936538696289
49.7692307692308 0.369207710027695
52.1378205128205 0.315311104059219
54.6217948717949 0.306672900915146
57.2211538461538 0.310634672641754
59.9455128205128 0.262401551008224
62.8012820512821 0.281118243932724
65.7916666666667 0.27843850851059
68.9230769230769 0.289015680551529
72.2051282051282 0.271875828504562
75.6442307692308 0.217997744679451
79.2467948717949 0.236163005232811
83.0192307692308 0.199122712016106
86.974358974359 0.217396453022957
91.1153846153846 0.212340220808983
95.4519230769231 0.166822001338005
100 0.159118488430977
};
\addlegendentry{mb 128, exact}
\addplot [, black, opacity=0.6, mark=*, mark size=0.5, mark options={solid}, only marks, forget plot]
table {%
1 0.982020795345306
1.04487179487179 0.953780591487885
1.09615384615385 0.987308979034424
1.1474358974359 0.990592777729034
1.20192307692308 0.992946445941925
1.25961538461538 0.991203784942627
1.32051282051282 0.989980518817902
1.38461538461538 0.990488648414612
1.44871794871795 0.99211198091507
1.51923076923077 0.990089416503906
1.58974358974359 0.988253116607666
1.66666666666667 0.987675189971924
1.74679487179487 0.986547589302063
1.83012820512821 0.98565673828125
1.91666666666667 0.985000550746918
2.00641025641026 0.983888268470764
2.1025641025641 0.980955302715302
2.20512820512821 0.981683909893036
2.30769230769231 0.975873172283173
2.41987179487179 0.978211998939514
2.53525641025641 0.974315464496613
2.65384615384615 0.967435479164124
2.78205128205128 0.967437088489532
2.91346153846154 0.953267276287079
3.05128205128205 0.949415385723114
3.19871794871795 0.875376641750336
3.34935897435897 0.924704968929291
3.50961538461538 0.895968735218048
3.67628205128205 0.92753928899765
3.8525641025641 0.923917770385742
4.03525641025641 0.917343556880951
4.2275641025641 0.846921145915985
4.42948717948718 0.902848660945892
4.64102564102564 0.898207008838654
4.86217948717949 0.904129326343536
5.09294871794872 0.891416072845459
5.33653846153846 0.925188660621643
5.58974358974359 0.896750271320343
5.85576923076923 0.880420506000519
6.13461538461539 0.877892136573792
6.42628205128205 0.904029667377472
6.73397435897436 0.839462876319885
7.05448717948718 0.85766077041626
7.38782051282051 0.849320232868195
7.74038461538461 0.863644242286682
8.10897435897436 0.847573757171631
8.49679487179487 0.859029591083527
8.90064102564103 0.847772538661957
9.32371794871795 0.852554261684418
9.76923076923077 0.770186126232147
10.2339743589744 0.763655602931976
10.7211538461538 0.726811528205872
11.2307692307692 0.776473820209503
11.7660256410256 0.761313438415527
12.3269230769231 0.79255622625351
12.9134615384615 0.743525981903076
13.5288461538462 0.68277233839035
14.1730769230769 0.739355981349945
14.849358974359 0.724602162837982
15.5544871794872 0.686427116394043
16.2948717948718 0.693134009838104
17.0705128205128 0.739706635475159
17.8846153846154 0.696715474128723
18.7371794871795 0.68716698884964
19.6282051282051 0.61690491437912
20.5641025641026 0.614159882068634
21.5416666666667 0.721479594707489
22.5673076923077 0.648724734783173
23.6442307692308 0.585017144680023
24.7692307692308 0.590175628662109
25.9487179487179 0.524260938167572
27.1826923076923 0.562411189079285
28.4775641025641 0.575817584991455
29.8333333333333 0.535391569137573
31.2564102564103 0.604447066783905
32.7435897435897 0.502214431762695
34.3044871794872 0.502260327339172
35.9358974358974 0.454190403223038
37.6474358974359 0.520226001739502
39.4391025641026 0.448612034320831
41.3173076923077 0.427878677845001
43.2852564102564 0.384945422410965
45.3461538461538 0.419080346822739
47.5064102564103 0.331002056598663
49.7692307692308 0.307637602090836
52.1378205128205 0.359945029020309
54.6217948717949 0.30944037437439
57.2211538461538 0.262731164693832
59.9455128205128 0.280141055583954
62.8012820512821 0.229214191436768
65.7916666666667 0.199491798877716
68.9230769230769 0.216717630624771
72.2051282051282 0.192475125193596
75.6442307692308 0.257141083478928
79.2467948717949 0.148212775588036
83.0192307692308 0.18046860396862
86.974358974359 0.187112584710121
91.1153846153846 0.136196181178093
95.4519230769231 0.145475029945374
100 0.141011148691177
};
\addplot [, black, opacity=0.6, mark=*, mark size=0.5, mark options={solid}, only marks, forget plot]
table {%
1 0.984170615673065
1.04487179487179 0.962684333324432
1.09615384615385 0.990463376045227
1.1474358974359 0.992333114147186
1.20192307692308 0.994071006774902
1.25961538461538 0.993533909320831
1.32051282051282 0.991840958595276
1.38461538461538 0.991823792457581
1.44871794871795 0.993282914161682
1.51923076923077 0.990627884864807
1.58974358974359 0.990484058856964
1.66666666666667 0.988605320453644
1.74679487179487 0.988952457904816
1.83012820512821 0.986732482910156
1.91666666666667 0.98617947101593
2.00641025641026 0.986214578151703
2.1025641025641 0.982601761817932
2.20512820512821 0.983159720897675
2.30769230769231 0.981453120708466
2.41987179487179 0.982396304607391
2.53525641025641 0.979023456573486
2.65384615384615 0.975050568580627
2.78205128205128 0.960993766784668
2.91346153846154 0.952524185180664
3.05128205128205 0.926346242427826
3.19871794871795 0.874208569526672
3.34935897435897 0.890124142169952
3.50961538461538 0.870679318904877
3.67628205128205 0.927426934242249
3.8525641025641 0.894946277141571
4.03525641025641 0.848156630992889
4.2275641025641 0.932078003883362
4.42948717948718 0.869904816150665
4.64102564102564 0.875738322734833
4.86217948717949 0.840713024139404
5.09294871794872 0.840531945228577
5.33653846153846 0.923191964626312
5.58974358974359 0.840282380580902
5.85576923076923 0.817100942134857
6.13461538461539 0.834160268306732
6.42628205128205 0.857244312763214
6.73397435897436 0.873536288738251
7.05448717948718 0.870953977108002
7.38782051282051 0.872702240943909
7.74038461538461 0.864868760108948
8.10897435897436 0.791886925697327
8.49679487179487 0.808820724487305
8.90064102564103 0.809190213680267
9.32371794871795 0.802600562572479
9.76923076923077 0.809189915657043
10.2339743589744 0.766855239868164
10.7211538461538 0.817921936511993
11.2307692307692 0.780461728572845
11.7660256410256 0.73890221118927
12.3269230769231 0.792637348175049
12.9134615384615 0.758075952529907
13.5288461538462 0.795771420001984
14.1730769230769 0.707125425338745
14.849358974359 0.722668051719666
15.5544871794872 0.65700376033783
16.2948717948718 0.729147553443909
17.0705128205128 0.699276089668274
17.8846153846154 0.688258171081543
18.7371794871795 0.630997002124786
19.6282051282051 0.641006469726562
20.5641025641026 0.641031563282013
21.5416666666667 0.611485123634338
22.5673076923077 0.636362671852112
23.6442307692308 0.579290926456451
24.7692307692308 0.580309808254242
25.9487179487179 0.571253001689911
27.1826923076923 0.566135585308075
28.4775641025641 0.604471206665039
29.8333333333333 0.568235576152802
31.2564102564103 0.54334819316864
32.7435897435897 0.529070019721985
34.3044871794872 0.544546067714691
35.9358974358974 0.482943207025528
37.6474358974359 0.538393676280975
39.4391025641026 0.463052660226822
41.3173076923077 0.472070604562759
43.2852564102564 0.442756474018097
45.3461538461538 0.433039039373398
47.5064102564103 0.403144657611847
49.7692307692308 0.368089973926544
52.1378205128205 0.375291228294373
54.6217948717949 0.373930543661118
57.2211538461538 0.284169286489487
59.9455128205128 0.306993454694748
62.8012820512821 0.282681196928024
65.7916666666667 0.288691371679306
68.9230769230769 0.285261958837509
72.2051282051282 0.256960988044739
75.6442307692308 0.230094075202942
79.2467948717949 0.2147067040205
83.0192307692308 0.226162105798721
86.974358974359 0.186426684260368
91.1153846153846 0.155625566840172
95.4519230769231 0.171306014060974
100 0.173411771655083
};
\addplot [, black, opacity=0.6, mark=*, mark size=0.5, mark options={solid}, only marks, forget plot]
table {%
1 0.984564304351807
1.04487179487179 0.975940704345703
1.09615384615385 0.990538895130157
1.1474358974359 0.991618752479553
1.20192307692308 0.993819415569305
1.25961538461538 0.993273675441742
1.32051282051282 0.991371631622314
1.38461538461538 0.991176605224609
1.44871794871795 0.993327617645264
1.51923076923077 0.990107357501984
1.58974358974359 0.98944091796875
1.66666666666667 0.987054467201233
1.74679487179487 0.987636685371399
1.83012820512821 0.98361474275589
1.91666666666667 0.983842968940735
2.00641025641026 0.983477294445038
2.1025641025641 0.97759610414505
2.20512820512821 0.981981098651886
2.30769230769231 0.977427303791046
2.41987179487179 0.94308990240097
2.53525641025641 0.885430157184601
2.65384615384615 0.940607070922852
2.78205128205128 0.879908382892609
2.91346153846154 0.963673532009125
3.05128205128205 0.902095317840576
3.19871794871795 0.928976535797119
3.34935897435897 0.907859265804291
3.50961538461538 0.874574661254883
3.67628205128205 0.870367050170898
3.8525641025641 0.943110883235931
4.03525641025641 0.854616105556488
4.2275641025641 0.914628028869629
4.42948717948718 0.825600624084473
4.64102564102564 0.887604176998138
4.86217948717949 0.835559844970703
5.09294871794872 0.886755168437958
5.33653846153846 0.882330358028412
5.58974358974359 0.816561877727509
5.85576923076923 0.81672191619873
6.13461538461539 0.833069980144501
6.42628205128205 0.844195187091827
6.73397435897436 0.819805800914764
7.05448717948718 0.808649480342865
7.38782051282051 0.807758152484894
7.74038461538461 0.759325206279755
8.10897435897436 0.799039661884308
8.49679487179487 0.880455434322357
8.90064102564103 0.810913264751434
9.32371794871795 0.796833634376526
9.76923076923077 0.757974803447723
10.2339743589744 0.751160323619843
10.7211538461538 0.791345059871674
11.2307692307692 0.769242703914642
11.7660256410256 0.763507902622223
12.3269230769231 0.753553450107574
12.9134615384615 0.752502381801605
13.5288461538462 0.724954068660736
14.1730769230769 0.736444771289825
14.849358974359 0.718579232692719
15.5544871794872 0.71355402469635
16.2948717948718 0.689074337482452
17.0705128205128 0.689262390136719
17.8846153846154 0.716107070446014
18.7371794871795 0.738085210323334
19.6282051282051 0.694091975688934
20.5641025641026 0.601486027240753
21.5416666666667 0.648961961269379
22.5673076923077 0.697732865810394
23.6442307692308 0.650158584117889
24.7692307692308 0.592214584350586
25.9487179487179 0.591780006885529
27.1826923076923 0.588055074214935
28.4775641025641 0.569010436534882
29.8333333333333 0.524317264556885
31.2564102564103 0.545189797878265
32.7435897435897 0.537986934185028
34.3044871794872 0.493544101715088
35.9358974358974 0.51635879278183
37.6474358974359 0.495911508798599
39.4391025641026 0.452186733484268
41.3173076923077 0.451370060443878
43.2852564102564 0.443919867277145
45.3461538461538 0.395095527172089
47.5064102564103 0.411824524402618
49.7692307692308 0.371033251285553
52.1378205128205 0.385174244642258
54.6217948717949 0.360418766736984
57.2211538461538 0.331921488046646
59.9455128205128 0.321492403745651
62.8012820512821 0.306410759687424
65.7916666666667 0.375531643629074
68.9230769230769 0.285832017660141
72.2051282051282 0.273515850305557
75.6442307692308 0.241822242736816
79.2467948717949 0.254934042692184
83.0192307692308 0.242678984999657
86.974358974359 0.241615250706673
91.1153846153846 0.212708428502083
95.4519230769231 0.185712978243828
100 0.191799759864807
};
\addplot [, black, opacity=0.6, mark=*, mark size=0.5, mark options={solid}, only marks, forget plot]
table {%
1 0.982150375843048
1.04487179487179 0.948779225349426
1.09615384615385 0.988999545574188
1.1474358974359 0.99137419462204
1.20192307692308 0.993538856506348
1.25961538461538 0.99172830581665
1.32051282051282 0.990062236785889
1.38461538461538 0.989916980266571
1.44871794871795 0.99199241399765
1.51923076923077 0.989880263805389
1.58974358974359 0.988528251647949
1.66666666666667 0.987747013568878
1.74679487179487 0.987243294715881
1.83012820512821 0.986089169979095
1.91666666666667 0.984426498413086
2.00641025641026 0.984810292720795
2.1025641025641 0.981416404247284
2.20512820512821 0.980676293373108
2.30769230769231 0.975415885448456
2.41987179487179 0.971500217914581
2.53525641025641 0.977910220623016
2.65384615384615 0.962833404541016
2.78205128205128 0.961658298969269
2.91346153846154 0.872323989868164
3.05128205128205 0.957341969013214
3.19871794871795 0.867337822914124
3.34935897435897 0.877156853675842
3.50961538461538 0.894341468811035
3.67628205128205 0.941383183002472
3.8525641025641 0.887681901454926
4.03525641025641 0.903548419475555
4.2275641025641 0.870131611824036
4.42948717948718 0.906260013580322
4.64102564102564 0.911598980426788
4.86217948717949 0.929806411266327
5.09294871794872 0.909204006195068
5.33653846153846 0.913798332214355
5.58974358974359 0.885871529579163
5.85576923076923 0.89288866519928
6.13461538461539 0.887316048145294
6.42628205128205 0.899638175964355
6.73397435897436 0.886183440685272
7.05448717948718 0.895258843898773
7.38782051282051 0.876403033733368
7.74038461538461 0.866112351417542
8.10897435897436 0.843766212463379
8.49679487179487 0.86292439699173
8.90064102564103 0.811742007732391
9.32371794871795 0.844043731689453
9.76923076923077 0.791018426418304
10.2339743589744 0.809145092964172
10.7211538461538 0.744627714157104
11.2307692307692 0.77544641494751
11.7660256410256 0.759141147136688
12.3269230769231 0.774510502815247
12.9134615384615 0.784630239009857
13.5288461538462 0.709848046302795
14.1730769230769 0.729115843772888
14.849358974359 0.771539866924286
15.5544871794872 0.777998089790344
16.2948717948718 0.736065566539764
17.0705128205128 0.726726353168488
17.8846153846154 0.71113646030426
18.7371794871795 0.703156292438507
19.6282051282051 0.703533232212067
20.5641025641026 0.665829658508301
21.5416666666667 0.686701118946075
22.5673076923077 0.672307372093201
23.6442307692308 0.647614181041718
24.7692307692308 0.611117362976074
25.9487179487179 0.640250205993652
27.1826923076923 0.631130695343018
28.4775641025641 0.613438129425049
29.8333333333333 0.572669923305511
31.2564102564103 0.557980358600616
32.7435897435897 0.562351405620575
34.3044871794872 0.532477676868439
35.9358974358974 0.478374540805817
37.6474358974359 0.473836660385132
39.4391025641026 0.408859342336655
41.3173076923077 0.420732021331787
43.2852564102564 0.414896339178085
45.3461538461538 0.41843256354332
47.5064102564103 0.381112337112427
49.7692307692308 0.402549088001251
52.1378205128205 0.326619625091553
54.6217948717949 0.366044819355011
57.2211538461538 0.355786979198456
59.9455128205128 0.324082583189011
62.8012820512821 0.283750951290131
65.7916666666667 0.292596429586411
68.9230769230769 0.273279219865799
72.2051282051282 0.233821824193001
75.6442307692308 0.210522279143333
79.2467948717949 0.2241500467062
83.0192307692308 0.253349304199219
86.974358974359 0.174529269337654
91.1153846153846 0.180669337511063
95.4519230769231 0.163924336433411
100 0.149136736989021
};
\end{axis}

\end{tikzpicture}

  \tikzexternaldisable

  \caption{ \textbf{Mini-batch \ggn versus full-batch \ggn{}:} Overlap between the
    top-$C$ eigenspaces of the mini-batch \ggn and full-batch \ggn during training
    of the \threecthreed network on \cifarten with \sgd{}. For each mini-batch
    size, $5$ different mini-batches are drawn. }\label{vivit::fig:approx_eigenspace_bs}
\end{figure}

% Mini-batch GGN versus full-batch GGN
%
\vivit uses mini-batching to compute a statistical estimator of the full-batch
\ggn{}. This approximation alters the top-$C$ eigenspace, as shown in
\Cref{vivit::fig:approx_eigenspace_bs}: with decreasing mini-batch size, the
approximation carries less and less structure of its full-batch counterpart, as
indicated by dropping overlaps. In addition, at constant batch size, a decrease
in approximation quality can be observed over the course of training. This might
be a valuable insight for the design of second-order optimization methods, where
this structural decay could lead to performance degradation over the course of
the optimization, which has to be compensated for by a growing batch-size (\eg
\citet{martens2010deep} reports that the optimal batch size grows during
training).


% ViViT versus full-batch GGN
%

\begin{figure}[t]
  \centering
  % \textbf{\cifarten \threecthreed \sgd}\\[1mm]
  % defines the pgfplots style "eigspacedefault"
\pgfkeys{/pgfplots/eigspacedefault/.style={
    width=1.0\linewidth,
    height=0.6\linewidth,
    every axis plot/.append style={line width = 1.5pt},
    tick pos = left,
    ylabel near ticks,
    xlabel near ticks,
    xtick align = inside,
    ytick align = inside,
    legend cell align = left,
    legend columns = 4,
    legend pos = south east,
    legend style = {
      fill opacity = 1,
      text opacity = 1,
      font = \footnotesize,
      at={(1, 1.025)},
      anchor=south east,
      column sep=0.25cm,
    },
    legend image post style={scale=2.5},
    xticklabel style = {font = \footnotesize},
    xlabel style = {font = \footnotesize},
    axis line style = {black},
    yticklabel style = {font = \footnotesize},
    ylabel style = {font = \footnotesize},
    title style = {font = \footnotesize},
    grid = major,
    grid style = {dashed}
  }
}

\pgfkeys{/pgfplots/eigspacedefaultapp/.style={
    eigspacedefault,
    height=0.6\linewidth,
    legend columns = 2,
  }
}

\pgfkeys{/pgfplots/eigspacenolegend/.style={
    legend image post style = {scale=0},
    legend style = {
      fill opacity = 0,
      draw opacity = 0,
      text opacity = 0,
      font = \footnotesize,
      at={(1, 1.025)},
      anchor=south east,
      column sep=0.25cm,
    },
  }
}
%%% Local Variables:
%%% mode: latex
%%% TeX-master: "../../thesis"
%%% End:

  \pgfkeys{/pgfplots/zmystyle/.style={
      eigspacedefault,
    }}
  \tikzexternalenable
  % This file was created by tikzplotlib v0.9.7.
\begin{tikzpicture}

\definecolor{color0}{rgb}{0.274509803921569,0.6,0.564705882352941}
\definecolor{color1}{rgb}{0.870588235294118,0.623529411764706,0.0862745098039216}
\definecolor{color2}{rgb}{0.501960784313725,0.184313725490196,0.6}

\begin{axis}[
axis line style={white!10!black},
legend columns=2,
legend style={fill opacity=0.8, draw opacity=1, text opacity=1, at={(0.03,0.03)}, anchor=south west, draw=white!80!black},
log basis x={10},
tick pos=left,
xlabel={epoch (log scale)},
xmajorgrids,
xmin=0.794328234724281, xmax=125.892541179417,
xmode=log,
ylabel={overlap},
ymajorgrids,
ymin=-0.05, ymax=1.05,
zmystyle
]
\addplot [, white!10!black, dashed, forget plot]
table {%
0.794328234724281 1
125.892541179417 1
};
\addplot [, white!10!black, dashed, forget plot]
table {%
0.794328234724281 0
125.892541179417 0
};
\addplot [, black, opacity=0.6, mark=*, mark size=0.5, mark options={solid}, only marks]
table {%
1 0.981747090816498
1.04487179487179 0.98182338476181
1.09615384615385 0.988020360469818
1.1474358974359 0.990345180034637
1.20192307692308 0.991780400276184
1.25961538461538 0.99025171995163
1.32051282051282 0.988504528999329
1.38461538461538 0.98907083272934
1.44871794871795 0.990697205066681
1.51923076923077 0.988103091716766
1.58974358974359 0.985977172851562
1.66666666666667 0.98432582616806
1.74679487179487 0.983087539672852
1.83012820512821 0.981574535369873
1.91666666666667 0.980182111263275
2.00641025641026 0.979159355163574
2.1025641025641 0.978555619716644
2.20512820512821 0.974648296833038
2.30769230769231 0.973026692867279
2.41987179487179 0.970817983150482
2.53525641025641 0.967321872711182
2.65384615384615 0.957172393798828
2.78205128205128 0.96312552690506
2.91346153846154 0.932680547237396
3.05128205128205 0.902878761291504
3.19871794871795 0.872137367725372
3.34935897435897 0.914182841777802
3.50961538461538 0.912082493305206
3.67628205128205 0.924856305122375
3.8525641025641 0.878294587135315
4.03525641025641 0.910778641700745
4.2275641025641 0.842510998249054
4.42948717948718 0.867378413677216
4.64102564102564 0.819855988025665
4.86217948717949 0.837713837623596
5.09294871794872 0.864742696285248
5.33653846153846 0.874320983886719
5.58974358974359 0.825038135051727
5.85576923076923 0.856293499469757
6.13461538461539 0.864015102386475
6.42628205128205 0.861823260784149
6.73397435897436 0.88582843542099
7.05448717948718 0.859955608844757
7.38782051282051 0.815200984477997
7.74038461538461 0.806585609912872
8.10897435897436 0.842734754085541
8.49679487179487 0.815576016902924
8.90064102564103 0.82606166601181
9.32371794871795 0.858394265174866
9.76923076923077 0.754096686840057
10.2339743589744 0.807255387306213
10.7211538461538 0.784341633319855
11.2307692307692 0.759694039821625
11.7660256410256 0.759757816791534
12.3269230769231 0.785604655742645
12.9134615384615 0.772199630737305
13.5288461538462 0.677051663398743
14.1730769230769 0.699923515319824
14.849358974359 0.711808145046234
15.5544871794872 0.689821422100067
16.2948717948718 0.719122231006622
17.0705128205128 0.755955517292023
17.8846153846154 0.685849487781525
18.7371794871795 0.674093425273895
19.6282051282051 0.663297057151794
20.5641025641026 0.651888072490692
21.5416666666667 0.679920017719269
22.5673076923077 0.673120439052582
23.6442307692308 0.54052209854126
24.7692307692308 0.550203800201416
25.9487179487179 0.58035671710968
27.1826923076923 0.525099635124207
28.4775641025641 0.566456913948059
29.8333333333333 0.538295209407806
31.2564102564103 0.560106933116913
32.7435897435897 0.494608640670776
34.3044871794872 0.478361934423447
35.9358974358974 0.481842517852783
37.6474358974359 0.464411735534668
39.4391025641026 0.383963644504547
41.3173076923077 0.414829462766647
43.2852564102564 0.420306831598282
45.3461538461538 0.395040482282639
47.5064102564103 0.368936538696289
49.7692307692308 0.369207710027695
52.1378205128205 0.315311104059219
54.6217948717949 0.306672900915146
57.2211538461538 0.310634672641754
59.9455128205128 0.262401551008224
62.8012820512821 0.281118243932724
65.7916666666667 0.27843850851059
68.9230769230769 0.289015680551529
72.2051282051282 0.271875828504562
75.6442307692308 0.217997744679451
79.2467948717949 0.236163005232811
83.0192307692308 0.199122712016106
86.974358974359 0.217396453022957
91.1153846153846 0.212340220808983
95.4519230769231 0.166822001338005
100 0.159118488430977
};
\addlegendentry{mb 128, exact}
\addplot [, black, opacity=0.6, mark=*, mark size=0.5, mark options={solid}, only marks, forget plot]
table {%
1 0.982020795345306
1.04487179487179 0.953780591487885
1.09615384615385 0.987308979034424
1.1474358974359 0.990592777729034
1.20192307692308 0.992946445941925
1.25961538461538 0.991203784942627
1.32051282051282 0.989980518817902
1.38461538461538 0.990488648414612
1.44871794871795 0.99211198091507
1.51923076923077 0.990089416503906
1.58974358974359 0.988253116607666
1.66666666666667 0.987675189971924
1.74679487179487 0.986547589302063
1.83012820512821 0.98565673828125
1.91666666666667 0.985000550746918
2.00641025641026 0.983888268470764
2.1025641025641 0.980955302715302
2.20512820512821 0.981683909893036
2.30769230769231 0.975873172283173
2.41987179487179 0.978211998939514
2.53525641025641 0.974315464496613
2.65384615384615 0.967435479164124
2.78205128205128 0.967437088489532
2.91346153846154 0.953267276287079
3.05128205128205 0.949415385723114
3.19871794871795 0.875376641750336
3.34935897435897 0.924704968929291
3.50961538461538 0.895968735218048
3.67628205128205 0.92753928899765
3.8525641025641 0.923917770385742
4.03525641025641 0.917343556880951
4.2275641025641 0.846921145915985
4.42948717948718 0.902848660945892
4.64102564102564 0.898207008838654
4.86217948717949 0.904129326343536
5.09294871794872 0.891416072845459
5.33653846153846 0.925188660621643
5.58974358974359 0.896750271320343
5.85576923076923 0.880420506000519
6.13461538461539 0.877892136573792
6.42628205128205 0.904029667377472
6.73397435897436 0.839462876319885
7.05448717948718 0.85766077041626
7.38782051282051 0.849320232868195
7.74038461538461 0.863644242286682
8.10897435897436 0.847573757171631
8.49679487179487 0.859029591083527
8.90064102564103 0.847772538661957
9.32371794871795 0.852554261684418
9.76923076923077 0.770186126232147
10.2339743589744 0.763655602931976
10.7211538461538 0.726811528205872
11.2307692307692 0.776473820209503
11.7660256410256 0.761313438415527
12.3269230769231 0.79255622625351
12.9134615384615 0.743525981903076
13.5288461538462 0.68277233839035
14.1730769230769 0.739355981349945
14.849358974359 0.724602162837982
15.5544871794872 0.686427116394043
16.2948717948718 0.693134009838104
17.0705128205128 0.739706635475159
17.8846153846154 0.696715474128723
18.7371794871795 0.68716698884964
19.6282051282051 0.61690491437912
20.5641025641026 0.614159882068634
21.5416666666667 0.721479594707489
22.5673076923077 0.648724734783173
23.6442307692308 0.585017144680023
24.7692307692308 0.590175628662109
25.9487179487179 0.524260938167572
27.1826923076923 0.562411189079285
28.4775641025641 0.575817584991455
29.8333333333333 0.535391569137573
31.2564102564103 0.604447066783905
32.7435897435897 0.502214431762695
34.3044871794872 0.502260327339172
35.9358974358974 0.454190403223038
37.6474358974359 0.520226001739502
39.4391025641026 0.448612034320831
41.3173076923077 0.427878677845001
43.2852564102564 0.384945422410965
45.3461538461538 0.419080346822739
47.5064102564103 0.331002056598663
49.7692307692308 0.307637602090836
52.1378205128205 0.359945029020309
54.6217948717949 0.30944037437439
57.2211538461538 0.262731164693832
59.9455128205128 0.280141055583954
62.8012820512821 0.229214191436768
65.7916666666667 0.199491798877716
68.9230769230769 0.216717630624771
72.2051282051282 0.192475125193596
75.6442307692308 0.257141083478928
79.2467948717949 0.148212775588036
83.0192307692308 0.18046860396862
86.974358974359 0.187112584710121
91.1153846153846 0.136196181178093
95.4519230769231 0.145475029945374
100 0.141011148691177
};
\addplot [, black, opacity=0.6, mark=*, mark size=0.5, mark options={solid}, only marks, forget plot]
table {%
1 0.984170615673065
1.04487179487179 0.962684333324432
1.09615384615385 0.990463376045227
1.1474358974359 0.992333114147186
1.20192307692308 0.994071006774902
1.25961538461538 0.993533909320831
1.32051282051282 0.991840958595276
1.38461538461538 0.991823792457581
1.44871794871795 0.993282914161682
1.51923076923077 0.990627884864807
1.58974358974359 0.990484058856964
1.66666666666667 0.988605320453644
1.74679487179487 0.988952457904816
1.83012820512821 0.986732482910156
1.91666666666667 0.98617947101593
2.00641025641026 0.986214578151703
2.1025641025641 0.982601761817932
2.20512820512821 0.983159720897675
2.30769230769231 0.981453120708466
2.41987179487179 0.982396304607391
2.53525641025641 0.979023456573486
2.65384615384615 0.975050568580627
2.78205128205128 0.960993766784668
2.91346153846154 0.952524185180664
3.05128205128205 0.926346242427826
3.19871794871795 0.874208569526672
3.34935897435897 0.890124142169952
3.50961538461538 0.870679318904877
3.67628205128205 0.927426934242249
3.8525641025641 0.894946277141571
4.03525641025641 0.848156630992889
4.2275641025641 0.932078003883362
4.42948717948718 0.869904816150665
4.64102564102564 0.875738322734833
4.86217948717949 0.840713024139404
5.09294871794872 0.840531945228577
5.33653846153846 0.923191964626312
5.58974358974359 0.840282380580902
5.85576923076923 0.817100942134857
6.13461538461539 0.834160268306732
6.42628205128205 0.857244312763214
6.73397435897436 0.873536288738251
7.05448717948718 0.870953977108002
7.38782051282051 0.872702240943909
7.74038461538461 0.864868760108948
8.10897435897436 0.791886925697327
8.49679487179487 0.808820724487305
8.90064102564103 0.809190213680267
9.32371794871795 0.802600562572479
9.76923076923077 0.809189915657043
10.2339743589744 0.766855239868164
10.7211538461538 0.817921936511993
11.2307692307692 0.780461728572845
11.7660256410256 0.73890221118927
12.3269230769231 0.792637348175049
12.9134615384615 0.758075952529907
13.5288461538462 0.795771420001984
14.1730769230769 0.707125425338745
14.849358974359 0.722668051719666
15.5544871794872 0.65700376033783
16.2948717948718 0.729147553443909
17.0705128205128 0.699276089668274
17.8846153846154 0.688258171081543
18.7371794871795 0.630997002124786
19.6282051282051 0.641006469726562
20.5641025641026 0.641031563282013
21.5416666666667 0.611485123634338
22.5673076923077 0.636362671852112
23.6442307692308 0.579290926456451
24.7692307692308 0.580309808254242
25.9487179487179 0.571253001689911
27.1826923076923 0.566135585308075
28.4775641025641 0.604471206665039
29.8333333333333 0.568235576152802
31.2564102564103 0.54334819316864
32.7435897435897 0.529070019721985
34.3044871794872 0.544546067714691
35.9358974358974 0.482943207025528
37.6474358974359 0.538393676280975
39.4391025641026 0.463052660226822
41.3173076923077 0.472070604562759
43.2852564102564 0.442756474018097
45.3461538461538 0.433039039373398
47.5064102564103 0.403144657611847
49.7692307692308 0.368089973926544
52.1378205128205 0.375291228294373
54.6217948717949 0.373930543661118
57.2211538461538 0.284169286489487
59.9455128205128 0.306993454694748
62.8012820512821 0.282681196928024
65.7916666666667 0.288691371679306
68.9230769230769 0.285261958837509
72.2051282051282 0.256960988044739
75.6442307692308 0.230094075202942
79.2467948717949 0.2147067040205
83.0192307692308 0.226162105798721
86.974358974359 0.186426684260368
91.1153846153846 0.155625566840172
95.4519230769231 0.171306014060974
100 0.173411771655083
};
\addplot [, black, opacity=0.6, mark=*, mark size=0.5, mark options={solid}, only marks, forget plot]
table {%
1 0.984564304351807
1.04487179487179 0.975940704345703
1.09615384615385 0.990538895130157
1.1474358974359 0.991618752479553
1.20192307692308 0.993819415569305
1.25961538461538 0.993273675441742
1.32051282051282 0.991371631622314
1.38461538461538 0.991176605224609
1.44871794871795 0.993327617645264
1.51923076923077 0.990107357501984
1.58974358974359 0.98944091796875
1.66666666666667 0.987054467201233
1.74679487179487 0.987636685371399
1.83012820512821 0.98361474275589
1.91666666666667 0.983842968940735
2.00641025641026 0.983477294445038
2.1025641025641 0.97759610414505
2.20512820512821 0.981981098651886
2.30769230769231 0.977427303791046
2.41987179487179 0.94308990240097
2.53525641025641 0.885430157184601
2.65384615384615 0.940607070922852
2.78205128205128 0.879908382892609
2.91346153846154 0.963673532009125
3.05128205128205 0.902095317840576
3.19871794871795 0.928976535797119
3.34935897435897 0.907859265804291
3.50961538461538 0.874574661254883
3.67628205128205 0.870367050170898
3.8525641025641 0.943110883235931
4.03525641025641 0.854616105556488
4.2275641025641 0.914628028869629
4.42948717948718 0.825600624084473
4.64102564102564 0.887604176998138
4.86217948717949 0.835559844970703
5.09294871794872 0.886755168437958
5.33653846153846 0.882330358028412
5.58974358974359 0.816561877727509
5.85576923076923 0.81672191619873
6.13461538461539 0.833069980144501
6.42628205128205 0.844195187091827
6.73397435897436 0.819805800914764
7.05448717948718 0.808649480342865
7.38782051282051 0.807758152484894
7.74038461538461 0.759325206279755
8.10897435897436 0.799039661884308
8.49679487179487 0.880455434322357
8.90064102564103 0.810913264751434
9.32371794871795 0.796833634376526
9.76923076923077 0.757974803447723
10.2339743589744 0.751160323619843
10.7211538461538 0.791345059871674
11.2307692307692 0.769242703914642
11.7660256410256 0.763507902622223
12.3269230769231 0.753553450107574
12.9134615384615 0.752502381801605
13.5288461538462 0.724954068660736
14.1730769230769 0.736444771289825
14.849358974359 0.718579232692719
15.5544871794872 0.71355402469635
16.2948717948718 0.689074337482452
17.0705128205128 0.689262390136719
17.8846153846154 0.716107070446014
18.7371794871795 0.738085210323334
19.6282051282051 0.694091975688934
20.5641025641026 0.601486027240753
21.5416666666667 0.648961961269379
22.5673076923077 0.697732865810394
23.6442307692308 0.650158584117889
24.7692307692308 0.592214584350586
25.9487179487179 0.591780006885529
27.1826923076923 0.588055074214935
28.4775641025641 0.569010436534882
29.8333333333333 0.524317264556885
31.2564102564103 0.545189797878265
32.7435897435897 0.537986934185028
34.3044871794872 0.493544101715088
35.9358974358974 0.51635879278183
37.6474358974359 0.495911508798599
39.4391025641026 0.452186733484268
41.3173076923077 0.451370060443878
43.2852564102564 0.443919867277145
45.3461538461538 0.395095527172089
47.5064102564103 0.411824524402618
49.7692307692308 0.371033251285553
52.1378205128205 0.385174244642258
54.6217948717949 0.360418766736984
57.2211538461538 0.331921488046646
59.9455128205128 0.321492403745651
62.8012820512821 0.306410759687424
65.7916666666667 0.375531643629074
68.9230769230769 0.285832017660141
72.2051282051282 0.273515850305557
75.6442307692308 0.241822242736816
79.2467948717949 0.254934042692184
83.0192307692308 0.242678984999657
86.974358974359 0.241615250706673
91.1153846153846 0.212708428502083
95.4519230769231 0.185712978243828
100 0.191799759864807
};
\addplot [, black, opacity=0.6, mark=*, mark size=0.5, mark options={solid}, only marks, forget plot]
table {%
1 0.982150375843048
1.04487179487179 0.948779225349426
1.09615384615385 0.988999545574188
1.1474358974359 0.99137419462204
1.20192307692308 0.993538856506348
1.25961538461538 0.99172830581665
1.32051282051282 0.990062236785889
1.38461538461538 0.989916980266571
1.44871794871795 0.99199241399765
1.51923076923077 0.989880263805389
1.58974358974359 0.988528251647949
1.66666666666667 0.987747013568878
1.74679487179487 0.987243294715881
1.83012820512821 0.986089169979095
1.91666666666667 0.984426498413086
2.00641025641026 0.984810292720795
2.1025641025641 0.981416404247284
2.20512820512821 0.980676293373108
2.30769230769231 0.975415885448456
2.41987179487179 0.971500217914581
2.53525641025641 0.977910220623016
2.65384615384615 0.962833404541016
2.78205128205128 0.961658298969269
2.91346153846154 0.872323989868164
3.05128205128205 0.957341969013214
3.19871794871795 0.867337822914124
3.34935897435897 0.877156853675842
3.50961538461538 0.894341468811035
3.67628205128205 0.941383183002472
3.8525641025641 0.887681901454926
4.03525641025641 0.903548419475555
4.2275641025641 0.870131611824036
4.42948717948718 0.906260013580322
4.64102564102564 0.911598980426788
4.86217948717949 0.929806411266327
5.09294871794872 0.909204006195068
5.33653846153846 0.913798332214355
5.58974358974359 0.885871529579163
5.85576923076923 0.89288866519928
6.13461538461539 0.887316048145294
6.42628205128205 0.899638175964355
6.73397435897436 0.886183440685272
7.05448717948718 0.895258843898773
7.38782051282051 0.876403033733368
7.74038461538461 0.866112351417542
8.10897435897436 0.843766212463379
8.49679487179487 0.86292439699173
8.90064102564103 0.811742007732391
9.32371794871795 0.844043731689453
9.76923076923077 0.791018426418304
10.2339743589744 0.809145092964172
10.7211538461538 0.744627714157104
11.2307692307692 0.77544641494751
11.7660256410256 0.759141147136688
12.3269230769231 0.774510502815247
12.9134615384615 0.784630239009857
13.5288461538462 0.709848046302795
14.1730769230769 0.729115843772888
14.849358974359 0.771539866924286
15.5544871794872 0.777998089790344
16.2948717948718 0.736065566539764
17.0705128205128 0.726726353168488
17.8846153846154 0.71113646030426
18.7371794871795 0.703156292438507
19.6282051282051 0.703533232212067
20.5641025641026 0.665829658508301
21.5416666666667 0.686701118946075
22.5673076923077 0.672307372093201
23.6442307692308 0.647614181041718
24.7692307692308 0.611117362976074
25.9487179487179 0.640250205993652
27.1826923076923 0.631130695343018
28.4775641025641 0.613438129425049
29.8333333333333 0.572669923305511
31.2564102564103 0.557980358600616
32.7435897435897 0.562351405620575
34.3044871794872 0.532477676868439
35.9358974358974 0.478374540805817
37.6474358974359 0.473836660385132
39.4391025641026 0.408859342336655
41.3173076923077 0.420732021331787
43.2852564102564 0.414896339178085
45.3461538461538 0.41843256354332
47.5064102564103 0.381112337112427
49.7692307692308 0.402549088001251
52.1378205128205 0.326619625091553
54.6217948717949 0.366044819355011
57.2211538461538 0.355786979198456
59.9455128205128 0.324082583189011
62.8012820512821 0.283750951290131
65.7916666666667 0.292596429586411
68.9230769230769 0.273279219865799
72.2051282051282 0.233821824193001
75.6442307692308 0.210522279143333
79.2467948717949 0.2241500467062
83.0192307692308 0.253349304199219
86.974358974359 0.174529269337654
91.1153846153846 0.180669337511063
95.4519230769231 0.163924336433411
100 0.149136736989021
};
\addplot [, color0, opacity=0.6, mark=diamond*, mark size=0.5, mark options={solid}, only marks]
table {%
1 0.888289570808411
1.04487179487179 0.850245416164398
1.09615384615385 0.929787933826447
1.1474358974359 0.935880959033966
1.20192307692308 0.943499863147736
1.25961538461538 0.929302036762238
1.32051282051282 0.91467422246933
1.38461538461538 0.918963551521301
1.44871794871795 0.93555736541748
1.51923076923077 0.922814548015594
1.58974358974359 0.914416491985321
1.66666666666667 0.912550449371338
1.74679487179487 0.902347981929779
1.83012820512821 0.901371419429779
1.91666666666667 0.895730793476105
2.00641025641026 0.887700259685516
2.1025641025641 0.792950987815857
2.20512820512821 0.85385400056839
2.30769230769231 0.823620975017548
2.41987179487179 0.763083279132843
2.53525641025641 0.811955094337463
2.65384615384615 0.785270512104034
2.78205128205128 0.759598016738892
2.91346153846154 0.765279948711395
3.05128205128205 0.741091549396515
3.19871794871795 0.73689204454422
3.34935897435897 0.735926568508148
3.50961538461538 0.729260742664337
3.67628205128205 0.66669362783432
3.8525641025641 0.702483832836151
4.03525641025641 0.658973634243011
4.2275641025641 0.651246964931488
4.42948717948718 0.647138714790344
4.64102564102564 0.641364753246307
4.86217948717949 0.639494895935059
5.09294871794872 0.648367762565613
5.33653846153846 0.65986031293869
5.58974358974359 0.581650614738464
5.85576923076923 0.556316316127777
6.13461538461539 0.604955017566681
6.42628205128205 0.562110304832458
6.73397435897436 0.52776974439621
7.05448717948718 0.533655226230621
7.38782051282051 0.529252648353577
7.74038461538461 0.506701648235321
8.10897435897436 0.551192760467529
8.49679487179487 0.511806011199951
8.90064102564103 0.494422018527985
9.32371794871795 0.448191255331039
9.76923076923077 0.395153999328613
10.2339743589744 0.447132438421249
10.7211538461538 0.464736312627792
11.2307692307692 0.418795645236969
11.7660256410256 0.376513570547104
12.3269230769231 0.406778633594513
12.9134615384615 0.364027082920074
13.5288461538462 0.381784528493881
14.1730769230769 0.329464435577393
14.849358974359 0.35334438085556
15.5544871794872 0.4212606549263
16.2948717948718 0.355994403362274
17.0705128205128 0.352870553731918
17.8846153846154 0.343120992183685
18.7371794871795 0.272719025611877
19.6282051282051 0.329196721315384
20.5641025641026 0.286043673753738
21.5416666666667 0.334159225225449
22.5673076923077 0.293594181537628
23.6442307692308 0.286035478115082
24.7692307692308 0.308116465806961
25.9487179487179 0.244923546910286
27.1826923076923 0.239613577723503
28.4775641025641 0.274717658758163
29.8333333333333 0.269978433847427
31.2564102564103 0.247582659125328
32.7435897435897 0.2376539260149
34.3044871794872 0.218981549143791
35.9358974358974 0.221796855330467
37.6474358974359 0.223196133971214
39.4391025641026 0.208313748240471
41.3173076923077 0.215252637863159
43.2852564102564 0.19606702029705
45.3461538461538 0.181623503565788
47.5064102564103 0.17289674282074
49.7692307692308 0.160922631621361
52.1378205128205 0.162959381937981
54.6217948717949 0.163500472903252
57.2211538461538 0.164205715060234
59.9455128205128 0.175009906291962
62.8012820512821 0.140065833926201
65.7916666666667 0.138864755630493
68.9230769230769 0.140384361147881
72.2051282051282 0.141933500766754
75.6442307692308 0.139193296432495
79.2467948717949 0.146768793463707
83.0192307692308 0.128423228859901
86.974358974359 0.157832458615303
91.1153846153846 0.121040962636471
95.4519230769231 0.108292534947395
100 0.112434543669224
};
\addlegendentry{sub 16, exact}
\addplot [, color0, opacity=0.6, mark=diamond*, mark size=0.5, mark options={solid}, only marks, forget plot]
table {%
1 0.893791139125824
1.04487179487179 0.854173302650452
1.09615384615385 0.928434371948242
1.1474358974359 0.936124265193939
1.20192307692308 0.951777875423431
1.25961538461538 0.949469029903412
1.32051282051282 0.934574723243713
1.38461538461538 0.93833714723587
1.44871794871795 0.948976516723633
1.51923076923077 0.937324702739716
1.58974358974359 0.919515609741211
1.66666666666667 0.908123672008514
1.74679487179487 0.890411198139191
1.83012820512821 0.876589953899384
1.91666666666667 0.890449941158295
2.00641025641026 0.867175877094269
2.1025641025641 0.783797204494476
2.20512820512821 0.867137432098389
2.30769230769231 0.816940784454346
2.41987179487179 0.76775187253952
2.53525641025641 0.754798829555511
2.65384615384615 0.783145546913147
2.78205128205128 0.714383900165558
2.91346153846154 0.729553878307343
3.05128205128205 0.700503766536713
3.19871794871795 0.735346615314484
3.34935897435897 0.740528881549835
3.50961538461538 0.723609924316406
3.67628205128205 0.666880667209625
3.8525641025641 0.656340777873993
4.03525641025641 0.642926037311554
4.2275641025641 0.650541007518768
4.42948717948718 0.601711690425873
4.64102564102564 0.618374109268188
4.86217948717949 0.616057097911835
5.09294871794872 0.58877432346344
5.33653846153846 0.576996743679047
5.58974358974359 0.57886803150177
5.85576923076923 0.533031165599823
6.13461538461539 0.549373686313629
6.42628205128205 0.572380721569061
6.73397435897436 0.520179390907288
7.05448717948718 0.518650531768799
7.38782051282051 0.518807888031006
7.74038461538461 0.473583698272705
8.10897435897436 0.481939882040024
8.49679487179487 0.468410223722458
8.90064102564103 0.464531630277634
9.32371794871795 0.461068838834763
9.76923076923077 0.438083410263062
10.2339743589744 0.419620484113693
10.7211538461538 0.430715948343277
11.2307692307692 0.413104265928268
11.7660256410256 0.383857488632202
12.3269230769231 0.369482487440109
12.9134615384615 0.372629731893539
13.5288461538462 0.350613206624985
14.1730769230769 0.354642152786255
14.849358974359 0.375771790742874
15.5544871794872 0.367782175540924
16.2948717948718 0.326392233371735
17.0705128205128 0.345154494047165
17.8846153846154 0.356812238693237
18.7371794871795 0.324780136346817
19.6282051282051 0.335758477449417
20.5641025641026 0.316658705472946
21.5416666666667 0.336475849151611
22.5673076923077 0.314805507659912
23.6442307692308 0.283265024423599
24.7692307692308 0.285132855176926
25.9487179487179 0.288173645734787
27.1826923076923 0.288888901472092
28.4775641025641 0.290199965238571
29.8333333333333 0.269994080066681
31.2564102564103 0.271203547716141
32.7435897435897 0.248423621058464
34.3044871794872 0.245234161615372
35.9358974358974 0.241854622960091
37.6474358974359 0.199167862534523
39.4391025641026 0.2262252420187
41.3173076923077 0.213113889098167
43.2852564102564 0.20268402993679
45.3461538461538 0.192482188344002
47.5064102564103 0.186557650566101
49.7692307692308 0.191979929804802
52.1378205128205 0.18522472679615
54.6217948717949 0.198280215263367
57.2211538461538 0.178179949522018
59.9455128205128 0.178136929869652
62.8012820512821 0.178591907024384
65.7916666666667 0.157031685113907
68.9230769230769 0.17550702393055
72.2051282051282 0.14932969212532
75.6442307692308 0.154043957591057
79.2467948717949 0.147724434733391
83.0192307692308 0.154264211654663
86.974358974359 0.131414070725441
91.1153846153846 0.137487486004829
95.4519230769231 0.148206621408463
100 0.140420719981194
};
\addplot [, color0, opacity=0.6, mark=diamond*, mark size=0.5, mark options={solid}, only marks, forget plot]
table {%
1 0.891717374324799
1.04487179487179 0.869342982769012
1.09615384615385 0.919399440288544
1.1474358974359 0.933474004268646
1.20192307692308 0.946918129920959
1.25961538461538 0.93341988325119
1.32051282051282 0.916859447956085
1.38461538461538 0.923144817352295
1.44871794871795 0.936473965644836
1.51923076923077 0.929627239704132
1.58974358974359 0.902821183204651
1.66666666666667 0.89877837896347
1.74679487179487 0.891434490680695
1.83012820512821 0.876318395137787
1.91666666666667 0.880023181438446
2.00641025641026 0.863736927509308
2.1025641025641 0.855601489543915
2.20512820512821 0.851014077663422
2.30769230769231 0.808860599994659
2.41987179487179 0.816703021526337
2.53525641025641 0.831770539283752
2.65384615384615 0.75491589307785
2.78205128205128 0.755659341812134
2.91346153846154 0.737481713294983
3.05128205128205 0.70359867811203
3.19871794871795 0.668097019195557
3.34935897435897 0.68815416097641
3.50961538461538 0.632323503494263
3.67628205128205 0.611617684364319
3.8525641025641 0.65784627199173
4.03525641025641 0.63075464963913
4.2275641025641 0.621694028377533
4.42948717948718 0.607796907424927
4.64102564102564 0.565841376781464
4.86217948717949 0.586063385009766
5.09294871794872 0.576301455497742
5.33653846153846 0.590676605701447
5.58974358974359 0.52356892824173
5.85576923076923 0.521511495113373
6.13461538461539 0.554125010967255
6.42628205128205 0.513071358203888
6.73397435897436 0.517736494541168
7.05448717948718 0.496909350156784
7.38782051282051 0.506617426872253
7.74038461538461 0.479956835508347
8.10897435897436 0.469402849674225
8.49679487179487 0.461047023534775
8.90064102564103 0.462979704141617
9.32371794871795 0.486325353384018
9.76923076923077 0.418221771717072
10.2339743589744 0.390541791915894
10.7211538461538 0.439920097589493
11.2307692307692 0.37975537776947
11.7660256410256 0.459371298551559
12.3269230769231 0.398192316293716
12.9134615384615 0.354040682315826
13.5288461538462 0.351535648107529
14.1730769230769 0.344239860773087
14.849358974359 0.382947832345963
15.5544871794872 0.344797998666763
16.2948717948718 0.305987119674683
17.0705128205128 0.370094865560532
17.8846153846154 0.336247593164444
18.7371794871795 0.327902793884277
19.6282051282051 0.346059888601303
20.5641025641026 0.298206001520157
21.5416666666667 0.306540340185165
22.5673076923077 0.291296809911728
23.6442307692308 0.279941380023956
24.7692307692308 0.271569490432739
25.9487179487179 0.269172310829163
27.1826923076923 0.247764781117439
28.4775641025641 0.239833638072014
29.8333333333333 0.243390426039696
31.2564102564103 0.238371476531029
32.7435897435897 0.233770713210106
34.3044871794872 0.221039280295372
35.9358974358974 0.256289690732956
37.6474358974359 0.220509633421898
39.4391025641026 0.216097861528397
41.3173076923077 0.194299146533012
43.2852564102564 0.227774649858475
45.3461538461538 0.197210982441902
47.5064102564103 0.196835562586784
49.7692307692308 0.178263410925865
52.1378205128205 0.15285649895668
54.6217948717949 0.186899214982986
57.2211538461538 0.157098278403282
59.9455128205128 0.169110685586929
62.8012820512821 0.148695811629295
65.7916666666667 0.15706293284893
68.9230769230769 0.119244933128357
72.2051282051282 0.145460233092308
75.6442307692308 0.131833106279373
79.2467948717949 0.146838709712029
83.0192307692308 0.137946113944054
86.974358974359 0.124879911541939
91.1153846153846 0.119742766022682
95.4519230769231 0.119828082621098
100 0.123383574187756
};
\addplot [, color0, opacity=0.6, mark=diamond*, mark size=0.5, mark options={solid}, only marks, forget plot]
table {%
1 0.893024384975433
1.04487179487179 0.910530865192413
1.09615384615385 0.927744090557098
1.1474358974359 0.944096386432648
1.20192307692308 0.955252647399902
1.25961538461538 0.958257853984833
1.32051282051282 0.938665211200714
1.38461538461538 0.943126678466797
1.44871794871795 0.951784312725067
1.51923076923077 0.925331115722656
1.58974358974359 0.930963158607483
1.66666666666667 0.833288013935089
1.74679487179487 0.909896373748779
1.83012820512821 0.898792266845703
1.91666666666667 0.821410953998566
2.00641025641026 0.892078995704651
2.1025641025641 0.844248235225677
2.20512820512821 0.850242078304291
2.30769230769231 0.863114833831787
2.41987179487179 0.824883937835693
2.53525641025641 0.835236191749573
2.65384615384615 0.76833975315094
2.78205128205128 0.82883894443512
2.91346153846154 0.761045396327972
3.05128205128205 0.724739491939545
3.19871794871795 0.70199453830719
3.34935897435897 0.702546060085297
3.50961538461538 0.671069800853729
3.67628205128205 0.702909350395203
3.8525641025641 0.666581273078918
4.03525641025641 0.676584422588348
4.2275641025641 0.634079396724701
4.42948717948718 0.654314935207367
4.64102564102564 0.595455348491669
4.86217948717949 0.63051849603653
5.09294871794872 0.571516454219818
5.33653846153846 0.576733529567719
5.58974358974359 0.55988883972168
5.85576923076923 0.546994626522064
6.13461538461539 0.537533521652222
6.42628205128205 0.574197113513947
6.73397435897436 0.569141805171967
7.05448717948718 0.524285912513733
7.38782051282051 0.53972989320755
7.74038461538461 0.519838452339172
8.10897435897436 0.464834213256836
8.49679487179487 0.453800588846207
8.90064102564103 0.463179796934128
9.32371794871795 0.502031803131104
9.76923076923077 0.397626042366028
10.2339743589744 0.406128019094467
10.7211538461538 0.443699091672897
11.2307692307692 0.344123691320419
11.7660256410256 0.414508253335953
12.3269230769231 0.377609103918076
12.9134615384615 0.339062422513962
13.5288461538462 0.397669851779938
14.1730769230769 0.373747766017914
14.849358974359 0.369808584451675
15.5544871794872 0.361682713031769
16.2948717948718 0.33836567401886
17.0705128205128 0.347890764474869
17.8846153846154 0.314610540866852
18.7371794871795 0.328371733427048
19.6282051282051 0.321426630020142
20.5641025641026 0.319091945886612
21.5416666666667 0.307366997003555
22.5673076923077 0.272907763719559
23.6442307692308 0.281391710042953
24.7692307692308 0.288220018148422
25.9487179487179 0.269860476255417
27.1826923076923 0.265939146280289
28.4775641025641 0.236077472567558
29.8333333333333 0.240741729736328
31.2564102564103 0.219699457287788
32.7435897435897 0.274763643741608
34.3044871794872 0.266452521085739
35.9358974358974 0.224049806594849
37.6474358974359 0.200556501746178
39.4391025641026 0.233722046017647
41.3173076923077 0.23330469429493
43.2852564102564 0.200644060969353
45.3461538461538 0.241697236895561
47.5064102564103 0.200261399149895
49.7692307692308 0.177268266677856
52.1378205128205 0.186943978071213
54.6217948717949 0.17124892771244
57.2211538461538 0.162881061434746
59.9455128205128 0.169677838683128
62.8012820512821 0.166471898555756
65.7916666666667 0.169251278042793
68.9230769230769 0.129498943686485
72.2051282051282 0.148728296160698
75.6442307692308 0.136617735028267
79.2467948717949 0.153871297836304
83.0192307692308 0.13096396625042
86.974358974359 0.142919063568115
91.1153846153846 0.133864000439644
95.4519230769231 0.120287202298641
100 0.12443633377552
};
\addplot [, color0, opacity=0.6, mark=diamond*, mark size=0.5, mark options={solid}, only marks, forget plot]
table {%
1 0.845916569232941
1.04487179487179 0.842062175273895
1.09615384615385 0.883130252361298
1.1474358974359 0.866719722747803
1.20192307692308 0.932085931301117
1.25961538461538 0.907064616680145
1.32051282051282 0.91059821844101
1.38461538461538 0.915305554866791
1.44871794871795 0.930165946483612
1.51923076923077 0.923670768737793
1.58974358974359 0.91000759601593
1.66666666666667 0.909145176410675
1.74679487179487 0.898723781108856
1.83012820512821 0.891693711280823
1.91666666666667 0.896263718605042
2.00641025641026 0.864108681678772
2.1025641025641 0.858955085277557
2.20512820512821 0.875910699367523
2.30769230769231 0.840061843395233
2.41987179487179 0.794665455818176
2.53525641025641 0.783516585826874
2.65384615384615 0.77463561296463
2.78205128205128 0.77858829498291
2.91346153846154 0.730015754699707
3.05128205128205 0.738743484020233
3.19871794871795 0.690069735050201
3.34935897435897 0.72129899263382
3.50961538461538 0.687578856945038
3.67628205128205 0.641691863536835
3.8525641025641 0.645978629589081
4.03525641025641 0.677906334400177
4.2275641025641 0.61445951461792
4.42948717948718 0.622399926185608
4.64102564102564 0.644111275672913
4.86217948717949 0.6007000207901
5.09294871794872 0.616980731487274
5.33653846153846 0.566904187202454
5.58974358974359 0.60029661655426
5.85576923076923 0.575075745582581
6.13461538461539 0.603174209594727
6.42628205128205 0.585057556629181
6.73397435897436 0.508771896362305
7.05448717948718 0.515156269073486
7.38782051282051 0.510006844997406
7.74038461538461 0.515608489513397
8.10897435897436 0.480144828557968
8.49679487179487 0.522282242774963
8.90064102564103 0.480451196432114
9.32371794871795 0.489544928073883
9.76923076923077 0.445629805326462
10.2339743589744 0.423765182495117
10.7211538461538 0.423987001180649
11.2307692307692 0.430731594562531
11.7660256410256 0.405017465353012
12.3269230769231 0.439331352710724
12.9134615384615 0.392494261264801
13.5288461538462 0.372928828001022
14.1730769230769 0.371385663747787
14.849358974359 0.352378219366074
15.5544871794872 0.366654723882675
16.2948717948718 0.350931406021118
17.0705128205128 0.378520339727402
17.8846153846154 0.348050802946091
18.7371794871795 0.291749387979507
19.6282051282051 0.347710102796555
20.5641025641026 0.307074934244156
21.5416666666667 0.374862015247345
22.5673076923077 0.269711971282959
23.6442307692308 0.2686927318573
24.7692307692308 0.261369079351425
25.9487179487179 0.259682148694992
27.1826923076923 0.249882504343987
28.4775641025641 0.248703479766846
29.8333333333333 0.226085141301155
31.2564102564103 0.258959621191025
32.7435897435897 0.226390510797501
34.3044871794872 0.220880016684532
35.9358974358974 0.236545950174332
37.6474358974359 0.222899720072746
39.4391025641026 0.222450017929077
41.3173076923077 0.201570630073547
43.2852564102564 0.190586537122726
45.3461538461538 0.201367244124413
47.5064102564103 0.190268754959106
49.7692307692308 0.172049760818481
52.1378205128205 0.175165176391602
54.6217948717949 0.186083629727364
57.2211538461538 0.165031909942627
59.9455128205128 0.163544625043869
62.8012820512821 0.163207098841667
65.7916666666667 0.149850562214851
68.9230769230769 0.133799344301224
72.2051282051282 0.141920894384384
75.6442307692308 0.150656268000603
79.2467948717949 0.156617060303688
83.0192307692308 0.146520391106606
86.974358974359 0.126060798764229
91.1153846153846 0.131151244044304
95.4519230769231 0.125335350632668
100 0.136466324329376
};
\addplot [, color1, opacity=0.6, mark=square*, mark size=0.5, mark options={solid}, only marks]
table {%
1 0.823391258716583
1.04487179487179 0.842215180397034
1.09615384615385 0.889334976673126
1.1474358974359 0.902535855770111
1.20192307692308 0.935880661010742
1.25961538461538 0.924263179302216
1.32051282051282 0.920876502990723
1.38461538461538 0.927456080913544
1.44871794871795 0.933768689632416
1.51923076923077 0.918422698974609
1.58974358974359 0.910473167896271
1.66666666666667 0.921622395515442
1.74679487179487 0.897418618202209
1.83012820512821 0.885995090007782
1.91666666666667 0.877628803253174
2.00641025641026 0.901624858379364
2.1025641025641 0.827723920345306
2.20512820512821 0.905844151973724
2.30769230769231 0.89058381319046
2.41987179487179 0.849358081817627
2.53525641025641 0.852563083171844
2.65384615384615 0.858367085456848
2.78205128205128 0.768650770187378
2.91346153846154 0.777651727199554
3.05128205128205 0.755616664886475
3.19871794871795 0.764954686164856
3.34935897435897 0.758058190345764
3.50961538461538 0.774447023868561
3.67628205128205 0.762955844402313
3.8525641025641 0.739735305309296
4.03525641025641 0.729641914367676
4.2275641025641 0.758092403411865
4.42948717948718 0.715419411659241
4.64102564102564 0.63623720407486
4.86217948717949 0.686236143112183
5.09294871794872 0.710280776023865
5.33653846153846 0.694095611572266
5.58974358974359 0.62928718328476
5.85576923076923 0.666550040245056
6.13461538461539 0.600450098514557
6.42628205128205 0.638733327388763
6.73397435897436 0.625997245311737
7.05448717948718 0.595963895320892
7.38782051282051 0.550736963748932
7.74038461538461 0.620732188224792
8.10897435897436 0.582258999347687
8.49679487179487 0.603067696094513
8.90064102564103 0.600725829601288
9.32371794871795 0.57937616109848
9.76923076923077 0.532348215579987
10.2339743589744 0.524379074573517
10.7211538461538 0.473377704620361
11.2307692307692 0.495040386915207
11.7660256410256 0.514038503170013
12.3269230769231 0.478236585855484
12.9134615384615 0.492099970579147
13.5288461538462 0.475808918476105
14.1730769230769 0.412688463926315
14.849358974359 0.432154387235641
15.5544871794872 0.474344223737717
16.2948717948718 0.440344244241714
17.0705128205128 0.492268294095993
17.8846153846154 0.40320560336113
18.7371794871795 0.369570881128311
19.6282051282051 0.338529855012894
20.5641025641026 0.320983022451401
21.5416666666667 0.413980633020401
22.5673076923077 0.378960222005844
23.6442307692308 0.365000575780869
24.7692307692308 0.341664463281631
25.9487179487179 0.303045481443405
27.1826923076923 0.322080224752426
28.4775641025641 0.256156295537949
29.8333333333333 0.299985557794571
31.2564102564103 0.30941778421402
32.7435897435897 0.295577198266983
34.3044871794872 0.285488218069077
35.9358974358974 0.253482073545456
37.6474358974359 0.209031566977501
39.4391025641026 0.19727049767971
41.3173076923077 0.220738098025322
43.2852564102564 0.199967280030251
45.3461538461538 0.201494887471199
47.5064102564103 0.215584620833397
49.7692307692308 0.198384687304497
52.1378205128205 0.203752189874649
54.6217948717949 0.163193479180336
57.2211538461538 0.20632492005825
59.9455128205128 0.152495786547661
62.8012820512821 0.170067459344864
65.7916666666667 0.143556877970695
68.9230769230769 0.156829744577408
72.2051282051282 0.135373160243034
75.6442307692308 0.141417667269707
79.2467948717949 0.140521839261055
83.0192307692308 0.148325487971306
86.974358974359 0.142252922058105
91.1153846153846 0.127506420016289
95.4519230769231 0.133091166615486
100 0.140429183840752
};
\addlegendentry{mb 128, mc 1}
\addplot [, color1, opacity=0.6, mark=square*, mark size=0.5, mark options={solid}, only marks, forget plot]
table {%
1 0.87957763671875
1.04487179487179 0.828774571418762
1.09615384615385 0.874753594398499
1.1474358974359 0.900301098823547
1.20192307692308 0.927363097667694
1.25961538461538 0.951544106006622
1.32051282051282 0.913272857666016
1.38461538461538 0.937634110450745
1.44871794871795 0.94763058423996
1.51923076923077 0.907957077026367
1.58974358974359 0.86978942155838
1.66666666666667 0.91970694065094
1.74679487179487 0.911797165870667
1.83012820512821 0.890633583068848
1.91666666666667 0.834814250469208
2.00641025641026 0.917069256305695
2.1025641025641 0.786681532859802
2.20512820512821 0.870474815368652
2.30769230769231 0.82436865568161
2.41987179487179 0.85211580991745
2.53525641025641 0.859180271625519
2.65384615384615 0.833840370178223
2.78205128205128 0.834819436073303
2.91346153846154 0.825025022029877
3.05128205128205 0.791481733322144
3.19871794871795 0.772403836250305
3.34935897435897 0.74486231803894
3.50961538461538 0.75981992483139
3.67628205128205 0.708748042583466
3.8525641025641 0.7497239112854
4.03525641025641 0.794406831264496
4.2275641025641 0.697329759597778
4.42948717948718 0.7423135638237
4.64102564102564 0.708115935325623
4.86217948717949 0.6645787358284
5.09294871794872 0.651331543922424
5.33653846153846 0.644330263137817
5.58974358974359 0.65204781293869
5.85576923076923 0.690769135951996
6.13461538461539 0.641214609146118
6.42628205128205 0.625691056251526
6.73397435897436 0.64350837469101
7.05448717948718 0.547377705574036
7.38782051282051 0.57829350233078
7.74038461538461 0.589384257793427
8.10897435897436 0.620327174663544
8.49679487179487 0.55165559053421
8.90064102564103 0.605317234992981
9.32371794871795 0.561595678329468
9.76923076923077 0.545445740222931
10.2339743589744 0.519560754299164
10.7211538461538 0.531488597393036
11.2307692307692 0.574404835700989
11.7660256410256 0.544439494609833
12.3269230769231 0.552535355091095
12.9134615384615 0.464041203260422
13.5288461538462 0.536853134632111
14.1730769230769 0.463635921478271
14.849358974359 0.43095263838768
15.5544871794872 0.471528291702271
16.2948717948718 0.489900797605515
17.0705128205128 0.454616516828537
17.8846153846154 0.413832873106003
18.7371794871795 0.414517611265182
19.6282051282051 0.403109401464462
20.5641025641026 0.423843622207642
21.5416666666667 0.411006897687912
22.5673076923077 0.412874311208725
23.6442307692308 0.391848534345627
24.7692307692308 0.362223953008652
25.9487179487179 0.322522342205048
27.1826923076923 0.354929178953171
28.4775641025641 0.290924519300461
29.8333333333333 0.307508558034897
31.2564102564103 0.338540941476822
32.7435897435897 0.304098606109619
34.3044871794872 0.269175976514816
35.9358974358974 0.290908694267273
37.6474358974359 0.244592472910881
39.4391025641026 0.240803003311157
41.3173076923077 0.263549894094467
43.2852564102564 0.235744342207909
45.3461538461538 0.221570774912834
47.5064102564103 0.258336931467056
49.7692307692308 0.205432698130608
52.1378205128205 0.200462728738785
54.6217948717949 0.24289308488369
57.2211538461538 0.182994619011879
59.9455128205128 0.194484278559685
62.8012820512821 0.163613274693489
65.7916666666667 0.172761157155037
68.9230769230769 0.141867563128471
72.2051282051282 0.132742926478386
75.6442307692308 0.138381108641624
79.2467948717949 0.158747658133507
83.0192307692308 0.142110258340836
86.974358974359 0.119228236377239
91.1153846153846 0.137062087655067
95.4519230769231 0.122919373214245
100 0.141973420977592
};
\addplot [, color1, opacity=0.6, mark=square*, mark size=0.5, mark options={solid}, only marks, forget plot]
table {%
1 0.813920974731445
1.04487179487179 0.853579938411713
1.09615384615385 0.877396106719971
1.1474358974359 0.909783542156219
1.20192307692308 0.931008338928223
1.25961538461538 0.945797920227051
1.32051282051282 0.923625767230988
1.38461538461538 0.927043855190277
1.44871794871795 0.936709403991699
1.51923076923077 0.876821219921112
1.58974358974359 0.930028557777405
1.66666666666667 0.902787208557129
1.74679487179487 0.920415878295898
1.83012820512821 0.880231082439423
1.91666666666667 0.860368192195892
2.00641025641026 0.904169499874115
2.1025641025641 0.817207753658295
2.20512820512821 0.884664058685303
2.30769230769231 0.884621262550354
2.41987179487179 0.859479129314423
2.53525641025641 0.851247012615204
2.65384615384615 0.856026649475098
2.78205128205128 0.778792023658752
2.91346153846154 0.771400511264801
3.05128205128205 0.760350406169891
3.19871794871795 0.776646554470062
3.34935897435897 0.772492825984955
3.50961538461538 0.723271310329437
3.67628205128205 0.740569770336151
3.8525641025641 0.740213572978973
4.03525641025641 0.734779536724091
4.2275641025641 0.677603423595428
4.42948717948718 0.688316643238068
4.64102564102564 0.695041358470917
4.86217948717949 0.691567659378052
5.09294871794872 0.711704015731812
5.33653846153846 0.709236085414886
5.58974358974359 0.67598968744278
5.85576923076923 0.658975541591644
6.13461538461539 0.66428929567337
6.42628205128205 0.605591952800751
6.73397435897436 0.580184936523438
7.05448717948718 0.58957177400589
7.38782051282051 0.617193818092346
7.74038461538461 0.583066523075104
8.10897435897436 0.543685674667358
8.49679487179487 0.534812569618225
8.90064102564103 0.572847068309784
9.32371794871795 0.535838425159454
9.76923076923077 0.47879084944725
10.2339743589744 0.419375628232956
10.7211538461538 0.475016415119171
11.2307692307692 0.482770442962646
11.7660256410256 0.442036211490631
12.3269230769231 0.461029440164566
12.9134615384615 0.487050443887711
13.5288461538462 0.439131170511246
14.1730769230769 0.415038108825684
14.849358974359 0.381323337554932
15.5544871794872 0.371467232704163
16.2948717948718 0.412198930978775
17.0705128205128 0.373337090015411
17.8846153846154 0.356059223413467
18.7371794871795 0.350715607404709
19.6282051282051 0.390187859535217
20.5641025641026 0.403272300958633
21.5416666666667 0.354211896657944
22.5673076923077 0.334307998418808
23.6442307692308 0.315705060958862
24.7692307692308 0.327926367521286
25.9487179487179 0.318911999464035
27.1826923076923 0.318706005811691
28.4775641025641 0.284682840108871
29.8333333333333 0.309139788150787
31.2564102564103 0.272540628910065
32.7435897435897 0.309625297784805
34.3044871794872 0.278086453676224
35.9358974358974 0.244552060961723
37.6474358974359 0.239249810576439
39.4391025641026 0.252473384141922
41.3173076923077 0.198404505848885
43.2852564102564 0.191971883177757
45.3461538461538 0.244633108377457
47.5064102564103 0.256941765546799
49.7692307692308 0.198323920369148
52.1378205128205 0.180555522441864
54.6217948717949 0.216850519180298
57.2211538461538 0.175423145294189
59.9455128205128 0.159764602780342
62.8012820512821 0.155464872717857
65.7916666666667 0.166446432471275
68.9230769230769 0.142659857869148
72.2051282051282 0.158121258020401
75.6442307692308 0.149655967950821
79.2467948717949 0.135649248957634
83.0192307692308 0.122383452951908
86.974358974359 0.1365697234869
91.1153846153846 0.137557059526443
95.4519230769231 0.105647042393684
100 0.121991850435734
};
\addplot [, color1, opacity=0.6, mark=square*, mark size=0.5, mark options={solid}, only marks, forget plot]
table {%
1 0.814961731433868
1.04487179487179 0.861804664134979
1.09615384615385 0.903393268585205
1.1474358974359 0.902513027191162
1.20192307692308 0.931597650051117
1.25961538461538 0.939505755901337
1.32051282051282 0.915442645549774
1.38461538461538 0.927477836608887
1.44871794871795 0.934845566749573
1.51923076923077 0.909969985485077
1.58974358974359 0.908619225025177
1.66666666666667 0.916705548763275
1.74679487179487 0.927593231201172
1.83012820512821 0.89570564031601
1.91666666666667 0.884585678577423
2.00641025641026 0.879919052124023
2.1025641025641 0.828971683979034
2.20512820512821 0.904527008533478
2.30769230769231 0.892909169197083
2.41987179487179 0.886720657348633
2.53525641025641 0.855257928371429
2.65384615384615 0.79511547088623
2.78205128205128 0.796644926071167
2.91346153846154 0.846791684627533
3.05128205128205 0.766516208648682
3.19871794871795 0.777772605419159
3.34935897435897 0.777154088020325
3.50961538461538 0.689082384109497
3.67628205128205 0.724733591079712
3.8525641025641 0.772805392742157
4.03525641025641 0.699395358562469
4.2275641025641 0.693999946117401
4.42948717948718 0.681119680404663
4.64102564102564 0.593441426753998
4.86217948717949 0.650995373725891
5.09294871794872 0.613939881324768
5.33653846153846 0.624313652515411
5.58974358974359 0.607631504535675
5.85576923076923 0.569897353649139
6.13461538461539 0.580474555492401
6.42628205128205 0.586857736110687
6.73397435897436 0.55461311340332
7.05448717948718 0.559203565120697
7.38782051282051 0.603739559650421
7.74038461538461 0.548218727111816
8.10897435897436 0.579538643360138
8.49679487179487 0.56089448928833
8.90064102564103 0.548054337501526
9.32371794871795 0.518293976783752
9.76923076923077 0.516376674175262
10.2339743589744 0.535245835781097
10.7211538461538 0.472121924161911
11.2307692307692 0.469164818525314
11.7660256410256 0.494981676340103
12.3269230769231 0.474410593509674
12.9134615384615 0.47300997376442
13.5288461538462 0.461851984262466
14.1730769230769 0.43564772605896
14.849358974359 0.467223167419434
15.5544871794872 0.432653158903122
16.2948717948718 0.398156851530075
17.0705128205128 0.406810134649277
17.8846153846154 0.330373674631119
18.7371794871795 0.386685580015182
19.6282051282051 0.3945472240448
20.5641025641026 0.36268812417984
21.5416666666667 0.363277703523636
22.5673076923077 0.337811142206192
23.6442307692308 0.319479703903198
24.7692307692308 0.328717321157455
25.9487179487179 0.340767592191696
27.1826923076923 0.304143756628036
28.4775641025641 0.262426614761353
29.8333333333333 0.313884079456329
31.2564102564103 0.252789109945297
32.7435897435897 0.278071075677872
34.3044871794872 0.237848803400993
35.9358974358974 0.266281992197037
37.6474358974359 0.207716226577759
39.4391025641026 0.244884833693504
41.3173076923077 0.241774350404739
43.2852564102564 0.18808887898922
45.3461538461538 0.201867386698723
47.5064102564103 0.191822364926338
49.7692307692308 0.205799534916878
52.1378205128205 0.193779021501541
54.6217948717949 0.201593667268753
57.2211538461538 0.157081291079521
59.9455128205128 0.148374885320663
62.8012820512821 0.172693774104118
65.7916666666667 0.154709547758102
68.9230769230769 0.135667085647583
72.2051282051282 0.175849035382271
75.6442307692308 0.156277179718018
79.2467948717949 0.130936414003372
83.0192307692308 0.116735257208347
86.974358974359 0.164088845252991
91.1153846153846 0.117777347564697
95.4519230769231 0.162224695086479
100 0.127921149134636
};
\addplot [, color1, opacity=0.6, mark=square*, mark size=0.5, mark options={solid}, only marks, forget plot]
table {%
1 0.814086556434631
1.04487179487179 0.820041000843048
1.09615384615385 0.905407845973969
1.1474358974359 0.907537460327148
1.20192307692308 0.946788311004639
1.25961538461538 0.93783712387085
1.32051282051282 0.91689395904541
1.38461538461538 0.937966763973236
1.44871794871795 0.947003304958344
1.51923076923077 0.924761474132538
1.58974358974359 0.933577656745911
1.66666666666667 0.92704576253891
1.74679487179487 0.923090755939484
1.83012820512821 0.894181668758392
1.91666666666667 0.815649151802063
2.00641025641026 0.916873753070831
2.1025641025641 0.903815448284149
2.20512820512821 0.899733364582062
2.30769230769231 0.890616238117218
2.41987179487179 0.8893723487854
2.53525641025641 0.889072597026825
2.65384615384615 0.820434033870697
2.78205128205128 0.873179614543915
2.91346153846154 0.823922336101532
3.05128205128205 0.806040227413177
3.19871794871795 0.782665848731995
3.34935897435897 0.776334941387177
3.50961538461538 0.754034340381622
3.67628205128205 0.772092461585999
3.8525641025641 0.782798588275909
4.03525641025641 0.788489043712616
4.2275641025641 0.636767566204071
4.42948717948718 0.682346522808075
4.64102564102564 0.727462291717529
4.86217948717949 0.693143367767334
5.09294871794872 0.73149311542511
5.33653846153846 0.672623693943024
5.58974358974359 0.672357559204102
5.85576923076923 0.679366767406464
6.13461538461539 0.636976003646851
6.42628205128205 0.669324219226837
6.73397435897436 0.667176425457001
7.05448717948718 0.621859848499298
7.38782051282051 0.685414969921112
7.74038461538461 0.610198140144348
8.10897435897436 0.628699958324432
8.49679487179487 0.644458293914795
8.90064102564103 0.589770257472992
9.32371794871795 0.596592903137207
9.76923076923077 0.556481063365936
10.2339743589744 0.547998666763306
10.7211538461538 0.543423593044281
11.2307692307692 0.557931125164032
11.7660256410256 0.533280551433563
12.3269230769231 0.496736437082291
12.9134615384615 0.469264358282089
13.5288461538462 0.430152714252472
14.1730769230769 0.480765789747238
14.849358974359 0.476478397846222
15.5544871794872 0.475318640470505
16.2948717948718 0.382304817438126
17.0705128205128 0.461870104074478
17.8846153846154 0.417780965566635
18.7371794871795 0.379417896270752
19.6282051282051 0.425748020410538
20.5641025641026 0.342852592468262
21.5416666666667 0.405255615711212
22.5673076923077 0.352134764194489
23.6442307692308 0.319680124521255
24.7692307692308 0.315171450376511
25.9487179487179 0.290882915258408
27.1826923076923 0.268134653568268
28.4775641025641 0.330175787210464
29.8333333333333 0.325382441282272
31.2564102564103 0.309806555509567
32.7435897435897 0.255855947732925
34.3044871794872 0.257487446069717
35.9358974358974 0.243169099092484
37.6474358974359 0.240131288766861
39.4391025641026 0.243887767195702
41.3173076923077 0.239833876490593
43.2852564102564 0.228459239006042
45.3461538461538 0.213240057229996
47.5064102564103 0.215318068861961
49.7692307692308 0.207732677459717
52.1378205128205 0.235179737210274
54.6217948717949 0.179309815168381
57.2211538461538 0.165841445326805
59.9455128205128 0.189157351851463
62.8012820512821 0.169837236404419
65.7916666666667 0.152424365282059
68.9230769230769 0.160150915384293
72.2051282051282 0.151564314961433
75.6442307692308 0.175773069262505
79.2467948717949 0.117102935910225
83.0192307692308 0.151815339922905
86.974358974359 0.143327280879021
91.1153846153846 0.153160110116005
95.4519230769231 0.13850474357605
100 0.130394786596298
};
\addplot [, color2, opacity=0.6, mark=triangle*, mark size=0.5, mark options={solid,rotate=180}, only marks]
table {%
1 0.442987680435181
1.04487179487179 0.453514963388443
1.09615384615385 0.508857190608978
1.1474358974359 0.51631635427475
1.20192307692308 0.548053860664368
1.25961538461538 0.581723034381866
1.32051282051282 0.56914758682251
1.38461538461538 0.624051988124847
1.44871794871795 0.603230059146881
1.51923076923077 0.575822055339813
1.58974358974359 0.519580662250519
1.66666666666667 0.538453280925751
1.74679487179487 0.577455639839172
1.83012820512821 0.465112030506134
1.91666666666667 0.501900315284729
2.00641025641026 0.480522364377975
2.1025641025641 0.498138517141342
2.20512820512821 0.451758146286011
2.30769230769231 0.489146769046783
2.41987179487179 0.506605327129364
2.53525641025641 0.459589689970016
2.65384615384615 0.460887730121613
2.78205128205128 0.521577060222626
2.91346153846154 0.501647651195526
3.05128205128205 0.446675300598145
3.19871794871795 0.404471546411514
3.34935897435897 0.425048023462296
3.50961538461538 0.43471547961235
3.67628205128205 0.426186770200729
3.8525641025641 0.389112055301666
4.03525641025641 0.36141037940979
4.2275641025641 0.386157214641571
4.42948717948718 0.39362621307373
4.64102564102564 0.336796820163727
4.86217948717949 0.326784700155258
5.09294871794872 0.373699545860291
5.33653846153846 0.336952775716782
5.58974358974359 0.320444732904434
5.85576923076923 0.34437757730484
6.13461538461539 0.317411661148071
6.42628205128205 0.366957485675812
6.73397435897436 0.315403163433075
7.05448717948718 0.310861319303513
7.38782051282051 0.294764757156372
7.74038461538461 0.299458742141724
8.10897435897436 0.30516916513443
8.49679487179487 0.313449114561081
8.90064102564103 0.304549932479858
9.32371794871795 0.291255563497543
9.76923076923077 0.311374515295029
10.2339743589744 0.287690550088882
10.7211538461538 0.274033337831497
11.2307692307692 0.303581774234772
11.7660256410256 0.291517466306686
12.3269230769231 0.275168538093567
12.9134615384615 0.250420093536377
13.5288461538462 0.276546329259872
14.1730769230769 0.246168091893196
14.849358974359 0.265483766794205
15.5544871794872 0.259662568569183
16.2948717948718 0.26627379655838
17.0705128205128 0.270033121109009
17.8846153846154 0.240415379405022
18.7371794871795 0.241636261343956
19.6282051282051 0.242404818534851
20.5641025641026 0.206714436411858
21.5416666666667 0.195420429110527
22.5673076923077 0.221726089715958
23.6442307692308 0.205098509788513
24.7692307692308 0.225963115692139
25.9487179487179 0.215676531195641
27.1826923076923 0.201812982559204
28.4775641025641 0.23456946015358
29.8333333333333 0.248164817690849
31.2564102564103 0.195357233285904
32.7435897435897 0.213364630937576
34.3044871794872 0.205700397491455
35.9358974358974 0.203671649098396
37.6474358974359 0.197671562433243
39.4391025641026 0.174669116735458
41.3173076923077 0.174284800887108
43.2852564102564 0.155085369944572
45.3461538461538 0.183879807591438
47.5064102564103 0.195349678397179
49.7692307692308 0.162939965724945
52.1378205128205 0.168420717120171
54.6217948717949 0.178241923451424
57.2211538461538 0.195703938603401
59.9455128205128 0.149719923734665
62.8012820512821 0.155604839324951
65.7916666666667 0.159613460302353
68.9230769230769 0.144018188118935
72.2051282051282 0.1764245480299
75.6442307692308 0.157525792717934
79.2467948717949 0.163749203085899
83.0192307692308 0.155408576130867
86.974358974359 0.16846863925457
91.1153846153846 0.14734773337841
95.4519230769231 0.14391665160656
100 0.156635954976082
};
\addlegendentry{sub 16, mc 1}
\addplot [, color2, opacity=0.6, mark=triangle*, mark size=0.5, mark options={solid,rotate=180}, only marks, forget plot]
table {%
1 0.45664244890213
1.04487179487179 0.488359540700912
1.09615384615385 0.474117964506149
1.1474358974359 0.533434987068176
1.20192307692308 0.560124218463898
1.25961538461538 0.620501935482025
1.32051282051282 0.520655274391174
1.38461538461538 0.525057733058929
1.44871794871795 0.544446647167206
1.51923076923077 0.522458970546722
1.58974358974359 0.513312101364136
1.66666666666667 0.565965414047241
1.74679487179487 0.542752921581268
1.83012820512821 0.520931720733643
1.91666666666667 0.547880411148071
2.00641025641026 0.488321304321289
2.1025641025641 0.548984467983246
2.20512820512821 0.502498388290405
2.30769230769231 0.501622080802917
2.41987179487179 0.483981221914291
2.53525641025641 0.495440691709518
2.65384615384615 0.4845130443573
2.78205128205128 0.510759770870209
2.91346153846154 0.499755680561066
3.05128205128205 0.497873395681381
3.19871794871795 0.398508280515671
3.34935897435897 0.414653271436691
3.50961538461538 0.413383692502975
3.67628205128205 0.415128141641617
3.8525641025641 0.408534914255142
4.03525641025641 0.361922264099121
4.2275641025641 0.427779734134674
4.42948717948718 0.389261990785599
4.64102564102564 0.403603881597519
4.86217948717949 0.369113564491272
5.09294871794872 0.3641017973423
5.33653846153846 0.354526996612549
5.58974358974359 0.382056146860123
5.85576923076923 0.360566049814224
6.13461538461539 0.314578324556351
6.42628205128205 0.364496558904648
6.73397435897436 0.328855037689209
7.05448717948718 0.332928508520126
7.38782051282051 0.336692869663239
7.74038461538461 0.336801588535309
8.10897435897436 0.352983444929123
8.49679487179487 0.321911811828613
8.90064102564103 0.316495060920715
9.32371794871795 0.315565556287766
9.76923076923077 0.336298942565918
10.2339743589744 0.304185122251511
10.7211538461538 0.313432425260544
11.2307692307692 0.302625179290771
11.7660256410256 0.293559044599533
12.3269230769231 0.295323193073273
12.9134615384615 0.250570684671402
13.5288461538462 0.274021446704865
14.1730769230769 0.278924137353897
14.849358974359 0.285097599029541
15.5544871794872 0.270343154668808
16.2948717948718 0.283934772014618
17.0705128205128 0.281418293714523
17.8846153846154 0.237061783671379
18.7371794871795 0.253382921218872
19.6282051282051 0.252987951040268
20.5641025641026 0.248651131987572
21.5416666666667 0.2256178855896
22.5673076923077 0.246026083827019
23.6442307692308 0.227260306477547
24.7692307692308 0.220754146575928
25.9487179487179 0.216524228453636
27.1826923076923 0.225194498896599
28.4775641025641 0.218571096658707
29.8333333333333 0.241528496146202
31.2564102564103 0.243770122528076
32.7435897435897 0.227981239557266
34.3044871794872 0.236047431826591
35.9358974358974 0.214144572615623
37.6474358974359 0.219956785440445
39.4391025641026 0.176924183964729
41.3173076923077 0.203414246439934
43.2852564102564 0.193079113960266
45.3461538461538 0.206042915582657
47.5064102564103 0.171578079462051
49.7692307692308 0.17240546643734
52.1378205128205 0.197866842150688
54.6217948717949 0.185724526643753
57.2211538461538 0.186682730913162
59.9455128205128 0.179983541369438
62.8012820512821 0.167170464992523
65.7916666666667 0.15029376745224
68.9230769230769 0.154174849390984
72.2051282051282 0.147461250424385
75.6442307692308 0.151970133185387
79.2467948717949 0.160637184977531
83.0192307692308 0.15930163860321
86.974358974359 0.148638397455215
91.1153846153846 0.125940963625908
95.4519230769231 0.136837914586067
100 0.154181644320488
};
\addplot [, color2, opacity=0.6, mark=triangle*, mark size=0.5, mark options={solid,rotate=180}, only marks, forget plot]
table {%
1 0.463568300008774
1.04487179487179 0.44919815659523
1.09615384615385 0.496870100498199
1.1474358974359 0.496309250593185
1.20192307692308 0.559290587902069
1.25961538461538 0.601864516735077
1.32051282051282 0.527247428894043
1.38461538461538 0.55091518163681
1.44871794871795 0.581195533275604
1.51923076923077 0.517044365406036
1.58974358974359 0.506243526935577
1.66666666666667 0.537494778633118
1.74679487179487 0.528838157653809
1.83012820512821 0.552605807781219
1.91666666666667 0.517944037914276
2.00641025641026 0.531650841236115
2.1025641025641 0.503016173839569
2.20512820512821 0.538962006568909
2.30769230769231 0.482532739639282
2.41987179487179 0.529026806354523
2.53525641025641 0.549134075641632
2.65384615384615 0.551187694072723
2.78205128205128 0.492201328277588
2.91346153846154 0.458098709583282
3.05128205128205 0.438568264245987
3.19871794871795 0.394972234964371
3.34935897435897 0.381661385297775
3.50961538461538 0.442925453186035
3.67628205128205 0.491434097290039
3.8525641025641 0.360328018665314
4.03525641025641 0.396867126226425
4.2275641025641 0.336165577173233
4.42948717948718 0.404158651828766
4.64102564102564 0.372887700796127
4.86217948717949 0.354875952005386
5.09294871794872 0.389184504747391
5.33653846153846 0.314355581998825
5.58974358974359 0.334769457578659
5.85576923076923 0.397382169961929
6.13461538461539 0.330030888319016
6.42628205128205 0.362342089414597
6.73397435897436 0.32244861125946
7.05448717948718 0.288790076971054
7.38782051282051 0.286737591028214
7.74038461538461 0.314424753189087
8.10897435897436 0.328387588262558
8.49679487179487 0.265753000974655
8.90064102564103 0.266751646995544
9.32371794871795 0.29220649600029
9.76923076923077 0.276252955198288
10.2339743589744 0.271228075027466
10.7211538461538 0.259152263402939
11.2307692307692 0.279304176568985
11.7660256410256 0.27250599861145
12.3269230769231 0.258378833532333
12.9134615384615 0.218941077589989
13.5288461538462 0.243590518832207
14.1730769230769 0.248198136687279
14.849358974359 0.263823598623276
15.5544871794872 0.269834011793137
16.2948717948718 0.247258976101875
17.0705128205128 0.252733558416367
17.8846153846154 0.235700473189354
18.7371794871795 0.242205217480659
19.6282051282051 0.241946563124657
20.5641025641026 0.217965230345726
21.5416666666667 0.222919389605522
22.5673076923077 0.214803293347359
23.6442307692308 0.209262043237686
24.7692307692308 0.224295094609261
25.9487179487179 0.208931088447571
27.1826923076923 0.189168974757195
28.4775641025641 0.195302471518517
29.8333333333333 0.22994776070118
31.2564102564103 0.234125301241875
32.7435897435897 0.190059661865234
34.3044871794872 0.189065620303154
35.9358974358974 0.187740713357925
37.6474358974359 0.190043240785599
39.4391025641026 0.171962007880211
41.3173076923077 0.178120836615562
43.2852564102564 0.16480877995491
45.3461538461538 0.167153149843216
47.5064102564103 0.167318314313889
49.7692307692308 0.173778459429741
52.1378205128205 0.150978028774261
54.6217948717949 0.1663988083601
57.2211538461538 0.159672975540161
59.9455128205128 0.167956575751305
62.8012820512821 0.136348500847816
65.7916666666667 0.149875909090042
68.9230769230769 0.141034036874771
72.2051282051282 0.152741074562073
75.6442307692308 0.145118221640587
79.2467948717949 0.153723910450935
83.0192307692308 0.143347293138504
86.974358974359 0.131125003099442
91.1153846153846 0.128940671682358
95.4519230769231 0.123244978487492
100 0.136153683066368
};
\addplot [, color2, opacity=0.6, mark=triangle*, mark size=0.5, mark options={solid,rotate=180}, only marks, forget plot]
table {%
1 0.442890256643295
1.04487179487179 0.465807527303696
1.09615384615385 0.491499274969101
1.1474358974359 0.492543518543243
1.20192307692308 0.531104981899261
1.25961538461538 0.5337815284729
1.32051282051282 0.473546892404556
1.38461538461538 0.506591796875
1.44871794871795 0.505273580551147
1.51923076923077 0.494515031576157
1.58974358974359 0.486622780561447
1.66666666666667 0.481880396604538
1.74679487179487 0.51718008518219
1.83012820512821 0.474972784519196
1.91666666666667 0.544571101665497
2.00641025641026 0.450578898191452
2.1025641025641 0.497553914785385
2.20512820512821 0.521302163600922
2.30769230769231 0.45562544465065
2.41987179487179 0.450323671102524
2.53525641025641 0.486208409070969
2.65384615384615 0.412342995405197
2.78205128205128 0.374674588441849
2.91346153846154 0.360656172037125
3.05128205128205 0.412627846002579
3.19871794871795 0.382128715515137
3.34935897435897 0.407662868499756
3.50961538461538 0.398483484983444
3.67628205128205 0.374871581792831
3.8525641025641 0.38334047794342
4.03525641025641 0.364080220460892
4.2275641025641 0.389536947011948
4.42948717948718 0.335583060979843
4.64102564102564 0.374951839447021
4.86217948717949 0.332113832235336
5.09294871794872 0.362540572881699
5.33653846153846 0.310377717018127
5.58974358974359 0.306521087884903
5.85576923076923 0.342299312353134
6.13461538461539 0.325338542461395
6.42628205128205 0.29438716173172
6.73397435897436 0.298663437366486
7.05448717948718 0.32064101099968
7.38782051282051 0.278215140104294
7.74038461538461 0.30039244890213
8.10897435897436 0.276203364133835
8.49679487179487 0.282353401184082
8.90064102564103 0.291979849338531
9.32371794871795 0.265044778585434
9.76923076923077 0.297161608934402
10.2339743589744 0.306463867425919
10.7211538461538 0.26125904917717
11.2307692307692 0.279982060194016
11.7660256410256 0.23257015645504
12.3269230769231 0.257759869098663
12.9134615384615 0.273136675357819
13.5288461538462 0.30076465010643
14.1730769230769 0.25873601436615
14.849358974359 0.240730032324791
15.5544871794872 0.237923502922058
16.2948717948718 0.25894159078598
17.0705128205128 0.20793791115284
17.8846153846154 0.251441925764084
18.7371794871795 0.231198355555534
19.6282051282051 0.229582354426384
20.5641025641026 0.20737612247467
21.5416666666667 0.218416377902031
22.5673076923077 0.200692698359489
23.6442307692308 0.206666097044945
24.7692307692308 0.204730197787285
25.9487179487179 0.202501535415649
27.1826923076923 0.187782526016235
28.4775641025641 0.202057287096977
29.8333333333333 0.196097046136856
31.2564102564103 0.181255623698235
32.7435897435897 0.187923938035965
34.3044871794872 0.160223871469498
35.9358974358974 0.176091715693474
37.6474358974359 0.164523750543594
39.4391025641026 0.182492956519127
41.3173076923077 0.165994361042976
43.2852564102564 0.159150034189224
45.3461538461538 0.171510845422745
47.5064102564103 0.140902370214462
49.7692307692308 0.138725787401199
52.1378205128205 0.134748324751854
54.6217948717949 0.144759640097618
57.2211538461538 0.145405247807503
59.9455128205128 0.144622012972832
62.8012820512821 0.166542336344719
65.7916666666667 0.141166135668755
68.9230769230769 0.133803591132164
72.2051282051282 0.149031579494476
75.6442307692308 0.151126965880394
79.2467948717949 0.152791857719421
83.0192307692308 0.146502569317818
86.974358974359 0.133316144347191
91.1153846153846 0.133518055081367
95.4519230769231 0.0990371704101562
100 0.120995998382568
};
\addplot [, color2, opacity=0.6, mark=triangle*, mark size=0.5, mark options={solid,rotate=180}, only marks, forget plot]
table {%
1 0.424876511096954
1.04487179487179 0.418141752481461
1.09615384615385 0.500442504882812
1.1474358974359 0.528636634349823
1.20192307692308 0.523607909679413
1.25961538461538 0.550910234451294
1.32051282051282 0.545012235641479
1.38461538461538 0.546003341674805
1.44871794871795 0.570225059986115
1.51923076923077 0.578176140785217
1.58974358974359 0.630516290664673
1.66666666666667 0.528062164783478
1.74679487179487 0.571726977825165
1.83012820512821 0.511195123195648
1.91666666666667 0.512043595314026
2.00641025641026 0.574773490428925
2.1025641025641 0.582970261573792
2.20512820512821 0.52328634262085
2.30769230769231 0.514954626560211
2.41987179487179 0.530214905738831
2.53525641025641 0.481572359800339
2.65384615384615 0.48029550909996
2.78205128205128 0.506241321563721
2.91346153846154 0.464868366718292
3.05128205128205 0.416675060987473
3.19871794871795 0.454407036304474
3.34935897435897 0.41616615653038
3.50961538461538 0.41492372751236
3.67628205128205 0.409368246793747
3.8525641025641 0.433226585388184
4.03525641025641 0.380622059106827
4.2275641025641 0.404396146535873
4.42948717948718 0.4060919880867
4.64102564102564 0.379947513341904
4.86217948717949 0.349557608366013
5.09294871794872 0.38306587934494
5.33653846153846 0.3383928835392
5.58974358974359 0.383353680372238
5.85576923076923 0.369841545820236
6.13461538461539 0.339458912611008
6.42628205128205 0.312626928091049
6.73397435897436 0.343037724494934
7.05448717948718 0.313010424375534
7.38782051282051 0.332656651735306
7.74038461538461 0.341195791959763
8.10897435897436 0.309576332569122
8.49679487179487 0.273302286863327
8.90064102564103 0.312992870807648
9.32371794871795 0.283845007419586
9.76923076923077 0.295323699712753
10.2339743589744 0.283484101295471
10.7211538461538 0.299267441034317
11.2307692307692 0.293455898761749
11.7660256410256 0.276124209165573
12.3269230769231 0.300295472145081
12.9134615384615 0.263891845941544
13.5288461538462 0.275073677301407
14.1730769230769 0.263921111822128
14.849358974359 0.243724510073662
15.5544871794872 0.221858888864517
16.2948717948718 0.27247616648674
17.0705128205128 0.263088583946228
17.8846153846154 0.259578466415405
18.7371794871795 0.217894420027733
19.6282051282051 0.248406037688255
20.5641025641026 0.228343531489372
21.5416666666667 0.250413149595261
22.5673076923077 0.235506936907768
23.6442307692308 0.233039379119873
24.7692307692308 0.207214340567589
25.9487179487179 0.209854394197464
27.1826923076923 0.199583813548088
28.4775641025641 0.209045320749283
29.8333333333333 0.198848739266396
31.2564102564103 0.187825068831444
32.7435897435897 0.192919611930847
34.3044871794872 0.17939218878746
35.9358974358974 0.186725944280624
37.6474358974359 0.173520252108574
39.4391025641026 0.196665167808533
41.3173076923077 0.184170290827751
43.2852564102564 0.162272915244102
45.3461538461538 0.187145337462425
47.5064102564103 0.145138785243034
49.7692307692308 0.16075287759304
52.1378205128205 0.156466946005821
54.6217948717949 0.177348092198372
57.2211538461538 0.155200436711311
59.9455128205128 0.134817287325859
62.8012820512821 0.149735972285271
65.7916666666667 0.148916438221931
68.9230769230769 0.117412738502026
72.2051282051282 0.13433600962162
75.6442307692308 0.114476136863232
79.2467948717949 0.12873487174511
83.0192307692308 0.134172335267067
86.974358974359 0.151155397295952
91.1153846153846 0.137189537286758
95.4519230769231 0.126561284065247
100 0.129756525158882
};
\end{axis}

\end{tikzpicture}

  \tikzexternaldisable

  \caption{ \textbf{Approximations versus full-batch \ggn{}:} Overlap between the
    top-$C$ eigenspaces of the mini-batch \ggn{}, \vivit{}'s approximations and
    the full-batch \ggn during training of the \threecthreed network on \cifarten
    with \sgd{}. Each approximation is evaluated on $5$ mini-batches.
  } \label{vivit::fig:approx_eigenspace_vivit}
\end{figure}

To allow for a fine-grained cost-accuracy trade-off, \vivit introduces
% curvature sub-sampling and an MC approximation as
\textit{further} approximations to the mini-batch \ggn{} (see
\Cref{vivit::sec:approximations}). \Cref{vivit::fig:approx_eigenspace_vivit}
shows the overlap between these \ggn approximations and the full-batch
\ggn{}\sidenote{ A comparison with the mini-batch \ggn as ground truth can be
  found in \Cref{vivit::sec:eigenspace_noise} }. The order of the approximations
is as expected: with increasing computational effort, the approximations improve
and, despite the greatly reduced computational effort compared to the exact
mini-batch \ggn{}, significant structure of the top-$C$ eigenspace is preserved.
Details and results for the other test problems are reported in
\Cref{vivit::sec:eigenspace_noise}.

So far, our analysis is based on the top-$C$ eigenspace of the curvature
matrices. We extend it by studying the effect of noise and approximations on the
curvature \textit{magnitude} along the top-$C$ directions in
\Cref{vivit::sec:curvature_noise}.

%%% Local Variables:
%%% mode: latex
%%% TeX-master: "../thesis"
%%% End:


\subsection{Per-sample Directional Derivatives}\label{vivit::subsec:directional_derivatives}
\begin{figure}
  \centering
  % \textbf{\cifarten \threecthreed \sgd}\\[1mm]
  % defines the pgfplots style "gammaslambdasdefault"
\pgfkeys{/pgfplots/gammaslambdasdefault/.style={
    width=1.0\linewidth,
    height=0.6\linewidth,
    every axis plot/.append style={line width = 1.5pt},
    mark size = 0.6,
    tick pos = left,
    ylabel near ticks,
    xlabel near ticks,
    xtick align = inside,
    ytick align = inside,
    legend cell align = left,
    legend columns = 1,
    legend pos = south east,
    legend style = {
      fill opacity = 0.7,
      text opacity = 1,
      font = \footnotesize,
    },
    xticklabel style = {font = \footnotesize},
    xlabel style = {font = \footnotesize},
    axis line style = {black},
    yticklabel style = {font = \footnotesize},
    ylabel style = {font = \footnotesize},
    title style = {font = \footnotesize},
    grid = major,
    grid style = {dashed}
  }
}
%%% Local Variables:
%%% mode: latex
%%% TeX-master: "../../thesis"
%%% End:

  \pgfkeys{/pgfplots/zmystyle/.style={
      gammaslambdasdefault
    }}
  \tikzexternalenable
  % This file was created by tikzplotlib v0.9.7.
\begin{tikzpicture}

\begin{axis}[
axis line style={white!10!black},
log basis x={10},
tick pos=left,
xlabel={epoch (log scale)},
xmajorgrids,
xmin=0.794328234724281, xmax=125.892541179417,
xmode=log,
ylabel={SNR[\(\displaystyle \lambda_{nk}\)] (log scale)},
ymajorgrids,
ymin=0.00584285483153274, ymax=7.42366888382601,
ymode=log,
zmystyle
]
\addplot [
mark=*,
only marks,
scatter,
scatter/@post marker code/.code={%
  \endscope
},
scatter/@pre marker code/.code={%
  \expanded{%
  \noexpand\definecolor{thispointdrawcolor}{RGB}{\drawcolor}%
  \noexpand\definecolor{thispointfillcolor}{RGB}{\fillcolor}%
  }%
  \scope[draw=thispointdrawcolor, fill=thispointfillcolor]%
},
visualization depends on={value \thisrow{draw} \as \drawcolor},
visualization depends on={value \thisrow{fill} \as \fillcolor}
]
table{%
x  y  draw  fill
1 0.975784003734589 0,0,0 0,0,0
1 4.81617498397827 25.8039215686274,14.8235294117647,12.8470588235294 25.8039215686274,14.8235294117647,12.8470588235294
1 5.36448431015015 51.6078431372549,29.6470588235294,25.6941176470588 51.6078431372549,29.6470588235294,25.6941176470588
1 5.23165082931519 78.3333333333333,45,39 78.3333333333333,45,39
1 4.53217601776123 104.137254901961,59.8235294117647,51.8470588235294 104.137254901961,59.8235294117647,51.8470588235294
1 4.9503321647644 130.862745098039,75.1764705882353,65.1529411764706 130.862745098039,75.1764705882353,65.1529411764706
1 4.64961767196655 156.666666666667,90,78 156.666666666667,90,78
1 4.98736238479614 183.392156862745,105.352941176471,91.3058823529412 183.392156862745,105.352941176471,91.3058823529412
1 4.43404388427734 209.196078431373,120.176470588235,104.152941176471 209.196078431373,120.176470588235,104.152941176471
1 3.36111760139465 235,135,117 235,135,117
1.04487179487179 1.08064937591553 0,0,0 0,0,0
1.04487179487179 5.07322311401367 25.8039215686274,14.8235294117647,12.8470588235294 25.8039215686274,14.8235294117647,12.8470588235294
1.04487179487179 4.59841156005859 51.6078431372549,29.6470588235294,25.6941176470588 51.6078431372549,29.6470588235294,25.6941176470588
1.04487179487179 4.47616338729858 78.3333333333333,45,39 78.3333333333333,45,39
1.04487179487179 4.52594327926636 104.137254901961,59.8235294117647,51.8470588235294 104.137254901961,59.8235294117647,51.8470588235294
1.04487179487179 4.97389841079712 130.862745098039,75.1764705882353,65.1529411764706 130.862745098039,75.1764705882353,65.1529411764706
1.04487179487179 4.35929489135742 156.666666666667,90,78 156.666666666667,90,78
1.04487179487179 4.74812793731689 183.392156862745,105.352941176471,91.3058823529412 183.392156862745,105.352941176471,91.3058823529412
1.04487179487179 3.9832592010498 209.196078431373,120.176470588235,104.152941176471 209.196078431373,120.176470588235,104.152941176471
1.04487179487179 3.92481541633606 235,135,117 235,135,117
1.09615384615385 1.51742148399353 0,0,0 0,0,0
1.09615384615385 4.44957447052002 25.8039215686274,14.8235294117647,12.8470588235294 25.8039215686274,14.8235294117647,12.8470588235294
1.09615384615385 4.36930656433105 51.6078431372549,29.6470588235294,25.6941176470588 51.6078431372549,29.6470588235294,25.6941176470588
1.09615384615385 3.95009589195251 78.3333333333333,45,39 78.3333333333333,45,39
1.09615384615385 4.10540246963501 104.137254901961,59.8235294117647,51.8470588235294 104.137254901961,59.8235294117647,51.8470588235294
1.09615384615385 4.36474990844727 130.862745098039,75.1764705882353,65.1529411764706 130.862745098039,75.1764705882353,65.1529411764706
1.09615384615385 3.50593447685242 156.666666666667,90,78 156.666666666667,90,78
1.09615384615385 4.21931886672974 183.392156862745,105.352941176471,91.3058823529412 183.392156862745,105.352941176471,91.3058823529412
1.09615384615385 3.77479767799377 209.196078431373,120.176470588235,104.152941176471 209.196078431373,120.176470588235,104.152941176471
1.09615384615385 3.17522883415222 235,135,117 235,135,117
1.1474358974359 1.38034021854401 0,0,0 0,0,0
1.1474358974359 5.0747275352478 25.8039215686274,14.8235294117647,12.8470588235294 25.8039215686274,14.8235294117647,12.8470588235294
1.1474358974359 3.91628170013428 51.6078431372549,29.6470588235294,25.6941176470588 51.6078431372549,29.6470588235294,25.6941176470588
1.1474358974359 4.29094266891479 78.3333333333333,45,39 78.3333333333333,45,39
1.1474358974359 3.87449312210083 104.137254901961,59.8235294117647,51.8470588235294 104.137254901961,59.8235294117647,51.8470588235294
1.1474358974359 4.54162359237671 130.862745098039,75.1764705882353,65.1529411764706 130.862745098039,75.1764705882353,65.1529411764706
1.1474358974359 3.64791703224182 156.666666666667,90,78 156.666666666667,90,78
1.1474358974359 3.16870093345642 183.392156862745,105.352941176471,91.3058823529412 183.392156862745,105.352941176471,91.3058823529412
1.1474358974359 2.85287046432495 209.196078431373,120.176470588235,104.152941176471 209.196078431373,120.176470588235,104.152941176471
1.1474358974359 2.64012289047241 235,135,117 235,135,117
1.20192307692308 0.79089879989624 0,0,0 0,0,0
1.20192307692308 3.73917365074158 25.8039215686274,14.8235294117647,12.8470588235294 25.8039215686274,14.8235294117647,12.8470588235294
1.20192307692308 4.88604164123535 51.6078431372549,29.6470588235294,25.6941176470588 51.6078431372549,29.6470588235294,25.6941176470588
1.20192307692308 3.14525508880615 78.3333333333333,45,39 78.3333333333333,45,39
1.20192307692308 3.75772309303284 104.137254901961,59.8235294117647,51.8470588235294 104.137254901961,59.8235294117647,51.8470588235294
1.20192307692308 4.00032138824463 130.862745098039,75.1764705882353,65.1529411764706 130.862745098039,75.1764705882353,65.1529411764706
1.20192307692308 2.76700234413147 156.666666666667,90,78 156.666666666667,90,78
1.20192307692308 3.53422427177429 183.392156862745,105.352941176471,91.3058823529412 183.392156862745,105.352941176471,91.3058823529412
1.20192307692308 2.31572914123535 209.196078431373,120.176470588235,104.152941176471 209.196078431373,120.176470588235,104.152941176471
1.20192307692308 2.00478529930115 235,135,117 235,135,117
1.25961538461538 1.09610772132874 0,0,0 0,0,0
1.25961538461538 0.886177718639374 25.8039215686274,14.8235294117647,12.8470588235294 25.8039215686274,14.8235294117647,12.8470588235294
1.25961538461538 3.01286220550537 51.6078431372549,29.6470588235294,25.6941176470588 51.6078431372549,29.6470588235294,25.6941176470588
1.25961538461538 3.05179691314697 78.3333333333333,45,39 78.3333333333333,45,39
1.25961538461538 3.63458371162415 104.137254901961,59.8235294117647,51.8470588235294 104.137254901961,59.8235294117647,51.8470588235294
1.25961538461538 3.7750518321991 130.862745098039,75.1764705882353,65.1529411764706 130.862745098039,75.1764705882353,65.1529411764706
1.25961538461538 3.86977648735046 156.666666666667,90,78 156.666666666667,90,78
1.25961538461538 2.07635998725891 183.392156862745,105.352941176471,91.3058823529412 183.392156862745,105.352941176471,91.3058823529412
1.25961538461538 4.65148544311523 209.196078431373,120.176470588235,104.152941176471 209.196078431373,120.176470588235,104.152941176471
1.25961538461538 1.15362024307251 235,135,117 235,135,117
1.32051282051282 2.83147716522217 0,0,0 0,0,0
1.32051282051282 1.46668863296509 25.8039215686274,14.8235294117647,12.8470588235294 25.8039215686274,14.8235294117647,12.8470588235294
1.32051282051282 2.50360822677612 51.6078431372549,29.6470588235294,25.6941176470588 51.6078431372549,29.6470588235294,25.6941176470588
1.32051282051282 0.847918629646301 78.3333333333333,45,39 78.3333333333333,45,39
1.32051282051282 2.59461426734924 104.137254901961,59.8235294117647,51.8470588235294 104.137254901961,59.8235294117647,51.8470588235294
1.32051282051282 2.48218321800232 130.862745098039,75.1764705882353,65.1529411764706 130.862745098039,75.1764705882353,65.1529411764706
1.32051282051282 4.08651208877563 156.666666666667,90,78 156.666666666667,90,78
1.32051282051282 1.95713353157043 183.392156862745,105.352941176471,91.3058823529412 183.392156862745,105.352941176471,91.3058823529412
1.32051282051282 1.41261315345764 209.196078431373,120.176470588235,104.152941176471 209.196078431373,120.176470588235,104.152941176471
1.32051282051282 0.97663152217865 235,135,117 235,135,117
1.38461538461538 3.6496798992157 0,0,0 0,0,0
1.38461538461538 1.87759959697723 25.8039215686274,14.8235294117647,12.8470588235294 25.8039215686274,14.8235294117647,12.8470588235294
1.38461538461538 3.64344024658203 51.6078431372549,29.6470588235294,25.6941176470588 51.6078431372549,29.6470588235294,25.6941176470588
1.38461538461538 2.70816874504089 78.3333333333333,45,39 78.3333333333333,45,39
1.38461538461538 1.43051779270172 104.137254901961,59.8235294117647,51.8470588235294 104.137254901961,59.8235294117647,51.8470588235294
1.38461538461538 1.18621563911438 130.862745098039,75.1764705882353,65.1529411764706 130.862745098039,75.1764705882353,65.1529411764706
1.38461538461538 2.39292788505554 156.666666666667,90,78 156.666666666667,90,78
1.38461538461538 1.72273135185242 183.392156862745,105.352941176471,91.3058823529412 183.392156862745,105.352941176471,91.3058823529412
1.38461538461538 0.928144812583923 209.196078431373,120.176470588235,104.152941176471 209.196078431373,120.176470588235,104.152941176471
1.38461538461538 1.05567860603333 235,135,117 235,135,117
1.44871794871795 2.09320998191833 0,0,0 0,0,0
1.44871794871795 2.07442283630371 25.8039215686274,14.8235294117647,12.8470588235294 25.8039215686274,14.8235294117647,12.8470588235294
1.44871794871795 3.09923481941223 51.6078431372549,29.6470588235294,25.6941176470588 51.6078431372549,29.6470588235294,25.6941176470588
1.44871794871795 0.960194051265717 78.3333333333333,45,39 78.3333333333333,45,39
1.44871794871795 1.39489448070526 104.137254901961,59.8235294117647,51.8470588235294 104.137254901961,59.8235294117647,51.8470588235294
1.44871794871795 2.27095365524292 130.862745098039,75.1764705882353,65.1529411764706 130.862745098039,75.1764705882353,65.1529411764706
1.44871794871795 1.67746150493622 156.666666666667,90,78 156.666666666667,90,78
1.44871794871795 0.820045948028564 183.392156862745,105.352941176471,91.3058823529412 183.392156862745,105.352941176471,91.3058823529412
1.44871794871795 1.10364508628845 209.196078431373,120.176470588235,104.152941176471 209.196078431373,120.176470588235,104.152941176471
1.44871794871795 1.16712415218353 235,135,117 235,135,117
1.51923076923077 1.31854009628296 0,0,0 0,0,0
1.51923076923077 1.26899337768555 25.8039215686274,14.8235294117647,12.8470588235294 25.8039215686274,14.8235294117647,12.8470588235294
1.51923076923077 1.57649421691895 51.6078431372549,29.6470588235294,25.6941176470588 51.6078431372549,29.6470588235294,25.6941176470588
1.51923076923077 1.95576333999634 78.3333333333333,45,39 78.3333333333333,45,39
1.51923076923077 1.41067063808441 104.137254901961,59.8235294117647,51.8470588235294 104.137254901961,59.8235294117647,51.8470588235294
1.51923076923077 1.09714448451996 130.862745098039,75.1764705882353,65.1529411764706 130.862745098039,75.1764705882353,65.1529411764706
1.51923076923077 1.13901102542877 156.666666666667,90,78 156.666666666667,90,78
1.51923076923077 1.19942700862885 183.392156862745,105.352941176471,91.3058823529412 183.392156862745,105.352941176471,91.3058823529412
1.51923076923077 0.841314136981964 209.196078431373,120.176470588235,104.152941176471 209.196078431373,120.176470588235,104.152941176471
1.51923076923077 1.16712582111359 235,135,117 235,135,117
1.58974358974359 2.71763777732849 0,0,0 0,0,0
1.58974358974359 2.69588589668274 25.8039215686274,14.8235294117647,12.8470588235294 25.8039215686274,14.8235294117647,12.8470588235294
1.58974358974359 3.5698356628418 51.6078431372549,29.6470588235294,25.6941176470588 51.6078431372549,29.6470588235294,25.6941176470588
1.58974358974359 0.312659859657288 78.3333333333333,45,39 78.3333333333333,45,39
1.58974358974359 1.60501623153687 104.137254901961,59.8235294117647,51.8470588235294 104.137254901961,59.8235294117647,51.8470588235294
1.58974358974359 2.90069818496704 130.862745098039,75.1764705882353,65.1529411764706 130.862745098039,75.1764705882353,65.1529411764706
1.58974358974359 1.25137579441071 156.666666666667,90,78 156.666666666667,90,78
1.58974358974359 2.91528820991516 183.392156862745,105.352941176471,91.3058823529412 183.392156862745,105.352941176471,91.3058823529412
1.58974358974359 0.786626100540161 209.196078431373,120.176470588235,104.152941176471 209.196078431373,120.176470588235,104.152941176471
1.58974358974359 0.798025131225586 235,135,117 235,135,117
1.66666666666667 0.953728497028351 0,0,0 0,0,0
1.66666666666667 1.56447541713715 25.8039215686274,14.8235294117647,12.8470588235294 25.8039215686274,14.8235294117647,12.8470588235294
1.66666666666667 3.58846759796143 51.6078431372549,29.6470588235294,25.6941176470588 51.6078431372549,29.6470588235294,25.6941176470588
1.66666666666667 1.98851418495178 78.3333333333333,45,39 78.3333333333333,45,39
1.66666666666667 1.50565123558044 104.137254901961,59.8235294117647,51.8470588235294 104.137254901961,59.8235294117647,51.8470588235294
1.66666666666667 1.94345831871033 130.862745098039,75.1764705882353,65.1529411764706 130.862745098039,75.1764705882353,65.1529411764706
1.66666666666667 0.542786419391632 156.666666666667,90,78 156.666666666667,90,78
1.66666666666667 1.86082851886749 183.392156862745,105.352941176471,91.3058823529412 183.392156862745,105.352941176471,91.3058823529412
1.66666666666667 0.543868005275726 209.196078431373,120.176470588235,104.152941176471 209.196078431373,120.176470588235,104.152941176471
1.66666666666667 0.698533058166504 235,135,117 235,135,117
1.74679487179487 1.98289561271667 0,0,0 0,0,0
1.74679487179487 0.989320278167725 25.8039215686274,14.8235294117647,12.8470588235294 25.8039215686274,14.8235294117647,12.8470588235294
1.74679487179487 2.46191382408142 51.6078431372549,29.6470588235294,25.6941176470588 51.6078431372549,29.6470588235294,25.6941176470588
1.74679487179487 1.88124489784241 78.3333333333333,45,39 78.3333333333333,45,39
1.74679487179487 0.726153612136841 104.137254901961,59.8235294117647,51.8470588235294 104.137254901961,59.8235294117647,51.8470588235294
1.74679487179487 2.44277215003967 130.862745098039,75.1764705882353,65.1529411764706 130.862745098039,75.1764705882353,65.1529411764706
1.74679487179487 3.08665680885315 156.666666666667,90,78 156.666666666667,90,78
1.74679487179487 1.92148435115814 183.392156862745,105.352941176471,91.3058823529412 183.392156862745,105.352941176471,91.3058823529412
1.74679487179487 0.999748766422272 209.196078431373,120.176470588235,104.152941176471 209.196078431373,120.176470588235,104.152941176471
1.74679487179487 0.692695915699005 235,135,117 235,135,117
1.83012820512821 0.897441446781158 0,0,0 0,0,0
1.83012820512821 1.37237954139709 25.8039215686274,14.8235294117647,12.8470588235294 25.8039215686274,14.8235294117647,12.8470588235294
1.83012820512821 3.20425200462341 51.6078431372549,29.6470588235294,25.6941176470588 51.6078431372549,29.6470588235294,25.6941176470588
1.83012820512821 0.580470144748688 78.3333333333333,45,39 78.3333333333333,45,39
1.83012820512821 1.11900365352631 104.137254901961,59.8235294117647,51.8470588235294 104.137254901961,59.8235294117647,51.8470588235294
1.83012820512821 1.01710796356201 130.862745098039,75.1764705882353,65.1529411764706 130.862745098039,75.1764705882353,65.1529411764706
1.83012820512821 2.08246994018555 156.666666666667,90,78 156.666666666667,90,78
1.83012820512821 3.68994092941284 183.392156862745,105.352941176471,91.3058823529412 183.392156862745,105.352941176471,91.3058823529412
1.83012820512821 0.833235919475555 209.196078431373,120.176470588235,104.152941176471 209.196078431373,120.176470588235,104.152941176471
1.83012820512821 0.736810684204102 235,135,117 235,135,117
1.91666666666667 1.41917562484741 0,0,0 0,0,0
1.91666666666667 1.89283299446106 25.8039215686274,14.8235294117647,12.8470588235294 25.8039215686274,14.8235294117647,12.8470588235294
1.91666666666667 1.57728016376495 51.6078431372549,29.6470588235294,25.6941176470588 51.6078431372549,29.6470588235294,25.6941176470588
1.91666666666667 0.651821076869965 78.3333333333333,45,39 78.3333333333333,45,39
1.91666666666667 1.28349924087524 104.137254901961,59.8235294117647,51.8470588235294 104.137254901961,59.8235294117647,51.8470588235294
1.91666666666667 0.956565380096436 130.862745098039,75.1764705882353,65.1529411764706 130.862745098039,75.1764705882353,65.1529411764706
1.91666666666667 2.07766723632812 156.666666666667,90,78 156.666666666667,90,78
1.91666666666667 1.82708513736725 183.392156862745,105.352941176471,91.3058823529412 183.392156862745,105.352941176471,91.3058823529412
1.91666666666667 1.78830301761627 209.196078431373,120.176470588235,104.152941176471 209.196078431373,120.176470588235,104.152941176471
1.91666666666667 0.578944683074951 235,135,117 235,135,117
2.00641025641026 0.957026362419128 0,0,0 0,0,0
2.00641025641026 2.15670013427734 25.8039215686274,14.8235294117647,12.8470588235294 25.8039215686274,14.8235294117647,12.8470588235294
2.00641025641026 0.628524839878082 51.6078431372549,29.6470588235294,25.6941176470588 51.6078431372549,29.6470588235294,25.6941176470588
2.00641025641026 2.081547498703 78.3333333333333,45,39 78.3333333333333,45,39
2.00641025641026 0.610617756843567 104.137254901961,59.8235294117647,51.8470588235294 104.137254901961,59.8235294117647,51.8470588235294
2.00641025641026 1.47205865383148 130.862745098039,75.1764705882353,65.1529411764706 130.862745098039,75.1764705882353,65.1529411764706
2.00641025641026 0.878404855728149 156.666666666667,90,78 156.666666666667,90,78
2.00641025641026 1.67288327217102 183.392156862745,105.352941176471,91.3058823529412 183.392156862745,105.352941176471,91.3058823529412
2.00641025641026 2.25432372093201 209.196078431373,120.176470588235,104.152941176471 209.196078431373,120.176470588235,104.152941176471
2.00641025641026 0.772728621959686 235,135,117 235,135,117
2.1025641025641 1.1981326341629 0,0,0 0,0,0
2.1025641025641 2.98220658302307 25.8039215686274,14.8235294117647,12.8470588235294 25.8039215686274,14.8235294117647,12.8470588235294
2.1025641025641 1.94990372657776 51.6078431372549,29.6470588235294,25.6941176470588 51.6078431372549,29.6470588235294,25.6941176470588
2.1025641025641 1.24430131912231 78.3333333333333,45,39 78.3333333333333,45,39
2.1025641025641 1.46133375167847 104.137254901961,59.8235294117647,51.8470588235294 104.137254901961,59.8235294117647,51.8470588235294
2.1025641025641 0.391487717628479 130.862745098039,75.1764705882353,65.1529411764706 130.862745098039,75.1764705882353,65.1529411764706
2.1025641025641 0.965483009815216 156.666666666667,90,78 156.666666666667,90,78
2.1025641025641 2.42984747886658 183.392156862745,105.352941176471,91.3058823529412 183.392156862745,105.352941176471,91.3058823529412
2.1025641025641 0.750842213630676 209.196078431373,120.176470588235,104.152941176471 209.196078431373,120.176470588235,104.152941176471
2.1025641025641 0.703916549682617 235,135,117 235,135,117
2.20512820512821 0.683538317680359 0,0,0 0,0,0
2.20512820512821 2.2261278629303 25.8039215686274,14.8235294117647,12.8470588235294 25.8039215686274,14.8235294117647,12.8470588235294
2.20512820512821 2.85879993438721 51.6078431372549,29.6470588235294,25.6941176470588 51.6078431372549,29.6470588235294,25.6941176470588
2.20512820512821 1.31451082229614 78.3333333333333,45,39 78.3333333333333,45,39
2.20512820512821 1.17713391780853 104.137254901961,59.8235294117647,51.8470588235294 104.137254901961,59.8235294117647,51.8470588235294
2.20512820512821 1.40217018127441 130.862745098039,75.1764705882353,65.1529411764706 130.862745098039,75.1764705882353,65.1529411764706
2.20512820512821 1.22811830043793 156.666666666667,90,78 156.666666666667,90,78
2.20512820512821 1.82753384113312 183.392156862745,105.352941176471,91.3058823529412 183.392156862745,105.352941176471,91.3058823529412
2.20512820512821 1.91878819465637 209.196078431373,120.176470588235,104.152941176471 209.196078431373,120.176470588235,104.152941176471
2.20512820512821 0.67274284362793 235,135,117 235,135,117
2.30769230769231 1.15941154956818 0,0,0 0,0,0
2.30769230769231 2.67057538032532 25.8039215686274,14.8235294117647,12.8470588235294 25.8039215686274,14.8235294117647,12.8470588235294
2.30769230769231 0.724619925022125 51.6078431372549,29.6470588235294,25.6941176470588 51.6078431372549,29.6470588235294,25.6941176470588
2.30769230769231 1.0580872297287 78.3333333333333,45,39 78.3333333333333,45,39
2.30769230769231 2.75676608085632 104.137254901961,59.8235294117647,51.8470588235294 104.137254901961,59.8235294117647,51.8470588235294
2.30769230769231 0.47296866774559 130.862745098039,75.1764705882353,65.1529411764706 130.862745098039,75.1764705882353,65.1529411764706
2.30769230769231 0.951368987560272 156.666666666667,90,78 156.666666666667,90,78
2.30769230769231 1.4808474779129 183.392156862745,105.352941176471,91.3058823529412 183.392156862745,105.352941176471,91.3058823529412
2.30769230769231 1.40100240707397 209.196078431373,120.176470588235,104.152941176471 209.196078431373,120.176470588235,104.152941176471
2.30769230769231 0.663636744022369 235,135,117 235,135,117
2.41987179487179 1.3608090877533 0,0,0 0,0,0
2.41987179487179 2.1333909034729 25.8039215686274,14.8235294117647,12.8470588235294 25.8039215686274,14.8235294117647,12.8470588235294
2.41987179487179 1.6856632232666 51.6078431372549,29.6470588235294,25.6941176470588 51.6078431372549,29.6470588235294,25.6941176470588
2.41987179487179 0.920447647571564 78.3333333333333,45,39 78.3333333333333,45,39
2.41987179487179 0.536633908748627 104.137254901961,59.8235294117647,51.8470588235294 104.137254901961,59.8235294117647,51.8470588235294
2.41987179487179 2.04160213470459 130.862745098039,75.1764705882353,65.1529411764706 130.862745098039,75.1764705882353,65.1529411764706
2.41987179487179 2.03159046173096 156.666666666667,90,78 156.666666666667,90,78
2.41987179487179 0.8911172747612 183.392156862745,105.352941176471,91.3058823529412 183.392156862745,105.352941176471,91.3058823529412
2.41987179487179 1.31990396976471 209.196078431373,120.176470588235,104.152941176471 209.196078431373,120.176470588235,104.152941176471
2.41987179487179 0.978793442249298 235,135,117 235,135,117
2.53525641025641 1.28656411170959 0,0,0 0,0,0
2.53525641025641 1.01132917404175 25.8039215686274,14.8235294117647,12.8470588235294 25.8039215686274,14.8235294117647,12.8470588235294
2.53525641025641 2.91121864318848 51.6078431372549,29.6470588235294,25.6941176470588 51.6078431372549,29.6470588235294,25.6941176470588
2.53525641025641 0.35634583234787 78.3333333333333,45,39 78.3333333333333,45,39
2.53525641025641 0.450347036123276 104.137254901961,59.8235294117647,51.8470588235294 104.137254901961,59.8235294117647,51.8470588235294
2.53525641025641 2.38254070281982 130.862745098039,75.1764705882353,65.1529411764706 130.862745098039,75.1764705882353,65.1529411764706
2.53525641025641 2.22307109832764 156.666666666667,90,78 156.666666666667,90,78
2.53525641025641 1.90956676006317 183.392156862745,105.352941176471,91.3058823529412 183.392156862745,105.352941176471,91.3058823529412
2.53525641025641 1.79042518138885 209.196078431373,120.176470588235,104.152941176471 209.196078431373,120.176470588235,104.152941176471
2.53525641025641 0.575985729694366 235,135,117 235,135,117
2.65384615384615 1.58030438423157 0,0,0 0,0,0
2.65384615384615 1.0937123298645 25.8039215686274,14.8235294117647,12.8470588235294 25.8039215686274,14.8235294117647,12.8470588235294
2.65384615384615 0.996086657047272 51.6078431372549,29.6470588235294,25.6941176470588 51.6078431372549,29.6470588235294,25.6941176470588
2.65384615384615 0.894850373268127 78.3333333333333,45,39 78.3333333333333,45,39
2.65384615384615 0.708906650543213 104.137254901961,59.8235294117647,51.8470588235294 104.137254901961,59.8235294117647,51.8470588235294
2.65384615384615 1.46594929695129 130.862745098039,75.1764705882353,65.1529411764706 130.862745098039,75.1764705882353,65.1529411764706
2.65384615384615 1.06093084812164 156.666666666667,90,78 156.666666666667,90,78
2.65384615384615 0.926005721092224 183.392156862745,105.352941176471,91.3058823529412 183.392156862745,105.352941176471,91.3058823529412
2.65384615384615 0.842970669269562 209.196078431373,120.176470588235,104.152941176471 209.196078431373,120.176470588235,104.152941176471
2.65384615384615 1.36941730976105 235,135,117 235,135,117
2.78205128205128 0.968191921710968 0,0,0 0,0,0
2.78205128205128 1.24065697193146 25.8039215686274,14.8235294117647,12.8470588235294 25.8039215686274,14.8235294117647,12.8470588235294
2.78205128205128 0.683271527290344 51.6078431372549,29.6470588235294,25.6941176470588 51.6078431372549,29.6470588235294,25.6941176470588
2.78205128205128 0.888547897338867 78.3333333333333,45,39 78.3333333333333,45,39
2.78205128205128 0.995721518993378 104.137254901961,59.8235294117647,51.8470588235294 104.137254901961,59.8235294117647,51.8470588235294
2.78205128205128 1.40969228744507 130.862745098039,75.1764705882353,65.1529411764706 130.862745098039,75.1764705882353,65.1529411764706
2.78205128205128 1.74354720115662 156.666666666667,90,78 156.666666666667,90,78
2.78205128205128 0.96014541387558 183.392156862745,105.352941176471,91.3058823529412 183.392156862745,105.352941176471,91.3058823529412
2.78205128205128 1.19853329658508 209.196078431373,120.176470588235,104.152941176471 209.196078431373,120.176470588235,104.152941176471
2.78205128205128 0.697073101997375 235,135,117 235,135,117
2.91346153846154 0.609635770320892 0,0,0 0,0,0
2.91346153846154 2.07473850250244 25.8039215686274,14.8235294117647,12.8470588235294 25.8039215686274,14.8235294117647,12.8470588235294
2.91346153846154 0.549867868423462 51.6078431372549,29.6470588235294,25.6941176470588 51.6078431372549,29.6470588235294,25.6941176470588
2.91346153846154 0.735077679157257 78.3333333333333,45,39 78.3333333333333,45,39
2.91346153846154 0.353789001703262 104.137254901961,59.8235294117647,51.8470588235294 104.137254901961,59.8235294117647,51.8470588235294
2.91346153846154 0.823186814785004 130.862745098039,75.1764705882353,65.1529411764706 130.862745098039,75.1764705882353,65.1529411764706
2.91346153846154 1.76903975009918 156.666666666667,90,78 156.666666666667,90,78
2.91346153846154 0.420060843229294 183.392156862745,105.352941176471,91.3058823529412 183.392156862745,105.352941176471,91.3058823529412
2.91346153846154 0.551012754440308 209.196078431373,120.176470588235,104.152941176471 209.196078431373,120.176470588235,104.152941176471
2.91346153846154 1.01113784313202 235,135,117 235,135,117
3.05128205128205 0.959796130657196 0,0,0 0,0,0
3.05128205128205 1.06861424446106 25.8039215686274,14.8235294117647,12.8470588235294 25.8039215686274,14.8235294117647,12.8470588235294
3.05128205128205 0.611390054225922 51.6078431372549,29.6470588235294,25.6941176470588 51.6078431372549,29.6470588235294,25.6941176470588
3.05128205128205 0.372309535741806 78.3333333333333,45,39 78.3333333333333,45,39
3.05128205128205 1.54838001728058 104.137254901961,59.8235294117647,51.8470588235294 104.137254901961,59.8235294117647,51.8470588235294
3.05128205128205 1.45186567306519 130.862745098039,75.1764705882353,65.1529411764706 130.862745098039,75.1764705882353,65.1529411764706
3.05128205128205 2.38362693786621 156.666666666667,90,78 156.666666666667,90,78
3.05128205128205 1.32113754749298 183.392156862745,105.352941176471,91.3058823529412 183.392156862745,105.352941176471,91.3058823529412
3.05128205128205 1.02174687385559 209.196078431373,120.176470588235,104.152941176471 209.196078431373,120.176470588235,104.152941176471
3.05128205128205 1.65574610233307 235,135,117 235,135,117
3.19871794871795 1.64644658565521 0,0,0 0,0,0
3.19871794871795 0.949063956737518 25.8039215686274,14.8235294117647,12.8470588235294 25.8039215686274,14.8235294117647,12.8470588235294
3.19871794871795 0.546768128871918 51.6078431372549,29.6470588235294,25.6941176470588 51.6078431372549,29.6470588235294,25.6941176470588
3.19871794871795 0.2589011490345 78.3333333333333,45,39 78.3333333333333,45,39
3.19871794871795 1.13936948776245 104.137254901961,59.8235294117647,51.8470588235294 104.137254901961,59.8235294117647,51.8470588235294
3.19871794871795 1.49578666687012 130.862745098039,75.1764705882353,65.1529411764706 130.862745098039,75.1764705882353,65.1529411764706
3.19871794871795 1.57845735549927 156.666666666667,90,78 156.666666666667,90,78
3.19871794871795 0.507057189941406 183.392156862745,105.352941176471,91.3058823529412 183.392156862745,105.352941176471,91.3058823529412
3.19871794871795 0.773656845092773 209.196078431373,120.176470588235,104.152941176471 209.196078431373,120.176470588235,104.152941176471
3.19871794871795 1.32569575309753 235,135,117 235,135,117
3.34935897435897 1.56769549846649 0,0,0 0,0,0
3.34935897435897 1.77553796768188 25.8039215686274,14.8235294117647,12.8470588235294 25.8039215686274,14.8235294117647,12.8470588235294
3.34935897435897 0.920381009578705 51.6078431372549,29.6470588235294,25.6941176470588 51.6078431372549,29.6470588235294,25.6941176470588
3.34935897435897 0.787599086761475 78.3333333333333,45,39 78.3333333333333,45,39
3.34935897435897 0.988805770874023 104.137254901961,59.8235294117647,51.8470588235294 104.137254901961,59.8235294117647,51.8470588235294
3.34935897435897 1.93014740943909 130.862745098039,75.1764705882353,65.1529411764706 130.862745098039,75.1764705882353,65.1529411764706
3.34935897435897 1.90421116352081 156.666666666667,90,78 156.666666666667,90,78
3.34935897435897 1.00538742542267 183.392156862745,105.352941176471,91.3058823529412 183.392156862745,105.352941176471,91.3058823529412
3.34935897435897 0.616127073764801 209.196078431373,120.176470588235,104.152941176471 209.196078431373,120.176470588235,104.152941176471
3.34935897435897 1.32753646373749 235,135,117 235,135,117
3.50961538461538 1.23502969741821 0,0,0 0,0,0
3.50961538461538 0.898199558258057 25.8039215686274,14.8235294117647,12.8470588235294 25.8039215686274,14.8235294117647,12.8470588235294
3.50961538461538 1.02644622325897 51.6078431372549,29.6470588235294,25.6941176470588 51.6078431372549,29.6470588235294,25.6941176470588
3.50961538461538 0.968848884105682 78.3333333333333,45,39 78.3333333333333,45,39
3.50961538461538 0.787219524383545 104.137254901961,59.8235294117647,51.8470588235294 104.137254901961,59.8235294117647,51.8470588235294
3.50961538461538 1.54163193702698 130.862745098039,75.1764705882353,65.1529411764706 130.862745098039,75.1764705882353,65.1529411764706
3.50961538461538 1.70604562759399 156.666666666667,90,78 156.666666666667,90,78
3.50961538461538 1.16676878929138 183.392156862745,105.352941176471,91.3058823529412 183.392156862745,105.352941176471,91.3058823529412
3.50961538461538 0.532797038555145 209.196078431373,120.176470588235,104.152941176471 209.196078431373,120.176470588235,104.152941176471
3.50961538461538 1.37712466716766 235,135,117 235,135,117
3.67628205128205 0.768164575099945 0,0,0 0,0,0
3.67628205128205 0.897670865058899 25.8039215686274,14.8235294117647,12.8470588235294 25.8039215686274,14.8235294117647,12.8470588235294
3.67628205128205 0.48084232211113 51.6078431372549,29.6470588235294,25.6941176470588 51.6078431372549,29.6470588235294,25.6941176470588
3.67628205128205 0.609206855297089 78.3333333333333,45,39 78.3333333333333,45,39
3.67628205128205 0.803423702716827 104.137254901961,59.8235294117647,51.8470588235294 104.137254901961,59.8235294117647,51.8470588235294
3.67628205128205 0.514328300952911 130.862745098039,75.1764705882353,65.1529411764706 130.862745098039,75.1764705882353,65.1529411764706
3.67628205128205 1.95982778072357 156.666666666667,90,78 156.666666666667,90,78
3.67628205128205 0.98697167634964 183.392156862745,105.352941176471,91.3058823529412 183.392156862745,105.352941176471,91.3058823529412
3.67628205128205 0.624035656452179 209.196078431373,120.176470588235,104.152941176471 209.196078431373,120.176470588235,104.152941176471
3.67628205128205 0.967042088508606 235,135,117 235,135,117
3.8525641025641 0.728654861450195 0,0,0 0,0,0
3.8525641025641 0.444164723157883 25.8039215686274,14.8235294117647,12.8470588235294 25.8039215686274,14.8235294117647,12.8470588235294
3.8525641025641 0.673896551132202 51.6078431372549,29.6470588235294,25.6941176470588 51.6078431372549,29.6470588235294,25.6941176470588
3.8525641025641 1.50595903396606 78.3333333333333,45,39 78.3333333333333,45,39
3.8525641025641 0.87969446182251 104.137254901961,59.8235294117647,51.8470588235294 104.137254901961,59.8235294117647,51.8470588235294
3.8525641025641 1.36471116542816 130.862745098039,75.1764705882353,65.1529411764706 130.862745098039,75.1764705882353,65.1529411764706
3.8525641025641 0.91891485452652 156.666666666667,90,78 156.666666666667,90,78
3.8525641025641 0.979045569896698 183.392156862745,105.352941176471,91.3058823529412 183.392156862745,105.352941176471,91.3058823529412
3.8525641025641 0.600068092346191 209.196078431373,120.176470588235,104.152941176471 209.196078431373,120.176470588235,104.152941176471
3.8525641025641 1.05551815032959 235,135,117 235,135,117
4.03525641025641 0.900466620922089 0,0,0 0,0,0
4.03525641025641 0.555265784263611 25.8039215686274,14.8235294117647,12.8470588235294 25.8039215686274,14.8235294117647,12.8470588235294
4.03525641025641 0.213296785950661 51.6078431372549,29.6470588235294,25.6941176470588 51.6078431372549,29.6470588235294,25.6941176470588
4.03525641025641 1.36506283283234 78.3333333333333,45,39 78.3333333333333,45,39
4.03525641025641 1.6247261762619 104.137254901961,59.8235294117647,51.8470588235294 104.137254901961,59.8235294117647,51.8470588235294
4.03525641025641 0.582535028457642 130.862745098039,75.1764705882353,65.1529411764706 130.862745098039,75.1764705882353,65.1529411764706
4.03525641025641 1.27524209022522 156.666666666667,90,78 156.666666666667,90,78
4.03525641025641 0.937963902950287 183.392156862745,105.352941176471,91.3058823529412 183.392156862745,105.352941176471,91.3058823529412
4.03525641025641 0.439503490924835 209.196078431373,120.176470588235,104.152941176471 209.196078431373,120.176470588235,104.152941176471
4.03525641025641 0.818723857402802 235,135,117 235,135,117
4.2275641025641 1.22950494289398 0,0,0 0,0,0
4.2275641025641 0.668096899986267 25.8039215686274,14.8235294117647,12.8470588235294 25.8039215686274,14.8235294117647,12.8470588235294
4.2275641025641 0.712359309196472 51.6078431372549,29.6470588235294,25.6941176470588 51.6078431372549,29.6470588235294,25.6941176470588
4.2275641025641 0.610431015491486 78.3333333333333,45,39 78.3333333333333,45,39
4.2275641025641 1.09898900985718 104.137254901961,59.8235294117647,51.8470588235294 104.137254901961,59.8235294117647,51.8470588235294
4.2275641025641 1.02358484268188 130.862745098039,75.1764705882353,65.1529411764706 130.862745098039,75.1764705882353,65.1529411764706
4.2275641025641 1.1459047794342 156.666666666667,90,78 156.666666666667,90,78
4.2275641025641 0.966608166694641 183.392156862745,105.352941176471,91.3058823529412 183.392156862745,105.352941176471,91.3058823529412
4.2275641025641 0.685954034328461 209.196078431373,120.176470588235,104.152941176471 209.196078431373,120.176470588235,104.152941176471
4.2275641025641 0.971087157726288 235,135,117 235,135,117
4.42948717948718 0.536157488822937 0,0,0 0,0,0
4.42948717948718 0.31957283616066 25.8039215686274,14.8235294117647,12.8470588235294 25.8039215686274,14.8235294117647,12.8470588235294
4.42948717948718 0.254982441663742 51.6078431372549,29.6470588235294,25.6941176470588 51.6078431372549,29.6470588235294,25.6941176470588
4.42948717948718 1.23078382015228 78.3333333333333,45,39 78.3333333333333,45,39
4.42948717948718 1.27418661117554 104.137254901961,59.8235294117647,51.8470588235294 104.137254901961,59.8235294117647,51.8470588235294
4.42948717948718 1.09501171112061 130.862745098039,75.1764705882353,65.1529411764706 130.862745098039,75.1764705882353,65.1529411764706
4.42948717948718 1.42944669723511 156.666666666667,90,78 156.666666666667,90,78
4.42948717948718 0.567227423191071 183.392156862745,105.352941176471,91.3058823529412 183.392156862745,105.352941176471,91.3058823529412
4.42948717948718 0.512123703956604 209.196078431373,120.176470588235,104.152941176471 209.196078431373,120.176470588235,104.152941176471
4.42948717948718 1.03413116931915 235,135,117 235,135,117
4.64102564102564 1.1321474313736 0,0,0 0,0,0
4.64102564102564 1.05999195575714 25.8039215686274,14.8235294117647,12.8470588235294 25.8039215686274,14.8235294117647,12.8470588235294
4.64102564102564 0.784232795238495 51.6078431372549,29.6470588235294,25.6941176470588 51.6078431372549,29.6470588235294,25.6941176470588
4.64102564102564 0.38313490152359 78.3333333333333,45,39 78.3333333333333,45,39
4.64102564102564 0.899316608905792 104.137254901961,59.8235294117647,51.8470588235294 104.137254901961,59.8235294117647,51.8470588235294
4.64102564102564 1.51124787330627 130.862745098039,75.1764705882353,65.1529411764706 130.862745098039,75.1764705882353,65.1529411764706
4.64102564102564 0.830731153488159 156.666666666667,90,78 156.666666666667,90,78
4.64102564102564 0.669659793376923 183.392156862745,105.352941176471,91.3058823529412 183.392156862745,105.352941176471,91.3058823529412
4.64102564102564 0.601336717605591 209.196078431373,120.176470588235,104.152941176471 209.196078431373,120.176470588235,104.152941176471
4.64102564102564 1.05722498893738 235,135,117 235,135,117
4.86217948717949 1.46660482883453 0,0,0 0,0,0
4.86217948717949 0.965067565441132 25.8039215686274,14.8235294117647,12.8470588235294 25.8039215686274,14.8235294117647,12.8470588235294
4.86217948717949 0.496772199869156 51.6078431372549,29.6470588235294,25.6941176470588 51.6078431372549,29.6470588235294,25.6941176470588
4.86217948717949 0.609667479991913 78.3333333333333,45,39 78.3333333333333,45,39
4.86217948717949 0.781067073345184 104.137254901961,59.8235294117647,51.8470588235294 104.137254901961,59.8235294117647,51.8470588235294
4.86217948717949 0.746175348758698 130.862745098039,75.1764705882353,65.1529411764706 130.862745098039,75.1764705882353,65.1529411764706
4.86217948717949 1.62621545791626 156.666666666667,90,78 156.666666666667,90,78
4.86217948717949 1.02460932731628 183.392156862745,105.352941176471,91.3058823529412 183.392156862745,105.352941176471,91.3058823529412
4.86217948717949 0.761037766933441 209.196078431373,120.176470588235,104.152941176471 209.196078431373,120.176470588235,104.152941176471
4.86217948717949 1.24410593509674 235,135,117 235,135,117
5.09294871794872 0.512161672115326 0,0,0 0,0,0
5.09294871794872 0.432282269001007 25.8039215686274,14.8235294117647,12.8470588235294 25.8039215686274,14.8235294117647,12.8470588235294
5.09294871794872 0.256528824567795 51.6078431372549,29.6470588235294,25.6941176470588 51.6078431372549,29.6470588235294,25.6941176470588
5.09294871794872 1.31927740573883 78.3333333333333,45,39 78.3333333333333,45,39
5.09294871794872 0.747238874435425 104.137254901961,59.8235294117647,51.8470588235294 104.137254901961,59.8235294117647,51.8470588235294
5.09294871794872 0.907392740249634 130.862745098039,75.1764705882353,65.1529411764706 130.862745098039,75.1764705882353,65.1529411764706
5.09294871794872 0.280877202749252 156.666666666667,90,78 156.666666666667,90,78
5.09294871794872 0.527149200439453 183.392156862745,105.352941176471,91.3058823529412 183.392156862745,105.352941176471,91.3058823529412
5.09294871794872 0.795610666275024 209.196078431373,120.176470588235,104.152941176471 209.196078431373,120.176470588235,104.152941176471
5.09294871794872 0.886933445930481 235,135,117 235,135,117
5.33653846153846 0.881524264812469 0,0,0 0,0,0
5.33653846153846 0.379370272159576 25.8039215686274,14.8235294117647,12.8470588235294 25.8039215686274,14.8235294117647,12.8470588235294
5.33653846153846 0.630515813827515 51.6078431372549,29.6470588235294,25.6941176470588 51.6078431372549,29.6470588235294,25.6941176470588
5.33653846153846 0.634531259536743 78.3333333333333,45,39 78.3333333333333,45,39
5.33653846153846 0.993321239948273 104.137254901961,59.8235294117647,51.8470588235294 104.137254901961,59.8235294117647,51.8470588235294
5.33653846153846 0.842267215251923 130.862745098039,75.1764705882353,65.1529411764706 130.862745098039,75.1764705882353,65.1529411764706
5.33653846153846 0.535592973232269 156.666666666667,90,78 156.666666666667,90,78
5.33653846153846 1.16408264636993 183.392156862745,105.352941176471,91.3058823529412 183.392156862745,105.352941176471,91.3058823529412
5.33653846153846 0.500023365020752 209.196078431373,120.176470588235,104.152941176471 209.196078431373,120.176470588235,104.152941176471
5.33653846153846 0.849346876144409 235,135,117 235,135,117
5.58974358974359 0.981884658336639 0,0,0 0,0,0
5.58974358974359 1.17341864109039 25.8039215686274,14.8235294117647,12.8470588235294 25.8039215686274,14.8235294117647,12.8470588235294
5.58974358974359 0.396986305713654 51.6078431372549,29.6470588235294,25.6941176470588 51.6078431372549,29.6470588235294,25.6941176470588
5.58974358974359 0.969923973083496 78.3333333333333,45,39 78.3333333333333,45,39
5.58974358974359 0.352045565843582 104.137254901961,59.8235294117647,51.8470588235294 104.137254901961,59.8235294117647,51.8470588235294
5.58974358974359 0.969979047775269 130.862745098039,75.1764705882353,65.1529411764706 130.862745098039,75.1764705882353,65.1529411764706
5.58974358974359 0.910520195960999 156.666666666667,90,78 156.666666666667,90,78
5.58974358974359 0.739221334457397 183.392156862745,105.352941176471,91.3058823529412 183.392156862745,105.352941176471,91.3058823529412
5.58974358974359 0.844757556915283 209.196078431373,120.176470588235,104.152941176471 209.196078431373,120.176470588235,104.152941176471
5.58974358974359 1.07917749881744 235,135,117 235,135,117
5.85576923076923 0.63684070110321 0,0,0 0,0,0
5.85576923076923 0.479411870241165 25.8039215686274,14.8235294117647,12.8470588235294 25.8039215686274,14.8235294117647,12.8470588235294
5.85576923076923 0.426124453544617 51.6078431372549,29.6470588235294,25.6941176470588 51.6078431372549,29.6470588235294,25.6941176470588
5.85576923076923 0.686657845973969 78.3333333333333,45,39 78.3333333333333,45,39
5.85576923076923 0.382183194160461 104.137254901961,59.8235294117647,51.8470588235294 104.137254901961,59.8235294117647,51.8470588235294
5.85576923076923 0.554038345813751 130.862745098039,75.1764705882353,65.1529411764706 130.862745098039,75.1764705882353,65.1529411764706
5.85576923076923 0.48877289891243 156.666666666667,90,78 156.666666666667,90,78
5.85576923076923 0.823561072349548 183.392156862745,105.352941176471,91.3058823529412 183.392156862745,105.352941176471,91.3058823529412
5.85576923076923 0.235067293047905 209.196078431373,120.176470588235,104.152941176471 209.196078431373,120.176470588235,104.152941176471
5.85576923076923 0.673173248767853 235,135,117 235,135,117
6.13461538461539 1.32324814796448 0,0,0 0,0,0
6.13461538461539 0.596458256244659 25.8039215686274,14.8235294117647,12.8470588235294 25.8039215686274,14.8235294117647,12.8470588235294
6.13461538461539 0.965082883834839 51.6078431372549,29.6470588235294,25.6941176470588 51.6078431372549,29.6470588235294,25.6941176470588
6.13461538461539 1.11405169963837 78.3333333333333,45,39 78.3333333333333,45,39
6.13461538461539 0.572742342948914 104.137254901961,59.8235294117647,51.8470588235294 104.137254901961,59.8235294117647,51.8470588235294
6.13461538461539 1.01978623867035 130.862745098039,75.1764705882353,65.1529411764706 130.862745098039,75.1764705882353,65.1529411764706
6.13461538461539 0.615030705928802 156.666666666667,90,78 156.666666666667,90,78
6.13461538461539 0.765961289405823 183.392156862745,105.352941176471,91.3058823529412 183.392156862745,105.352941176471,91.3058823529412
6.13461538461539 0.431054890155792 209.196078431373,120.176470588235,104.152941176471 209.196078431373,120.176470588235,104.152941176471
6.13461538461539 0.892988204956055 235,135,117 235,135,117
6.42628205128205 0.716674864292145 0,0,0 0,0,0
6.42628205128205 0.913927018642426 25.8039215686274,14.8235294117647,12.8470588235294 25.8039215686274,14.8235294117647,12.8470588235294
6.42628205128205 0.85632336139679 51.6078431372549,29.6470588235294,25.6941176470588 51.6078431372549,29.6470588235294,25.6941176470588
6.42628205128205 0.448912739753723 78.3333333333333,45,39 78.3333333333333,45,39
6.42628205128205 0.517470598220825 104.137254901961,59.8235294117647,51.8470588235294 104.137254901961,59.8235294117647,51.8470588235294
6.42628205128205 0.915785610675812 130.862745098039,75.1764705882353,65.1529411764706 130.862745098039,75.1764705882353,65.1529411764706
6.42628205128205 0.453579485416412 156.666666666667,90,78 156.666666666667,90,78
6.42628205128205 0.949284672737122 183.392156862745,105.352941176471,91.3058823529412 183.392156862745,105.352941176471,91.3058823529412
6.42628205128205 0.721194982528687 209.196078431373,120.176470588235,104.152941176471 209.196078431373,120.176470588235,104.152941176471
6.42628205128205 0.445933938026428 235,135,117 235,135,117
6.73397435897436 0.336776793003082 0,0,0 0,0,0
6.73397435897436 0.663623213768005 25.8039215686274,14.8235294117647,12.8470588235294 25.8039215686274,14.8235294117647,12.8470588235294
6.73397435897436 0.522590100765228 51.6078431372549,29.6470588235294,25.6941176470588 51.6078431372549,29.6470588235294,25.6941176470588
6.73397435897436 0.361402273178101 78.3333333333333,45,39 78.3333333333333,45,39
6.73397435897436 0.483754307031631 104.137254901961,59.8235294117647,51.8470588235294 104.137254901961,59.8235294117647,51.8470588235294
6.73397435897436 1.03270316123962 130.862745098039,75.1764705882353,65.1529411764706 130.862745098039,75.1764705882353,65.1529411764706
6.73397435897436 0.411600589752197 156.666666666667,90,78 156.666666666667,90,78
6.73397435897436 0.592099785804749 183.392156862745,105.352941176471,91.3058823529412 183.392156862745,105.352941176471,91.3058823529412
6.73397435897436 0.767435133457184 209.196078431373,120.176470588235,104.152941176471 209.196078431373,120.176470588235,104.152941176471
6.73397435897436 0.860516786575317 235,135,117 235,135,117
7.05448717948718 0.589427530765533 0,0,0 0,0,0
7.05448717948718 0.733646333217621 25.8039215686274,14.8235294117647,12.8470588235294 25.8039215686274,14.8235294117647,12.8470588235294
7.05448717948718 0.365888655185699 51.6078431372549,29.6470588235294,25.6941176470588 51.6078431372549,29.6470588235294,25.6941176470588
7.05448717948718 0.565677523612976 78.3333333333333,45,39 78.3333333333333,45,39
7.05448717948718 0.849868595600128 104.137254901961,59.8235294117647,51.8470588235294 104.137254901961,59.8235294117647,51.8470588235294
7.05448717948718 0.390950590372086 130.862745098039,75.1764705882353,65.1529411764706 130.862745098039,75.1764705882353,65.1529411764706
7.05448717948718 0.565177261829376 156.666666666667,90,78 156.666666666667,90,78
7.05448717948718 0.888440728187561 183.392156862745,105.352941176471,91.3058823529412 183.392156862745,105.352941176471,91.3058823529412
7.05448717948718 0.722141921520233 209.196078431373,120.176470588235,104.152941176471 209.196078431373,120.176470588235,104.152941176471
7.05448717948718 0.696915686130524 235,135,117 235,135,117
7.38782051282051 0.524853229522705 0,0,0 0,0,0
7.38782051282051 1.03068447113037 25.8039215686274,14.8235294117647,12.8470588235294 25.8039215686274,14.8235294117647,12.8470588235294
7.38782051282051 0.327024638652802 51.6078431372549,29.6470588235294,25.6941176470588 51.6078431372549,29.6470588235294,25.6941176470588
7.38782051282051 0.482303887605667 78.3333333333333,45,39 78.3333333333333,45,39
7.38782051282051 0.759714841842651 104.137254901961,59.8235294117647,51.8470588235294 104.137254901961,59.8235294117647,51.8470588235294
7.38782051282051 0.900131404399872 130.862745098039,75.1764705882353,65.1529411764706 130.862745098039,75.1764705882353,65.1529411764706
7.38782051282051 0.526843011379242 156.666666666667,90,78 156.666666666667,90,78
7.38782051282051 0.532286405563354 183.392156862745,105.352941176471,91.3058823529412 183.392156862745,105.352941176471,91.3058823529412
7.38782051282051 0.230667129158974 209.196078431373,120.176470588235,104.152941176471 209.196078431373,120.176470588235,104.152941176471
7.38782051282051 0.635480761528015 235,135,117 235,135,117
7.74038461538461 0.468379110097885 0,0,0 0,0,0
7.74038461538461 0.194263786077499 25.8039215686274,14.8235294117647,12.8470588235294 25.8039215686274,14.8235294117647,12.8470588235294
7.74038461538461 0.516550302505493 51.6078431372549,29.6470588235294,25.6941176470588 51.6078431372549,29.6470588235294,25.6941176470588
7.74038461538461 0.36815682053566 78.3333333333333,45,39 78.3333333333333,45,39
7.74038461538461 0.906022489070892 104.137254901961,59.8235294117647,51.8470588235294 104.137254901961,59.8235294117647,51.8470588235294
7.74038461538461 0.518999755382538 130.862745098039,75.1764705882353,65.1529411764706 130.862745098039,75.1764705882353,65.1529411764706
7.74038461538461 0.32777351140976 156.666666666667,90,78 156.666666666667,90,78
7.74038461538461 0.882981300354004 183.392156862745,105.352941176471,91.3058823529412 183.392156862745,105.352941176471,91.3058823529412
7.74038461538461 0.670071482658386 209.196078431373,120.176470588235,104.152941176471 209.196078431373,120.176470588235,104.152941176471
7.74038461538461 0.514330744743347 235,135,117 235,135,117
8.10897435897436 0.654615223407745 0,0,0 0,0,0
8.10897435897436 0.50330376625061 25.8039215686274,14.8235294117647,12.8470588235294 25.8039215686274,14.8235294117647,12.8470588235294
8.10897435897436 0.135971948504448 51.6078431372549,29.6470588235294,25.6941176470588 51.6078431372549,29.6470588235294,25.6941176470588
8.10897435897436 0.328311055898666 78.3333333333333,45,39 78.3333333333333,45,39
8.10897435897436 0.724754989147186 104.137254901961,59.8235294117647,51.8470588235294 104.137254901961,59.8235294117647,51.8470588235294
8.10897435897436 0.514231741428375 130.862745098039,75.1764705882353,65.1529411764706 130.862745098039,75.1764705882353,65.1529411764706
8.10897435897436 0.634259700775146 156.666666666667,90,78 156.666666666667,90,78
8.10897435897436 0.582264304161072 183.392156862745,105.352941176471,91.3058823529412 183.392156862745,105.352941176471,91.3058823529412
8.10897435897436 0.12478443980217 209.196078431373,120.176470588235,104.152941176471 209.196078431373,120.176470588235,104.152941176471
8.10897435897436 0.464908361434937 235,135,117 235,135,117
8.49679487179487 0.960468590259552 0,0,0 0,0,0
8.49679487179487 0.250853180885315 25.8039215686274,14.8235294117647,12.8470588235294 25.8039215686274,14.8235294117647,12.8470588235294
8.49679487179487 1.0983270406723 51.6078431372549,29.6470588235294,25.6941176470588 51.6078431372549,29.6470588235294,25.6941176470588
8.49679487179487 0.523638606071472 78.3333333333333,45,39 78.3333333333333,45,39
8.49679487179487 1.07649481296539 104.137254901961,59.8235294117647,51.8470588235294 104.137254901961,59.8235294117647,51.8470588235294
8.49679487179487 0.947211146354675 130.862745098039,75.1764705882353,65.1529411764706 130.862745098039,75.1764705882353,65.1529411764706
8.49679487179487 0.730439841747284 156.666666666667,90,78 156.666666666667,90,78
8.49679487179487 0.536979138851166 183.392156862745,105.352941176471,91.3058823529412 183.392156862745,105.352941176471,91.3058823529412
8.49679487179487 0.680288255214691 209.196078431373,120.176470588235,104.152941176471 209.196078431373,120.176470588235,104.152941176471
8.49679487179487 0.695093154907227 235,135,117 235,135,117
8.90064102564103 0.777355372905731 0,0,0 0,0,0
8.90064102564103 0.968128442764282 25.8039215686274,14.8235294117647,12.8470588235294 25.8039215686274,14.8235294117647,12.8470588235294
8.90064102564103 0.393322944641113 51.6078431372549,29.6470588235294,25.6941176470588 51.6078431372549,29.6470588235294,25.6941176470588
8.90064102564103 0.438248574733734 78.3333333333333,45,39 78.3333333333333,45,39
8.90064102564103 0.375376403331757 104.137254901961,59.8235294117647,51.8470588235294 104.137254901961,59.8235294117647,51.8470588235294
8.90064102564103 0.614651083946228 130.862745098039,75.1764705882353,65.1529411764706 130.862745098039,75.1764705882353,65.1529411764706
8.90064102564103 0.736077189445496 156.666666666667,90,78 156.666666666667,90,78
8.90064102564103 0.755953013896942 183.392156862745,105.352941176471,91.3058823529412 183.392156862745,105.352941176471,91.3058823529412
8.90064102564103 0.723952114582062 209.196078431373,120.176470588235,104.152941176471 209.196078431373,120.176470588235,104.152941176471
8.90064102564103 0.631002724170685 235,135,117 235,135,117
9.32371794871795 0.766370475292206 0,0,0 0,0,0
9.32371794871795 0.673695206642151 25.8039215686274,14.8235294117647,12.8470588235294 25.8039215686274,14.8235294117647,12.8470588235294
9.32371794871795 0.341137737035751 51.6078431372549,29.6470588235294,25.6941176470588 51.6078431372549,29.6470588235294,25.6941176470588
9.32371794871795 0.372631847858429 78.3333333333333,45,39 78.3333333333333,45,39
9.32371794871795 0.786072731018066 104.137254901961,59.8235294117647,51.8470588235294 104.137254901961,59.8235294117647,51.8470588235294
9.32371794871795 0.492206126451492 130.862745098039,75.1764705882353,65.1529411764706 130.862745098039,75.1764705882353,65.1529411764706
9.32371794871795 0.232465401291847 156.666666666667,90,78 156.666666666667,90,78
9.32371794871795 0.525389432907104 183.392156862745,105.352941176471,91.3058823529412 183.392156862745,105.352941176471,91.3058823529412
9.32371794871795 0.86659187078476 209.196078431373,120.176470588235,104.152941176471 209.196078431373,120.176470588235,104.152941176471
9.32371794871795 0.781038641929626 235,135,117 235,135,117
9.76923076923077 0.436046451330185 0,0,0 0,0,0
9.76923076923077 0.396347314119339 25.8039215686274,14.8235294117647,12.8470588235294 25.8039215686274,14.8235294117647,12.8470588235294
9.76923076923077 0.444849222898483 51.6078431372549,29.6470588235294,25.6941176470588 51.6078431372549,29.6470588235294,25.6941176470588
9.76923076923077 0.211970046162605 78.3333333333333,45,39 78.3333333333333,45,39
9.76923076923077 0.542597711086273 104.137254901961,59.8235294117647,51.8470588235294 104.137254901961,59.8235294117647,51.8470588235294
9.76923076923077 0.656075060367584 130.862745098039,75.1764705882353,65.1529411764706 130.862745098039,75.1764705882353,65.1529411764706
9.76923076923077 0.611439943313599 156.666666666667,90,78 156.666666666667,90,78
9.76923076923077 0.311808377504349 183.392156862745,105.352941176471,91.3058823529412 183.392156862745,105.352941176471,91.3058823529412
9.76923076923077 0.5491943359375 209.196078431373,120.176470588235,104.152941176471 209.196078431373,120.176470588235,104.152941176471
9.76923076923077 0.531759440898895 235,135,117 235,135,117
10.2339743589744 0.375014394521713 0,0,0 0,0,0
10.2339743589744 0.932224452495575 25.8039215686274,14.8235294117647,12.8470588235294 25.8039215686274,14.8235294117647,12.8470588235294
10.2339743589744 0.302491992712021 51.6078431372549,29.6470588235294,25.6941176470588 51.6078431372549,29.6470588235294,25.6941176470588
10.2339743589744 0.669188618659973 78.3333333333333,45,39 78.3333333333333,45,39
10.2339743589744 0.277532428503036 104.137254901961,59.8235294117647,51.8470588235294 104.137254901961,59.8235294117647,51.8470588235294
10.2339743589744 0.796482741832733 130.862745098039,75.1764705882353,65.1529411764706 130.862745098039,75.1764705882353,65.1529411764706
10.2339743589744 0.448923826217651 156.666666666667,90,78 156.666666666667,90,78
10.2339743589744 0.681187927722931 183.392156862745,105.352941176471,91.3058823529412 183.392156862745,105.352941176471,91.3058823529412
10.2339743589744 0.347121387720108 209.196078431373,120.176470588235,104.152941176471 209.196078431373,120.176470588235,104.152941176471
10.2339743589744 0.554070651531219 235,135,117 235,135,117
10.7211538461538 0.689081966876984 0,0,0 0,0,0
10.7211538461538 0.892086446285248 25.8039215686274,14.8235294117647,12.8470588235294 25.8039215686274,14.8235294117647,12.8470588235294
10.7211538461538 0.375030159950256 51.6078431372549,29.6470588235294,25.6941176470588 51.6078431372549,29.6470588235294,25.6941176470588
10.7211538461538 0.52287882566452 78.3333333333333,45,39 78.3333333333333,45,39
10.7211538461538 0.268375307321548 104.137254901961,59.8235294117647,51.8470588235294 104.137254901961,59.8235294117647,51.8470588235294
10.7211538461538 0.259581953287125 130.862745098039,75.1764705882353,65.1529411764706 130.862745098039,75.1764705882353,65.1529411764706
10.7211538461538 0.505916357040405 156.666666666667,90,78 156.666666666667,90,78
10.7211538461538 0.379882037639618 183.392156862745,105.352941176471,91.3058823529412 183.392156862745,105.352941176471,91.3058823529412
10.7211538461538 0.639497756958008 209.196078431373,120.176470588235,104.152941176471 209.196078431373,120.176470588235,104.152941176471
10.7211538461538 0.28718575835228 235,135,117 235,135,117
11.2307692307692 0.354478567838669 0,0,0 0,0,0
11.2307692307692 0.164844661951065 25.8039215686274,14.8235294117647,12.8470588235294 25.8039215686274,14.8235294117647,12.8470588235294
11.2307692307692 0.51285582780838 51.6078431372549,29.6470588235294,25.6941176470588 51.6078431372549,29.6470588235294,25.6941176470588
11.2307692307692 0.417955696582794 78.3333333333333,45,39 78.3333333333333,45,39
11.2307692307692 0.398293703794479 104.137254901961,59.8235294117647,51.8470588235294 104.137254901961,59.8235294117647,51.8470588235294
11.2307692307692 0.492564499378204 130.862745098039,75.1764705882353,65.1529411764706 130.862745098039,75.1764705882353,65.1529411764706
11.2307692307692 0.422768414020538 156.666666666667,90,78 156.666666666667,90,78
11.2307692307692 0.54209291934967 183.392156862745,105.352941176471,91.3058823529412 183.392156862745,105.352941176471,91.3058823529412
11.2307692307692 0.340652078390121 209.196078431373,120.176470588235,104.152941176471 209.196078431373,120.176470588235,104.152941176471
11.2307692307692 0.415873557329178 235,135,117 235,135,117
11.7660256410256 0.497313439846039 0,0,0 0,0,0
11.7660256410256 0.25340873003006 25.8039215686274,14.8235294117647,12.8470588235294 25.8039215686274,14.8235294117647,12.8470588235294
11.7660256410256 0.36448135972023 51.6078431372549,29.6470588235294,25.6941176470588 51.6078431372549,29.6470588235294,25.6941176470588
11.7660256410256 0.201116472482681 78.3333333333333,45,39 78.3333333333333,45,39
11.7660256410256 0.27964198589325 104.137254901961,59.8235294117647,51.8470588235294 104.137254901961,59.8235294117647,51.8470588235294
11.7660256410256 0.166851162910461 130.862745098039,75.1764705882353,65.1529411764706 130.862745098039,75.1764705882353,65.1529411764706
11.7660256410256 0.302189558744431 156.666666666667,90,78 156.666666666667,90,78
11.7660256410256 0.494708031415939 183.392156862745,105.352941176471,91.3058823529412 183.392156862745,105.352941176471,91.3058823529412
11.7660256410256 0.38817036151886 209.196078431373,120.176470588235,104.152941176471 209.196078431373,120.176470588235,104.152941176471
11.7660256410256 0.480544656515121 235,135,117 235,135,117
12.3269230769231 0.537239551544189 0,0,0 0,0,0
12.3269230769231 0.183179184794426 25.8039215686274,14.8235294117647,12.8470588235294 25.8039215686274,14.8235294117647,12.8470588235294
12.3269230769231 0.54598468542099 51.6078431372549,29.6470588235294,25.6941176470588 51.6078431372549,29.6470588235294,25.6941176470588
12.3269230769231 0.224608317017555 78.3333333333333,45,39 78.3333333333333,45,39
12.3269230769231 0.336753576993942 104.137254901961,59.8235294117647,51.8470588235294 104.137254901961,59.8235294117647,51.8470588235294
12.3269230769231 0.818406343460083 130.862745098039,75.1764705882353,65.1529411764706 130.862745098039,75.1764705882353,65.1529411764706
12.3269230769231 0.45431461930275 156.666666666667,90,78 156.666666666667,90,78
12.3269230769231 0.400381803512573 183.392156862745,105.352941176471,91.3058823529412 183.392156862745,105.352941176471,91.3058823529412
12.3269230769231 0.262286573648453 209.196078431373,120.176470588235,104.152941176471 209.196078431373,120.176470588235,104.152941176471
12.3269230769231 0.402088671922684 235,135,117 235,135,117
12.9134615384615 0.319041311740875 0,0,0 0,0,0
12.9134615384615 0.549960255622864 25.8039215686274,14.8235294117647,12.8470588235294 25.8039215686274,14.8235294117647,12.8470588235294
12.9134615384615 0.431554824113846 51.6078431372549,29.6470588235294,25.6941176470588 51.6078431372549,29.6470588235294,25.6941176470588
12.9134615384615 0.883669495582581 78.3333333333333,45,39 78.3333333333333,45,39
12.9134615384615 0.379375547170639 104.137254901961,59.8235294117647,51.8470588235294 104.137254901961,59.8235294117647,51.8470588235294
12.9134615384615 0.339861005544662 130.862745098039,75.1764705882353,65.1529411764706 130.862745098039,75.1764705882353,65.1529411764706
12.9134615384615 0.718375265598297 156.666666666667,90,78 156.666666666667,90,78
12.9134615384615 0.670551478862762 183.392156862745,105.352941176471,91.3058823529412 183.392156862745,105.352941176471,91.3058823529412
12.9134615384615 0.560630261898041 209.196078431373,120.176470588235,104.152941176471 209.196078431373,120.176470588235,104.152941176471
12.9134615384615 0.567080795764923 235,135,117 235,135,117
13.5288461538462 0.243825942277908 0,0,0 0,0,0
13.5288461538462 0.333544135093689 25.8039215686274,14.8235294117647,12.8470588235294 25.8039215686274,14.8235294117647,12.8470588235294
13.5288461538462 0.350484818220139 51.6078431372549,29.6470588235294,25.6941176470588 51.6078431372549,29.6470588235294,25.6941176470588
13.5288461538462 0.173688262701035 78.3333333333333,45,39 78.3333333333333,45,39
13.5288461538462 0.287692010402679 104.137254901961,59.8235294117647,51.8470588235294 104.137254901961,59.8235294117647,51.8470588235294
13.5288461538462 0.304149448871613 130.862745098039,75.1764705882353,65.1529411764706 130.862745098039,75.1764705882353,65.1529411764706
13.5288461538462 0.467170864343643 156.666666666667,90,78 156.666666666667,90,78
13.5288461538462 0.366013497114182 183.392156862745,105.352941176471,91.3058823529412 183.392156862745,105.352941176471,91.3058823529412
13.5288461538462 0.260251075029373 209.196078431373,120.176470588235,104.152941176471 209.196078431373,120.176470588235,104.152941176471
13.5288461538462 0.517615377902985 235,135,117 235,135,117
14.1730769230769 0.147898152470589 0,0,0 0,0,0
14.1730769230769 0.392569422721863 25.8039215686274,14.8235294117647,12.8470588235294 25.8039215686274,14.8235294117647,12.8470588235294
14.1730769230769 0.147306516766548 51.6078431372549,29.6470588235294,25.6941176470588 51.6078431372549,29.6470588235294,25.6941176470588
14.1730769230769 0.451515316963196 78.3333333333333,45,39 78.3333333333333,45,39
14.1730769230769 0.509858131408691 104.137254901961,59.8235294117647,51.8470588235294 104.137254901961,59.8235294117647,51.8470588235294
14.1730769230769 0.4457146525383 130.862745098039,75.1764705882353,65.1529411764706 130.862745098039,75.1764705882353,65.1529411764706
14.1730769230769 0.346548140048981 156.666666666667,90,78 156.666666666667,90,78
14.1730769230769 0.614377856254578 183.392156862745,105.352941176471,91.3058823529412 183.392156862745,105.352941176471,91.3058823529412
14.1730769230769 0.270308017730713 209.196078431373,120.176470588235,104.152941176471 209.196078431373,120.176470588235,104.152941176471
14.1730769230769 0.417016506195068 235,135,117 235,135,117
14.849358974359 0.438409239053726 0,0,0 0,0,0
14.849358974359 0.270187199115753 25.8039215686274,14.8235294117647,12.8470588235294 25.8039215686274,14.8235294117647,12.8470588235294
14.849358974359 0.139068409800529 51.6078431372549,29.6470588235294,25.6941176470588 51.6078431372549,29.6470588235294,25.6941176470588
14.849358974359 0.134812369942665 78.3333333333333,45,39 78.3333333333333,45,39
14.849358974359 0.352992415428162 104.137254901961,59.8235294117647,51.8470588235294 104.137254901961,59.8235294117647,51.8470588235294
14.849358974359 0.369162261486053 130.862745098039,75.1764705882353,65.1529411764706 130.862745098039,75.1764705882353,65.1529411764706
14.849358974359 0.281582832336426 156.666666666667,90,78 156.666666666667,90,78
14.849358974359 0.250238686800003 183.392156862745,105.352941176471,91.3058823529412 183.392156862745,105.352941176471,91.3058823529412
14.849358974359 0.152493461966515 209.196078431373,120.176470588235,104.152941176471 209.196078431373,120.176470588235,104.152941176471
14.849358974359 0.35007119178772 235,135,117 235,135,117
15.5544871794872 0.375635981559753 0,0,0 0,0,0
15.5544871794872 0.303830116987228 25.8039215686274,14.8235294117647,12.8470588235294 25.8039215686274,14.8235294117647,12.8470588235294
15.5544871794872 0.169918224215508 51.6078431372549,29.6470588235294,25.6941176470588 51.6078431372549,29.6470588235294,25.6941176470588
15.5544871794872 0.318712651729584 78.3333333333333,45,39 78.3333333333333,45,39
15.5544871794872 0.223490595817566 104.137254901961,59.8235294117647,51.8470588235294 104.137254901961,59.8235294117647,51.8470588235294
15.5544871794872 0.232829406857491 130.862745098039,75.1764705882353,65.1529411764706 130.862745098039,75.1764705882353,65.1529411764706
15.5544871794872 0.220886662602425 156.666666666667,90,78 156.666666666667,90,78
15.5544871794872 0.36919978260994 183.392156862745,105.352941176471,91.3058823529412 183.392156862745,105.352941176471,91.3058823529412
15.5544871794872 0.279243230819702 209.196078431373,120.176470588235,104.152941176471 209.196078431373,120.176470588235,104.152941176471
15.5544871794872 0.316171526908875 235,135,117 235,135,117
16.2948717948718 0.336017370223999 0,0,0 0,0,0
16.2948717948718 0.205741941928864 25.8039215686274,14.8235294117647,12.8470588235294 25.8039215686274,14.8235294117647,12.8470588235294
16.2948717948718 0.270815253257751 51.6078431372549,29.6470588235294,25.6941176470588 51.6078431372549,29.6470588235294,25.6941176470588
16.2948717948718 0.215436205267906 78.3333333333333,45,39 78.3333333333333,45,39
16.2948717948718 0.220368713140488 104.137254901961,59.8235294117647,51.8470588235294 104.137254901961,59.8235294117647,51.8470588235294
16.2948717948718 0.357287257909775 130.862745098039,75.1764705882353,65.1529411764706 130.862745098039,75.1764705882353,65.1529411764706
16.2948717948718 0.311433881521225 156.666666666667,90,78 156.666666666667,90,78
16.2948717948718 0.334268689155579 183.392156862745,105.352941176471,91.3058823529412 183.392156862745,105.352941176471,91.3058823529412
16.2948717948718 0.0516838133335114 209.196078431373,120.176470588235,104.152941176471 209.196078431373,120.176470588235,104.152941176471
16.2948717948718 0.194836392998695 235,135,117 235,135,117
17.0705128205128 0.466751962900162 0,0,0 0,0,0
17.0705128205128 0.36104679107666 25.8039215686274,14.8235294117647,12.8470588235294 25.8039215686274,14.8235294117647,12.8470588235294
17.0705128205128 0.0340119898319244 51.6078431372549,29.6470588235294,25.6941176470588 51.6078431372549,29.6470588235294,25.6941176470588
17.0705128205128 0.0929097235202789 78.3333333333333,45,39 78.3333333333333,45,39
17.0705128205128 0.554856956005096 104.137254901961,59.8235294117647,51.8470588235294 104.137254901961,59.8235294117647,51.8470588235294
17.0705128205128 0.307357758283615 130.862745098039,75.1764705882353,65.1529411764706 130.862745098039,75.1764705882353,65.1529411764706
17.0705128205128 0.389662057161331 156.666666666667,90,78 156.666666666667,90,78
17.0705128205128 0.227753654122353 183.392156862745,105.352941176471,91.3058823529412 183.392156862745,105.352941176471,91.3058823529412
17.0705128205128 0.227551057934761 209.196078431373,120.176470588235,104.152941176471 209.196078431373,120.176470588235,104.152941176471
17.0705128205128 0.415609657764435 235,135,117 235,135,117
17.8846153846154 0.552225887775421 0,0,0 0,0,0
17.8846153846154 0.243320181965828 25.8039215686274,14.8235294117647,12.8470588235294 25.8039215686274,14.8235294117647,12.8470588235294
17.8846153846154 0.533170521259308 51.6078431372549,29.6470588235294,25.6941176470588 51.6078431372549,29.6470588235294,25.6941176470588
17.8846153846154 0.219667300581932 78.3333333333333,45,39 78.3333333333333,45,39
17.8846153846154 0.207200229167938 104.137254901961,59.8235294117647,51.8470588235294 104.137254901961,59.8235294117647,51.8470588235294
17.8846153846154 0.441452324390411 130.862745098039,75.1764705882353,65.1529411764706 130.862745098039,75.1764705882353,65.1529411764706
17.8846153846154 0.400682836771011 156.666666666667,90,78 156.666666666667,90,78
17.8846153846154 0.545804858207703 183.392156862745,105.352941176471,91.3058823529412 183.392156862745,105.352941176471,91.3058823529412
17.8846153846154 0.176205024123192 209.196078431373,120.176470588235,104.152941176471 209.196078431373,120.176470588235,104.152941176471
17.8846153846154 0.22774963080883 235,135,117 235,135,117
18.7371794871795 0.315650254487991 0,0,0 0,0,0
18.7371794871795 0.239748045802116 25.8039215686274,14.8235294117647,12.8470588235294 25.8039215686274,14.8235294117647,12.8470588235294
18.7371794871795 0.179958537220955 51.6078431372549,29.6470588235294,25.6941176470588 51.6078431372549,29.6470588235294,25.6941176470588
18.7371794871795 0.211378902196884 78.3333333333333,45,39 78.3333333333333,45,39
18.7371794871795 0.289680898189545 104.137254901961,59.8235294117647,51.8470588235294 104.137254901961,59.8235294117647,51.8470588235294
18.7371794871795 0.312140107154846 130.862745098039,75.1764705882353,65.1529411764706 130.862745098039,75.1764705882353,65.1529411764706
18.7371794871795 0.260847717523575 156.666666666667,90,78 156.666666666667,90,78
18.7371794871795 0.228098526597023 183.392156862745,105.352941176471,91.3058823529412 183.392156862745,105.352941176471,91.3058823529412
18.7371794871795 0.138547718524933 209.196078431373,120.176470588235,104.152941176471 209.196078431373,120.176470588235,104.152941176471
18.7371794871795 0.157880410552025 235,135,117 235,135,117
19.6282051282051 0.351776659488678 0,0,0 0,0,0
19.6282051282051 0.273958504199982 25.8039215686274,14.8235294117647,12.8470588235294 25.8039215686274,14.8235294117647,12.8470588235294
19.6282051282051 0.168734595179558 51.6078431372549,29.6470588235294,25.6941176470588 51.6078431372549,29.6470588235294,25.6941176470588
19.6282051282051 0.12004042416811 78.3333333333333,45,39 78.3333333333333,45,39
19.6282051282051 0.453694969415665 104.137254901961,59.8235294117647,51.8470588235294 104.137254901961,59.8235294117647,51.8470588235294
19.6282051282051 0.226570695638657 130.862745098039,75.1764705882353,65.1529411764706 130.862745098039,75.1764705882353,65.1529411764706
19.6282051282051 0.149635702371597 156.666666666667,90,78 156.666666666667,90,78
19.6282051282051 0.0828923135995865 183.392156862745,105.352941176471,91.3058823529412 183.392156862745,105.352941176471,91.3058823529412
19.6282051282051 0.127526983618736 209.196078431373,120.176470588235,104.152941176471 209.196078431373,120.176470588235,104.152941176471
19.6282051282051 0.162903144955635 235,135,117 235,135,117
20.5641025641026 0.197232410311699 0,0,0 0,0,0
20.5641025641026 0.159589141607285 25.8039215686274,14.8235294117647,12.8470588235294 25.8039215686274,14.8235294117647,12.8470588235294
20.5641025641026 0.151643097400665 51.6078431372549,29.6470588235294,25.6941176470588 51.6078431372549,29.6470588235294,25.6941176470588
20.5641025641026 0.132980465888977 78.3333333333333,45,39 78.3333333333333,45,39
20.5641025641026 0.260623216629028 104.137254901961,59.8235294117647,51.8470588235294 104.137254901961,59.8235294117647,51.8470588235294
20.5641025641026 0.442532420158386 130.862745098039,75.1764705882353,65.1529411764706 130.862745098039,75.1764705882353,65.1529411764706
20.5641025641026 0.5415398478508 156.666666666667,90,78 156.666666666667,90,78
20.5641025641026 0.228515341877937 183.392156862745,105.352941176471,91.3058823529412 183.392156862745,105.352941176471,91.3058823529412
20.5641025641026 0.117429852485657 209.196078431373,120.176470588235,104.152941176471 209.196078431373,120.176470588235,104.152941176471
20.5641025641026 0.208738714456558 235,135,117 235,135,117
21.5416666666667 0.252879053354263 0,0,0 0,0,0
21.5416666666667 0.604028761386871 25.8039215686274,14.8235294117647,12.8470588235294 25.8039215686274,14.8235294117647,12.8470588235294
21.5416666666667 0.378509670495987 51.6078431372549,29.6470588235294,25.6941176470588 51.6078431372549,29.6470588235294,25.6941176470588
21.5416666666667 0.207104906439781 78.3333333333333,45,39 78.3333333333333,45,39
21.5416666666667 0.113723017275333 104.137254901961,59.8235294117647,51.8470588235294 104.137254901961,59.8235294117647,51.8470588235294
21.5416666666667 0.602006733417511 130.862745098039,75.1764705882353,65.1529411764706 130.862745098039,75.1764705882353,65.1529411764706
21.5416666666667 0.335463434457779 156.666666666667,90,78 156.666666666667,90,78
21.5416666666667 0.163503721356392 183.392156862745,105.352941176471,91.3058823529412 183.392156862745,105.352941176471,91.3058823529412
21.5416666666667 0.119618266820908 209.196078431373,120.176470588235,104.152941176471 209.196078431373,120.176470588235,104.152941176471
21.5416666666667 0.129860326647758 235,135,117 235,135,117
22.5673076923077 0.0646372213959694 0,0,0 0,0,0
22.5673076923077 0.178423672914505 25.8039215686274,14.8235294117647,12.8470588235294 25.8039215686274,14.8235294117647,12.8470588235294
22.5673076923077 0.114612556993961 51.6078431372549,29.6470588235294,25.6941176470588 51.6078431372549,29.6470588235294,25.6941176470588
22.5673076923077 0.0614951811730862 78.3333333333333,45,39 78.3333333333333,45,39
22.5673076923077 0.214148089289665 104.137254901961,59.8235294117647,51.8470588235294 104.137254901961,59.8235294117647,51.8470588235294
22.5673076923077 0.112378686666489 130.862745098039,75.1764705882353,65.1529411764706 130.862745098039,75.1764705882353,65.1529411764706
22.5673076923077 0.380411267280579 156.666666666667,90,78 156.666666666667,90,78
22.5673076923077 0.224997207522392 183.392156862745,105.352941176471,91.3058823529412 183.392156862745,105.352941176471,91.3058823529412
22.5673076923077 0.228796765208244 209.196078431373,120.176470588235,104.152941176471 209.196078431373,120.176470588235,104.152941176471
22.5673076923077 0.212361961603165 235,135,117 235,135,117
23.6442307692308 0.12228225171566 0,0,0 0,0,0
23.6442307692308 0.224083378911018 25.8039215686274,14.8235294117647,12.8470588235294 25.8039215686274,14.8235294117647,12.8470588235294
23.6442307692308 0.274574279785156 51.6078431372549,29.6470588235294,25.6941176470588 51.6078431372549,29.6470588235294,25.6941176470588
23.6442307692308 0.25410521030426 78.3333333333333,45,39 78.3333333333333,45,39
23.6442307692308 0.0759283974766731 104.137254901961,59.8235294117647,51.8470588235294 104.137254901961,59.8235294117647,51.8470588235294
23.6442307692308 0.329486578702927 130.862745098039,75.1764705882353,65.1529411764706 130.862745098039,75.1764705882353,65.1529411764706
23.6442307692308 0.325658529996872 156.666666666667,90,78 156.666666666667,90,78
23.6442307692308 0.220740586519241 183.392156862745,105.352941176471,91.3058823529412 183.392156862745,105.352941176471,91.3058823529412
23.6442307692308 0.086956113576889 209.196078431373,120.176470588235,104.152941176471 209.196078431373,120.176470588235,104.152941176471
23.6442307692308 0.0506661534309387 235,135,117 235,135,117
24.7692307692308 0.119727112352848 0,0,0 0,0,0
24.7692307692308 0.102132126688957 25.8039215686274,14.8235294117647,12.8470588235294 25.8039215686274,14.8235294117647,12.8470588235294
24.7692307692308 0.118928253650665 51.6078431372549,29.6470588235294,25.6941176470588 51.6078431372549,29.6470588235294,25.6941176470588
24.7692307692308 0.392418384552002 78.3333333333333,45,39 78.3333333333333,45,39
24.7692307692308 0.224873453378677 104.137254901961,59.8235294117647,51.8470588235294 104.137254901961,59.8235294117647,51.8470588235294
24.7692307692308 0.275590151548386 130.862745098039,75.1764705882353,65.1529411764706 130.862745098039,75.1764705882353,65.1529411764706
24.7692307692308 0.0926736369729042 156.666666666667,90,78 156.666666666667,90,78
24.7692307692308 0.240820869803429 183.392156862745,105.352941176471,91.3058823529412 183.392156862745,105.352941176471,91.3058823529412
24.7692307692308 0.18152329325676 209.196078431373,120.176470588235,104.152941176471 209.196078431373,120.176470588235,104.152941176471
24.7692307692308 0.0570325627923012 235,135,117 235,135,117
25.9487179487179 0.0506958775222301 0,0,0 0,0,0
25.9487179487179 0.10757390409708 25.8039215686274,14.8235294117647,12.8470588235294 25.8039215686274,14.8235294117647,12.8470588235294
25.9487179487179 0.222792968153954 51.6078431372549,29.6470588235294,25.6941176470588 51.6078431372549,29.6470588235294,25.6941176470588
25.9487179487179 0.283831149339676 78.3333333333333,45,39 78.3333333333333,45,39
25.9487179487179 0.127395942807198 104.137254901961,59.8235294117647,51.8470588235294 104.137254901961,59.8235294117647,51.8470588235294
25.9487179487179 0.286481857299805 130.862745098039,75.1764705882353,65.1529411764706 130.862745098039,75.1764705882353,65.1529411764706
25.9487179487179 0.222583249211311 156.666666666667,90,78 156.666666666667,90,78
25.9487179487179 0.198974683880806 183.392156862745,105.352941176471,91.3058823529412 183.392156862745,105.352941176471,91.3058823529412
25.9487179487179 0.112334348261356 209.196078431373,120.176470588235,104.152941176471 209.196078431373,120.176470588235,104.152941176471
25.9487179487179 0.353535354137421 235,135,117 235,135,117
27.1826923076923 0.14485040307045 0,0,0 0,0,0
27.1826923076923 0.230795383453369 25.8039215686274,14.8235294117647,12.8470588235294 25.8039215686274,14.8235294117647,12.8470588235294
27.1826923076923 0.200821235775948 51.6078431372549,29.6470588235294,25.6941176470588 51.6078431372549,29.6470588235294,25.6941176470588
27.1826923076923 0.130327269434929 78.3333333333333,45,39 78.3333333333333,45,39
27.1826923076923 0.120531186461449 104.137254901961,59.8235294117647,51.8470588235294 104.137254901961,59.8235294117647,51.8470588235294
27.1826923076923 0.239658087491989 130.862745098039,75.1764705882353,65.1529411764706 130.862745098039,75.1764705882353,65.1529411764706
27.1826923076923 0.0288295242935419 156.666666666667,90,78 156.666666666667,90,78
27.1826923076923 0.128580003976822 183.392156862745,105.352941176471,91.3058823529412 183.392156862745,105.352941176471,91.3058823529412
27.1826923076923 0.142585903406143 209.196078431373,120.176470588235,104.152941176471 209.196078431373,120.176470588235,104.152941176471
27.1826923076923 0.238067969679832 235,135,117 235,135,117
28.4775641025641 0.0822729468345642 0,0,0 0,0,0
28.4775641025641 0.126973271369934 25.8039215686274,14.8235294117647,12.8470588235294 25.8039215686274,14.8235294117647,12.8470588235294
28.4775641025641 0.304915100336075 51.6078431372549,29.6470588235294,25.6941176470588 51.6078431372549,29.6470588235294,25.6941176470588
28.4775641025641 0.227368086576462 78.3333333333333,45,39 78.3333333333333,45,39
28.4775641025641 0.241228774189949 104.137254901961,59.8235294117647,51.8470588235294 104.137254901961,59.8235294117647,51.8470588235294
28.4775641025641 0.0380346029996872 130.862745098039,75.1764705882353,65.1529411764706 130.862745098039,75.1764705882353,65.1529411764706
28.4775641025641 0.13489992916584 156.666666666667,90,78 156.666666666667,90,78
28.4775641025641 0.0865603685379028 183.392156862745,105.352941176471,91.3058823529412 183.392156862745,105.352941176471,91.3058823529412
28.4775641025641 0.19671793282032 209.196078431373,120.176470588235,104.152941176471 209.196078431373,120.176470588235,104.152941176471
28.4775641025641 0.0473198182880878 235,135,117 235,135,117
29.8333333333333 0.109053455293179 0,0,0 0,0,0
29.8333333333333 0.0378567092120647 25.8039215686274,14.8235294117647,12.8470588235294 25.8039215686274,14.8235294117647,12.8470588235294
29.8333333333333 0.326532304286957 51.6078431372549,29.6470588235294,25.6941176470588 51.6078431372549,29.6470588235294,25.6941176470588
29.8333333333333 0.19535069167614 78.3333333333333,45,39 78.3333333333333,45,39
29.8333333333333 0.126152515411377 104.137254901961,59.8235294117647,51.8470588235294 104.137254901961,59.8235294117647,51.8470588235294
29.8333333333333 0.153328225016594 130.862745098039,75.1764705882353,65.1529411764706 130.862745098039,75.1764705882353,65.1529411764706
29.8333333333333 0.0422515161335468 156.666666666667,90,78 156.666666666667,90,78
29.8333333333333 0.157725185155869 183.392156862745,105.352941176471,91.3058823529412 183.392156862745,105.352941176471,91.3058823529412
29.8333333333333 0.147372007369995 209.196078431373,120.176470588235,104.152941176471 209.196078431373,120.176470588235,104.152941176471
29.8333333333333 0.0441205538809299 235,135,117 235,135,117
31.2564102564103 0.0705294758081436 0,0,0 0,0,0
31.2564102564103 0.0804430693387985 25.8039215686274,14.8235294117647,12.8470588235294 25.8039215686274,14.8235294117647,12.8470588235294
31.2564102564103 0.356323778629303 51.6078431372549,29.6470588235294,25.6941176470588 51.6078431372549,29.6470588235294,25.6941176470588
31.2564102564103 0.254549533128738 78.3333333333333,45,39 78.3333333333333,45,39
31.2564102564103 0.170110508799553 104.137254901961,59.8235294117647,51.8470588235294 104.137254901961,59.8235294117647,51.8470588235294
31.2564102564103 0.127747893333435 130.862745098039,75.1764705882353,65.1529411764706 130.862745098039,75.1764705882353,65.1529411764706
31.2564102564103 0.124140240252018 156.666666666667,90,78 156.666666666667,90,78
31.2564102564103 0.163027718663216 183.392156862745,105.352941176471,91.3058823529412 183.392156862745,105.352941176471,91.3058823529412
31.2564102564103 0.0449127443134785 209.196078431373,120.176470588235,104.152941176471 209.196078431373,120.176470588235,104.152941176471
31.2564102564103 0.0825785920023918 235,135,117 235,135,117
32.7435897435897 0.191301763057709 0,0,0 0,0,0
32.7435897435897 0.194513127207756 25.8039215686274,14.8235294117647,12.8470588235294 25.8039215686274,14.8235294117647,12.8470588235294
32.7435897435897 0.0770542398095131 51.6078431372549,29.6470588235294,25.6941176470588 51.6078431372549,29.6470588235294,25.6941176470588
32.7435897435897 0.200528010725975 78.3333333333333,45,39 78.3333333333333,45,39
32.7435897435897 0.133365616202354 104.137254901961,59.8235294117647,51.8470588235294 104.137254901961,59.8235294117647,51.8470588235294
32.7435897435897 0.0466050319373608 130.862745098039,75.1764705882353,65.1529411764706 130.862745098039,75.1764705882353,65.1529411764706
32.7435897435897 0.162162408232689 156.666666666667,90,78 156.666666666667,90,78
32.7435897435897 0.168131917715073 183.392156862745,105.352941176471,91.3058823529412 183.392156862745,105.352941176471,91.3058823529412
32.7435897435897 0.123800598084927 209.196078431373,120.176470588235,104.152941176471 209.196078431373,120.176470588235,104.152941176471
32.7435897435897 0.0978487059473991 235,135,117 235,135,117
34.3044871794872 0.0787762030959129 0,0,0 0,0,0
34.3044871794872 0.14290851354599 25.8039215686274,14.8235294117647,12.8470588235294 25.8039215686274,14.8235294117647,12.8470588235294
34.3044871794872 0.113800406455994 51.6078431372549,29.6470588235294,25.6941176470588 51.6078431372549,29.6470588235294,25.6941176470588
34.3044871794872 0.246655121445656 78.3333333333333,45,39 78.3333333333333,45,39
34.3044871794872 0.232390269637108 104.137254901961,59.8235294117647,51.8470588235294 104.137254901961,59.8235294117647,51.8470588235294
34.3044871794872 0.202954471111298 130.862745098039,75.1764705882353,65.1529411764706 130.862745098039,75.1764705882353,65.1529411764706
34.3044871794872 0.199400752782822 156.666666666667,90,78 156.666666666667,90,78
34.3044871794872 0.116207957267761 183.392156862745,105.352941176471,91.3058823529412 183.392156862745,105.352941176471,91.3058823529412
34.3044871794872 0.204152628779411 209.196078431373,120.176470588235,104.152941176471 209.196078431373,120.176470588235,104.152941176471
34.3044871794872 0.0153717948123813 235,135,117 235,135,117
35.9358974358974 0.0870254933834076 0,0,0 0,0,0
35.9358974358974 0.0683745220303535 25.8039215686274,14.8235294117647,12.8470588235294 25.8039215686274,14.8235294117647,12.8470588235294
35.9358974358974 0.079672209918499 51.6078431372549,29.6470588235294,25.6941176470588 51.6078431372549,29.6470588235294,25.6941176470588
35.9358974358974 0.246928170323372 78.3333333333333,45,39 78.3333333333333,45,39
35.9358974358974 0.0935479253530502 104.137254901961,59.8235294117647,51.8470588235294 104.137254901961,59.8235294117647,51.8470588235294
35.9358974358974 0.0673164650797844 130.862745098039,75.1764705882353,65.1529411764706 130.862745098039,75.1764705882353,65.1529411764706
35.9358974358974 0.133186683058739 156.666666666667,90,78 156.666666666667,90,78
35.9358974358974 0.164324730634689 183.392156862745,105.352941176471,91.3058823529412 183.392156862745,105.352941176471,91.3058823529412
35.9358974358974 0.0227599907666445 209.196078431373,120.176470588235,104.152941176471 209.196078431373,120.176470588235,104.152941176471
35.9358974358974 0.0935291722416878 235,135,117 235,135,117
37.6474358974359 0.0599369630217552 0,0,0 0,0,0
37.6474358974359 0.182949036359787 25.8039215686274,14.8235294117647,12.8470588235294 25.8039215686274,14.8235294117647,12.8470588235294
37.6474358974359 0.0820496678352356 51.6078431372549,29.6470588235294,25.6941176470588 51.6078431372549,29.6470588235294,25.6941176470588
37.6474358974359 0.130189165472984 78.3333333333333,45,39 78.3333333333333,45,39
37.6474358974359 0.152782499790192 104.137254901961,59.8235294117647,51.8470588235294 104.137254901961,59.8235294117647,51.8470588235294
37.6474358974359 0.221665799617767 130.862745098039,75.1764705882353,65.1529411764706 130.862745098039,75.1764705882353,65.1529411764706
37.6474358974359 0.101278856396675 156.666666666667,90,78 156.666666666667,90,78
37.6474358974359 0.0819317251443863 183.392156862745,105.352941176471,91.3058823529412 183.392156862745,105.352941176471,91.3058823529412
37.6474358974359 0.07819963991642 209.196078431373,120.176470588235,104.152941176471 209.196078431373,120.176470588235,104.152941176471
37.6474358974359 0.0840277746319771 235,135,117 235,135,117
39.4391025641026 0.228082180023193 0,0,0 0,0,0
39.4391025641026 0.082749143242836 25.8039215686274,14.8235294117647,12.8470588235294 25.8039215686274,14.8235294117647,12.8470588235294
39.4391025641026 0.154853343963623 51.6078431372549,29.6470588235294,25.6941176470588 51.6078431372549,29.6470588235294,25.6941176470588
39.4391025641026 0.0796608626842499 78.3333333333333,45,39 78.3333333333333,45,39
39.4391025641026 0.156124800443649 104.137254901961,59.8235294117647,51.8470588235294 104.137254901961,59.8235294117647,51.8470588235294
39.4391025641026 0.154243722558022 130.862745098039,75.1764705882353,65.1529411764706 130.862745098039,75.1764705882353,65.1529411764706
39.4391025641026 0.0511416718363762 156.666666666667,90,78 156.666666666667,90,78
39.4391025641026 0.0464367642998695 183.392156862745,105.352941176471,91.3058823529412 183.392156862745,105.352941176471,91.3058823529412
39.4391025641026 0.171126991510391 209.196078431373,120.176470588235,104.152941176471 209.196078431373,120.176470588235,104.152941176471
39.4391025641026 0.0410487428307533 235,135,117 235,135,117
41.3173076923077 0.0728199481964111 0,0,0 0,0,0
41.3173076923077 0.0356121882796288 25.8039215686274,14.8235294117647,12.8470588235294 25.8039215686274,14.8235294117647,12.8470588235294
41.3173076923077 0.147451981902122 51.6078431372549,29.6470588235294,25.6941176470588 51.6078431372549,29.6470588235294,25.6941176470588
41.3173076923077 0.119195900857449 78.3333333333333,45,39 78.3333333333333,45,39
41.3173076923077 0.131420403718948 104.137254901961,59.8235294117647,51.8470588235294 104.137254901961,59.8235294117647,51.8470588235294
41.3173076923077 0.137727826833725 130.862745098039,75.1764705882353,65.1529411764706 130.862745098039,75.1764705882353,65.1529411764706
41.3173076923077 0.0394698902964592 156.666666666667,90,78 156.666666666667,90,78
41.3173076923077 0.0881710574030876 183.392156862745,105.352941176471,91.3058823529412 183.392156862745,105.352941176471,91.3058823529412
41.3173076923077 0.0417782925069332 209.196078431373,120.176470588235,104.152941176471 209.196078431373,120.176470588235,104.152941176471
41.3173076923077 0.0560717098414898 235,135,117 235,135,117
43.2852564102564 0.0814539641141891 0,0,0 0,0,0
43.2852564102564 0.0906740203499794 25.8039215686274,14.8235294117647,12.8470588235294 25.8039215686274,14.8235294117647,12.8470588235294
43.2852564102564 0.0582966394722462 51.6078431372549,29.6470588235294,25.6941176470588 51.6078431372549,29.6470588235294,25.6941176470588
43.2852564102564 0.168130055069923 78.3333333333333,45,39 78.3333333333333,45,39
43.2852564102564 0.13024590909481 104.137254901961,59.8235294117647,51.8470588235294 104.137254901961,59.8235294117647,51.8470588235294
43.2852564102564 0.0479000806808472 130.862745098039,75.1764705882353,65.1529411764706 130.862745098039,75.1764705882353,65.1529411764706
43.2852564102564 0.106295309960842 156.666666666667,90,78 156.666666666667,90,78
43.2852564102564 0.102547593414783 183.392156862745,105.352941176471,91.3058823529412 183.392156862745,105.352941176471,91.3058823529412
43.2852564102564 0.157536342740059 209.196078431373,120.176470588235,104.152941176471 209.196078431373,120.176470588235,104.152941176471
43.2852564102564 0.108162567019463 235,135,117 235,135,117
45.3461538461538 0.135500684380531 0,0,0 0,0,0
45.3461538461538 0.0712881237268448 25.8039215686274,14.8235294117647,12.8470588235294 25.8039215686274,14.8235294117647,12.8470588235294
45.3461538461538 0.108187831938267 51.6078431372549,29.6470588235294,25.6941176470588 51.6078431372549,29.6470588235294,25.6941176470588
45.3461538461538 0.0746166184544563 78.3333333333333,45,39 78.3333333333333,45,39
45.3461538461538 0.0936126485466957 104.137254901961,59.8235294117647,51.8470588235294 104.137254901961,59.8235294117647,51.8470588235294
45.3461538461538 0.0503127612173557 130.862745098039,75.1764705882353,65.1529411764706 130.862745098039,75.1764705882353,65.1529411764706
45.3461538461538 0.129724636673927 156.666666666667,90,78 156.666666666667,90,78
45.3461538461538 0.0677567273378372 183.392156862745,105.352941176471,91.3058823529412 183.392156862745,105.352941176471,91.3058823529412
45.3461538461538 0.133995831012726 209.196078431373,120.176470588235,104.152941176471 209.196078431373,120.176470588235,104.152941176471
45.3461538461538 0.0775809586048126 235,135,117 235,135,117
47.5064102564103 0.0879152417182922 0,0,0 0,0,0
47.5064102564103 0.0613650940358639 25.8039215686274,14.8235294117647,12.8470588235294 25.8039215686274,14.8235294117647,12.8470588235294
47.5064102564103 0.0766095370054245 51.6078431372549,29.6470588235294,25.6941176470588 51.6078431372549,29.6470588235294,25.6941176470588
47.5064102564103 0.173701852560043 78.3333333333333,45,39 78.3333333333333,45,39
47.5064102564103 0.0641912519931793 104.137254901961,59.8235294117647,51.8470588235294 104.137254901961,59.8235294117647,51.8470588235294
47.5064102564103 0.0168549977242947 130.862745098039,75.1764705882353,65.1529411764706 130.862745098039,75.1764705882353,65.1529411764706
47.5064102564103 0.124533459544182 156.666666666667,90,78 156.666666666667,90,78
47.5064102564103 0.0628927573561668 183.392156862745,105.352941176471,91.3058823529412 183.392156862745,105.352941176471,91.3058823529412
47.5064102564103 0.0759809091687202 209.196078431373,120.176470588235,104.152941176471 209.196078431373,120.176470588235,104.152941176471
47.5064102564103 0.0583058446645737 235,135,117 235,135,117
49.7692307692308 0.0775977224111557 0,0,0 0,0,0
49.7692307692308 0.123262368142605 25.8039215686274,14.8235294117647,12.8470588235294 25.8039215686274,14.8235294117647,12.8470588235294
49.7692307692308 0.186306223273277 51.6078431372549,29.6470588235294,25.6941176470588 51.6078431372549,29.6470588235294,25.6941176470588
49.7692307692308 0.187439873814583 78.3333333333333,45,39 78.3333333333333,45,39
49.7692307692308 0.12071692943573 104.137254901961,59.8235294117647,51.8470588235294 104.137254901961,59.8235294117647,51.8470588235294
49.7692307692308 0.0337477289140224 130.862745098039,75.1764705882353,65.1529411764706 130.862745098039,75.1764705882353,65.1529411764706
49.7692307692308 0.0982203707098961 156.666666666667,90,78 156.666666666667,90,78
49.7692307692308 0.188566669821739 183.392156862745,105.352941176471,91.3058823529412 183.392156862745,105.352941176471,91.3058823529412
49.7692307692308 0.0367709398269653 209.196078431373,120.176470588235,104.152941176471 209.196078431373,120.176470588235,104.152941176471
49.7692307692308 0.0213430784642696 235,135,117 235,135,117
52.1378205128205 0.0804963409900665 0,0,0 0,0,0
52.1378205128205 0.106464222073555 25.8039215686274,14.8235294117647,12.8470588235294 25.8039215686274,14.8235294117647,12.8470588235294
52.1378205128205 0.11439548432827 51.6078431372549,29.6470588235294,25.6941176470588 51.6078431372549,29.6470588235294,25.6941176470588
52.1378205128205 0.0422885678708553 78.3333333333333,45,39 78.3333333333333,45,39
52.1378205128205 0.149507746100426 104.137254901961,59.8235294117647,51.8470588235294 104.137254901961,59.8235294117647,51.8470588235294
52.1378205128205 0.0772893205285072 130.862745098039,75.1764705882353,65.1529411764706 130.862745098039,75.1764705882353,65.1529411764706
52.1378205128205 0.118644796311855 156.666666666667,90,78 156.666666666667,90,78
52.1378205128205 0.0764889419078827 183.392156862745,105.352941176471,91.3058823529412 183.392156862745,105.352941176471,91.3058823529412
52.1378205128205 0.0492008626461029 209.196078431373,120.176470588235,104.152941176471 209.196078431373,120.176470588235,104.152941176471
52.1378205128205 0.0648493096232414 235,135,117 235,135,117
54.6217948717949 0.0774566456675529 0,0,0 0,0,0
54.6217948717949 0.125383287668228 25.8039215686274,14.8235294117647,12.8470588235294 25.8039215686274,14.8235294117647,12.8470588235294
54.6217948717949 0.140533328056335 51.6078431372549,29.6470588235294,25.6941176470588 51.6078431372549,29.6470588235294,25.6941176470588
54.6217948717949 0.167792648077011 78.3333333333333,45,39 78.3333333333333,45,39
54.6217948717949 0.068349651992321 104.137254901961,59.8235294117647,51.8470588235294 104.137254901961,59.8235294117647,51.8470588235294
54.6217948717949 0.0571270473301411 130.862745098039,75.1764705882353,65.1529411764706 130.862745098039,75.1764705882353,65.1529411764706
54.6217948717949 0.134534001350403 156.666666666667,90,78 156.666666666667,90,78
54.6217948717949 0.0864002481102943 183.392156862745,105.352941176471,91.3058823529412 183.392156862745,105.352941176471,91.3058823529412
54.6217948717949 0.0753994882106781 209.196078431373,120.176470588235,104.152941176471 209.196078431373,120.176470588235,104.152941176471
54.6217948717949 0.0598373003304005 235,135,117 235,135,117
57.2211538461538 0.0515536442399025 0,0,0 0,0,0
57.2211538461538 0.0686461105942726 25.8039215686274,14.8235294117647,12.8470588235294 25.8039215686274,14.8235294117647,12.8470588235294
57.2211538461538 0.100483745336533 51.6078431372549,29.6470588235294,25.6941176470588 51.6078431372549,29.6470588235294,25.6941176470588
57.2211538461538 0.0865549743175507 78.3333333333333,45,39 78.3333333333333,45,39
57.2211538461538 0.0600200407207012 104.137254901961,59.8235294117647,51.8470588235294 104.137254901961,59.8235294117647,51.8470588235294
57.2211538461538 0.036415446549654 130.862745098039,75.1764705882353,65.1529411764706 130.862745098039,75.1764705882353,65.1529411764706
57.2211538461538 0.0372415035963058 156.666666666667,90,78 156.666666666667,90,78
57.2211538461538 0.0452808775007725 183.392156862745,105.352941176471,91.3058823529412 183.392156862745,105.352941176471,91.3058823529412
57.2211538461538 0.0991490855813026 209.196078431373,120.176470588235,104.152941176471 209.196078431373,120.176470588235,104.152941176471
57.2211538461538 0.0101491622626781 235,135,117 235,135,117
59.9455128205128 0.143939718604088 0,0,0 0,0,0
59.9455128205128 0.0591123476624489 25.8039215686274,14.8235294117647,12.8470588235294 25.8039215686274,14.8235294117647,12.8470588235294
59.9455128205128 0.202153012156487 51.6078431372549,29.6470588235294,25.6941176470588 51.6078431372549,29.6470588235294,25.6941176470588
59.9455128205128 0.0693584382534027 78.3333333333333,45,39 78.3333333333333,45,39
59.9455128205128 0.0501086339354515 104.137254901961,59.8235294117647,51.8470588235294 104.137254901961,59.8235294117647,51.8470588235294
59.9455128205128 0.0273427311331034 130.862745098039,75.1764705882353,65.1529411764706 130.862745098039,75.1764705882353,65.1529411764706
59.9455128205128 0.141048192977905 156.666666666667,90,78 156.666666666667,90,78
59.9455128205128 0.0754390954971313 183.392156862745,105.352941176471,91.3058823529412 183.392156862745,105.352941176471,91.3058823529412
59.9455128205128 0.0333353839814663 209.196078431373,120.176470588235,104.152941176471 209.196078431373,120.176470588235,104.152941176471
59.9455128205128 0.0642862692475319 235,135,117 235,135,117
62.8012820512821 0.0407466702163219 0,0,0 0,0,0
62.8012820512821 0.045787263661623 25.8039215686274,14.8235294117647,12.8470588235294 25.8039215686274,14.8235294117647,12.8470588235294
62.8012820512821 0.034728717058897 51.6078431372549,29.6470588235294,25.6941176470588 51.6078431372549,29.6470588235294,25.6941176470588
62.8012820512821 0.0885457172989845 78.3333333333333,45,39 78.3333333333333,45,39
62.8012820512821 0.0272695533931255 104.137254901961,59.8235294117647,51.8470588235294 104.137254901961,59.8235294117647,51.8470588235294
62.8012820512821 0.0464835539460182 130.862745098039,75.1764705882353,65.1529411764706 130.862745098039,75.1764705882353,65.1529411764706
62.8012820512821 0.0209010522812605 156.666666666667,90,78 156.666666666667,90,78
62.8012820512821 0.0715994611382484 183.392156862745,105.352941176471,91.3058823529412 183.392156862745,105.352941176471,91.3058823529412
62.8012820512821 0.0113705527037382 209.196078431373,120.176470588235,104.152941176471 209.196078431373,120.176470588235,104.152941176471
62.8012820512821 0.0173634625971317 235,135,117 235,135,117
65.7916666666667 0.111600190401077 0,0,0 0,0,0
65.7916666666667 0.091223232448101 25.8039215686274,14.8235294117647,12.8470588235294 25.8039215686274,14.8235294117647,12.8470588235294
65.7916666666667 0.0637151226401329 51.6078431372549,29.6470588235294,25.6941176470588 51.6078431372549,29.6470588235294,25.6941176470588
65.7916666666667 0.0900863036513329 78.3333333333333,45,39 78.3333333333333,45,39
65.7916666666667 0.0330133438110352 104.137254901961,59.8235294117647,51.8470588235294 104.137254901961,59.8235294117647,51.8470588235294
65.7916666666667 0.0520493946969509 130.862745098039,75.1764705882353,65.1529411764706 130.862745098039,75.1764705882353,65.1529411764706
65.7916666666667 0.0569294653832912 156.666666666667,90,78 156.666666666667,90,78
65.7916666666667 0.0382050834596157 183.392156862745,105.352941176471,91.3058823529412 183.392156862745,105.352941176471,91.3058823529412
65.7916666666667 0.0557978786528111 209.196078431373,120.176470588235,104.152941176471 209.196078431373,120.176470588235,104.152941176471
65.7916666666667 0.0111792730167508 235,135,117 235,135,117
68.9230769230769 0.0401521660387516 0,0,0 0,0,0
68.9230769230769 0.146928250789642 25.8039215686274,14.8235294117647,12.8470588235294 25.8039215686274,14.8235294117647,12.8470588235294
68.9230769230769 0.0615935809910297 51.6078431372549,29.6470588235294,25.6941176470588 51.6078431372549,29.6470588235294,25.6941176470588
68.9230769230769 0.0334985330700874 78.3333333333333,45,39 78.3333333333333,45,39
68.9230769230769 0.0353465899825096 104.137254901961,59.8235294117647,51.8470588235294 104.137254901961,59.8235294117647,51.8470588235294
68.9230769230769 0.0955769717693329 130.862745098039,75.1764705882353,65.1529411764706 130.862745098039,75.1764705882353,65.1529411764706
68.9230769230769 0.0461437478661537 156.666666666667,90,78 156.666666666667,90,78
68.9230769230769 0.0568266697227955 183.392156862745,105.352941176471,91.3058823529412 183.392156862745,105.352941176471,91.3058823529412
68.9230769230769 0.0431615747511387 209.196078431373,120.176470588235,104.152941176471 209.196078431373,120.176470588235,104.152941176471
68.9230769230769 0.0128111308440566 235,135,117 235,135,117
72.2051282051282 0.0376274436712265 0,0,0 0,0,0
72.2051282051282 0.0410486347973347 25.8039215686274,14.8235294117647,12.8470588235294 25.8039215686274,14.8235294117647,12.8470588235294
72.2051282051282 0.0303088519722223 51.6078431372549,29.6470588235294,25.6941176470588 51.6078431372549,29.6470588235294,25.6941176470588
72.2051282051282 0.0483927689492702 78.3333333333333,45,39 78.3333333333333,45,39
72.2051282051282 0.0579156242311001 104.137254901961,59.8235294117647,51.8470588235294 104.137254901961,59.8235294117647,51.8470588235294
72.2051282051282 0.0739426836371422 130.862745098039,75.1764705882353,65.1529411764706 130.862745098039,75.1764705882353,65.1529411764706
72.2051282051282 0.0441956296563148 156.666666666667,90,78 156.666666666667,90,78
72.2051282051282 0.0138805769383907 183.392156862745,105.352941176471,91.3058823529412 183.392156862745,105.352941176471,91.3058823529412
72.2051282051282 0.0229110848158598 209.196078431373,120.176470588235,104.152941176471 209.196078431373,120.176470588235,104.152941176471
72.2051282051282 0.0215059034526348 235,135,117 235,135,117
75.6442307692308 0.0416424721479416 0,0,0 0,0,0
75.6442307692308 0.110561817884445 25.8039215686274,14.8235294117647,12.8470588235294 25.8039215686274,14.8235294117647,12.8470588235294
75.6442307692308 0.0273461546748877 51.6078431372549,29.6470588235294,25.6941176470588 51.6078431372549,29.6470588235294,25.6941176470588
75.6442307692308 0.0279228612780571 78.3333333333333,45,39 78.3333333333333,45,39
75.6442307692308 0.0270750112831593 104.137254901961,59.8235294117647,51.8470588235294 104.137254901961,59.8235294117647,51.8470588235294
75.6442307692308 0.0827696323394775 130.862745098039,75.1764705882353,65.1529411764706 130.862745098039,75.1764705882353,65.1529411764706
75.6442307692308 0.043198000639677 156.666666666667,90,78 156.666666666667,90,78
75.6442307692308 0.0489821918308735 183.392156862745,105.352941176471,91.3058823529412 183.392156862745,105.352941176471,91.3058823529412
75.6442307692308 0.0161643028259277 209.196078431373,120.176470588235,104.152941176471 209.196078431373,120.176470588235,104.152941176471
75.6442307692308 0.0331812165677547 235,135,117 235,135,117
79.2467948717949 0.0331645645201206 0,0,0 0,0,0
79.2467948717949 0.0627862364053726 25.8039215686274,14.8235294117647,12.8470588235294 25.8039215686274,14.8235294117647,12.8470588235294
79.2467948717949 0.0783738866448402 51.6078431372549,29.6470588235294,25.6941176470588 51.6078431372549,29.6470588235294,25.6941176470588
79.2467948717949 0.0606013387441635 78.3333333333333,45,39 78.3333333333333,45,39
79.2467948717949 0.0259740501642227 104.137254901961,59.8235294117647,51.8470588235294 104.137254901961,59.8235294117647,51.8470588235294
79.2467948717949 0.0151487151160836 130.862745098039,75.1764705882353,65.1529411764706 130.862745098039,75.1764705882353,65.1529411764706
79.2467948717949 0.0449281744658947 156.666666666667,90,78 156.666666666667,90,78
79.2467948717949 0.0170947462320328 183.392156862745,105.352941176471,91.3058823529412 183.392156862745,105.352941176471,91.3058823529412
79.2467948717949 0.0161969531327486 209.196078431373,120.176470588235,104.152941176471 209.196078431373,120.176470588235,104.152941176471
79.2467948717949 0.00977244228124619 235,135,117 235,135,117
83.0192307692308 0.0545338839292526 0,0,0 0,0,0
83.0192307692308 0.0130007080733776 25.8039215686274,14.8235294117647,12.8470588235294 25.8039215686274,14.8235294117647,12.8470588235294
83.0192307692308 0.0465932562947273 51.6078431372549,29.6470588235294,25.6941176470588 51.6078431372549,29.6470588235294,25.6941176470588
83.0192307692308 0.0189707912504673 78.3333333333333,45,39 78.3333333333333,45,39
83.0192307692308 0.0228762924671173 104.137254901961,59.8235294117647,51.8470588235294 104.137254901961,59.8235294117647,51.8470588235294
83.0192307692308 0.0304686296731234 130.862745098039,75.1764705882353,65.1529411764706 130.862745098039,75.1764705882353,65.1529411764706
83.0192307692308 0.039107047021389 156.666666666667,90,78 156.666666666667,90,78
83.0192307692308 0.0128621403127909 183.392156862745,105.352941176471,91.3058823529412 183.392156862745,105.352941176471,91.3058823529412
83.0192307692308 0.0182233229279518 209.196078431373,120.176470588235,104.152941176471 209.196078431373,120.176470588235,104.152941176471
83.0192307692308 0.00825722981244326 235,135,117 235,135,117
86.974358974359 0.0498495772480965 0,0,0 0,0,0
86.974358974359 0.039225846529007 25.8039215686274,14.8235294117647,12.8470588235294 25.8039215686274,14.8235294117647,12.8470588235294
86.974358974359 0.0489772744476795 51.6078431372549,29.6470588235294,25.6941176470588 51.6078431372549,29.6470588235294,25.6941176470588
86.974358974359 0.0757955461740494 78.3333333333333,45,39 78.3333333333333,45,39
86.974358974359 0.0306103881448507 104.137254901961,59.8235294117647,51.8470588235294 104.137254901961,59.8235294117647,51.8470588235294
86.974358974359 0.0237225517630577 130.862745098039,75.1764705882353,65.1529411764706 130.862745098039,75.1764705882353,65.1529411764706
86.974358974359 0.0197017565369606 156.666666666667,90,78 156.666666666667,90,78
86.974358974359 0.0649831593036652 183.392156862745,105.352941176471,91.3058823529412 183.392156862745,105.352941176471,91.3058823529412
86.974358974359 0.0199153255671263 209.196078431373,120.176470588235,104.152941176471 209.196078431373,120.176470588235,104.152941176471
86.974358974359 0.00862924661487341 235,135,117 235,135,117
91.1153846153846 0.0731098502874374 0,0,0 0,0,0
91.1153846153846 0.0323996990919113 25.8039215686274,14.8235294117647,12.8470588235294 25.8039215686274,14.8235294117647,12.8470588235294
91.1153846153846 0.0321961976587772 51.6078431372549,29.6470588235294,25.6941176470588 51.6078431372549,29.6470588235294,25.6941176470588
91.1153846153846 0.0376484394073486 78.3333333333333,45,39 78.3333333333333,45,39
91.1153846153846 0.0263667125254869 104.137254901961,59.8235294117647,51.8470588235294 104.137254901961,59.8235294117647,51.8470588235294
91.1153846153846 0.0418588779866695 130.862745098039,75.1764705882353,65.1529411764706 130.862745098039,75.1764705882353,65.1529411764706
91.1153846153846 0.038524117320776 156.666666666667,90,78 156.666666666667,90,78
91.1153846153846 0.00932418648153543 183.392156862745,105.352941176471,91.3058823529412 183.392156862745,105.352941176471,91.3058823529412
91.1153846153846 0.0101931672543287 209.196078431373,120.176470588235,104.152941176471 209.196078431373,120.176470588235,104.152941176471
91.1153846153846 0.00892436970025301 235,135,117 235,135,117
95.4519230769231 0.0384651608765125 0,0,0 0,0,0
95.4519230769231 0.0212954841554165 25.8039215686274,14.8235294117647,12.8470588235294 25.8039215686274,14.8235294117647,12.8470588235294
95.4519230769231 0.0639208257198334 51.6078431372549,29.6470588235294,25.6941176470588 51.6078431372549,29.6470588235294,25.6941176470588
95.4519230769231 0.0118824178352952 78.3333333333333,45,39 78.3333333333333,45,39
95.4519230769231 0.0212286729365587 104.137254901961,59.8235294117647,51.8470588235294 104.137254901961,59.8235294117647,51.8470588235294
95.4519230769231 0.0145912701264024 130.862745098039,75.1764705882353,65.1529411764706 130.862745098039,75.1764705882353,65.1529411764706
95.4519230769231 0.087324395775795 156.666666666667,90,78 156.666666666667,90,78
95.4519230769231 0.0737019032239914 183.392156862745,105.352941176471,91.3058823529412 183.392156862745,105.352941176471,91.3058823529412
95.4519230769231 0.0112798046320677 209.196078431373,120.176470588235,104.152941176471 209.196078431373,120.176470588235,104.152941176471
95.4519230769231 0.0109028639271855 235,135,117 235,135,117
100 0.0364880599081516 0,0,0 0,0,0
100 0.0200623665004969 25.8039215686274,14.8235294117647,12.8470588235294 25.8039215686274,14.8235294117647,12.8470588235294
100 0.0419906713068485 51.6078431372549,29.6470588235294,25.6941176470588 51.6078431372549,29.6470588235294,25.6941176470588
100 0.034723736345768 78.3333333333333,45,39 78.3333333333333,45,39
100 0.0528703331947327 104.137254901961,59.8235294117647,51.8470588235294 104.137254901961,59.8235294117647,51.8470588235294
100 0.029904393479228 130.862745098039,75.1764705882353,65.1529411764706 130.862745098039,75.1764705882353,65.1529411764706
100 0.0263231229037046 156.666666666667,90,78 156.666666666667,90,78
100 0.0245629977434874 183.392156862745,105.352941176471,91.3058823529412 183.392156862745,105.352941176471,91.3058823529412
100 0.00935054570436478 209.196078431373,120.176470588235,104.152941176471 209.196078431373,120.176470588235,104.152941176471
100 0.00808566436171532 235,135,117 235,135,117
};
\end{axis}

\end{tikzpicture}

  \tikzexternaldisable
  \caption{ \textbf{Directional curvature SNRs:} Curvature SNRs along each of
    the mini-batch \ggn{}'s top-$C$ eigenvectors during training of the
    \threecthreed network on \cifarten with \sgd{}. At fixed epoch, the SNR for
    the most curved direction is shown in
    {\protect\tikz{\protect\draw[white,fill={light_red},line width=0mm] (0,0)
        circle (.8ex);}} and the SNR for the direction with the smallest
    curvature is shown in {\protect\tikz{\protect\draw [white,fill=black] (0,0)
        circle (.8ex);}}. } \label{vivit::fig:directional_derivatives}
\end{figure}

A unique feature of \vivit{}'s quantities is that they provide a notion of
\textit{curvature uncertainty} through \textit{per-sample} first- and
second-order directional derivatives (\Cref{vivit::eq:gammas-lambdas}). To
quantify noise in these derivatives, we compute their signal-to-noise ratios
(SNRs). For each direction $\ve_k$, the SNR is given by the squared empirical
mean divided by the empirical variance of the $N$ mini-batch samples
$\{\gamma_{n,k}\}_{n=1}^N$ and $\{\lambda_{n,k}\}_{n=1}^N$, respectively.

\Cref{vivit::fig:directional_derivatives} shows curvature SNRs during training
the \threecthreed network on \cifarten with \sgd. The curvature signal along the
top-$C$ eigenvectors decreases from $\text{SNR} > 1$ by two orders of magnitude.
In comparison, the directional gradients do not exhibit such a pattern (see
\Cref{vivit::sec:directional_derivatives}). Results for the other test cases can
be found in \Cref{vivit::sec:directional_derivatives}.

In this section, we have given a glimpse of the \textit{very rich} quantities
that can be efficiently computed under \vivit's concept. In
\Cref{vivit::sec:use_cases}, we discuss their practical use---curvature
uncertainty in particular.

%%% Local Variables:
%%% mode: latex
%%% TeX-master: "../thesis"
%%% End:


%%% Local Variables:
%%% mode: latex
%%% TeX-master: "../thesis"
%%% End:
