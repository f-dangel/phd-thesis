One design principle of \cockpit is its easy integration with conventional
\pytorch training loops. \Cref{fig:cockpit::Codeblock} shows a working example
of a standard training loop with \cockpit. More examples and tutorials are
described in the documentation. \cockpit's syntax is inspired by \backpack: it
can be used interchangeably with the library responsible for most back-end
computations. Changes to the code are straightforward:

\lstinputlisting[ language=Python,
caption={\textbf{Complete training loop with
    \cockpittitle in \pytorch.} Line changes are highlighted in blue
  (\textcolor{maincolor!25}{\ding{122}}).},
label=fig:cockpit::Codeblock,
linebackgroundcolor={\ifnum\value{lstnumber}=5\color{maincolor!25}\fi\ifnum\value{lstnumber}=6\color{maincolor!25}\fi\ifnum\value{lstnumber}=7\color{maincolor!25}\fi\ifnum\value{lstnumber}=10\color{maincolor!25}\fi\ifnum\value{lstnumber}=11\color{maincolor!25}\fi\ifnum\value{lstnumber}=12\color{maincolor!25}\fi\ifnum\value{lstnumber}=15\color{maincolor!25}\fi\ifnum\value{lstnumber}=16\color{maincolor!25}\fi\ifnum\value{lstnumber}=26\color{maincolor!25}\fi\ifnum\value{lstnumber}=27\color{maincolor!25}\fi\ifnum\value{lstnumber}=28\color{maincolor!25}\fi\ifnum\value{lstnumber}=29\color{maincolor!25}\fi\ifnum\value{lstnumber}=30\color{maincolor!25}\fi\ifnum\value{lstnumber}=31\color{maincolor!25}\fi\ifnum\value{lstnumber}=32\color{maincolor!25}\fi\ifnum\value{lstnumber}=33\color{maincolor!25}\fi\ifnum\value{lstnumber}=34\color{maincolor!25}\fi\ifnum\value{lstnumber}=35\color{maincolor!25}\fi\ifnum\value{lstnumber}=36\color{maincolor!25}\fi\ifnum\value{lstnumber}=37\color{maincolor!25}\fi\ifnum\value{lstnumber}=45\color{maincolor!25}\fi\ifnum\value{lstnumber}=46\color{maincolor!25}\fi},
]{../repos/cockpit-paper/tex/appendix/example_code.py}

\begin{itemize}
\item \textbf{Importing} (\textit{Lines 5, 6} and \textit{7}): besides importing
  \cockpit we also need to import \backpack which is required for extending
  (parts of) the model (see next step).
\item \textbf{Extending} (\textit{Lines 10} and \textit{11}): when defining the
  model and the loss function, we need to \emph{extend} both of them using
  \backpack. This is as trivial as wrapping them in the \inlinecode{extend()}
  function provided by \backpack and lets \backpack know that additional
  quantities (such as the individual gradients) should be computed for them.
  Note, that while applying \backpack is easy, it currently does not support all
  possible model architectures and layer types. Specifically, \emph{batch norm}
  layers are not supported since using them results in ill-defined individual
  gradients.
\item \textbf{Individual losses} (\textit{Line 12}): for the \inlinecode{Alpha}
  quantity, \cockpit also requires the individual loss values (to estimate the
  variance of the loss estimate). This can be computed cheaply but is not
  usually part of a conventional training loop. Creating this loss is done
  analogously to creating any other loss, with the only exception of setting
  \inlinecode{reduction="none"}. Since we don't differentiate this loss, we don't
  need to extend it.
\item \textbf{Cockpit configuration} (\textit{Line 15} and \textit{16}):
  Initializing the \cockpit requires passing them (extended) model parameters as
  well as a list of quantities that should be tracked.
  \Cref{cockpit::tab:overview-quantities} provides an overview of all possible
  quantities. In this example, we use one of the pre-defined configurations
  offered by \cockpit. Separately, we initialize the plotting part of \cockpit.
  We deliberately detached the visualization from the tracking to allow greater
  flexibility.
\item \textbf{Quantity computation} (\textit{Line 26} to \textit{37}):
  Performing the training is very similar to a regular training loop, with the
  only difference being that the backward pass should be surrounded by the
  \cockpit context (\inlinecode{with cockpit():}). Additionally to the
  \inlinecode{global\_step} we also pass a few additional information to the
  \cockpit that are computed anyway and can be re-used, such as the batch size,
  the individual losses, or the optimizer itself. After the backward pass (when
  the context is left) all \cockpit quantities are automatically computed.
\item \textbf{Logging \& visualizing} (\textit{Line 45} and \textit{46}): at any
  point during training---here at the end---we can write all quantities to a log
  file. We can use this log file, or alternatively the \cockpit directly, to
  visualize all quantities in a status screen similar to
  \Cref{cockpit::fig:showcase}.
\end{itemize}

%%% Local Variables:
%%% mode: latex
%%% TeX-master: "../thesis"
%%% End:
