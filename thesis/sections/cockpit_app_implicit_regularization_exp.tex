\subsection{Detecting Implicit Regularization of The
  Optimizer}\label{cockpit::app:implicit_regularization_exp}

In non-convex optimization, optimizers can converge to local minima with
different properties. Here, we illustrate this by investigating the effect of
sub-sampling noise on a simple task from
\cite{mulayoff2020unique,ginsburg2020regularization}.

We generate synthetic data $\sD = \{(x_n, y_n) \in \sR \times \sR
\}_{n=1}^{N=100}$ for a regression task with $x \sim \gN(\giventhat{x}{0, 1})$
with noisy observations $y = 1.4 x + \epsilon$ where $\epsilon \sim
\gN(\giventhat{\epsilon}{0,1})$. The model is a scalar net with parameters
$\vtheta = \begin{pmatrix} w_1 & w_2 \end{pmatrix}^\top \in \sR^2$, initialized
at $\vtheta_0 = \begin{pmatrix} 0.1 & 1.7 \end{pmatrix}^\top$, that produces
predictions $f_{\vtheta}(x) = w_2 w_1 x$. We seek to minimize the mean squared
error
\begin{equation*}
  \gL_\sD(\vtheta) = \frac{1}{N} \sum_{n=1}^{N} \left( f_{\vtheta}(x_n) - y_n \right)^2
\end{equation*}
and compare \sgd ($|\sB|=95$) with \gd ($|\sB|= N =100$) at a learning rate of
$0.1$ (see \Cref{cockpit::fig:implicit-regularization}).

We observe that the loss of both \sgd and \gd is almost identical. Using a noisy
gradient regularizes the Hessian's maximum eigenvalue though. It decreases in
later stages where the loss curve suggests that training has converged. This
regularization effect constitutes an important phenomenon that cannot be
observed by monitoring only the loss.

\pgfkeys{/pgfplots/regularizationdefault/.style={
    width=1.05\linewidth,
    height=0.8\linewidth,
    every axis plot/.append style={line width = 1.5pt},
    every axis background/.style={fill=white},
    ymajorticks=true,
    xmajorticks=true,
    tick pos = left,
    ylabel near ticks,
    xlabel near ticks,
    xtick align = inside,
    ytick align = inside,
    legend cell align = left,
    legend columns = 1,
    legend pos = north east,
    legend style = {
      fill opacity = 0.7,
      text opacity = 1,
      font = \footnotesize,
    },
    colorbar style = {font = \footnotesize},
    title style = {font = \footnotesize, inner ysep = 0ex},
    xticklabel style = {font = \footnotesize, inner xsep = 0ex},
    xlabel style = {font = \footnotesize},
    axis line style = {black},
    yticklabel style = {font = \footnotesize, inner ysep = -4ex},
    ylabel style = {font = \footnotesize},
    grid = major,
    grid style = {dashed}
  }
}

\begin{figure}[!th]
  \centering
	\begin{subfigure}[t]{0.495\textwidth}
    \centering
		\pgfkeys{/pgfplots/zmystyle/.style={regularizationdefault, ymin=0.6, ymax=1.1}}
    \tikzexternalenable
		% This file was created by tikzplotlib v0.9.7.
\begin{tikzpicture}

\definecolor{color0}{rgb}{0.12156862745098,0.466666666666667,0.705882352941177}
\definecolor{color1}{rgb}{1,0.498039215686275,0.0549019607843137}

\begin{axis}[
axis line style={white},
legend style={fill opacity=0.8, draw opacity=1, text opacity=1, draw=white!80!black},
log basis x={10},
tick align=outside,
xlabel={Iteration},
xmajorticks=false,
xmin=0.563970595937194, xmax=167345.603972783,
xmode=log,
xtick style={color=white!15!black},
ylabel={Mini-Batch Loss},
ymajorticks=false,
ymin=0.614918631315231, ymax=1.72726913094521,
zmystyle
]
\addplot [, color0]
table {%
0 1.65519058704376
1 1.01682019233704
2 0.780193150043488
3 0.775982201099396
4 0.737731337547302
5 0.771629452705383
6 0.738641858100891
7 0.745266854763031
8 0.772922933101654
9 0.748066663742065
10 0.77786260843277
11 0.769368886947632
12 0.775993466377258
13 0.769854426383972
14 0.780099034309387
15 0.732372939586639
16 0.76972508430481
17 0.77314555644989
18 0.787652790546417
19 0.731954038143158
20 0.753742933273315
21 0.760406374931335
22 0.772047460079193
24 0.752689599990845
25 0.768256425857544
27 0.772502541542053
28 0.745130240917206
30 0.73753297328949
32 0.775065541267395
34 0.7575803399086
36 0.748904526233673
38 0.78914475440979
40 0.774977564811707
42 0.770744025707245
45 0.771463930606842
48 0.724553644657135
51 0.750621318817139
54 0.786653280258179
57 0.740645587444305
60 0.780794203281403
64 0.765990436077118
68 0.772936582565308
72 0.764917433261871
76 0.75068473815918
81 0.763052582740784
86 0.766409575939178
91 0.779118478298187
96 0.779287576675415
102 0.694868266582489
108 0.753131628036499
114 0.745236694812775
121 0.769624173641205
128 0.733922481536865
136 0.763625741004944
144 0.771767735481262
153 0.778103351593018
162 0.76782751083374
172 0.770959913730621
182 0.778098523616791
193 0.747863173484802
204 0.767393231391907
217 0.747031450271606
230 0.761284053325653
243 0.788995981216431
258 0.78480863571167
273 0.725325763225555
289 0.788911044597626
307 0.757820010185242
325 0.701482534408569
344 0.786840677261353
365 0.75093948841095
387 0.793320834636688
410 0.733399331569672
434 0.790173828601837
460 0.789791882038116
488 0.772913932800293
517 0.747723639011383
547 0.767966389656067
580 0.769767463207245
615 0.777678191661835
651 0.732484042644501
690 0.765304982662201
731 0.779313802719116
775 0.752284646034241
821 0.751649916172028
870 0.763818562030792
922 0.763033270835876
977 0.76042252779007
1035 0.794110774993896
1096 0.770817458629608
1162 0.719465613365173
1231 0.773622274398804
1304 0.777041494846344
1382 0.752636194229126
1464 0.746034681797028
1552 0.756577551364899
1644 0.710349500179291
1742 0.788471639156342
1846 0.749858379364014
1956 0.751055300235748
2072 0.77653980255127
2196 0.743179202079773
2327 0.777839779853821
2465 0.765402734279633
2612 0.784416019916534
2768 0.675363481044769
2933 0.767980337142944
3107 0.77206939458847
3292 0.761081337928772
3489 0.780567407608032
3696 0.759546518325806
3917 0.772010743618011
4150 0.731160163879395
4397 0.778384268283844
4659 0.792823255062103
4937 0.780224800109863
5231 0.716187834739685
5542 0.732295513153076
5872 0.757798075675964
6222 0.783280313014984
6593 0.774591028690338
6985 0.76332038640976
7401 0.73084568977356
7842 0.744604051113129
8309 0.77698802947998
8804 0.770139813423157
9329 0.7605100274086
9884 0.745865881443024
10473 0.772988736629486
11097 0.784264206886292
11758 0.73901504278183
12458 0.665480017662048
13200 0.74626225233078
13987 0.765305399894714
14820 0.788454294204712
15702 0.783531785011292
16638 0.790920555591583
17629 0.723659753799438
18679 0.79127299785614
19791 0.776141047477722
20970 0.772833287715912
22219 0.73369562625885
23542 0.771220922470093
24945 0.785363912582397
26430 0.72164660692215
28005 0.788875877857208
29673 0.772274553775787
31440 0.764169812202454
33312 0.723604738712311
35297 0.758412897586823
37399 0.769568562507629
39626 0.782877564430237
41987 0.774317562580109
44487 0.74730920791626
47137 0.759216368198395
49945 0.75877833366394
52919 0.77678370475769
56071 0.770803034305573
59411 0.757794678211212
62949 0.775205314159393
66699 0.787786245346069
70671 0.761527419090271
74881 0.740764439105988
79340 0.766364872455597
84066 0.771571159362793
89073 0.774646580219269
94378 0.766631364822388
};
\addlegendentry{SGD}
\addplot [, color1]
table {%
0 1.67670774459839
1 1.00288474559784
2 0.814482688903809
3 0.769742786884308
4 0.761073589324951
5 0.759601593017578
6 0.759367525577545
7 0.759331345558167
8 0.759325861930847
9 0.759325087070465
10 0.759324848651886
11 0.759324848651886
12 0.759324908256531
13 0.759324848651886
14 0.759324848651886
15 0.759324848651886
16 0.759324848651886
17 0.759324848651886
18 0.759324908256531
19 0.759324848651886
20 0.759324848651886
21 0.759324848651886
22 0.759324848651886
24 0.759324789047241
25 0.759324848651886
27 0.759324848651886
28 0.759324848651886
30 0.759324848651886
32 0.759324848651886
34 0.759324848651886
36 0.759324908256531
38 0.759324848651886
40 0.759324848651886
42 0.759324848651886
45 0.759324908256531
48 0.759324848651886
51 0.759324848651886
54 0.759324848651886
57 0.759324848651886
60 0.759324848651886
64 0.759324848651886
68 0.759324848651886
72 0.759324848651886
76 0.759324848651886
81 0.759324908256531
86 0.759324789047241
91 0.759324848651886
96 0.759324908256531
102 0.759324848651886
108 0.759324848651886
114 0.759324908256531
121 0.759324848651886
128 0.759324848651886
136 0.759324789047241
144 0.759324908256531
153 0.759324848651886
162 0.759324848651886
172 0.759324908256531
182 0.759324908256531
193 0.759324848651886
204 0.759324908256531
217 0.759324908256531
230 0.759324908256531
243 0.759324848651886
258 0.759324848651886
273 0.759324908256531
289 0.759324848651886
307 0.759324848651886
325 0.759324848651886
344 0.759324789047241
365 0.759324848651886
387 0.759324848651886
410 0.759324908256531
434 0.759324848651886
460 0.759324848651886
488 0.759324848651886
517 0.759324848651886
547 0.759324848651886
580 0.759324908256531
615 0.759324908256531
651 0.759324848651886
690 0.759324848651886
731 0.759324848651886
775 0.759324789047241
821 0.759324848651886
870 0.759324848651886
922 0.759324848651886
977 0.759324848651886
1035 0.759324908256531
1096 0.759324908256531
1162 0.759324789047241
1231 0.759324848651886
1304 0.759324848651886
1382 0.759324848651886
1464 0.759324848651886
1552 0.759324848651886
1644 0.759324848651886
1742 0.759324908256531
1846 0.759324848651886
1956 0.759324908256531
2072 0.759324908256531
2196 0.759324908256531
2327 0.759324789047241
2465 0.759324908256531
2612 0.759324848651886
2768 0.759324848651886
2933 0.759324848651886
3107 0.759324848651886
3292 0.759324908256531
3489 0.759324908256531
3696 0.759324908256531
3917 0.759324848651886
4150 0.759324848651886
4397 0.759324848651886
4659 0.759324848651886
4937 0.759324848651886
5231 0.759324848651886
5542 0.759324848651886
5872 0.759324908256531
6222 0.759324908256531
6593 0.759324848651886
6985 0.759324908256531
7401 0.759324908256531
7842 0.759324848651886
8309 0.759324848651886
8804 0.759324908256531
9329 0.759324848651886
9884 0.759324908256531
10473 0.759324848651886
11097 0.759324908256531
11758 0.759324848651886
12458 0.759324848651886
13200 0.759324848651886
13987 0.759324908256531
14820 0.759324908256531
15702 0.759324848651886
16638 0.759324848651886
17629 0.759324848651886
18679 0.759324908256531
19791 0.759324908256531
20970 0.759324848651886
22219 0.759324848651886
23542 0.759324848651886
24945 0.759324848651886
26430 0.759324908256531
28005 0.759324848651886
29673 0.759324848651886
31440 0.759324789047241
33312 0.759324848651886
35297 0.759324848651886
37399 0.759324789047241
39626 0.759324908256531
41987 0.759324848651886
44487 0.759324848651886
47137 0.759324848651886
49945 0.759324848651886
52919 0.759324848651886
56071 0.759324848651886
59411 0.759324908256531
62949 0.759324789047241
66699 0.759324848651886
70671 0.759324848651886
74881 0.759324848651886
79340 0.759324848651886
84066 0.759324848651886
89073 0.759324848651886
94378 0.759324908256531
};
\addlegendentry{GD}
\end{axis}

\end{tikzpicture}

    \tikzexternaldisable
	\end{subfigure}
	\hfill
	\begin{subfigure}[t]{0.495\textwidth}
    \centering
		\pgfkeys{/pgfplots/zmystyle/.style={regularizationdefault,
        legend style = {
          opacity = 0,
          fill opacity = 0,
          text opacity = 0,
        },
        ylabel = {Max.\,Hessian eigenvalue}
      }}
    \tikzexternalenable
		% This file was created by tikzplotlib v0.9.7.
\begin{tikzpicture}

\definecolor{color0}{rgb}{0.12156862745098,0.466666666666667,0.705882352941177}
\definecolor{color1}{rgb}{1,0.498039215686275,0.0549019607843137}

\begin{axis}[
axis line style={white},
legend style={fill opacity=0.8, draw opacity=1, text opacity=1, draw=white!80!black},
log basis x={10},
tick align=outside,
xlabel={Iteration},
xmajorticks=false,
xmin=0.563970595937194, xmax=167345.603972783,
xmode=log,
xtick style={color=white!15!black},
ylabel={Maximum Hessian eigenvalue},
ymajorticks=false,
ymin=3.85510742664337, ymax=6.68519914150238,
zmystyle
]
\addplot [, color0]
table {%
0 5.01858520507812
1 4.91631031036377
2 5.16143560409546
3 5.60115337371826
4 5.79127502441406
5 6.0561056137085
6 6.26061916351318
7 6.19245529174805
8 6.1329493522644
9 6.05357074737549
10 5.97355842590332
11 6.05519437789917
12 6.31369209289551
13 6.22404289245605
14 6.33904695510864
15 5.68460321426392
16 6.16733026504517
17 6.17319679260254
18 6.14776849746704
19 6.2695107460022
20 6.03493785858154
21 6.15226316452026
22 6.19907712936401
24 5.59337615966797
25 6.36163806915283
27 6.35177993774414
28 6.22747230529785
30 5.91255331039429
32 6.19828081130981
34 6.39521312713623
36 6.0371265411377
38 6.43193244934082
40 6.14503479003906
42 6.25523281097412
45 6.05292510986328
48 5.67981719970703
51 6.36472225189209
54 6.06869220733643
57 6.18863964080811
60 5.93778562545776
64 6.12580823898315
68 6.18363857269287
72 6.01109552383423
76 6.02523136138916
81 6.21799850463867
86 6.18191480636597
91 6.55655860900879
96 6.31975746154785
102 6.05267763137817
108 6.0778636932373
114 6.01467657089233
121 6.24698925018311
128 6.04640293121338
136 6.24112749099731
144 6.31688690185547
153 6.28657817840576
162 6.13792037963867
172 5.65208387374878
182 6.30600261688232
193 5.98742771148682
204 5.71659326553345
217 5.84029388427734
230 6.0068564414978
243 6.3081259727478
258 6.39043521881104
273 6.14346885681152
289 6.18961191177368
307 6.13053369522095
325 6.17688846588135
344 5.90940189361572
365 5.89127063751221
387 5.98564434051514
410 6.25674438476562
434 5.93166160583496
460 6.01510143280029
488 6.29282379150391
517 6.3133373260498
547 5.89875364303589
580 5.78323602676392
615 6.22842979431152
651 6.12653160095215
690 6.13379669189453
731 6.08839702606201
775 6.19680023193359
821 6.05588006973267
870 6.21298170089722
922 6.0586724281311
977 5.70560264587402
1035 6.05281639099121
1096 6.12884950637817
1162 5.72979116439819
1231 5.75020885467529
1304 5.78489637374878
1382 6.24677515029907
1464 5.72618103027344
1552 6.16928005218506
1644 5.99916076660156
1742 6.31236791610718
1846 5.42526912689209
1956 6.01460218429565
2072 6.17887306213379
2196 5.93650770187378
2327 6.21088695526123
2465 5.92963600158691
2612 5.98927402496338
2768 6.04936790466309
2933 5.9457950592041
3107 6.15447568893433
3292 6.04442024230957
3489 5.88113212585449
3696 6.0181360244751
3917 5.82596063613892
4150 5.97462749481201
4397 5.84421348571777
4659 5.84129619598389
4937 5.80539751052856
5231 6.00247669219971
5542 5.8760838508606
5872 5.92244338989258
6222 5.69848346710205
6593 5.64358997344971
6985 5.6120719909668
7401 5.73087453842163
7842 5.67744302749634
8309 5.43394136428833
8804 5.7706823348999
9329 5.68305492401123
9884 5.22680282592773
10473 5.09199523925781
11097 5.7391529083252
11758 5.49506330490112
12458 5.29003858566284
13200 5.21850109100342
13987 5.15451717376709
14820 5.25626420974731
15702 5.2664647102356
16638 5.09290647506714
17629 5.06191682815552
18679 5.22023582458496
19791 4.99110126495361
20970 5.13607454299927
22219 4.82361221313477
23542 4.96726989746094
24945 4.517502784729
26430 4.89909362792969
28005 4.69721984863281
29673 4.44070720672607
31440 4.71606779098511
33312 4.43321752548218
35297 4.52913951873779
37399 4.35525035858154
39626 4.48494052886963
41987 4.43455600738525
44487 4.4298300743103
47137 4.41794347763062
49945 4.16174602508545
52919 4.41815376281738
56071 4.30182790756226
59411 4.18179321289062
62949 4.38689374923706
66699 4.22925329208374
70671 4.26896953582764
74881 4.2410717010498
79340 4.02750778198242
84066 3.98374795913696
89073 4.10416793823242
94378 4.25959587097168
};
\addlegendentry{SGD}
\addplot [, color1]
table {%
0 5.19412231445312
1 4.89158582687378
2 5.36561346054077
3 5.75625705718994
4 5.95861196517944
5 6.04718065261841
6 6.08332633972168
7 6.09766292572021
8 6.10328483581543
9 6.10547971725464
10 6.10633707046509
11 6.10666942596436
12 6.10679912567139
13 6.106849193573
14 6.10686922073364
15 6.10687732696533
16 6.10687875747681
17 6.1068811416626
18 6.1068811416626
19 6.10688161849976
20 6.10688161849976
21 6.10688161849976
22 6.10688161849976
24 6.10688161849976
25 6.1068811416626
27 6.10688209533691
28 6.1068811416626
30 6.10688161849976
32 6.10688161849976
34 6.10688161849976
36 6.10688161849976
38 6.10688161849976
40 6.1068811416626
42 6.10688161849976
45 6.10688161849976
48 6.10688257217407
51 6.10688209533691
54 6.10688161849976
57 6.10688209533691
60 6.10688161849976
64 6.10688209533691
68 6.1068811416626
72 6.10688161849976
76 6.10688161849976
81 6.10688257217407
86 6.10688161849976
91 6.1068811416626
96 6.10688209533691
102 6.10688257217407
108 6.10688161849976
114 6.10688209533691
121 6.1068811416626
128 6.10688257217407
136 6.10688161849976
144 6.10688209533691
153 6.10688161849976
162 6.10688209533691
172 6.10688161849976
182 6.10688161849976
193 6.1068811416626
204 6.10688161849976
217 6.10688257217407
230 6.1068811416626
243 6.10688209533691
258 6.1068811416626
273 6.1068811416626
289 6.10688161849976
307 6.10688161849976
325 6.1068811416626
344 6.10688161849976
365 6.10688161849976
387 6.10688161849976
410 6.10688209533691
434 6.10688161849976
460 6.10688161849976
488 6.1068811416626
517 6.10688161849976
547 6.10688161849976
580 6.10688209533691
615 6.10688161849976
651 6.1068811416626
690 6.1068811416626
731 6.1068811416626
775 6.1068811416626
821 6.1068811416626
870 6.10688209533691
922 6.10688161849976
977 6.10688161849976
1035 6.10688161849976
1096 6.10688161849976
1162 6.10688161849976
1231 6.1068811416626
1304 6.10688209533691
1382 6.10688161849976
1464 6.10688161849976
1552 6.1068811416626
1644 6.1068811416626
1742 6.10688161849976
1846 6.10688209533691
1956 6.10688161849976
2072 6.1068811416626
2196 6.10688161849976
2327 6.1068811416626
2465 6.10688161849976
2612 6.10688161849976
2768 6.10688161849976
2933 6.10688209533691
3107 6.10688257217407
3292 6.10688161849976
3489 6.10688257217407
3696 6.10688161849976
3917 6.1068811416626
4150 6.10688209533691
4397 6.1068811416626
4659 6.10688161849976
4937 6.10688209533691
5231 6.10688161849976
5542 6.1068811416626
5872 6.10688161849976
6222 6.10688209533691
6593 6.1068811416626
6985 6.10688161849976
7401 6.10688209533691
7842 6.10688161849976
8309 6.10688209533691
8804 6.10688161849976
9329 6.10688161849976
9884 6.1068811416626
10473 6.1068811416626
11097 6.10688161849976
11758 6.10688161849976
12458 6.10688161849976
13200 6.1068811416626
13987 6.1068811416626
14820 6.1068811416626
15702 6.10688161849976
16638 6.10688161849976
17629 6.10688161849976
18679 6.10688161849976
19791 6.1068811416626
20970 6.1068811416626
22219 6.1068811416626
23542 6.10688209533691
24945 6.10688161849976
26430 6.10688209533691
28005 6.10688209533691
29673 6.1068811416626
31440 6.10688161849976
33312 6.10688209533691
35297 6.10688161849976
37399 6.10688161849976
39626 6.10688209533691
41987 6.10688161849976
44487 6.10688161849976
47137 6.10688161849976
49945 6.1068811416626
52919 6.10688209533691
56071 6.10688209533691
59411 6.10688161849976
62949 6.10688161849976
66699 6.10688161849976
70671 6.10688161849976
74881 6.10688161849976
79340 6.10688161849976
84066 6.10688161849976
89073 6.1068811416626
94378 6.10688161849976
};
\addlegendentry{GD}
\end{axis}

\end{tikzpicture}

    \tikzexternaldisable
	\end{subfigure}
  \vspace{1ex}
  \begin{subfigure}{1.0\linewidth}
    \centering
		\pgfkeys{/pgfplots/zmystyle/.style={regularizationdefault,
        width = 0.91\linewidth,
        height = 0.6695*0.91*\linewidth,
        legend style = {
          opacity = 0,
          fill opacity = 0,
          text opacity = 0,
        },
        xmajorticks=true,
        ymajorticks=true,
      }}
    \tikzexternalenable
    % This file was created by tikzplotlib v0.9.7.
\begin{tikzpicture}

\definecolor{color0}{rgb}{0.12156862745098,0.466666666666667,0.705882352941177}
\definecolor{color1}{rgb}{1,0.498039215686275,0.0549019607843137}

\begin{axis}[
axis line style={white},
colorbar,
colormap={mymap}{[1pt]
  rgb(0pt)=(1,1,0.850980392156863);
  rgb(1pt)=(0.929411764705882,0.972549019607843,0.694117647058824);
  rgb(2pt)=(0.780392156862745,0.913725490196078,0.705882352941177);
  rgb(3pt)=(0.498039215686275,0.803921568627451,0.733333333333333);
  rgb(4pt)=(0.254901960784314,0.713725490196078,0.768627450980392);
  rgb(5pt)=(0.113725490196078,0.568627450980392,0.752941176470588);
  rgb(6pt)=(0.133333333333333,0.368627450980392,0.658823529411765);
  rgb(7pt)=(0.145098039215686,0.203921568627451,0.580392156862745);
  rgb(8pt)=(0.0313725490196078,0.113725490196078,0.345098039215686)
},
legend style={fill opacity=0.8, draw opacity=1, text opacity=1, draw=white!80!black},
point meta max=33.3554496765137,
point meta min=0.759324848651886,
tick align=outside,
title={Loss landscape},
xlabel={\(\displaystyle \theta_1\)},
xmajorticks=false,
xmin=-1, xmax=2.5,
xtick style={color=white!15!black},
ylabel={\(\displaystyle \theta_2\)},
ymajorticks=false,
ymin=-0.5, ymax=3,
zmystyle
]
\addplot graphics [,xmin=-1, xmax=2.5, ymin=-0.5, ymax=3] {Trajectory-000.png};
\addplot [, color0]
table {%
0.100000001490116 1.70000004768372
0.390511572360992 1.7170889377594
0.546473205089569 1.75255870819092
0.601417779922485 1.76969122886658
0.639493048191071 1.78263092041016
0.645879983901978 1.78492212295532
0.667922377586365 1.79289829730988
0.669110357761383 1.79334080219269
0.682194948196411 1.79822278022766
0.674466907978058 1.79529094696045
0.663153052330017 1.79104053974152
0.673606514930725 1.79491102695465
0.675003290176392 1.79543519020081
0.678699016571045 1.79682457447052
0.685095965862274 1.79924082756042
0.687204122543335 1.80004358291626
0.691844761371613 1.80181527137756
0.671525537967682 1.79401326179504
0.675070106983185 1.79534006118774
0.67482715845108 1.79524874687195
0.678539335727692 1.79664409160614
0.685560762882233 1.7992959022522
0.67389988899231 1.79485297203064
0.677536308765411 1.79621696472168
0.700529396533966 1.80489003658295
0.682436764240265 1.79790055751801
0.681613385677338 1.7975879907608
0.677117347717285 1.79583013057709
0.692475318908691 1.80160808563232
0.680085301399231 1.79686343669891
0.690684139728546 1.80087828636169
0.689901888370514 1.80054640769958
0.68423718214035 1.79837930202484
0.682424306869507 1.79768741130829
0.675705194473267 1.79513609409332
0.675473153591156 1.79498684406281
0.698002457618713 1.80348515510559
0.677664995193481 1.79566705226898
0.671689748764038 1.79340803623199
0.670520484447479 1.79295492172241
0.678237795829773 1.79584860801697
0.672419190406799 1.79344725608826
0.685892820358276 1.79848670959473
0.680178642272949 1.7962874174118
0.685978591442108 1.79844427108765
0.694296002388 1.80159783363342
0.711118221282959 1.80797255039215
0.675868630409241 1.79425501823425
0.684127986431122 1.79733657836914
0.679848849773407 1.79566979408264
0.669373571872711 1.79165542125702
0.679888606071472 1.79551303386688
0.662135541439056 1.78879928588867
0.686899542808533 1.79779970645905
0.682054400444031 1.79579043388367
0.672400176525116 1.7919909954071
0.680885851383209 1.79514026641846
0.667153120040894 1.78990614414215
0.679344117641449 1.79442346096039
0.673126935958862 1.79201233386993
0.670550942420959 1.79096555709839
0.69197142124176 1.79880881309509
0.672725915908813 1.79119396209717
0.676092565059662 1.79214191436768
0.688663899898529 1.79669153690338
0.681666493415833 1.79389941692352
0.677907943725586 1.79229915142059
0.673436224460602 1.79007911682129
0.68997323513031 1.79620814323425
0.667576611042023 1.78752589225769
0.682157278060913 1.79277646541595
0.679268181324005 1.79123222827911
0.681079208850861 1.79162502288818
0.677574813365936 1.78988242149353
0.67927086353302 1.79004609584808
0.697462141513824 1.79662251472473
0.69139176607132 1.79365348815918
0.687649428844452 1.79169237613678
0.682355523109436 1.78918492794037
0.675151944160461 1.78578960895538
0.67628139257431 1.78551888465881
0.675973236560822 1.78479135036469
0.665274679660797 1.77992808818817
0.678472220897675 1.78403651714325
0.689973175525665 1.78784000873566
0.681166648864746 1.78333806991577
0.691523969173431 1.78662049770355
0.698261976242065 1.78783130645752
0.690599858760834 1.78375256061554
0.680019974708557 1.77813601493835
0.688908398151398 1.78057253360748
0.698104679584503 1.78271543979645
0.67931067943573 1.77420496940613
0.684651434421539 1.77530860900879
0.693451106548309 1.77726054191589
0.689227163791656 1.77445876598358
0.680094182491302 1.76916575431824
0.696024477481842 1.77381980419159
0.676648080348969 1.76468467712402
0.681325316429138 1.7649313211441
0.68862909078598 1.76631033420563
0.693156957626343 1.76604175567627
0.700364649295807 1.76705718040466
0.7011958360672 1.76529848575592
0.687162697315216 1.75772929191589
0.703916668891907 1.76169896125793
0.689925253391266 1.75330626964569
0.70111882686615 1.75461566448212
0.695941746234894 1.74926269054413
0.699183106422424 1.74723780155182
0.696930766105652 1.74242079257965
0.710143327713013 1.74268937110901
0.698858141899109 1.73455560207367
0.709864377975464 1.73439800739288
0.704060018062592 1.72810399532318
0.698992013931274 1.72063815593719
0.712297260761261 1.72095036506653
0.717435717582703 1.7179012298584
0.712529122829437 1.71006381511688
0.719913721084595 1.70747792720795
0.719109177589417 1.70160639286041
0.719201445579529 1.69564807415009
0.726583480834961 1.69265925884247
0.724139869213104 1.68445897102356
0.719537675380707 1.67515122890472
0.74134373664856 1.67590343952179
0.740000069141388 1.66538965702057
0.730468451976776 1.65302991867065
0.740104854106903 1.64824032783508
0.76548433303833 1.65183210372925
0.744733691215515 1.63146710395813
0.746873617172241 1.62325477600098
0.753918051719666 1.61548483371735
0.765423655509949 1.61043667793274
0.773338854312897 1.60266101360321
0.773339629173279 1.59142887592316
0.76962685585022 1.57759571075439
0.778478145599365 1.56945598125458
0.793064177036285 1.56236958503723
0.792321801185608 1.54660928249359
0.79296088218689 1.53167915344238
0.792017102241516 1.51534283161163
0.815773010253906 1.51425313949585
0.818939983844757 1.50040066242218
0.82924497127533 1.48936939239502
0.819665551185608 1.46620404720306
0.837881088256836 1.45742213726044
0.848906576633453 1.44540286064148
0.84745717048645 1.42533433437347
0.871433675289154 1.42067611217499
0.862388014793396 1.39525926113129
0.880145847797394 1.38803493976593
0.897489786148071 1.3785206079483
0.903302729129791 1.36190438270569
0.913372099399567 1.34719264507294
0.916217088699341 1.32893395423889
0.953232765197754 1.33372986316681
0.942984879016876 1.3056834936142
0.944588005542755 1.28611779212952
0.962793827056885 1.27936899662018
0.963907539844513 1.2592408657074
0.999159157276154 1.26485371589661
1.00393545627594 1.24771106243134
0.999233841896057 1.22480249404907
1.00601613521576 1.21015644073486
1.01214456558228 1.1964567899704
1.02476096153259 1.18976318836212
};
\addlegendentry{SGD}
\addplot [, color1]
table {%
0.100000001490116 1.70000004768372
0.396201580762863 1.71742367744446
0.5503870844841 1.7529935836792
0.625281095504761 1.77650809288025
0.658266365528107 1.78811800479889
0.671868860721588 1.79312551021576
0.677294850349426 1.79515862464905
0.679428040981293 1.79596340656281
0.680261731147766 1.79627883434296
0.680586755275726 1.79640197753906
0.680713355541229 1.79644989967346
0.680762708187103 1.79646861553192
0.680781900882721 1.79647588729858
0.680789351463318 1.79647874832153
0.680792272090912 1.79647982120514
0.680793404579163 1.7964802980423
0.680793821811676 1.79648041725159
0.68079400062561 1.79648053646088
0.680794060230255 1.79648053646088
0.6807941198349 1.79648053646088
0.6807941198349 1.79648053646088
0.6807941198349 1.79648053646088
0.6807941198349 1.79648053646088
0.6807941198349 1.79648053646088
0.6807941198349 1.79648053646088
0.6807941198349 1.79648053646088
0.6807941198349 1.79648053646088
0.6807941198349 1.79648053646088
0.6807941198349 1.79648053646088
0.6807941198349 1.79648053646088
0.6807941198349 1.79648053646088
0.6807941198349 1.79648053646088
0.6807941198349 1.79648053646088
0.6807941198349 1.79648053646088
0.6807941198349 1.79648053646088
0.6807941198349 1.79648053646088
0.6807941198349 1.79648053646088
0.6807941198349 1.79648053646088
0.6807941198349 1.79648053646088
0.6807941198349 1.79648053646088
0.6807941198349 1.79648053646088
0.6807941198349 1.79648053646088
0.6807941198349 1.79648053646088
0.6807941198349 1.79648053646088
0.6807941198349 1.79648053646088
0.6807941198349 1.79648053646088
0.6807941198349 1.79648053646088
0.6807941198349 1.79648053646088
0.6807941198349 1.79648053646088
0.6807941198349 1.79648053646088
0.6807941198349 1.79648053646088
0.6807941198349 1.79648053646088
0.6807941198349 1.79648053646088
0.6807941198349 1.79648053646088
0.6807941198349 1.79648053646088
0.6807941198349 1.79648053646088
0.6807941198349 1.79648053646088
0.6807941198349 1.79648053646088
0.6807941198349 1.79648053646088
0.6807941198349 1.79648053646088
0.6807941198349 1.79648053646088
0.6807941198349 1.79648053646088
0.6807941198349 1.79648053646088
0.6807941198349 1.79648053646088
0.6807941198349 1.79648053646088
0.6807941198349 1.79648053646088
0.6807941198349 1.79648053646088
0.6807941198349 1.79648053646088
0.6807941198349 1.79648053646088
0.6807941198349 1.79648053646088
0.6807941198349 1.79648053646088
0.6807941198349 1.79648053646088
0.6807941198349 1.79648053646088
0.6807941198349 1.79648053646088
0.6807941198349 1.79648053646088
0.6807941198349 1.79648053646088
0.6807941198349 1.79648053646088
0.6807941198349 1.79648053646088
0.6807941198349 1.79648053646088
0.6807941198349 1.79648053646088
0.6807941198349 1.79648053646088
0.6807941198349 1.79648053646088
0.6807941198349 1.79648053646088
0.6807941198349 1.79648053646088
0.6807941198349 1.79648053646088
0.6807941198349 1.79648053646088
0.6807941198349 1.79648053646088
0.6807941198349 1.79648053646088
0.6807941198349 1.79648053646088
0.6807941198349 1.79648053646088
0.6807941198349 1.79648053646088
0.6807941198349 1.79648053646088
0.6807941198349 1.79648053646088
0.6807941198349 1.79648053646088
0.6807941198349 1.79648053646088
0.6807941198349 1.79648053646088
0.6807941198349 1.79648053646088
0.6807941198349 1.79648053646088
0.6807941198349 1.79648053646088
0.6807941198349 1.79648053646088
0.6807941198349 1.79648053646088
0.6807941198349 1.79648053646088
0.6807941198349 1.79648053646088
0.6807941198349 1.79648053646088
0.6807941198349 1.79648053646088
0.6807941198349 1.79648053646088
0.6807941198349 1.79648053646088
0.6807941198349 1.79648053646088
0.6807941198349 1.79648053646088
0.6807941198349 1.79648053646088
0.6807941198349 1.79648053646088
0.6807941198349 1.79648053646088
0.6807941198349 1.79648053646088
0.6807941198349 1.79648053646088
0.6807941198349 1.79648053646088
0.6807941198349 1.79648053646088
0.6807941198349 1.79648053646088
0.6807941198349 1.79648053646088
0.6807941198349 1.79648053646088
0.6807941198349 1.79648053646088
0.6807941198349 1.79648053646088
0.6807941198349 1.79648053646088
0.6807941198349 1.79648053646088
0.6807941198349 1.79648053646088
0.6807941198349 1.79648053646088
0.6807941198349 1.79648053646088
0.6807941198349 1.79648053646088
0.6807941198349 1.79648053646088
0.6807941198349 1.79648053646088
0.6807941198349 1.79648053646088
0.6807941198349 1.79648053646088
0.6807941198349 1.79648053646088
0.6807941198349 1.79648053646088
0.6807941198349 1.79648053646088
0.6807941198349 1.79648053646088
0.6807941198349 1.79648053646088
0.6807941198349 1.79648053646088
0.6807941198349 1.79648053646088
0.6807941198349 1.79648053646088
0.6807941198349 1.79648053646088
0.6807941198349 1.79648053646088
0.6807941198349 1.79648053646088
0.6807941198349 1.79648053646088
0.6807941198349 1.79648053646088
0.6807941198349 1.79648053646088
0.6807941198349 1.79648053646088
0.6807941198349 1.79648053646088
0.6807941198349 1.79648053646088
0.6807941198349 1.79648053646088
0.6807941198349 1.79648053646088
0.6807941198349 1.79648053646088
0.6807941198349 1.79648053646088
0.6807941198349 1.79648053646088
0.6807941198349 1.79648053646088
0.6807941198349 1.79648053646088
0.6807941198349 1.79648053646088
0.6807941198349 1.79648053646088
0.6807941198349 1.79648053646088
0.6807941198349 1.79648053646088
0.6807941198349 1.79648053646088
0.6807941198349 1.79648053646088
0.6807941198349 1.79648053646088
0.6807941198349 1.79648053646088
0.6807941198349 1.79648053646088
0.6807941198349 1.79648053646088
0.6807941198349 1.79648053646088
0.6807941198349 1.79648053646088
};
\addlegendentry{GD}
\end{axis}

\end{tikzpicture}

    \tikzexternaldisable
  \end{subfigure}
  \caption{\textbf{Observing implicit regularization of the optimizer with
      \cockpittitle} through a comparison of \sgd and \gd on a synthetic problem
    inspired by \cite{mulayoff2020unique, ginsburg2020regularization} (details
    in the text). \textit{Top left:} The mini-batch loss of both optimizers
    looks similar. \textit{Top right:} Noise due to mini-batching regularizes
    the Hessian's maximum eigenvalue in stages where the loss suggests that
    training has converged. \textit{Bottom:} Optimization trajectories in
    parameter space. SGD is attracted to the flattest minimum.}
	  \label{cockpit::fig:implicit-regularization}
\end{figure}

%%% Local Variables:
%%% mode: latex
%%% TeX-master: "../thesis"
%%% End:
