\begin{figure}
  \centering
  % \textbf{\cifarten \threecthreed \sgd}\\[1mm]
  % defines the pgfplots style "gammaslambdasdefault"
\pgfkeys{/pgfplots/gammaslambdasdefault/.style={
    width=1.0\linewidth,
    height=0.6\linewidth,
    every axis plot/.append style={line width = 1.5pt},
    mark size = 0.6,
    tick pos = left,
    ylabel near ticks,
    xlabel near ticks,
    xtick align = inside,
    ytick align = inside,
    legend cell align = left,
    legend columns = 1,
    legend pos = south east,
    legend style = {
      fill opacity = 0.7,
      text opacity = 1,
      font = \footnotesize,
    },
    xticklabel style = {font = \footnotesize},
    xlabel style = {font = \footnotesize},
    axis line style = {black},
    yticklabel style = {font = \footnotesize},
    ylabel style = {font = \footnotesize},
    title style = {font = \footnotesize},
    grid = major,
    grid style = {dashed}
  }
}
%%% Local Variables:
%%% mode: latex
%%% TeX-master: "../../thesis"
%%% End:

  \pgfkeys{/pgfplots/zmystyle/.style={
      gammaslambdasdefault
    }}
  \tikzexternalenable
  % This file was created by tikzplotlib v0.9.7.
\begin{tikzpicture}

\begin{axis}[
axis line style={white!10!black},
log basis x={10},
tick pos=left,
xlabel={epoch (log scale)},
xmajorgrids,
xmin=0.794328234724281, xmax=125.892541179417,
xmode=log,
ylabel={SNR[\(\displaystyle \lambda_{nk}\)] (log scale)},
ymajorgrids,
ymin=0.00584285483153274, ymax=7.42366888382601,
ymode=log,
zmystyle
]
\addplot [
mark=*,
only marks,
scatter,
scatter/@post marker code/.code={%
  \endscope
},
scatter/@pre marker code/.code={%
  \expanded{%
  \noexpand\definecolor{thispointdrawcolor}{RGB}{\drawcolor}%
  \noexpand\definecolor{thispointfillcolor}{RGB}{\fillcolor}%
  }%
  \scope[draw=thispointdrawcolor, fill=thispointfillcolor]%
},
visualization depends on={value \thisrow{draw} \as \drawcolor},
visualization depends on={value \thisrow{fill} \as \fillcolor}
]
table{%
x  y  draw  fill
1 0.975784003734589 0,0,0 0,0,0
1 4.81617498397827 25.8039215686274,14.8235294117647,12.8470588235294 25.8039215686274,14.8235294117647,12.8470588235294
1 5.36448431015015 51.6078431372549,29.6470588235294,25.6941176470588 51.6078431372549,29.6470588235294,25.6941176470588
1 5.23165082931519 78.3333333333333,45,39 78.3333333333333,45,39
1 4.53217601776123 104.137254901961,59.8235294117647,51.8470588235294 104.137254901961,59.8235294117647,51.8470588235294
1 4.9503321647644 130.862745098039,75.1764705882353,65.1529411764706 130.862745098039,75.1764705882353,65.1529411764706
1 4.64961767196655 156.666666666667,90,78 156.666666666667,90,78
1 4.98736238479614 183.392156862745,105.352941176471,91.3058823529412 183.392156862745,105.352941176471,91.3058823529412
1 4.43404388427734 209.196078431373,120.176470588235,104.152941176471 209.196078431373,120.176470588235,104.152941176471
1 3.36111760139465 235,135,117 235,135,117
1.04487179487179 1.08064937591553 0,0,0 0,0,0
1.04487179487179 5.07322311401367 25.8039215686274,14.8235294117647,12.8470588235294 25.8039215686274,14.8235294117647,12.8470588235294
1.04487179487179 4.59841156005859 51.6078431372549,29.6470588235294,25.6941176470588 51.6078431372549,29.6470588235294,25.6941176470588
1.04487179487179 4.47616338729858 78.3333333333333,45,39 78.3333333333333,45,39
1.04487179487179 4.52594327926636 104.137254901961,59.8235294117647,51.8470588235294 104.137254901961,59.8235294117647,51.8470588235294
1.04487179487179 4.97389841079712 130.862745098039,75.1764705882353,65.1529411764706 130.862745098039,75.1764705882353,65.1529411764706
1.04487179487179 4.35929489135742 156.666666666667,90,78 156.666666666667,90,78
1.04487179487179 4.74812793731689 183.392156862745,105.352941176471,91.3058823529412 183.392156862745,105.352941176471,91.3058823529412
1.04487179487179 3.9832592010498 209.196078431373,120.176470588235,104.152941176471 209.196078431373,120.176470588235,104.152941176471
1.04487179487179 3.92481541633606 235,135,117 235,135,117
1.09615384615385 1.51742148399353 0,0,0 0,0,0
1.09615384615385 4.44957447052002 25.8039215686274,14.8235294117647,12.8470588235294 25.8039215686274,14.8235294117647,12.8470588235294
1.09615384615385 4.36930656433105 51.6078431372549,29.6470588235294,25.6941176470588 51.6078431372549,29.6470588235294,25.6941176470588
1.09615384615385 3.95009589195251 78.3333333333333,45,39 78.3333333333333,45,39
1.09615384615385 4.10540246963501 104.137254901961,59.8235294117647,51.8470588235294 104.137254901961,59.8235294117647,51.8470588235294
1.09615384615385 4.36474990844727 130.862745098039,75.1764705882353,65.1529411764706 130.862745098039,75.1764705882353,65.1529411764706
1.09615384615385 3.50593447685242 156.666666666667,90,78 156.666666666667,90,78
1.09615384615385 4.21931886672974 183.392156862745,105.352941176471,91.3058823529412 183.392156862745,105.352941176471,91.3058823529412
1.09615384615385 3.77479767799377 209.196078431373,120.176470588235,104.152941176471 209.196078431373,120.176470588235,104.152941176471
1.09615384615385 3.17522883415222 235,135,117 235,135,117
1.1474358974359 1.38034021854401 0,0,0 0,0,0
1.1474358974359 5.0747275352478 25.8039215686274,14.8235294117647,12.8470588235294 25.8039215686274,14.8235294117647,12.8470588235294
1.1474358974359 3.91628170013428 51.6078431372549,29.6470588235294,25.6941176470588 51.6078431372549,29.6470588235294,25.6941176470588
1.1474358974359 4.29094266891479 78.3333333333333,45,39 78.3333333333333,45,39
1.1474358974359 3.87449312210083 104.137254901961,59.8235294117647,51.8470588235294 104.137254901961,59.8235294117647,51.8470588235294
1.1474358974359 4.54162359237671 130.862745098039,75.1764705882353,65.1529411764706 130.862745098039,75.1764705882353,65.1529411764706
1.1474358974359 3.64791703224182 156.666666666667,90,78 156.666666666667,90,78
1.1474358974359 3.16870093345642 183.392156862745,105.352941176471,91.3058823529412 183.392156862745,105.352941176471,91.3058823529412
1.1474358974359 2.85287046432495 209.196078431373,120.176470588235,104.152941176471 209.196078431373,120.176470588235,104.152941176471
1.1474358974359 2.64012289047241 235,135,117 235,135,117
1.20192307692308 0.79089879989624 0,0,0 0,0,0
1.20192307692308 3.73917365074158 25.8039215686274,14.8235294117647,12.8470588235294 25.8039215686274,14.8235294117647,12.8470588235294
1.20192307692308 4.88604164123535 51.6078431372549,29.6470588235294,25.6941176470588 51.6078431372549,29.6470588235294,25.6941176470588
1.20192307692308 3.14525508880615 78.3333333333333,45,39 78.3333333333333,45,39
1.20192307692308 3.75772309303284 104.137254901961,59.8235294117647,51.8470588235294 104.137254901961,59.8235294117647,51.8470588235294
1.20192307692308 4.00032138824463 130.862745098039,75.1764705882353,65.1529411764706 130.862745098039,75.1764705882353,65.1529411764706
1.20192307692308 2.76700234413147 156.666666666667,90,78 156.666666666667,90,78
1.20192307692308 3.53422427177429 183.392156862745,105.352941176471,91.3058823529412 183.392156862745,105.352941176471,91.3058823529412
1.20192307692308 2.31572914123535 209.196078431373,120.176470588235,104.152941176471 209.196078431373,120.176470588235,104.152941176471
1.20192307692308 2.00478529930115 235,135,117 235,135,117
1.25961538461538 1.09610772132874 0,0,0 0,0,0
1.25961538461538 0.886177718639374 25.8039215686274,14.8235294117647,12.8470588235294 25.8039215686274,14.8235294117647,12.8470588235294
1.25961538461538 3.01286220550537 51.6078431372549,29.6470588235294,25.6941176470588 51.6078431372549,29.6470588235294,25.6941176470588
1.25961538461538 3.05179691314697 78.3333333333333,45,39 78.3333333333333,45,39
1.25961538461538 3.63458371162415 104.137254901961,59.8235294117647,51.8470588235294 104.137254901961,59.8235294117647,51.8470588235294
1.25961538461538 3.7750518321991 130.862745098039,75.1764705882353,65.1529411764706 130.862745098039,75.1764705882353,65.1529411764706
1.25961538461538 3.86977648735046 156.666666666667,90,78 156.666666666667,90,78
1.25961538461538 2.07635998725891 183.392156862745,105.352941176471,91.3058823529412 183.392156862745,105.352941176471,91.3058823529412
1.25961538461538 4.65148544311523 209.196078431373,120.176470588235,104.152941176471 209.196078431373,120.176470588235,104.152941176471
1.25961538461538 1.15362024307251 235,135,117 235,135,117
1.32051282051282 2.83147716522217 0,0,0 0,0,0
1.32051282051282 1.46668863296509 25.8039215686274,14.8235294117647,12.8470588235294 25.8039215686274,14.8235294117647,12.8470588235294
1.32051282051282 2.50360822677612 51.6078431372549,29.6470588235294,25.6941176470588 51.6078431372549,29.6470588235294,25.6941176470588
1.32051282051282 0.847918629646301 78.3333333333333,45,39 78.3333333333333,45,39
1.32051282051282 2.59461426734924 104.137254901961,59.8235294117647,51.8470588235294 104.137254901961,59.8235294117647,51.8470588235294
1.32051282051282 2.48218321800232 130.862745098039,75.1764705882353,65.1529411764706 130.862745098039,75.1764705882353,65.1529411764706
1.32051282051282 4.08651208877563 156.666666666667,90,78 156.666666666667,90,78
1.32051282051282 1.95713353157043 183.392156862745,105.352941176471,91.3058823529412 183.392156862745,105.352941176471,91.3058823529412
1.32051282051282 1.41261315345764 209.196078431373,120.176470588235,104.152941176471 209.196078431373,120.176470588235,104.152941176471
1.32051282051282 0.97663152217865 235,135,117 235,135,117
1.38461538461538 3.6496798992157 0,0,0 0,0,0
1.38461538461538 1.87759959697723 25.8039215686274,14.8235294117647,12.8470588235294 25.8039215686274,14.8235294117647,12.8470588235294
1.38461538461538 3.64344024658203 51.6078431372549,29.6470588235294,25.6941176470588 51.6078431372549,29.6470588235294,25.6941176470588
1.38461538461538 2.70816874504089 78.3333333333333,45,39 78.3333333333333,45,39
1.38461538461538 1.43051779270172 104.137254901961,59.8235294117647,51.8470588235294 104.137254901961,59.8235294117647,51.8470588235294
1.38461538461538 1.18621563911438 130.862745098039,75.1764705882353,65.1529411764706 130.862745098039,75.1764705882353,65.1529411764706
1.38461538461538 2.39292788505554 156.666666666667,90,78 156.666666666667,90,78
1.38461538461538 1.72273135185242 183.392156862745,105.352941176471,91.3058823529412 183.392156862745,105.352941176471,91.3058823529412
1.38461538461538 0.928144812583923 209.196078431373,120.176470588235,104.152941176471 209.196078431373,120.176470588235,104.152941176471
1.38461538461538 1.05567860603333 235,135,117 235,135,117
1.44871794871795 2.09320998191833 0,0,0 0,0,0
1.44871794871795 2.07442283630371 25.8039215686274,14.8235294117647,12.8470588235294 25.8039215686274,14.8235294117647,12.8470588235294
1.44871794871795 3.09923481941223 51.6078431372549,29.6470588235294,25.6941176470588 51.6078431372549,29.6470588235294,25.6941176470588
1.44871794871795 0.960194051265717 78.3333333333333,45,39 78.3333333333333,45,39
1.44871794871795 1.39489448070526 104.137254901961,59.8235294117647,51.8470588235294 104.137254901961,59.8235294117647,51.8470588235294
1.44871794871795 2.27095365524292 130.862745098039,75.1764705882353,65.1529411764706 130.862745098039,75.1764705882353,65.1529411764706
1.44871794871795 1.67746150493622 156.666666666667,90,78 156.666666666667,90,78
1.44871794871795 0.820045948028564 183.392156862745,105.352941176471,91.3058823529412 183.392156862745,105.352941176471,91.3058823529412
1.44871794871795 1.10364508628845 209.196078431373,120.176470588235,104.152941176471 209.196078431373,120.176470588235,104.152941176471
1.44871794871795 1.16712415218353 235,135,117 235,135,117
1.51923076923077 1.31854009628296 0,0,0 0,0,0
1.51923076923077 1.26899337768555 25.8039215686274,14.8235294117647,12.8470588235294 25.8039215686274,14.8235294117647,12.8470588235294
1.51923076923077 1.57649421691895 51.6078431372549,29.6470588235294,25.6941176470588 51.6078431372549,29.6470588235294,25.6941176470588
1.51923076923077 1.95576333999634 78.3333333333333,45,39 78.3333333333333,45,39
1.51923076923077 1.41067063808441 104.137254901961,59.8235294117647,51.8470588235294 104.137254901961,59.8235294117647,51.8470588235294
1.51923076923077 1.09714448451996 130.862745098039,75.1764705882353,65.1529411764706 130.862745098039,75.1764705882353,65.1529411764706
1.51923076923077 1.13901102542877 156.666666666667,90,78 156.666666666667,90,78
1.51923076923077 1.19942700862885 183.392156862745,105.352941176471,91.3058823529412 183.392156862745,105.352941176471,91.3058823529412
1.51923076923077 0.841314136981964 209.196078431373,120.176470588235,104.152941176471 209.196078431373,120.176470588235,104.152941176471
1.51923076923077 1.16712582111359 235,135,117 235,135,117
1.58974358974359 2.71763777732849 0,0,0 0,0,0
1.58974358974359 2.69588589668274 25.8039215686274,14.8235294117647,12.8470588235294 25.8039215686274,14.8235294117647,12.8470588235294
1.58974358974359 3.5698356628418 51.6078431372549,29.6470588235294,25.6941176470588 51.6078431372549,29.6470588235294,25.6941176470588
1.58974358974359 0.312659859657288 78.3333333333333,45,39 78.3333333333333,45,39
1.58974358974359 1.60501623153687 104.137254901961,59.8235294117647,51.8470588235294 104.137254901961,59.8235294117647,51.8470588235294
1.58974358974359 2.90069818496704 130.862745098039,75.1764705882353,65.1529411764706 130.862745098039,75.1764705882353,65.1529411764706
1.58974358974359 1.25137579441071 156.666666666667,90,78 156.666666666667,90,78
1.58974358974359 2.91528820991516 183.392156862745,105.352941176471,91.3058823529412 183.392156862745,105.352941176471,91.3058823529412
1.58974358974359 0.786626100540161 209.196078431373,120.176470588235,104.152941176471 209.196078431373,120.176470588235,104.152941176471
1.58974358974359 0.798025131225586 235,135,117 235,135,117
1.66666666666667 0.953728497028351 0,0,0 0,0,0
1.66666666666667 1.56447541713715 25.8039215686274,14.8235294117647,12.8470588235294 25.8039215686274,14.8235294117647,12.8470588235294
1.66666666666667 3.58846759796143 51.6078431372549,29.6470588235294,25.6941176470588 51.6078431372549,29.6470588235294,25.6941176470588
1.66666666666667 1.98851418495178 78.3333333333333,45,39 78.3333333333333,45,39
1.66666666666667 1.50565123558044 104.137254901961,59.8235294117647,51.8470588235294 104.137254901961,59.8235294117647,51.8470588235294
1.66666666666667 1.94345831871033 130.862745098039,75.1764705882353,65.1529411764706 130.862745098039,75.1764705882353,65.1529411764706
1.66666666666667 0.542786419391632 156.666666666667,90,78 156.666666666667,90,78
1.66666666666667 1.86082851886749 183.392156862745,105.352941176471,91.3058823529412 183.392156862745,105.352941176471,91.3058823529412
1.66666666666667 0.543868005275726 209.196078431373,120.176470588235,104.152941176471 209.196078431373,120.176470588235,104.152941176471
1.66666666666667 0.698533058166504 235,135,117 235,135,117
1.74679487179487 1.98289561271667 0,0,0 0,0,0
1.74679487179487 0.989320278167725 25.8039215686274,14.8235294117647,12.8470588235294 25.8039215686274,14.8235294117647,12.8470588235294
1.74679487179487 2.46191382408142 51.6078431372549,29.6470588235294,25.6941176470588 51.6078431372549,29.6470588235294,25.6941176470588
1.74679487179487 1.88124489784241 78.3333333333333,45,39 78.3333333333333,45,39
1.74679487179487 0.726153612136841 104.137254901961,59.8235294117647,51.8470588235294 104.137254901961,59.8235294117647,51.8470588235294
1.74679487179487 2.44277215003967 130.862745098039,75.1764705882353,65.1529411764706 130.862745098039,75.1764705882353,65.1529411764706
1.74679487179487 3.08665680885315 156.666666666667,90,78 156.666666666667,90,78
1.74679487179487 1.92148435115814 183.392156862745,105.352941176471,91.3058823529412 183.392156862745,105.352941176471,91.3058823529412
1.74679487179487 0.999748766422272 209.196078431373,120.176470588235,104.152941176471 209.196078431373,120.176470588235,104.152941176471
1.74679487179487 0.692695915699005 235,135,117 235,135,117
1.83012820512821 0.897441446781158 0,0,0 0,0,0
1.83012820512821 1.37237954139709 25.8039215686274,14.8235294117647,12.8470588235294 25.8039215686274,14.8235294117647,12.8470588235294
1.83012820512821 3.20425200462341 51.6078431372549,29.6470588235294,25.6941176470588 51.6078431372549,29.6470588235294,25.6941176470588
1.83012820512821 0.580470144748688 78.3333333333333,45,39 78.3333333333333,45,39
1.83012820512821 1.11900365352631 104.137254901961,59.8235294117647,51.8470588235294 104.137254901961,59.8235294117647,51.8470588235294
1.83012820512821 1.01710796356201 130.862745098039,75.1764705882353,65.1529411764706 130.862745098039,75.1764705882353,65.1529411764706
1.83012820512821 2.08246994018555 156.666666666667,90,78 156.666666666667,90,78
1.83012820512821 3.68994092941284 183.392156862745,105.352941176471,91.3058823529412 183.392156862745,105.352941176471,91.3058823529412
1.83012820512821 0.833235919475555 209.196078431373,120.176470588235,104.152941176471 209.196078431373,120.176470588235,104.152941176471
1.83012820512821 0.736810684204102 235,135,117 235,135,117
1.91666666666667 1.41917562484741 0,0,0 0,0,0
1.91666666666667 1.89283299446106 25.8039215686274,14.8235294117647,12.8470588235294 25.8039215686274,14.8235294117647,12.8470588235294
1.91666666666667 1.57728016376495 51.6078431372549,29.6470588235294,25.6941176470588 51.6078431372549,29.6470588235294,25.6941176470588
1.91666666666667 0.651821076869965 78.3333333333333,45,39 78.3333333333333,45,39
1.91666666666667 1.28349924087524 104.137254901961,59.8235294117647,51.8470588235294 104.137254901961,59.8235294117647,51.8470588235294
1.91666666666667 0.956565380096436 130.862745098039,75.1764705882353,65.1529411764706 130.862745098039,75.1764705882353,65.1529411764706
1.91666666666667 2.07766723632812 156.666666666667,90,78 156.666666666667,90,78
1.91666666666667 1.82708513736725 183.392156862745,105.352941176471,91.3058823529412 183.392156862745,105.352941176471,91.3058823529412
1.91666666666667 1.78830301761627 209.196078431373,120.176470588235,104.152941176471 209.196078431373,120.176470588235,104.152941176471
1.91666666666667 0.578944683074951 235,135,117 235,135,117
2.00641025641026 0.957026362419128 0,0,0 0,0,0
2.00641025641026 2.15670013427734 25.8039215686274,14.8235294117647,12.8470588235294 25.8039215686274,14.8235294117647,12.8470588235294
2.00641025641026 0.628524839878082 51.6078431372549,29.6470588235294,25.6941176470588 51.6078431372549,29.6470588235294,25.6941176470588
2.00641025641026 2.081547498703 78.3333333333333,45,39 78.3333333333333,45,39
2.00641025641026 0.610617756843567 104.137254901961,59.8235294117647,51.8470588235294 104.137254901961,59.8235294117647,51.8470588235294
2.00641025641026 1.47205865383148 130.862745098039,75.1764705882353,65.1529411764706 130.862745098039,75.1764705882353,65.1529411764706
2.00641025641026 0.878404855728149 156.666666666667,90,78 156.666666666667,90,78
2.00641025641026 1.67288327217102 183.392156862745,105.352941176471,91.3058823529412 183.392156862745,105.352941176471,91.3058823529412
2.00641025641026 2.25432372093201 209.196078431373,120.176470588235,104.152941176471 209.196078431373,120.176470588235,104.152941176471
2.00641025641026 0.772728621959686 235,135,117 235,135,117
2.1025641025641 1.1981326341629 0,0,0 0,0,0
2.1025641025641 2.98220658302307 25.8039215686274,14.8235294117647,12.8470588235294 25.8039215686274,14.8235294117647,12.8470588235294
2.1025641025641 1.94990372657776 51.6078431372549,29.6470588235294,25.6941176470588 51.6078431372549,29.6470588235294,25.6941176470588
2.1025641025641 1.24430131912231 78.3333333333333,45,39 78.3333333333333,45,39
2.1025641025641 1.46133375167847 104.137254901961,59.8235294117647,51.8470588235294 104.137254901961,59.8235294117647,51.8470588235294
2.1025641025641 0.391487717628479 130.862745098039,75.1764705882353,65.1529411764706 130.862745098039,75.1764705882353,65.1529411764706
2.1025641025641 0.965483009815216 156.666666666667,90,78 156.666666666667,90,78
2.1025641025641 2.42984747886658 183.392156862745,105.352941176471,91.3058823529412 183.392156862745,105.352941176471,91.3058823529412
2.1025641025641 0.750842213630676 209.196078431373,120.176470588235,104.152941176471 209.196078431373,120.176470588235,104.152941176471
2.1025641025641 0.703916549682617 235,135,117 235,135,117
2.20512820512821 0.683538317680359 0,0,0 0,0,0
2.20512820512821 2.2261278629303 25.8039215686274,14.8235294117647,12.8470588235294 25.8039215686274,14.8235294117647,12.8470588235294
2.20512820512821 2.85879993438721 51.6078431372549,29.6470588235294,25.6941176470588 51.6078431372549,29.6470588235294,25.6941176470588
2.20512820512821 1.31451082229614 78.3333333333333,45,39 78.3333333333333,45,39
2.20512820512821 1.17713391780853 104.137254901961,59.8235294117647,51.8470588235294 104.137254901961,59.8235294117647,51.8470588235294
2.20512820512821 1.40217018127441 130.862745098039,75.1764705882353,65.1529411764706 130.862745098039,75.1764705882353,65.1529411764706
2.20512820512821 1.22811830043793 156.666666666667,90,78 156.666666666667,90,78
2.20512820512821 1.82753384113312 183.392156862745,105.352941176471,91.3058823529412 183.392156862745,105.352941176471,91.3058823529412
2.20512820512821 1.91878819465637 209.196078431373,120.176470588235,104.152941176471 209.196078431373,120.176470588235,104.152941176471
2.20512820512821 0.67274284362793 235,135,117 235,135,117
2.30769230769231 1.15941154956818 0,0,0 0,0,0
2.30769230769231 2.67057538032532 25.8039215686274,14.8235294117647,12.8470588235294 25.8039215686274,14.8235294117647,12.8470588235294
2.30769230769231 0.724619925022125 51.6078431372549,29.6470588235294,25.6941176470588 51.6078431372549,29.6470588235294,25.6941176470588
2.30769230769231 1.0580872297287 78.3333333333333,45,39 78.3333333333333,45,39
2.30769230769231 2.75676608085632 104.137254901961,59.8235294117647,51.8470588235294 104.137254901961,59.8235294117647,51.8470588235294
2.30769230769231 0.47296866774559 130.862745098039,75.1764705882353,65.1529411764706 130.862745098039,75.1764705882353,65.1529411764706
2.30769230769231 0.951368987560272 156.666666666667,90,78 156.666666666667,90,78
2.30769230769231 1.4808474779129 183.392156862745,105.352941176471,91.3058823529412 183.392156862745,105.352941176471,91.3058823529412
2.30769230769231 1.40100240707397 209.196078431373,120.176470588235,104.152941176471 209.196078431373,120.176470588235,104.152941176471
2.30769230769231 0.663636744022369 235,135,117 235,135,117
2.41987179487179 1.3608090877533 0,0,0 0,0,0
2.41987179487179 2.1333909034729 25.8039215686274,14.8235294117647,12.8470588235294 25.8039215686274,14.8235294117647,12.8470588235294
2.41987179487179 1.6856632232666 51.6078431372549,29.6470588235294,25.6941176470588 51.6078431372549,29.6470588235294,25.6941176470588
2.41987179487179 0.920447647571564 78.3333333333333,45,39 78.3333333333333,45,39
2.41987179487179 0.536633908748627 104.137254901961,59.8235294117647,51.8470588235294 104.137254901961,59.8235294117647,51.8470588235294
2.41987179487179 2.04160213470459 130.862745098039,75.1764705882353,65.1529411764706 130.862745098039,75.1764705882353,65.1529411764706
2.41987179487179 2.03159046173096 156.666666666667,90,78 156.666666666667,90,78
2.41987179487179 0.8911172747612 183.392156862745,105.352941176471,91.3058823529412 183.392156862745,105.352941176471,91.3058823529412
2.41987179487179 1.31990396976471 209.196078431373,120.176470588235,104.152941176471 209.196078431373,120.176470588235,104.152941176471
2.41987179487179 0.978793442249298 235,135,117 235,135,117
2.53525641025641 1.28656411170959 0,0,0 0,0,0
2.53525641025641 1.01132917404175 25.8039215686274,14.8235294117647,12.8470588235294 25.8039215686274,14.8235294117647,12.8470588235294
2.53525641025641 2.91121864318848 51.6078431372549,29.6470588235294,25.6941176470588 51.6078431372549,29.6470588235294,25.6941176470588
2.53525641025641 0.35634583234787 78.3333333333333,45,39 78.3333333333333,45,39
2.53525641025641 0.450347036123276 104.137254901961,59.8235294117647,51.8470588235294 104.137254901961,59.8235294117647,51.8470588235294
2.53525641025641 2.38254070281982 130.862745098039,75.1764705882353,65.1529411764706 130.862745098039,75.1764705882353,65.1529411764706
2.53525641025641 2.22307109832764 156.666666666667,90,78 156.666666666667,90,78
2.53525641025641 1.90956676006317 183.392156862745,105.352941176471,91.3058823529412 183.392156862745,105.352941176471,91.3058823529412
2.53525641025641 1.79042518138885 209.196078431373,120.176470588235,104.152941176471 209.196078431373,120.176470588235,104.152941176471
2.53525641025641 0.575985729694366 235,135,117 235,135,117
2.65384615384615 1.58030438423157 0,0,0 0,0,0
2.65384615384615 1.0937123298645 25.8039215686274,14.8235294117647,12.8470588235294 25.8039215686274,14.8235294117647,12.8470588235294
2.65384615384615 0.996086657047272 51.6078431372549,29.6470588235294,25.6941176470588 51.6078431372549,29.6470588235294,25.6941176470588
2.65384615384615 0.894850373268127 78.3333333333333,45,39 78.3333333333333,45,39
2.65384615384615 0.708906650543213 104.137254901961,59.8235294117647,51.8470588235294 104.137254901961,59.8235294117647,51.8470588235294
2.65384615384615 1.46594929695129 130.862745098039,75.1764705882353,65.1529411764706 130.862745098039,75.1764705882353,65.1529411764706
2.65384615384615 1.06093084812164 156.666666666667,90,78 156.666666666667,90,78
2.65384615384615 0.926005721092224 183.392156862745,105.352941176471,91.3058823529412 183.392156862745,105.352941176471,91.3058823529412
2.65384615384615 0.842970669269562 209.196078431373,120.176470588235,104.152941176471 209.196078431373,120.176470588235,104.152941176471
2.65384615384615 1.36941730976105 235,135,117 235,135,117
2.78205128205128 0.968191921710968 0,0,0 0,0,0
2.78205128205128 1.24065697193146 25.8039215686274,14.8235294117647,12.8470588235294 25.8039215686274,14.8235294117647,12.8470588235294
2.78205128205128 0.683271527290344 51.6078431372549,29.6470588235294,25.6941176470588 51.6078431372549,29.6470588235294,25.6941176470588
2.78205128205128 0.888547897338867 78.3333333333333,45,39 78.3333333333333,45,39
2.78205128205128 0.995721518993378 104.137254901961,59.8235294117647,51.8470588235294 104.137254901961,59.8235294117647,51.8470588235294
2.78205128205128 1.40969228744507 130.862745098039,75.1764705882353,65.1529411764706 130.862745098039,75.1764705882353,65.1529411764706
2.78205128205128 1.74354720115662 156.666666666667,90,78 156.666666666667,90,78
2.78205128205128 0.96014541387558 183.392156862745,105.352941176471,91.3058823529412 183.392156862745,105.352941176471,91.3058823529412
2.78205128205128 1.19853329658508 209.196078431373,120.176470588235,104.152941176471 209.196078431373,120.176470588235,104.152941176471
2.78205128205128 0.697073101997375 235,135,117 235,135,117
2.91346153846154 0.609635770320892 0,0,0 0,0,0
2.91346153846154 2.07473850250244 25.8039215686274,14.8235294117647,12.8470588235294 25.8039215686274,14.8235294117647,12.8470588235294
2.91346153846154 0.549867868423462 51.6078431372549,29.6470588235294,25.6941176470588 51.6078431372549,29.6470588235294,25.6941176470588
2.91346153846154 0.735077679157257 78.3333333333333,45,39 78.3333333333333,45,39
2.91346153846154 0.353789001703262 104.137254901961,59.8235294117647,51.8470588235294 104.137254901961,59.8235294117647,51.8470588235294
2.91346153846154 0.823186814785004 130.862745098039,75.1764705882353,65.1529411764706 130.862745098039,75.1764705882353,65.1529411764706
2.91346153846154 1.76903975009918 156.666666666667,90,78 156.666666666667,90,78
2.91346153846154 0.420060843229294 183.392156862745,105.352941176471,91.3058823529412 183.392156862745,105.352941176471,91.3058823529412
2.91346153846154 0.551012754440308 209.196078431373,120.176470588235,104.152941176471 209.196078431373,120.176470588235,104.152941176471
2.91346153846154 1.01113784313202 235,135,117 235,135,117
3.05128205128205 0.959796130657196 0,0,0 0,0,0
3.05128205128205 1.06861424446106 25.8039215686274,14.8235294117647,12.8470588235294 25.8039215686274,14.8235294117647,12.8470588235294
3.05128205128205 0.611390054225922 51.6078431372549,29.6470588235294,25.6941176470588 51.6078431372549,29.6470588235294,25.6941176470588
3.05128205128205 0.372309535741806 78.3333333333333,45,39 78.3333333333333,45,39
3.05128205128205 1.54838001728058 104.137254901961,59.8235294117647,51.8470588235294 104.137254901961,59.8235294117647,51.8470588235294
3.05128205128205 1.45186567306519 130.862745098039,75.1764705882353,65.1529411764706 130.862745098039,75.1764705882353,65.1529411764706
3.05128205128205 2.38362693786621 156.666666666667,90,78 156.666666666667,90,78
3.05128205128205 1.32113754749298 183.392156862745,105.352941176471,91.3058823529412 183.392156862745,105.352941176471,91.3058823529412
3.05128205128205 1.02174687385559 209.196078431373,120.176470588235,104.152941176471 209.196078431373,120.176470588235,104.152941176471
3.05128205128205 1.65574610233307 235,135,117 235,135,117
3.19871794871795 1.64644658565521 0,0,0 0,0,0
3.19871794871795 0.949063956737518 25.8039215686274,14.8235294117647,12.8470588235294 25.8039215686274,14.8235294117647,12.8470588235294
3.19871794871795 0.546768128871918 51.6078431372549,29.6470588235294,25.6941176470588 51.6078431372549,29.6470588235294,25.6941176470588
3.19871794871795 0.2589011490345 78.3333333333333,45,39 78.3333333333333,45,39
3.19871794871795 1.13936948776245 104.137254901961,59.8235294117647,51.8470588235294 104.137254901961,59.8235294117647,51.8470588235294
3.19871794871795 1.49578666687012 130.862745098039,75.1764705882353,65.1529411764706 130.862745098039,75.1764705882353,65.1529411764706
3.19871794871795 1.57845735549927 156.666666666667,90,78 156.666666666667,90,78
3.19871794871795 0.507057189941406 183.392156862745,105.352941176471,91.3058823529412 183.392156862745,105.352941176471,91.3058823529412
3.19871794871795 0.773656845092773 209.196078431373,120.176470588235,104.152941176471 209.196078431373,120.176470588235,104.152941176471
3.19871794871795 1.32569575309753 235,135,117 235,135,117
3.34935897435897 1.56769549846649 0,0,0 0,0,0
3.34935897435897 1.77553796768188 25.8039215686274,14.8235294117647,12.8470588235294 25.8039215686274,14.8235294117647,12.8470588235294
3.34935897435897 0.920381009578705 51.6078431372549,29.6470588235294,25.6941176470588 51.6078431372549,29.6470588235294,25.6941176470588
3.34935897435897 0.787599086761475 78.3333333333333,45,39 78.3333333333333,45,39
3.34935897435897 0.988805770874023 104.137254901961,59.8235294117647,51.8470588235294 104.137254901961,59.8235294117647,51.8470588235294
3.34935897435897 1.93014740943909 130.862745098039,75.1764705882353,65.1529411764706 130.862745098039,75.1764705882353,65.1529411764706
3.34935897435897 1.90421116352081 156.666666666667,90,78 156.666666666667,90,78
3.34935897435897 1.00538742542267 183.392156862745,105.352941176471,91.3058823529412 183.392156862745,105.352941176471,91.3058823529412
3.34935897435897 0.616127073764801 209.196078431373,120.176470588235,104.152941176471 209.196078431373,120.176470588235,104.152941176471
3.34935897435897 1.32753646373749 235,135,117 235,135,117
3.50961538461538 1.23502969741821 0,0,0 0,0,0
3.50961538461538 0.898199558258057 25.8039215686274,14.8235294117647,12.8470588235294 25.8039215686274,14.8235294117647,12.8470588235294
3.50961538461538 1.02644622325897 51.6078431372549,29.6470588235294,25.6941176470588 51.6078431372549,29.6470588235294,25.6941176470588
3.50961538461538 0.968848884105682 78.3333333333333,45,39 78.3333333333333,45,39
3.50961538461538 0.787219524383545 104.137254901961,59.8235294117647,51.8470588235294 104.137254901961,59.8235294117647,51.8470588235294
3.50961538461538 1.54163193702698 130.862745098039,75.1764705882353,65.1529411764706 130.862745098039,75.1764705882353,65.1529411764706
3.50961538461538 1.70604562759399 156.666666666667,90,78 156.666666666667,90,78
3.50961538461538 1.16676878929138 183.392156862745,105.352941176471,91.3058823529412 183.392156862745,105.352941176471,91.3058823529412
3.50961538461538 0.532797038555145 209.196078431373,120.176470588235,104.152941176471 209.196078431373,120.176470588235,104.152941176471
3.50961538461538 1.37712466716766 235,135,117 235,135,117
3.67628205128205 0.768164575099945 0,0,0 0,0,0
3.67628205128205 0.897670865058899 25.8039215686274,14.8235294117647,12.8470588235294 25.8039215686274,14.8235294117647,12.8470588235294
3.67628205128205 0.48084232211113 51.6078431372549,29.6470588235294,25.6941176470588 51.6078431372549,29.6470588235294,25.6941176470588
3.67628205128205 0.609206855297089 78.3333333333333,45,39 78.3333333333333,45,39
3.67628205128205 0.803423702716827 104.137254901961,59.8235294117647,51.8470588235294 104.137254901961,59.8235294117647,51.8470588235294
3.67628205128205 0.514328300952911 130.862745098039,75.1764705882353,65.1529411764706 130.862745098039,75.1764705882353,65.1529411764706
3.67628205128205 1.95982778072357 156.666666666667,90,78 156.666666666667,90,78
3.67628205128205 0.98697167634964 183.392156862745,105.352941176471,91.3058823529412 183.392156862745,105.352941176471,91.3058823529412
3.67628205128205 0.624035656452179 209.196078431373,120.176470588235,104.152941176471 209.196078431373,120.176470588235,104.152941176471
3.67628205128205 0.967042088508606 235,135,117 235,135,117
3.8525641025641 0.728654861450195 0,0,0 0,0,0
3.8525641025641 0.444164723157883 25.8039215686274,14.8235294117647,12.8470588235294 25.8039215686274,14.8235294117647,12.8470588235294
3.8525641025641 0.673896551132202 51.6078431372549,29.6470588235294,25.6941176470588 51.6078431372549,29.6470588235294,25.6941176470588
3.8525641025641 1.50595903396606 78.3333333333333,45,39 78.3333333333333,45,39
3.8525641025641 0.87969446182251 104.137254901961,59.8235294117647,51.8470588235294 104.137254901961,59.8235294117647,51.8470588235294
3.8525641025641 1.36471116542816 130.862745098039,75.1764705882353,65.1529411764706 130.862745098039,75.1764705882353,65.1529411764706
3.8525641025641 0.91891485452652 156.666666666667,90,78 156.666666666667,90,78
3.8525641025641 0.979045569896698 183.392156862745,105.352941176471,91.3058823529412 183.392156862745,105.352941176471,91.3058823529412
3.8525641025641 0.600068092346191 209.196078431373,120.176470588235,104.152941176471 209.196078431373,120.176470588235,104.152941176471
3.8525641025641 1.05551815032959 235,135,117 235,135,117
4.03525641025641 0.900466620922089 0,0,0 0,0,0
4.03525641025641 0.555265784263611 25.8039215686274,14.8235294117647,12.8470588235294 25.8039215686274,14.8235294117647,12.8470588235294
4.03525641025641 0.213296785950661 51.6078431372549,29.6470588235294,25.6941176470588 51.6078431372549,29.6470588235294,25.6941176470588
4.03525641025641 1.36506283283234 78.3333333333333,45,39 78.3333333333333,45,39
4.03525641025641 1.6247261762619 104.137254901961,59.8235294117647,51.8470588235294 104.137254901961,59.8235294117647,51.8470588235294
4.03525641025641 0.582535028457642 130.862745098039,75.1764705882353,65.1529411764706 130.862745098039,75.1764705882353,65.1529411764706
4.03525641025641 1.27524209022522 156.666666666667,90,78 156.666666666667,90,78
4.03525641025641 0.937963902950287 183.392156862745,105.352941176471,91.3058823529412 183.392156862745,105.352941176471,91.3058823529412
4.03525641025641 0.439503490924835 209.196078431373,120.176470588235,104.152941176471 209.196078431373,120.176470588235,104.152941176471
4.03525641025641 0.818723857402802 235,135,117 235,135,117
4.2275641025641 1.22950494289398 0,0,0 0,0,0
4.2275641025641 0.668096899986267 25.8039215686274,14.8235294117647,12.8470588235294 25.8039215686274,14.8235294117647,12.8470588235294
4.2275641025641 0.712359309196472 51.6078431372549,29.6470588235294,25.6941176470588 51.6078431372549,29.6470588235294,25.6941176470588
4.2275641025641 0.610431015491486 78.3333333333333,45,39 78.3333333333333,45,39
4.2275641025641 1.09898900985718 104.137254901961,59.8235294117647,51.8470588235294 104.137254901961,59.8235294117647,51.8470588235294
4.2275641025641 1.02358484268188 130.862745098039,75.1764705882353,65.1529411764706 130.862745098039,75.1764705882353,65.1529411764706
4.2275641025641 1.1459047794342 156.666666666667,90,78 156.666666666667,90,78
4.2275641025641 0.966608166694641 183.392156862745,105.352941176471,91.3058823529412 183.392156862745,105.352941176471,91.3058823529412
4.2275641025641 0.685954034328461 209.196078431373,120.176470588235,104.152941176471 209.196078431373,120.176470588235,104.152941176471
4.2275641025641 0.971087157726288 235,135,117 235,135,117
4.42948717948718 0.536157488822937 0,0,0 0,0,0
4.42948717948718 0.31957283616066 25.8039215686274,14.8235294117647,12.8470588235294 25.8039215686274,14.8235294117647,12.8470588235294
4.42948717948718 0.254982441663742 51.6078431372549,29.6470588235294,25.6941176470588 51.6078431372549,29.6470588235294,25.6941176470588
4.42948717948718 1.23078382015228 78.3333333333333,45,39 78.3333333333333,45,39
4.42948717948718 1.27418661117554 104.137254901961,59.8235294117647,51.8470588235294 104.137254901961,59.8235294117647,51.8470588235294
4.42948717948718 1.09501171112061 130.862745098039,75.1764705882353,65.1529411764706 130.862745098039,75.1764705882353,65.1529411764706
4.42948717948718 1.42944669723511 156.666666666667,90,78 156.666666666667,90,78
4.42948717948718 0.567227423191071 183.392156862745,105.352941176471,91.3058823529412 183.392156862745,105.352941176471,91.3058823529412
4.42948717948718 0.512123703956604 209.196078431373,120.176470588235,104.152941176471 209.196078431373,120.176470588235,104.152941176471
4.42948717948718 1.03413116931915 235,135,117 235,135,117
4.64102564102564 1.1321474313736 0,0,0 0,0,0
4.64102564102564 1.05999195575714 25.8039215686274,14.8235294117647,12.8470588235294 25.8039215686274,14.8235294117647,12.8470588235294
4.64102564102564 0.784232795238495 51.6078431372549,29.6470588235294,25.6941176470588 51.6078431372549,29.6470588235294,25.6941176470588
4.64102564102564 0.38313490152359 78.3333333333333,45,39 78.3333333333333,45,39
4.64102564102564 0.899316608905792 104.137254901961,59.8235294117647,51.8470588235294 104.137254901961,59.8235294117647,51.8470588235294
4.64102564102564 1.51124787330627 130.862745098039,75.1764705882353,65.1529411764706 130.862745098039,75.1764705882353,65.1529411764706
4.64102564102564 0.830731153488159 156.666666666667,90,78 156.666666666667,90,78
4.64102564102564 0.669659793376923 183.392156862745,105.352941176471,91.3058823529412 183.392156862745,105.352941176471,91.3058823529412
4.64102564102564 0.601336717605591 209.196078431373,120.176470588235,104.152941176471 209.196078431373,120.176470588235,104.152941176471
4.64102564102564 1.05722498893738 235,135,117 235,135,117
4.86217948717949 1.46660482883453 0,0,0 0,0,0
4.86217948717949 0.965067565441132 25.8039215686274,14.8235294117647,12.8470588235294 25.8039215686274,14.8235294117647,12.8470588235294
4.86217948717949 0.496772199869156 51.6078431372549,29.6470588235294,25.6941176470588 51.6078431372549,29.6470588235294,25.6941176470588
4.86217948717949 0.609667479991913 78.3333333333333,45,39 78.3333333333333,45,39
4.86217948717949 0.781067073345184 104.137254901961,59.8235294117647,51.8470588235294 104.137254901961,59.8235294117647,51.8470588235294
4.86217948717949 0.746175348758698 130.862745098039,75.1764705882353,65.1529411764706 130.862745098039,75.1764705882353,65.1529411764706
4.86217948717949 1.62621545791626 156.666666666667,90,78 156.666666666667,90,78
4.86217948717949 1.02460932731628 183.392156862745,105.352941176471,91.3058823529412 183.392156862745,105.352941176471,91.3058823529412
4.86217948717949 0.761037766933441 209.196078431373,120.176470588235,104.152941176471 209.196078431373,120.176470588235,104.152941176471
4.86217948717949 1.24410593509674 235,135,117 235,135,117
5.09294871794872 0.512161672115326 0,0,0 0,0,0
5.09294871794872 0.432282269001007 25.8039215686274,14.8235294117647,12.8470588235294 25.8039215686274,14.8235294117647,12.8470588235294
5.09294871794872 0.256528824567795 51.6078431372549,29.6470588235294,25.6941176470588 51.6078431372549,29.6470588235294,25.6941176470588
5.09294871794872 1.31927740573883 78.3333333333333,45,39 78.3333333333333,45,39
5.09294871794872 0.747238874435425 104.137254901961,59.8235294117647,51.8470588235294 104.137254901961,59.8235294117647,51.8470588235294
5.09294871794872 0.907392740249634 130.862745098039,75.1764705882353,65.1529411764706 130.862745098039,75.1764705882353,65.1529411764706
5.09294871794872 0.280877202749252 156.666666666667,90,78 156.666666666667,90,78
5.09294871794872 0.527149200439453 183.392156862745,105.352941176471,91.3058823529412 183.392156862745,105.352941176471,91.3058823529412
5.09294871794872 0.795610666275024 209.196078431373,120.176470588235,104.152941176471 209.196078431373,120.176470588235,104.152941176471
5.09294871794872 0.886933445930481 235,135,117 235,135,117
5.33653846153846 0.881524264812469 0,0,0 0,0,0
5.33653846153846 0.379370272159576 25.8039215686274,14.8235294117647,12.8470588235294 25.8039215686274,14.8235294117647,12.8470588235294
5.33653846153846 0.630515813827515 51.6078431372549,29.6470588235294,25.6941176470588 51.6078431372549,29.6470588235294,25.6941176470588
5.33653846153846 0.634531259536743 78.3333333333333,45,39 78.3333333333333,45,39
5.33653846153846 0.993321239948273 104.137254901961,59.8235294117647,51.8470588235294 104.137254901961,59.8235294117647,51.8470588235294
5.33653846153846 0.842267215251923 130.862745098039,75.1764705882353,65.1529411764706 130.862745098039,75.1764705882353,65.1529411764706
5.33653846153846 0.535592973232269 156.666666666667,90,78 156.666666666667,90,78
5.33653846153846 1.16408264636993 183.392156862745,105.352941176471,91.3058823529412 183.392156862745,105.352941176471,91.3058823529412
5.33653846153846 0.500023365020752 209.196078431373,120.176470588235,104.152941176471 209.196078431373,120.176470588235,104.152941176471
5.33653846153846 0.849346876144409 235,135,117 235,135,117
5.58974358974359 0.981884658336639 0,0,0 0,0,0
5.58974358974359 1.17341864109039 25.8039215686274,14.8235294117647,12.8470588235294 25.8039215686274,14.8235294117647,12.8470588235294
5.58974358974359 0.396986305713654 51.6078431372549,29.6470588235294,25.6941176470588 51.6078431372549,29.6470588235294,25.6941176470588
5.58974358974359 0.969923973083496 78.3333333333333,45,39 78.3333333333333,45,39
5.58974358974359 0.352045565843582 104.137254901961,59.8235294117647,51.8470588235294 104.137254901961,59.8235294117647,51.8470588235294
5.58974358974359 0.969979047775269 130.862745098039,75.1764705882353,65.1529411764706 130.862745098039,75.1764705882353,65.1529411764706
5.58974358974359 0.910520195960999 156.666666666667,90,78 156.666666666667,90,78
5.58974358974359 0.739221334457397 183.392156862745,105.352941176471,91.3058823529412 183.392156862745,105.352941176471,91.3058823529412
5.58974358974359 0.844757556915283 209.196078431373,120.176470588235,104.152941176471 209.196078431373,120.176470588235,104.152941176471
5.58974358974359 1.07917749881744 235,135,117 235,135,117
5.85576923076923 0.63684070110321 0,0,0 0,0,0
5.85576923076923 0.479411870241165 25.8039215686274,14.8235294117647,12.8470588235294 25.8039215686274,14.8235294117647,12.8470588235294
5.85576923076923 0.426124453544617 51.6078431372549,29.6470588235294,25.6941176470588 51.6078431372549,29.6470588235294,25.6941176470588
5.85576923076923 0.686657845973969 78.3333333333333,45,39 78.3333333333333,45,39
5.85576923076923 0.382183194160461 104.137254901961,59.8235294117647,51.8470588235294 104.137254901961,59.8235294117647,51.8470588235294
5.85576923076923 0.554038345813751 130.862745098039,75.1764705882353,65.1529411764706 130.862745098039,75.1764705882353,65.1529411764706
5.85576923076923 0.48877289891243 156.666666666667,90,78 156.666666666667,90,78
5.85576923076923 0.823561072349548 183.392156862745,105.352941176471,91.3058823529412 183.392156862745,105.352941176471,91.3058823529412
5.85576923076923 0.235067293047905 209.196078431373,120.176470588235,104.152941176471 209.196078431373,120.176470588235,104.152941176471
5.85576923076923 0.673173248767853 235,135,117 235,135,117
6.13461538461539 1.32324814796448 0,0,0 0,0,0
6.13461538461539 0.596458256244659 25.8039215686274,14.8235294117647,12.8470588235294 25.8039215686274,14.8235294117647,12.8470588235294
6.13461538461539 0.965082883834839 51.6078431372549,29.6470588235294,25.6941176470588 51.6078431372549,29.6470588235294,25.6941176470588
6.13461538461539 1.11405169963837 78.3333333333333,45,39 78.3333333333333,45,39
6.13461538461539 0.572742342948914 104.137254901961,59.8235294117647,51.8470588235294 104.137254901961,59.8235294117647,51.8470588235294
6.13461538461539 1.01978623867035 130.862745098039,75.1764705882353,65.1529411764706 130.862745098039,75.1764705882353,65.1529411764706
6.13461538461539 0.615030705928802 156.666666666667,90,78 156.666666666667,90,78
6.13461538461539 0.765961289405823 183.392156862745,105.352941176471,91.3058823529412 183.392156862745,105.352941176471,91.3058823529412
6.13461538461539 0.431054890155792 209.196078431373,120.176470588235,104.152941176471 209.196078431373,120.176470588235,104.152941176471
6.13461538461539 0.892988204956055 235,135,117 235,135,117
6.42628205128205 0.716674864292145 0,0,0 0,0,0
6.42628205128205 0.913927018642426 25.8039215686274,14.8235294117647,12.8470588235294 25.8039215686274,14.8235294117647,12.8470588235294
6.42628205128205 0.85632336139679 51.6078431372549,29.6470588235294,25.6941176470588 51.6078431372549,29.6470588235294,25.6941176470588
6.42628205128205 0.448912739753723 78.3333333333333,45,39 78.3333333333333,45,39
6.42628205128205 0.517470598220825 104.137254901961,59.8235294117647,51.8470588235294 104.137254901961,59.8235294117647,51.8470588235294
6.42628205128205 0.915785610675812 130.862745098039,75.1764705882353,65.1529411764706 130.862745098039,75.1764705882353,65.1529411764706
6.42628205128205 0.453579485416412 156.666666666667,90,78 156.666666666667,90,78
6.42628205128205 0.949284672737122 183.392156862745,105.352941176471,91.3058823529412 183.392156862745,105.352941176471,91.3058823529412
6.42628205128205 0.721194982528687 209.196078431373,120.176470588235,104.152941176471 209.196078431373,120.176470588235,104.152941176471
6.42628205128205 0.445933938026428 235,135,117 235,135,117
6.73397435897436 0.336776793003082 0,0,0 0,0,0
6.73397435897436 0.663623213768005 25.8039215686274,14.8235294117647,12.8470588235294 25.8039215686274,14.8235294117647,12.8470588235294
6.73397435897436 0.522590100765228 51.6078431372549,29.6470588235294,25.6941176470588 51.6078431372549,29.6470588235294,25.6941176470588
6.73397435897436 0.361402273178101 78.3333333333333,45,39 78.3333333333333,45,39
6.73397435897436 0.483754307031631 104.137254901961,59.8235294117647,51.8470588235294 104.137254901961,59.8235294117647,51.8470588235294
6.73397435897436 1.03270316123962 130.862745098039,75.1764705882353,65.1529411764706 130.862745098039,75.1764705882353,65.1529411764706
6.73397435897436 0.411600589752197 156.666666666667,90,78 156.666666666667,90,78
6.73397435897436 0.592099785804749 183.392156862745,105.352941176471,91.3058823529412 183.392156862745,105.352941176471,91.3058823529412
6.73397435897436 0.767435133457184 209.196078431373,120.176470588235,104.152941176471 209.196078431373,120.176470588235,104.152941176471
6.73397435897436 0.860516786575317 235,135,117 235,135,117
7.05448717948718 0.589427530765533 0,0,0 0,0,0
7.05448717948718 0.733646333217621 25.8039215686274,14.8235294117647,12.8470588235294 25.8039215686274,14.8235294117647,12.8470588235294
7.05448717948718 0.365888655185699 51.6078431372549,29.6470588235294,25.6941176470588 51.6078431372549,29.6470588235294,25.6941176470588
7.05448717948718 0.565677523612976 78.3333333333333,45,39 78.3333333333333,45,39
7.05448717948718 0.849868595600128 104.137254901961,59.8235294117647,51.8470588235294 104.137254901961,59.8235294117647,51.8470588235294
7.05448717948718 0.390950590372086 130.862745098039,75.1764705882353,65.1529411764706 130.862745098039,75.1764705882353,65.1529411764706
7.05448717948718 0.565177261829376 156.666666666667,90,78 156.666666666667,90,78
7.05448717948718 0.888440728187561 183.392156862745,105.352941176471,91.3058823529412 183.392156862745,105.352941176471,91.3058823529412
7.05448717948718 0.722141921520233 209.196078431373,120.176470588235,104.152941176471 209.196078431373,120.176470588235,104.152941176471
7.05448717948718 0.696915686130524 235,135,117 235,135,117
7.38782051282051 0.524853229522705 0,0,0 0,0,0
7.38782051282051 1.03068447113037 25.8039215686274,14.8235294117647,12.8470588235294 25.8039215686274,14.8235294117647,12.8470588235294
7.38782051282051 0.327024638652802 51.6078431372549,29.6470588235294,25.6941176470588 51.6078431372549,29.6470588235294,25.6941176470588
7.38782051282051 0.482303887605667 78.3333333333333,45,39 78.3333333333333,45,39
7.38782051282051 0.759714841842651 104.137254901961,59.8235294117647,51.8470588235294 104.137254901961,59.8235294117647,51.8470588235294
7.38782051282051 0.900131404399872 130.862745098039,75.1764705882353,65.1529411764706 130.862745098039,75.1764705882353,65.1529411764706
7.38782051282051 0.526843011379242 156.666666666667,90,78 156.666666666667,90,78
7.38782051282051 0.532286405563354 183.392156862745,105.352941176471,91.3058823529412 183.392156862745,105.352941176471,91.3058823529412
7.38782051282051 0.230667129158974 209.196078431373,120.176470588235,104.152941176471 209.196078431373,120.176470588235,104.152941176471
7.38782051282051 0.635480761528015 235,135,117 235,135,117
7.74038461538461 0.468379110097885 0,0,0 0,0,0
7.74038461538461 0.194263786077499 25.8039215686274,14.8235294117647,12.8470588235294 25.8039215686274,14.8235294117647,12.8470588235294
7.74038461538461 0.516550302505493 51.6078431372549,29.6470588235294,25.6941176470588 51.6078431372549,29.6470588235294,25.6941176470588
7.74038461538461 0.36815682053566 78.3333333333333,45,39 78.3333333333333,45,39
7.74038461538461 0.906022489070892 104.137254901961,59.8235294117647,51.8470588235294 104.137254901961,59.8235294117647,51.8470588235294
7.74038461538461 0.518999755382538 130.862745098039,75.1764705882353,65.1529411764706 130.862745098039,75.1764705882353,65.1529411764706
7.74038461538461 0.32777351140976 156.666666666667,90,78 156.666666666667,90,78
7.74038461538461 0.882981300354004 183.392156862745,105.352941176471,91.3058823529412 183.392156862745,105.352941176471,91.3058823529412
7.74038461538461 0.670071482658386 209.196078431373,120.176470588235,104.152941176471 209.196078431373,120.176470588235,104.152941176471
7.74038461538461 0.514330744743347 235,135,117 235,135,117
8.10897435897436 0.654615223407745 0,0,0 0,0,0
8.10897435897436 0.50330376625061 25.8039215686274,14.8235294117647,12.8470588235294 25.8039215686274,14.8235294117647,12.8470588235294
8.10897435897436 0.135971948504448 51.6078431372549,29.6470588235294,25.6941176470588 51.6078431372549,29.6470588235294,25.6941176470588
8.10897435897436 0.328311055898666 78.3333333333333,45,39 78.3333333333333,45,39
8.10897435897436 0.724754989147186 104.137254901961,59.8235294117647,51.8470588235294 104.137254901961,59.8235294117647,51.8470588235294
8.10897435897436 0.514231741428375 130.862745098039,75.1764705882353,65.1529411764706 130.862745098039,75.1764705882353,65.1529411764706
8.10897435897436 0.634259700775146 156.666666666667,90,78 156.666666666667,90,78
8.10897435897436 0.582264304161072 183.392156862745,105.352941176471,91.3058823529412 183.392156862745,105.352941176471,91.3058823529412
8.10897435897436 0.12478443980217 209.196078431373,120.176470588235,104.152941176471 209.196078431373,120.176470588235,104.152941176471
8.10897435897436 0.464908361434937 235,135,117 235,135,117
8.49679487179487 0.960468590259552 0,0,0 0,0,0
8.49679487179487 0.250853180885315 25.8039215686274,14.8235294117647,12.8470588235294 25.8039215686274,14.8235294117647,12.8470588235294
8.49679487179487 1.0983270406723 51.6078431372549,29.6470588235294,25.6941176470588 51.6078431372549,29.6470588235294,25.6941176470588
8.49679487179487 0.523638606071472 78.3333333333333,45,39 78.3333333333333,45,39
8.49679487179487 1.07649481296539 104.137254901961,59.8235294117647,51.8470588235294 104.137254901961,59.8235294117647,51.8470588235294
8.49679487179487 0.947211146354675 130.862745098039,75.1764705882353,65.1529411764706 130.862745098039,75.1764705882353,65.1529411764706
8.49679487179487 0.730439841747284 156.666666666667,90,78 156.666666666667,90,78
8.49679487179487 0.536979138851166 183.392156862745,105.352941176471,91.3058823529412 183.392156862745,105.352941176471,91.3058823529412
8.49679487179487 0.680288255214691 209.196078431373,120.176470588235,104.152941176471 209.196078431373,120.176470588235,104.152941176471
8.49679487179487 0.695093154907227 235,135,117 235,135,117
8.90064102564103 0.777355372905731 0,0,0 0,0,0
8.90064102564103 0.968128442764282 25.8039215686274,14.8235294117647,12.8470588235294 25.8039215686274,14.8235294117647,12.8470588235294
8.90064102564103 0.393322944641113 51.6078431372549,29.6470588235294,25.6941176470588 51.6078431372549,29.6470588235294,25.6941176470588
8.90064102564103 0.438248574733734 78.3333333333333,45,39 78.3333333333333,45,39
8.90064102564103 0.375376403331757 104.137254901961,59.8235294117647,51.8470588235294 104.137254901961,59.8235294117647,51.8470588235294
8.90064102564103 0.614651083946228 130.862745098039,75.1764705882353,65.1529411764706 130.862745098039,75.1764705882353,65.1529411764706
8.90064102564103 0.736077189445496 156.666666666667,90,78 156.666666666667,90,78
8.90064102564103 0.755953013896942 183.392156862745,105.352941176471,91.3058823529412 183.392156862745,105.352941176471,91.3058823529412
8.90064102564103 0.723952114582062 209.196078431373,120.176470588235,104.152941176471 209.196078431373,120.176470588235,104.152941176471
8.90064102564103 0.631002724170685 235,135,117 235,135,117
9.32371794871795 0.766370475292206 0,0,0 0,0,0
9.32371794871795 0.673695206642151 25.8039215686274,14.8235294117647,12.8470588235294 25.8039215686274,14.8235294117647,12.8470588235294
9.32371794871795 0.341137737035751 51.6078431372549,29.6470588235294,25.6941176470588 51.6078431372549,29.6470588235294,25.6941176470588
9.32371794871795 0.372631847858429 78.3333333333333,45,39 78.3333333333333,45,39
9.32371794871795 0.786072731018066 104.137254901961,59.8235294117647,51.8470588235294 104.137254901961,59.8235294117647,51.8470588235294
9.32371794871795 0.492206126451492 130.862745098039,75.1764705882353,65.1529411764706 130.862745098039,75.1764705882353,65.1529411764706
9.32371794871795 0.232465401291847 156.666666666667,90,78 156.666666666667,90,78
9.32371794871795 0.525389432907104 183.392156862745,105.352941176471,91.3058823529412 183.392156862745,105.352941176471,91.3058823529412
9.32371794871795 0.86659187078476 209.196078431373,120.176470588235,104.152941176471 209.196078431373,120.176470588235,104.152941176471
9.32371794871795 0.781038641929626 235,135,117 235,135,117
9.76923076923077 0.436046451330185 0,0,0 0,0,0
9.76923076923077 0.396347314119339 25.8039215686274,14.8235294117647,12.8470588235294 25.8039215686274,14.8235294117647,12.8470588235294
9.76923076923077 0.444849222898483 51.6078431372549,29.6470588235294,25.6941176470588 51.6078431372549,29.6470588235294,25.6941176470588
9.76923076923077 0.211970046162605 78.3333333333333,45,39 78.3333333333333,45,39
9.76923076923077 0.542597711086273 104.137254901961,59.8235294117647,51.8470588235294 104.137254901961,59.8235294117647,51.8470588235294
9.76923076923077 0.656075060367584 130.862745098039,75.1764705882353,65.1529411764706 130.862745098039,75.1764705882353,65.1529411764706
9.76923076923077 0.611439943313599 156.666666666667,90,78 156.666666666667,90,78
9.76923076923077 0.311808377504349 183.392156862745,105.352941176471,91.3058823529412 183.392156862745,105.352941176471,91.3058823529412
9.76923076923077 0.5491943359375 209.196078431373,120.176470588235,104.152941176471 209.196078431373,120.176470588235,104.152941176471
9.76923076923077 0.531759440898895 235,135,117 235,135,117
10.2339743589744 0.375014394521713 0,0,0 0,0,0
10.2339743589744 0.932224452495575 25.8039215686274,14.8235294117647,12.8470588235294 25.8039215686274,14.8235294117647,12.8470588235294
10.2339743589744 0.302491992712021 51.6078431372549,29.6470588235294,25.6941176470588 51.6078431372549,29.6470588235294,25.6941176470588
10.2339743589744 0.669188618659973 78.3333333333333,45,39 78.3333333333333,45,39
10.2339743589744 0.277532428503036 104.137254901961,59.8235294117647,51.8470588235294 104.137254901961,59.8235294117647,51.8470588235294
10.2339743589744 0.796482741832733 130.862745098039,75.1764705882353,65.1529411764706 130.862745098039,75.1764705882353,65.1529411764706
10.2339743589744 0.448923826217651 156.666666666667,90,78 156.666666666667,90,78
10.2339743589744 0.681187927722931 183.392156862745,105.352941176471,91.3058823529412 183.392156862745,105.352941176471,91.3058823529412
10.2339743589744 0.347121387720108 209.196078431373,120.176470588235,104.152941176471 209.196078431373,120.176470588235,104.152941176471
10.2339743589744 0.554070651531219 235,135,117 235,135,117
10.7211538461538 0.689081966876984 0,0,0 0,0,0
10.7211538461538 0.892086446285248 25.8039215686274,14.8235294117647,12.8470588235294 25.8039215686274,14.8235294117647,12.8470588235294
10.7211538461538 0.375030159950256 51.6078431372549,29.6470588235294,25.6941176470588 51.6078431372549,29.6470588235294,25.6941176470588
10.7211538461538 0.52287882566452 78.3333333333333,45,39 78.3333333333333,45,39
10.7211538461538 0.268375307321548 104.137254901961,59.8235294117647,51.8470588235294 104.137254901961,59.8235294117647,51.8470588235294
10.7211538461538 0.259581953287125 130.862745098039,75.1764705882353,65.1529411764706 130.862745098039,75.1764705882353,65.1529411764706
10.7211538461538 0.505916357040405 156.666666666667,90,78 156.666666666667,90,78
10.7211538461538 0.379882037639618 183.392156862745,105.352941176471,91.3058823529412 183.392156862745,105.352941176471,91.3058823529412
10.7211538461538 0.639497756958008 209.196078431373,120.176470588235,104.152941176471 209.196078431373,120.176470588235,104.152941176471
10.7211538461538 0.28718575835228 235,135,117 235,135,117
11.2307692307692 0.354478567838669 0,0,0 0,0,0
11.2307692307692 0.164844661951065 25.8039215686274,14.8235294117647,12.8470588235294 25.8039215686274,14.8235294117647,12.8470588235294
11.2307692307692 0.51285582780838 51.6078431372549,29.6470588235294,25.6941176470588 51.6078431372549,29.6470588235294,25.6941176470588
11.2307692307692 0.417955696582794 78.3333333333333,45,39 78.3333333333333,45,39
11.2307692307692 0.398293703794479 104.137254901961,59.8235294117647,51.8470588235294 104.137254901961,59.8235294117647,51.8470588235294
11.2307692307692 0.492564499378204 130.862745098039,75.1764705882353,65.1529411764706 130.862745098039,75.1764705882353,65.1529411764706
11.2307692307692 0.422768414020538 156.666666666667,90,78 156.666666666667,90,78
11.2307692307692 0.54209291934967 183.392156862745,105.352941176471,91.3058823529412 183.392156862745,105.352941176471,91.3058823529412
11.2307692307692 0.340652078390121 209.196078431373,120.176470588235,104.152941176471 209.196078431373,120.176470588235,104.152941176471
11.2307692307692 0.415873557329178 235,135,117 235,135,117
11.7660256410256 0.497313439846039 0,0,0 0,0,0
11.7660256410256 0.25340873003006 25.8039215686274,14.8235294117647,12.8470588235294 25.8039215686274,14.8235294117647,12.8470588235294
11.7660256410256 0.36448135972023 51.6078431372549,29.6470588235294,25.6941176470588 51.6078431372549,29.6470588235294,25.6941176470588
11.7660256410256 0.201116472482681 78.3333333333333,45,39 78.3333333333333,45,39
11.7660256410256 0.27964198589325 104.137254901961,59.8235294117647,51.8470588235294 104.137254901961,59.8235294117647,51.8470588235294
11.7660256410256 0.166851162910461 130.862745098039,75.1764705882353,65.1529411764706 130.862745098039,75.1764705882353,65.1529411764706
11.7660256410256 0.302189558744431 156.666666666667,90,78 156.666666666667,90,78
11.7660256410256 0.494708031415939 183.392156862745,105.352941176471,91.3058823529412 183.392156862745,105.352941176471,91.3058823529412
11.7660256410256 0.38817036151886 209.196078431373,120.176470588235,104.152941176471 209.196078431373,120.176470588235,104.152941176471
11.7660256410256 0.480544656515121 235,135,117 235,135,117
12.3269230769231 0.537239551544189 0,0,0 0,0,0
12.3269230769231 0.183179184794426 25.8039215686274,14.8235294117647,12.8470588235294 25.8039215686274,14.8235294117647,12.8470588235294
12.3269230769231 0.54598468542099 51.6078431372549,29.6470588235294,25.6941176470588 51.6078431372549,29.6470588235294,25.6941176470588
12.3269230769231 0.224608317017555 78.3333333333333,45,39 78.3333333333333,45,39
12.3269230769231 0.336753576993942 104.137254901961,59.8235294117647,51.8470588235294 104.137254901961,59.8235294117647,51.8470588235294
12.3269230769231 0.818406343460083 130.862745098039,75.1764705882353,65.1529411764706 130.862745098039,75.1764705882353,65.1529411764706
12.3269230769231 0.45431461930275 156.666666666667,90,78 156.666666666667,90,78
12.3269230769231 0.400381803512573 183.392156862745,105.352941176471,91.3058823529412 183.392156862745,105.352941176471,91.3058823529412
12.3269230769231 0.262286573648453 209.196078431373,120.176470588235,104.152941176471 209.196078431373,120.176470588235,104.152941176471
12.3269230769231 0.402088671922684 235,135,117 235,135,117
12.9134615384615 0.319041311740875 0,0,0 0,0,0
12.9134615384615 0.549960255622864 25.8039215686274,14.8235294117647,12.8470588235294 25.8039215686274,14.8235294117647,12.8470588235294
12.9134615384615 0.431554824113846 51.6078431372549,29.6470588235294,25.6941176470588 51.6078431372549,29.6470588235294,25.6941176470588
12.9134615384615 0.883669495582581 78.3333333333333,45,39 78.3333333333333,45,39
12.9134615384615 0.379375547170639 104.137254901961,59.8235294117647,51.8470588235294 104.137254901961,59.8235294117647,51.8470588235294
12.9134615384615 0.339861005544662 130.862745098039,75.1764705882353,65.1529411764706 130.862745098039,75.1764705882353,65.1529411764706
12.9134615384615 0.718375265598297 156.666666666667,90,78 156.666666666667,90,78
12.9134615384615 0.670551478862762 183.392156862745,105.352941176471,91.3058823529412 183.392156862745,105.352941176471,91.3058823529412
12.9134615384615 0.560630261898041 209.196078431373,120.176470588235,104.152941176471 209.196078431373,120.176470588235,104.152941176471
12.9134615384615 0.567080795764923 235,135,117 235,135,117
13.5288461538462 0.243825942277908 0,0,0 0,0,0
13.5288461538462 0.333544135093689 25.8039215686274,14.8235294117647,12.8470588235294 25.8039215686274,14.8235294117647,12.8470588235294
13.5288461538462 0.350484818220139 51.6078431372549,29.6470588235294,25.6941176470588 51.6078431372549,29.6470588235294,25.6941176470588
13.5288461538462 0.173688262701035 78.3333333333333,45,39 78.3333333333333,45,39
13.5288461538462 0.287692010402679 104.137254901961,59.8235294117647,51.8470588235294 104.137254901961,59.8235294117647,51.8470588235294
13.5288461538462 0.304149448871613 130.862745098039,75.1764705882353,65.1529411764706 130.862745098039,75.1764705882353,65.1529411764706
13.5288461538462 0.467170864343643 156.666666666667,90,78 156.666666666667,90,78
13.5288461538462 0.366013497114182 183.392156862745,105.352941176471,91.3058823529412 183.392156862745,105.352941176471,91.3058823529412
13.5288461538462 0.260251075029373 209.196078431373,120.176470588235,104.152941176471 209.196078431373,120.176470588235,104.152941176471
13.5288461538462 0.517615377902985 235,135,117 235,135,117
14.1730769230769 0.147898152470589 0,0,0 0,0,0
14.1730769230769 0.392569422721863 25.8039215686274,14.8235294117647,12.8470588235294 25.8039215686274,14.8235294117647,12.8470588235294
14.1730769230769 0.147306516766548 51.6078431372549,29.6470588235294,25.6941176470588 51.6078431372549,29.6470588235294,25.6941176470588
14.1730769230769 0.451515316963196 78.3333333333333,45,39 78.3333333333333,45,39
14.1730769230769 0.509858131408691 104.137254901961,59.8235294117647,51.8470588235294 104.137254901961,59.8235294117647,51.8470588235294
14.1730769230769 0.4457146525383 130.862745098039,75.1764705882353,65.1529411764706 130.862745098039,75.1764705882353,65.1529411764706
14.1730769230769 0.346548140048981 156.666666666667,90,78 156.666666666667,90,78
14.1730769230769 0.614377856254578 183.392156862745,105.352941176471,91.3058823529412 183.392156862745,105.352941176471,91.3058823529412
14.1730769230769 0.270308017730713 209.196078431373,120.176470588235,104.152941176471 209.196078431373,120.176470588235,104.152941176471
14.1730769230769 0.417016506195068 235,135,117 235,135,117
14.849358974359 0.438409239053726 0,0,0 0,0,0
14.849358974359 0.270187199115753 25.8039215686274,14.8235294117647,12.8470588235294 25.8039215686274,14.8235294117647,12.8470588235294
14.849358974359 0.139068409800529 51.6078431372549,29.6470588235294,25.6941176470588 51.6078431372549,29.6470588235294,25.6941176470588
14.849358974359 0.134812369942665 78.3333333333333,45,39 78.3333333333333,45,39
14.849358974359 0.352992415428162 104.137254901961,59.8235294117647,51.8470588235294 104.137254901961,59.8235294117647,51.8470588235294
14.849358974359 0.369162261486053 130.862745098039,75.1764705882353,65.1529411764706 130.862745098039,75.1764705882353,65.1529411764706
14.849358974359 0.281582832336426 156.666666666667,90,78 156.666666666667,90,78
14.849358974359 0.250238686800003 183.392156862745,105.352941176471,91.3058823529412 183.392156862745,105.352941176471,91.3058823529412
14.849358974359 0.152493461966515 209.196078431373,120.176470588235,104.152941176471 209.196078431373,120.176470588235,104.152941176471
14.849358974359 0.35007119178772 235,135,117 235,135,117
15.5544871794872 0.375635981559753 0,0,0 0,0,0
15.5544871794872 0.303830116987228 25.8039215686274,14.8235294117647,12.8470588235294 25.8039215686274,14.8235294117647,12.8470588235294
15.5544871794872 0.169918224215508 51.6078431372549,29.6470588235294,25.6941176470588 51.6078431372549,29.6470588235294,25.6941176470588
15.5544871794872 0.318712651729584 78.3333333333333,45,39 78.3333333333333,45,39
15.5544871794872 0.223490595817566 104.137254901961,59.8235294117647,51.8470588235294 104.137254901961,59.8235294117647,51.8470588235294
15.5544871794872 0.232829406857491 130.862745098039,75.1764705882353,65.1529411764706 130.862745098039,75.1764705882353,65.1529411764706
15.5544871794872 0.220886662602425 156.666666666667,90,78 156.666666666667,90,78
15.5544871794872 0.36919978260994 183.392156862745,105.352941176471,91.3058823529412 183.392156862745,105.352941176471,91.3058823529412
15.5544871794872 0.279243230819702 209.196078431373,120.176470588235,104.152941176471 209.196078431373,120.176470588235,104.152941176471
15.5544871794872 0.316171526908875 235,135,117 235,135,117
16.2948717948718 0.336017370223999 0,0,0 0,0,0
16.2948717948718 0.205741941928864 25.8039215686274,14.8235294117647,12.8470588235294 25.8039215686274,14.8235294117647,12.8470588235294
16.2948717948718 0.270815253257751 51.6078431372549,29.6470588235294,25.6941176470588 51.6078431372549,29.6470588235294,25.6941176470588
16.2948717948718 0.215436205267906 78.3333333333333,45,39 78.3333333333333,45,39
16.2948717948718 0.220368713140488 104.137254901961,59.8235294117647,51.8470588235294 104.137254901961,59.8235294117647,51.8470588235294
16.2948717948718 0.357287257909775 130.862745098039,75.1764705882353,65.1529411764706 130.862745098039,75.1764705882353,65.1529411764706
16.2948717948718 0.311433881521225 156.666666666667,90,78 156.666666666667,90,78
16.2948717948718 0.334268689155579 183.392156862745,105.352941176471,91.3058823529412 183.392156862745,105.352941176471,91.3058823529412
16.2948717948718 0.0516838133335114 209.196078431373,120.176470588235,104.152941176471 209.196078431373,120.176470588235,104.152941176471
16.2948717948718 0.194836392998695 235,135,117 235,135,117
17.0705128205128 0.466751962900162 0,0,0 0,0,0
17.0705128205128 0.36104679107666 25.8039215686274,14.8235294117647,12.8470588235294 25.8039215686274,14.8235294117647,12.8470588235294
17.0705128205128 0.0340119898319244 51.6078431372549,29.6470588235294,25.6941176470588 51.6078431372549,29.6470588235294,25.6941176470588
17.0705128205128 0.0929097235202789 78.3333333333333,45,39 78.3333333333333,45,39
17.0705128205128 0.554856956005096 104.137254901961,59.8235294117647,51.8470588235294 104.137254901961,59.8235294117647,51.8470588235294
17.0705128205128 0.307357758283615 130.862745098039,75.1764705882353,65.1529411764706 130.862745098039,75.1764705882353,65.1529411764706
17.0705128205128 0.389662057161331 156.666666666667,90,78 156.666666666667,90,78
17.0705128205128 0.227753654122353 183.392156862745,105.352941176471,91.3058823529412 183.392156862745,105.352941176471,91.3058823529412
17.0705128205128 0.227551057934761 209.196078431373,120.176470588235,104.152941176471 209.196078431373,120.176470588235,104.152941176471
17.0705128205128 0.415609657764435 235,135,117 235,135,117
17.8846153846154 0.552225887775421 0,0,0 0,0,0
17.8846153846154 0.243320181965828 25.8039215686274,14.8235294117647,12.8470588235294 25.8039215686274,14.8235294117647,12.8470588235294
17.8846153846154 0.533170521259308 51.6078431372549,29.6470588235294,25.6941176470588 51.6078431372549,29.6470588235294,25.6941176470588
17.8846153846154 0.219667300581932 78.3333333333333,45,39 78.3333333333333,45,39
17.8846153846154 0.207200229167938 104.137254901961,59.8235294117647,51.8470588235294 104.137254901961,59.8235294117647,51.8470588235294
17.8846153846154 0.441452324390411 130.862745098039,75.1764705882353,65.1529411764706 130.862745098039,75.1764705882353,65.1529411764706
17.8846153846154 0.400682836771011 156.666666666667,90,78 156.666666666667,90,78
17.8846153846154 0.545804858207703 183.392156862745,105.352941176471,91.3058823529412 183.392156862745,105.352941176471,91.3058823529412
17.8846153846154 0.176205024123192 209.196078431373,120.176470588235,104.152941176471 209.196078431373,120.176470588235,104.152941176471
17.8846153846154 0.22774963080883 235,135,117 235,135,117
18.7371794871795 0.315650254487991 0,0,0 0,0,0
18.7371794871795 0.239748045802116 25.8039215686274,14.8235294117647,12.8470588235294 25.8039215686274,14.8235294117647,12.8470588235294
18.7371794871795 0.179958537220955 51.6078431372549,29.6470588235294,25.6941176470588 51.6078431372549,29.6470588235294,25.6941176470588
18.7371794871795 0.211378902196884 78.3333333333333,45,39 78.3333333333333,45,39
18.7371794871795 0.289680898189545 104.137254901961,59.8235294117647,51.8470588235294 104.137254901961,59.8235294117647,51.8470588235294
18.7371794871795 0.312140107154846 130.862745098039,75.1764705882353,65.1529411764706 130.862745098039,75.1764705882353,65.1529411764706
18.7371794871795 0.260847717523575 156.666666666667,90,78 156.666666666667,90,78
18.7371794871795 0.228098526597023 183.392156862745,105.352941176471,91.3058823529412 183.392156862745,105.352941176471,91.3058823529412
18.7371794871795 0.138547718524933 209.196078431373,120.176470588235,104.152941176471 209.196078431373,120.176470588235,104.152941176471
18.7371794871795 0.157880410552025 235,135,117 235,135,117
19.6282051282051 0.351776659488678 0,0,0 0,0,0
19.6282051282051 0.273958504199982 25.8039215686274,14.8235294117647,12.8470588235294 25.8039215686274,14.8235294117647,12.8470588235294
19.6282051282051 0.168734595179558 51.6078431372549,29.6470588235294,25.6941176470588 51.6078431372549,29.6470588235294,25.6941176470588
19.6282051282051 0.12004042416811 78.3333333333333,45,39 78.3333333333333,45,39
19.6282051282051 0.453694969415665 104.137254901961,59.8235294117647,51.8470588235294 104.137254901961,59.8235294117647,51.8470588235294
19.6282051282051 0.226570695638657 130.862745098039,75.1764705882353,65.1529411764706 130.862745098039,75.1764705882353,65.1529411764706
19.6282051282051 0.149635702371597 156.666666666667,90,78 156.666666666667,90,78
19.6282051282051 0.0828923135995865 183.392156862745,105.352941176471,91.3058823529412 183.392156862745,105.352941176471,91.3058823529412
19.6282051282051 0.127526983618736 209.196078431373,120.176470588235,104.152941176471 209.196078431373,120.176470588235,104.152941176471
19.6282051282051 0.162903144955635 235,135,117 235,135,117
20.5641025641026 0.197232410311699 0,0,0 0,0,0
20.5641025641026 0.159589141607285 25.8039215686274,14.8235294117647,12.8470588235294 25.8039215686274,14.8235294117647,12.8470588235294
20.5641025641026 0.151643097400665 51.6078431372549,29.6470588235294,25.6941176470588 51.6078431372549,29.6470588235294,25.6941176470588
20.5641025641026 0.132980465888977 78.3333333333333,45,39 78.3333333333333,45,39
20.5641025641026 0.260623216629028 104.137254901961,59.8235294117647,51.8470588235294 104.137254901961,59.8235294117647,51.8470588235294
20.5641025641026 0.442532420158386 130.862745098039,75.1764705882353,65.1529411764706 130.862745098039,75.1764705882353,65.1529411764706
20.5641025641026 0.5415398478508 156.666666666667,90,78 156.666666666667,90,78
20.5641025641026 0.228515341877937 183.392156862745,105.352941176471,91.3058823529412 183.392156862745,105.352941176471,91.3058823529412
20.5641025641026 0.117429852485657 209.196078431373,120.176470588235,104.152941176471 209.196078431373,120.176470588235,104.152941176471
20.5641025641026 0.208738714456558 235,135,117 235,135,117
21.5416666666667 0.252879053354263 0,0,0 0,0,0
21.5416666666667 0.604028761386871 25.8039215686274,14.8235294117647,12.8470588235294 25.8039215686274,14.8235294117647,12.8470588235294
21.5416666666667 0.378509670495987 51.6078431372549,29.6470588235294,25.6941176470588 51.6078431372549,29.6470588235294,25.6941176470588
21.5416666666667 0.207104906439781 78.3333333333333,45,39 78.3333333333333,45,39
21.5416666666667 0.113723017275333 104.137254901961,59.8235294117647,51.8470588235294 104.137254901961,59.8235294117647,51.8470588235294
21.5416666666667 0.602006733417511 130.862745098039,75.1764705882353,65.1529411764706 130.862745098039,75.1764705882353,65.1529411764706
21.5416666666667 0.335463434457779 156.666666666667,90,78 156.666666666667,90,78
21.5416666666667 0.163503721356392 183.392156862745,105.352941176471,91.3058823529412 183.392156862745,105.352941176471,91.3058823529412
21.5416666666667 0.119618266820908 209.196078431373,120.176470588235,104.152941176471 209.196078431373,120.176470588235,104.152941176471
21.5416666666667 0.129860326647758 235,135,117 235,135,117
22.5673076923077 0.0646372213959694 0,0,0 0,0,0
22.5673076923077 0.178423672914505 25.8039215686274,14.8235294117647,12.8470588235294 25.8039215686274,14.8235294117647,12.8470588235294
22.5673076923077 0.114612556993961 51.6078431372549,29.6470588235294,25.6941176470588 51.6078431372549,29.6470588235294,25.6941176470588
22.5673076923077 0.0614951811730862 78.3333333333333,45,39 78.3333333333333,45,39
22.5673076923077 0.214148089289665 104.137254901961,59.8235294117647,51.8470588235294 104.137254901961,59.8235294117647,51.8470588235294
22.5673076923077 0.112378686666489 130.862745098039,75.1764705882353,65.1529411764706 130.862745098039,75.1764705882353,65.1529411764706
22.5673076923077 0.380411267280579 156.666666666667,90,78 156.666666666667,90,78
22.5673076923077 0.224997207522392 183.392156862745,105.352941176471,91.3058823529412 183.392156862745,105.352941176471,91.3058823529412
22.5673076923077 0.228796765208244 209.196078431373,120.176470588235,104.152941176471 209.196078431373,120.176470588235,104.152941176471
22.5673076923077 0.212361961603165 235,135,117 235,135,117
23.6442307692308 0.12228225171566 0,0,0 0,0,0
23.6442307692308 0.224083378911018 25.8039215686274,14.8235294117647,12.8470588235294 25.8039215686274,14.8235294117647,12.8470588235294
23.6442307692308 0.274574279785156 51.6078431372549,29.6470588235294,25.6941176470588 51.6078431372549,29.6470588235294,25.6941176470588
23.6442307692308 0.25410521030426 78.3333333333333,45,39 78.3333333333333,45,39
23.6442307692308 0.0759283974766731 104.137254901961,59.8235294117647,51.8470588235294 104.137254901961,59.8235294117647,51.8470588235294
23.6442307692308 0.329486578702927 130.862745098039,75.1764705882353,65.1529411764706 130.862745098039,75.1764705882353,65.1529411764706
23.6442307692308 0.325658529996872 156.666666666667,90,78 156.666666666667,90,78
23.6442307692308 0.220740586519241 183.392156862745,105.352941176471,91.3058823529412 183.392156862745,105.352941176471,91.3058823529412
23.6442307692308 0.086956113576889 209.196078431373,120.176470588235,104.152941176471 209.196078431373,120.176470588235,104.152941176471
23.6442307692308 0.0506661534309387 235,135,117 235,135,117
24.7692307692308 0.119727112352848 0,0,0 0,0,0
24.7692307692308 0.102132126688957 25.8039215686274,14.8235294117647,12.8470588235294 25.8039215686274,14.8235294117647,12.8470588235294
24.7692307692308 0.118928253650665 51.6078431372549,29.6470588235294,25.6941176470588 51.6078431372549,29.6470588235294,25.6941176470588
24.7692307692308 0.392418384552002 78.3333333333333,45,39 78.3333333333333,45,39
24.7692307692308 0.224873453378677 104.137254901961,59.8235294117647,51.8470588235294 104.137254901961,59.8235294117647,51.8470588235294
24.7692307692308 0.275590151548386 130.862745098039,75.1764705882353,65.1529411764706 130.862745098039,75.1764705882353,65.1529411764706
24.7692307692308 0.0926736369729042 156.666666666667,90,78 156.666666666667,90,78
24.7692307692308 0.240820869803429 183.392156862745,105.352941176471,91.3058823529412 183.392156862745,105.352941176471,91.3058823529412
24.7692307692308 0.18152329325676 209.196078431373,120.176470588235,104.152941176471 209.196078431373,120.176470588235,104.152941176471
24.7692307692308 0.0570325627923012 235,135,117 235,135,117
25.9487179487179 0.0506958775222301 0,0,0 0,0,0
25.9487179487179 0.10757390409708 25.8039215686274,14.8235294117647,12.8470588235294 25.8039215686274,14.8235294117647,12.8470588235294
25.9487179487179 0.222792968153954 51.6078431372549,29.6470588235294,25.6941176470588 51.6078431372549,29.6470588235294,25.6941176470588
25.9487179487179 0.283831149339676 78.3333333333333,45,39 78.3333333333333,45,39
25.9487179487179 0.127395942807198 104.137254901961,59.8235294117647,51.8470588235294 104.137254901961,59.8235294117647,51.8470588235294
25.9487179487179 0.286481857299805 130.862745098039,75.1764705882353,65.1529411764706 130.862745098039,75.1764705882353,65.1529411764706
25.9487179487179 0.222583249211311 156.666666666667,90,78 156.666666666667,90,78
25.9487179487179 0.198974683880806 183.392156862745,105.352941176471,91.3058823529412 183.392156862745,105.352941176471,91.3058823529412
25.9487179487179 0.112334348261356 209.196078431373,120.176470588235,104.152941176471 209.196078431373,120.176470588235,104.152941176471
25.9487179487179 0.353535354137421 235,135,117 235,135,117
27.1826923076923 0.14485040307045 0,0,0 0,0,0
27.1826923076923 0.230795383453369 25.8039215686274,14.8235294117647,12.8470588235294 25.8039215686274,14.8235294117647,12.8470588235294
27.1826923076923 0.200821235775948 51.6078431372549,29.6470588235294,25.6941176470588 51.6078431372549,29.6470588235294,25.6941176470588
27.1826923076923 0.130327269434929 78.3333333333333,45,39 78.3333333333333,45,39
27.1826923076923 0.120531186461449 104.137254901961,59.8235294117647,51.8470588235294 104.137254901961,59.8235294117647,51.8470588235294
27.1826923076923 0.239658087491989 130.862745098039,75.1764705882353,65.1529411764706 130.862745098039,75.1764705882353,65.1529411764706
27.1826923076923 0.0288295242935419 156.666666666667,90,78 156.666666666667,90,78
27.1826923076923 0.128580003976822 183.392156862745,105.352941176471,91.3058823529412 183.392156862745,105.352941176471,91.3058823529412
27.1826923076923 0.142585903406143 209.196078431373,120.176470588235,104.152941176471 209.196078431373,120.176470588235,104.152941176471
27.1826923076923 0.238067969679832 235,135,117 235,135,117
28.4775641025641 0.0822729468345642 0,0,0 0,0,0
28.4775641025641 0.126973271369934 25.8039215686274,14.8235294117647,12.8470588235294 25.8039215686274,14.8235294117647,12.8470588235294
28.4775641025641 0.304915100336075 51.6078431372549,29.6470588235294,25.6941176470588 51.6078431372549,29.6470588235294,25.6941176470588
28.4775641025641 0.227368086576462 78.3333333333333,45,39 78.3333333333333,45,39
28.4775641025641 0.241228774189949 104.137254901961,59.8235294117647,51.8470588235294 104.137254901961,59.8235294117647,51.8470588235294
28.4775641025641 0.0380346029996872 130.862745098039,75.1764705882353,65.1529411764706 130.862745098039,75.1764705882353,65.1529411764706
28.4775641025641 0.13489992916584 156.666666666667,90,78 156.666666666667,90,78
28.4775641025641 0.0865603685379028 183.392156862745,105.352941176471,91.3058823529412 183.392156862745,105.352941176471,91.3058823529412
28.4775641025641 0.19671793282032 209.196078431373,120.176470588235,104.152941176471 209.196078431373,120.176470588235,104.152941176471
28.4775641025641 0.0473198182880878 235,135,117 235,135,117
29.8333333333333 0.109053455293179 0,0,0 0,0,0
29.8333333333333 0.0378567092120647 25.8039215686274,14.8235294117647,12.8470588235294 25.8039215686274,14.8235294117647,12.8470588235294
29.8333333333333 0.326532304286957 51.6078431372549,29.6470588235294,25.6941176470588 51.6078431372549,29.6470588235294,25.6941176470588
29.8333333333333 0.19535069167614 78.3333333333333,45,39 78.3333333333333,45,39
29.8333333333333 0.126152515411377 104.137254901961,59.8235294117647,51.8470588235294 104.137254901961,59.8235294117647,51.8470588235294
29.8333333333333 0.153328225016594 130.862745098039,75.1764705882353,65.1529411764706 130.862745098039,75.1764705882353,65.1529411764706
29.8333333333333 0.0422515161335468 156.666666666667,90,78 156.666666666667,90,78
29.8333333333333 0.157725185155869 183.392156862745,105.352941176471,91.3058823529412 183.392156862745,105.352941176471,91.3058823529412
29.8333333333333 0.147372007369995 209.196078431373,120.176470588235,104.152941176471 209.196078431373,120.176470588235,104.152941176471
29.8333333333333 0.0441205538809299 235,135,117 235,135,117
31.2564102564103 0.0705294758081436 0,0,0 0,0,0
31.2564102564103 0.0804430693387985 25.8039215686274,14.8235294117647,12.8470588235294 25.8039215686274,14.8235294117647,12.8470588235294
31.2564102564103 0.356323778629303 51.6078431372549,29.6470588235294,25.6941176470588 51.6078431372549,29.6470588235294,25.6941176470588
31.2564102564103 0.254549533128738 78.3333333333333,45,39 78.3333333333333,45,39
31.2564102564103 0.170110508799553 104.137254901961,59.8235294117647,51.8470588235294 104.137254901961,59.8235294117647,51.8470588235294
31.2564102564103 0.127747893333435 130.862745098039,75.1764705882353,65.1529411764706 130.862745098039,75.1764705882353,65.1529411764706
31.2564102564103 0.124140240252018 156.666666666667,90,78 156.666666666667,90,78
31.2564102564103 0.163027718663216 183.392156862745,105.352941176471,91.3058823529412 183.392156862745,105.352941176471,91.3058823529412
31.2564102564103 0.0449127443134785 209.196078431373,120.176470588235,104.152941176471 209.196078431373,120.176470588235,104.152941176471
31.2564102564103 0.0825785920023918 235,135,117 235,135,117
32.7435897435897 0.191301763057709 0,0,0 0,0,0
32.7435897435897 0.194513127207756 25.8039215686274,14.8235294117647,12.8470588235294 25.8039215686274,14.8235294117647,12.8470588235294
32.7435897435897 0.0770542398095131 51.6078431372549,29.6470588235294,25.6941176470588 51.6078431372549,29.6470588235294,25.6941176470588
32.7435897435897 0.200528010725975 78.3333333333333,45,39 78.3333333333333,45,39
32.7435897435897 0.133365616202354 104.137254901961,59.8235294117647,51.8470588235294 104.137254901961,59.8235294117647,51.8470588235294
32.7435897435897 0.0466050319373608 130.862745098039,75.1764705882353,65.1529411764706 130.862745098039,75.1764705882353,65.1529411764706
32.7435897435897 0.162162408232689 156.666666666667,90,78 156.666666666667,90,78
32.7435897435897 0.168131917715073 183.392156862745,105.352941176471,91.3058823529412 183.392156862745,105.352941176471,91.3058823529412
32.7435897435897 0.123800598084927 209.196078431373,120.176470588235,104.152941176471 209.196078431373,120.176470588235,104.152941176471
32.7435897435897 0.0978487059473991 235,135,117 235,135,117
34.3044871794872 0.0787762030959129 0,0,0 0,0,0
34.3044871794872 0.14290851354599 25.8039215686274,14.8235294117647,12.8470588235294 25.8039215686274,14.8235294117647,12.8470588235294
34.3044871794872 0.113800406455994 51.6078431372549,29.6470588235294,25.6941176470588 51.6078431372549,29.6470588235294,25.6941176470588
34.3044871794872 0.246655121445656 78.3333333333333,45,39 78.3333333333333,45,39
34.3044871794872 0.232390269637108 104.137254901961,59.8235294117647,51.8470588235294 104.137254901961,59.8235294117647,51.8470588235294
34.3044871794872 0.202954471111298 130.862745098039,75.1764705882353,65.1529411764706 130.862745098039,75.1764705882353,65.1529411764706
34.3044871794872 0.199400752782822 156.666666666667,90,78 156.666666666667,90,78
34.3044871794872 0.116207957267761 183.392156862745,105.352941176471,91.3058823529412 183.392156862745,105.352941176471,91.3058823529412
34.3044871794872 0.204152628779411 209.196078431373,120.176470588235,104.152941176471 209.196078431373,120.176470588235,104.152941176471
34.3044871794872 0.0153717948123813 235,135,117 235,135,117
35.9358974358974 0.0870254933834076 0,0,0 0,0,0
35.9358974358974 0.0683745220303535 25.8039215686274,14.8235294117647,12.8470588235294 25.8039215686274,14.8235294117647,12.8470588235294
35.9358974358974 0.079672209918499 51.6078431372549,29.6470588235294,25.6941176470588 51.6078431372549,29.6470588235294,25.6941176470588
35.9358974358974 0.246928170323372 78.3333333333333,45,39 78.3333333333333,45,39
35.9358974358974 0.0935479253530502 104.137254901961,59.8235294117647,51.8470588235294 104.137254901961,59.8235294117647,51.8470588235294
35.9358974358974 0.0673164650797844 130.862745098039,75.1764705882353,65.1529411764706 130.862745098039,75.1764705882353,65.1529411764706
35.9358974358974 0.133186683058739 156.666666666667,90,78 156.666666666667,90,78
35.9358974358974 0.164324730634689 183.392156862745,105.352941176471,91.3058823529412 183.392156862745,105.352941176471,91.3058823529412
35.9358974358974 0.0227599907666445 209.196078431373,120.176470588235,104.152941176471 209.196078431373,120.176470588235,104.152941176471
35.9358974358974 0.0935291722416878 235,135,117 235,135,117
37.6474358974359 0.0599369630217552 0,0,0 0,0,0
37.6474358974359 0.182949036359787 25.8039215686274,14.8235294117647,12.8470588235294 25.8039215686274,14.8235294117647,12.8470588235294
37.6474358974359 0.0820496678352356 51.6078431372549,29.6470588235294,25.6941176470588 51.6078431372549,29.6470588235294,25.6941176470588
37.6474358974359 0.130189165472984 78.3333333333333,45,39 78.3333333333333,45,39
37.6474358974359 0.152782499790192 104.137254901961,59.8235294117647,51.8470588235294 104.137254901961,59.8235294117647,51.8470588235294
37.6474358974359 0.221665799617767 130.862745098039,75.1764705882353,65.1529411764706 130.862745098039,75.1764705882353,65.1529411764706
37.6474358974359 0.101278856396675 156.666666666667,90,78 156.666666666667,90,78
37.6474358974359 0.0819317251443863 183.392156862745,105.352941176471,91.3058823529412 183.392156862745,105.352941176471,91.3058823529412
37.6474358974359 0.07819963991642 209.196078431373,120.176470588235,104.152941176471 209.196078431373,120.176470588235,104.152941176471
37.6474358974359 0.0840277746319771 235,135,117 235,135,117
39.4391025641026 0.228082180023193 0,0,0 0,0,0
39.4391025641026 0.082749143242836 25.8039215686274,14.8235294117647,12.8470588235294 25.8039215686274,14.8235294117647,12.8470588235294
39.4391025641026 0.154853343963623 51.6078431372549,29.6470588235294,25.6941176470588 51.6078431372549,29.6470588235294,25.6941176470588
39.4391025641026 0.0796608626842499 78.3333333333333,45,39 78.3333333333333,45,39
39.4391025641026 0.156124800443649 104.137254901961,59.8235294117647,51.8470588235294 104.137254901961,59.8235294117647,51.8470588235294
39.4391025641026 0.154243722558022 130.862745098039,75.1764705882353,65.1529411764706 130.862745098039,75.1764705882353,65.1529411764706
39.4391025641026 0.0511416718363762 156.666666666667,90,78 156.666666666667,90,78
39.4391025641026 0.0464367642998695 183.392156862745,105.352941176471,91.3058823529412 183.392156862745,105.352941176471,91.3058823529412
39.4391025641026 0.171126991510391 209.196078431373,120.176470588235,104.152941176471 209.196078431373,120.176470588235,104.152941176471
39.4391025641026 0.0410487428307533 235,135,117 235,135,117
41.3173076923077 0.0728199481964111 0,0,0 0,0,0
41.3173076923077 0.0356121882796288 25.8039215686274,14.8235294117647,12.8470588235294 25.8039215686274,14.8235294117647,12.8470588235294
41.3173076923077 0.147451981902122 51.6078431372549,29.6470588235294,25.6941176470588 51.6078431372549,29.6470588235294,25.6941176470588
41.3173076923077 0.119195900857449 78.3333333333333,45,39 78.3333333333333,45,39
41.3173076923077 0.131420403718948 104.137254901961,59.8235294117647,51.8470588235294 104.137254901961,59.8235294117647,51.8470588235294
41.3173076923077 0.137727826833725 130.862745098039,75.1764705882353,65.1529411764706 130.862745098039,75.1764705882353,65.1529411764706
41.3173076923077 0.0394698902964592 156.666666666667,90,78 156.666666666667,90,78
41.3173076923077 0.0881710574030876 183.392156862745,105.352941176471,91.3058823529412 183.392156862745,105.352941176471,91.3058823529412
41.3173076923077 0.0417782925069332 209.196078431373,120.176470588235,104.152941176471 209.196078431373,120.176470588235,104.152941176471
41.3173076923077 0.0560717098414898 235,135,117 235,135,117
43.2852564102564 0.0814539641141891 0,0,0 0,0,0
43.2852564102564 0.0906740203499794 25.8039215686274,14.8235294117647,12.8470588235294 25.8039215686274,14.8235294117647,12.8470588235294
43.2852564102564 0.0582966394722462 51.6078431372549,29.6470588235294,25.6941176470588 51.6078431372549,29.6470588235294,25.6941176470588
43.2852564102564 0.168130055069923 78.3333333333333,45,39 78.3333333333333,45,39
43.2852564102564 0.13024590909481 104.137254901961,59.8235294117647,51.8470588235294 104.137254901961,59.8235294117647,51.8470588235294
43.2852564102564 0.0479000806808472 130.862745098039,75.1764705882353,65.1529411764706 130.862745098039,75.1764705882353,65.1529411764706
43.2852564102564 0.106295309960842 156.666666666667,90,78 156.666666666667,90,78
43.2852564102564 0.102547593414783 183.392156862745,105.352941176471,91.3058823529412 183.392156862745,105.352941176471,91.3058823529412
43.2852564102564 0.157536342740059 209.196078431373,120.176470588235,104.152941176471 209.196078431373,120.176470588235,104.152941176471
43.2852564102564 0.108162567019463 235,135,117 235,135,117
45.3461538461538 0.135500684380531 0,0,0 0,0,0
45.3461538461538 0.0712881237268448 25.8039215686274,14.8235294117647,12.8470588235294 25.8039215686274,14.8235294117647,12.8470588235294
45.3461538461538 0.108187831938267 51.6078431372549,29.6470588235294,25.6941176470588 51.6078431372549,29.6470588235294,25.6941176470588
45.3461538461538 0.0746166184544563 78.3333333333333,45,39 78.3333333333333,45,39
45.3461538461538 0.0936126485466957 104.137254901961,59.8235294117647,51.8470588235294 104.137254901961,59.8235294117647,51.8470588235294
45.3461538461538 0.0503127612173557 130.862745098039,75.1764705882353,65.1529411764706 130.862745098039,75.1764705882353,65.1529411764706
45.3461538461538 0.129724636673927 156.666666666667,90,78 156.666666666667,90,78
45.3461538461538 0.0677567273378372 183.392156862745,105.352941176471,91.3058823529412 183.392156862745,105.352941176471,91.3058823529412
45.3461538461538 0.133995831012726 209.196078431373,120.176470588235,104.152941176471 209.196078431373,120.176470588235,104.152941176471
45.3461538461538 0.0775809586048126 235,135,117 235,135,117
47.5064102564103 0.0879152417182922 0,0,0 0,0,0
47.5064102564103 0.0613650940358639 25.8039215686274,14.8235294117647,12.8470588235294 25.8039215686274,14.8235294117647,12.8470588235294
47.5064102564103 0.0766095370054245 51.6078431372549,29.6470588235294,25.6941176470588 51.6078431372549,29.6470588235294,25.6941176470588
47.5064102564103 0.173701852560043 78.3333333333333,45,39 78.3333333333333,45,39
47.5064102564103 0.0641912519931793 104.137254901961,59.8235294117647,51.8470588235294 104.137254901961,59.8235294117647,51.8470588235294
47.5064102564103 0.0168549977242947 130.862745098039,75.1764705882353,65.1529411764706 130.862745098039,75.1764705882353,65.1529411764706
47.5064102564103 0.124533459544182 156.666666666667,90,78 156.666666666667,90,78
47.5064102564103 0.0628927573561668 183.392156862745,105.352941176471,91.3058823529412 183.392156862745,105.352941176471,91.3058823529412
47.5064102564103 0.0759809091687202 209.196078431373,120.176470588235,104.152941176471 209.196078431373,120.176470588235,104.152941176471
47.5064102564103 0.0583058446645737 235,135,117 235,135,117
49.7692307692308 0.0775977224111557 0,0,0 0,0,0
49.7692307692308 0.123262368142605 25.8039215686274,14.8235294117647,12.8470588235294 25.8039215686274,14.8235294117647,12.8470588235294
49.7692307692308 0.186306223273277 51.6078431372549,29.6470588235294,25.6941176470588 51.6078431372549,29.6470588235294,25.6941176470588
49.7692307692308 0.187439873814583 78.3333333333333,45,39 78.3333333333333,45,39
49.7692307692308 0.12071692943573 104.137254901961,59.8235294117647,51.8470588235294 104.137254901961,59.8235294117647,51.8470588235294
49.7692307692308 0.0337477289140224 130.862745098039,75.1764705882353,65.1529411764706 130.862745098039,75.1764705882353,65.1529411764706
49.7692307692308 0.0982203707098961 156.666666666667,90,78 156.666666666667,90,78
49.7692307692308 0.188566669821739 183.392156862745,105.352941176471,91.3058823529412 183.392156862745,105.352941176471,91.3058823529412
49.7692307692308 0.0367709398269653 209.196078431373,120.176470588235,104.152941176471 209.196078431373,120.176470588235,104.152941176471
49.7692307692308 0.0213430784642696 235,135,117 235,135,117
52.1378205128205 0.0804963409900665 0,0,0 0,0,0
52.1378205128205 0.106464222073555 25.8039215686274,14.8235294117647,12.8470588235294 25.8039215686274,14.8235294117647,12.8470588235294
52.1378205128205 0.11439548432827 51.6078431372549,29.6470588235294,25.6941176470588 51.6078431372549,29.6470588235294,25.6941176470588
52.1378205128205 0.0422885678708553 78.3333333333333,45,39 78.3333333333333,45,39
52.1378205128205 0.149507746100426 104.137254901961,59.8235294117647,51.8470588235294 104.137254901961,59.8235294117647,51.8470588235294
52.1378205128205 0.0772893205285072 130.862745098039,75.1764705882353,65.1529411764706 130.862745098039,75.1764705882353,65.1529411764706
52.1378205128205 0.118644796311855 156.666666666667,90,78 156.666666666667,90,78
52.1378205128205 0.0764889419078827 183.392156862745,105.352941176471,91.3058823529412 183.392156862745,105.352941176471,91.3058823529412
52.1378205128205 0.0492008626461029 209.196078431373,120.176470588235,104.152941176471 209.196078431373,120.176470588235,104.152941176471
52.1378205128205 0.0648493096232414 235,135,117 235,135,117
54.6217948717949 0.0774566456675529 0,0,0 0,0,0
54.6217948717949 0.125383287668228 25.8039215686274,14.8235294117647,12.8470588235294 25.8039215686274,14.8235294117647,12.8470588235294
54.6217948717949 0.140533328056335 51.6078431372549,29.6470588235294,25.6941176470588 51.6078431372549,29.6470588235294,25.6941176470588
54.6217948717949 0.167792648077011 78.3333333333333,45,39 78.3333333333333,45,39
54.6217948717949 0.068349651992321 104.137254901961,59.8235294117647,51.8470588235294 104.137254901961,59.8235294117647,51.8470588235294
54.6217948717949 0.0571270473301411 130.862745098039,75.1764705882353,65.1529411764706 130.862745098039,75.1764705882353,65.1529411764706
54.6217948717949 0.134534001350403 156.666666666667,90,78 156.666666666667,90,78
54.6217948717949 0.0864002481102943 183.392156862745,105.352941176471,91.3058823529412 183.392156862745,105.352941176471,91.3058823529412
54.6217948717949 0.0753994882106781 209.196078431373,120.176470588235,104.152941176471 209.196078431373,120.176470588235,104.152941176471
54.6217948717949 0.0598373003304005 235,135,117 235,135,117
57.2211538461538 0.0515536442399025 0,0,0 0,0,0
57.2211538461538 0.0686461105942726 25.8039215686274,14.8235294117647,12.8470588235294 25.8039215686274,14.8235294117647,12.8470588235294
57.2211538461538 0.100483745336533 51.6078431372549,29.6470588235294,25.6941176470588 51.6078431372549,29.6470588235294,25.6941176470588
57.2211538461538 0.0865549743175507 78.3333333333333,45,39 78.3333333333333,45,39
57.2211538461538 0.0600200407207012 104.137254901961,59.8235294117647,51.8470588235294 104.137254901961,59.8235294117647,51.8470588235294
57.2211538461538 0.036415446549654 130.862745098039,75.1764705882353,65.1529411764706 130.862745098039,75.1764705882353,65.1529411764706
57.2211538461538 0.0372415035963058 156.666666666667,90,78 156.666666666667,90,78
57.2211538461538 0.0452808775007725 183.392156862745,105.352941176471,91.3058823529412 183.392156862745,105.352941176471,91.3058823529412
57.2211538461538 0.0991490855813026 209.196078431373,120.176470588235,104.152941176471 209.196078431373,120.176470588235,104.152941176471
57.2211538461538 0.0101491622626781 235,135,117 235,135,117
59.9455128205128 0.143939718604088 0,0,0 0,0,0
59.9455128205128 0.0591123476624489 25.8039215686274,14.8235294117647,12.8470588235294 25.8039215686274,14.8235294117647,12.8470588235294
59.9455128205128 0.202153012156487 51.6078431372549,29.6470588235294,25.6941176470588 51.6078431372549,29.6470588235294,25.6941176470588
59.9455128205128 0.0693584382534027 78.3333333333333,45,39 78.3333333333333,45,39
59.9455128205128 0.0501086339354515 104.137254901961,59.8235294117647,51.8470588235294 104.137254901961,59.8235294117647,51.8470588235294
59.9455128205128 0.0273427311331034 130.862745098039,75.1764705882353,65.1529411764706 130.862745098039,75.1764705882353,65.1529411764706
59.9455128205128 0.141048192977905 156.666666666667,90,78 156.666666666667,90,78
59.9455128205128 0.0754390954971313 183.392156862745,105.352941176471,91.3058823529412 183.392156862745,105.352941176471,91.3058823529412
59.9455128205128 0.0333353839814663 209.196078431373,120.176470588235,104.152941176471 209.196078431373,120.176470588235,104.152941176471
59.9455128205128 0.0642862692475319 235,135,117 235,135,117
62.8012820512821 0.0407466702163219 0,0,0 0,0,0
62.8012820512821 0.045787263661623 25.8039215686274,14.8235294117647,12.8470588235294 25.8039215686274,14.8235294117647,12.8470588235294
62.8012820512821 0.034728717058897 51.6078431372549,29.6470588235294,25.6941176470588 51.6078431372549,29.6470588235294,25.6941176470588
62.8012820512821 0.0885457172989845 78.3333333333333,45,39 78.3333333333333,45,39
62.8012820512821 0.0272695533931255 104.137254901961,59.8235294117647,51.8470588235294 104.137254901961,59.8235294117647,51.8470588235294
62.8012820512821 0.0464835539460182 130.862745098039,75.1764705882353,65.1529411764706 130.862745098039,75.1764705882353,65.1529411764706
62.8012820512821 0.0209010522812605 156.666666666667,90,78 156.666666666667,90,78
62.8012820512821 0.0715994611382484 183.392156862745,105.352941176471,91.3058823529412 183.392156862745,105.352941176471,91.3058823529412
62.8012820512821 0.0113705527037382 209.196078431373,120.176470588235,104.152941176471 209.196078431373,120.176470588235,104.152941176471
62.8012820512821 0.0173634625971317 235,135,117 235,135,117
65.7916666666667 0.111600190401077 0,0,0 0,0,0
65.7916666666667 0.091223232448101 25.8039215686274,14.8235294117647,12.8470588235294 25.8039215686274,14.8235294117647,12.8470588235294
65.7916666666667 0.0637151226401329 51.6078431372549,29.6470588235294,25.6941176470588 51.6078431372549,29.6470588235294,25.6941176470588
65.7916666666667 0.0900863036513329 78.3333333333333,45,39 78.3333333333333,45,39
65.7916666666667 0.0330133438110352 104.137254901961,59.8235294117647,51.8470588235294 104.137254901961,59.8235294117647,51.8470588235294
65.7916666666667 0.0520493946969509 130.862745098039,75.1764705882353,65.1529411764706 130.862745098039,75.1764705882353,65.1529411764706
65.7916666666667 0.0569294653832912 156.666666666667,90,78 156.666666666667,90,78
65.7916666666667 0.0382050834596157 183.392156862745,105.352941176471,91.3058823529412 183.392156862745,105.352941176471,91.3058823529412
65.7916666666667 0.0557978786528111 209.196078431373,120.176470588235,104.152941176471 209.196078431373,120.176470588235,104.152941176471
65.7916666666667 0.0111792730167508 235,135,117 235,135,117
68.9230769230769 0.0401521660387516 0,0,0 0,0,0
68.9230769230769 0.146928250789642 25.8039215686274,14.8235294117647,12.8470588235294 25.8039215686274,14.8235294117647,12.8470588235294
68.9230769230769 0.0615935809910297 51.6078431372549,29.6470588235294,25.6941176470588 51.6078431372549,29.6470588235294,25.6941176470588
68.9230769230769 0.0334985330700874 78.3333333333333,45,39 78.3333333333333,45,39
68.9230769230769 0.0353465899825096 104.137254901961,59.8235294117647,51.8470588235294 104.137254901961,59.8235294117647,51.8470588235294
68.9230769230769 0.0955769717693329 130.862745098039,75.1764705882353,65.1529411764706 130.862745098039,75.1764705882353,65.1529411764706
68.9230769230769 0.0461437478661537 156.666666666667,90,78 156.666666666667,90,78
68.9230769230769 0.0568266697227955 183.392156862745,105.352941176471,91.3058823529412 183.392156862745,105.352941176471,91.3058823529412
68.9230769230769 0.0431615747511387 209.196078431373,120.176470588235,104.152941176471 209.196078431373,120.176470588235,104.152941176471
68.9230769230769 0.0128111308440566 235,135,117 235,135,117
72.2051282051282 0.0376274436712265 0,0,0 0,0,0
72.2051282051282 0.0410486347973347 25.8039215686274,14.8235294117647,12.8470588235294 25.8039215686274,14.8235294117647,12.8470588235294
72.2051282051282 0.0303088519722223 51.6078431372549,29.6470588235294,25.6941176470588 51.6078431372549,29.6470588235294,25.6941176470588
72.2051282051282 0.0483927689492702 78.3333333333333,45,39 78.3333333333333,45,39
72.2051282051282 0.0579156242311001 104.137254901961,59.8235294117647,51.8470588235294 104.137254901961,59.8235294117647,51.8470588235294
72.2051282051282 0.0739426836371422 130.862745098039,75.1764705882353,65.1529411764706 130.862745098039,75.1764705882353,65.1529411764706
72.2051282051282 0.0441956296563148 156.666666666667,90,78 156.666666666667,90,78
72.2051282051282 0.0138805769383907 183.392156862745,105.352941176471,91.3058823529412 183.392156862745,105.352941176471,91.3058823529412
72.2051282051282 0.0229110848158598 209.196078431373,120.176470588235,104.152941176471 209.196078431373,120.176470588235,104.152941176471
72.2051282051282 0.0215059034526348 235,135,117 235,135,117
75.6442307692308 0.0416424721479416 0,0,0 0,0,0
75.6442307692308 0.110561817884445 25.8039215686274,14.8235294117647,12.8470588235294 25.8039215686274,14.8235294117647,12.8470588235294
75.6442307692308 0.0273461546748877 51.6078431372549,29.6470588235294,25.6941176470588 51.6078431372549,29.6470588235294,25.6941176470588
75.6442307692308 0.0279228612780571 78.3333333333333,45,39 78.3333333333333,45,39
75.6442307692308 0.0270750112831593 104.137254901961,59.8235294117647,51.8470588235294 104.137254901961,59.8235294117647,51.8470588235294
75.6442307692308 0.0827696323394775 130.862745098039,75.1764705882353,65.1529411764706 130.862745098039,75.1764705882353,65.1529411764706
75.6442307692308 0.043198000639677 156.666666666667,90,78 156.666666666667,90,78
75.6442307692308 0.0489821918308735 183.392156862745,105.352941176471,91.3058823529412 183.392156862745,105.352941176471,91.3058823529412
75.6442307692308 0.0161643028259277 209.196078431373,120.176470588235,104.152941176471 209.196078431373,120.176470588235,104.152941176471
75.6442307692308 0.0331812165677547 235,135,117 235,135,117
79.2467948717949 0.0331645645201206 0,0,0 0,0,0
79.2467948717949 0.0627862364053726 25.8039215686274,14.8235294117647,12.8470588235294 25.8039215686274,14.8235294117647,12.8470588235294
79.2467948717949 0.0783738866448402 51.6078431372549,29.6470588235294,25.6941176470588 51.6078431372549,29.6470588235294,25.6941176470588
79.2467948717949 0.0606013387441635 78.3333333333333,45,39 78.3333333333333,45,39
79.2467948717949 0.0259740501642227 104.137254901961,59.8235294117647,51.8470588235294 104.137254901961,59.8235294117647,51.8470588235294
79.2467948717949 0.0151487151160836 130.862745098039,75.1764705882353,65.1529411764706 130.862745098039,75.1764705882353,65.1529411764706
79.2467948717949 0.0449281744658947 156.666666666667,90,78 156.666666666667,90,78
79.2467948717949 0.0170947462320328 183.392156862745,105.352941176471,91.3058823529412 183.392156862745,105.352941176471,91.3058823529412
79.2467948717949 0.0161969531327486 209.196078431373,120.176470588235,104.152941176471 209.196078431373,120.176470588235,104.152941176471
79.2467948717949 0.00977244228124619 235,135,117 235,135,117
83.0192307692308 0.0545338839292526 0,0,0 0,0,0
83.0192307692308 0.0130007080733776 25.8039215686274,14.8235294117647,12.8470588235294 25.8039215686274,14.8235294117647,12.8470588235294
83.0192307692308 0.0465932562947273 51.6078431372549,29.6470588235294,25.6941176470588 51.6078431372549,29.6470588235294,25.6941176470588
83.0192307692308 0.0189707912504673 78.3333333333333,45,39 78.3333333333333,45,39
83.0192307692308 0.0228762924671173 104.137254901961,59.8235294117647,51.8470588235294 104.137254901961,59.8235294117647,51.8470588235294
83.0192307692308 0.0304686296731234 130.862745098039,75.1764705882353,65.1529411764706 130.862745098039,75.1764705882353,65.1529411764706
83.0192307692308 0.039107047021389 156.666666666667,90,78 156.666666666667,90,78
83.0192307692308 0.0128621403127909 183.392156862745,105.352941176471,91.3058823529412 183.392156862745,105.352941176471,91.3058823529412
83.0192307692308 0.0182233229279518 209.196078431373,120.176470588235,104.152941176471 209.196078431373,120.176470588235,104.152941176471
83.0192307692308 0.00825722981244326 235,135,117 235,135,117
86.974358974359 0.0498495772480965 0,0,0 0,0,0
86.974358974359 0.039225846529007 25.8039215686274,14.8235294117647,12.8470588235294 25.8039215686274,14.8235294117647,12.8470588235294
86.974358974359 0.0489772744476795 51.6078431372549,29.6470588235294,25.6941176470588 51.6078431372549,29.6470588235294,25.6941176470588
86.974358974359 0.0757955461740494 78.3333333333333,45,39 78.3333333333333,45,39
86.974358974359 0.0306103881448507 104.137254901961,59.8235294117647,51.8470588235294 104.137254901961,59.8235294117647,51.8470588235294
86.974358974359 0.0237225517630577 130.862745098039,75.1764705882353,65.1529411764706 130.862745098039,75.1764705882353,65.1529411764706
86.974358974359 0.0197017565369606 156.666666666667,90,78 156.666666666667,90,78
86.974358974359 0.0649831593036652 183.392156862745,105.352941176471,91.3058823529412 183.392156862745,105.352941176471,91.3058823529412
86.974358974359 0.0199153255671263 209.196078431373,120.176470588235,104.152941176471 209.196078431373,120.176470588235,104.152941176471
86.974358974359 0.00862924661487341 235,135,117 235,135,117
91.1153846153846 0.0731098502874374 0,0,0 0,0,0
91.1153846153846 0.0323996990919113 25.8039215686274,14.8235294117647,12.8470588235294 25.8039215686274,14.8235294117647,12.8470588235294
91.1153846153846 0.0321961976587772 51.6078431372549,29.6470588235294,25.6941176470588 51.6078431372549,29.6470588235294,25.6941176470588
91.1153846153846 0.0376484394073486 78.3333333333333,45,39 78.3333333333333,45,39
91.1153846153846 0.0263667125254869 104.137254901961,59.8235294117647,51.8470588235294 104.137254901961,59.8235294117647,51.8470588235294
91.1153846153846 0.0418588779866695 130.862745098039,75.1764705882353,65.1529411764706 130.862745098039,75.1764705882353,65.1529411764706
91.1153846153846 0.038524117320776 156.666666666667,90,78 156.666666666667,90,78
91.1153846153846 0.00932418648153543 183.392156862745,105.352941176471,91.3058823529412 183.392156862745,105.352941176471,91.3058823529412
91.1153846153846 0.0101931672543287 209.196078431373,120.176470588235,104.152941176471 209.196078431373,120.176470588235,104.152941176471
91.1153846153846 0.00892436970025301 235,135,117 235,135,117
95.4519230769231 0.0384651608765125 0,0,0 0,0,0
95.4519230769231 0.0212954841554165 25.8039215686274,14.8235294117647,12.8470588235294 25.8039215686274,14.8235294117647,12.8470588235294
95.4519230769231 0.0639208257198334 51.6078431372549,29.6470588235294,25.6941176470588 51.6078431372549,29.6470588235294,25.6941176470588
95.4519230769231 0.0118824178352952 78.3333333333333,45,39 78.3333333333333,45,39
95.4519230769231 0.0212286729365587 104.137254901961,59.8235294117647,51.8470588235294 104.137254901961,59.8235294117647,51.8470588235294
95.4519230769231 0.0145912701264024 130.862745098039,75.1764705882353,65.1529411764706 130.862745098039,75.1764705882353,65.1529411764706
95.4519230769231 0.087324395775795 156.666666666667,90,78 156.666666666667,90,78
95.4519230769231 0.0737019032239914 183.392156862745,105.352941176471,91.3058823529412 183.392156862745,105.352941176471,91.3058823529412
95.4519230769231 0.0112798046320677 209.196078431373,120.176470588235,104.152941176471 209.196078431373,120.176470588235,104.152941176471
95.4519230769231 0.0109028639271855 235,135,117 235,135,117
100 0.0364880599081516 0,0,0 0,0,0
100 0.0200623665004969 25.8039215686274,14.8235294117647,12.8470588235294 25.8039215686274,14.8235294117647,12.8470588235294
100 0.0419906713068485 51.6078431372549,29.6470588235294,25.6941176470588 51.6078431372549,29.6470588235294,25.6941176470588
100 0.034723736345768 78.3333333333333,45,39 78.3333333333333,45,39
100 0.0528703331947327 104.137254901961,59.8235294117647,51.8470588235294 104.137254901961,59.8235294117647,51.8470588235294
100 0.029904393479228 130.862745098039,75.1764705882353,65.1529411764706 130.862745098039,75.1764705882353,65.1529411764706
100 0.0263231229037046 156.666666666667,90,78 156.666666666667,90,78
100 0.0245629977434874 183.392156862745,105.352941176471,91.3058823529412 183.392156862745,105.352941176471,91.3058823529412
100 0.00935054570436478 209.196078431373,120.176470588235,104.152941176471 209.196078431373,120.176470588235,104.152941176471
100 0.00808566436171532 235,135,117 235,135,117
};
\end{axis}

\end{tikzpicture}

  \tikzexternaldisable
  \caption{ \textbf{Directional curvature SNRs:} Curvature SNRs along each of
    the mini-batch \ggn{}'s top-$C$ eigenvectors during training of the
    \threecthreed network on \cifarten with \sgd{}. At fixed epoch, the SNR for
    the most curved direction is shown in
    {\protect\tikz{\protect\draw[white,fill={light_red},line width=0mm] (0,0)
        circle (.8ex);}} and the SNR for the direction with the smallest
    curvature is shown in {\protect\tikz{\protect\draw [white,fill=black] (0,0)
        circle (.8ex);}}. } \label{vivit::fig:directional_derivatives}
\end{figure}

A unique feature of \vivit{}'s quantities is that they provide a notion of
\textit{curvature uncertainty} through \textit{per-sample} first- and
second-order directional derivatives (\Cref{vivit::eq:gammas-lambdas}). To
quantify noise in these derivatives, we compute their signal-to-noise ratios
(SNRs). For each direction $\ve_k$, the SNR is given by the squared empirical
mean divided by the empirical variance of the $N$ mini-batch samples
$\{\gamma_{n,k}\}_{n=1}^N$ and $\{\lambda_{n,k}\}_{n=1}^N$, respectively.

\Cref{vivit::fig:directional_derivatives} shows curvature SNRs during training
the \threecthreed network on \cifarten with \sgd. The curvature signal along the
top-$C$ eigenvectors decreases from $\text{SNR} > 1$ by two orders of magnitude.
In comparison, the directional gradients do not exhibit such a pattern (see
\Cref{vivit::sec:directional_derivatives}). Results for the other test cases can
be found in \Cref{vivit::sec:directional_derivatives}.

In this section, we have given a glimpse of the \textit{very rich} quantities
that can be efficiently computed under \vivit's concept. In
\Cref{vivit::sec:use_cases}, we discuss their practical use---curvature
uncertainty in particular.

%%% Local Variables:
%%% mode: latex
%%% TeX-master: "../thesis"
%%% End:
