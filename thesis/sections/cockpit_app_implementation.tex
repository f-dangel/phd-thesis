In this section, we provide more details about our implementation
(\Cref{cockpit::app:hooks_benchmarks}) to access the desired quantities with as
little overhead as possible. Additionally, we present more benchmarks for
individual instruments (\Cref{cockpit::app:benchmark-instruments}) and \cockpit
configurations (\Cref{cockpit::app:benchmark-configuration}). These are similar
but extended versions of the ones presented in
\Cref{cockpit::fig:benchmark-instruments,cockpit::fig:benchmark_heatmap} in the
main text. Lastly, we benchmark different implementations of computing the
two-dimensional gradient histogram (\Cref{cockpit::app:histograms}), identifying
a computational bottleneck for its current GPU implementation.

\subsubsection{Hardware Details}

We conducted benchmarks on the following setup:
\begin{itemize}
\item \textbf{CPU:} Intel Core i7-8700K CPU @ 3.70\,GHz × 12 (32\,GB)
\item \textbf{GPU:} NVIDIA GeForce RTX 2080 Ti (11\,GB)
\end{itemize}

\subsubsection{Test Problem Details}

The experiments in this paper rely mostly on optimization problems provided by
the \deepobs benchmark suite \citep{schneider2019deepobs}. If not stated
otherwise, we use the default training details suggested by \deepobs, that are
summarized below. For more details see the original paper.
\begin{itemize}
\item \textbf{Quadratic Deep:} A stochastic quadratic problem with an
  eigenspectrum similar to what has been reported for neural nets. Default batch
  size $128$, default number of epochs $100$.
\item \textbf{\mnist Log. Reg.:} Multinomial logistic regression on \mnist
  \citep{lecun1998gradient}. Default batch size $128$, default number of epochs $50$.
\item \textbf{\mnist \mlp:} Multi-layer perceptron on \mnist.
  Default batch size $128$, default number of epochs $100$.
\item \textbf{\fmnist \mlp:} Multi-layer perceptron on \fmnist
  \citep{xiao2017fashion}. Default batch size $128$, default number of epochs $100$.
\item \textbf{\fmnist \twoctwod:} A two convolutional and two dense layered
  neural network on \fmnist. Default batch size $128$, default number of epochs
  $100$.
\item \textbf{\cifarten \threecthreed:} A three convolutional and three dense
  layered neural network on \cifarten \citep{krizhevsky2009learning}. Default batch size
  $128$, default number of epochs $100$.
\item \textbf{\cifarhun \allcnnc:} All Convolutional Neural Network C (\allcnnc
  \citep{springenberg2015striving}) on \cifarhun \citep{krizhevsky2009learning}. Default batch
  size $256$, default number of epochs $350$.
\item \textbf{\svhn \threecthreed:} A three convolutional and three dense
  layered neural network on \svhn \citep{netzer2011reading}. Default batch size $128$,
  default number of epochs $100$.
\end{itemize}

\subsection{Hooks \& Memory Benchmarks}\label{cockpit::app:hooks_benchmarks}

To improve memory consumption, we compact information during the backward pass
by adding hooks to the neural network's layers. These are executed after
\backpack extensions and have access to the quantities computed therein. They
compress information to what is requested by a quantity and free the memory
occupied by \backpack buffers. Such savings primarily depend on the parameter
distribution over layers, and are bigger for more balanced architectures
(compare \Cref{cockpit::fig:memory-benchmark}).


\subsubsection{Example}

Say, we want to compute a histogram over the $|\sB| \times D$ individual
gradient elements of a network. Suppose that $|\sB| = 128$ and the model is
\deepobs's \cifarten \threecthreed test problem with $895,210$ parameters. Given
that every parameter is stored in single precision, the model requires $895,210
\times 4\,\text{Bytes} \approx 3.41\,\text{MB}$.
% computation performed with https://www.gbmb.org/bytes-to-mb, MB in binary
Storing the individual gradients will require $128 \times 895,210 \times
4\,\text{Bytes} \approx 437\,\text{MB}$ (for larger networks this quickly
exceeds the available memory as the individual gradients occupy $|\sB|$ times
the model size). If instead, the layer-wise individual gradients are condensed
into histograms of negligible size and immediately freed afterwards during
backpropagation, the maximum memory overhead reduces to storing the individual
gradients of the largest layer. For our example, the largest layer has
$589,824$ parameters, and the associated individual gradients will require $128
\times 589,824\times 4\,\text{Bytes} \approx 288\,\text{MB}$, saving roughly
$149\,\text{MB}$ of RAM. In practice, we observe these expected savings, see
\Cref{cockpit::fig:memory-benchmark-cifar10}.

\pgfkeys{/pgfplots/memorybenchmarkdefault/.style={
    enlarge x limits=-0.05,
    width=1.0\linewidth,
    height=0.45\linewidth,
    every axis plot/.append style={line width = 1.5pt},
    ymin=0,
    tick pos = left,
    ylabel near ticks,
    xlabel near ticks,
    xtick align = inside,
    ytick align = inside,
    legend cell align = left,
    legend columns = 1,
    legend pos = south east,
    legend style = {
      fill opacity = 0.7,
      text opacity = 1,
      font = \footnotesize,
    },
    xticklabel style = {font = \footnotesize, inner xsep = -5ex},
    xlabel style = {font = \footnotesize},
    axis line style = {black},
    yticklabel style = {font = \footnotesize, inner ysep = 0ex},
    ylabel style = {font = \footnotesize},
    title style = {font = \footnotesize, inner ysep = -3ex},
    grid = major,
    grid style = {dashed}
  }
}

\captionsetup[subfigure]{justification=justified,singlelinecheck=false}

\begin{figure}[!bt]
  \centering
  \begin{subfigure}[t]{0.99\textwidth}
    \pgfkeys{/pgfplots/zmystyle/.style={ memorybenchmarkdefault, xlabel = {},
        legend pos = north west}}
    \caption{\fmnist \twoctwod}
    \tikzexternalenable
    % This file was created by tikzplotlib v0.9.8.
\begin{tikzpicture}

\definecolor{color0}{rgb}{0.894117647058824,0.101960784313725,0.109803921568627}
\definecolor{color1}{rgb}{0.301960784313725,0.686274509803922,0.290196078431373}
\definecolor{color2}{rgb}{0.215686274509804,0.494117647058824,0.72156862745098}

\begin{axis}[
axis line style={white!80!black},
legend style={
  fill opacity=0.8,
  draw opacity=1,
  text opacity=1,
  at={(0.5,0.09)},
  anchor=south,
  draw=white!80!black
},
tick pos=left,
xlabel={Time [s]},
xmin=-0.08, xmax=1.68,
xtick style={color=gray},
ylabel={Memory [MB]},
ymin=101.97587418374, ymax=3915.72539214146,
zmystyle
]
\path [draw=color0, fill=color0, opacity=0.4]
(axis cs:0,3742.3731413252)
--(axis cs:0,3707.8409211748)
--(axis cs:1.6,3707.8409211748)
--(axis cs:1.6,3742.3731413252)
--(axis cs:1.6,3742.3731413252)
--(axis cs:0,3742.3731413252)
--cycle;

\path [draw=color1, fill=color1, opacity=0.4]
(axis cs:0,3676.40934791577)
--(axis cs:0,3665.15862083423)
--(axis cs:1.6,3665.15862083423)
--(axis cs:1.6,3676.40934791577)
--(axis cs:1.6,3676.40934791577)
--(axis cs:0,3676.40934791577)
--cycle;

\path [draw=color2, fill=color2, opacity=0.4]
(axis cs:0,586.554612878292)
--(axis cs:0,539.237574621708)
--(axis cs:1.6,539.237574621708)
--(axis cs:1.6,586.554612878292)
--(axis cs:1.6,586.554612878292)
--(axis cs:0,586.554612878292)
--cycle;

\addplot [, color0, dotted, forget plot]
table {%
0 275.95703125
0.01 276.16015625
0.02 297.875
0.03 332.80078125
0.04 364.51171875
0.05 371.203125
0.06 371.203125
0.07 382.01953125
0.08 390.52734375
0.09 401.46484375
0.1 415.46484375
0.11 427.34765625
0.12 464.16796875
0.13 479.6796875
0.14 483.93359375
0.15 490.1796875
0.16 496.1796875
0.17 500.1796875
0.18 488.2109375
0.19 502.6484375
0.2 420.02734375
0.21 420.54296875
0.22 420.54296875
0.23 427.94140625
0.24 434.12890625
0.25 436.33984375
0.26 436.33984375
0.27 436.33984375
0.28 436.33984375
0.29 436.33984375
0.3 436.78515625
0.31 440.265625
0.32 489.015625
0.33 526.80078125
0.34 698.76171875
0.35 893.41015625
0.36 1089.08984375
0.37 1286.05859375
0.38 1481.73828125
0.39 1678.19140625
0.4 1875.41796875
0.41 2072.12890625
0.42 2120.44921875
0.43 2122.8515625
0.44 2128.0234375
0.45 2165.72265625
0.46 2173.41015625
0.47 2173.66796875
0.48 2165.0078125
0.49 2300.1015625
0.5 2435.7109375
0.51 2571.3203125
0.52 2706.4140625
0.53 2841.765625
0.54 2977.890625
0.55 3114.53125
0.56 3250.65625
0.57 3386.5234375
0.58 3522.6484375
0.59 3658.2578125
0.6 3717.5546875
0.61 3717.5546875
0.62 3717.5546875
0.63 3717.5546875
0.64 3717.5546875
0.65 3717.5546875
0.66 3717.5546875
0.67 3717.5546875
0.68 3717.5546875
0.69 3717.5546875
0.7 3717.5546875
0.71 3717.5546875
0.72 3717.5546875
0.73 3717.5546875
0.74 3717.5546875
0.75 3717.5546875
0.76 3717.5546875
0.77 3717.5546875
0.78 3717.5546875
0.79 3717.5546875
0.8 3717.5546875
0.81 3717.5546875
0.82 3717.5546875
0.83 3717.5546875
0.84 3717.5546875
0.85 3717.5546875
0.86 3717.5546875
0.87 3717.5546875
0.88 3717.5546875
0.89 3717.5546875
0.9 3717.5546875
0.91 3717.5546875
0.92 3717.5546875
0.93 3717.5546875
0.94 3717.5546875
0.95 3717.5546875
0.96 3717.5546875
0.97 3717.5546875
0.98 3717.5546875
0.99 3717.5546875
1 3717.5546875
1.01 3717.5546875
1.02 3717.5546875
1.03 3717.5546875
1.04 3717.5546875
1.05 3717.5546875
1.06 3717.5546875
1.07 3717.5546875
1.08 3717.5546875
1.09 3717.5546875
1.1 3717.5546875
1.11 3717.5546875
1.12 3717.5546875
1.13 3717.5546875
1.14 3717.5546875
1.15 3717.5546875
1.16 3717.5546875
1.17 3717.5546875
1.18 3717.5546875
1.19 3717.5546875
1.2 3717.5546875
1.21 3717.5546875
1.22 3717.5546875
1.23 3717.5546875
1.24 3717.5546875
1.25 3717.5546875
1.26 3717.5546875
1.27 3717.5546875
1.28 3717.5546875
1.29 3717.5546875
1.3 3717.5546875
1.31 3717.5546875
1.32 3717.5546875
1.33 3717.5546875
1.34 3717.5546875
1.35 3717.5546875
1.36 3717.5546875
1.37 3717.5546875
1.38 3717.5546875
1.39 3717.5546875
1.4 3717.5546875
1.41 3717.5546875
1.42 3717.5546875
1.43 3717.5546875
1.44 3717.5546875
1.45 3255.71875
1.46 2732.7734375
1.47 2240.296875
1.48 1808.77734375
1.49 1387.71875
1.5 975.35546875
1.51 581.71875
1.52 581.7578125
};
\addplot [, color0, dotted, forget plot]
table {%
0 275.47265625
0.01 275.68359375
0.02 296.27734375
0.03 331.19140625
0.04 367.28515625
0.05 370.5625
0.06 374.46875
0.07 386.453125
0.08 395.5
0.09 408.9140625
0.1 420.93359375
0.11 436.796875
0.12 503.8046875
0.13 480.84375
0.14 486.84375
0.15 492.84375
0.16 498.84375
0.17 478.80859375
0.18 498.40234375
0.19 419.5703125
0.2 419.9765625
0.21 419.9765625
0.22 421.18359375
0.23 433.04296875
0.24 435.1015625
0.25 435.76953125
0.26 435.76953125
0.27 435.76953125
0.28 435.76953125
0.29 435.76953125
0.3 436.36328125
0.31 471.87109375
0.32 503.52734375
0.33 602.63671875
0.34 799.08984375
0.35 996.05859375
0.36 1193.54296875
0.37 1391.54296875
0.38 1589.02734375
0.39 1787.80078125
0.4 1986.31640625
0.41 2120.86328125
0.42 2120.86328125
0.43 2123.8046875
0.44 2142.62109375
0.45 2162.30859375
0.46 2186.90625
0.47 2186.90625
0.48 2249.71875
0.49 2385.84375
0.5 2520.9375
0.51 2656.8046875
0.52 2792.15625
0.53 2927.25
0.54 3063.375
0.55 3199.2421875
0.56 3335.8828125
0.57 3471.75
0.58 3607.6171875
0.59 3731.109375
0.6 3731.109375
0.61 3731.109375
0.62 3731.109375
0.63 3731.109375
0.64 3731.109375
0.65 3731.109375
0.66 3731.109375
0.67 3731.109375
0.68 3731.109375
0.69 3731.109375
0.7 3731.109375
0.71 3731.109375
0.72 3731.109375
0.73 3731.109375
0.74 3731.109375
0.75 3731.109375
0.76 3731.109375
0.77 3731.109375
0.78 3731.109375
0.79 3731.109375
0.8 3731.109375
0.81 3731.109375
0.82 3731.109375
0.83 3731.109375
0.84 3731.109375
0.85 3731.109375
0.86 3731.109375
0.87 3731.109375
0.88 3731.109375
0.89 3731.109375
0.9 3731.109375
0.91 3731.109375
0.92 3731.109375
0.93 3731.109375
0.94 3731.109375
0.95 3731.109375
0.96 3731.109375
0.97 3731.109375
0.98 3731.109375
0.99 3731.109375
1 3731.109375
1.01 3731.109375
1.02 3731.109375
1.03 3731.109375
1.04 3731.109375
1.05 3731.109375
1.06 3731.109375
1.07 3731.109375
1.08 3731.109375
1.09 3731.109375
1.1 3731.109375
1.11 3731.109375
1.12 3731.109375
1.13 3731.109375
1.14 3731.109375
1.15 3731.109375
1.16 3731.109375
1.17 3731.109375
1.18 3731.109375
1.19 3731.109375
1.2 3731.109375
1.21 3731.109375
1.22 3731.109375
1.23 3731.109375
1.24 3731.109375
1.25 3731.109375
1.26 3731.109375
1.27 3731.109375
1.28 3731.109375
1.29 3731.109375
1.3 3731.109375
1.31 3731.109375
1.32 3731.109375
1.33 3731.109375
1.34 3731.109375
1.35 3731.109375
1.36 3731.109375
1.37 3731.109375
1.38 3731.109375
1.39 3731.109375
1.4 3731.109375
1.41 3731.109375
1.42 3731.109375
1.43 3731.109375
1.44 3428.21875
1.45 2935.7421875
1.46 2413.2734375
1.47 1973.86328125
1.48 1557.15234375
1.49 1102.55859375
1.5 685.84765625
1.51 595.37109375
1.52 595.37109375
};
\addplot [, color0, dotted, forget plot]
table {%
0 275.8359375
0.01 276.046875
0.02 298.9921875
0.03 333.2109375
0.04 368.75
0.05 371.0390625
0.06 377.7421875
0.07 387.41796875
0.08 396.5078125
0.09 408.9609375
0.1 422.203125
0.11 439.6171875
0.12 503.609375
0.13 481.140625
0.14 487.140625
0.15 493.140625
0.16 499.140625
0.17 479.62109375
0.18 499.21484375
0.19 419.609375
0.2 420.09375
0.21 420.09375
0.22 421.8671875
0.23 433.2109375
0.24 435.52734375
0.25 435.88671875
0.26 435.88671875
0.27 435.88671875
0.28 435.88671875
0.29 435.88671875
0.3 436.6171875
0.31 473.8125
0.32 503.90234375
0.33 611.09765625
0.34 807.03515625
0.35 1003.48828125
0.36 1200.71484375
0.37 1398.19921875
0.38 1596.45703125
0.39 1794.19921875
0.4 1992.71484375
0.41 2120.6015625
0.42 2120.6015625
0.43 2123.4375
0.44 2142.2265625
0.45 2166.80859375
0.46 2187.08203125
0.47 2187.08203125
0.48 2257.1328125
0.49 2392.2265625
0.5 2527.578125
0.51 2662.9296875
0.52 2798.5390625
0.53 2933.890625
0.54 3069.5
0.55 3205.8828125
0.56 3341.75
0.57 3477.875
0.58 3613.2265625
0.59 3730.53125
0.6 3730.53125
0.61 3730.53125
0.62 3730.53125
0.63 3730.53125
0.64 3730.53125
0.65 3730.53125
0.66 3730.53125
0.67 3730.53125
0.68 3730.53125
0.69 3730.53125
0.7 3730.53125
0.71 3730.53125
0.72 3730.53125
0.73 3730.53125
0.74 3730.53125
0.75 3730.53125
0.76 3730.53125
0.77 3730.53125
0.78 3730.53125
0.79 3730.53125
0.8 3730.53125
0.81 3730.53125
0.82 3730.53125
0.83 3730.53125
0.84 3730.53125
0.85 3730.53125
0.86 3730.53125
0.87 3730.53125
0.88 3730.53125
0.89 3730.53125
0.9 3730.53125
0.91 3730.53125
0.92 3730.53125
0.93 3730.53125
0.94 3730.53125
0.95 3730.53125
0.96 3730.53125
0.97 3730.53125
0.98 3730.53125
0.99 3730.53125
1 3730.53125
1.01 3730.53125
1.02 3730.53125
1.03 3730.53125
1.04 3730.53125
1.05 3730.53125
1.06 3730.53125
1.07 3730.53125
1.08 3730.53125
1.09 3730.53125
1.1 3730.53125
1.11 3730.53125
1.12 3730.53125
1.13 3730.53125
1.14 3730.53125
1.15 3730.53125
1.16 3730.53125
1.17 3730.53125
1.18 3730.53125
1.19 3730.53125
1.2 3730.53125
1.21 3730.53125
1.22 3730.53125
1.23 3730.53125
1.24 3730.53125
1.25 3730.53125
1.26 3730.53125
1.27 3730.53125
1.28 3730.53125
1.29 3730.53125
1.3 3730.53125
1.31 3730.53125
1.32 3730.53125
1.33 3730.53125
1.34 3730.53125
1.35 3730.53125
1.36 3730.53125
1.37 3730.53125
1.38 3730.53125
1.39 3730.53125
1.4 3730.53125
1.41 3730.53125
1.42 3730.53125
1.43 3730.53125
1.44 3617.0546875
1.45 3086.6953125
1.46 2594.21875
1.47 2120.6953125
1.48 1674.6953125
1.49 1253.51171875
1.5 836.80078125
1.51 594.6953125
1.52 594.73046875
};
\addplot [, color0, dotted, forget plot]
table {%
0 275.58984375
0.01 275.73046875
0.02 298.13671875
0.03 331.1328125
0.04 367.2265625
0.05 370.5
0.06 374.9140625
0.07 386
0.08 395.78125
0.09 408.96484375
0.1 420.96484375
0.11 433.453125
0.12 502.4140625
0.13 480.4609375
0.14 486.71875
0.15 492.9609375
0.16 496.9609375
0.17 477.66015625
0.18 496.9921875
0.19 443.953125
0.2 419.91796875
0.21 419.91796875
0.22 419.91796875
0.23 433.0234375
0.24 434.82421875
0.25 435.71484375
0.26 435.71484375
0.27 435.71484375
0.28 435.71484375
0.29 435.71484375
0.3 436.9765625
0.31 463.35546875
0.32 508.26171875
0.33 583.11328125
0.34 776.47265625
0.35 971.12109375
0.36 1165.25390625
0.37 1361.19140625
0.38 1557.90234375
0.39 1754.09765625
0.4 1951.83984375
0.41 2116.953125
0.42 2124.80078125
0.43 2127.50390625
0.44 2141.70703125
0.45 2153.66796875
0.46 2178.62109375
0.47 2178.62109375
0.48 2224.8515625
0.49 2359.9453125
0.5 2495.296875
0.51 2630.6484375
0.52 2765.2265625
0.53 2900.8359375
0.54 3037.21875
0.55 3173.34375
0.56 3309.2109375
0.57 3445.3359375
0.58 3580.9453125
0.59 3714.4921875
0.6 3722.7421875
0.61 3722.7421875
0.62 3722.7421875
0.63 3722.7421875
0.64 3722.7421875
0.65 3722.7421875
0.66 3722.7421875
0.67 3722.7421875
0.68 3722.7421875
0.69 3722.7421875
0.7 3722.7421875
0.71 3722.7421875
0.72 3722.7421875
0.73 3722.7421875
0.74 3722.7421875
0.75 3722.7421875
0.76 3722.7421875
0.77 3722.7421875
0.78 3722.7421875
0.79 3722.7421875
0.8 3722.7421875
0.81 3722.7421875
0.82 3722.7421875
0.83 3722.7421875
0.84 3722.7421875
0.85 3722.7421875
0.86 3722.7421875
0.87 3722.7421875
0.88 3722.7421875
0.89 3722.7421875
0.9 3722.7421875
0.91 3722.7421875
0.92 3722.7421875
0.93 3722.7421875
0.94 3722.7421875
0.95 3722.7421875
0.96 3722.7421875
0.97 3722.7421875
0.98 3722.7421875
0.99 3722.7421875
1 3722.7421875
1.01 3722.7421875
1.02 3722.7421875
1.03 3722.7421875
1.04 3722.7421875
1.05 3722.7421875
1.06 3722.7421875
1.07 3722.7421875
1.08 3722.7421875
1.09 3722.7421875
1.1 3722.7421875
1.11 3722.7421875
1.12 3722.7421875
1.13 3722.7421875
1.14 3722.7421875
1.15 3722.7421875
1.16 3722.7421875
1.17 3722.7421875
1.18 3722.7421875
1.19 3722.7421875
1.2 3722.7421875
1.21 3722.7421875
1.22 3722.7421875
1.23 3722.7421875
1.24 3722.7421875
1.25 3722.7421875
1.26 3722.7421875
1.27 3722.7421875
1.28 3722.7421875
1.29 3722.7421875
1.3 3722.7421875
1.31 3722.7421875
1.32 3722.7421875
1.33 3722.7421875
1.34 3722.7421875
1.35 3722.7421875
1.36 3722.7421875
1.37 3722.7421875
1.38 3722.7421875
1.39 3722.7421875
1.4 3722.7421875
1.41 3722.7421875
1.42 3722.7421875
1.43 3722.7421875
1.44 3722.7421875
1.45 3216.90625
1.46 2700.078125
1.47 2207.6015625
1.48 1794.90625
1.49 1359.37109375
1.5 980.54296875
1.51 586.90625
1.52 586.98828125
};
\addplot [, color0, dotted, forget plot]
table {%
0 275.328125
0.01 275.53515625
0.02 296.48828125
0.03 331.10546875
0.04 367.45703125
0.05 370.2734375
0.06 371.31640625
0.07 385.4140625
0.08 394.51953125
0.09 408.43359375
0.1 420.43359375
0.11 430.5625
0.12 496.859375
0.13 480.0859375
0.14 486.34375
0.15 492.34375
0.16 498.34375
0.17 501.8515625
0.18 496.35546875
0.19 443.83203125
0.2 419.85546875
0.21 419.85546875
0.22 419.85546875
0.23 432.90234375
0.24 434.703125
0.25 435.65234375
0.26 435.65234375
0.27 435.65234375
0.28 435.65234375
0.29 435.65234375
0.3 437
0.31 475.74609375
0.32 505.41015625
0.33 603.234375
0.34 798.3984375
0.35 995.3671875
0.36 1192.3359375
0.37 1385.6953125
0.38 1583.6953125
0.39 1782.2109375
0.4 1979.4375
0.41 2120.109375
0.42 2120.36328125
0.43 2123.03125
0.44 2140.05078125
0.45 2160.86328125
0.46 2173.71875
0.47 2173.71875
0.48 2237.79296875
0.49 2373.40234375
0.5 2509.01171875
0.51 2644.36328125
0.52 2779.45703125
0.53 2915.32421875
0.54 3051.44921875
0.55 3187.57421875
0.56 3323.18359375
0.57 3459.30859375
0.58 3595.43359375
0.59 3716.60546875
0.6 3717.89453125
0.61 3717.89453125
0.62 3717.89453125
0.63 3717.89453125
0.64 3717.89453125
0.65 3717.89453125
0.66 3717.89453125
0.67 3717.89453125
0.68 3717.89453125
0.69 3717.89453125
0.7 3717.89453125
0.71 3717.89453125
0.72 3717.89453125
0.73 3717.89453125
0.74 3717.89453125
0.75 3717.89453125
0.76 3717.89453125
0.77 3717.89453125
0.78 3717.89453125
0.79 3717.89453125
0.8 3717.89453125
0.81 3717.89453125
0.82 3717.89453125
0.83 3717.89453125
0.84 3717.89453125
0.85 3717.89453125
0.86 3717.89453125
0.87 3717.89453125
0.88 3717.89453125
0.89 3717.89453125
0.9 3717.89453125
0.91 3717.89453125
0.92 3717.89453125
0.93 3717.89453125
0.94 3717.89453125
0.95 3717.89453125
0.96 3717.89453125
0.97 3717.89453125
0.98 3717.89453125
0.99 3717.89453125
1 3717.89453125
1.01 3717.89453125
1.02 3717.89453125
1.03 3717.89453125
1.04 3717.89453125
1.05 3717.89453125
1.06 3717.89453125
1.07 3717.89453125
1.08 3717.89453125
1.09 3717.89453125
1.1 3717.89453125
1.11 3717.89453125
1.12 3717.89453125
1.13 3717.89453125
1.14 3717.89453125
1.15 3717.89453125
1.16 3717.89453125
1.17 3717.89453125
1.18 3717.89453125
1.19 3717.89453125
1.2 3717.89453125
1.21 3717.89453125
1.22 3717.89453125
1.23 3717.89453125
1.24 3717.89453125
1.25 3717.89453125
1.26 3717.89453125
1.27 3717.89453125
1.28 3717.89453125
1.29 3717.89453125
1.3 3717.89453125
1.31 3717.89453125
1.32 3717.89453125
1.33 3717.89453125
1.34 3717.89453125
1.35 3717.89453125
1.36 3717.89453125
1.37 3717.89453125
1.38 3717.89453125
1.39 3717.89453125
1.4 3717.89453125
1.41 3717.89453125
1.42 3717.89453125
1.43 3717.89453125
1.44 3566.53515625
1.45 3036.17578125
1.46 2543.69921875
1.47 2074.296875
1.48 1652.05859375
1.49 1240.875
1.5 810.05859375
1.51 582.05859375
1.52 582.109375
};
\addplot [, color0, dotted, forget plot]
table {%
0 275.796875
0.01 276.00390625
0.02 297.68359375
0.03 331.3828125
0.04 366.703125
0.05 371.01171875
0.06 373.34765625
0.07 386.3125
0.08 396.0234375
0.09 409.1640625
0.1 420.15234375
0.11 430.3515625
0.12 496.3671875
0.13 481.61328125
0.14 487.61328125
0.15 491.61328125
0.16 497.61328125
0.17 502.0859375
0.18 496.9140625
0.19 444.1328125
0.2 420.10546875
0.21 420.10546875
0.22 420.10546875
0.23 433.24609375
0.24 435.046875
0.25 435.90625
0.26 435.90625
0.27 435.90625
0.28 435.90625
0.29 435.90625
0.3 436.62890625
0.31 476.328125
0.32 506.015625
0.33 605.58203125
0.34 800.23046875
0.35 995.91015625
0.36 1192.36328125
0.37 1389.33203125
0.38 1587.07421875
0.39 1783.78515625
0.4 1981.52734375
0.41 2121.01953125
0.42 2121.01953125
0.43 2124.00390625
0.44 2142.37109375
0.45 2166.2890625
0.46 2174.1953125
0.47 2174.1953125
0.48 2239.59765625
0.49 2374.94921875
0.5 2509.01171875
0.51 2642.30078125
0.52 2775.58984375
0.53 2911.19921875
0.54 3047.06640625
0.55 3182.67578125
0.56 3318.28515625
0.57 3453.63671875
0.58 3588.98828125
0.59 3717.37890625
0.6 3718.41015625
0.61 3718.41015625
0.62 3718.41015625
0.63 3718.41015625
0.64 3718.41015625
0.65 3718.41015625
0.66 3718.41015625
0.67 3718.41015625
0.68 3718.41015625
0.69 3718.41015625
0.7 3718.41015625
0.71 3718.41015625
0.72 3718.41015625
0.73 3718.41015625
0.74 3718.41015625
0.75 3718.41015625
0.76 3718.41015625
0.77 3718.41015625
0.78 3718.41015625
0.79 3718.41015625
0.8 3718.41015625
0.81 3718.41015625
0.82 3718.41015625
0.83 3718.41015625
0.84 3718.41015625
0.85 3718.41015625
0.86 3718.41015625
0.87 3718.41015625
0.88 3718.41015625
0.89 3718.41015625
0.9 3718.41015625
0.91 3718.41015625
0.92 3718.41015625
0.93 3718.41015625
0.94 3718.41015625
0.95 3718.41015625
0.96 3718.41015625
0.97 3718.41015625
0.98 3718.41015625
0.99 3718.41015625
1 3718.41015625
1.01 3718.41015625
1.02 3718.41015625
1.03 3718.41015625
1.04 3718.41015625
1.05 3718.41015625
1.06 3718.41015625
1.07 3718.41015625
1.08 3718.41015625
1.09 3718.41015625
1.1 3718.41015625
1.11 3718.41015625
1.12 3718.41015625
1.13 3718.41015625
1.14 3718.41015625
1.15 3718.41015625
1.16 3718.41015625
1.17 3718.41015625
1.18 3718.41015625
1.19 3718.41015625
1.2 3718.41015625
1.21 3718.41015625
1.22 3718.41015625
1.23 3718.41015625
1.24 3718.41015625
1.25 3718.41015625
1.26 3718.41015625
1.27 3718.41015625
1.28 3718.41015625
1.29 3718.41015625
1.3 3718.41015625
1.31 3718.41015625
1.32 3718.41015625
1.33 3718.41015625
1.34 3718.41015625
1.35 3718.41015625
1.36 3718.41015625
1.37 3718.41015625
1.38 3718.41015625
1.39 3718.41015625
1.4 3718.41015625
1.41 3718.41015625
1.42 3718.41015625
1.43 3718.41015625
1.44 3718.41015625
1.45 3226.10546875
1.46 2708.57421875
1.47 2203.26953125
1.48 1794.57421875
1.49 1355.0390625
1.5 938.328125
1.51 582.57421875
1.52 582.61328125
};
\addplot [, color0, dotted, forget plot]
table {%
0 275.45703125
0.01 275.6640625
0.02 298.68359375
0.03 332.51953125
0.04 368.28515625
0.05 370.64453125
0.06 376.59375
0.07 386.609375
0.08 396.4765625
0.09 408.96875
0.1 422.2109375
0.11 441.65234375
0.12 503.64453125
0.13 481.375
0.14 487.375
0.15 493.375
0.16 499.375
0.17 480.171875
0.18 500.28125
0.19 424.09375
0.2 420.24609375
0.21 420.24609375
0.22 422.84375
0.23 433.4140625
0.24 435.98828125
0.25 436.04296875
0.26 436.04296875
0.27 436.04296875
0.28 436.04296875
0.29 436.04296875
0.3 436.76953125
0.31 478.5859375
0.32 503.12109375
0.33 625.07421875
0.34 820.75390625
0.35 1017.98046875
0.36 1215.46484375
0.37 1412.94921875
0.38 1611.72265625
0.39 1809.46484375
0.4 2008.75390625
0.41 2120.09375
0.42 2121.1015625
0.43 2123.3671875
0.44 2144.984375
0.45 2172.8671875
0.46 2173.33984375
0.47 2173.33984375
0.48 2264.03515625
0.49 2399.38671875
0.5 2535.25390625
0.51 2671.12109375
0.52 2805.44140625
0.53 2942.33984375
0.54 3077.69140625
0.55 3213.81640625
0.56 3349.94140625
0.57 3485.55078125
0.58 3621.41796875
0.59 3717.32421875
0.6 3717.32421875
0.61 3717.32421875
0.62 3717.32421875
0.63 3717.32421875
0.64 3717.32421875
0.65 3717.32421875
0.66 3717.32421875
0.67 3717.32421875
0.68 3717.32421875
0.69 3717.32421875
0.7 3717.32421875
0.71 3717.32421875
0.72 3717.32421875
0.73 3717.32421875
0.74 3717.32421875
0.75 3717.32421875
0.76 3717.32421875
0.77 3717.32421875
0.78 3717.32421875
0.79 3717.32421875
0.8 3717.32421875
0.81 3717.32421875
0.82 3717.32421875
0.83 3717.32421875
0.84 3717.32421875
0.85 3717.32421875
0.86 3717.32421875
0.87 3717.32421875
0.88 3717.32421875
0.89 3717.32421875
0.9 3717.32421875
0.91 3717.32421875
0.92 3717.32421875
0.93 3717.32421875
0.94 3717.32421875
0.95 3717.32421875
0.96 3717.32421875
0.97 3717.32421875
0.98 3717.32421875
0.99 3717.32421875
1 3717.32421875
1.01 3717.32421875
1.02 3717.32421875
1.03 3717.32421875
1.04 3717.32421875
1.05 3717.32421875
1.06 3717.32421875
1.07 3717.32421875
1.08 3717.32421875
1.09 3717.32421875
1.1 3717.32421875
1.11 3717.32421875
1.12 3717.32421875
1.13 3717.32421875
1.14 3717.32421875
1.15 3717.32421875
1.16 3717.32421875
1.17 3717.32421875
1.18 3717.32421875
1.19 3717.32421875
1.2 3717.32421875
1.21 3717.32421875
1.22 3717.32421875
1.23 3717.32421875
1.24 3717.32421875
1.25 3717.32421875
1.26 3717.32421875
1.27 3717.32421875
1.28 3717.32421875
1.29 3717.32421875
1.3 3717.32421875
1.31 3717.32421875
1.32 3717.32421875
1.33 3717.32421875
1.34 3717.32421875
1.35 3717.32421875
1.36 3717.32421875
1.37 3717.32421875
1.38 3717.32421875
1.39 3717.32421875
1.4 3717.32421875
1.41 3717.32421875
1.42 3717.32421875
1.43 3717.32421875
1.44 3528.08203125
1.45 3031.48828125
1.46 2505.24609375
1.47 2045.48828125
1.48 1619.1328125
1.49 1202.421875
1.5 785.7109375
1.51 581.48828125
1.52 581.53125
};
\addplot [, color0, dotted, forget plot]
table {%
0 275.375
0.01 275.58203125
0.02 296.7734375
0.03 331.55859375
0.04 367.640625
0.05 370.15625
0.06 375.33203125
0.07 386.265625
0.08 395.23046875
0.09 408.5859375
0.1 420.5859375
0.11 434.4375
0.12 502.453125
0.13 480.94140625
0.14 486.94140625
0.15 492.94140625
0.16 496.94140625
0.17 477.67578125
0.18 497.52734375
0.19 430.70703125
0.2 420.02734375
0.21 420.02734375
0.22 420.76171875
0.23 433.13671875
0.24 435.1953125
0.25 435.828125
0.26 435.828125
0.27 435.828125
0.28 435.828125
0.29 435.828125
0.3 436.55078125
0.31 476
0.32 505.6171875
0.33 618.0234375
0.34 814.21875
0.35 1009.8984375
0.36 1208.4140625
0.37 1406.15625
0.38 1603.640625
0.39 1801.8984375
0.4 2000.9296875
0.41 2120.7265625
0.42 2121.3828125
0.43 2123.4296875
0.44 2142.48046875
0.45 2176.90234375
0.46 2193.05078125
0.47 2193.05078125
0.48 2269.48828125
0.49 2404.83984375
0.5 2540.70703125
0.51 2675.80078125
0.52 2811.92578125
0.53 2947.79296875
0.54 3083.66015625
0.55 3219.78515625
0.56 3356.42578125
0.57 3492.03515625
0.58 3628.41796875
0.59 3736.95703125
0.6 3736.95703125
0.61 3736.95703125
0.62 3736.95703125
0.63 3736.95703125
0.64 3736.95703125
0.65 3736.95703125
0.66 3736.95703125
0.67 3736.95703125
0.68 3736.95703125
0.69 3736.95703125
0.7 3736.95703125
0.71 3736.95703125
0.72 3736.95703125
0.73 3736.95703125
0.74 3736.95703125
0.75 3736.95703125
0.76 3736.95703125
0.77 3736.95703125
0.78 3736.95703125
0.79 3736.95703125
0.8 3736.95703125
0.81 3736.95703125
0.82 3736.95703125
0.83 3736.95703125
0.84 3736.95703125
0.85 3736.95703125
0.86 3736.95703125
0.87 3736.95703125
0.88 3736.95703125
0.89 3736.95703125
0.9 3736.95703125
0.91 3736.95703125
0.92 3736.95703125
0.93 3736.95703125
0.94 3736.95703125
0.95 3736.95703125
0.96 3736.95703125
0.97 3736.95703125
0.98 3736.95703125
0.99 3736.95703125
1 3736.95703125
1.01 3736.95703125
1.02 3736.95703125
1.03 3736.95703125
1.04 3736.95703125
1.05 3736.95703125
1.06 3736.95703125
1.07 3736.95703125
1.08 3736.95703125
1.09 3736.95703125
1.1 3736.95703125
1.11 3736.95703125
1.12 3736.95703125
1.13 3736.95703125
1.14 3736.95703125
1.15 3736.95703125
1.16 3736.95703125
1.17 3736.95703125
1.18 3736.95703125
1.19 3736.95703125
1.2 3736.95703125
1.21 3736.95703125
1.22 3736.95703125
1.23 3736.95703125
1.24 3736.95703125
1.25 3736.95703125
1.26 3736.95703125
1.27 3736.95703125
1.28 3736.95703125
1.29 3736.95703125
1.3 3736.95703125
1.31 3736.95703125
1.32 3736.95703125
1.33 3736.95703125
1.34 3736.95703125
1.35 3736.95703125
1.36 3736.95703125
1.37 3736.95703125
1.38 3736.95703125
1.39 3736.95703125
1.4 3736.95703125
1.41 3736.95703125
1.42 3736.95703125
1.43 3736.95703125
1.44 3509.83203125
1.45 3017.35546875
1.46 2524.87890625
1.47 2055.4765625
1.48 1613.12109375
1.49 1191.12109375
1.5 767.4609375
1.51 601.12109375
1.52 601.171875
};
\addplot [, color0, dotted, forget plot]
table {%
0 275.95703125
0.01 276.17578125
0.02 298.859375
0.03 331.53125
0.04 367.625
0.05 371.15234375
0.06 375.81640625
0.07 386.421875
0.08 396
0.09 409.2421875
0.1 421.2421875
0.11 433.90625
0.12 502.8671875
0.13 481.8828125
0.14 488.13671875
0.15 492.13671875
0.16 498.13671875
0.17 478.62109375
0.18 498.21484375
0.19 431.39453125
0.2 420.38671875
0.21 420.38671875
0.22 420.90625
0.23 433.5390625
0.24 435.59765625
0.25 436.1796875
0.26 436.1796875
0.27 436.1796875
0.28 436.1796875
0.29 436.1796875
0.3 436.91015625
0.31 470.734375
0.32 503.66796875
0.33 603.3359375
0.34 799.53125
0.35 995.2109375
0.36 1189.6015625
0.37 1387.6015625
0.38 1586.1171875
0.39 1784.1171875
0.4 1982.6328125
0.41 2120.578125
0.42 2120.578125
0.43 2123.3046875
0.44 2140.05859375
0.45 2160.71484375
0.46 2173.51953125
0.47 2173.51953125
0.48 2238.015625
0.49 2373.3671875
0.5 2508.71875
0.51 2644.328125
0.52 2779.421875
0.53 2915.2890625
0.54 3051.4140625
0.55 3187.28125
0.56 3323.921875
0.57 3460.3046875
0.58 3596.171875
0.59 3717.34375
0.6 3717.34375
0.61 3717.34375
0.62 3717.34375
0.63 3717.34375
0.64 3717.34375
0.65 3717.34375
0.66 3717.34375
0.67 3717.34375
0.68 3717.34375
0.69 3717.34375
0.7 3717.34375
0.71 3717.34375
0.72 3717.34375
0.73 3717.34375
0.74 3717.34375
0.75 3717.34375
0.76 3717.34375
0.77 3717.34375
0.78 3717.34375
0.79 3717.34375
0.8 3717.34375
0.81 3717.34375
0.82 3717.34375
0.83 3717.34375
0.84 3717.34375
0.85 3717.34375
0.86 3717.34375
0.87 3717.34375
0.88 3717.34375
0.89 3717.34375
0.9 3717.34375
0.91 3717.34375
0.92 3717.34375
0.93 3717.34375
0.94 3717.34375
0.95 3717.34375
0.96 3717.34375
0.97 3717.34375
0.98 3717.34375
0.99 3717.34375
1 3717.34375
1.01 3717.34375
1.02 3717.34375
1.03 3717.34375
1.04 3717.34375
1.05 3717.34375
1.06 3717.34375
1.07 3717.34375
1.08 3717.34375
1.09 3717.34375
1.1 3717.34375
1.11 3717.34375
1.12 3717.34375
1.13 3717.34375
1.14 3717.34375
1.15 3717.34375
1.16 3717.34375
1.17 3717.34375
1.18 3717.34375
1.19 3717.34375
1.2 3717.34375
1.21 3717.34375
1.22 3717.34375
1.23 3717.34375
1.24 3717.34375
1.25 3717.34375
1.26 3717.34375
1.27 3717.34375
1.28 3717.34375
1.29 3717.34375
1.3 3717.34375
1.31 3717.34375
1.32 3717.34375
1.33 3717.34375
1.34 3717.34375
1.35 3717.34375
1.36 3717.34375
1.37 3717.34375
1.38 3717.34375
1.39 3717.34375
1.4 3717.34375
1.41 3717.34375
1.42 3717.34375
1.43 3717.34375
1.44 3717.34375
1.45 3338.6875
1.46 2846.2109375
1.47 2353.734375
1.48 1915.5078125
1.49 1505.50390625
1.5 1061.5078125
1.51 639.5078125
1.52 581.5546875
};
\addplot [, color0, dotted, forget plot]
table {%
0 275.7265625
0.01 275.93359375
0.02 298.75
0.03 332.46875
0.04 364.1796875
0.05 370.80078125
0.06 370.80078125
0.07 381.75
0.08 390.515625
0.09 400.97265625
0.1 415.20703125
0.11 427.37109375
0.12 464.9296875
0.13 479.41796875
0.14 484.703125
0.15 490.703125
0.16 494.703125
0.17 500.703125
0.18 487.17578125
0.19 502.38671875
0.2 420.03515625
0.21 420.55078125
0.22 420.55078125
0.23 427.23828125
0.24 434.19921875
0.25 436.34765625
0.26 436.34765625
0.27 436.34765625
0.28 436.34765625
0.29 436.34765625
0.3 436.41796875
0.31 437.9609375
0.32 487.875
0.33 526.6953125
0.34 693.7578125
0.35 889.953125
0.36 1085.375
0.37 1283.1171875
0.38 1480.6015625
0.39 1678.0859375
0.4 1876.6015625
0.41 2074.34375
0.42 2124.98046875
0.43 2127.234375
0.44 2132.1796875
0.45 2172.3984375
0.46 2184.4375
0.47 2197.0625
0.48 2202.0625
0.49 2334.3203125
0.5 2470.1875
0.51 2605.796875
0.52 2740.890625
0.53 2876.7578125
0.54 3013.3984375
0.55 3149.5234375
0.56 3285.6484375
0.57 3421.515625
0.58 3557.640625
0.59 3693.5078125
0.6 3741.203125
0.61 3741.203125
0.62 3741.203125
0.63 3741.203125
0.64 3741.203125
0.65 3741.203125
0.66 3741.203125
0.67 3741.203125
0.68 3741.203125
0.69 3741.203125
0.7 3741.203125
0.71 3741.203125
0.72 3741.203125
0.73 3741.203125
0.74 3741.203125
0.75 3741.203125
0.76 3741.203125
0.77 3741.203125
0.78 3741.203125
0.79 3741.203125
0.8 3741.203125
0.81 3741.203125
0.82 3741.203125
0.83 3741.203125
0.84 3741.203125
0.85 3741.203125
0.86 3741.203125
0.87 3741.203125
0.88 3741.203125
0.89 3741.203125
0.9 3741.203125
0.91 3741.203125
0.92 3741.203125
0.93 3741.203125
0.94 3741.203125
0.95 3741.203125
0.96 3741.203125
0.97 3741.203125
0.98 3741.203125
0.99 3741.203125
1 3741.203125
1.01 3741.203125
1.02 3741.203125
1.03 3741.203125
1.04 3741.203125
1.05 3741.203125
1.06 3741.203125
1.07 3741.203125
1.08 3741.203125
1.09 3741.203125
1.1 3741.203125
1.11 3741.203125
1.12 3741.203125
1.13 3741.203125
1.14 3741.203125
1.15 3741.203125
1.16 3741.203125
1.17 3741.203125
1.18 3741.203125
1.19 3741.203125
1.2 3741.203125
1.21 3741.203125
1.22 3741.203125
1.23 3741.203125
1.24 3741.203125
1.25 3741.203125
1.26 3741.203125
1.27 3741.203125
1.28 3741.203125
1.29 3741.203125
1.3 3741.203125
1.31 3741.203125
1.32 3741.203125
1.33 3741.203125
1.34 3741.203125
1.35 3741.203125
1.36 3741.203125
1.37 3741.203125
1.38 3741.203125
1.39 3741.203125
1.4 3741.203125
1.41 3741.203125
1.42 3741.203125
1.43 3741.203125
1.44 3741.203125
1.45 3305.3671875
1.46 2794.3046875
1.47 2301.828125
1.48 1870.30859375
1.49 1445.3671875
1.5 1036.88671875
1.51 605.3671875
1.52 605.41015625
};
\addplot [, color1, dotted, forget plot]
table {%
0 275.76953125
0.01 275.97265625
0.02 298.5390625
0.03 334.11328125
0.04 368.64453125
0.05 370.90625
0.06 376.0234375
0.07 386.484375
0.08 395.7265625
0.09 409.08203125
0.1 421.08203125
0.11 433.77734375
0.12 502.22265625
0.13 482.01171875
0.14 488.01171875
0.15 492.01171875
0.16 498.01171875
0.17 478.234375
0.18 498.0859375
0.19 439.265625
0.2 420.35546875
0.21 420.35546875
0.22 421.90234375
0.23 433.50390625
0.24 435.56640625
0.25 436.16015625
0.26 436.16015625
0.27 436.16015625
0.28 436.16015625
0.29 436.16015625
0.3 436.7578125
0.31 473.53125
0.32 509.59375
0.33 571.609375
0.34 765.484375
0.35 962.1953125
0.36 1159.1640625
0.37 1356.1328125
0.38 1553.359375
0.39 1751.1015625
0.4 1949.359375
0.41 2122.8671875
0.42 2258.734375
0.43 2394.34375
0.44 2529.4375
0.45 2664.7890625
0.46 2799.8828125
0.47 2936.0078125
0.48 3072.1328125
0.49 3208.515625
0.5 3344.8984375
0.51 3481.0234375
0.52 3617.1484375
0.53 3672.0625
0.54 3672.0625
0.55 3672.0625
0.56 3672.0625
0.57 3672.0625
0.58 3672.0625
0.59 3672.0625
0.6 3672.0625
0.61 3672.0625
0.62 3672.0625
0.63 3672.0625
0.64 3672.0625
0.65 3672.0625
0.66 3672.0625
0.67 3672.0625
0.68 3672.0625
0.69 3672.0625
0.7 3672.0625
0.71 3672.0625
0.72 3672.0625
0.73 3672.0625
0.74 3672.0625
0.75 3672.0625
0.76 3672.0625
0.77 3672.0625
0.78 3672.0625
0.79 3672.0625
0.8 3672.0625
0.81 3672.0625
0.82 3672.0625
0.83 3672.0625
0.84 3672.0625
0.85 3672.0625
0.86 3672.0625
0.87 3672.0625
0.88 3672.0625
0.89 3672.0625
0.9 3672.0625
0.91 3672.0625
0.92 3672.0625
0.93 3672.0625
0.94 3672.0625
0.95 3672.0625
0.96 3672.0625
0.97 3672.0625
0.98 3672.0625
0.99 3672.0625
1 3672.0625
1.01 3672.0625
1.02 3672.0625
1.03 3672.0625
1.04 3672.0625
1.05 3672.0625
1.06 3672.0625
1.07 3672.0625
1.08 3672.0625
1.09 3672.0625
1.1 3672.0625
1.11 3672.0625
1.12 3672.0625
1.13 3672.0625
1.14 3672.0625
1.15 3672.0625
1.16 3672.0625
1.17 3672.0625
1.18 3672.0625
1.19 3672.0625
1.2 3672.0625
1.21 3672.0625
1.22 3672.0625
1.23 3672.0625
1.24 3672.0625
1.25 3672.0625
1.26 3672.0625
1.27 3672.0625
1.28 3672.0625
1.29 3672.0625
1.3 3672.0625
1.31 3672.0625
1.32 3672.0625
1.33 3672.0625
1.34 3672.0625
1.35 3672.0625
1.36 3672.0625
1.37 3672.0625
1.38 3230.12890625
1.39 2725.06640625
1.4 2232.58984375
1.41 1801.0703125
1.42 1384.359375
1.43 967.6484375
1.44 536.12890625
1.45 548.34765625
1.46 550.40234375
1.47 559.3515625
1.48 602.1484375
1.49 594.84765625
1.5 569.91015625
1.51 593.87890625
1.52 594.41015625
};
\addplot [, color1, dotted, forget plot]
table {%
0 275.80078125
0.01 276
0.02 298.6796875
0.03 334.0078125
0.04 368.0390625
0.05 371.05078125
0.06 375.13671875
0.07 385.671875
0.08 394.69921875
0.09 407.28125
0.1 420.02734375
0.11 430.46875
0.12 493.27734375
0.13 481.4765625
0.14 487.4765625
0.15 491.4765625
0.16 497.4765625
0.17 502.203125
0.18 497.2890625
0.19 444.25
0.2 420.59375
0.21 420.59375
0.22 421.609375
0.23 433.7265625
0.24 435.78515625
0.25 436.39453125
0.26 436.39453125
0.27 436.39453125
0.28 436.39453125
0.29 436.39453125
0.3 436.98828125
0.31 467.8359375
0.32 489.734375
0.33 522.30859375
0.34 579.31640625
0.35 758.23828125
0.36 938.19140625
0.37 1119.17578125
0.38 1300.67578125
0.39 1482.43359375
0.4 1665.48046875
0.41 1848.26953125
0.42 1969.95703125
0.43 2023.06640625
0.44 2077.20703125
0.45 2164.60546875
0.46 2294.80078125
0.47 2425.25390625
0.48 2555.96484375
0.49 2686.67578125
0.5 2816.87109375
0.51 2948.09765625
0.52 3079.58203125
0.53 3209.77734375
0.54 3341.00390625
0.55 3471.71484375
0.56 3552.15234375
0.57 3599.07421875
0.58 3646.51171875
0.59 3672.03515625
0.6 3672.03515625
0.61 3672.03515625
0.62 3672.03515625
0.63 3672.03515625
0.64 3672.03515625
0.65 3672.03515625
0.66 3672.03515625
0.67 3672.03515625
0.68 3672.03515625
0.69 3672.03515625
0.7 3672.03515625
0.71 3672.03515625
0.72 3672.03515625
0.73 3672.03515625
0.74 3672.03515625
0.75 3672.03515625
0.76 3672.03515625
0.77 3672.03515625
0.78 3672.03515625
0.79 3672.03515625
0.8 3672.03515625
0.81 3672.03515625
0.82 3672.03515625
0.83 3672.03515625
0.84 3672.03515625
0.85 3672.03515625
0.86 3672.03515625
0.87 3672.03515625
0.88 3672.03515625
0.89 3672.03515625
0.9 3672.03515625
0.91 3672.03515625
0.92 3672.03515625
0.93 3672.03515625
0.94 3672.03515625
0.95 3672.03515625
0.96 3672.03515625
0.97 3672.03515625
0.98 3672.03515625
0.99 3672.03515625
1 3672.03515625
1.01 3672.03515625
1.02 3672.03515625
1.03 3672.03515625
1.04 3672.03515625
1.05 3672.03515625
1.06 3672.03515625
1.07 3672.03515625
1.08 3672.03515625
1.09 3672.03515625
1.1 3672.03515625
1.11 3672.03515625
1.12 3672.03515625
1.13 3672.03515625
1.14 3672.03515625
1.15 3672.03515625
1.16 3672.03515625
1.17 3672.03515625
1.18 3672.03515625
1.19 3672.03515625
1.2 3672.03515625
1.21 3672.03515625
1.22 3672.03515625
1.23 3672.03515625
1.24 3672.03515625
1.25 3672.03515625
1.26 3672.03515625
1.27 3672.03515625
1.28 3672.03515625
1.29 3672.03515625
1.3 3672.03515625
1.31 3672.03515625
1.32 3672.03515625
1.33 3672.03515625
1.34 3672.03515625
1.35 3672.03515625
1.36 3672.03515625
1.37 3672.03515625
1.38 3672.03515625
1.39 3672.03515625
1.4 3672.03515625
1.41 3672.03515625
1.42 3672.03515625
1.43 3672.03515625
1.44 3514.1171875
1.45 2990.234375
1.46 2497.7578125
1.47 2028.35546875
1.48 1606.1171875
1.49 1194.93359375
1.5 772.1171875
1.51 560.45703125
1.52 560.7109375
1.53 562.31640625
1.54 563.765625
1.55 585.64453125
1.56 610.13671875
1.57 607.9296875
1.58 582.9921875
1.59 582.9921875
1.6 583.515625
};
\addplot [, color1, dotted, forget plot]
table {%
0 275.58984375
0.01 275.7890625
0.02 297.72265625
0.03 332.89453125
0.04 368.40234375
0.05 370.46484375
0.06 375.57421875
0.07 386.57421875
0.08 395.51171875
0.09 408.61328125
0.1 421.85546875
0.11 439.046875
0.12 503.5546875
0.13 482.0546875
0.14 488.0546875
0.15 494.0546875
0.16 498.0546875
0.17 481.37109375
0.18 500.96484375
0.19 429.11328125
0.2 420.4140625
0.21 420.4140625
0.22 424.4765625
0.23 433.5
0.24 436.21484375
0.25 436.21484375
0.26 436.21484375
0.27 436.21484375
0.28 436.21484375
0.29 436.21484375
0.3 436.765625
0.31 484.45703125
0.32 520.19140625
0.33 600.34765625
0.34 794.99609375
0.35 991.19140625
0.36 1188.41796875
0.37 1385.90234375
0.38 1582.61328125
0.39 1781.38671875
0.4 1979.64453125
0.41 2138.45703125
0.42 2273.55078125
0.43 2409.16015625
0.44 2544.51171875
0.45 2679.60546875
0.46 2814.95703125
0.47 2951.08203125
0.48 3086.69140625
0.49 3222.81640625
0.5 3358.68359375
0.51 3494.80859375
0.52 3630.67578125
0.53 3672.44140625
0.54 3672.44140625
0.55 3672.44140625
0.56 3672.44140625
0.57 3672.44140625
0.58 3672.44140625
0.59 3672.44140625
0.6 3672.44140625
0.61 3672.44140625
0.62 3672.44140625
0.63 3672.44140625
0.64 3672.44140625
0.65 3672.44140625
0.66 3672.44140625
0.67 3672.44140625
0.68 3672.44140625
0.69 3672.44140625
0.7 3672.44140625
0.71 3672.44140625
0.72 3672.44140625
0.73 3672.44140625
0.74 3672.44140625
0.75 3672.44140625
0.76 3672.44140625
0.77 3672.44140625
0.78 3672.44140625
0.79 3672.44140625
0.8 3672.44140625
0.81 3672.44140625
0.82 3672.44140625
0.83 3672.44140625
0.84 3672.44140625
0.85 3672.44140625
0.86 3672.44140625
0.87 3672.44140625
0.88 3672.44140625
0.89 3672.44140625
0.9 3672.44140625
0.91 3672.44140625
0.92 3672.44140625
0.93 3672.44140625
0.94 3672.44140625
0.95 3672.44140625
0.96 3672.44140625
0.97 3672.44140625
0.98 3672.44140625
0.99 3672.44140625
1 3672.44140625
1.01 3672.44140625
1.02 3672.44140625
1.03 3672.44140625
1.04 3672.44140625
1.05 3672.44140625
1.06 3672.44140625
1.07 3672.44140625
1.08 3672.44140625
1.09 3672.44140625
1.1 3672.44140625
1.11 3672.44140625
1.12 3672.44140625
1.13 3672.44140625
1.14 3672.44140625
1.15 3672.44140625
1.16 3672.44140625
1.17 3672.44140625
1.18 3672.44140625
1.19 3672.44140625
1.2 3672.44140625
1.21 3672.44140625
1.22 3672.44140625
1.23 3672.44140625
1.24 3672.44140625
1.25 3672.44140625
1.26 3672.44140625
1.27 3672.44140625
1.28 3672.44140625
1.29 3672.44140625
1.3 3672.44140625
1.31 3672.44140625
1.32 3672.44140625
1.33 3672.44140625
1.34 3672.44140625
1.35 3672.44140625
1.36 3672.44140625
1.37 3672.44140625
1.38 3180.0625
1.39 2649.703125
1.4 2157.2265625
1.41 1725.70703125
1.42 1308.99609375
1.43 892.28515625
1.44 542.82421875
1.45 561.23828125
1.46 563.50390625
1.47 575.09375
1.48 595.90234375
1.49 609.05078125
1.5 584.11328125
1.51 584.8359375
};
\addplot [, color1, dotted, forget plot]
table {%
0 275.83203125
0.01 275.9609375
0.02 297.47265625
0.03 331.96875
0.04 367.8046875
0.05 370.6171875
0.06 374.1953125
0.07 386.0234375
0.08 395.859375
0.09 409.21875
0.1 421.21875
0.11 433.82421875
0.12 502.78515625
0.13 482.31640625
0.14 488.31640625
0.15 492.31640625
0.16 498.31640625
0.17 479.0546875
0.18 498.6484375
0.19 419.81640625
0.2 420.65625
0.21 420.65625
0.22 422.41015625
0.23 433.75390625
0.24 436.0703125
0.25 436.45703125
0.26 436.45703125
0.27 436.45703125
0.28 436.45703125
0.29 436.45703125
0.3 437.23828125
0.31 472.10546875
0.32 508.68359375
0.33 559.625
0.34 754.015625
0.35 948.921875
0.36 1144.6015625
0.37 1341.5703125
0.38 1538.0234375
0.39 1734.734375
0.4 1932.21875
0.41 2105.46875
0.42 2240.3046875
0.43 2376.4296875
0.44 2511.5234375
0.45 2646.875
0.46 2782.2265625
0.47 2918.3515625
0.48 3053.703125
0.49 3190.0859375
0.5 3325.953125
0.51 3461.5625
0.52 3597.6875
0.53 3672.453125
0.54 3672.453125
0.55 3672.453125
0.56 3672.453125
0.57 3672.453125
0.58 3672.453125
0.59 3672.453125
0.6 3672.453125
0.61 3672.453125
0.62 3672.453125
0.63 3672.453125
0.64 3672.453125
0.65 3672.453125
0.66 3672.453125
0.67 3672.453125
0.68 3672.453125
0.69 3672.453125
0.7 3672.453125
0.71 3672.453125
0.72 3672.453125
0.73 3672.453125
0.74 3672.453125
0.75 3672.453125
0.76 3672.453125
0.77 3672.453125
0.78 3672.453125
0.79 3672.453125
0.8 3672.453125
0.81 3672.453125
0.82 3672.453125
0.83 3672.453125
0.84 3672.453125
0.85 3672.453125
0.86 3672.453125
0.87 3672.453125
0.88 3672.453125
0.89 3672.453125
0.9 3672.453125
0.91 3672.453125
0.92 3672.453125
0.93 3672.453125
0.94 3672.453125
0.95 3672.453125
0.96 3672.453125
0.97 3672.453125
0.98 3672.453125
0.99 3672.453125
1 3672.453125
1.01 3672.453125
1.02 3672.453125
1.03 3672.453125
1.04 3672.453125
1.05 3672.453125
1.06 3672.453125
1.07 3672.453125
1.08 3672.453125
1.09 3672.453125
1.1 3672.453125
1.11 3672.453125
1.12 3672.453125
1.13 3672.453125
1.14 3672.453125
1.15 3672.453125
1.16 3672.453125
1.17 3672.453125
1.18 3672.453125
1.19 3672.453125
1.2 3672.453125
1.21 3672.453125
1.22 3672.453125
1.23 3672.453125
1.24 3672.453125
1.25 3672.453125
1.26 3672.453125
1.27 3672.453125
1.28 3672.453125
1.29 3672.453125
1.3 3672.453125
1.31 3672.453125
1.32 3672.453125
1.33 3672.453125
1.34 3672.453125
1.35 3672.453125
1.36 3672.453125
1.37 3672.453125
1.38 3369.53515625
1.39 2877.05859375
1.4 2368.58984375
1.41 1915.1796875
1.42 1498.46875
1.43 1052.58984375
1.44 648.58984375
1.45 561.3046875
1.46 562.421875
1.47 564.25
1.48 589.37890625
1.49 608.88671875
1.5 583.88671875
1.51 583.88671875
1.52 584.45703125
};
\addplot [, color1, dotted, forget plot]
table {%
0 275.44140625
0.01 275.640625
0.02 298.18359375
0.03 333.3671875
0.04 367.65625
0.05 370.47265625
0.06 374.30859375
0.07 385.33203125
0.08 394
0.09 406.734375
0.1 420.734375
0.11 430.34375
0.12 496.359375
0.13 481.60546875
0.14 487.60546875
0.15 491.60546875
0.16 497.60546875
0.17 501.8203125
0.18 496.1328125
0.19 444.125
0.2 420.2109375
0.21 420.2109375
0.22 420.66796875
0.23 433.30078125
0.24 435.359375
0.25 436.00390625
0.26 436.00390625
0.27 436.00390625
0.28 436.00390625
0.29 436.00390625
0.3 436.5625
0.31 469.796875
0.32 508.484375
0.33 552.87890625
0.34 746.49609375
0.35 942.43359375
0.36 1138.62890625
0.37 1335.85546875
0.38 1533.59765625
0.39 1731.33984375
0.4 1928.56640625
0.41 2103.87890625
0.42 2236.91015625
0.43 2372.26171875
0.44 2507.35546875
0.45 2642.96484375
0.46 2778.31640625
0.47 2914.18359375
0.48 3050.82421875
0.49 3186.94921875
0.5 3322.55859375
0.51 3459.45703125
0.52 3595.58203125
0.53 3672.15234375
0.54 3672.15234375
0.55 3672.15234375
0.56 3672.15234375
0.57 3672.15234375
0.58 3672.15234375
0.59 3672.15234375
0.6 3672.15234375
0.61 3672.15234375
0.62 3672.15234375
0.63 3672.15234375
0.64 3672.15234375
0.65 3672.15234375
0.66 3672.15234375
0.67 3672.15234375
0.68 3672.15234375
0.69 3672.15234375
0.7 3672.15234375
0.71 3672.15234375
0.72 3672.15234375
0.73 3672.15234375
0.74 3672.15234375
0.75 3672.15234375
0.76 3672.15234375
0.77 3672.15234375
0.78 3672.15234375
0.79 3672.15234375
0.8 3672.15234375
0.81 3672.15234375
0.82 3672.15234375
0.83 3672.15234375
0.84 3672.15234375
0.85 3672.15234375
0.86 3672.15234375
0.87 3672.15234375
0.88 3672.15234375
0.89 3672.15234375
0.9 3672.15234375
0.91 3672.15234375
0.92 3672.15234375
0.93 3672.15234375
0.94 3672.15234375
0.95 3672.15234375
0.96 3672.15234375
0.97 3672.15234375
0.98 3672.15234375
0.99 3672.15234375
1 3672.15234375
1.01 3672.15234375
1.02 3672.15234375
1.03 3672.15234375
1.04 3672.15234375
1.05 3672.15234375
1.06 3672.15234375
1.07 3672.15234375
1.08 3672.15234375
1.09 3672.15234375
1.1 3672.15234375
1.11 3672.15234375
1.12 3672.15234375
1.13 3672.15234375
1.14 3672.15234375
1.15 3672.15234375
1.16 3672.15234375
1.17 3672.15234375
1.18 3672.15234375
1.19 3672.15234375
1.2 3672.15234375
1.21 3672.15234375
1.22 3672.15234375
1.23 3672.15234375
1.24 3672.15234375
1.25 3672.15234375
1.26 3672.15234375
1.27 3672.15234375
1.28 3672.15234375
1.29 3672.15234375
1.3 3672.15234375
1.31 3672.15234375
1.32 3672.15234375
1.33 3672.15234375
1.34 3672.15234375
1.35 3672.15234375
1.36 3672.15234375
1.37 3672.15234375
1.38 3407.109375
1.39 2914.6328125
1.4 2384.2734375
1.41 1948.28125
1.42 1498.16015625
1.43 1081.44921875
1.44 664.73828125
1.45 548.3046875
1.46 550.85546875
1.47 551.94140625
1.48 576.6328125
1.49 594.8515625
1.5 569.9140625
1.51 594.140625
1.52 594.72265625
};
\addplot [, color1, dotted, forget plot]
table {%
0 275.73046875
0.01 275.93359375
0.02 298.01953125
0.03 333.078125
0.04 368.640625
0.05 370.90234375
0.06 376.046875
0.07 386.76953125
0.08 395.54296875
0.09 408.90625
0.1 421.12890625
0.11 435.6171875
0.12 503.3515625
0.13 481.8515625
0.14 487.8515625
0.15 491.8515625
0.16 497.8515625
0.17 479.109375
0.18 498.703125
0.19 419.61328125
0.2 420.35546875
0.21 420.35546875
0.22 422.2421875
0.23 433.5859375
0.24 435.90234375
0.25 436.16015625
0.26 436.16015625
0.27 436.16015625
0.28 436.16015625
0.29 436.16015625
0.3 436.8203125
0.31 478.72265625
0.32 509.90625
0.33 576.69921875
0.34 771.86328125
0.35 967.54296875
0.36 1165.28515625
0.37 1362.76953125
0.38 1560.51171875
0.39 1758.76953125
0.4 1957.02734375
0.41 2127.69921875
0.42 2262.79296875
0.43 2398.66015625
0.44 2534.26953125
0.45 2669.36328125
0.46 2804.45703125
0.47 2940.58203125
0.48 3076.96484375
0.49 3212.31640625
0.5 3348.69921875
0.51 3481.73046875
0.52 3617.85546875
0.53 3672.25390625
0.54 3672.25390625
0.55 3672.25390625
0.56 3672.25390625
0.57 3672.25390625
0.58 3672.25390625
0.59 3672.25390625
0.6 3672.25390625
0.61 3672.25390625
0.62 3672.25390625
0.63 3672.25390625
0.64 3672.25390625
0.65 3672.25390625
0.66 3672.25390625
0.67 3672.25390625
0.68 3672.25390625
0.69 3672.25390625
0.7 3672.25390625
0.71 3672.25390625
0.72 3672.25390625
0.73 3672.25390625
0.74 3672.25390625
0.75 3672.25390625
0.76 3672.25390625
0.77 3672.25390625
0.78 3672.25390625
0.79 3672.25390625
0.8 3672.25390625
0.81 3672.25390625
0.82 3672.25390625
0.83 3672.25390625
0.84 3672.25390625
0.85 3672.25390625
0.86 3672.25390625
0.87 3672.25390625
0.88 3672.25390625
0.89 3672.25390625
0.9 3672.25390625
0.91 3672.25390625
0.92 3672.25390625
0.93 3672.25390625
0.94 3672.25390625
0.95 3672.25390625
0.96 3672.25390625
0.97 3672.25390625
0.98 3672.25390625
0.99 3672.25390625
1 3672.25390625
1.01 3672.25390625
1.02 3672.25390625
1.03 3672.25390625
1.04 3672.25390625
1.05 3672.25390625
1.06 3672.25390625
1.07 3672.25390625
1.08 3672.25390625
1.09 3672.25390625
1.1 3672.25390625
1.11 3672.25390625
1.12 3672.25390625
1.13 3672.25390625
1.14 3672.25390625
1.15 3672.25390625
1.16 3672.25390625
1.17 3672.25390625
1.18 3672.25390625
1.19 3672.25390625
1.2 3672.25390625
1.21 3672.25390625
1.22 3672.25390625
1.23 3672.25390625
1.24 3672.25390625
1.25 3672.25390625
1.26 3672.25390625
1.27 3672.25390625
1.28 3672.25390625
1.29 3672.25390625
1.3 3672.25390625
1.31 3672.25390625
1.32 3672.25390625
1.33 3672.25390625
1.34 3672.25390625
1.35 3672.25390625
1.36 3672.25390625
1.37 3672.25390625
1.38 3407.19140625
1.39 2876.83203125
1.4 2384.35546875
1.41 1942.36328125
1.42 1498.2421875
1.43 1081.53125
1.44 664.8203125
1.45 561.37890625
1.46 562.75390625
1.47 564.4296875
1.48 588.6640625
1.49 608.140625
1.5 583.203125
1.51 595.05859375
1.52 595.7109375
};
\addplot [, color1, dotted, forget plot]
table {%
0 275.48828125
0.01 275.61328125
0.02 298.80859375
0.03 333.921875
0.04 368.1484375
0.05 370.2109375
0.06 374.80078125
0.07 385.9453125
0.08 395.3203125
0.09 408.67578125
0.1 420.6953125
0.11 434.83984375
0.12 501.09375
0.13 481.078125
0.14 487.078125
0.15 493.078125
0.16 499.078125
0.17 478.84375
0.18 498.4375
0.19 419.34765625
0.2 420.078125
0.21 420.078125
0.22 421.96875
0.23 433.3125
0.24 435.62890625
0.25 435.87109375
0.26 435.87109375
0.27 435.87109375
0.28 435.87109375
0.29 435.87109375
0.3 436.41796875
0.31 481.8671875
0.32 507.47265625
0.33 574.359375
0.34 768.75
0.35 965.203125
0.36 1162.6875
0.37 1360.4296875
0.38 1558.4296875
0.39 1756.6875
0.4 1954.171875
0.41 2125.6171875
0.42 2261.2265625
0.43 2393.7421875
0.44 2528.578125
0.45 2664.1875
0.46 2799.5390625
0.47 2935.1484375
0.48 3071.015625
0.49 3207.3984375
0.5 3343.0078125
0.51 3479.1328125
0.52 3615.515625
0.53 3664.2421875
0.54 3664.2421875
0.55 3664.2421875
0.56 3664.2421875
0.57 3664.2421875
0.58 3664.2421875
0.59 3664.2421875
0.6 3664.2421875
0.61 3664.2421875
0.62 3664.2421875
0.63 3664.2421875
0.64 3664.2421875
0.65 3664.2421875
0.66 3664.2421875
0.67 3664.2421875
0.68 3664.2421875
0.69 3664.2421875
0.7 3664.2421875
0.71 3664.2421875
0.72 3664.2421875
0.73 3664.2421875
0.74 3664.2421875
0.75 3664.2421875
0.76 3664.2421875
0.77 3664.2421875
0.78 3664.2421875
0.79 3664.2421875
0.8 3664.2421875
0.81 3664.2421875
0.82 3664.2421875
0.83 3664.2421875
0.84 3664.2421875
0.85 3664.2421875
0.86 3664.2421875
0.87 3664.2421875
0.88 3664.2421875
0.89 3664.2421875
0.9 3664.2421875
0.91 3664.2421875
0.92 3664.2421875
0.93 3664.2421875
0.94 3664.2421875
0.95 3664.2421875
0.96 3664.2421875
0.97 3664.2421875
0.98 3664.2421875
0.99 3664.2421875
1 3664.2421875
1.01 3664.2421875
1.02 3664.2421875
1.03 3664.2421875
1.04 3664.2421875
1.05 3664.2421875
1.06 3664.2421875
1.07 3664.2421875
1.08 3664.2421875
1.09 3664.2421875
1.1 3664.2421875
1.11 3664.2421875
1.12 3664.2421875
1.13 3664.2421875
1.14 3664.2421875
1.15 3664.2421875
1.16 3664.2421875
1.17 3664.2421875
1.18 3664.2421875
1.19 3664.2421875
1.2 3664.2421875
1.21 3664.2421875
1.22 3664.2421875
1.23 3664.2421875
1.24 3664.2421875
1.25 3664.2421875
1.26 3664.2421875
1.27 3664.2421875
1.28 3664.2421875
1.29 3664.2421875
1.3 3664.2421875
1.31 3664.2421875
1.32 3664.2421875
1.33 3664.2421875
1.34 3664.2421875
1.35 3664.2421875
1.36 3664.2421875
1.37 3664.2421875
1.38 3550.7890625
1.39 3058.3125
1.4 2562.4296875
1.41 2058.55078125
1.42 1641.83984375
1.43 1225.12890625
1.44 808.41796875
1.45 552.625
1.46 553.140625
1.47 556.04296875
1.48 570.29296875
1.49 600.09765625
1.5 600.61328125
1.51 586.95703125
1.52 587.78515625
};
\addplot [, color1, dotted, forget plot]
table {%
0 275.7265625
0.01 275.92578125
0.02 298.09375
0.03 333.66796875
0.04 368.21484375
0.05 370.98046875
0.06 374.80859375
0.07 386.06640625
0.08 395.59765625
0.09 409.16015625
0.1 421.16015625
0.11 430.65234375
0.12 500.91796875
0.13 481.91796875
0.14 487.91796875
0.15 491.91796875
0.16 497.91796875
0.17 477.88671875
0.18 497.21875
0.19 444.1796875
0.2 420.53515625
0.21 420.53515625
0.22 421.24609375
0.23 433.62109375
0.24 435.6796875
0.25 436.33203125
0.26 436.33203125
0.27 436.33203125
0.28 436.33203125
0.29 436.33203125
0.3 436.92578125
0.31 471.3828125
0.32 508.4609375
0.33 554.78515625
0.34 748.14453125
0.35 944.08203125
0.36 1140.01953125
0.37 1336.21484375
0.38 1532.92578125
0.39 1730.15234375
0.4 1927.12109375
0.41 2104.23828125
0.42 2239.07421875
0.43 2374.16796875
0.44 2509.00390625
0.45 2644.61328125
0.46 2779.70703125
0.47 2915.31640625
0.48 3051.44140625
0.49 3187.56640625
0.5 3323.43359375
0.51 3459.30078125
0.52 3595.68359375
0.53 3671.73828125
0.54 3671.73828125
0.55 3671.73828125
0.56 3671.73828125
0.57 3671.73828125
0.58 3671.73828125
0.59 3671.73828125
0.6 3671.73828125
0.61 3671.73828125
0.62 3671.73828125
0.63 3671.73828125
0.64 3671.73828125
0.65 3671.73828125
0.66 3671.73828125
0.67 3671.73828125
0.68 3671.73828125
0.69 3671.73828125
0.7 3671.73828125
0.71 3671.73828125
0.72 3671.73828125
0.73 3671.73828125
0.74 3671.73828125
0.75 3671.73828125
0.76 3671.73828125
0.77 3671.73828125
0.78 3671.73828125
0.79 3671.73828125
0.8 3671.73828125
0.81 3671.73828125
0.82 3671.73828125
0.83 3671.73828125
0.84 3671.73828125
0.85 3671.73828125
0.86 3671.73828125
0.87 3671.73828125
0.88 3671.73828125
0.89 3671.73828125
0.9 3671.73828125
0.91 3671.73828125
0.92 3671.73828125
0.93 3671.73828125
0.94 3671.73828125
0.95 3671.73828125
0.96 3671.73828125
0.97 3671.73828125
0.98 3671.73828125
0.99 3671.73828125
1 3671.73828125
1.01 3671.73828125
1.02 3671.73828125
1.03 3671.73828125
1.04 3671.73828125
1.05 3671.73828125
1.06 3671.73828125
1.07 3671.73828125
1.08 3671.73828125
1.09 3671.73828125
1.1 3671.73828125
1.11 3671.73828125
1.12 3671.73828125
1.13 3671.73828125
1.14 3671.73828125
1.15 3671.73828125
1.16 3671.73828125
1.17 3671.73828125
1.18 3671.73828125
1.19 3671.73828125
1.2 3671.73828125
1.21 3671.73828125
1.22 3671.73828125
1.23 3671.73828125
1.24 3671.73828125
1.25 3671.73828125
1.26 3671.73828125
1.27 3671.73828125
1.28 3671.73828125
1.29 3671.73828125
1.3 3671.73828125
1.31 3671.73828125
1.32 3671.73828125
1.33 3671.73828125
1.34 3671.73828125
1.35 3671.73828125
1.36 3671.73828125
1.37 3671.73828125
1.38 3633.93359375
1.39 3141.45703125
1.4 2611.09765625
1.41 2118.62109375
1.42 1707.80859375
1.43 1270.390625
1.44 865.80859375
1.45 551.06640625
1.46 560.65625
1.47 562.90234375
1.48 575.7890625
1.49 602.1953125
1.5 607.8671875
1.51 582.9296875
1.52 583.453125
};
\addplot [, color1, dotted, forget plot]
table {%
0 276.06640625
0.01 276.265625
0.02 297.73046875
0.03 331.3828125
0.04 366.9609375
0.05 371.0703125
0.06 372.04296875
0.07 386.0234375
0.08 395.1015625
0.09 407.5078125
0.1 421.5078125
0.11 431.15625
0.12 499.421875
0.13 482.421875
0.14 488.421875
0.15 492.421875
0.16 498.421875
0.17 479.88671875
0.18 497.72265625
0.19 444.68359375
0.2 421.01953125
0.21 421.01953125
0.22 422.296875
0.23 434.15625
0.24 436.21484375
0.25 436.8203125
0.26 436.8203125
0.27 436.8203125
0.28 436.8203125
0.29 436.8203125
0.3 437.3671875
0.31 472.0859375
0.32 508.91796875
0.33 564.02734375
0.34 759.96484375
0.35 956.41796875
0.36 1153.38671875
0.37 1350.87109375
0.38 1549.12890625
0.39 1747.38671875
0.4 1945.64453125
0.41 2128.17578125
0.42 2263.26953125
0.43 2398.87890625
0.44 2534.48828125
0.45 2669.83984375
0.46 2804.93359375
0.47 2941.31640625
0.48 3076.92578125
0.49 3213.30859375
0.5 3349.17578125
0.51 3485.55859375
0.52 3621.42578125
0.53 3672.21484375
0.54 3672.21484375
0.55 3672.21484375
0.56 3672.21484375
0.57 3672.21484375
0.58 3672.21484375
0.59 3672.21484375
0.6 3672.21484375
0.61 3672.21484375
0.62 3672.21484375
0.63 3672.21484375
0.64 3672.21484375
0.65 3672.21484375
0.66 3672.21484375
0.67 3672.21484375
0.68 3672.21484375
0.69 3672.21484375
0.7 3672.21484375
0.71 3672.21484375
0.72 3672.21484375
0.73 3672.21484375
0.74 3672.21484375
0.75 3672.21484375
0.76 3672.21484375
0.77 3672.21484375
0.78 3672.21484375
0.79 3672.21484375
0.8 3672.21484375
0.81 3672.21484375
0.82 3672.21484375
0.83 3672.21484375
0.84 3672.21484375
0.85 3672.21484375
0.86 3672.21484375
0.87 3672.21484375
0.88 3672.21484375
0.89 3672.21484375
0.9 3672.21484375
0.91 3672.21484375
0.92 3672.21484375
0.93 3672.21484375
0.94 3672.21484375
0.95 3672.21484375
0.96 3672.21484375
0.97 3672.21484375
0.98 3672.21484375
0.99 3672.21484375
1 3672.21484375
1.01 3672.21484375
1.02 3672.21484375
1.03 3672.21484375
1.04 3672.21484375
1.05 3672.21484375
1.06 3672.21484375
1.07 3672.21484375
1.08 3672.21484375
1.09 3672.21484375
1.1 3672.21484375
1.11 3672.21484375
1.12 3672.21484375
1.13 3672.21484375
1.14 3672.21484375
1.15 3672.21484375
1.16 3672.21484375
1.17 3672.21484375
1.18 3672.21484375
1.19 3672.21484375
1.2 3672.21484375
1.21 3672.21484375
1.22 3672.21484375
1.23 3672.21484375
1.24 3672.21484375
1.25 3672.21484375
1.26 3672.21484375
1.27 3672.21484375
1.28 3672.21484375
1.29 3672.21484375
1.3 3672.21484375
1.31 3672.21484375
1.32 3672.21484375
1.33 3672.21484375
1.34 3672.21484375
1.35 3672.21484375
1.36 3672.21484375
1.37 3672.21484375
1.38 3672.21484375
1.39 3255.578125
1.4 2763.1015625
1.41 2270.625
1.42 1828.28125
1.43 1422.39453125
1.44 1000.28125
1.45 556.28125
1.46 560.93359375
1.47 563.046875
1.48 568.2578125
1.49 607.703125
1.5 608.08984375
1.51 583.15234375
1.52 583.15234375
1.53 583.73046875
};
\addplot [, color1, dotted, forget plot]
table {%
0 275.421875
0.01 275.55859375
0.02 298.48828125
0.03 332.78125
0.04 364.75
0.05 370.08984375
0.06 370.08984375
0.07 380.4609375
0.08 388.96875
0.09 400.60546875
0.1 412.60546875
0.11 425.45703125
0.12 453.48046875
0.13 503.24609375
0.14 482.26171875
0.15 488.26171875
0.16 494.26171875
0.17 498.26171875
0.18 482.61328125
0.19 501.94921875
0.2 419.1640625
0.21 419.77734375
0.22 419.77734375
0.23 424.03515625
0.24 433.05859375
0.25 435.56640625
0.26 435.56640625
0.27 435.56640625
0.28 435.56640625
0.29 435.56640625
0.3 435.56640625
0.31 436.16796875
0.32 485.00390625
0.33 513.71875
0.34 594.92578125
0.35 790.08984375
0.36 986.28515625
0.37 1183.76953125
0.38 1381.51171875
0.39 1579.76953125
0.4 1777.76953125
0.41 1975.76953125
0.42 2136.12890625
0.43 2271.73828125
0.44 2407.60546875
0.45 2543.21484375
0.46 2678.56640625
0.47 2813.66015625
0.48 2949.78515625
0.49 3085.13671875
0.5 3221.51953125
0.51 3357.64453125
0.52 3493.25390625
0.53 3629.12109375
0.54 3666.24609375
0.55 3666.24609375
0.56 3666.24609375
0.57 3666.24609375
0.58 3666.24609375
0.59 3666.24609375
0.6 3666.24609375
0.61 3666.24609375
0.62 3666.24609375
0.63 3666.24609375
0.64 3666.24609375
0.65 3666.24609375
0.66 3666.24609375
0.67 3666.24609375
0.68 3666.24609375
0.69 3666.24609375
0.7 3666.24609375
0.71 3666.24609375
0.72 3666.24609375
0.73 3666.24609375
0.74 3666.24609375
0.75 3666.24609375
0.76 3666.24609375
0.77 3666.24609375
0.78 3666.24609375
0.79 3666.24609375
0.8 3666.24609375
0.81 3666.24609375
0.82 3666.24609375
0.83 3666.24609375
0.84 3666.24609375
0.85 3666.24609375
0.86 3666.24609375
0.87 3666.24609375
0.88 3666.24609375
0.89 3666.24609375
0.9 3666.24609375
0.91 3666.24609375
0.92 3666.24609375
0.93 3666.24609375
0.94 3666.24609375
0.95 3666.24609375
0.96 3666.24609375
0.97 3666.24609375
0.98 3666.24609375
0.99 3666.24609375
1 3666.24609375
1.01 3666.24609375
1.02 3666.24609375
1.03 3666.24609375
1.04 3666.24609375
1.05 3666.24609375
1.06 3666.24609375
1.07 3666.24609375
1.08 3666.24609375
1.09 3666.24609375
1.1 3666.24609375
1.11 3666.24609375
1.12 3666.24609375
1.13 3666.24609375
1.14 3666.24609375
1.15 3666.24609375
1.16 3666.24609375
1.17 3666.24609375
1.18 3666.24609375
1.19 3666.24609375
1.2 3666.24609375
1.21 3666.24609375
1.22 3666.24609375
1.23 3666.24609375
1.24 3666.24609375
1.25 3666.24609375
1.26 3666.24609375
1.27 3666.24609375
1.28 3666.24609375
1.29 3666.24609375
1.3 3666.24609375
1.31 3666.24609375
1.32 3666.24609375
1.33 3666.24609375
1.34 3666.24609375
1.35 3666.24609375
1.36 3666.24609375
1.37 3666.24609375
1.38 3666.24609375
1.39 3174.01953125
1.4 2643.66015625
1.41 2151.18359375
1.42 1736.48828125
1.43 1302.953125
1.44 886.2421875
1.45 541.16796875
1.46 555.19921875
1.47 557.37109375
1.48 568.93359375
1.49 589.46875
1.5 602.875
1.51 587.21484375
1.52 590.3515625
1.53 590.3515625
};
\addplot [, color0, dashed]
table {%
0 3725.10703125
1.6 3725.10703125
};
\addlegendentry{expensive: $3725 \pm 9$ MB}
\addplot [, color1, dashed]
table {%
0 3670.783984375
1.6 3670.783984375
};
\addlegendentry{optimized: $3671 \pm 3$ MB}
\addplot [, color2, dashed]
table {%
0 562.89609375
1.6 562.89609375
};
\addlegendentry{baseline: $563 \pm 12$ MB}
\end{axis}

\end{tikzpicture}

    \tikzexternaldisable
    \label{cockpit::fig:memory-benchmark-fmnist}
  \end{subfigure}
  \hfill
  \begin{subfigure}[t]{0.99\textwidth}
    \pgfkeys{/pgfplots/zmystyle/.style={ memorybenchmarkdefault, xlabel = {},
        legend pos = north west}}
    \caption{\mnist \mlp}
    \tikzexternalenable
    % This file was created by tikzplotlib v0.9.8.
\begin{tikzpicture}

\definecolor{color0}{rgb}{0.894117647058824,0.101960784313725,0.109803921568627}
\definecolor{color1}{rgb}{0.301960784313725,0.686274509803922,0.290196078431373}
\definecolor{color2}{rgb}{0.215686274509804,0.494117647058824,0.72156862745098}

\begin{axis}[
axis line style={white!80!black},
legend style={
  fill opacity=0.8,
  draw opacity=1,
  text opacity=1,
  at={(0.97,0.03)},
  anchor=south east,
  draw=white!80!black
},
tick pos=left,
xlabel={Time [s]},
xmin=-0.0375, xmax=0.7875,
xtick style={color=gray},
ylabel={Memory [MB]},
ymin=215.406523127481, ymax=1529.90051432289,
zmystyle
]
\path [draw=color0, fill=color0, opacity=0.4]
(axis cs:0,1470.15078745037)
--(axis cs:0,1468.57577504963)
--(axis cs:0.75,1468.57577504963)
--(axis cs:0.75,1470.15078745037)
--(axis cs:0.75,1470.15078745037)
--(axis cs:0,1470.15078745037)
--cycle;

\path [draw=color1, fill=color1, opacity=0.4]
(axis cs:0,1202.00730407789)
--(axis cs:0,1200.78332092211)
--(axis cs:0.75,1200.78332092211)
--(axis cs:0.75,1202.00730407789)
--(axis cs:0.75,1202.00730407789)
--(axis cs:0,1202.00730407789)
--cycle;

\path [draw=color2, fill=color2, opacity=0.4]
(axis cs:0,436.04903634219)
--(axis cs:0,433.92361990781)
--(axis cs:0.75,433.92361990781)
--(axis cs:0.75,436.04903634219)
--(axis cs:0.75,436.04903634219)
--(axis cs:0,436.04903634219)
--cycle;

\addplot [, color0, dotted, forget plot]
table {%
0 275.484375
0.01 275.69140625
0.02 298.61328125
0.03 334.296875
0.04 368.26953125
0.05 370.33203125
0.06 374.4765625
0.07 386.49609375
0.08 406.99609375
0.09 411.82421875
0.1 392.609375
0.11 403.94140625
0.12 391.046875
0.13 395
0.14 402.734375
0.15 415.625
0.16 425.6796875
0.17 427.99609375
0.18 428.375
0.19 428.375
0.2 428.375
0.21 428.375
0.22 428.375
0.23 429.08984375
0.24 480.5703125
0.25 674.703125
0.26 854.3984375
0.27 1051.3671875
0.28 1192.14453125
0.29 1326.46484375
0.3 1460.52734375
0.31 1469.55078125
0.32 1469.55078125
0.33 1469.55078125
0.34 1469.55078125
0.35 1469.55078125
0.36 1469.55078125
0.37 1469.55078125
0.38 1469.55078125
0.39 1469.55078125
0.4 1469.55078125
0.41 1469.55078125
0.42 1469.55078125
0.43 1469.55078125
0.44 1469.55078125
0.45 1469.55078125
0.46 1469.55078125
0.47 1469.55078125
0.48 1469.55078125
0.49 1469.55078125
0.5 1469.55078125
0.51 1469.55078125
0.52 1356.65625
0.53 935.95703125
0.54 752.10546875
0.55 886.42578125
0.56 948.81640625
0.57 948.81640625
0.58 948.81640625
0.59 948.81640625
0.6 948.81640625
0.61 948.81640625
0.62 948.81640625
0.63 948.81640625
0.64 948.81640625
0.65 948.81640625
0.66 948.81640625
0.67 948.81640625
0.68 948.81640625
0.69 948.81640625
0.7 667.00390625
0.71 485.23046875
0.72 485.23046875
0.73 436.68359375
};
\addplot [, color0, dotted, forget plot]
table {%
0 275.921875
0.01 276.0546875
0.02 299.09765625
0.03 334.5546875
0.04 367.8125
0.05 370.64453125
0.06 370.64453125
0.07 371.13671875
0.08 385.6875
0.09 412.5234375
0.1 411.6328125
0.11 393.046875
0.12 404.37890625
0.13 391.41796875
0.14 394.546875
0.15 402.0234375
0.16 414.65625
0.17 426
0.18 428.05859375
0.19 428.7578125
0.2 428.7578125
0.21 428.7578125
0.22 428.7578125
0.23 428.7578125
0.24 429.5234375
0.25 459.15625
0.26 647.875
0.27 826.28125
0.28 1023.25
0.29 1170.87109375
0.3 1304.93359375
0.31 1439.76953125
0.32 1469.67578125
0.33 1469.67578125
0.34 1469.67578125
0.35 1469.67578125
0.36 1469.67578125
0.37 1469.67578125
0.38 1469.67578125
0.39 1469.67578125
0.4 1469.67578125
0.41 1469.67578125
0.42 1469.67578125
0.43 1469.67578125
0.44 1469.67578125
0.45 1469.67578125
0.46 1469.67578125
0.47 1469.67578125
0.48 1469.67578125
0.49 1469.67578125
0.5 1469.67578125
0.51 1469.67578125
0.52 1469.67578125
0.53 1469.67578125
0.54 1026.69921875
0.55 717.73046875
0.56 848.44140625
0.57 948.98828125
0.58 948.98828125
0.59 948.98828125
0.6 948.98828125
0.61 948.98828125
0.62 948.98828125
0.63 948.98828125
0.64 948.98828125
0.65 948.98828125
0.66 948.98828125
0.67 948.98828125
0.68 948.98828125
0.69 948.98828125
0.7 948.98828125
0.71 759.7890625
0.72 469.67578125
0.73 485.40234375
0.74 436.8515625
};
\addplot [, color0, dotted, forget plot]
table {%
0 275.76171875
0.01 275.96875
0.02 298.89453125
0.03 334.4140625
0.04 368.67578125
0.05 370.96484375
0.06 374.88671875
0.07 386.7890625
0.08 413.13671875
0.09 411.765625
0.1 392.6796875
0.11 404.015625
0.12 391.046875
0.13 394.59765625
0.14 402.33203125
0.15 415.73828125
0.16 425.53515625
0.17 427.8515625
0.18 428.203125
0.19 428.203125
0.2 428.203125
0.21 428.203125
0.22 428.203125
0.23 428.9765625
0.24 458.5
0.25 637.1640625
0.26 744.671875
0.27 926.171875
0.28 1066.1640625
0.29 1162.7890625
0.3 1293.7578125
0.31 1424.984375
0.32 1469.0703125
0.33 1469.0703125
0.34 1469.0703125
0.35 1469.0703125
0.36 1469.0703125
0.37 1469.0703125
0.38 1469.0703125
0.39 1469.0703125
0.4 1469.0703125
0.41 1469.0703125
0.42 1469.0703125
0.43 1469.0703125
0.44 1469.0703125
0.45 1469.0703125
0.46 1469.0703125
0.47 1469.0703125
0.48 1469.0703125
0.49 1469.0703125
0.5 1469.0703125
0.51 1469.0703125
0.52 1469.0703125
0.53 1469.0703125
0.54 1170.0703125
0.55 783.96875
0.56 802.1796875
0.57 926.9609375
0.58 948.359375
0.59 948.359375
0.6 948.359375
0.61 948.359375
0.62 948.359375
0.63 948.359375
0.64 948.359375
0.65 948.359375
0.66 948.359375
0.67 948.359375
0.68 948.359375
0.69 948.359375
0.7 948.359375
0.71 936.28515625
0.72 580.28515625
0.73 484.7734375
0.74 464.140625
0.75 436.16796875
};
\addplot [, color0, dotted, forget plot]
table {%
0 276.10546875
0.01 276.32421875
0.02 297.87109375
0.03 333.76171875
0.04 369.59375
0.05 371.140625
0.06 378.62890625
0.07 390.16796875
0.08 407.7734375
0.09 412.79296875
0.1 394.33984375
0.11 404.65234375
0.12 391.703125
0.13 395.85546875
0.14 403.58984375
0.15 417.25390625
0.16 426.27734375
0.17 428.8515625
0.18 428.96875
0.19 428.96875
0.2 428.96875
0.21 428.96875
0.22 428.96875
0.23 429.60546875
0.24 490.69921875
0.25 685.60546875
0.26 865.81640625
0.27 1063.81640625
0.28 1198.8515625
0.29 1333.6875
0.3 1467.234375
0.31 1469.609375
0.32 1469.609375
0.33 1469.609375
0.34 1469.609375
0.35 1469.609375
0.36 1469.609375
0.37 1469.609375
0.38 1469.609375
0.39 1469.609375
0.4 1469.609375
0.41 1469.609375
0.42 1469.609375
0.43 1469.609375
0.44 1469.609375
0.45 1469.609375
0.46 1469.609375
0.47 1469.609375
0.48 1469.609375
0.49 1469.609375
0.5 1469.609375
0.51 1469.609375
0.52 1356.51171875
0.53 935.8125
0.54 747.578125
0.55 881.8984375
0.56 948.671875
0.57 948.671875
0.58 948.671875
0.59 948.671875
0.6 948.671875
0.61 948.671875
0.62 948.671875
0.63 948.671875
0.64 948.671875
0.65 948.671875
0.66 948.671875
0.67 948.671875
0.68 948.671875
0.69 948.671875
0.7 702.59765625
0.71 485.0859375
0.72 485.0859375
0.73 436.4765625
};
\addplot [, color0, dotted, forget plot]
table {%
0 275.4296875
0.01 275.63671875
0.02 299.09375
0.03 333.8828125
0.04 366.109375
0.05 370.40625
0.06 370.40625
0.07 381.5234375
0.08 393.671875
0.09 409.90234375
0.1 394.15234375
0.11 403.69921875
0.12 390.57421875
0.13 391.3828125
0.14 398.51171875
0.15 408.30859375
0.16 421.71484375
0.17 425.83984375
0.18 428.015625
0.19 428.015625
0.2 428.015625
0.21 428.015625
0.22 428.015625
0.23 428.3046875
0.24 430.7421875
0.25 570.54296875
0.26 747.40234375
0.27 945.40234375
0.28 1117.8203125
0.29 1252.140625
0.3 1386.203125
0.31 1468.9609375
0.32 1468.9609375
0.33 1468.9609375
0.34 1468.9609375
0.35 1468.9609375
0.36 1468.9609375
0.37 1468.9609375
0.38 1468.9609375
0.39 1468.9609375
0.4 1468.9609375
0.41 1468.9609375
0.42 1468.9609375
0.43 1468.9609375
0.44 1468.9609375
0.45 1468.9609375
0.46 1468.9609375
0.47 1468.9609375
0.48 1468.9609375
0.49 1468.9609375
0.5 1468.9609375
0.51 1468.9609375
0.52 1468.9609375
0.53 1204.46875
0.54 807.87109375
0.55 795.01953125
0.56 928.56640625
0.57 948.16015625
0.58 948.16015625
0.59 948.16015625
0.6 948.16015625
0.61 948.16015625
0.62 948.16015625
0.63 948.16015625
0.64 948.16015625
0.65 948.16015625
0.66 948.16015625
0.67 948.16015625
0.68 948.16015625
0.69 948.16015625
0.7 890.0859375
0.71 552.69921875
0.72 484.57421875
0.73 435.94140625
0.74 435.96484375
};
\addplot [, color0, dotted, forget plot]
table {%
0 276.04296875
0.01 276.24609375
0.02 299.80078125
0.03 335.171875
0.04 367.65625
0.05 371.00390625
0.06 371.00390625
0.07 383.97265625
0.08 401.33984375
0.09 411.703125
0.1 393.25
0.11 404.58203125
0.12 391.6171875
0.13 393.3671875
0.14 401.1015625
0.15 412.4453125
0.16 425.8515625
0.17 427.39453125
0.18 428.89453125
0.19 428.89453125
0.2 428.89453125
0.21 428.89453125
0.22 428.89453125
0.23 429.44921875
0.24 433.26953125
0.25 610.46875
0.26 792.484375
0.27 989.7109375
0.28 1149.23828125
0.29 1284.33203125
0.3 1418.65234375
0.31 1469.44140625
0.32 1469.44140625
0.33 1469.44140625
0.34 1469.44140625
0.35 1469.44140625
0.36 1469.44140625
0.37 1469.44140625
0.38 1469.44140625
0.39 1469.44140625
0.4 1469.44140625
0.41 1469.44140625
0.42 1469.44140625
0.43 1469.44140625
0.44 1469.44140625
0.45 1469.44140625
0.46 1469.44140625
0.47 1469.44140625
0.48 1469.44140625
0.49 1469.44140625
0.5 1469.44140625
0.51 1469.44140625
0.52 1469.44140625
0.53 1091.28515625
0.54 708.46875
0.55 831.0625
0.56 948.625
0.57 948.625
0.58 948.625
0.59 948.625
0.6 948.625
0.61 948.625
0.62 948.625
0.63 948.625
0.64 948.625
0.65 948.625
0.66 948.625
0.67 948.625
0.68 948.625
0.69 948.625
0.7 797.30859375
0.71 460.55078125
0.72 485.0390625
0.73 436.4296875
};
\addplot [, color0, dotted, forget plot]
table {%
0 275.296875
0.01 275.49609375
0.02 297.46484375
0.03 333.109375
0.04 368.15625
0.05 370.21875
0.06 377.18359375
0.07 389.66796875
0.08 406.91015625
0.09 406.65234375
0.1 395.54296875
0.11 404.56640625
0.12 391.15234375
0.13 395.0859375
0.14 402.5625
0.15 415.70703125
0.16 425.50390625
0.17 427.5625
0.18 428.03125
0.19 428.03125
0.2 428.03125
0.21 428.03125
0.22 428.03125
0.23 428.5625
0.24 482.9453125
0.25 677.59375
0.26 857.03125
0.27 1055.2890625
0.28 1193.12109375
0.29 1327.69921875
0.3 1462.53515625
0.31 1468.72265625
0.32 1468.72265625
0.33 1468.72265625
0.34 1468.72265625
0.35 1468.72265625
0.36 1468.72265625
0.37 1468.72265625
0.38 1468.72265625
0.39 1468.72265625
0.4 1468.72265625
0.41 1468.72265625
0.42 1468.72265625
0.43 1468.72265625
0.44 1468.72265625
0.45 1468.72265625
0.46 1468.72265625
0.47 1468.72265625
0.48 1468.72265625
0.49 1468.72265625
0.5 1468.72265625
0.51 1468.72265625
0.52 1393.671875
0.53 972.97265625
0.54 738.86328125
0.55 873.44140625
0.56 947.94921875
0.57 947.94921875
0.58 947.94921875
0.59 947.94921875
0.6 947.94921875
0.61 947.94921875
0.62 947.94921875
0.63 947.94921875
0.64 947.94921875
0.65 947.94921875
0.66 947.94921875
0.67 947.94921875
0.68 947.94921875
0.69 947.94921875
0.7 689.875
0.71 484.36328125
0.72 484.36328125
0.73 435.765625
};
\addplot [, color0, dotted, forget plot]
table {%
0 275.43359375
0.01 275.58984375
0.02 297.7421875
0.03 332.7734375
0.04 367.5703125
0.05 370.09375
0.06 375.25390625
0.07 388.5625
0.08 406.72265625
0.09 411.80859375
0.1 393.35546875
0.11 403.66796875
0.12 390.75390625
0.13 394.734375
0.14 402.2109375
0.15 415.359375
0.16 425.4140625
0.17 427.47265625
0.18 427.96875
0.19 427.96875
0.2 427.96875
0.21 427.96875
0.22 427.96875
0.23 429.046875
0.24 476.1015625
0.25 670.4921875
0.26 849.9296875
0.27 1047.15625
0.28 1189.17578125
0.29 1322.98046875
0.3 1456.78515625
0.31 1468.90234375
0.32 1468.90234375
0.33 1468.90234375
0.34 1468.90234375
0.35 1468.90234375
0.36 1468.90234375
0.37 1468.90234375
0.38 1468.90234375
0.39 1468.90234375
0.4 1468.90234375
0.41 1468.90234375
0.42 1468.90234375
0.43 1468.90234375
0.44 1468.90234375
0.45 1468.90234375
0.46 1468.90234375
0.47 1468.90234375
0.48 1468.90234375
0.49 1468.90234375
0.5 1468.90234375
0.51 1468.90234375
0.52 1393.859375
0.53 973.16015625
0.54 733.63671875
0.55 866.92578125
0.56 948.13671875
0.57 948.13671875
0.58 948.13671875
0.59 948.13671875
0.6 948.13671875
0.61 948.13671875
0.62 948.13671875
0.63 948.13671875
0.64 948.13671875
0.65 948.13671875
0.66 948.13671875
0.67 948.13671875
0.68 948.13671875
0.69 948.13671875
0.7 721.0546875
0.71 476.30078125
0.72 484.55078125
0.73 436.00390625
};
\addplot [, color0, dotted, forget plot]
table {%
0 275.75390625
0.01 275.953125
0.02 298.5
0.03 334.2890625
0.04 370.109375
0.05 370.625
0.06 376.5703125
0.07 389.9765625
0.08 407.546875
0.09 412.375
0.1 393.3984375
0.11 404.484375
0.12 391.609375
0.13 395.96484375
0.14 403.69921875
0.15 417.10546875
0.16 426.38671875
0.17 428.703125
0.18 428.91796875
0.19 428.91796875
0.2 428.91796875
0.21 428.91796875
0.22 428.91796875
0.23 429.5625
0.24 487.87109375
0.25 681.74609375
0.26 861.18359375
0.27 1058.92578125
0.28 1178.203125
0.29 1285.7109375
0.3 1420.8046875
0.31 1469.7890625
0.32 1469.7890625
0.33 1469.7890625
0.34 1469.7890625
0.35 1469.7890625
0.36 1469.7890625
0.37 1469.7890625
0.38 1469.7890625
0.39 1469.7890625
0.4 1469.7890625
0.41 1469.7890625
0.42 1469.7890625
0.43 1469.7890625
0.44 1469.7890625
0.45 1469.7890625
0.46 1469.7890625
0.47 1469.7890625
0.48 1469.7890625
0.49 1469.7890625
0.5 1469.7890625
0.51 1469.7890625
0.52 1469.7890625
0.53 1167.3671875
0.54 780.65234375
0.55 806.88671875
0.56 941.20703125
0.57 948.94140625
0.58 948.94140625
0.59 948.94140625
0.6 948.94140625
0.61 948.94140625
0.62 948.94140625
0.63 948.94140625
0.64 948.94140625
0.65 948.94140625
0.66 948.94140625
0.67 948.94140625
0.68 948.94140625
0.69 948.94140625
0.7 873.390625
0.71 515.59765625
0.72 485.35546875
0.73 436.72265625
0.74 436.74609375
};
\addplot [, color0, dotted, forget plot]
table {%
0 276.15234375
0.01 276.359375
0.02 299.6953125
0.03 335.34375
0.04 369.1015625
0.05 371.1640625
0.06 378.41796875
0.07 390.66015625
0.08 407.81640625
0.09 407.06640625
0.1 394.8984375
0.11 404.6953125
0.12 392.12109375
0.13 396.078125
0.14 404.0703125
0.15 417.734375
0.16 426.2421875
0.17 429.00390625
0.18 429.00390625
0.19 429.00390625
0.2 429.00390625
0.21 429.00390625
0.22 429.00390625
0.23 429.640625
0.24 477.37109375
0.25 656.80859375
0.26 765.86328125
0.27 947.36328125
0.28 1074.20703125
0.29 1183.203125
0.3 1313.65625
0.31 1437.1484375
0.32 1469.91015625
0.33 1469.91015625
0.34 1469.91015625
0.35 1469.91015625
0.36 1469.91015625
0.37 1469.91015625
0.38 1469.91015625
0.39 1469.91015625
0.4 1469.91015625
0.41 1469.91015625
0.42 1469.91015625
0.43 1469.91015625
0.44 1469.91015625
0.45 1469.91015625
0.46 1469.91015625
0.47 1469.91015625
0.48 1469.91015625
0.49 1469.91015625
0.5 1469.91015625
0.51 1469.91015625
0.52 1469.91015625
0.53 1469.91015625
0.54 1175.77734375
0.55 810.01953125
0.56 789.9140625
0.57 920.3671875
0.58 948.7265625
0.59 948.7265625
0.6 948.7265625
0.61 948.7265625
0.62 948.7265625
0.63 948.7265625
0.64 948.7265625
0.65 948.7265625
0.66 948.7265625
0.67 948.7265625
0.68 948.7265625
0.69 948.7265625
0.7 948.7265625
0.71 948.7265625
0.72 629.03125
0.73 485.140625
0.74 485.140625
0.75 436.53125
};
\addplot [, color1, dotted, forget plot]
table {%
0 275.7734375
0.01 275.9765625
0.02 298.93359375
0.03 335.0234375
0.04 368.99609375
0.05 370.80078125
0.06 376.96875
0.07 387.36328125
0.08 407.19140625
0.09 406.44140625
0.1 394.7421875
0.11 404.0234375
0.12 391.44921875
0.13 395.640625
0.14 403.890625
0.15 417.5546875
0.16 425.546875
0.17 428.2109375
0.18 428.2109375
0.19 428.2109375
0.2 428.2109375
0.21 428.2109375
0.22 428.2109375
0.23 429.0546875
0.24 483.96484375
0.25 483.96484375
0.26 523.53515625
0.27 702.45703125
0.28 836.77734375
0.29 923.66015625
0.3 923.66015625
0.31 923.66015625
0.32 923.66015625
0.33 923.66015625
0.34 923.66015625
0.35 923.66015625
0.36 923.66015625
0.37 923.66015625
0.38 923.66015625
0.39 923.66015625
0.4 923.66015625
0.41 923.66015625
0.42 923.66015625
0.43 679.68359375
0.44 463.89453125
0.45 659.83203125
0.46 840.81640625
0.47 974.87890625
0.48 1109.19921875
0.49 1201.23828125
0.5 1201.23828125
0.51 1201.23828125
0.52 1201.23828125
0.53 1201.23828125
0.54 1201.23828125
0.55 1201.23828125
0.56 1201.23828125
0.57 1201.23828125
0.58 1201.23828125
0.59 1201.23828125
0.6 1201.23828125
0.61 1201.23828125
0.62 1201.23828125
0.63 1201.23828125
0.64 1201.23828125
0.65 1201.23828125
0.66 1201.23828125
0.67 1201.23828125
0.68 1201.23828125
0.69 1201.23828125
0.7 1201.23828125
0.71 974.1875
0.72 591.37109375
0.73 436.1796875
};
\addplot [, color1, dotted, forget plot]
table {%
0 275.15625
0.01 275.3515625
0.02 297.83984375
0.03 332.265625
0.04 368.34765625
0.05 370.15234375
0.06 375.30078125
0.07 387.83203125
0.08 406.7421875
0.09 411.82421875
0.1 393.05859375
0.11 403.62890625
0.12 390.99609375
0.13 395.22265625
0.14 402.69921875
0.15 416.10546875
0.16 425.12890625
0.17 427.703125
0.18 427.875
0.19 427.875
0.2 427.875
0.21 427.875
0.22 427.875
0.23 428.58984375
0.24 483.4453125
0.25 483.4453125
0.26 496.44921875
0.27 683.36328125
0.28 816.91015625
0.29 923.12890625
0.3 923.12890625
0.31 923.12890625
0.32 923.12890625
0.33 923.12890625
0.34 923.12890625
0.35 923.12890625
0.36 923.12890625
0.37 923.12890625
0.38 923.12890625
0.39 923.12890625
0.4 923.12890625
0.41 923.12890625
0.42 923.12890625
0.43 733.8828125
0.44 435.265625
0.45 620.37109375
0.46 815.79296875
0.47 944.95703125
0.48 1078.50390625
0.49 1200.70703125
0.5 1200.70703125
0.51 1200.70703125
0.52 1200.70703125
0.53 1200.70703125
0.54 1200.70703125
0.55 1200.70703125
0.56 1200.70703125
0.57 1200.70703125
0.58 1200.70703125
0.59 1200.70703125
0.6 1200.70703125
0.61 1200.70703125
0.62 1200.70703125
0.63 1200.70703125
0.64 1200.70703125
0.65 1200.70703125
0.66 1200.70703125
0.67 1200.70703125
0.68 1200.70703125
0.69 1200.70703125
0.7 1200.70703125
0.71 1087.3046875
0.72 704.48828125
0.73 435.59765625
0.74 435.59765625
};
\addplot [, color1, dotted, forget plot]
table {%
0 275.9296875
0.01 276.05859375
0.02 299.62890625
0.03 335.4140625
0.04 368.66015625
0.05 370.72265625
0.06 370.72265625
0.07 378.796875
0.08 389.46484375
0.09 409.28125
0.1 409.60546875
0.11 399.88671875
0.12 407.10546875
0.13 392.0078125
0.14 397.328125
0.15 406.34765625
0.16 419.75390625
0.17 425.94140625
0.18 428.63671875
0.19 428.63671875
0.2 428.63671875
0.21 428.63671875
0.22 428.63671875
0.23 428.63671875
0.24 429.15234375
0.25 484.54296875
0.26 484.54296875
0.27 536.16796875
0.28 709.16015625
0.29 836.00390625
0.3 924.17578125
0.31 924.17578125
0.32 924.17578125
0.33 924.17578125
0.34 924.17578125
0.35 924.17578125
0.36 924.17578125
0.37 924.17578125
0.38 924.17578125
0.39 924.17578125
0.4 924.17578125
0.41 924.17578125
0.42 924.17578125
0.43 924.17578125
0.44 810.6953125
0.45 466.0546875
0.46 583.77734375
0.47 780.23046875
0.48 919.96484375
0.49 1052.73828125
0.5 1186.54296875
0.51 1201.75390625
0.52 1201.75390625
0.53 1201.75390625
0.54 1201.75390625
0.55 1201.75390625
0.56 1201.75390625
0.57 1201.75390625
0.58 1201.75390625
0.59 1201.75390625
0.6 1201.75390625
0.61 1201.75390625
0.62 1201.75390625
0.63 1201.75390625
0.64 1201.75390625
0.65 1201.75390625
0.66 1201.75390625
0.67 1201.75390625
0.68 1201.75390625
0.69 1201.75390625
0.7 1201.75390625
0.71 1201.75390625
0.72 1188.3671875
0.73 781.30078125
0.74 440.35546875
0.75 436.6953125
};
\addplot [, color1, dotted, forget plot]
table {%
0 275.890625
0.01 276.08984375
0.02 299.65625
0.03 335.4375
0.04 368.9453125
0.05 371.0078125
0.06 371.0078125
0.07 378.96484375
0.08 389.0390625
0.09 409.4140625
0.1 408.44921875
0.11 397.90234375
0.12 406.15234375
0.13 391.96484375
0.14 396.921875
0.15 405.4296875
0.16 419.09375
0.17 426.0546875
0.18 428.59375
0.19 428.59375
0.2 428.59375
0.21 428.59375
0.22 428.59375
0.23 428.59375
0.24 429.140625
0.25 484.40234375
0.26 484.40234375
0.27 535.75
0.28 711.0625
0.29 844.3515625
0.3 924.2734375
0.31 924.2734375
0.32 924.2734375
0.33 924.2734375
0.34 924.2734375
0.35 924.2734375
0.36 924.2734375
0.37 924.2734375
0.38 924.2734375
0.39 924.2734375
0.4 924.2734375
0.41 924.2734375
0.42 924.2734375
0.43 924.2734375
0.44 676.15234375
0.45 468.375
0.46 663.0234375
0.47 842.71875
0.48 976.5234375
0.49 1107.75
0.5 1201.59375
0.51 1201.59375
0.52 1201.59375
0.53 1201.59375
0.54 1201.59375
0.55 1201.59375
0.56 1201.59375
0.57 1201.59375
0.58 1201.59375
0.59 1201.59375
0.6 1201.59375
0.61 1201.59375
0.62 1201.59375
0.63 1201.59375
0.64 1201.59375
0.65 1201.59375
0.66 1201.59375
0.67 1201.59375
0.68 1201.59375
0.69 1201.59375
0.7 1201.59375
0.71 1201.59375
0.72 1002.20703125
0.73 629.609375
0.74 436.4765625
};
\addplot [, color1, dotted, forget plot]
table {%
0 275.890625
0.01 276.01953125
0.02 296.9765625
0.03 333.01171875
0.04 368.328125
0.05 370.6484375
0.06 374.73046875
0.07 387.65234375
0.08 407.23828125
0.09 411.80859375
0.1 392.88671875
0.11 404.22265625
0.12 391.36328125
0.13 395.703125
0.14 403.43359375
0.15 416.83984375
0.16 425.86328125
0.17 428.4375
0.18 428.59375
0.19 428.59375
0.2 428.59375
0.21 428.59375
0.22 428.59375
0.23 429.79296875
0.24 484.28515625
0.25 484.54296875
0.26 511.46484375
0.27 693.99609375
0.28 828.05859375
0.29 923.96484375
0.3 923.96484375
0.31 923.96484375
0.32 923.96484375
0.33 923.96484375
0.34 923.96484375
0.35 923.96484375
0.36 923.96484375
0.37 923.96484375
0.38 923.96484375
0.39 923.96484375
0.4 923.96484375
0.41 923.96484375
0.42 923.96484375
0.43 734.71875
0.44 436.359375
0.45 631.26171875
0.46 819.20703125
0.47 952.23828125
0.48 1086.04296875
0.49 1201.80078125
0.5 1201.80078125
0.51 1201.80078125
0.52 1201.80078125
0.53 1201.80078125
0.54 1201.80078125
0.55 1201.80078125
0.56 1201.80078125
0.57 1201.80078125
0.58 1201.80078125
0.59 1201.80078125
0.6 1201.80078125
0.61 1201.80078125
0.62 1201.80078125
0.63 1201.80078125
0.64 1201.80078125
0.65 1201.80078125
0.66 1201.80078125
0.67 1201.80078125
0.68 1201.80078125
0.69 1201.80078125
0.7 1201.80078125
0.71 1074.4140625
0.72 674.4140625
0.73 436.75
};
\addplot [, color1, dotted, forget plot]
table {%
0 275.7890625
0.01 275.98828125
0.02 298.78515625
0.03 335.14453125
0.04 367.62890625
0.05 370.9140625
0.06 370.9140625
0.07 371.4609375
0.08 385.86328125
0.09 412.92578125
0.1 411.90625
0.11 392.62109375
0.12 403.95703125
0.13 391.0078125
0.14 394.9765625
0.15 402.453125
0.16 415.6015625
0.17 425.3984375
0.18 427.71484375
0.19 428.14453125
0.2 428.14453125
0.21 428.14453125
0.22 428.14453125
0.23 428.14453125
0.24 429.109375
0.25 476.51953125
0.26 483.99609375
0.27 501.421875
0.28 685.7578125
0.29 819.3046875
0.3 923.71875
0.31 923.71875
0.32 923.71875
0.33 923.71875
0.34 923.71875
0.35 923.71875
0.36 923.71875
0.37 923.71875
0.38 923.71875
0.39 923.71875
0.4 923.71875
0.41 923.71875
0.42 923.71875
0.43 923.71875
0.44 734.47265625
0.45 435.85546875
0.46 624.828125
0.47 817.9296875
0.48 949.4140625
0.49 1079.3515625
0.5 1201.296875
0.51 1201.296875
0.52 1201.296875
0.53 1201.296875
0.54 1201.296875
0.55 1201.296875
0.56 1201.296875
0.57 1201.296875
0.58 1201.296875
0.59 1201.296875
0.6 1201.296875
0.61 1201.296875
0.62 1201.296875
0.63 1201.296875
0.64 1201.296875
0.65 1201.296875
0.66 1201.296875
0.67 1201.296875
0.68 1201.296875
0.69 1201.296875
0.7 1201.296875
0.71 1201.296875
0.72 1087.89453125
0.73 705.078125
0.74 436.24609375
0.75 436.24609375
};
\addplot [, color1, dotted, forget plot]
table {%
0 275.75
0.01 275.94921875
0.02 299.1875
0.03 335.54296875
0.04 368.2734375
0.05 370.78515625
0.06 370.78515625
0.07 378.90234375
0.08 390.26953125
0.09 407.1640625
0.1 408.7109375
0.11 398.5703125
0.12 406.5625
0.13 391.53125
0.14 396.828125
0.15 405.8515625
0.16 419.2578125
0.17 425.4453125
0.18 428.16015625
0.19 428.16015625
0.2 428.16015625
0.21 428.16015625
0.22 428.16015625
0.23 428.16015625
0.24 428.9296875
0.25 484.08984375
0.26 484.08984375
0.27 546.9609375
0.28 718.1484375
0.29 850.921875
0.3 923.8828125
0.31 923.8828125
0.32 923.8828125
0.33 923.8828125
0.34 923.8828125
0.35 923.8828125
0.36 923.8828125
0.37 923.8828125
0.38 923.8828125
0.39 923.8828125
0.4 923.8828125
0.41 923.8828125
0.42 923.8828125
0.43 923.8828125
0.44 649.76171875
0.45 473.3984375
0.46 668.3046875
0.47 845.421875
0.48 979.2265625
0.49 1113.546875
0.5 1201.203125
0.51 1201.203125
0.52 1201.203125
0.53 1201.203125
0.54 1201.203125
0.55 1201.203125
0.56 1201.203125
0.57 1201.203125
0.58 1201.203125
0.59 1201.203125
0.6 1201.203125
0.61 1201.203125
0.62 1201.203125
0.63 1201.203125
0.64 1201.203125
0.65 1201.203125
0.66 1201.203125
0.67 1201.203125
0.68 1201.203125
0.69 1201.203125
0.7 1201.203125
0.71 1201.203125
0.72 974.15234375
0.73 591.3359375
0.74 436.140625
};
\addplot [, color1, dotted, forget plot]
table {%
0 275.375
0.01 275.578125
0.02 296.98828125
0.03 332.98828125
0.04 368.74609375
0.05 370.29296875
0.06 376.4453125
0.07 386.16796875
0.08 406.88671875
0.09 411.45703125
0.1 392.49609375
0.11 403.83203125
0.12 391.04296875
0.13 395.21484375
0.14 402.69140625
0.15 416.09375
0.16 425.6328125
0.17 427.94921875
0.18 428.265625
0.19 428.265625
0.2 428.265625
0.21 428.265625
0.22 428.265625
0.23 428.81640625
0.24 476.8203125
0.25 484.296875
0.26 501.65234375
0.27 687.79296875
0.28 821.85546875
0.29 924.20703125
0.3 924.20703125
0.31 924.20703125
0.32 924.20703125
0.33 924.20703125
0.34 924.20703125
0.35 924.20703125
0.36 924.20703125
0.37 924.20703125
0.38 924.20703125
0.39 924.20703125
0.4 924.20703125
0.41 924.20703125
0.42 924.20703125
0.43 734.9609375
0.44 436.0859375
0.45 628.66796875
0.46 819.70703125
0.47 952.22265625
0.48 1086.28515625
0.49 1201.52734375
0.5 1201.52734375
0.51 1201.52734375
0.52 1201.52734375
0.53 1201.52734375
0.54 1201.52734375
0.55 1201.52734375
0.56 1201.52734375
0.57 1201.52734375
0.58 1201.52734375
0.59 1201.52734375
0.6 1201.52734375
0.61 1201.52734375
0.62 1201.52734375
0.63 1201.52734375
0.64 1201.52734375
0.65 1201.52734375
0.66 1201.52734375
0.67 1201.52734375
0.68 1201.52734375
0.69 1201.52734375
0.7 1201.52734375
0.71 1012.359375
0.72 629.54296875
0.73 436.6015625
};
\addplot [, color1, dotted, forget plot]
table {%
0 275.40234375
0.01 275.6015625
0.02 297.515625
0.03 333.76953125
0.04 368.25
0.05 370.3125
0.06 376.234375
0.07 388.14453125
0.08 406.8984375
0.09 411.7265625
0.1 393.51953125
0.11 403.83203125
0.12 391.25
0.13 395.83984375
0.14 403.83203125
0.15 417.49609375
0.16 425.74609375
0.17 428.265625
0.18 428.265625
0.19 428.265625
0.2 428.265625
0.21 428.265625
0.22 428.265625
0.23 429.05078125
0.24 484.0078125
0.25 484.265625
0.26 508.12890625
0.27 692.46484375
0.28 825.49609375
0.29 923.72265625
0.3 923.72265625
0.31 923.72265625
0.32 923.72265625
0.33 923.72265625
0.34 923.72265625
0.35 923.72265625
0.36 923.72265625
0.37 923.72265625
0.38 923.72265625
0.39 923.72265625
0.4 923.72265625
0.41 923.72265625
0.42 923.72265625
0.43 734.4765625
0.44 436.1171875
0.45 631.01953125
0.46 822.57421875
0.47 955.86328125
0.48 1090.18359375
0.49 1201.55859375
0.5 1201.55859375
0.51 1201.55859375
0.52 1201.55859375
0.53 1201.55859375
0.54 1201.55859375
0.55 1201.55859375
0.56 1201.55859375
0.57 1201.55859375
0.58 1201.55859375
0.59 1201.55859375
0.6 1201.55859375
0.61 1201.55859375
0.62 1201.55859375
0.63 1201.55859375
0.64 1201.55859375
0.65 1201.55859375
0.66 1201.55859375
0.67 1201.55859375
0.68 1201.55859375
0.69 1201.55859375
0.7 1201.55859375
0.71 1012.390625
0.72 629.57421875
0.73 436.62890625
};
\addplot [, color1, dotted, forget plot]
table {%
0 275.7734375
0.01 275.97265625
0.02 297.4921875
0.03 334.1015625
0.04 369.61328125
0.05 370.90234375
0.06 378.05078125
0.07 389.82421875
0.08 407.19140625
0.09 407.44921875
0.1 396.53515625
0.11 405.30078125
0.12 391.5625
0.13 396.30078125
0.14 404.8046875
0.15 418.2109375
0.16 425.6875
0.17 428.1953125
0.18 428.1953125
0.19 428.1953125
0.2 428.1953125
0.21 428.1953125
0.22 428.1953125
0.23 428.96484375
0.24 484.17578125
0.25 484.17578125
0.26 530.7890625
0.27 707.390625
0.28 841.453125
0.29 923.6953125
0.3 923.6953125
0.31 923.6953125
0.32 923.6953125
0.33 923.6953125
0.34 923.6953125
0.35 923.6953125
0.36 923.6953125
0.37 923.6953125
0.38 923.6953125
0.39 923.6953125
0.4 923.6953125
0.41 923.6953125
0.42 923.6953125
0.43 679.71875
0.44 458.7734375
0.45 654.1953125
0.46 836.46875
0.47 969.5
0.48 1103.8203125
0.49 1201.2734375
0.5 1201.2734375
0.51 1201.2734375
0.52 1201.2734375
0.53 1201.2734375
0.54 1201.2734375
0.55 1201.2734375
0.56 1201.2734375
0.57 1201.2734375
0.58 1201.2734375
0.59 1201.2734375
0.6 1201.2734375
0.61 1201.2734375
0.62 1201.2734375
0.63 1201.2734375
0.64 1201.2734375
0.65 1201.2734375
0.66 1201.2734375
0.67 1201.2734375
0.68 1201.2734375
0.69 1201.2734375
0.7 1201.2734375
0.71 1012.10546875
0.72 629.2890625
0.73 436.21875
};
\addplot [, color0, dashed]
table {%
0 1469.36328125
0.75 1469.36328125
};
\addlegendentry{expensive: $1469 \pm 0$ MB}
\addplot [, color1, dashed]
table {%
0 1201.3953125
0.75 1201.3953125
};
\addlegendentry{optimized: $1201 \pm 0$ MB}
\addplot [, color2, dashed]
table {%
0 434.986328125
0.75 434.986328125
};
\addlegendentry{baseline: $435 \pm 1$ MB}
\end{axis}

\end{tikzpicture}

    \tikzexternaldisable
    \label{cockpit::fig:memory-benchmark-mnist}
  \end{subfigure}
  \hfill
  \begin{subfigure}[t]{0.99\textwidth}
    \pgfkeys{/pgfplots/zmystyle/.style={ memorybenchmarkdefault, xlabel = {},
        legend pos = north west}}
    \caption{\cifarten \threecthreed}
    \tikzexternalenable
    % This file was created by tikzplotlib v0.9.8.
\begin{tikzpicture}

\definecolor{color0}{rgb}{0.894117647058824,0.101960784313725,0.109803921568627}
\definecolor{color1}{rgb}{0.301960784313725,0.686274509803922,0.290196078431373}
\definecolor{color2}{rgb}{0.215686274509804,0.494117647058824,0.72156862745098}

\begin{axis}[
axis line style={white!80!black},
legend style={
  fill opacity=0.8,
  draw opacity=1,
  text opacity=1,
  at={(0.97,0.03)},
  anchor=south east,
  draw=white!80!black
},
tick pos=left,
xlabel={Time [s]},
xmin=-0.107, xmax=2.247,
xtick style={color=gray},
ylabel={Memory [MB]},
ymin=210.544028216097, ymax=1632.52853246196,
zmystyle
]
\path [draw=color0, fill=color0, opacity=0.4]
(axis cs:0,1567.89287317805)
--(axis cs:0,1461.59931432195)
--(axis cs:2.14,1461.59931432195)
--(axis cs:2.14,1567.89287317805)
--(axis cs:2.14,1567.89287317805)
--(axis cs:0,1567.89287317805)
--cycle;

\path [draw=color1, fill=color1, opacity=0.4]
(axis cs:0,1364.19421421592)
--(axis cs:0,1358.35734828408)
--(axis cs:2.14,1358.35734828408)
--(axis cs:2.14,1364.19421421592)
--(axis cs:2.14,1364.19421421592)
--(axis cs:0,1364.19421421592)
--cycle;

\path [draw=color2, fill=color2, opacity=0.4]
(axis cs:0,849.451090957299)
--(axis cs:0,840.805159042701)
--(axis cs:2.14,840.805159042701)
--(axis cs:2.14,849.451090957299)
--(axis cs:2.14,849.451090957299)
--(axis cs:0,849.451090957299)
--cycle;

\addplot [, color0, dotted, forget plot]
table {%
0 275.5078125
0.01 275.6484375
0.02 277.9921875
0.03 277.9921875
0.04 277.9921875
0.05 277.9921875
0.06 277.9921875
0.07 277.9921875
0.08 277.9921875
0.09 277.9921875
0.1 277.9921875
0.11 277.9921875
0.12 277.9921875
0.13 277.9921875
0.14 277.9921875
0.15 277.9921875
0.16 277.9921875
0.17 277.9921875
0.18 277.9921875
0.19 277.9921875
0.2 277.9921875
0.21 277.9921875
0.22 277.9921875
0.23 277.9921875
0.24 277.9921875
0.25 277.9921875
0.26 277.9921875
0.27 277.9921875
0.28 277.9921875
0.29 277.9921875
0.3 277.9921875
0.31 277.9921875
0.32 277.9921875
0.33 277.9921875
0.34 277.9921875
0.35 277.9921875
0.36 277.9921875
0.37 277.9921875
0.38 277.9921875
0.39 277.9921875
0.4 277.9921875
0.41 277.9921875
0.42 277.9921875
0.43 277.9921875
0.44 277.9921875
0.45 297.73828125
0.46 327.12890625
0.47 322.9140625
0.48 321.2734375
0.49 352.7265625
0.5 365.7421875
0.51 390.23046875
0.52 399.20703125
0.53 424.73046875
0.54 436.84375
0.55 458.6328125
0.56 525.2265625
0.57 593.2265625
0.58 600.4765625
0.59 600.4765625
0.6 600.4765625
0.61 600.4765625
0.62 600.4765625
0.63 600.4765625
0.64 600.4765625
0.65 600.4765625
0.66 600.4765625
0.67 600.4765625
0.68 600.4765625
0.69 600.4765625
0.7 600.4765625
0.71 600.4765625
0.72 600.4765625
0.73 600.4765625
0.74 600.4765625
0.75 600.4765625
0.76 600.4765625
0.77 600.4765625
0.78 600.4765625
0.79 600.4765625
0.8 600.4765625
0.81 600.4765625
0.82 600.4765625
0.83 600.4765625
0.84 600.4765625
0.85 600.4765625
0.86 600.4765625
0.87 600.4765625
0.88 600.4765625
0.89 600.4765625
0.9 600.4765625
0.91 600.4765625
0.92 600.4765625
0.93 600.4765625
0.94 600.4765625
0.95 600.4765625
0.96 600.4765625
0.97 600.4765625
0.98 600.4765625
0.99 600.4765625
1 600.4765625
1.01 600.4765625
1.02 600.98046875
1.03 600.9765625
1.04 619.01953125
1.05 650.87890625
1.06 715.8515625
1.07 744.15625
1.08 718.671875
1.09 718.87890625
1.1 718.87890625
1.11 718.87890625
1.12 718.87890625
1.13 718.87890625
1.14 718.87890625
1.15 718.87890625
1.16 718.87890625
1.17 718.87890625
1.18 718.87890625
1.19 718.87890625
1.2 718.87890625
1.21 718.87890625
1.22 718.87890625
1.23 718.87890625
1.24 718.87890625
1.25 718.87890625
1.26 718.87890625
1.27 718.87890625
1.28 718.87890625
1.29 718.87890625
1.3 718.87890625
1.31 718.87890625
1.32 718.87890625
1.33 718.87890625
1.34 718.87890625
1.35 718.87890625
1.36 718.87890625
1.37 718.87890625
1.38 718.87890625
1.39 718.87890625
1.4 718.87890625
1.41 718.87890625
1.42 718.87890625
1.43 718.87890625
1.44 718.87890625
1.45 718.87890625
1.46 718.87890625
1.47 718.87890625
1.48 718.87890625
1.49 718.87890625
1.5 718.87890625
1.51 718.87890625
1.52 719.3984375
1.53 722.15625
1.54 723.46875
1.55 723.46875
1.56 723.46875
1.57 726.015625
1.58 726.015625
1.59 726.015625
1.6 726.015625
1.61 726.015625
1.62 726.015625
1.63 727.05078125
1.64 727.05078125
1.65 727.05078125
1.66 727.05078125
1.67 727.05078125
1.68 727.05078125
1.69 727.05078125
1.7 730.21484375
1.71 780.5859375
1.72 781.11328125
1.73 850.76171875
1.74 1047.21484375
1.75 1149.640625
1.76 1152.09765625
1.77 1193.921875
1.78 1218.94921875
1.79 1218.94921875
1.8 1246.53515625
1.81 1289.58984375
1.82 1289.58984375
1.83 1290.015625
1.84 1290.015625
1.85 1344.24609375
1.86 1344.24609375
1.87 1344.24609375
1.88 1279.31640625
1.89 1413.89453125
1.9 1524.75390625
1.91 1524.75390625
1.92 1524.75390625
1.93 1524.75390625
1.94 1524.75390625
1.95 1524.75390625
1.96 1524.75390625
1.97 1524.75390625
1.98 1524.75390625
1.99 1524.75390625
2 1524.75390625
2.01 1524.75390625
2.02 1524.75390625
2.03 1524.75390625
2.04 1524.75390625
2.05 1524.75390625
2.06 1524.75390625
2.07 1198.91796875
2.08 983.59765625
2.09 1012.47265625
2.1 1012.47265625
2.11 1012.47265625
2.12 948.56640625
2.13 884.60546875
};
\addplot [, color0, dotted, forget plot]
table {%
0 276.03515625
0.01 276.23828125
0.02 278.53125
0.03 278.53125
0.04 278.53125
0.05 278.53125
0.06 278.53125
0.07 278.53125
0.08 278.53125
0.09 278.53125
0.1 278.53125
0.11 278.53125
0.12 278.53125
0.13 278.53125
0.14 278.53125
0.15 278.53125
0.16 278.53125
0.17 278.53125
0.18 278.53125
0.19 278.53125
0.2 278.53125
0.21 278.53125
0.22 278.53125
0.23 278.53125
0.24 278.53125
0.25 278.53125
0.26 278.53125
0.27 278.53125
0.28 278.53125
0.29 278.53125
0.3 278.53125
0.31 278.53125
0.32 278.53125
0.33 278.53125
0.34 278.53125
0.35 278.53125
0.36 278.53125
0.37 278.53125
0.38 278.53125
0.39 278.53125
0.4 278.53125
0.41 278.53125
0.42 278.53125
0.43 278.53125
0.44 278.53125
0.45 308.59765625
0.46 336.44140625
0.47 332.03125
0.48 332.97265625
0.49 361.84765625
0.5 370.9921875
0.51 395.7421875
0.52 407.8125
0.53 425.6015625
0.54 444.421875
0.55 475.25
0.56 541.8359375
0.57 601.0859375
0.58 601.0859375
0.59 601.0859375
0.6 601.0859375
0.61 601.0859375
0.62 601.0859375
0.63 601.0859375
0.64 601.0859375
0.65 601.0859375
0.66 601.0859375
0.67 601.0859375
0.68 601.0859375
0.69 601.0859375
0.7 601.0859375
0.71 601.0859375
0.72 601.0859375
0.73 601.0859375
0.74 601.0859375
0.75 601.0859375
0.76 601.0859375
0.77 601.0859375
0.78 601.0859375
0.79 601.0859375
0.8 601.0859375
0.81 601.0859375
0.82 601.0859375
0.83 601.0859375
0.84 601.0859375
0.85 601.0859375
0.86 601.0859375
0.87 601.0859375
0.88 601.0859375
0.89 601.0859375
0.9 601.0859375
0.91 601.0859375
0.92 601.0859375
0.93 601.0859375
0.94 601.0859375
0.95 601.0859375
0.96 601.0859375
0.97 601.0859375
0.98 601.0859375
0.99 601.0859375
1 601.0859375
1.01 601.0859375
1.02 601.58984375
1.03 601.5859375
1.04 621.17578125
1.05 653.48828125
1.06 720.4609375
1.07 740.1171875
1.08 719.27734375
1.09 719.4921875
1.1 719.4921875
1.11 719.4921875
1.12 719.4921875
1.13 719.4921875
1.14 719.4921875
1.15 719.4921875
1.16 719.4921875
1.17 719.4921875
1.18 719.4921875
1.19 719.4921875
1.2 719.4921875
1.21 719.4921875
1.22 719.4921875
1.23 719.4921875
1.24 719.4921875
1.25 719.4921875
1.26 719.4921875
1.27 719.4921875
1.28 719.4921875
1.29 719.4921875
1.3 719.4921875
1.31 719.4921875
1.32 719.4921875
1.33 719.4921875
1.34 719.4921875
1.35 719.4921875
1.36 719.4921875
1.37 719.4921875
1.38 719.4921875
1.39 719.4921875
1.4 719.4921875
1.41 719.4921875
1.42 719.4921875
1.43 719.4921875
1.44 719.4921875
1.45 719.4921875
1.46 719.4921875
1.47 719.4921875
1.48 719.4921875
1.49 719.4921875
1.5 719.4921875
1.51 719.4921875
1.52 719.984375
1.53 722.52734375
1.54 724.2109375
1.55 724.2109375
1.56 724.2109375
1.57 726.5859375
1.58 726.8203125
1.59 726.8203125
1.6 726.8203125
1.61 726.8203125
1.62 726.8203125
1.63 727.98828125
1.64 727.98828125
1.65 727.98828125
1.66 727.98828125
1.67 727.98828125
1.68 727.98828125
1.69 727.98828125
1.7 731.11328125
1.71 757.3828125
1.72 765.57421875
1.73 843.51953125
1.74 1026.08984375
1.75 1145.4765625
1.76 1147.92578125
1.77 1182.05078125
1.78 1207.9765625
1.79 1214.41796875
1.8 1232.20703125
1.81 1276.80859375
1.82 1284.54296875
1.83 1285.0625
1.84 1285.0625
1.85 1339.6875
1.86 1339.6875
1.87 1339.6875
1.88 1248.96875
1.89 1382.7734375
1.9 1516.3203125
1.91 1520.1875
1.92 1520.1875
1.93 1520.1875
1.94 1520.1875
1.95 1520.1875
1.96 1520.1875
1.97 1520.1875
1.98 1520.1875
1.99 1520.1875
2 1520.1875
2.01 1520.1875
2.02 1520.1875
2.03 1520.1875
2.04 1520.1875
2.05 1520.1875
2.06 1520.1875
2.07 1282.2265625
2.08 951.9609375
2.09 1008.1640625
2.1 1008.1640625
2.11 1008.1640625
2.12 1008.1640625
2.13 880.28515625
};
\addplot [, color0, dotted, forget plot]
table {%
0 275.72265625
0.01 275.92578125
0.02 278.29296875
0.03 278.29296875
0.04 278.29296875
0.05 278.29296875
0.06 278.29296875
0.07 278.29296875
0.08 278.29296875
0.09 278.29296875
0.1 278.29296875
0.11 278.29296875
0.12 278.29296875
0.13 278.29296875
0.14 278.29296875
0.15 278.29296875
0.16 278.29296875
0.17 278.29296875
0.18 278.29296875
0.19 278.29296875
0.2 278.29296875
0.21 278.29296875
0.22 278.29296875
0.23 278.29296875
0.24 278.29296875
0.25 278.29296875
0.26 278.29296875
0.27 278.29296875
0.28 278.29296875
0.29 278.29296875
0.3 278.29296875
0.31 278.29296875
0.32 278.29296875
0.33 278.29296875
0.34 278.29296875
0.35 278.29296875
0.36 278.29296875
0.37 278.29296875
0.38 278.29296875
0.39 278.29296875
0.4 278.29296875
0.41 278.29296875
0.42 278.29296875
0.43 278.29296875
0.44 278.29296875
0.45 302.41796875
0.46 330.77734375
0.47 326.2890625
0.48 325.16796875
0.49 356.10546875
0.5 366.0234375
0.51 393.60546875
0.52 402.58203125
0.53 425.01171875
0.54 439.96484375
0.55 464.91796875
0.56 533.515625
0.57 599.515625
0.58 600.765625
0.59 600.765625
0.6 600.765625
0.61 600.765625
0.62 600.765625
0.63 600.765625
0.64 600.765625
0.65 600.765625
0.66 600.765625
0.67 600.765625
0.68 600.765625
0.69 600.765625
0.7 600.765625
0.71 600.765625
0.72 600.765625
0.73 600.765625
0.74 600.765625
0.75 600.765625
0.76 600.765625
0.77 600.765625
0.78 600.765625
0.79 600.765625
0.8 600.765625
0.81 600.765625
0.82 600.765625
0.83 600.765625
0.84 600.765625
0.85 600.765625
0.86 600.765625
0.87 600.765625
0.88 600.765625
0.89 600.765625
0.9 600.765625
0.91 600.765625
0.92 600.765625
0.93 600.765625
0.94 600.765625
0.95 600.765625
0.96 600.765625
0.97 600.765625
0.98 600.765625
0.99 600.765625
1 600.765625
1.01 600.765625
1.02 601.26953125
1.03 601.265625
1.04 619.05078125
1.05 649.16796875
1.06 716.140625
1.07 756.40625
1.08 718.95703125
1.09 719.16015625
1.1 719.16015625
1.11 719.16015625
1.12 719.16015625
1.13 719.16015625
1.14 719.16015625
1.15 719.16015625
1.16 719.16015625
1.17 719.16015625
1.18 719.16015625
1.19 719.16015625
1.2 719.16015625
1.21 719.16015625
1.22 719.16015625
1.23 719.16015625
1.24 719.16015625
1.25 719.16015625
1.26 719.16015625
1.27 719.16015625
1.28 719.16015625
1.29 719.16015625
1.3 719.16015625
1.31 719.16015625
1.32 719.16015625
1.33 719.16015625
1.34 719.16015625
1.35 719.16015625
1.36 719.16015625
1.37 719.16015625
1.38 719.16015625
1.39 719.16015625
1.4 719.16015625
1.41 719.16015625
1.42 719.16015625
1.43 719.16015625
1.44 719.16015625
1.45 719.16015625
1.46 719.16015625
1.47 719.16015625
1.48 719.16015625
1.49 719.16015625
1.5 719.16015625
1.51 719.16015625
1.52 719.18359375
1.53 722.40625
1.54 724.01953125
1.55 724.01953125
1.56 724.01953125
1.57 726.18359375
1.58 726.6328125
1.59 726.6328125
1.6 726.6328125
1.61 726.6328125
1.62 726.6328125
1.63 728.66796875
1.64 728.66796875
1.65 728.66796875
1.66 728.66796875
1.67 728.66796875
1.68 728.66796875
1.69 728.66796875
1.7 730.953125
1.71 757.7265625
1.72 765.62109375
1.73 814.20703125
1.74 989.44140625
1.75 1139.34375
1.76 1147.9140625
1.77 1164.41015625
1.78 1219.1328125
1.79 1214.609375
1.8 1226.2109375
1.81 1261.015625
1.82 1285.25
1.83 1285.671875
1.84 1285.671875
1.85 1285.671875
1.86 1285.671875
1.87 1285.671875
1.88 1248.46484375
1.89 1243.6171875
1.9 1377.6796875
1.91 1435.4296875
1.92 1435.4296875
1.93 1435.4296875
1.94 1435.4296875
1.95 1435.4296875
1.96 1435.4296875
1.97 1435.4296875
1.98 1435.4296875
1.99 1435.4296875
2 1435.4296875
2.01 1435.4296875
2.02 1435.4296875
2.03 1435.4296875
2.04 1435.4296875
2.05 1435.4296875
2.06 1435.4296875
2.07 1359.71484375
2.08 961.46875
2.09 922.890625
2.1 922.890625
2.11 922.890625
2.12 922.890625
2.13 794.98046875
2.14 795.01953125
};
\addplot [, color0, dotted, forget plot]
table {%
0 275.28515625
0.01 275.4921875
0.02 277.83203125
0.03 277.83203125
0.04 277.83203125
0.05 277.83203125
0.06 277.83203125
0.07 277.83203125
0.08 277.83203125
0.09 277.83203125
0.1 277.83203125
0.11 277.83203125
0.12 277.83203125
0.13 277.83203125
0.14 277.83203125
0.15 277.83203125
0.16 277.83203125
0.17 277.83203125
0.18 277.83203125
0.19 277.83203125
0.2 277.83203125
0.21 277.83203125
0.22 277.83203125
0.23 277.83203125
0.24 277.83203125
0.25 277.83203125
0.26 277.83203125
0.27 277.83203125
0.28 277.83203125
0.29 277.83203125
0.3 277.83203125
0.31 277.83203125
0.32 277.83203125
0.33 277.83203125
0.34 277.83203125
0.35 277.83203125
0.36 277.83203125
0.37 277.83203125
0.38 277.83203125
0.39 277.83203125
0.4 277.83203125
0.41 277.83203125
0.42 277.83203125
0.43 277.83203125
0.44 277.83203125
0.45 297.0625
0.46 326.453125
0.47 321.6796875
0.48 319.265625
0.49 351.234375
0.5 365.5390625
0.51 388.22265625
0.52 397.45703125
0.53 424.52734375
0.54 434.8359375
0.55 454.4296875
0.56 521.02734375
0.57 589.02734375
0.58 600.27734375
0.59 600.27734375
0.6 600.27734375
0.61 600.27734375
0.62 600.27734375
0.63 600.27734375
0.64 600.27734375
0.65 600.27734375
0.66 600.27734375
0.67 600.27734375
0.68 600.27734375
0.69 600.27734375
0.7 600.27734375
0.71 600.27734375
0.72 600.27734375
0.73 600.27734375
0.74 600.27734375
0.75 600.27734375
0.76 600.27734375
0.77 600.27734375
0.78 600.27734375
0.79 600.27734375
0.8 600.27734375
0.81 600.27734375
0.82 600.27734375
0.83 600.27734375
0.84 600.27734375
0.85 600.27734375
0.86 600.27734375
0.87 600.27734375
0.88 600.27734375
0.89 600.27734375
0.9 600.27734375
0.91 600.27734375
0.92 600.27734375
0.93 600.27734375
0.94 600.27734375
0.95 600.27734375
0.96 600.27734375
0.97 600.27734375
0.98 600.27734375
0.99 600.27734375
1 600.27734375
1.01 600.27734375
1.02 600.78125
1.03 600.77734375
1.04 613.1484375
1.05 636.6796875
1.06 701.65234375
1.07 765.02734375
1.08 718.46484375
1.09 718.6796875
1.1 718.6796875
1.11 718.6796875
1.12 718.6796875
1.13 718.6796875
1.14 718.6796875
1.15 718.6796875
1.16 718.6796875
1.17 718.6796875
1.18 718.6796875
1.19 718.6796875
1.2 718.6796875
1.21 718.6796875
1.22 718.6796875
1.23 718.6796875
1.24 718.6796875
1.25 718.6796875
1.26 718.6796875
1.27 718.6796875
1.28 718.6796875
1.29 718.6796875
1.3 718.6796875
1.31 718.6796875
1.32 718.6796875
1.33 718.6796875
1.34 718.6796875
1.35 718.6796875
1.36 718.6796875
1.37 718.6796875
1.38 718.6796875
1.39 718.6796875
1.4 718.6796875
1.41 718.6796875
1.42 718.6796875
1.43 718.6796875
1.44 718.6796875
1.45 718.6796875
1.46 718.6796875
1.47 718.6796875
1.48 718.6796875
1.49 718.6796875
1.5 718.6796875
1.51 718.6796875
1.52 718.6796875
1.53 721.53125
1.54 723.2890625
1.55 723.2890625
1.56 723.2890625
1.57 724.65234375
1.58 725.8984375
1.59 725.8984375
1.6 725.8984375
1.61 725.8984375
1.62 725.8984375
1.63 726.125
1.64 726.125
1.65 726.125
1.66 726.125
1.67 726.125
1.68 726.125
1.69 726.125
1.7 730.1328125
1.71 780.2890625
1.72 780.2890625
1.73 806.91796875
1.74 982.48828125
1.75 1142.90234375
1.76 1151.578125
1.77 1167.8359375
1.78 1222.77734375
1.79 1218.51171875
1.8 1230.11328125
1.81 1264.66015625
1.82 1288.89453125
1.83 1289.31640625
1.84 1289.31640625
1.85 1343.73828125
1.86 1343.73828125
1.87 1343.73828125
1.88 1251.875
1.89 1358.984375
1.9 1493.046875
1.91 1524.2421875
1.92 1524.2421875
1.93 1524.2421875
1.94 1524.2421875
1.95 1524.2421875
1.96 1524.2421875
1.97 1524.2421875
1.98 1524.2421875
1.99 1524.2421875
2 1524.2421875
2.01 1524.2421875
2.02 1524.2421875
2.03 1524.2421875
2.04 1524.2421875
2.05 1524.2421875
2.06 1524.2421875
2.07 1346.28125
2.08 971.109375
2.09 1011.9609375
2.1 1011.9609375
2.11 1011.9609375
2.12 1011.9609375
2.13 884.09375
};
\addplot [, color0, dotted, forget plot]
table {%
0 275.51953125
0.01 275.6640625
0.02 277.94921875
0.03 277.94921875
0.04 277.94921875
0.05 277.94921875
0.06 277.94921875
0.07 277.94921875
0.08 277.94921875
0.09 277.94921875
0.1 277.94921875
0.11 277.94921875
0.12 277.94921875
0.13 277.94921875
0.14 277.94921875
0.15 277.94921875
0.16 277.94921875
0.17 277.94921875
0.18 277.94921875
0.19 277.94921875
0.2 277.94921875
0.21 277.94921875
0.22 277.94921875
0.23 277.94921875
0.24 277.94921875
0.25 277.94921875
0.26 277.94921875
0.27 277.94921875
0.28 277.94921875
0.29 277.94921875
0.3 277.94921875
0.31 277.94921875
0.32 277.94921875
0.33 277.94921875
0.34 277.94921875
0.35 277.94921875
0.36 277.94921875
0.37 277.94921875
0.38 277.94921875
0.39 277.94921875
0.4 277.94921875
0.41 277.94921875
0.42 277.94921875
0.43 277.94921875
0.44 277.94921875
0.45 299.4921875
0.46 328.3671875
0.47 323.6953125
0.48 321.80078125
0.49 353.51171875
0.5 365.74609375
0.51 391.5234375
0.52 401.015625
0.53 424.734375
0.54 438.65625
0.55 462.640625
0.56 531.23828125
0.57 597.23828125
0.58 600.48828125
0.59 600.48828125
0.6 600.48828125
0.61 600.48828125
0.62 600.48828125
0.63 600.48828125
0.64 600.48828125
0.65 600.48828125
0.66 600.48828125
0.67 600.48828125
0.68 600.48828125
0.69 600.48828125
0.7 600.48828125
0.71 600.48828125
0.72 600.48828125
0.73 600.48828125
0.74 600.48828125
0.75 600.48828125
0.76 600.48828125
0.77 600.48828125
0.78 600.48828125
0.79 600.48828125
0.8 600.48828125
0.81 600.48828125
0.82 600.48828125
0.83 600.48828125
0.84 600.48828125
0.85 600.48828125
0.86 600.48828125
0.87 600.48828125
0.88 600.48828125
0.89 600.48828125
0.9 600.48828125
0.91 600.48828125
0.92 600.48828125
0.93 600.48828125
0.94 600.48828125
0.95 600.48828125
0.96 600.48828125
0.97 600.48828125
0.98 600.48828125
0.99 600.48828125
1 600.48828125
1.01 600.48828125
1.02 600.9921875
1.03 600.98828125
1.04 620.578125
1.05 654.890625
1.06 719.86328125
1.07 739.5234375
1.08 718.67578125
1.09 718.890625
1.1 718.890625
1.11 718.890625
1.12 718.890625
1.13 718.890625
1.14 718.890625
1.15 718.890625
1.16 718.890625
1.17 718.890625
1.18 718.890625
1.19 718.890625
1.2 718.890625
1.21 718.890625
1.22 718.890625
1.23 718.890625
1.24 718.890625
1.25 718.890625
1.26 718.890625
1.27 718.890625
1.28 718.890625
1.29 718.890625
1.3 718.890625
1.31 718.890625
1.32 718.890625
1.33 718.890625
1.34 718.890625
1.35 718.890625
1.36 718.890625
1.37 718.890625
1.38 718.890625
1.39 718.890625
1.4 718.890625
1.41 718.890625
1.42 718.890625
1.43 718.890625
1.44 718.890625
1.45 718.890625
1.46 718.890625
1.47 718.890625
1.48 718.890625
1.49 718.890625
1.5 718.890625
1.51 718.890625
1.52 719.65234375
1.53 721.80859375
1.54 723.3984375
1.55 723.3984375
1.56 723.3984375
1.57 725.8828125
1.58 725.8828125
1.59 725.8828125
1.6 725.8828125
1.61 725.8828125
1.62 725.8828125
1.63 727.02734375
1.64 727.02734375
1.65 727.02734375
1.66 727.02734375
1.67 727.02734375
1.68 727.02734375
1.69 727.02734375
1.7 740.328125
1.71 780.78125
1.72 781.30078125
1.73 863.12890625
1.74 1058.29296875
1.75 1149.8984375
1.76 1152.47265625
1.77 1196.5546875
1.78 1219.5390625
1.79 1219.5390625
1.8 1248.671875
1.81 1289.6640625
1.82 1289.6640625
1.83 1292.72265625
1.84 1292.72265625
1.85 1347.26953125
1.86 1347.26953125
1.87 1347.26953125
1.88 1285.69921875
1.89 1418.73046875
1.9 1527.78515625
1.91 1527.78515625
1.92 1527.78515625
1.93 1527.78515625
1.94 1527.78515625
1.95 1527.78515625
1.96 1527.78515625
1.97 1527.78515625
1.98 1527.78515625
1.99 1527.78515625
2 1527.78515625
2.01 1527.78515625
2.02 1527.78515625
2.03 1527.78515625
2.04 1527.78515625
2.05 1527.78515625
2.06 1527.78515625
2.07 1201.94921875
2.08 989.72265625
2.09 1015.50390625
2.1 1015.50390625
2.11 1015.50390625
2.12 947.59375
2.13 887.62890625
};
\addplot [, color0, dotted, forget plot]
table {%
0 275.92578125
0.01 276.13671875
0.02 278.5546875
0.03 278.5546875
0.04 278.5546875
0.05 278.5546875
0.06 278.5546875
0.07 278.5546875
0.08 278.5546875
0.09 278.5546875
0.1 278.5546875
0.11 278.5546875
0.12 278.5546875
0.13 278.5546875
0.14 278.5546875
0.15 278.5546875
0.16 278.5546875
0.17 278.5546875
0.18 278.5546875
0.19 278.5546875
0.2 278.5546875
0.21 278.5546875
0.22 278.5546875
0.23 278.5546875
0.24 278.5546875
0.25 278.5546875
0.26 278.5546875
0.27 278.5546875
0.28 278.5546875
0.29 278.5546875
0.3 278.5546875
0.31 278.5546875
0.32 278.5546875
0.33 278.5546875
0.34 278.5546875
0.35 278.5546875
0.36 278.5546875
0.37 278.5546875
0.38 278.5546875
0.39 278.5546875
0.4 278.5546875
0.41 278.5546875
0.42 278.5546875
0.43 278.5546875
0.44 278.5546875
0.45 309.12890625
0.46 336.97265625
0.47 332.45703125
0.48 333.66015625
0.49 362.27734375
0.5 371.67578125
0.51 395.65234375
0.52 409.26953125
0.53 425.51171875
0.54 446.39453125
0.55 477.26171875
0.56 543.74609375
0.57 600.99609375
0.58 600.99609375
0.59 600.99609375
0.6 600.99609375
0.61 600.99609375
0.62 600.99609375
0.63 600.99609375
0.64 600.99609375
0.65 600.99609375
0.66 600.99609375
0.67 600.99609375
0.68 600.99609375
0.69 600.99609375
0.7 600.99609375
0.71 600.99609375
0.72 600.99609375
0.73 600.99609375
0.74 600.99609375
0.75 600.99609375
0.76 600.99609375
0.77 600.99609375
0.78 600.99609375
0.79 600.99609375
0.8 600.99609375
0.81 600.99609375
0.82 600.99609375
0.83 600.99609375
0.84 600.99609375
0.85 600.99609375
0.86 600.99609375
0.87 600.99609375
0.88 600.99609375
0.89 600.99609375
0.9 600.99609375
0.91 600.99609375
0.92 600.99609375
0.93 600.99609375
0.94 600.99609375
0.95 600.99609375
0.96 600.99609375
0.97 600.99609375
0.98 600.99609375
0.99 600.99609375
1 600.99609375
1.01 601.5
1.02 601.49609375
1.03 601.49609375
1.04 627.53125
1.05 668.37109375
1.06 736.37109375
1.07 719.18359375
1.08 719.18359375
1.09 719.40625
1.1 719.40625
1.11 719.40625
1.12 719.40625
1.13 719.40625
1.14 719.40625
1.15 719.40625
1.16 719.40625
1.17 719.40625
1.18 719.40625
1.19 719.40625
1.2 719.40625
1.21 719.40625
1.22 719.40625
1.23 719.40625
1.24 719.40625
1.25 719.40625
1.26 719.40625
1.27 719.40625
1.28 719.40625
1.29 719.40625
1.3 719.40625
1.31 719.40625
1.32 719.40625
1.33 719.40625
1.34 719.40625
1.35 719.40625
1.36 719.40625
1.37 719.40625
1.38 719.40625
1.39 719.40625
1.4 719.40625
1.41 719.40625
1.42 719.40625
1.43 719.40625
1.44 719.40625
1.45 719.40625
1.46 719.40625
1.47 719.40625
1.48 719.40625
1.49 719.40625
1.5 719.40625
1.51 719.40625
1.52 722.3125
1.53 723.984375
1.54 724.05859375
1.55 724.05859375
1.56 724.8515625
1.57 726.66796875
1.58 726.66796875
1.59 726.66796875
1.6 726.66796875
1.61 726.66796875
1.62 728.2265625
1.63 728.2265625
1.64 728.2265625
1.65 728.2265625
1.66 728.2265625
1.67 728.2265625
1.68 728.2265625
1.69 728.2265625
1.7 740.48828125
1.71 757.546875
1.72 773.15625
1.73 889.9296875
1.74 1086.3828125
1.75 1146.125
1.76 1148.12109375
1.77 1202.828125
1.78 1214.39453125
1.79 1216.71484375
1.8 1254.87109375
1.81 1284.77734375
1.82 1284.77734375
1.83 1285.078125
1.84 1285.078125
1.85 1339.85546875
1.86 1339.85546875
1.87 1339.85546875
1.88 1285.2421875
1.89 1419.046875
1.9 1519.8515625
1.91 1519.8515625
1.92 1519.8515625
1.93 1519.8515625
1.94 1519.8515625
1.95 1519.8515625
1.96 1519.8515625
1.97 1519.8515625
1.98 1519.8515625
1.99 1519.8515625
2 1519.8515625
2.01 1519.8515625
2.02 1519.8515625
2.03 1519.8515625
2.04 1519.8515625
2.05 1519.8515625
2.06 1519.8515625
2.07 1231.890625
2.08 970.9609375
2.09 1007.828125
2.1 1007.828125
2.11 1007.828125
2.12 970.04296875
2.13 879.9609375
};
\addplot [, color0, dotted, forget plot]
table {%
0 275.9609375
0.01 276.1875
0.02 278.5703125
0.03 278.5703125
0.04 278.5703125
0.05 278.5703125
0.06 278.5703125
0.07 278.5703125
0.08 278.5703125
0.09 278.5703125
0.1 278.5703125
0.11 278.5703125
0.12 278.5703125
0.13 278.5703125
0.14 278.5703125
0.15 278.5703125
0.16 278.5703125
0.17 278.5703125
0.18 278.5703125
0.19 278.5703125
0.2 278.5703125
0.21 278.5703125
0.22 278.5703125
0.23 278.5703125
0.24 278.5703125
0.25 278.5703125
0.26 278.5703125
0.27 278.5703125
0.28 278.5703125
0.29 278.5703125
0.3 278.5703125
0.31 278.5703125
0.32 278.5703125
0.33 278.5703125
0.34 278.5703125
0.35 278.5703125
0.36 278.5703125
0.37 278.5703125
0.38 278.5703125
0.39 278.5703125
0.4 278.5703125
0.41 278.5703125
0.42 278.5703125
0.43 278.5703125
0.44 278.5703125
0.45 293.15625
0.46 323.3203125
0.47 318.8359375
0.48 315.13671875
0.49 348.13671875
0.5 366.30859375
0.51 385.640625
0.52 395.91015625
0.53 422.4609375
0.54 431.7421875
0.55 455.203125
0.56 513.796875
0.57 579.796875
0.58 601.046875
0.59 601.046875
0.6 601.046875
0.61 601.046875
0.62 601.046875
0.63 601.046875
0.64 601.046875
0.65 601.046875
0.66 601.046875
0.67 601.046875
0.68 601.046875
0.69 601.046875
0.7 601.046875
0.71 601.046875
0.72 601.046875
0.73 601.046875
0.74 601.046875
0.75 601.046875
0.76 601.046875
0.77 601.046875
0.78 601.046875
0.79 601.046875
0.8 601.046875
0.81 601.046875
0.82 601.046875
0.83 601.046875
0.84 601.046875
0.85 601.046875
0.86 601.046875
0.87 601.046875
0.88 601.046875
0.89 601.046875
0.9 601.046875
0.91 601.046875
0.92 601.046875
0.93 601.046875
0.94 601.046875
0.95 601.046875
0.96 601.046875
0.97 601.046875
0.98 601.046875
0.99 601.046875
1 601.046875
1.01 601.046875
1.02 601.55078125
1.03 601.546875
1.04 601.546875
1.05 627.32421875
1.06 668.421875
1.07 736.421875
1.08 719.23828125
1.09 719.23828125
1.1 719.44921875
1.11 719.44921875
1.12 719.44921875
1.13 719.44921875
1.14 719.44921875
1.15 719.44921875
1.16 719.44921875
1.17 719.44921875
1.18 719.44921875
1.19 719.44921875
1.2 719.44921875
1.21 719.44921875
1.22 719.44921875
1.23 719.44921875
1.24 719.44921875
1.25 719.44921875
1.26 719.44921875
1.27 719.44921875
1.28 719.44921875
1.29 719.44921875
1.3 719.44921875
1.31 719.44921875
1.32 719.44921875
1.33 719.44921875
1.34 719.44921875
1.35 719.44921875
1.36 719.44921875
1.37 719.44921875
1.38 719.44921875
1.39 719.44921875
1.4 719.44921875
1.41 719.44921875
1.42 719.44921875
1.43 719.44921875
1.44 719.44921875
1.45 719.44921875
1.46 719.44921875
1.47 719.44921875
1.48 719.44921875
1.49 719.44921875
1.5 719.44921875
1.51 719.44921875
1.52 719.44921875
1.53 720.2109375
1.54 723.34375
1.55 724.1015625
1.56 724.1015625
1.57 724.2578125
1.58 726.7109375
1.59 726.7109375
1.6 726.7109375
1.61 726.7109375
1.62 726.7109375
1.63 726.7109375
1.64 727.0234375
1.65 727.0234375
1.66 727.0234375
1.67 727.0234375
1.68 727.0234375
1.69 727.0234375
1.7 727.0234375
1.71 730.93359375
1.72 757.58984375
1.73 770.125
1.74 863.40625
1.75 1060.375
1.76 1145.62890625
1.77 1147.82421875
1.78 1194.58984375
1.79 1214.40234375
1.8 1214.40234375
1.81 1246.11328125
1.82 1285.30078125
1.83 1285.30078125
1.84 1285.71875
1.85 1285.71875
1.86 1339.86328125
1.87 1339.86328125
1.88 1339.86328125
1.89 1279.0546875
1.9 1413.375
1.91 1520.109375
1.92 1520.109375
1.93 1520.109375
1.94 1520.109375
1.95 1520.109375
1.96 1520.109375
1.97 1520.109375
1.98 1520.109375
1.99 1520.109375
2 1520.109375
2.01 1520.109375
2.02 1520.109375
2.03 1520.109375
2.04 1520.109375
2.05 1520.109375
2.06 1520.109375
2.07 1520.109375
2.08 1194.2734375
2.09 979.984375
2.1 1007.828125
2.11 1007.828125
2.12 1007.828125
2.13 941.97265625
2.14 879.99609375
};
\addplot [, color0, dotted, forget plot]
table {%
0 275.26171875
0.01 275.46484375
0.02 277.73828125
0.03 277.73828125
0.04 277.73828125
0.05 277.73828125
0.06 277.73828125
0.07 277.73828125
0.08 277.73828125
0.09 277.73828125
0.1 277.73828125
0.11 277.73828125
0.12 277.73828125
0.13 277.73828125
0.14 277.73828125
0.15 277.73828125
0.16 277.73828125
0.17 277.73828125
0.18 277.73828125
0.19 277.73828125
0.2 277.73828125
0.21 277.73828125
0.22 277.73828125
0.23 277.73828125
0.24 277.73828125
0.25 277.73828125
0.26 277.73828125
0.27 277.73828125
0.28 277.73828125
0.29 277.73828125
0.3 277.73828125
0.31 277.73828125
0.32 277.73828125
0.33 277.73828125
0.34 277.73828125
0.35 277.73828125
0.36 277.73828125
0.37 277.73828125
0.38 277.73828125
0.39 277.73828125
0.4 277.73828125
0.41 277.73828125
0.42 277.73828125
0.43 277.73828125
0.44 277.73828125
0.45 308.0625
0.46 335.6484375
0.47 331.4453125
0.48 332.125
0.49 360.7421875
0.5 369.88671875
0.51 394.89453125
0.52 406.44921875
0.53 424.75
0.54 442.5390625
0.55 472.3984375
0.56 541
0.57 600.25
0.58 600.25
0.59 600.25
0.6 600.25
0.61 600.25
0.62 600.25
0.63 600.25
0.64 600.25
0.65 600.25
0.66 600.25
0.67 600.25
0.68 600.25
0.69 600.25
0.7 600.25
0.71 600.25
0.72 600.25
0.73 600.25
0.74 600.25
0.75 600.25
0.76 600.25
0.77 600.25
0.78 600.25
0.79 600.25
0.8 600.25
0.81 600.25
0.82 600.25
0.83 600.25
0.84 600.25
0.85 600.25
0.86 600.25
0.87 600.25
0.88 600.25
0.89 600.25
0.9 600.25
0.91 600.25
0.92 600.25
0.93 600.25
0.94 600.25
0.95 600.25
0.96 600.25
0.97 600.25
0.98 600.25
0.99 600.25
1 600.25
1.01 600.25
1.02 600.75390625
1.03 600.75
1.04 618.79296875
1.05 650.65234375
1.06 715.625
1.07 764.46875
1.08 718.4375
1.09 718.64453125
1.1 718.64453125
1.11 718.64453125
1.12 718.64453125
1.13 718.64453125
1.14 718.64453125
1.15 718.64453125
1.16 718.64453125
1.17 718.64453125
1.18 718.64453125
1.19 718.64453125
1.2 718.64453125
1.21 718.64453125
1.22 718.64453125
1.23 718.64453125
1.24 718.64453125
1.25 718.64453125
1.26 718.64453125
1.27 718.64453125
1.28 718.64453125
1.29 718.64453125
1.3 718.64453125
1.31 718.64453125
1.32 718.64453125
1.33 718.64453125
1.34 718.64453125
1.35 718.64453125
1.36 718.64453125
1.37 718.64453125
1.38 718.64453125
1.39 718.64453125
1.4 718.64453125
1.41 718.64453125
1.42 718.64453125
1.43 718.64453125
1.44 718.64453125
1.45 718.64453125
1.46 718.64453125
1.47 718.64453125
1.48 718.64453125
1.49 718.64453125
1.5 718.64453125
1.51 718.64453125
1.52 719.18359375
1.53 721.69140625
1.54 723.2578125
1.55 723.2578125
1.56 723.2578125
1.57 725.8671875
1.58 725.8671875
1.59 725.8671875
1.6 725.8671875
1.61 725.8671875
1.62 725.8671875
1.63 726.94921875
1.64 726.94921875
1.65 726.94921875
1.66 726.94921875
1.67 726.94921875
1.68 726.94921875
1.69 726.94921875
1.7 735.09765625
1.71 780.24609375
1.72 780.76171875
1.73 860.5078125
1.74 1056.703125
1.75 1149.27734375
1.76 1151.60546875
1.77 1196.91796875
1.78 1218.515625
1.79 1218.515625
1.8 1250.7421875
1.81 1288.8984375
1.82 1288.8984375
1.83 1289.3203125
1.84 1289.3203125
1.85 1343.7421875
1.86 1343.7421875
1.87 1343.7421875
1.88 1285.25390625
1.89 1419.57421875
1.9 1523.98828125
1.91 1523.98828125
1.92 1523.98828125
1.93 1523.98828125
1.94 1523.98828125
1.95 1523.98828125
1.96 1523.98828125
1.97 1523.98828125
1.98 1523.98828125
1.99 1523.98828125
2 1523.98828125
2.01 1523.98828125
2.02 1523.98828125
2.03 1523.98828125
2.04 1523.98828125
2.05 1523.98828125
2.06 1523.98828125
2.07 1162.03125
2.08 869.27734375
2.09 920.58203125
2.1 920.58203125
2.11 920.58203125
2.12 896.67578125
2.13 792.71484375
};
\addplot [, color0, dotted, forget plot]
table {%
0 275.21484375
0.01 275.421875
0.02 277.75
0.03 277.75
0.04 277.75
0.05 277.75
0.06 277.75
0.07 277.75
0.08 277.75
0.09 277.75
0.1 277.75
0.11 277.75
0.12 277.75
0.13 277.75
0.14 277.75
0.15 277.75
0.16 277.75
0.17 277.75
0.18 277.75
0.19 277.75
0.2 277.75
0.21 277.75
0.22 277.75
0.23 277.75
0.24 277.75
0.25 277.75
0.26 277.75
0.27 277.75
0.28 277.75
0.29 277.75
0.3 277.75
0.31 277.75
0.32 277.75
0.33 277.75
0.34 277.75
0.35 277.75
0.36 277.75
0.37 277.75
0.38 277.75
0.39 277.75
0.4 277.75
0.41 277.75
0.42 277.75
0.43 277.75
0.44 277.75
0.45 302.9140625
0.46 331.015625
0.47 326.51171875
0.48 325.90234375
0.49 356.32421875
0.5 365.4765625
0.51 393.57421875
0.52 402.55078125
0.53 424.46484375
0.54 440.1875
0.55 468.3671875
0.56 534.95703125
0.57 600.95703125
0.58 600.20703125
0.59 600.20703125
0.6 600.20703125
0.61 600.20703125
0.62 600.20703125
0.63 600.20703125
0.64 600.20703125
0.65 600.20703125
0.66 600.20703125
0.67 600.20703125
0.68 600.20703125
0.69 600.20703125
0.7 600.20703125
0.71 600.20703125
0.72 600.20703125
0.73 600.20703125
0.74 600.20703125
0.75 600.20703125
0.76 600.20703125
0.77 600.20703125
0.78 600.20703125
0.79 600.20703125
0.8 600.20703125
0.81 600.20703125
0.82 600.20703125
0.83 600.20703125
0.84 600.20703125
0.85 600.20703125
0.86 600.20703125
0.87 600.20703125
0.88 600.20703125
0.89 600.20703125
0.9 600.20703125
0.91 600.20703125
0.92 600.20703125
0.93 600.20703125
0.94 600.20703125
0.95 600.20703125
0.96 600.20703125
0.97 600.20703125
0.98 600.20703125
0.99 600.20703125
1 600.20703125
1.01 600.20703125
1.02 600.7109375
1.03 600.70703125
1.04 617.4609375
1.05 646.609375
1.06 713.58203125
1.07 776.95703125
1.08 718.40234375
1.09 718.61328125
1.1 718.61328125
1.11 718.61328125
1.12 718.61328125
1.13 718.61328125
1.14 718.61328125
1.15 718.61328125
1.16 718.61328125
1.17 718.61328125
1.18 718.61328125
1.19 718.61328125
1.2 718.61328125
1.21 718.61328125
1.22 718.61328125
1.23 718.61328125
1.24 718.61328125
1.25 718.61328125
1.26 718.61328125
1.27 718.61328125
1.28 718.61328125
1.29 718.61328125
1.3 718.61328125
1.31 718.61328125
1.32 718.61328125
1.33 718.61328125
1.34 718.61328125
1.35 718.61328125
1.36 718.61328125
1.37 718.61328125
1.38 718.61328125
1.39 718.61328125
1.4 718.61328125
1.41 718.61328125
1.42 718.61328125
1.43 718.61328125
1.44 718.61328125
1.45 718.61328125
1.46 718.61328125
1.47 718.61328125
1.48 718.61328125
1.49 718.61328125
1.5 718.61328125
1.51 718.61328125
1.52 718.61328125
1.53 722.03515625
1.54 723.22265625
1.55 723.22265625
1.56 723.22265625
1.57 725.35546875
1.58 725.8359375
1.59 725.8359375
1.6 725.8359375
1.61 725.8359375
1.62 725.8359375
1.63 726.85546875
1.64 726.85546875
1.65 726.85546875
1.66 726.85546875
1.67 726.85546875
1.68 726.85546875
1.69 726.85546875
1.7 730.078125
1.71 780.22265625
1.72 780.73828125
1.73 848.9296875
1.74 1026.046875
1.75 1149.296875
1.76 1151.8359375
1.77 1186.12890625
1.78 1211.59375
1.79 1218.55078125
1.8 1235.82421875
1.81 1273.20703125
1.82 1288.67578125
1.83 1289.15625
1.84 1289.15625
1.85 1343.78125
1.86 1343.78125
1.87 1343.78125
1.88 1247.1328125
1.89 1381.1953125
1.9 1515.515625
1.91 1523.765625
1.92 1523.765625
1.93 1523.765625
1.94 1523.765625
1.95 1523.765625
1.96 1523.765625
1.97 1523.765625
1.98 1523.765625
1.99 1523.765625
2 1523.765625
2.01 1523.765625
2.02 1523.765625
2.03 1523.765625
2.04 1523.765625
2.05 1523.765625
2.06 1523.765625
2.07 1296.51953125
2.08 947.80859375
2.09 1011.7421875
2.1 1011.7421875
2.11 1011.7421875
2.12 1011.7421875
2.13 883.859375
};
\addplot [, color0, dotted, forget plot]
table {%
0 275.48828125
0.01 275.62890625
0.02 277.93359375
0.03 277.93359375
0.04 277.93359375
0.05 277.93359375
0.06 277.93359375
0.07 277.93359375
0.08 277.93359375
0.09 277.93359375
0.1 277.93359375
0.11 277.93359375
0.12 277.93359375
0.13 277.93359375
0.14 277.93359375
0.15 277.93359375
0.16 277.93359375
0.17 277.93359375
0.18 277.93359375
0.19 277.93359375
0.2 277.93359375
0.21 277.93359375
0.22 277.93359375
0.23 277.93359375
0.24 277.93359375
0.25 277.93359375
0.26 277.93359375
0.27 277.93359375
0.28 277.93359375
0.29 277.93359375
0.3 277.93359375
0.31 277.93359375
0.32 277.93359375
0.33 277.93359375
0.34 277.93359375
0.35 277.93359375
0.36 277.93359375
0.37 277.93359375
0.38 277.93359375
0.39 277.93359375
0.4 277.93359375
0.41 277.93359375
0.42 277.93359375
0.43 277.93359375
0.44 277.93359375
0.45 302.0546875
0.46 330.4140625
0.47 325.46875
0.48 323.83203125
0.49 355.02734375
0.5 365.71875
0.51 392.26953125
0.52 401.76171875
0.53 424.70703125
0.54 439.14453125
0.55 464.61328125
0.56 531.20703125
0.57 597.20703125
0.58 600.45703125
0.59 600.45703125
0.6 600.45703125
0.61 600.45703125
0.62 600.45703125
0.63 600.45703125
0.64 600.45703125
0.65 600.45703125
0.66 600.45703125
0.67 600.45703125
0.68 600.45703125
0.69 600.45703125
0.7 600.45703125
0.71 600.45703125
0.72 600.45703125
0.73 600.45703125
0.74 600.45703125
0.75 600.45703125
0.76 600.45703125
0.77 600.45703125
0.78 600.45703125
0.79 600.45703125
0.8 600.45703125
0.81 600.45703125
0.82 600.45703125
0.83 600.45703125
0.84 600.45703125
0.85 600.45703125
0.86 600.45703125
0.87 600.45703125
0.88 600.45703125
0.89 600.45703125
0.9 600.45703125
0.91 600.45703125
0.92 600.45703125
0.93 600.45703125
0.94 600.45703125
0.95 600.45703125
0.96 600.45703125
0.97 600.45703125
0.98 600.45703125
0.99 600.45703125
1 600.45703125
1.01 600.45703125
1.02 600.9609375
1.03 600.95703125
1.04 621.3203125
1.05 654.859375
1.06 721.83203125
1.07 739.4921875
1.08 718.65234375
1.09 718.859375
1.1 718.859375
1.11 718.859375
1.12 718.859375
1.13 718.859375
1.14 718.859375
1.15 718.859375
1.16 718.859375
1.17 718.859375
1.18 718.859375
1.19 718.859375
1.2 718.859375
1.21 718.859375
1.22 718.859375
1.23 718.859375
1.24 718.859375
1.25 718.859375
1.26 718.859375
1.27 718.859375
1.28 718.859375
1.29 718.859375
1.3 718.859375
1.31 718.859375
1.32 718.859375
1.33 718.859375
1.34 718.859375
1.35 718.859375
1.36 718.859375
1.37 718.859375
1.38 718.859375
1.39 718.859375
1.4 718.859375
1.41 718.859375
1.42 718.859375
1.43 718.859375
1.44 718.859375
1.45 718.859375
1.46 718.859375
1.47 718.859375
1.48 718.859375
1.49 718.859375
1.5 718.859375
1.51 718.859375
1.52 719.41796875
1.53 721.60546875
1.54 723.328125
1.55 723.328125
1.56 723.328125
1.57 725.87890625
1.58 725.87890625
1.59 725.87890625
1.6 725.87890625
1.61 725.87890625
1.62 725.87890625
1.63 726.84375
1.64 726.84375
1.65 726.84375
1.66 726.84375
1.67 726.84375
1.68 726.84375
1.69 726.84375
1.7 735.05078125
1.71 780.40234375
1.72 780.91796875
1.73 859.4921875
1.74 1055.9453125
1.75 1149.41015625
1.76 1152.30859375
1.77 1195.7890625
1.78 1218.9609375
1.79 1218.9609375
1.8 1238.0390625
1.81 1289.6015625
1.82 1289.6015625
1.83 1292.66015625
1.84 1292.66015625
1.85 1346.83984375
1.86 1346.83984375
1.87 1346.83984375
1.88 1250.19921875
1.89 1383.74609375
1.9 1518.06640625
1.91 1527.34765625
1.92 1527.34765625
1.93 1527.34765625
1.94 1527.34765625
1.95 1527.34765625
1.96 1527.34765625
1.97 1527.34765625
1.98 1527.34765625
1.99 1527.34765625
2 1527.34765625
2.01 1527.34765625
2.02 1527.34765625
2.03 1527.34765625
2.04 1527.34765625
2.05 1527.34765625
2.06 1527.34765625
2.07 1337.984375
2.08 997.38671875
2.09 1014.80859375
2.1 1014.80859375
2.11 1014.80859375
2.12 1014.80859375
2.13 886.9375
};
\addplot [, color1, dotted, forget plot]
table {%
0 275.8671875
0.01 276.07421875
0.02 278.4375
0.03 278.4375
0.04 278.4375
0.05 278.4375
0.06 278.4375
0.07 278.4375
0.08 278.4375
0.09 278.4375
0.1 278.4375
0.11 278.4375
0.12 278.4375
0.13 278.4375
0.14 278.4375
0.15 278.4375
0.16 278.4375
0.17 278.4375
0.18 278.4375
0.19 278.4375
0.2 278.4375
0.21 278.4375
0.22 278.4375
0.23 278.4375
0.24 278.4375
0.25 278.4375
0.26 278.4375
0.27 278.4375
0.28 278.4375
0.29 278.4375
0.3 278.4375
0.31 278.4375
0.32 278.4375
0.33 278.4375
0.34 278.4375
0.35 278.4375
0.36 278.4375
0.37 278.4375
0.38 278.4375
0.39 278.4375
0.4 278.4375
0.41 278.4375
0.42 278.4375
0.43 278.4375
0.44 278.4375
0.45 300.25390625
0.46 328.61328125
0.47 324.1328125
0.48 322.23828125
0.49 353.69140625
0.5 366.1875
0.51 390.671875
0.52 399.65234375
0.53 425.17578125
0.54 436.2578125
0.55 455.078125
0.56 523.6875
0.57 589.6875
0.58 600.9453125
0.59 600.9453125
0.6 600.9453125
0.61 600.9453125
0.62 600.9453125
0.63 600.9453125
0.64 600.9453125
0.65 600.9453125
0.66 600.9453125
0.67 600.9453125
0.68 600.9453125
0.69 600.9453125
0.7 600.9453125
0.71 600.9453125
0.72 600.9453125
0.73 600.9453125
0.74 600.9453125
0.75 600.9453125
0.76 600.9453125
0.77 600.9453125
0.78 600.9453125
0.79 600.9453125
0.8 600.9453125
0.81 600.9453125
0.82 600.9453125
0.83 600.9453125
0.84 600.9453125
0.85 600.9453125
0.86 600.9453125
0.87 600.9453125
0.88 600.9453125
0.89 600.9453125
0.9 600.9453125
0.91 600.9453125
0.92 600.9453125
0.93 600.9453125
0.94 600.9453125
0.95 600.9453125
0.96 600.9453125
0.97 600.9453125
0.98 600.9453125
0.99 600.9453125
1 600.9453125
1.01 600.9453125
1.02 601.453125
1.03 601.4453125
1.04 613.81640625
1.05 637.34765625
1.06 702.3203125
1.07 767.6953125
1.08 719.140625
1.09 719.3359375
1.1 719.3359375
1.11 719.3359375
1.12 719.3359375
1.13 719.3359375
1.14 719.3359375
1.15 719.3359375
1.16 719.3359375
1.17 719.3359375
1.18 719.3359375
1.19 719.3359375
1.2 719.3359375
1.21 719.3359375
1.22 719.3359375
1.23 719.3359375
1.24 719.3359375
1.25 719.3359375
1.26 719.3359375
1.27 719.3359375
1.28 719.3359375
1.29 719.3359375
1.3 719.3359375
1.31 719.3359375
1.32 719.3359375
1.33 719.3359375
1.34 719.3359375
1.35 719.3359375
1.36 719.3359375
1.37 719.3359375
1.38 719.3359375
1.39 719.3359375
1.4 719.3359375
1.41 719.3359375
1.42 719.3359375
1.43 719.3359375
1.44 719.3359375
1.45 719.3359375
1.46 719.3359375
1.47 719.3359375
1.48 719.3359375
1.49 719.3359375
1.5 719.3359375
1.51 719.3359375
1.52 719.3359375
1.53 722.3046875
1.54 724.17578125
1.55 724.17578125
1.56 724.17578125
1.57 725.46484375
1.58 726.7890625
1.59 726.7890625
1.6 726.7890625
1.61 726.7890625
1.62 726.7890625
1.63 727.08984375
1.64 727.08984375
1.65 727.08984375
1.66 727.08984375
1.67 727.08984375
1.68 727.08984375
1.69 727.08984375
1.7 729.05078125
1.71 781.0625
1.72 781.3203125
1.73 784.78125
1.74 915.51953125
1.75 915.51953125
1.76 915.51953125
1.77 915.51953125
1.78 787.62109375
1.79 971.6953125
1.8 1135.921875
1.81 1270.2421875
1.82 1363.0546875
1.83 1363.0546875
1.84 1363.0546875
1.85 1363.0546875
1.86 1363.0546875
1.87 1363.0546875
1.88 1363.0546875
1.89 1363.0546875
1.9 1363.0546875
1.91 1363.0546875
1.92 1363.0546875
1.93 1363.0546875
1.94 1363.0546875
1.95 1363.0546875
1.96 1363.0546875
1.97 1363.0546875
1.98 1363.0546875
1.99 999.48828125
2 793.2890625
2.01 798.66015625
2.02 822.6953125
2.03 853.50390625
2.04 907.4375
2.05 907.4375
2.06 907.4375
2.07 805.6015625
2.08 805.6015625
2.09 836.796875
2.1 878.8203125
2.11 878.8203125
2.12 878.8203125
2.13 878.8203125
2.14 878.94921875
};
\addplot [, color1, dotted, forget plot]
table {%
0 275.55078125
0.01 275.6875
0.02 278.11328125
0.03 278.11328125
0.04 278.11328125
0.05 278.11328125
0.06 278.11328125
0.07 278.11328125
0.08 278.11328125
0.09 278.11328125
0.1 278.11328125
0.11 278.11328125
0.12 278.11328125
0.13 278.11328125
0.14 278.11328125
0.15 278.11328125
0.16 278.11328125
0.17 278.11328125
0.18 278.11328125
0.19 278.11328125
0.2 278.11328125
0.21 278.11328125
0.22 278.11328125
0.23 278.11328125
0.24 278.11328125
0.25 278.11328125
0.26 278.11328125
0.27 278.11328125
0.28 278.11328125
0.29 278.11328125
0.3 278.11328125
0.31 278.11328125
0.32 278.11328125
0.33 278.11328125
0.34 278.11328125
0.35 278.11328125
0.36 278.11328125
0.37 278.11328125
0.38 278.11328125
0.39 278.11328125
0.4 278.11328125
0.41 278.11328125
0.42 278.11328125
0.43 278.11328125
0.44 278.11328125
0.45 306.1015625
0.46 333.9453125
0.47 329.640625
0.48 330.328125
0.49 359.71875
0.5 368.85546875
0.51 395.15234375
0.52 405.9375
0.53 425.01171875
0.54 443.31640625
0.55 474.796875
0.56 541.26953125
0.57 600.53125
0.58 600.53125
0.59 600.53125
0.6 600.53125
0.61 600.53125
0.62 600.53125
0.63 600.53125
0.64 600.53125
0.65 600.53125
0.66 600.53125
0.67 600.53125
0.68 600.53125
0.69 600.53125
0.7 600.53125
0.71 600.53125
0.72 600.53125
0.73 600.53125
0.74 600.53125
0.75 600.53125
0.76 600.53125
0.77 600.53125
0.78 600.53125
0.79 600.53125
0.8 600.53125
0.81 600.53125
0.82 600.53125
0.83 600.53125
0.84 600.53125
0.85 600.53125
0.86 600.53125
0.87 600.53125
0.88 600.53125
0.89 600.53125
0.9 600.53125
0.91 600.53125
0.92 600.53125
0.93 600.53125
0.94 600.53125
0.95 600.53125
0.96 600.53125
0.97 600.53125
0.98 600.53125
0.99 600.53125
1 600.53125
1.01 600.53125
1.02 601.0390625
1.03 601.03125
1.04 617.01171875
1.05 644.93359375
1.06 711.90625
1.07 777.28125
1.08 718.71875
1.09 718.9140625
1.1 718.9140625
1.11 718.9140625
1.12 718.9140625
1.13 718.9140625
1.14 718.9140625
1.15 718.9140625
1.16 718.9140625
1.17 718.9140625
1.18 718.9140625
1.19 718.9140625
1.2 718.9140625
1.21 718.9140625
1.22 718.9140625
1.23 718.9140625
1.24 718.9140625
1.25 718.9140625
1.26 718.9140625
1.27 718.9140625
1.28 718.9140625
1.29 718.9140625
1.3 718.9140625
1.31 718.9140625
1.32 718.9140625
1.33 718.9140625
1.34 718.9140625
1.35 718.9140625
1.36 718.9140625
1.37 718.9140625
1.38 718.9140625
1.39 718.9140625
1.4 718.9140625
1.41 718.9140625
1.42 718.9140625
1.43 718.9140625
1.44 718.9140625
1.45 718.9140625
1.46 718.9140625
1.47 718.9140625
1.48 718.9140625
1.49 718.9140625
1.5 718.9140625
1.51 718.9140625
1.52 718.9140625
1.53 722.01953125
1.54 723.3828125
1.55 723.3828125
1.56 723.3828125
1.57 725.16015625
1.58 725.93359375
1.59 725.93359375
1.6 725.93359375
1.61 725.93359375
1.62 725.93359375
1.63 726.109375
1.64 726.109375
1.65 726.109375
1.66 726.109375
1.67 726.109375
1.68 726.109375
1.69 726.109375
1.7 730.2109375
1.71 780.46484375
1.72 780.98828125
1.73 796.95703125
1.74 914.77734375
1.75 914.77734375
1.76 914.77734375
1.77 914.77734375
1.78 811.640625
1.79 1008.3515625
1.8 1159.6875
1.81 1293.75
1.82 1362.328125
1.83 1362.328125
1.84 1362.328125
1.85 1362.328125
1.86 1362.328125
1.87 1362.328125
1.88 1362.328125
1.89 1362.328125
1.9 1362.328125
1.91 1362.328125
1.92 1362.328125
1.93 1362.328125
1.94 1362.328125
1.95 1362.328125
1.96 1362.328125
1.97 1362.328125
1.98 1324.65234375
1.99 960.8828125
2 795.81640625
2.01 797.7890625
2.02 836.93359375
2.03 873.203125
2.04 907.234375
2.05 907.234375
2.06 907.234375
2.07 805.39453125
2.08 809.00390625
2.09 851.80078125
2.1 878.61328125
2.11 878.61328125
2.12 878.61328125
2.13 878.61328125
2.14 878.69140625
};
\addplot [, color1, dotted, forget plot]
table {%
0 275.734375
0.01 275.9375
0.02 278.3359375
0.03 278.3359375
0.04 278.3359375
0.05 278.3359375
0.06 278.3359375
0.07 278.3359375
0.08 278.3359375
0.09 278.3359375
0.1 278.3359375
0.11 278.3359375
0.12 278.3359375
0.13 278.3359375
0.14 278.3359375
0.15 278.3359375
0.16 278.3359375
0.17 278.3359375
0.18 278.3359375
0.19 278.3359375
0.2 278.3359375
0.21 278.3359375
0.22 278.3359375
0.23 278.3359375
0.24 278.3359375
0.25 278.3359375
0.26 278.3359375
0.27 278.3359375
0.28 278.3359375
0.29 278.3359375
0.3 278.3359375
0.31 278.3359375
0.32 278.3359375
0.33 278.3359375
0.34 278.3359375
0.35 278.3359375
0.36 278.3359375
0.37 278.3359375
0.38 278.3359375
0.39 278.3359375
0.4 278.3359375
0.41 278.3359375
0.42 278.3359375
0.43 278.3359375
0.44 278.3359375
0.45 305.30078125
0.46 332.88671875
0.47 327.87109375
0.48 326.23046875
0.49 356.65234375
0.5 366.5703125
0.51 394.15625
0.52 403.6484375
0.53 425.046875
0.54 441.02734375
0.55 469.078125
0.56 535.5546875
0.57 601.5546875
0.58 600.8125
0.59 600.8125
0.6 600.8125
0.61 600.8125
0.62 600.8125
0.63 600.8125
0.64 600.8125
0.65 600.8125
0.66 600.8125
0.67 600.8125
0.68 600.8125
0.69 600.8125
0.7 600.8125
0.71 600.8125
0.72 600.8125
0.73 600.8125
0.74 600.8125
0.75 600.8125
0.76 600.8125
0.77 600.8125
0.78 600.8125
0.79 600.8125
0.8 600.8125
0.81 600.8125
0.82 600.8125
0.83 600.8125
0.84 600.8125
0.85 600.8125
0.86 600.8125
0.87 600.8125
0.88 600.8125
0.89 600.8125
0.9 600.8125
0.91 600.8125
0.92 600.8125
0.93 600.8125
0.94 600.8125
0.95 600.8125
0.96 600.8125
0.97 600.8125
0.98 600.8125
0.99 600.8125
1 600.8125
1.01 600.8125
1.02 601.32421875
1.03 601.3125
1.04 619.09765625
1.05 649.21484375
1.06 714.1875
1.07 777.5625
1.08 719.00390625
1.09 719.203125
1.1 719.203125
1.11 719.203125
1.12 719.203125
1.13 719.203125
1.14 719.203125
1.15 719.203125
1.16 719.203125
1.17 719.203125
1.18 719.203125
1.19 719.203125
1.2 719.203125
1.21 719.203125
1.22 719.203125
1.23 719.203125
1.24 719.203125
1.25 719.203125
1.26 719.203125
1.27 719.203125
1.28 719.203125
1.29 719.203125
1.3 719.203125
1.31 719.203125
1.32 719.203125
1.33 719.203125
1.34 719.203125
1.35 719.203125
1.36 719.203125
1.37 719.203125
1.38 719.203125
1.39 719.203125
1.4 719.203125
1.41 719.203125
1.42 719.203125
1.43 719.203125
1.44 719.203125
1.45 719.203125
1.46 719.203125
1.47 719.203125
1.48 719.203125
1.49 719.203125
1.5 719.203125
1.51 719.203125
1.52 719.71875
1.53 722.17578125
1.54 724.04296875
1.55 724.04296875
1.56 724.04296875
1.57 726.4453125
1.58 726.65625
1.59 726.65625
1.6 726.65625
1.61 726.65625
1.62 726.65625
1.63 727.84765625
1.64 727.84765625
1.65 727.84765625
1.66 727.84765625
1.67 727.84765625
1.68 727.84765625
1.69 727.84765625
1.7 731.10546875
1.71 757.75390625
1.72 765.6484375
1.73 805.77734375
1.74 908.38671875
1.75 908.38671875
1.76 908.38671875
1.77 908.38671875
1.78 818.47265625
1.79 1013.37890625
1.8 1161.87890625
1.81 1295.68359375
1.82 1358.84765625
1.83 1358.84765625
1.84 1358.84765625
1.85 1358.84765625
1.86 1358.84765625
1.87 1358.84765625
1.88 1358.84765625
1.89 1358.84765625
1.9 1358.84765625
1.91 1358.84765625
1.92 1358.84765625
1.93 1358.84765625
1.94 1358.84765625
1.95 1358.84765625
1.96 1358.84765625
1.97 1358.84765625
1.98 1283.1875
1.99 919.41796875
2 795.12890625
2.01 797.57421875
2.02 841.33984375
2.03 881.09765625
2.04 906.10546875
2.05 906.10546875
2.06 906.10546875
2.07 809.68359375
2.08 813.55078125
2.09 858.40625
2.1 882.8984375
2.11 882.8984375
2.12 882.8984375
2.13 882.8984375
2.14 883.10546875
};
\addplot [, color1, dotted, forget plot]
table {%
0 275.44140625
0.01 275.5703125
0.02 277.90234375
0.03 277.90234375
0.04 277.90234375
0.05 277.90234375
0.06 277.90234375
0.07 277.90234375
0.08 277.90234375
0.09 277.90234375
0.1 277.90234375
0.11 277.90234375
0.12 277.90234375
0.13 277.90234375
0.14 277.90234375
0.15 277.90234375
0.16 277.90234375
0.17 277.90234375
0.18 277.90234375
0.19 277.90234375
0.2 277.90234375
0.21 277.90234375
0.22 277.90234375
0.23 277.90234375
0.24 277.90234375
0.25 277.90234375
0.26 277.90234375
0.27 277.90234375
0.28 277.90234375
0.29 277.90234375
0.3 277.90234375
0.31 277.90234375
0.32 277.90234375
0.33 277.90234375
0.34 277.90234375
0.35 277.90234375
0.36 277.90234375
0.37 277.90234375
0.38 277.90234375
0.39 277.90234375
0.4 277.90234375
0.41 277.90234375
0.42 277.90234375
0.43 277.90234375
0.44 277.90234375
0.45 301.77734375
0.46 330.39453125
0.47 325.66015625
0.48 323.765625
0.49 354.703125
0.5 365.65625
0.51 392.46484375
0.52 401.953125
0.53 424.640625
0.54 439.078125
0.55 464.546875
0.56 531.16015625
0.57 597.16015625
0.58 600.41015625
0.59 600.41015625
0.6 600.41015625
0.61 600.41015625
0.62 600.41015625
0.63 600.41015625
0.64 600.41015625
0.65 600.41015625
0.66 600.41015625
0.67 600.41015625
0.68 600.41015625
0.69 600.41015625
0.7 600.41015625
0.71 600.41015625
0.72 600.41015625
0.73 600.41015625
0.74 600.41015625
0.75 600.41015625
0.76 600.41015625
0.77 600.41015625
0.78 600.41015625
0.79 600.41015625
0.8 600.41015625
0.81 600.41015625
0.82 600.41015625
0.83 600.41015625
0.84 600.41015625
0.85 600.41015625
0.86 600.41015625
0.87 600.41015625
0.88 600.41015625
0.89 600.41015625
0.9 600.41015625
0.91 600.41015625
0.92 600.41015625
0.93 600.41015625
0.94 600.41015625
0.95 600.41015625
0.96 600.41015625
0.97 600.41015625
0.98 600.41015625
0.99 600.41015625
1 600.41015625
1.01 600.41015625
1.02 600.91796875
1.03 600.91015625
1.04 617.6640625
1.05 646.8125
1.06 711.78515625
1.07 777.16015625
1.08 718.59765625
1.09 718.796875
1.1 718.796875
1.11 718.796875
1.12 718.796875
1.13 718.796875
1.14 718.796875
1.15 718.796875
1.16 718.796875
1.17 718.796875
1.18 718.796875
1.19 718.796875
1.2 718.796875
1.21 718.796875
1.22 718.796875
1.23 718.796875
1.24 718.796875
1.25 718.796875
1.26 718.796875
1.27 718.796875
1.28 718.796875
1.29 718.796875
1.3 718.796875
1.31 718.796875
1.32 718.796875
1.33 718.796875
1.34 718.796875
1.35 718.796875
1.36 718.796875
1.37 718.796875
1.38 718.796875
1.39 718.796875
1.4 718.796875
1.41 718.796875
1.42 718.796875
1.43 718.796875
1.44 718.796875
1.45 718.796875
1.46 718.796875
1.47 718.796875
1.48 718.796875
1.49 718.796875
1.5 718.796875
1.51 718.796875
1.52 718.796875
1.53 721.8203125
1.54 723.30078125
1.55 723.30078125
1.56 723.30078125
1.57 725.4375
1.58 725.79296875
1.59 725.79296875
1.6 725.79296875
1.61 725.79296875
1.62 725.79296875
1.63 726.32421875
1.64 726.32421875
1.65 726.32421875
1.66 726.32421875
1.67 726.32421875
1.68 726.32421875
1.69 726.32421875
1.7 730.375
1.71 780.92578125
1.72 781.453125
1.73 804.15625
1.74 915.015625
1.75 915.015625
1.76 915.015625
1.77 915.015625
1.78 817.83203125
1.79 1014.54296875
1.8 1163.55859375
1.81 1298.13671875
1.82 1362.58984375
1.83 1362.58984375
1.84 1362.58984375
1.85 1362.58984375
1.86 1362.58984375
1.87 1362.58984375
1.88 1362.58984375
1.89 1362.58984375
1.9 1362.58984375
1.91 1362.58984375
1.92 1362.58984375
1.93 1362.58984375
1.94 1362.58984375
1.95 1362.58984375
1.96 1362.58984375
1.97 1362.58984375
1.98 1308.7890625
1.99 948.7890625
2 799.2421875
2.01 801.20703125
2.02 840.98046875
2.03 878.3046875
2.04 910.53125
2.05 910.53125
2.06 910.53125
2.07 813.8515625
2.08 816.4296875
2.09 857.9375
2.1 887.5859375
2.11 887.5859375
2.12 887.5859375
2.13 887.5859375
2.14 887.79296875
};
\addplot [, color1, dotted, forget plot]
table {%
0 275.4296875
0.01 275.62890625
0.02 278
0.03 278
0.04 278
0.05 278
0.06 278
0.07 278
0.08 278
0.09 278
0.1 278
0.11 278
0.12 278
0.13 278
0.14 278
0.15 278
0.16 278
0.17 278
0.18 278
0.19 278
0.2 278
0.21 278
0.22 278
0.23 278
0.24 278
0.25 278
0.26 278
0.27 278
0.28 278
0.29 278
0.3 278
0.31 278
0.32 278
0.33 278
0.34 278
0.35 278
0.36 278
0.37 278
0.38 278
0.39 278
0.4 278
0.41 278
0.42 278
0.43 278
0.44 278
0.45 300.3125
0.46 328.671875
0.47 324.4765625
0.48 323.35546875
0.49 354.29296875
0.5 365.7578125
0.51 391.79296875
0.52 401.0234375
0.53 424.7421875
0.54 438.40234375
0.55 458.64453125
0.56 525.25
0.57 591.25
0.58 600.515625
0.59 600.515625
0.6 600.515625
0.61 600.515625
0.62 600.515625
0.63 600.515625
0.64 600.515625
0.65 600.515625
0.66 600.515625
0.67 600.515625
0.68 600.515625
0.69 600.515625
0.7 600.515625
0.71 600.515625
0.72 600.515625
0.73 600.515625
0.74 600.515625
0.75 600.515625
0.76 600.515625
0.77 600.515625
0.78 600.515625
0.79 600.515625
0.8 600.515625
0.81 600.515625
0.82 600.515625
0.83 600.515625
0.84 600.515625
0.85 600.515625
0.86 600.515625
0.87 600.515625
0.88 600.515625
0.89 600.515625
0.9 600.515625
0.91 600.515625
0.92 600.515625
0.93 600.515625
0.94 600.515625
0.95 600.515625
0.96 600.515625
0.97 600.515625
0.98 600.515625
0.99 600.515625
1 600.515625
1.01 600.515625
1.02 601.0234375
1.03 601.015625
1.04 612.61328125
1.05 634.91796875
1.06 699.890625
1.07 765.265625
1.08 718.703125
1.09 718.90234375
1.1 718.90234375
1.11 718.90234375
1.12 718.90234375
1.13 718.90234375
1.14 718.90234375
1.15 718.90234375
1.16 718.90234375
1.17 718.90234375
1.18 718.90234375
1.19 718.90234375
1.2 718.90234375
1.21 718.90234375
1.22 718.90234375
1.23 718.90234375
1.24 718.90234375
1.25 718.90234375
1.26 718.90234375
1.27 718.90234375
1.28 718.90234375
1.29 718.90234375
1.3 718.90234375
1.31 718.90234375
1.32 718.90234375
1.33 718.90234375
1.34 718.90234375
1.35 718.90234375
1.36 718.90234375
1.37 718.90234375
1.38 718.90234375
1.39 718.90234375
1.4 718.90234375
1.41 718.90234375
1.42 718.90234375
1.43 718.90234375
1.44 718.90234375
1.45 718.90234375
1.46 718.90234375
1.47 718.90234375
1.48 718.90234375
1.49 718.90234375
1.5 718.90234375
1.51 718.90234375
1.52 718.90234375
1.53 721.765625
1.54 723.6171875
1.55 723.62109375
1.56 723.62109375
1.57 724.4765625
1.58 726.23828125
1.59 726.23828125
1.6 726.23828125
1.61 726.23828125
1.62 726.23828125
1.63 728.32421875
1.64 728.32421875
1.65 728.32421875
1.66 728.32421875
1.67 728.32421875
1.68 728.32421875
1.69 728.32421875
1.7 728.32421875
1.71 756.87890625
1.72 757.61328125
1.73 776.20703125
1.74 881.84765625
1.75 910.46484375
1.76 910.46484375
1.77 910.46484375
1.78 846.54296875
1.79 916.66796875
1.8 1096.62109375
1.81 1230.94140625
1.82 1361.39453125
1.83 1361.39453125
1.84 1361.39453125
1.85 1361.39453125
1.86 1361.39453125
1.87 1361.39453125
1.88 1361.39453125
1.89 1361.39453125
1.9 1361.39453125
1.91 1361.39453125
1.92 1361.39453125
1.93 1361.39453125
1.94 1361.39453125
1.95 1361.39453125
1.96 1361.39453125
1.97 1361.39453125
1.98 1361.39453125
1.99 1096.2421875
2 788.046875
2.01 792.2734375
2.02 798.015625
2.03 891.6015625
2.04 901.39453125
2.05 901.39453125
2.06 901.39453125
2.07 793.6328125
2.08 803.16796875
2.09 817.60546875
2.1 876.12890625
2.11 876.12890625
2.12 876.12890625
2.13 876.12890625
2.14 876.33203125
};
\addplot [, color1, dotted, forget plot]
table {%
0 275.22265625
0.01 275.421875
0.02 277.73828125
0.03 277.73828125
0.04 277.73828125
0.05 277.73828125
0.06 277.73828125
0.07 277.73828125
0.08 277.73828125
0.09 277.73828125
0.1 277.73828125
0.11 277.73828125
0.12 277.73828125
0.13 277.73828125
0.14 277.73828125
0.15 277.73828125
0.16 277.73828125
0.17 277.73828125
0.18 277.73828125
0.19 277.73828125
0.2 277.73828125
0.21 277.73828125
0.22 277.73828125
0.23 277.73828125
0.24 277.73828125
0.25 277.73828125
0.26 277.73828125
0.27 277.73828125
0.28 277.73828125
0.29 277.73828125
0.3 277.73828125
0.31 277.73828125
0.32 277.73828125
0.33 277.73828125
0.34 277.73828125
0.35 277.73828125
0.36 277.73828125
0.37 277.73828125
0.38 277.73828125
0.39 277.73828125
0.4 277.73828125
0.41 277.73828125
0.42 277.73828125
0.43 277.73828125
0.44 277.73828125
0.45 309.6015625
0.46 315.68359375
0.47 332.6875
0.48 333.62890625
0.49 361.98828125
0.5 371.39453125
0.51 394.85546875
0.52 408.47265625
0.53 424.7109375
0.54 443.53125
0.55 474.48828125
0.56 542.96875
0.57 600.22265625
0.58 600.22265625
0.59 600.22265625
0.6 600.22265625
0.61 600.22265625
0.62 600.22265625
0.63 600.22265625
0.64 600.22265625
0.65 600.22265625
0.66 600.22265625
0.67 600.22265625
0.68 600.22265625
0.69 600.22265625
0.7 600.22265625
0.71 600.22265625
0.72 600.22265625
0.73 600.22265625
0.74 600.22265625
0.75 600.22265625
0.76 600.22265625
0.77 600.22265625
0.78 600.22265625
0.79 600.22265625
0.8 600.22265625
0.81 600.22265625
0.82 600.22265625
0.83 600.22265625
0.84 600.22265625
0.85 600.22265625
0.86 600.22265625
0.87 600.22265625
0.88 600.22265625
0.89 600.22265625
0.9 600.22265625
0.91 600.22265625
0.92 600.22265625
0.93 600.22265625
0.94 600.22265625
0.95 600.22265625
0.96 600.22265625
0.97 600.22265625
0.98 600.22265625
0.99 600.22265625
1 600.22265625
1.01 600.73046875
1.02 600.72265625
1.03 600.72265625
1.04 627.2734375
1.05 669.59375
1.06 737.59375
1.07 718.4140625
1.08 718.4140625
1.09 718.609375
1.1 718.609375
1.11 718.609375
1.12 718.609375
1.13 718.609375
1.14 718.609375
1.15 718.609375
1.16 718.609375
1.17 718.609375
1.18 718.609375
1.19 718.609375
1.2 718.609375
1.21 718.609375
1.22 718.609375
1.23 718.609375
1.24 718.609375
1.25 718.609375
1.26 718.609375
1.27 718.609375
1.28 718.609375
1.29 718.609375
1.3 718.609375
1.31 718.609375
1.32 718.609375
1.33 718.609375
1.34 718.609375
1.35 718.609375
1.36 718.609375
1.37 718.609375
1.38 718.609375
1.39 718.609375
1.4 718.609375
1.41 718.609375
1.42 718.609375
1.43 718.609375
1.44 718.609375
1.45 718.609375
1.46 718.609375
1.47 718.609375
1.48 718.609375
1.49 718.609375
1.5 718.609375
1.51 718.609375
1.52 721.4609375
1.53 722.46484375
1.54 723.21875
1.55 723.21875
1.56 723.8046875
1.57 725.83203125
1.58 725.83203125
1.59 725.83203125
1.6 725.83203125
1.61 725.83203125
1.62 726.0546875
1.63 726.0546875
1.64 726.0546875
1.65 726.0546875
1.66 726.0546875
1.67 726.0546875
1.68 726.0546875
1.69 726.0546875
1.7 756.234375
1.71 756.734375
1.72 772.75
1.73 880.265625
1.74 908.3671875
1.75 908.3671875
1.76 908.3671875
1.77 844.51953125
1.78 903.00390625
1.79 1086.56640625
1.8 1220.11328125
1.81 1353.91796875
1.82 1359.33203125
1.83 1359.33203125
1.84 1359.33203125
1.85 1359.33203125
1.86 1359.33203125
1.87 1359.33203125
1.88 1359.33203125
1.89 1359.33203125
1.9 1359.33203125
1.91 1359.33203125
1.92 1359.33203125
1.93 1359.33203125
1.94 1359.33203125
1.95 1359.33203125
1.96 1359.33203125
1.97 1359.33203125
1.98 1132.0625
1.99 783.3515625
2 795.96484375
2.01 797.30859375
2.02 882.1171875
2.03 905.97265625
2.04 905.97265625
2.05 905.97265625
2.06 814.328125
2.07 810.0625
2.08 820.890625
2.09 883.0234375
2.1 883.0234375
2.11 883.0234375
2.12 883.0234375
2.13 883.16796875
};
\addplot [, color1, dotted, forget plot]
table {%
0 275.1796875
0.01 275.375
0.02 277.6953125
0.03 277.6953125
0.04 277.6953125
0.05 277.6953125
0.06 277.6953125
0.07 277.6953125
0.08 277.6953125
0.09 277.6953125
0.1 277.6953125
0.11 277.6953125
0.12 277.6953125
0.13 277.6953125
0.14 277.6953125
0.15 277.6953125
0.16 277.6953125
0.17 277.6953125
0.18 277.6953125
0.19 277.6953125
0.2 277.6953125
0.21 277.6953125
0.22 277.6953125
0.23 277.6953125
0.24 277.6953125
0.25 277.6953125
0.26 277.6953125
0.27 277.6953125
0.28 277.6953125
0.29 277.6953125
0.3 277.6953125
0.31 277.6953125
0.32 277.6953125
0.33 277.6953125
0.34 277.6953125
0.35 277.6953125
0.36 277.6953125
0.37 277.6953125
0.38 277.6953125
0.39 277.6953125
0.4 277.6953125
0.41 277.6953125
0.42 277.6953125
0.43 277.6953125
0.44 277.6953125
0.45 307.48828125
0.46 335.33203125
0.47 330.3125
0.48 329.96875
0.49 359.6171875
0.5 369.53515625
0.51 394.80078125
0.52 406.60546875
0.53 424.65234375
0.54 443.73046875
0.55 476.44140625
0.56 542.90625
0.57 600.17578125
0.58 600.17578125
0.59 600.17578125
0.6 600.17578125
0.61 600.17578125
0.62 600.17578125
0.63 600.17578125
0.64 600.17578125
0.65 600.17578125
0.66 600.17578125
0.67 600.17578125
0.68 600.17578125
0.69 600.17578125
0.7 600.17578125
0.71 600.17578125
0.72 600.17578125
0.73 600.17578125
0.74 600.17578125
0.75 600.17578125
0.76 600.17578125
0.77 600.17578125
0.78 600.17578125
0.79 600.17578125
0.8 600.17578125
0.81 600.17578125
0.82 600.17578125
0.83 600.17578125
0.84 600.17578125
0.85 600.17578125
0.86 600.17578125
0.87 600.17578125
0.88 600.17578125
0.89 600.17578125
0.9 600.17578125
0.91 600.17578125
0.92 600.17578125
0.93 600.17578125
0.94 600.17578125
0.95 600.17578125
0.96 600.17578125
0.97 600.17578125
0.98 600.17578125
0.99 600.17578125
1 600.17578125
1.01 600.17578125
1.02 600.68359375
1.03 600.67578125
1.04 616.9140625
1.05 646.578125
1.06 711.55078125
1.07 774.92578125
1.08 718.359375
1.09 718.5546875
1.1 718.5546875
1.11 718.5546875
1.12 718.5546875
1.13 718.5546875
1.14 718.5546875
1.15 718.5546875
1.16 718.5546875
1.17 718.5546875
1.18 718.5546875
1.19 718.5546875
1.2 718.5546875
1.21 718.5546875
1.22 718.5546875
1.23 718.5546875
1.24 718.5546875
1.25 718.5546875
1.26 718.5546875
1.27 718.5546875
1.28 718.5546875
1.29 718.5546875
1.3 718.5546875
1.31 718.5546875
1.32 718.5546875
1.33 718.5546875
1.34 718.5546875
1.35 718.5546875
1.36 718.5546875
1.37 718.5546875
1.38 718.5546875
1.39 718.5546875
1.4 718.5546875
1.41 718.5546875
1.42 718.5546875
1.43 718.5546875
1.44 718.5546875
1.45 718.5546875
1.46 718.5546875
1.47 718.5546875
1.48 718.5546875
1.49 718.5546875
1.5 718.5546875
1.51 718.5546875
1.52 718.5546875
1.53 721.84765625
1.54 723.1640625
1.55 723.1640625
1.56 723.1640625
1.57 724.97265625
1.58 725.78125
1.59 725.78125
1.6 725.78125
1.61 725.78125
1.62 725.78125
1.63 727.91015625
1.64 727.91015625
1.65 727.91015625
1.66 727.91015625
1.67 727.91015625
1.68 727.91015625
1.69 727.91015625
1.7 730.0234375
1.71 780.16015625
1.72 780.16015625
1.73 787.66796875
1.74 914.51171875
1.75 914.51171875
1.76 914.51171875
1.77 914.51171875
1.78 800.7890625
1.79 994.1484375
1.8 1145.2265625
1.81 1278.7734375
1.82 1362.046875
1.83 1362.046875
1.84 1362.046875
1.85 1362.046875
1.86 1362.046875
1.87 1362.046875
1.88 1362.046875
1.89 1362.046875
1.9 1362.046875
1.91 1362.046875
1.92 1362.046875
1.93 1362.046875
1.94 1362.046875
1.95 1362.046875
1.96 1362.046875
1.97 1362.046875
1.98 1362.046875
1.99 998.48828125
2 795.42578125
2.01 800.88671875
2.02 830.56640625
2.03 861.86328125
2.04 909.55859375
2.05 909.55859375
2.06 909.55859375
2.07 812.875
2.08 814.6796875
2.09 853.09375
2.1 886.609375
2.11 886.609375
2.12 886.609375
2.13 886.609375
2.14 886.75390625
};
\addplot [, color1, dotted, forget plot]
table {%
0 275.37890625
0.01 275.58203125
0.02 278.00390625
0.03 278.00390625
0.04 278.00390625
0.05 278.00390625
0.06 278.00390625
0.07 278.00390625
0.08 278.00390625
0.09 278.00390625
0.1 278.00390625
0.11 278.00390625
0.12 278.00390625
0.13 278.00390625
0.14 278.00390625
0.15 278.00390625
0.16 278.00390625
0.17 278.00390625
0.18 278.00390625
0.19 278.00390625
0.2 278.00390625
0.21 278.00390625
0.22 278.00390625
0.23 278.00390625
0.24 278.00390625
0.25 278.00390625
0.26 278.00390625
0.27 278.00390625
0.28 278.00390625
0.29 278.00390625
0.3 278.00390625
0.31 278.00390625
0.32 278.00390625
0.33 278.00390625
0.34 278.00390625
0.35 278.00390625
0.36 278.00390625
0.37 278.00390625
0.38 278.00390625
0.39 278.00390625
0.4 278.00390625
0.41 278.00390625
0.42 278.00390625
0.43 278.00390625
0.44 278.00390625
0.45 303.9375
0.46 331.78125
0.47 327.2578125
0.48 326.91015625
0.49 356.81640625
0.5 365.953125
0.51 393.5390625
0.52 402.51953125
0.53 424.69140625
0.54 440.15625
0.55 466.73046875
0.56 533.203125
0.57 601.203125
0.58 600.46484375
0.59 600.46484375
0.6 600.46484375
0.61 600.46484375
0.62 600.46484375
0.63 600.46484375
0.64 600.46484375
0.65 600.46484375
0.66 600.46484375
0.67 600.46484375
0.68 600.46484375
0.69 600.46484375
0.7 600.46484375
0.71 600.46484375
0.72 600.46484375
0.73 600.46484375
0.74 600.46484375
0.75 600.46484375
0.76 600.46484375
0.77 600.46484375
0.78 600.46484375
0.79 600.46484375
0.8 600.46484375
0.81 600.46484375
0.82 600.46484375
0.83 600.46484375
0.84 600.46484375
0.85 600.46484375
0.86 600.46484375
0.87 600.46484375
0.88 600.46484375
0.89 600.46484375
0.9 600.46484375
0.91 600.46484375
0.92 600.46484375
0.93 600.46484375
0.94 600.46484375
0.95 600.46484375
0.96 600.46484375
0.97 600.46484375
0.98 600.46484375
0.99 600.46484375
1 600.46484375
1.01 600.46484375
1.02 600.97265625
1.03 600.96484375
1.04 616.171875
1.05 642.8671875
1.06 707.83984375
1.07 773.21484375
1.08 718.65625
1.09 718.84765625
1.1 718.84765625
1.11 718.84765625
1.12 718.84765625
1.13 718.84765625
1.14 718.84765625
1.15 718.84765625
1.16 718.84765625
1.17 718.84765625
1.18 718.84765625
1.19 718.84765625
1.2 718.84765625
1.21 718.84765625
1.22 718.84765625
1.23 718.84765625
1.24 718.84765625
1.25 718.84765625
1.26 718.84765625
1.27 718.84765625
1.28 718.84765625
1.29 718.84765625
1.3 718.84765625
1.31 718.84765625
1.32 718.84765625
1.33 718.84765625
1.34 718.84765625
1.35 718.84765625
1.36 718.84765625
1.37 718.84765625
1.38 718.84765625
1.39 718.84765625
1.4 718.84765625
1.41 718.84765625
1.42 718.84765625
1.43 718.84765625
1.44 718.84765625
1.45 718.84765625
1.46 718.84765625
1.47 718.84765625
1.48 718.84765625
1.49 718.84765625
1.5 718.84765625
1.51 718.84765625
1.52 718.84765625
1.53 721.7109375
1.54 723.56640625
1.55 723.56640625
1.56 723.56640625
1.57 725.16015625
1.58 726.1796875
1.59 726.1796875
1.6 726.1796875
1.61 726.1796875
1.62 726.1796875
1.63 727.1796875
1.64 727.1796875
1.65 727.1796875
1.66 727.1796875
1.67 727.1796875
1.68 727.1796875
1.69 727.1796875
1.7 730.6015625
1.71 780.8046875
1.72 780.8046875
1.73 786.5625
1.74 914.80859375
1.75 914.80859375
1.76 914.80859375
1.77 914.80859375
1.78 794.12109375
1.79 988.76953125
1.8 1146.03515625
1.81 1279.83984375
1.82 1362.85546875
1.83 1362.85546875
1.84 1362.85546875
1.85 1362.85546875
1.86 1362.85546875
1.87 1362.85546875
1.88 1362.85546875
1.89 1362.85546875
1.9 1362.85546875
1.91 1362.85546875
1.92 1362.85546875
1.93 1362.85546875
1.94 1362.85546875
1.95 1362.85546875
1.96 1362.85546875
1.97 1362.85546875
1.98 1362.85546875
1.99 999.29296875
2 795.9765625
2.01 797.80859375
2.02 830.3125
2.03 858.1328125
2.04 907.375
2.05 907.375
2.06 907.375
2.07 805.796875
2.08 805.796875
2.09 830.546875
2.1 879.015625
2.11 879.015625
2.12 879.015625
2.13 879.015625
2.14 879.15234375
};
\addplot [, color1, dotted, forget plot]
table {%
0 275.92578125
0.01 276.125
0.02 278.48046875
0.03 278.48046875
0.04 278.48046875
0.05 278.48046875
0.06 278.48046875
0.07 278.48046875
0.08 278.48046875
0.09 278.48046875
0.1 278.48046875
0.11 278.48046875
0.12 278.48046875
0.13 278.48046875
0.14 278.48046875
0.15 278.48046875
0.16 278.48046875
0.17 278.48046875
0.18 278.48046875
0.19 278.48046875
0.2 278.48046875
0.21 278.48046875
0.22 278.48046875
0.23 278.48046875
0.24 278.48046875
0.25 278.48046875
0.26 278.48046875
0.27 278.48046875
0.28 278.48046875
0.29 278.48046875
0.3 278.48046875
0.31 278.48046875
0.32 278.48046875
0.33 278.48046875
0.34 278.48046875
0.35 278.48046875
0.36 278.48046875
0.37 278.48046875
0.38 278.48046875
0.39 278.48046875
0.4 278.48046875
0.41 278.48046875
0.42 278.48046875
0.43 278.48046875
0.44 278.48046875
0.45 301.06640625
0.46 329.68359375
0.47 325.2109375
0.48 323.3125
0.49 354.5078125
0.5 366.2421875
0.51 392.01953125
0.52 401.25
0.53 425.2265625
0.54 438.62890625
0.55 463.12890625
0.56 529.73828125
0.57 595.73828125
0.58 600.9921875
0.59 600.9921875
0.6 600.9921875
0.61 600.9921875
0.62 600.9921875
0.63 600.9921875
0.64 600.9921875
0.65 600.9921875
0.66 600.9921875
0.67 600.9921875
0.68 600.9921875
0.69 600.9921875
0.7 600.9921875
0.71 600.9921875
0.72 600.9921875
0.73 600.9921875
0.74 600.9921875
0.75 600.9921875
0.76 600.9921875
0.77 600.9921875
0.78 600.9921875
0.79 600.9921875
0.8 600.9921875
0.81 600.9921875
0.82 600.9921875
0.83 600.9921875
0.84 600.9921875
0.85 600.9921875
0.86 600.9921875
0.87 600.9921875
0.88 600.9921875
0.89 600.9921875
0.9 600.9921875
0.91 600.9921875
0.92 600.9921875
0.93 600.9921875
0.94 600.9921875
0.95 600.9921875
0.96 600.9921875
0.97 600.9921875
0.98 600.9921875
0.99 600.9921875
1 600.9921875
1.01 600.9921875
1.02 601.5
1.03 601.4921875
1.04 620.82421875
1.05 653.39453125
1.06 720.36328125
1.07 740.0234375
1.08 719.18359375
1.09 719.37890625
1.1 719.37890625
1.11 719.37890625
1.12 719.37890625
1.13 719.37890625
1.14 719.37890625
1.15 719.37890625
1.16 719.37890625
1.17 719.37890625
1.18 719.37890625
1.19 719.37890625
1.2 719.37890625
1.21 719.37890625
1.22 719.37890625
1.23 719.37890625
1.24 719.37890625
1.25 719.37890625
1.26 719.37890625
1.27 719.37890625
1.28 719.37890625
1.29 719.37890625
1.3 719.37890625
1.31 719.37890625
1.32 719.37890625
1.33 719.37890625
1.34 719.37890625
1.35 719.37890625
1.36 719.37890625
1.37 719.37890625
1.38 719.37890625
1.39 719.37890625
1.4 719.37890625
1.41 719.37890625
1.42 719.37890625
1.43 719.37890625
1.44 719.37890625
1.45 719.37890625
1.46 719.37890625
1.47 719.37890625
1.48 719.37890625
1.49 719.37890625
1.5 719.37890625
1.51 719.37890625
1.52 719.3984375
1.53 722.53515625
1.54 723.9140625
1.55 723.9140625
1.56 723.9140625
1.57 726.3046875
1.58 726.52734375
1.59 726.52734375
1.6 726.52734375
1.61 726.52734375
1.62 726.52734375
1.63 726.75390625
1.64 728.91796875
1.65 728.91796875
1.66 728.91796875
1.67 728.91796875
1.68 728.91796875
1.69 728.91796875
1.7 731.0625
1.71 757.63671875
1.72 765.41015625
1.73 799.8515625
1.74 909.1640625
1.75 909.1640625
1.76 909.1640625
1.77 909.1640625
1.78 812.28125
1.79 1007.4453125
1.8 1157.234375
1.81 1291.5546875
1.82 1360.1328125
1.83 1360.1328125
1.84 1360.1328125
1.85 1360.1328125
1.86 1360.1328125
1.87 1360.1328125
1.88 1360.1328125
1.89 1360.1328125
1.9 1360.1328125
1.91 1360.1328125
1.92 1360.1328125
1.93 1360.1328125
1.94 1360.1328125
1.95 1360.1328125
1.96 1360.1328125
1.97 1360.1328125
1.98 1314.15234375
1.99 956.15234375
2 795.1953125
2.01 797.44140625
2.02 836.42578125
2.03 872.1328125
2.04 906.9375
2.05 906.9375
2.06 906.9375
2.07 811.03125
2.08 812.8359375
2.09 853.05078125
2.1 883.98828125
2.11 883.98828125
2.12 883.98828125
2.13 883.98828125
2.14 884.13671875
};
\addplot [, color1, dotted, forget plot]
table {%
0 275.546875
0.01 275.74609375
0.02 278.03515625
0.03 278.03515625
0.04 278.03515625
0.05 278.03515625
0.06 278.03515625
0.07 278.03515625
0.08 278.03515625
0.09 278.03515625
0.1 278.03515625
0.11 278.03515625
0.12 278.03515625
0.13 278.03515625
0.14 278.03515625
0.15 278.03515625
0.16 278.03515625
0.17 278.03515625
0.18 278.03515625
0.19 278.03515625
0.2 278.03515625
0.21 278.03515625
0.22 278.03515625
0.23 278.03515625
0.24 278.03515625
0.25 278.03515625
0.26 278.03515625
0.27 278.03515625
0.28 278.03515625
0.29 278.03515625
0.3 278.03515625
0.31 278.03515625
0.32 278.03515625
0.33 278.03515625
0.34 278.03515625
0.35 278.03515625
0.36 278.03515625
0.37 278.03515625
0.38 278.03515625
0.39 278.03515625
0.4 278.03515625
0.41 278.03515625
0.42 278.03515625
0.43 278.03515625
0.44 278.03515625
0.45 306.02734375
0.46 333.87109375
0.47 329.68359375
0.48 329.3359375
0.49 358.984375
0.5 367.62109375
0.51 395.20703125
0.52 404.6953125
0.53 424.8046875
0.54 441.30078125
0.55 468.83984375
0.56 535.31640625
0.57 601.31640625
0.58 600.57421875
0.59 600.57421875
0.6 600.57421875
0.61 600.57421875
0.62 600.57421875
0.63 600.57421875
0.64 600.57421875
0.65 600.57421875
0.66 600.57421875
0.67 600.57421875
0.68 600.57421875
0.69 600.57421875
0.7 600.57421875
0.71 600.57421875
0.72 600.57421875
0.73 600.57421875
0.74 600.57421875
0.75 600.57421875
0.76 600.57421875
0.77 600.57421875
0.78 600.57421875
0.79 600.57421875
0.8 600.57421875
0.81 600.57421875
0.82 600.57421875
0.83 600.57421875
0.84 600.57421875
0.85 600.57421875
0.86 600.57421875
0.87 600.57421875
0.88 600.57421875
0.89 600.57421875
0.9 600.57421875
0.91 600.57421875
0.92 600.57421875
0.93 600.57421875
0.94 600.57421875
0.95 600.57421875
0.96 600.57421875
0.97 600.57421875
0.98 600.57421875
0.99 600.57421875
1 600.57421875
1.01 600.57421875
1.02 601.08203125
1.03 601.07421875
1.04 621.953125
1.05 656.9765625
1.06 723.9453125
1.07 729.765625
1.08 718.765625
1.09 718.95703125
1.1 718.95703125
1.11 718.95703125
1.12 718.95703125
1.13 718.95703125
1.14 718.95703125
1.15 718.95703125
1.16 718.95703125
1.17 718.95703125
1.18 718.95703125
1.19 718.95703125
1.2 718.95703125
1.21 718.95703125
1.22 718.95703125
1.23 718.95703125
1.24 718.95703125
1.25 718.95703125
1.26 718.95703125
1.27 718.95703125
1.28 718.95703125
1.29 718.95703125
1.3 718.95703125
1.31 718.95703125
1.32 718.95703125
1.33 718.95703125
1.34 718.95703125
1.35 718.95703125
1.36 718.95703125
1.37 718.95703125
1.38 718.95703125
1.39 718.95703125
1.4 718.95703125
1.41 718.95703125
1.42 718.95703125
1.43 718.95703125
1.44 718.95703125
1.45 718.95703125
1.46 718.95703125
1.47 718.95703125
1.48 718.95703125
1.49 718.95703125
1.5 718.95703125
1.51 718.95703125
1.52 719.4921875
1.53 722.171875
1.54 723.734375
1.55 723.734375
1.56 723.734375
1.57 726.09375
1.58 726.34765625
1.59 726.34765625
1.6 726.34765625
1.61 726.34765625
1.62 726.34765625
1.63 726.38671875
1.64 726.38671875
1.65 728.546875
1.66 728.546875
1.67 728.546875
1.68 728.546875
1.69 728.546875
1.7 730.8203125
1.71 757.59765625
1.72 765.58984375
1.73 816.28125
1.74 909.09375
1.75 909.09375
1.76 909.09375
1.77 909.09375
1.78 823.41015625
1.79 1019.86328125
1.8 1162.94921875
1.81 1296.49609375
1.82 1360.17578125
1.83 1360.17578125
1.84 1360.17578125
1.85 1360.17578125
1.86 1360.17578125
1.87 1360.17578125
1.88 1360.17578125
1.89 1360.17578125
1.9 1360.17578125
1.91 1360.17578125
1.92 1360.17578125
1.93 1360.17578125
1.94 1360.17578125
1.95 1360.17578125
1.96 1360.17578125
1.97 1360.17578125
1.98 1284.4140625
1.99 920.64453125
2 795.375
2.01 797.9921875
2.02 842.33203125
2.03 884.52734375
2.04 906.95703125
2.05 906.95703125
2.06 906.95703125
2.07 810.7890625
2.08 816.203125
2.09 872.40625
2.1 883.4921875
2.11 883.4921875
2.12 883.4921875
2.13 883.75390625
2.14 883.75390625
};
\addplot [, color0, dashed]
table {%
0 1514.74609375
2.14 1514.74609375
};
\addlegendentry{expensive: $1515 \pm 27$ MB}
\addplot [, color1, dashed]
table {%
0 1361.27578125
2.14 1361.27578125
};
\addlegendentry{optimized: $1361 \pm 1$ MB}
\addplot [, color2, dashed]
table {%
0 845.128125
2.14 845.128125
};
\addlegendentry{baseline: $845 \pm 2$ MB}
\end{axis}

\end{tikzpicture}

    \tikzexternaldisable
    \label{cockpit::fig:memory-benchmark-cifar10}
  \end{subfigure}
  \hfill
  \begin{subfigure}[t]{0.99\textwidth}
    \pgfkeys{/pgfplots/zmystyle/.style={ memorybenchmarkdefault,
        legend pos = north west}}
    \caption{\cifarhun \allcnnc}
    \tikzexternalenable
    % This file was created by tikzplotlib v0.9.8.
\begin{tikzpicture}

\definecolor{color0}{rgb}{0.894117647058824,0.101960784313725,0.109803921568627}
\definecolor{color1}{rgb}{0.301960784313725,0.686274509803922,0.290196078431373}
\definecolor{color2}{rgb}{0.215686274509804,0.494117647058824,0.72156862745098}

\begin{axis}[
axis line style={white!80!black},
legend style={
  fill opacity=0.8,
  draw opacity=1,
  text opacity=1,
  at={(0.97,0.03)},
  anchor=south east,
  draw=white!80!black
},
tick pos=left,
xlabel={Time [s]},
xmin=-0.2505, xmax=5.2605,
xtick style={color=gray},
ylabel={Memory [MB]},
ymin=124.1970703125, ymax=3448.2208984375,
zmystyle
]
\path [draw=color0, fill=color0, opacity=0.4]
(axis cs:0,3295.35944248751)
--(axis cs:0,3169.15149501249)
--(axis cs:5.01,3169.15149501249)
--(axis cs:5.01,3295.35944248751)
--(axis cs:5.01,3295.35944248751)
--(axis cs:0,3295.35944248751)
--cycle;

\path [draw=color1, fill=color1, opacity=0.4]
(axis cs:0,2647.07672318069)
--(axis cs:0,2554.65530806931)
--(axis cs:5.01,2554.65530806931)
--(axis cs:5.01,2647.07672318069)
--(axis cs:5.01,2647.07672318069)
--(axis cs:0,2647.07672318069)
--cycle;

\path [draw=color2, fill=color2, opacity=0.4]
(axis cs:0,1982.25571022163)
--(axis cs:0,1932.55210227837)
--(axis cs:5.01,1932.55210227837)
--(axis cs:5.01,1982.25571022163)
--(axis cs:5.01,1982.25571022163)
--(axis cs:0,1982.25571022163)
--cycle;

\addplot [, color0, dotted, forget plot]
table {%
0 275.3828125
0.02 277.8828125
0.04 277.8828125
0.06 277.8828125
0.08 277.8828125
0.1 277.8828125
0.12 277.8828125
0.14 277.8828125
0.16 277.8828125
0.18 277.8828125
0.2 277.8828125
0.22 277.8828125
0.24 277.8828125
0.26 277.8828125
0.28 277.8828125
0.3 277.8828125
0.32 277.8828125
0.34 277.8828125
0.36 277.8828125
0.38 277.8828125
0.4 277.8828125
0.42 277.8828125
0.44 277.8828125
0.46 290.6484375
0.48 356.6484375
0.5 425.2265625
0.52 481.6875
0.54 538.1484375
0.56 435.328125
0.58 491.015625
0.6 546.9609375
0.62 461.5625
0.64 494.84765625
0.66 425.92578125
0.68 425.92578125
0.7 425.92578125
0.72 425.92578125
0.74 425.92578125
0.76 425.92578125
0.78 425.92578125
0.8 425.92578125
0.82 425.92578125
0.84 425.92578125
0.86 425.92578125
0.88 425.92578125
0.9 425.92578125
0.92 425.92578125
0.94 425.92578125
0.96 425.92578125
0.98 425.92578125
1 425.92578125
1.02 425.92578125
1.04 425.92578125
1.06 425.92578125
1.08 428.24609375
1.1 489.60546875
1.12 558.18359375
1.14 617.22265625
1.16 673.42578125
1.18 608.578125
1.2 626.5234375
1.22 680.921875
1.24 579.0234375
1.26 713.203125
1.28 572.01953125
1.3 574.0390625
1.32 574.0390625
1.34 574.0390625
1.36 574.0390625
1.38 574.0390625
1.4 574.0390625
1.42 574.0390625
1.44 574.0390625
1.46 574.0390625
1.48 574.0390625
1.5 574.0390625
1.52 574.0390625
1.54 574.0390625
1.56 574.0390625
1.58 574.0390625
1.6 574.0390625
1.62 574.0390625
1.64 574.0390625
1.66 574.0390625
1.68 574.0390625
1.7 574.0390625
1.72 585.640625
1.74 612.80859375
1.76 634.0546875
1.78 635.9453125
1.8 642.09375
1.82 644.546875
1.84 644.546875
1.86 644.546875
1.88 645.8984375
1.9 645.8984375
1.92 645.8984375
1.94 645.8984375
1.96 645.8984375
1.98 645.8984375
2 645.8984375
2.02 806.98828125
2.04 953.1328125
2.06 1070.6875
2.08 1101.625
2.1 1133.3359375
2.12 1064.84375
2.14 1196.5078125
2.16 1348.875
2.18 1298.02734375
2.2 1425.40234375
2.22 1455.82421875
2.24 1445.640625
2.26 1618.875
2.28 1648.0078125
2.3 1663.4765625
2.32 1663.4765625
2.34 1704.953125
2.36 1759.09375
2.38 1752.8984375
2.4 1825.625
2.42 1887.75
2.44 1902.9609375
2.46 1902.9609375
2.48 1979.62890625
2.5 2146.17578125
2.52 2349.84765625
2.54 2354.87109375
2.56 2256.7734375
2.58 2311.15234375
2.6 2337.19140625
2.62 2396.44921875
2.64 2518.91015625
2.66 2643.43359375
2.68 2875.20703125
2.7 2634.953125
2.72 2550.4609375
2.74 2569.5390625
2.76 2589.1328125
2.78 2561.5859375
2.8 2667.7109375
2.82 2667.7109375
2.84 2667.7109375
2.86 2667.7109375
2.88 2636.63671875
2.9 2680.72265625
2.92 2725.58203125
2.94 2770.18359375
2.96 2815.04296875
2.98 2859.38671875
3 2904.50390625
3.02 2943.94921875
3.04 3203.30859375
3.06 2826.2578125
3.08 2716.265625
3.1 2731.734375
3.12 2797.96484375
3.14 2797.96484375
3.16 2750.05078125
3.18 2797.2265625
3.2 2841.0546875
3.22 2886.6875
3.24 3007.859375
3.26 2749.03515625
3.28 2817.87109375
3.3 2851.15625
3.32 2953.24609375
3.34 2915.08984375
3.36 2986.24609375
3.38 3023.62890625
3.4 2969.9296875
3.42 2919.8046875
3.44 2836.7265625
3.46 2871.015625
3.48 2906.078125
3.5 2856.78515625
3.52 2962.66796875
3.54 3040.78515625
3.56 3040.78515625
3.58 3040.78515625
3.6 3002.9453125
3.62 2948.33203125
3.64 2958.64453125
3.66 2968.69921875
3.68 2978.49609375
3.7 2988.80859375
3.72 2999.12109375
3.74 3009.17578125
3.76 3019.23046875
3.78 3071.56640625
3.8 2819.07421875
3.82 2566.43359375
3.84 2508.49609375
3.86 2355.671875
3.88 2398.46875
3.9 2398.46875
3.92 2317.33203125
3.94 2317.33203125
3.96 2317.33203125
3.98 2317.58203125
4 2317.58203125
4.02 2317.58203125
4.04 2317.58203125
4.06 2317.58203125
4.08 2093.01953125
4.1 2317.05859375
4.12 2317.05859375
4.14 2317.05859375
4.16 2317.05859375
4.18 2317.05859375
4.2 2317.05859375
4.22 2317.05859375
4.24 2317.05859375
4.26 2317.05859375
4.28 2061.17578125
4.3 1812.51171875
4.32 1993.49609375
4.34 1993.49609375
4.36 1993.49609375
4.38 1993.49609375
4.4 1993.49609375
4.42 1993.49609375
4.44 1993.49609375
4.46 1993.49609375
4.48 1993.49609375
4.5 1690.5546875
4.52 1511.12109375
4.54 1669.16015625
4.56 1669.16015625
4.58 1669.16015625
4.6 1669.16015625
4.62 1669.16015625
4.64 1669.16015625
4.66 1669.16015625
4.68 1669.16015625
4.7 1669.16015625
4.72 1345.27734375
4.74 1021.2734375
4.76 985.26953125
};
\addplot [, color0, dotted, forget plot]
table {%
0 276.03515625
0.02 278.52734375
0.04 278.52734375
0.06 278.52734375
0.08 278.52734375
0.1 278.52734375
0.12 278.52734375
0.14 278.52734375
0.16 278.52734375
0.18 278.52734375
0.2 278.52734375
0.22 278.52734375
0.24 278.52734375
0.26 278.52734375
0.28 278.52734375
0.3 278.52734375
0.32 278.52734375
0.34 278.52734375
0.36 278.52734375
0.38 278.52734375
0.4 278.52734375
0.42 278.52734375
0.44 278.52734375
0.46 292.33203125
0.48 358.58984375
0.5 426.65234375
0.52 482.59765625
0.54 538.54296875
0.56 435.1953125
0.58 490.3671875
0.6 545.5390625
0.62 460.19921875
0.64 495.484375
0.66 426.5625
0.68 426.5625
0.7 426.5625
0.72 426.5625
0.74 426.5625
0.76 426.5625
0.78 426.5625
0.8 426.5625
0.82 426.5625
0.84 426.5625
0.86 426.5625
0.88 426.5625
0.9 426.5625
0.92 426.5625
0.94 426.5625
0.96 426.5625
0.98 426.5625
1 426.5625
1.02 426.5625
1.04 426.5625
1.06 426.5625
1.08 429.65625
1.1 495.3984375
1.12 563.203125
1.14 620.1796875
1.16 675.609375
1.18 609.21484375
1.2 627.41796875
1.22 682.07421875
1.24 581.66015625
1.26 713.83984375
1.28 572.640625
1.3 574.6640625
1.32 574.6640625
1.34 574.6640625
1.36 574.6640625
1.38 574.6640625
1.4 574.6640625
1.42 574.6640625
1.44 574.6640625
1.46 574.6640625
1.48 574.6640625
1.5 574.6640625
1.52 574.6640625
1.54 574.6640625
1.56 574.6640625
1.58 574.6640625
1.6 574.6640625
1.62 574.6640625
1.64 574.6640625
1.66 574.6640625
1.68 574.6640625
1.7 574.6640625
1.72 584.97265625
1.74 612.41796875
1.76 634.51171875
1.78 636.54296875
1.8 642.77734375
1.82 645.1484375
1.84 645.1484375
1.86 645.1484375
1.88 646.94921875
1.9 646.94921875
1.92 646.94921875
1.94 646.94921875
1.96 646.94921875
1.98 646.94921875
2 646.94921875
2.02 814.54296875
2.04 953.86328125
2.06 1066.78125
2.08 1097.9765625
2.1 1129.6875
2.12 1120.54296875
2.14 1145.41015625
2.16 1312.98828125
2.18 1372.02734375
2.2 1399.05078125
2.22 1451.64453125
2.24 1462.90234375
2.26 1604.39453125
2.28 1643.06640625
2.3 1664.20703125
2.32 1664.20703125
2.34 1705.93359375
2.36 1760.07421875
2.38 1753.8828125
2.4 1822.14453125
2.42 1877.66796875
2.44 1903.68359375
2.46 1903.68359375
2.48 1949.49609375
2.5 2116.30078125
2.52 2308.37109375
2.54 2477.66015625
2.56 2250.1015625
2.58 2283.6015625
2.6 2337.73828125
2.62 2381.78515625
2.64 2506.05078125
2.66 2630.05859375
2.68 2849.45703125
2.7 2613.0703125
2.72 2549.4609375
2.74 2569.3125
2.76 2588.1328125
2.78 2607.984375
2.8 2668.7734375
2.82 2668.7734375
2.84 2668.7734375
2.86 2668.7734375
2.88 2635.37890625
2.9 2680.49609375
2.92 2725.61328125
2.94 2770.47265625
2.96 2814.30078125
2.98 2858.90234375
3 2904.27734375
3.02 2944.49609375
3.04 3194.31640625
3.06 2818.0390625
3.08 2717.0703125
3.1 2732.28125
3.12 2799.02734375
3.14 2799.02734375
3.16 2754.71875
3.18 2799.3203125
3.2 2843.6640625
3.22 2889.296875
3.24 3006.859375
3.26 2753.19140625
3.28 2835.17578125
3.3 2869.62109375
3.32 2971.7109375
3.34 2928.3984375
3.36 3003.6796875
3.38 3039.7734375
3.4 2973.18359375
3.42 2927.04296875
3.44 2852.61328125
3.46 2886.90234375
3.48 2921.96484375
3.5 2840.875
3.52 2866.71875
3.54 2987.375
3.56 2987.375
3.58 2987.375
3.6 2987.375
3.62 2893.375
3.64 2903.171875
3.66 2913.2265625
3.68 2923.5390625
3.7 2933.8515625
3.72 2944.1640625
3.74 2953.9609375
3.76 2964.015625
3.78 2972.0078125
3.8 2717.7109375
3.82 2643.0234375
3.84 2551.3203125
3.86 2263.34375
3.88 2344.796875
3.9 2344.796875
3.92 2246.13671875
3.94 2263.92578125
3.96 2263.92578125
3.98 2235.04296875
4 2263.66015625
4.02 2263.66015625
4.04 2263.66015625
4.06 2263.66015625
4.08 1987.79296875
4.1 2254.11328125
4.12 2264.16796875
4.14 2264.16796875
4.16 2264.16796875
4.18 2264.16796875
4.2 2264.16796875
4.22 2264.16796875
4.24 2264.16796875
4.26 2264.16796875
4.28 2150.640625
4.3 1708.83203125
4.32 1939.83203125
4.34 1939.83203125
4.36 1939.83203125
4.38 1939.83203125
4.4 1939.83203125
4.42 1939.83203125
4.44 1939.83203125
4.46 1939.83203125
4.48 1939.83203125
4.5 1750.5390625
4.52 1405.89453125
4.54 1616.01171875
4.56 1616.01171875
4.58 1616.01171875
4.6 1616.01171875
4.62 1616.01171875
4.64 1616.01171875
4.66 1616.01171875
4.68 1616.01171875
4.7 1616.01171875
4.72 1464.12890625
4.74 968.125
4.76 932.12109375
4.78 932.14453125
};
\addplot [, color0, dotted, forget plot]
table {%
0 275.484375
0.02 277.95703125
0.04 277.95703125
0.06 277.95703125
0.08 277.95703125
0.1 277.95703125
0.12 277.95703125
0.14 277.95703125
0.16 277.95703125
0.18 277.95703125
0.2 277.95703125
0.22 277.95703125
0.24 277.95703125
0.26 277.95703125
0.28 277.95703125
0.3 277.95703125
0.32 277.95703125
0.34 277.95703125
0.36 277.95703125
0.38 277.95703125
0.4 277.95703125
0.42 277.95703125
0.44 277.95703125
0.46 287.109375
0.48 352.8515625
0.5 420.3984375
0.52 477.1171875
0.54 532.8046875
0.56 429.41796875
0.58 485.87890625
0.6 541.56640625
0.62 451.58203125
0.64 532.75390625
0.66 425.94921875
0.68 425.94921875
0.7 425.94921875
0.72 425.94921875
0.74 425.94921875
0.76 425.94921875
0.78 425.94921875
0.8 425.94921875
0.82 425.94921875
0.84 425.94921875
0.86 425.94921875
0.88 425.94921875
0.9 425.94921875
0.92 425.94921875
0.94 425.94921875
0.96 425.94921875
0.98 425.94921875
1 425.94921875
1.02 425.94921875
1.04 425.94921875
1.06 425.94921875
1.08 428.52734375
1.1 490.14453125
1.12 558.46484375
1.14 617.50390625
1.16 673.70703125
1.18 608.6015625
1.2 627.3203125
1.22 682.4921875
1.24 583.046875
1.26 717.2265625
1.28 572.03515625
1.3 574.05859375
1.32 574.05859375
1.34 574.05859375
1.36 574.05859375
1.38 574.05859375
1.4 574.05859375
1.42 574.05859375
1.44 574.05859375
1.46 574.05859375
1.48 574.05859375
1.5 574.05859375
1.52 574.05859375
1.54 574.05859375
1.56 574.05859375
1.58 574.05859375
1.6 574.05859375
1.62 574.05859375
1.64 574.05859375
1.66 574.05859375
1.68 574.05859375
1.7 574.05859375
1.72 590.8125
1.74 616.6953125
1.76 634.0703125
1.78 635.85546875
1.8 642.28125
1.82 644.4609375
1.84 644.4609375
1.86 644.4609375
1.88 646.68359375
1.9 646.68359375
1.92 646.68359375
1.94 646.68359375
1.96 646.68359375
1.98 646.68359375
2 646.68359375
2.02 768.88671875
2.04 953.0703125
2.06 1071.91796875
2.08 1102.85546875
2.1 1134.56640625
2.12 1069.17578125
2.14 1215.79296875
2.16 1350.37109375
2.18 1279.15234375
2.2 1403.375
2.22 1432.5078125
2.24 1426.05078125
2.26 1595.4296875
2.28 1624.8203125
2.3 1639
2.32 1639
2.34 1680.46875
2.36 1746.46875
2.38 1747.69140625
2.4 1816.21875
2.42 1876.16015625
2.44 1895.50390625
2.46 1921.39453125
2.48 1994.13671875
2.5 2160.16796875
2.52 2359.19921875
2.54 2389.0234375
2.56 2273.18359375
2.58 2314.67578125
2.6 2355.41015625
2.62 2406.41796875
2.64 2528.62109375
2.66 2653.14453125
2.68 2871.76953125
2.7 2633.0625
2.72 2566.875
2.74 2586.46875
2.76 2606.0625
2.78 2625.3984375
2.8 2686.1875
2.82 2686.1875
2.84 2686.1875
2.86 2686.1875
2.88 2651.76171875
2.9 2695.58984375
2.92 2740.70703125
2.94 2785.05078125
2.96 2829.65234375
2.98 2874.25390625
3 2919.88671875
3.02 2962.16796875
3.04 3202.96484375
3.06 2828.4921875
3.08 2733.453125
3.1 2745.3125
3.12 2803.80859375
3.14 2811.28515625
3.16 2763.37109375
3.18 2806.1640625
3.2 2850.25
3.22 2895.625
3.24 2991.015625
3.26 2798.75
3.28 2859.33203125
3.3 2927.19140625
3.32 3002.94921875
3.34 2933.33984375
3.36 3031.30859375
3.38 3068.43359375
3.4 2983.5390625
3.42 2943.84375
3.44 2881.53125
3.46 2916.078125
3.48 2950.3671875
3.5 2923.734375
3.52 2973.44140625
3.54 3090.48828125
3.56 3090.48828125
3.58 3090.48828125
3.6 3090.48828125
3.62 2996.48828125
3.64 3006.28515625
3.66 3016.59765625
3.68 3026.91015625
3.7 3037.22265625
3.72 3047.27734375
3.74 3057.58984375
3.76 3067.64453125
3.78 3082.85546875
3.8 2833.71484375
3.82 2744.27734375
3.84 2654.17578125
3.86 2366.45703125
3.88 2448.1171875
3.9 2448.1171875
3.92 2362.3359375
3.94 2367.234375
3.96 2367.234375
3.98 2359.4921875
4 2366.96875
4.02 2366.96875
4.04 2366.96875
4.06 2366.96875
4.08 2109.921875
4.1 2366.9609375
4.12 2366.9609375
4.14 2366.9609375
4.16 2366.9609375
4.18 2366.9609375
4.2 2366.9609375
4.22 2366.9609375
4.24 2366.9609375
4.26 2366.9609375
4.28 2215.55078125
4.3 1833.28125
4.32 2043.140625
4.34 2043.140625
4.36 2043.140625
4.38 2043.140625
4.4 2043.140625
4.42 2043.140625
4.44 2043.140625
4.46 2043.140625
4.48 2043.140625
4.5 1831.2578125
4.52 1527.765625
4.54 1719.0625
4.56 1719.0625
4.58 1719.0625
4.6 1719.0625
4.62 1719.0625
4.64 1719.0625
4.66 1719.0625
4.68 1719.0625
4.7 1719.0625
4.72 1473.1796875
4.74 1071.17578125
4.76 1035.171875
};
\addplot [, color0, dotted, forget plot]
table {%
0 275.83203125
0.02 278.36328125
0.04 278.36328125
0.06 278.36328125
0.08 278.36328125
0.1 278.36328125
0.12 278.36328125
0.14 278.36328125
0.16 278.36328125
0.18 278.36328125
0.2 278.36328125
0.22 278.36328125
0.24 278.36328125
0.26 278.36328125
0.28 278.36328125
0.3 278.36328125
0.32 278.36328125
0.34 278.36328125
0.36 278.36328125
0.38 278.36328125
0.4 278.36328125
0.42 278.36328125
0.44 278.36328125
0.46 300.9296875
0.48 367.9609375
0.5 435.25
0.52 490.421875
0.54 546.3671875
0.56 443.77734375
0.58 500.49609375
0.6 556.95703125
0.62 484.0234375
0.64 425.3515625
0.66 426.3828125
0.68 426.3828125
0.7 426.3828125
0.72 426.3828125
0.74 426.3828125
0.76 426.3828125
0.78 426.3828125
0.8 426.3828125
0.82 426.3828125
0.84 426.3828125
0.86 426.3828125
0.88 426.3828125
0.9 426.3828125
0.92 426.3828125
0.94 426.3828125
0.96 426.3828125
0.98 426.3828125
1 426.3828125
1.02 426.3828125
1.04 426.3828125
1.06 426.3828125
1.08 442.3671875
1.1 508.8828125
1.12 577.203125
1.14 632.375
1.16 688.0625
1.18 585.46875
1.2 640.640625
1.22 696.328125
1.24 615.4765625
1.26 604.80078125
1.28 574.45703125
1.3 574.45703125
1.32 574.45703125
1.34 574.45703125
1.36 574.45703125
1.38 574.45703125
1.4 574.45703125
1.42 574.45703125
1.44 574.45703125
1.46 574.45703125
1.48 574.45703125
1.5 574.45703125
1.52 574.45703125
1.54 574.45703125
1.56 574.45703125
1.58 574.45703125
1.6 574.45703125
1.62 574.45703125
1.64 574.45703125
1.66 574.45703125
1.68 574.45703125
1.7 574.45703125
1.72 607.453125
1.74 631.5703125
1.76 634.62109375
1.78 636.3203125
1.8 644.00390625
1.82 644.9140625
1.84 644.9140625
1.86 646.1171875
1.88 646.1171875
1.9 646.1171875
1.92 646.1171875
1.94 646.1171875
1.96 646.1171875
1.98 646.1171875
2 686.02734375
2.02 809.77734375
2.04 977.34765625
2.06 1082.27734375
2.08 1113.73046875
2.1 1144.92578125
2.12 1132.76953125
2.14 1247.78125
2.16 1358.8984375
2.18 1343.59375
2.2 1437.171875
2.22 1465.7890625
2.24 1546.34765625
2.26 1628.07421875
2.28 1656.17578125
2.3 1663.65234375
2.32 1627.0546875
2.34 1759.265625
2.36 1717.234375
2.38 1767.09375
2.4 1852.48046875
2.42 1902.03515625
2.44 1903.16796875
2.46 1903.76953125
2.48 2016.00390625
2.5 2182.03515625
2.52 2410.71484375
2.54 2228.890625
2.56 2269.421875
2.58 2337.46484375
2.6 2337.46484375
2.62 2425.33984375
2.64 2549.86328125
2.66 2667.94140625
2.68 2938.38671875
2.7 2551.859375
2.72 2555.1171875
2.74 2574.7109375
2.76 2594.046875
2.78 2618.3203125
2.8 2668.5
2.82 2668.5
2.84 2668.5
2.86 2668.5
2.88 2647.99609375
2.9 2693.11328125
2.92 2737.97265625
2.94 2783.34765625
2.96 2827.17578125
2.98 2871.77734375
3 2916.89453125
3.02 3006.09765625
3.04 3230.48828125
3.06 2729.0078125
3.08 2720.6640625
3.1 2736.390625
3.12 2798.49609375
3.14 2798.49609375
3.16 2799.3046875
3.18 2845.1953125
3.2 2889.796875
3.22 2934.9140625
3.24 2991.03125
3.26 2834.33203125
3.28 2919.41015625
3.3 3004.76171875
3.32 3020.48828125
3.34 3030.28515625
3.36 3067.15234375
3.38 3109.43359375
3.4 2993.875
3.42 2820.953125
3.44 2916.34375
3.46 2950.375
3.48 2984.921875
3.5 3002.91796875
3.52 3108.02734375
3.54 3108.02734375
3.56 3108.02734375
3.58 3108.02734375
3.6 2903.6875
3.62 3018.92578125
3.64 3028.98046875
3.66 3039.80859375
3.68 3049.86328125
3.7 3059.91796875
3.72 3069.97265625
3.74 3080.02734375
3.76 3091.11328125
3.78 2984.484375
3.8 2680.40234375
3.82 2635.36328125
3.84 2576.25390625
3.86 2465.9453125
3.88 2465.9453125
3.9 2465.9453125
3.92 2384.546875
3.94 2384.546875
3.96 2241.96875
3.98 2384.796875
4 2384.796875
4.02 2384.796875
4.04 2384.796875
4.06 2271.33984375
4.08 2256.9140625
4.1 2384.7890625
4.12 2384.7890625
4.14 2384.7890625
4.16 2384.7890625
4.18 2384.7890625
4.2 2384.7890625
4.22 2384.7890625
4.24 2384.7890625
4.26 2384.7890625
4.28 1833.609375
4.3 1971.765625
4.32 2060.96875
4.34 2060.96875
4.36 2060.96875
4.38 2060.96875
4.4 2060.96875
4.42 2060.96875
4.44 2060.96875
4.46 2060.96875
4.48 2060.96875
4.5 1434.0234375
4.52 1664.1875
4.54 1736.6328125
4.56 1736.6328125
4.58 1736.6328125
4.6 1736.6328125
4.62 1736.6328125
4.64 1736.6328125
4.66 1736.6328125
4.68 1736.6328125
4.7 1736.6328125
4.72 1109.6875
4.74 1088.74609375
4.76 1052.7421875
};
\addplot [, color0, dotted, forget plot]
table {%
0 275.59765625
0.02 278.046875
0.04 278.046875
0.06 278.046875
0.08 278.046875
0.1 278.046875
0.12 278.046875
0.14 278.046875
0.16 278.046875
0.18 278.046875
0.2 278.046875
0.22 278.046875
0.24 278.046875
0.26 278.046875
0.28 278.046875
0.3 278.046875
0.32 278.046875
0.34 278.046875
0.36 278.046875
0.38 278.046875
0.4 278.046875
0.42 278.046875
0.44 278.046875
0.46 311.69140625
0.48 379.49609375
0.5 444.20703125
0.52 500.15234375
0.54 556.35546875
0.56 453.25
0.58 509.96875
0.6 565.9140625
0.62 509.6953125
0.64 426.0625
0.66 426.0625
0.68 426.0625
0.7 426.0625
0.72 426.0625
0.74 426.0625
0.76 426.0625
0.78 426.0625
0.8 426.0625
0.82 426.0625
0.84 426.0625
0.86 426.0625
0.88 426.0625
0.9 426.0625
0.92 426.0625
0.94 426.0625
0.96 426.0625
0.98 426.0625
1 426.0625
1.02 426.0625
1.04 426.0625
1.06 426.0625
1.08 463.703125
1.1 532.0234375
1.12 595.703125
1.14 651.6484375
1.16 708.109375
1.18 605.00390625
1.2 661.46484375
1.22 717.66796875
1.24 669.33984375
1.26 572.13671875
1.28 574.16796875
1.3 574.16796875
1.32 574.16796875
1.34 574.16796875
1.36 574.16796875
1.38 574.16796875
1.4 574.16796875
1.42 574.16796875
1.44 574.16796875
1.46 574.16796875
1.48 574.16796875
1.5 574.16796875
1.52 574.16796875
1.54 574.16796875
1.56 574.16796875
1.58 574.16796875
1.6 574.16796875
1.62 574.16796875
1.64 574.16796875
1.66 574.16796875
1.68 574.16796875
1.7 574.16796875
1.72 630.3671875
1.74 634.12890625
1.76 635.8671875
1.78 639.35546875
1.8 644.46484375
1.82 644.46484375
1.84 644.46484375
1.86 644.96875
1.88 644.96875
1.9 644.96875
1.92 644.96875
1.94 644.96875
1.96 644.96875
1.98 646.8984375
2 786.98828125
2.02 940.015625
2.04 1062.47265625
2.06 1093.66796875
2.08 1124.86328125
2.1 1155.80078125
2.12 1114.81640625
2.14 1283.68359375
2.16 1369.01953125
2.18 1376.390625
2.2 1447.8046875
2.22 1475.90625
2.24 1571.93359375
2.26 1638.70703125
2.28 1663.45703125
2.3 1663.45703125
2.32 1664.1953125
2.34 1759.0703125
2.36 1752.87109375
2.38 1810.05078125
2.4 1866.8046875
2.42 1902.97265625
2.44 1902.97265625
2.46 1918.375
2.48 2084.1484375
2.5 2253.2734375
2.52 2523.71875
2.54 2240.28515625
2.56 2311.95703125
2.58 2337.4609375
2.6 2351.859375
2.62 2475.8671875
2.64 2600.6484375
2.66 2778.796875
2.68 2795.90625
2.7 2513.08984375
2.72 2563.36328125
2.74 2582.95703125
2.76 2602.80859375
2.78 2648.64453125
2.8 2668.49609375
2.82 2668.49609375
2.84 2668.49609375
2.86 2620.40625
2.88 2669.90625
2.9 2714.5078125
2.92 2758.8515625
2.94 2803.453125
2.96 2848.5703125
2.98 2893.9453125
3 2938.2890625
3.02 3135.515625
3.04 2845.7265625
3.06 2713.18359375
3.08 2728.65234375
3.1 2779.671875
3.12 2799.0078125
3.14 2751.09375
3.16 2791.30859375
3.18 2836.16796875
3.2 2880.51171875
3.22 2967.39453125
3.24 2786.47265625
3.26 2845.25
3.28 2924.08984375
3.3 2990.671875
3.32 2913.328125
3.34 3017.7421875
3.36 3055.125
3.38 2995.53515625
3.4 2923.05859375
3.42 2866.93359375
3.44 2901.99609375
3.46 2936.28515625
3.48 2945.19921875
3.5 2948.7890625
3.52 3078.2109375
3.54 3078.2109375
3.56 3078.2109375
3.58 3078.2109375
3.6 2983.953125
3.62 2993.4921875
3.64 3003.546875
3.66 3014.375
3.68 3024.171875
3.7 3034.484375
3.72 3044.0234375
3.74 3054.8515625
3.76 3063.1015625
3.78 2808.546875
3.8 2734.1171875
3.82 2641.8984375
3.84 2353.921875
3.86 2435.12890625
3.88 2435.12890625
3.9 2333.109375
3.92 2354.765625
3.94 2354.765625
3.96 2328.4609375
3.98 2355.015625
4 2355.015625
4.02 2355.015625
4.04 2355.015625
4.06 2082.2421875
4.08 2348.3046875
4.1 2354.75
4.12 2354.75
4.14 2354.75
4.16 2354.75
4.18 2354.75
4.2 2354.75
4.22 2354.75
4.24 2354.75
4.26 2241.22265625
4.28 1807.921875
4.3 2030.15625
4.32 2030.15625
4.34 2030.15625
4.36 2030.15625
4.38 2030.15625
4.4 2030.15625
4.42 2030.15625
4.44 2030.15625
4.46 2030.15625
4.48 1916.62890625
4.5 1482.5546875
4.52 1706.8515625
4.54 1706.8515625
4.56 1706.8515625
4.58 1706.8515625
4.6 1706.8515625
4.62 1706.8515625
4.64 1706.8515625
4.66 1706.8515625
4.68 1706.8515625
4.7 1555.44140625
4.72 1058.96484375
4.74 1022.9609375
4.76 1022.984375
};
\addplot [, color0, dotted, forget plot]
table {%
0 275.66015625
0.02 278.24609375
0.04 278.24609375
0.06 278.24609375
0.08 278.24609375
0.1 278.24609375
0.12 278.24609375
0.14 278.24609375
0.16 278.24609375
0.18 278.24609375
0.2 278.24609375
0.22 278.24609375
0.24 278.24609375
0.26 278.24609375
0.28 278.24609375
0.3 278.24609375
0.32 278.24609375
0.34 278.24609375
0.36 278.24609375
0.38 278.24609375
0.4 278.24609375
0.42 278.24609375
0.44 278.24609375
0.46 290.48046875
0.48 356.73828125
0.5 425.31640625
0.52 481.77734375
0.54 538.49609375
0.56 435.85546875
0.58 491.80078125
0.6 548.26171875
0.62 465.83203125
0.64 457.23828125
0.66 426.19921875
0.68 426.19921875
0.7 426.19921875
0.72 426.19921875
0.74 426.19921875
0.76 426.19921875
0.78 426.19921875
0.8 426.19921875
0.82 426.19921875
0.84 426.19921875
0.86 426.19921875
0.88 426.19921875
0.9 426.19921875
0.92 426.19921875
0.94 426.19921875
0.96 426.19921875
0.98 426.19921875
1 426.19921875
1.02 426.19921875
1.04 426.19921875
1.06 426.19921875
1.08 431.35546875
1.1 497.61328125
1.12 566.19140625
1.14 623.68359375
1.16 680.40234375
1.18 577.5546875
1.2 634.015625
1.22 689.9609375
1.24 601.296875
1.26 680.1328125
1.28 573.265625
1.3 574.296875
1.32 574.296875
1.34 574.296875
1.36 574.296875
1.38 574.296875
1.4 574.296875
1.42 574.296875
1.44 574.296875
1.46 574.296875
1.48 574.296875
1.5 574.296875
1.52 574.296875
1.54 574.296875
1.56 574.296875
1.58 574.296875
1.6 574.296875
1.62 574.296875
1.64 574.296875
1.66 574.296875
1.68 574.296875
1.7 574.296875
1.72 594.6640625
1.74 620.8125
1.76 634.015625
1.78 635.78515625
1.8 642.61328125
1.82 644.58984375
1.84 644.58984375
1.86 645.21875
1.88 645.21875
1.9 645.21875
1.92 645.21875
1.94 645.21875
1.96 645.21875
1.98 645.21875
2 645.21875
2.02 748.828125
2.04 953.1875
2.06 1073.83984375
2.08 1104.77734375
2.1 1136.23046875
2.12 1080.12109375
2.14 1235.50390625
2.16 1351.00390625
2.18 1274.8515625
2.2 1404.78125
2.22 1433.65625
2.24 1433.90234375
2.26 1597.09765625
2.28 1624.94140625
2.3 1639.12109375
2.32 1639.12109375
2.34 1681.6171875
2.36 1746.5859375
2.38 1747.546875
2.4 1813.5
2.42 1870.21484375
2.44 1894.890625
2.46 1908.8125
2.48 1983.24609375
2.5 2148.76171875
2.52 2351.66015625
2.54 2369.015625
2.56 2261.03515625
2.58 2302.265625
2.6 2343.2578125
2.62 2395.0390625
2.64 2518.53125
2.66 2642.5390625
2.68 2864.2578125
2.7 2621.16796875
2.72 2554.72265625
2.74 2574.31640625
2.76 2593.39453125
2.78 2612.98828125
2.8 2673.77734375
2.82 2673.77734375
2.84 2673.77734375
2.86 2673.77734375
2.88 2640.125
2.9 2685.2421875
2.92 2730.1015625
2.94 2774.4453125
2.96 2819.8203125
2.98 2864.421875
3 2908.765625
3.02 2949.7578125
3.04 3202.9296875
3.06 2826.39453125
3.08 2719.75390625
3.1 2719.75390625
3.12 2780.05078125
3.14 2780.05078125
3.16 2739.234375
3.18 2783.578125
3.2 2828.6953125
3.22 2872.78125
3.24 3013.2890625
3.26 2731.00390625
3.28 2802.41796875
3.3 2833.125
3.32 2935.21484375
3.34 2871.27734375
3.36 2964.08984375
3.38 3001.47265625
3.4 2923.0234375
3.42 2879.9765625
3.44 2814.3125
3.46 2848.859375
3.48 2883.6640625
3.5 2838.37109375
3.52 2908.80078125
3.54 3022.49609375
3.56 3022.49609375
3.58 3022.49609375
3.6 3022.49609375
3.62 2928.75390625
3.64 2939.06640625
3.66 2948.86328125
3.68 2959.17578125
3.7 2969.74609375
3.72 2979.80078125
3.74 2989.59765625
3.76 3000.16796875
3.78 3017.69921875
3.8 2767.78515625
3.82 2644.6640625
3.84 2548.5859375
3.86 2298.46484375
3.88 2380.19140625
3.9 2380.19140625
3.92 2291.83203125
3.94 2299.56640625
3.96 2299.56640625
3.98 2288.47265625
4 2299.04296875
4.02 2299.04296875
4.04 2299.04296875
4.06 2299.04296875
4.08 2038.90234375
4.1 2299.29296875
4.12 2299.29296875
4.14 2299.29296875
4.16 2299.29296875
4.18 2299.29296875
4.2 2299.29296875
4.22 2299.29296875
4.24 2299.29296875
4.26 2299.29296875
4.28 2147.8828125
4.3 1755.30078125
4.32 1975.21484375
4.34 1975.21484375
4.36 1975.21484375
4.38 1975.21484375
4.4 1975.21484375
4.42 1975.21484375
4.44 1975.21484375
4.46 1975.21484375
4.48 1975.21484375
4.5 1785.921875
4.52 1441.01953125
4.54 1651.39453125
4.56 1651.39453125
4.58 1651.39453125
4.6 1651.39453125
4.62 1651.39453125
4.64 1651.39453125
4.66 1651.39453125
4.68 1651.39453125
4.7 1651.39453125
4.72 1462.1015625
4.74 1003.5078125
4.76 967.50390625
4.78 967.53125
};
\addplot [, color0, dotted, forget plot]
table {%
0 275.859375
0.02 278.38671875
0.04 278.38671875
0.06 278.38671875
0.08 278.38671875
0.1 278.38671875
0.12 278.38671875
0.14 278.38671875
0.16 278.38671875
0.18 278.38671875
0.2 278.38671875
0.22 278.38671875
0.24 278.38671875
0.26 278.38671875
0.28 278.38671875
0.3 278.38671875
0.32 278.38671875
0.34 278.38671875
0.36 278.38671875
0.38 278.38671875
0.4 278.38671875
0.42 278.38671875
0.44 278.38671875
0.46 298.109375
0.48 364.8828125
0.5 432.4296875
0.52 488.375
0.54 545.09375
0.56 442.2578125
0.58 498.71875
0.6 554.6640625
0.62 484.05078125
0.64 425.3828125
0.66 426.4140625
0.68 426.4140625
0.7 426.4140625
0.72 426.4140625
0.74 426.4140625
0.76 426.4140625
0.78 426.4140625
0.8 426.4140625
0.82 426.4140625
0.84 426.4140625
0.86 426.4140625
0.88 426.4140625
0.9 426.4140625
0.92 426.4140625
0.94 426.4140625
0.96 426.4140625
0.98 426.4140625
1 426.4140625
1.02 426.4140625
1.04 426.4140625
1.06 426.4140625
1.08 447.296875
1.1 514.0703125
1.12 581.1015625
1.14 637.046875
1.16 693.5078125
1.18 590.66015625
1.2 647.12109375
1.22 703.06640625
1.24 633.69140625
1.26 572.5078125
1.28 574.52734375
1.3 574.52734375
1.32 574.52734375
1.34 574.52734375
1.36 574.52734375
1.38 574.52734375
1.4 574.52734375
1.42 574.52734375
1.44 574.52734375
1.46 574.52734375
1.48 574.52734375
1.5 574.52734375
1.52 574.52734375
1.54 574.52734375
1.56 574.52734375
1.58 574.52734375
1.6 574.52734375
1.62 574.52734375
1.64 574.52734375
1.66 574.52734375
1.68 574.52734375
1.7 574.52734375
1.72 619.640625
1.74 623.47265625
1.76 635.65234375
1.78 636.35546875
1.8 644.94921875
1.82 644.94921875
1.84 644.94921875
1.86 646.94921875
1.88 646.94921875
1.9 646.94921875
1.92 646.94921875
1.94 646.94921875
1.96 646.94921875
1.98 646.94921875
2 759.73828125
2.02 887.86328125
2.04 1039.70703125
2.06 1089.46484375
2.08 1120.91796875
2.1 1152.62890625
2.12 1084.56640625
2.14 1247.24609375
2.16 1364.55078125
2.18 1343.0546875
2.2 1418.33203125
2.22 1446.69140625
2.24 1530.7265625
2.26 1609.6171875
2.28 1637.71875
2.3 1639.0078125
2.32 1590.4921875
2.34 1740.5390625
2.36 1728.61328125
2.38 1781.12890625
2.4 1862.1328125
2.42 1905.0859375
2.44 1929.05859375
2.46 1930.17578125
2.48 2082.3671875
2.5 2249.9453125
2.52 2497.1875
2.54 2256.2578125
2.56 2304.46875
2.58 2363.48828125
2.6 2330.70703125
2.62 2475.59765625
2.64 2601.15234375
2.66 2758.41796875
2.68 2961.07421875
2.7 2615.4296875
2.72 2585.78125
2.74 2605.375
2.76 2624.7109375
2.78 2623.109375
2.8 2694.265625
2.82 2694.265625
2.84 2694.265625
2.86 2590.74609375
2.88 2685.62109375
2.9 2730.99609375
2.92 2775.33984375
2.94 2819.94140625
2.96 2865.31640625
2.98 2909.40234375
3 2954.26171875
3.02 3109.98046875
3.04 3028.95703125
3.06 2702.15625
3.08 2739.984375
3.1 2703.859375
3.12 2800.0234375
3.14 2800.0234375
3.16 2770.03515625
3.18 2815.41015625
3.2 2860.52734375
3.22 2905.64453125
3.24 2973.76953125
3.26 2762.3203125
3.28 2848.4296875
3.3 2917.546875
3.32 2955.4453125
3.34 2961.1171875
3.36 2998.5
3.38 3035.8828125
3.4 2992.53125
3.42 2730.1328125
3.44 2846.91796875
3.46 2881.20703125
3.48 2916.01171875
3.5 2911.578125
3.52 3022.359375
3.54 3042.984375
3.56 3042.984375
3.58 3042.984375
3.6 2838.64453125
3.62 2952.3359375
3.64 2962.6484375
3.66 2972.9609375
3.68 2983.53125
3.7 2993.328125
3.72 3003.3828125
3.74 3013.953125
3.76 3023.75
3.78 3033.31640625
3.8 2677.3515625
3.82 2533.96875
3.84 2510.6953125
3.86 2400.6484375
3.88 2400.6484375
3.9 2400.6484375
3.92 2319.765625
3.94 2319.765625
3.96 2244.09765625
3.98 2319.5
4 2319.5
4.02 2319.5
4.04 2319.5
4.06 2319.5
4.08 2142.1171875
4.1 2319.4921875
4.12 2319.4921875
4.14 2319.4921875
4.16 2319.4921875
4.18 2319.4921875
4.2 2319.4921875
4.22 2319.4921875
4.24 2319.4921875
4.26 2319.4921875
4.28 2016.55078125
4.3 1761.8359375
4.32 1995.671875
4.34 1995.671875
4.36 1995.671875
4.38 1995.671875
4.4 1995.671875
4.42 1995.671875
4.44 1995.671875
4.46 1995.671875
4.48 1995.671875
4.5 1995.671875
4.52 1368.1484375
4.54 1634.984375
4.56 1671.59375
4.58 1671.59375
4.6 1671.59375
4.62 1671.59375
4.64 1671.59375
4.66 1671.59375
4.68 1671.59375
4.7 1671.59375
4.72 1671.59375
4.74 1023.70703125
4.76 1023.70703125
4.78 987.73046875
};
\addplot [, color0, dotted, forget plot]
table {%
0 275.53515625
0.02 278.05078125
0.04 278.05078125
0.06 278.05078125
0.08 278.05078125
0.1 278.05078125
0.12 278.05078125
0.14 278.05078125
0.16 278.05078125
0.18 278.05078125
0.2 278.05078125
0.22 278.05078125
0.24 278.05078125
0.26 278.05078125
0.28 278.05078125
0.3 278.05078125
0.32 278.05078125
0.34 278.05078125
0.36 278.05078125
0.38 278.05078125
0.4 278.05078125
0.42 278.05078125
0.44 278.05078125
0.46 290.80859375
0.48 356.80859375
0.5 424.87109375
0.52 480.55859375
0.54 535.98828125
0.56 432.390625
0.58 488.8515625
0.6 544.796875
0.62 459.7109375
0.64 496.28125
0.66 426.08203125
0.68 426.08203125
0.7 426.08203125
0.72 426.08203125
0.74 426.08203125
0.76 426.08203125
0.78 426.08203125
0.8 426.08203125
0.82 426.08203125
0.84 426.08203125
0.86 426.08203125
0.88 426.08203125
0.9 426.08203125
0.92 426.08203125
0.94 426.08203125
0.96 426.08203125
0.98 426.08203125
1 426.08203125
1.02 426.08203125
1.04 426.08203125
1.06 426.08203125
1.08 428.91796875
1.1 490.53515625
1.12 558.33984375
1.14 616.60546875
1.16 671.77734375
1.18 646.6171875
1.2 623.5859375
1.22 678.7578125
1.24 575.89453125
1.26 705.359375
1.28 572.16796875
1.3 574.1796875
1.32 574.1796875
1.34 574.1796875
1.36 574.1796875
1.38 574.1796875
1.4 574.1796875
1.42 574.1796875
1.44 574.1796875
1.46 574.1796875
1.48 574.1796875
1.5 574.1796875
1.52 574.1796875
1.54 574.1796875
1.56 574.1796875
1.58 574.1796875
1.6 574.1796875
1.62 574.1796875
1.64 574.1796875
1.66 574.1796875
1.68 574.1796875
1.7 574.1796875
1.72 574.1796875
1.74 631.4140625
1.76 634.2109375
1.78 636.09375
1.8 640.125
1.82 644.6953125
1.84 644.6953125
1.86 644.6953125
1.88 645.48828125
1.9 645.48828125
1.92 645.48828125
1.94 645.48828125
1.96 645.48828125
1.98 645.48828125
2 645.48828125
2.02 784.390625
2.04 932.16015625
2.06 1061.578125
2.08 1092.515625
2.1 1123.7109375
2.12 1155.421875
2.14 1107.734375
2.16 1274.27734375
2.18 1367.60546875
2.2 1367.76953125
2.22 1446.71484375
2.24 1475.58984375
2.26 1564.13671875
2.28 1637.35546875
2.3 1663.65234375
2.32 1663.65234375
2.34 1647.6328125
2.36 1759.265625
2.38 1753.06640625
2.4 1807.50390625
2.42 1867.05859375
2.44 1903.03125
2.46 1903.03125
2.48 1907.51953125
2.5 2075.35546875
2.52 2237.51953125
2.54 2507.19140625
2.56 2236.90625
2.58 2291.8203125
2.6 2337.69140625
2.62 2341.26171875
2.64 2466.81640625
2.66 2592.11328125
2.68 2758.91796875
2.7 2851.53515625
2.72 2509.0078125
2.74 2561.7890625
2.76 2581.3828125
2.78 2600.9765625
2.8 2629.28125
2.82 2668.46875
2.84 2668.46875
2.86 2668.46875
2.88 2600.52734375
2.9 2665.49609375
2.92 2710.87109375
2.94 2755.98828125
2.96 2801.36328125
2.98 2844.93359375
3 2890.30859375
3.02 2935.42578125
3.04 3116.92578125
3.06 2906.453125
3.08 2692.53125
3.1 2727.3359375
3.12 2758.24609375
3.14 2798.20703125
3.16 2750.29296875
3.18 2787.4140625
3.2 2832.015625
3.22 2876.359375
3.24 2940.5546875
3.26 2768.6484375
3.28 2793.62109375
3.3 2928.97265625
3.32 2953.48828125
3.34 2877.73046875
3.36 2974.37109375
3.38 3011.49609375
3.4 3077.49609375
3.42 2858.03125
3.44 2825.109375
3.46 2859.3984375
3.48 2893.9453125
3.5 2928.75
3.52 2871.90234375
3.54 3041.28515625
3.56 3041.28515625
3.58 3041.28515625
3.6 3041.28515625
3.62 2908.09765625
3.64 2955.27734375
3.66 2965.07421875
3.68 2974.87109375
3.7 2985.69921875
3.72 2995.75390625
3.74 3005.80859375
3.76 3015.86328125
3.78 3026.17578125
3.8 2729.08203125
3.82 2664.96484375
3.84 2604.97265625
3.86 2413
3.88 2398.7109375
3.9 2398.7109375
3.92 2252.34765625
3.94 2318.08984375
3.96 2318.08984375
3.98 2245.12109375
4 2317.82421875
4.02 2317.82421875
4.04 2317.82421875
4.06 2317.82421875
4.08 2004.48046875
4.1 2255.68359375
4.12 2317.81640625
4.14 2317.81640625
4.16 2317.81640625
4.18 2317.81640625
4.2 2317.81640625
4.22 2317.81640625
4.24 2317.81640625
4.26 2317.81640625
4.28 2317.81640625
4.3 1707.05078125
4.32 1973.88671875
4.34 1993.73828125
4.36 1993.73828125
4.38 1993.73828125
4.4 1993.73828125
4.42 1993.73828125
4.44 1993.73828125
4.46 1993.73828125
4.48 1993.73828125
4.5 1955.9765625
4.52 1407.98046875
4.54 1665.53515625
4.56 1669.91796875
4.58 1669.91796875
4.6 1669.91796875
4.62 1669.91796875
4.64 1669.91796875
4.66 1669.91796875
4.68 1669.91796875
4.7 1669.91796875
4.72 1669.91796875
4.74 1022.03125
4.76 1022.03125
4.78 986.0546875
};
\addplot [, color0, dotted, forget plot]
table {%
0 275.5
0.02 278.05859375
0.04 278.05859375
0.06 278.05859375
0.08 278.05859375
0.1 278.05859375
0.12 278.05859375
0.14 278.05859375
0.16 278.05859375
0.18 278.05859375
0.2 278.05859375
0.22 278.05859375
0.24 278.05859375
0.26 278.05859375
0.28 278.05859375
0.3 278.05859375
0.32 278.05859375
0.34 278.05859375
0.36 278.05859375
0.38 278.05859375
0.4 278.05859375
0.42 278.05859375
0.44 278.05859375
0.46 298.5546875
0.48 364.5546875
0.5 431.5859375
0.52 486.5
0.54 541.4140625
0.56 437.7578125
0.58 493.1875
0.6 547.84375
0.62 461.6640625
0.64 494.9609375
0.66 426.0390625
0.68 426.0390625
0.7 426.0390625
0.72 426.0390625
0.74 426.0390625
0.76 426.0390625
0.78 426.0390625
0.8 426.0390625
0.82 426.0390625
0.84 426.0390625
0.86 426.0390625
0.88 426.0390625
0.9 426.0390625
0.92 426.0390625
0.94 426.0390625
0.96 426.0390625
0.98 426.0390625
1 426.0390625
1.02 426.0390625
1.04 426.0390625
1.06 426.0390625
1.08 427.84375
1.1 487.140625
1.12 555.4609375
1.14 615.015625
1.16 670.9609375
1.18 646.57421875
1.2 623.54296875
1.22 678.45703125
1.24 575.8515625
1.26 707.31640625
1.28 572.12890625
1.3 574.1484375
1.32 574.1484375
1.34 574.1484375
1.36 574.1484375
1.38 574.1484375
1.4 574.1484375
1.42 574.1484375
1.44 574.1484375
1.46 574.1484375
1.48 574.1484375
1.5 574.1484375
1.52 574.1484375
1.54 574.1484375
1.56 574.1484375
1.58 574.1484375
1.6 574.1484375
1.62 574.1484375
1.64 574.1484375
1.66 574.1484375
1.68 574.1484375
1.7 574.1484375
1.72 574.6640625
1.74 613.84375
1.76 634.16796875
1.78 636.05078125
1.8 640.79296875
1.82 644.65234375
1.84 644.65234375
1.86 644.65234375
1.88 645.140625
1.9 645.140625
1.92 645.140625
1.94 645.140625
1.96 645.140625
1.98 646.99609375
2 646.99609375
2.02 782.78125
2.04 929.52734375
2.06 1061.5234375
2.08 1092.4609375
2.1 1123.9140625
2.12 1154.8515625
2.14 1105.6171875
2.16 1274.99609375
2.18 1367.55078125
2.2 1342.9609375
2.22 1422.1640625
2.24 1451.296875
2.26 1540.4921875
2.28 1612.9375
2.3 1639.234375
2.32 1639.234375
2.34 1626.5625
2.36 1744.3828125
2.38 1741.9921875
2.4 1789.72265625
2.42 1849.015625
2.44 1885.0390625
2.46 1909.01171875
2.48 1910.66796875
2.5 2071.54296875
2.52 2234.73828125
2.54 2493.32421875
2.56 2239.671875
2.58 2287.8828125
2.6 2343.29296875
2.62 2332.16796875
2.64 2464.68359375
2.66 2589.20703125
2.68 2748.53515625
2.7 2915.64453125
2.72 2562.87109375
2.74 2566.359375
2.76 2585.953125
2.78 2605.2890625
2.8 2612.7109375
2.82 2674.328125
2.84 2674.328125
2.86 2674.328125
2.88 2568.74609375
2.9 2665.68359375
2.92 2710.28515625
2.94 2755.66015625
2.96 2800.26171875
2.98 2845.12109375
3 2890.49609375
3.02 2934.32421875
3.04 3091.33203125
3.06 3009.01953125
3.08 2682.4765625
3.1 2720.3046875
3.12 2724.9140625
3.14 2780.34375
3.16 2780.5703125
3.18 2766.59765625
3.2 2810.94140625
3.22 2855.80078125
3.24 2912.26171875
3.26 2767.94921875
3.28 2773.578125
3.3 2907.3828125
3.32 2935.5078125
3.34 2859.75
3.36 2955.1015625
3.38 2993
3.4 3088.65234375
3.42 2834.89453125
3.44 2806.09765625
3.46 2840.90234375
3.48 2875.19140625
3.5 2909.22265625
3.52 2845.4140625
3.54 3022.7890625
3.56 3022.7890625
3.58 3022.7890625
3.6 3022.7890625
3.62 2881.8671875
3.64 2936.78125
3.66 2947.09375
3.68 2956.890625
3.7 2966.9453125
3.72 2977.2578125
3.74 2987.5703125
3.76 2998.140625
3.78 3007.421875
3.8 2711.359375
3.82 2648.015625
3.84 2586.9921875
3.86 2395.01953125
3.88 2380.4765625
3.9 2380.4765625
3.92 2233.078125
3.94 2299.8515625
3.96 2299.8515625
3.98 2227.65625
4 2299.328125
4.02 2299.328125
4.04 2299.328125
4.06 2299.328125
4.08 1977.5703125
4.1 2244.6640625
4.12 2299.8359375
4.14 2299.8359375
4.16 2299.8359375
4.18 2299.8359375
4.2 2299.8359375
4.22 2299.8359375
4.24 2299.8359375
4.26 2299.8359375
4.28 2299.8359375
4.3 1697.3203125
4.32 1963.3828125
4.34 1975.5
4.36 1975.5
4.38 1975.5
4.4 1975.5
4.42 1975.5
4.44 1975.5
4.46 1975.5
4.48 1975.5
4.5 1931.6171875
4.52 1394.125
4.54 1649.359375
4.56 1651.9375
4.58 1651.9375
4.6 1651.9375
4.62 1651.9375
4.64 1651.9375
4.66 1651.9375
4.68 1651.9375
4.7 1651.9375
4.72 1632.0546875
4.74 1004.05078125
4.76 1004.05078125
4.78 968.0703125
};
\addplot [, color0, dotted, forget plot]
table {%
0 275.91015625
0.02 277.93359375
0.04 278.44921875
0.06 278.44921875
0.08 278.44921875
0.1 278.44921875
0.12 278.44921875
0.14 278.44921875
0.16 278.44921875
0.18 278.44921875
0.2 278.44921875
0.22 278.44921875
0.24 278.44921875
0.26 278.44921875
0.28 278.44921875
0.3 278.44921875
0.32 278.44921875
0.34 278.44921875
0.36 278.44921875
0.38 278.44921875
0.4 278.44921875
0.42 278.44921875
0.44 278.44921875
0.46 278.44921875
0.48 278.44921875
0.5 278.44921875
0.52 278.44921875
0.54 278.44921875
0.56 278.44921875
0.58 278.44921875
0.6 278.44921875
0.62 278.44921875
0.64 278.44921875
0.66 278.44921875
0.68 278.44921875
0.7 326.5390625
0.72 394.6015625
0.74 456.4765625
0.76 512.6796875
0.78 569.140625
0.8 466.2890625
0.82 522.75
0.84 537.78125
0.86 536.16015625
0.88 426.46484375
0.9 426.46484375
0.92 426.46484375
0.94 426.46484375
0.96 426.46484375
0.98 426.46484375
1 426.46484375
1.02 426.46484375
1.04 426.46484375
1.06 426.46484375
1.08 426.46484375
1.1 426.46484375
1.12 426.46484375
1.14 426.46484375
1.16 426.46484375
1.18 426.46484375
1.2 426.46484375
1.22 426.46484375
1.24 426.46484375
1.26 426.46484375
1.28 426.46484375
1.3 426.46484375
1.32 458.43359375
1.34 525.98046875
1.36 591.20703125
1.38 647.15234375
1.4 703.61328125
1.42 600.50390625
1.44 656.44921875
1.46 711.62109375
1.48 651.73828125
1.5 572.546875
1.52 574.546875
1.54 574.546875
1.56 574.546875
1.58 574.546875
1.6 574.546875
1.62 574.546875
1.64 574.546875
1.66 574.546875
1.68 574.546875
1.7 574.546875
1.72 574.546875
1.74 574.546875
1.76 574.546875
1.78 574.546875
1.8 574.546875
1.82 574.546875
1.84 574.546875
1.86 574.546875
1.88 574.546875
1.9 574.546875
1.92 574.546875
1.94 574.546875
1.96 619.40625
1.98 623.5234375
2 635.8125
2.02 636.640625
2.04 645.23046875
2.06 645.23046875
2.08 645.23046875
2.1 645.61328125
2.12 645.61328125
2.14 645.61328125
2.16 645.61328125
2.18 647.63671875
2.2 647.63671875
2.22 647.63671875
2.24 696.61328125
2.26 817.07421875
2.28 984.12890625
2.3 1083.64453125
2.32 1114.58203125
2.34 1145.51953125
2.36 1107.7421875
2.38 1247.609375
2.4 1360.015625
2.42 1320.21875
2.44 1413.02734375
2.46 1442.41796875
2.48 1524.90625
2.5 1603.796875
2.52 1633.1875
2.54 1639.6328125
2.56 1599.421875
2.58 1735.5
2.6 1695.984375
2.62 1749.8671875
2.64 1820.1796875
2.66 1903.44921875
2.68 1916.98046875
2.7 1942.2890625
2.72 2027.48828125
2.74 2195.58203125
2.76 2406.47265625
2.78 2326.15234375
2.8 2299.03515625
2.82 2362.43359375
2.84 2375.83984375
2.86 2441.02734375
2.88 2566.58203125
2.9 2691.10546875
2.92 2937.57421875
2.94 2683.890625
2.96 2590.65625
2.98 2610.5078125
3 2630.1015625
3.02 2623.6953125
3.04 2706.875
3.06 2706.875
3.08 2706.875
3.1 2706.875
3.12 2683.01953125
3.14 2726.84765625
3.16 2771.96484375
3.18 2817.08203125
3.2 2862.19921875
3.22 2906.02734375
3.24 2950.88671875
3.26 3026.16796875
3.28 3297.12890625
3.3 2755.78125
3.32 2752.59375
3.34 2752.59375
3.36 2800.515625
3.38 2800.515625
3.4 2771
3.42 2815.859375
3.44 2860.71875
3.46 2905.8359375
3.48 2970.09765625
3.5 2763.28515625
3.52 2849.65234375
3.54 2921.86328125
3.56 2955.89453125
3.58 2962.59765625
3.6 3000.23828125
3.62 3036.84765625
3.64 2992.98046875
3.66 2734.4453125
3.68 2848.140625
3.7 2881.3984375
3.72 2915.9453125
3.74 2868.71484375
3.76 2976.66015625
3.78 3043.69140625
3.8 3043.69140625
3.82 3043.69140625
3.84 2967.96875
3.86 2950.98046875
3.88 2961.55078125
3.9 2971.60546875
3.92 2982.17578125
3.94 2991.71484375
3.96 3001.51171875
3.98 3012.85546875
4 3022.39453125
4.02 3086.07421875
4.04 2786.1796875
4.06 2569.33984375
4.08 2511.40234375
4.1 2365.40234375
4.12 2400.72265625
4.14 2400.72265625
4.16 2320.08984375
4.18 2320.08984375
4.2 2320.08984375
4.22 2320.08203125
4.24 2320.08203125
4.26 2320.08203125
4.28 2320.08203125
4.3 2320.08203125
4.32 2101.19140625
4.34 2320.07421875
4.36 2320.07421875
4.38 2320.07421875
4.4 2320.07421875
4.42 2320.07421875
4.44 2320.07421875
4.46 2320.07421875
4.48 2320.07421875
4.5 2320.07421875
4.52 2055.015625
4.54 1824.29296875
4.56 1995.73828125
4.58 1995.73828125
4.6 1995.73828125
4.62 1995.73828125
4.64 1995.73828125
4.66 1995.73828125
4.68 1995.73828125
4.7 1995.73828125
4.72 1995.73828125
4.74 1671.85546875
4.76 1521.61328125
4.78 1671.91796875
4.8 1671.91796875
4.82 1671.91796875
4.84 1671.91796875
4.86 1671.91796875
4.88 1671.91796875
4.9 1671.91796875
4.92 1671.91796875
4.94 1671.91796875
4.96 1310.15234375
4.98 1059.86328125
5 987.9296875
};
\addplot [, color1, dotted, forget plot]
table {%
0 275.8984375
0.02 278.4921875
0.04 278.4921875
0.06 278.4921875
0.08 278.4921875
0.1 278.4921875
0.12 278.4921875
0.14 278.4921875
0.16 278.4921875
0.18 278.4921875
0.2 278.4921875
0.22 278.4921875
0.24 278.4921875
0.26 278.4921875
0.28 278.4921875
0.3 278.4921875
0.32 278.4921875
0.34 278.4921875
0.36 278.4921875
0.38 278.4921875
0.4 278.4921875
0.42 278.4921875
0.44 278.4921875
0.46 297.421875
0.48 364.1953125
0.5 432
0.52 487.9453125
0.54 544.6640625
0.56 442.03515625
0.58 498.49609375
0.6 554.69921875
0.62 482.08203125
0.64 425.4140625
0.66 426.4453125
0.68 426.4453125
0.7 426.4453125
0.72 426.4453125
0.74 426.4453125
0.76 426.4453125
0.78 426.4453125
0.8 426.4453125
0.82 426.4453125
0.84 426.4453125
0.86 426.4453125
0.88 426.4453125
0.9 426.4453125
0.92 426.4453125
0.94 426.4453125
0.96 426.4453125
0.98 426.4453125
1 426.4453125
1.02 426.4453125
1.04 426.4453125
1.06 426.4453125
1.08 447.5859375
1.1 514.875
1.12 582.1640625
1.14 638.109375
1.16 694.828125
1.18 592.23828125
1.2 648.69921875
1.22 704.90234375
1.24 637.72265625
1.26 572.5390625
1.28 574.54296875
1.3 574.54296875
1.32 574.54296875
1.34 574.54296875
1.36 574.54296875
1.38 574.54296875
1.4 574.54296875
1.42 574.54296875
1.44 574.54296875
1.46 574.54296875
1.48 574.54296875
1.5 574.54296875
1.52 574.54296875
1.54 574.54296875
1.56 574.54296875
1.58 574.54296875
1.6 574.54296875
1.62 574.54296875
1.64 574.54296875
1.66 574.54296875
1.68 574.54296875
1.7 574.54296875
1.72 623.78515625
1.74 632.21875
1.76 636.5234375
1.78 637.28515625
1.8 645.21875
1.82 645.21875
1.84 645.21875
1.86 646.16015625
1.88 646.16015625
1.9 646.16015625
1.92 646.16015625
1.94 646.16015625
1.96 646.16015625
1.98 646.16015625
2 768.2734375
2.02 907.75390625
2.04 1052.12109375
2.06 1091.56640625
2.08 1122.76171875
2.1 1154.73046875
2.12 1098.53125
2.14 1267.65625
2.16 1366.9140625
2.18 1368.359375
2.2 1446.46875
2.22 1475.0859375
2.24 1558.7421875
2.26 1637.1171875
2.28 1664.1875
2.3 1664.1875
2.32 1646.62109375
2.34 1759.80078125
2.36 1753.60546875
2.38 1806.0859375
2.4 1865.8359375
2.42 1887.1015625
2.44 1938.48828125
2.46 1866.5546875
2.48 1866.5546875
2.5 1900.75390625
2.52 2066.52734375
2.54 2247.76953125
2.56 2515.12109375
2.58 2349.5390625
2.6 2516.34375
2.62 2516.34375
2.64 2516.34375
2.66 2516.34375
2.68 2516.34375
2.7 2516.34375
2.72 2516.34375
2.74 2516.34375
2.76 2516.34375
2.78 2116.6953125
2.8 1891.3984375
2.82 1922.83203125
2.84 1976.7109375
2.86 1980.5390625
2.88 2100.6796875
2.9 2225.4609375
2.92 2348.953125
2.94 2618.3671875
2.96 2383.52734375
2.98 2570.69921875
3 2570.69921875
3.02 2570.69921875
3.04 2570.69921875
3.06 2570.69921875
3.08 2570.69921875
3.1 2570.69921875
3.12 2570.69921875
3.14 2570.69921875
3.16 2171.0546875
3.18 1814.34765625
3.2 1877.765625
3.22 1897.1015625
3.24 1916.4375
3.26 1942.421875
3.28 1983.9296875
3.3 1983.9296875
3.32 1983.9296875
3.34 1915.21484375
3.36 1980.95703125
3.38 2025.81640625
3.4 2070.93359375
3.42 2114.76171875
3.44 2160.13671875
3.46 2205.51171875
3.48 2250.37109375
3.5 2432.90234375
3.52 2221.9140625
3.54 2456.59375
3.56 2522.59375
3.58 2522.59375
3.6 2522.59375
3.62 2522.59375
3.64 2522.59375
3.66 2522.59375
3.68 2522.59375
3.7 2522.59375
3.72 2522.59375
3.74 1806.28125
3.76 1706.859375
3.78 1713.3046875
3.8 1777.98828125
3.82 1777.98828125
3.84 1733.6796875
3.86 1777.5078125
3.88 1823.3984375
3.9 1868.515625
3.92 2000.2578125
3.94 2053.2265625
3.96 2053.7421875
3.98 2053.7421875
4 2053.7421875
4.02 2053.7421875
4.04 1565.65625
4.06 1613.55078125
4.08 1754.05859375
4.1 1771.09765625
4.12 1669.0078125
4.14 1792.75390625
4.16 1830.13671875
4.18 1869.296875
4.2 1830.8359375
4.22 1830.8359375
4.24 1705.2734375
4.26 1656.5546875
4.28 1573.1015625
4.3 1607.6484375
4.32 1642.1953125
4.34 1590.58203125
4.36 1694.66015625
4.38 1777.41796875
4.4 1777.41796875
4.42 1777.41796875
4.44 1754.68359375
4.46 1684.44921875
4.48 1694.76171875
4.5 1705.07421875
4.52 1715.12890625
4.54 1724.92578125
4.56 1735.49609375
4.58 1745.55078125
4.6 1755.86328125
4.62 1812.06640625
4.64 1735.33984375
4.66 1735.33984375
4.68 1578.66796875
4.7 1265.484375
4.72 1219.9296875
4.74 1164.4296875
4.76 972.5234375
};
\addplot [, color1, dotted, forget plot]
table {%
0 275.58984375
0.02 278.03125
0.04 278.03125
0.06 278.03125
0.08 278.03125
0.1 278.03125
0.12 278.03125
0.14 278.03125
0.16 278.03125
0.18 278.03125
0.2 278.03125
0.22 278.03125
0.24 278.03125
0.26 278.03125
0.28 278.03125
0.3 278.03125
0.32 278.03125
0.34 278.03125
0.36 278.03125
0.38 278.03125
0.4 278.03125
0.42 278.03125
0.44 278.03125
0.46 292.8359375
0.48 359.09375
0.5 427.4140625
0.52 483.1015625
0.54 539.5625
0.56 436.71484375
0.58 492.66015625
0.6 547.83203125
0.62 463.66796875
0.64 494.6953125
0.66 425.7734375
0.68 425.7734375
0.7 425.7734375
0.72 425.7734375
0.74 425.7734375
0.76 425.7734375
0.78 425.7734375
0.8 425.7734375
0.82 425.7734375
0.84 425.7734375
0.86 425.7734375
0.88 425.7734375
0.9 425.7734375
0.92 425.7734375
0.94 425.7734375
0.96 425.7734375
0.98 425.7734375
1 425.7734375
1.02 425.7734375
1.04 425.7734375
1.06 425.7734375
1.08 427.8359375
1.1 487.1328125
1.12 555.453125
1.14 615.0078125
1.16 671.2109375
1.18 646.5546875
1.2 623.265625
1.22 678.1796875
1.24 575.83203125
1.26 707.296875
1.28 572.109375
1.3 574.12109375
1.32 574.12109375
1.34 574.12109375
1.36 574.12109375
1.38 574.12109375
1.4 574.12109375
1.42 574.12109375
1.44 574.12109375
1.46 574.12109375
1.48 574.12109375
1.5 574.12109375
1.52 574.12109375
1.54 574.12109375
1.56 574.12109375
1.58 574.12109375
1.6 574.12109375
1.62 574.12109375
1.64 574.12109375
1.66 574.12109375
1.68 574.12109375
1.7 574.12109375
1.72 574.12109375
1.74 632.90234375
1.76 633.9609375
1.78 635.875
1.8 639.8828125
1.82 644.45703125
1.84 644.45703125
1.86 644.45703125
1.88 645.3671875
1.9 645.3671875
1.92 645.3671875
1.94 645.3671875
1.96 645.3671875
1.98 645.3671875
2 645.3671875
2.02 794.203125
2.04 952.10546875
2.06 1062.703125
2.08 1094.15625
2.1 1125.09375
2.12 1156.03125
2.14 1118.125
2.16 1287.5078125
2.18 1368.71875
2.2 1372.98828125
2.22 1446.72265625
2.24 1476.11328125
2.26 1570.0859375
2.28 1637.890625
2.3 1663.671875
2.32 1663.671875
2.34 1660.02734375
2.36 1759.28125
2.38 1753.08984375
2.4 1791.8046875
2.42 1873.23046875
2.44 1890.23828125
2.46 1945.7109375
2.48 1873.77734375
2.5 1873.77734375
2.52 1896.08984375
2.54 2063.41015625
2.56 2231.76171875
2.58 2501.17578125
2.6 2335.59375
2.62 2523.28125
2.64 2523.28125
2.66 2523.28125
2.68 2523.28125
2.7 2523.28125
2.72 2523.28125
2.74 2523.28125
2.76 2523.28125
2.78 2523.28125
2.8 2199.3984375
2.82 1895.81640625
2.84 1929.578125
2.86 1983.71484375
2.88 2022.86328125
2.9 2146.87109375
2.92 2271.39453125
2.94 2479.70703125
2.96 2269.6875
2.98 2511.1875
3 2577.703125
3.02 2577.703125
3.04 2577.703125
3.06 2577.703125
3.08 2577.703125
3.1 2577.703125
3.12 2577.703125
3.14 2577.703125
3.16 2577.703125
3.18 1969
3.2 1873.94140625
3.22 1893.27734375
3.24 1912.87109375
3.26 1900.01953125
3.28 1990.67578125
3.3 1990.67578125
3.32 1990.67578125
3.34 1990.67578125
3.36 1961.921875
3.38 2007.0390625
3.4 2051.125
3.42 2094.953125
3.44 2139.8125
3.46 2185.1875
3.48 2230.8203125
3.5 2270.5234375
3.52 2540.96875
3.54 2297.30859375
3.56 2529.59765625
3.58 2529.59765625
3.6 2529.59765625
3.62 2529.59765625
3.64 2529.59765625
3.66 2529.59765625
3.68 2529.59765625
3.7 2529.59765625
3.72 2529.59765625
3.74 2248.9609375
3.76 1736.12890625
3.78 1715.66796875
3.8 1723.91796875
3.82 1784.4765625
3.84 1784.4765625
3.86 1755.12109375
3.88 1799.98046875
3.9 1843.29296875
3.92 1889.18359375
3.94 1985.33984375
3.96 2060.48828125
3.98 2060.48828125
4 2060.48828125
4.02 2060.48828125
4.04 2060.48828125
4.06 1549.66015625
4.08 1619.0078125
4.1 1651.76953125
4.12 1753.859375
4.14 1714.671875
4.16 1785.828125
4.18 1823.984375
4.2 1767.70703125
4.22 1813.85546875
4.24 1813.85546875
4.26 1705.375
4.28 1456.34765625
4.3 1566.17578125
4.32 1600.72265625
4.34 1635.01171875
4.36 1642.953125
4.38 1754.765625
4.4 1760.4375
4.42 1760.4375
4.44 1760.4375
4.46 1556.09765625
4.48 1670.3046875
4.5 1680.6171875
4.52 1690.9296875
4.54 1700.984375
4.56 1711.0390625
4.58 1720.8359375
4.6 1732.1796875
4.62 1741.9765625
4.64 1751.28515625
4.66 1718.1015625
4.68 1718.1015625
4.7 1460.1875
4.72 1278.1484375
4.74 1241.875
4.76 1147.1875
};
\addplot [, color1, dotted, forget plot]
table {%
0 276.01953125
0.02 278.55078125
0.04 278.55078125
0.06 278.55078125
0.08 278.55078125
0.1 278.55078125
0.12 278.55078125
0.14 278.55078125
0.16 278.55078125
0.18 278.55078125
0.2 278.55078125
0.22 278.55078125
0.24 278.55078125
0.26 278.55078125
0.28 278.55078125
0.3 278.55078125
0.32 278.55078125
0.34 278.55078125
0.36 278.55078125
0.38 278.55078125
0.4 278.55078125
0.42 278.55078125
0.44 278.55078125
0.46 310.125
0.48 377.671875
0.5 442.8984375
0.52 498.84375
0.54 555.046875
0.56 451.421875
0.58 507.8828125
0.6 564.34375
0.62 506.1875
0.64 426.3046875
0.66 426.3046875
0.68 426.3046875
0.7 426.3046875
0.72 426.3046875
0.74 426.3046875
0.76 426.3046875
0.78 426.3046875
0.8 426.3046875
0.82 426.3046875
0.84 426.3046875
0.86 426.3046875
0.88 426.3046875
0.9 426.3046875
0.92 426.3046875
0.94 426.3046875
0.96 426.3046875
0.98 426.3046875
1 426.3046875
1.02 426.3046875
1.04 426.3046875
1.06 426.3046875
1.08 455.6953125
1.1 523.2421875
1.12 588.984375
1.14 644.671875
1.16 700.875
1.18 597.7578125
1.2 654.4765625
1.22 710.6796875
1.24 651.828125
1.26 572.6328125
1.28 574.6640625
1.3 574.6640625
1.32 574.6640625
1.34 574.6640625
1.36 574.6640625
1.38 574.6640625
1.4 574.6640625
1.42 574.6640625
1.44 574.6640625
1.46 574.6640625
1.48 574.6640625
1.5 574.6640625
1.52 574.6640625
1.54 574.6640625
1.56 574.6640625
1.58 574.6640625
1.6 574.6640625
1.62 574.6640625
1.64 574.6640625
1.66 574.6640625
1.68 574.6640625
1.7 574.6640625
1.72 622.6171875
1.74 629.60546875
1.76 636.26171875
1.78 636.62890625
1.8 645.203125
1.82 645.203125
1.84 645.203125
1.86 647.375
1.88 647.375
1.9 647.375
1.92 647.375
1.94 647.375
1.96 647.375
1.98 647.375
2 749.515625
2.02 866.81640625
2.04 1025.8828125
2.06 1088.2734375
2.08 1119.46875
2.1 1150.6640625
2.12 1063.25390625
2.14 1247.58984375
2.16 1363.34765625
2.18 1368.14453125
2.2 1442.44921875
2.22 1470.80859375
2.24 1554.9765625
2.26 1633.09375
2.28 1661.96875
2.3 1664.03125
2.32 1606.25
2.34 1759.6484375
2.36 1741.32421875
2.38 1789.12890625
2.4 1873.10546875
2.42 1882.88671875
2.44 1945.92578125
2.46 1873.9921875
2.48 1873.9921875
2.5 1875.43359375
2.52 2044.3515625
2.54 2199.5546875
2.56 2467.6796875
2.58 2301.07421875
2.6 2523.30859375
2.62 2523.30859375
2.64 2523.30859375
2.66 2523.30859375
2.68 2523.30859375
2.7 2523.30859375
2.72 2523.30859375
2.74 2523.30859375
2.76 2523.30859375
2.78 2296.1328125
2.8 1888.625
2.82 1967
2.84 1983.99609375
2.86 2004.58203125
2.88 2128.33203125
2.9 2253.62890625
2.92 2449.82421875
2.94 2366.67578125
2.96 2482.3359375
2.98 2577.7265625
3 2577.7265625
3.02 2577.7265625
3.04 2577.7265625
3.06 2577.7265625
3.08 2577.7265625
3.1 2577.7265625
3.12 2577.7265625
3.14 2577.7265625
3.16 1938.34765625
3.18 1871.90234375
3.2 1891.49609375
3.22 1911.08984375
3.24 1902.74609375
3.26 1990.95703125
3.28 1990.95703125
3.3 1990.95703125
3.32 1990.95703125
3.34 1959.625
3.36 2004.2265625
3.38 2048.5703125
3.4 2093.6875
3.42 2138.546875
3.44 2183.40625
3.46 2228.5234375
3.48 2266.6796875
3.5 2533.515625
3.52 2288.30859375
3.54 2529.87890625
3.56 2529.87890625
3.58 2529.87890625
3.6 2529.87890625
3.62 2529.87890625
3.64 2529.87890625
3.66 2529.87890625
3.68 2529.87890625
3.7 2529.87890625
3.72 2264.8203125
3.74 1731.515625
3.76 1715.953125
3.78 1723.9453125
3.8 1785.01953125
3.82 1785.01953125
3.84 1757.2109375
3.86 1801.296875
3.88 1847.1875
3.9 1891.2734375
3.92 1909.6015625
3.94 2061.03125
3.96 2061.03125
3.98 2061.03125
4 2061.03125
4.02 1947.57421875
4.04 1610.828125
4.06 1696.16015625
4.08 1746.96484375
4.1 1815.02734375
4.12 1815.02734375
4.14 1852.92578125
4.16 1891.08203125
4.18 1870.640625
4.2 1874.5078125
4.22 1874.5078125
4.24 1627.734375
4.26 1560.828125
4.28 1633.015625
4.3 1667.3046875
4.32 1701.8515625
4.34 1649.51953125
4.36 1821.34765625
4.38 1821.34765625
4.4 1821.34765625
4.42 1821.34765625
4.44 1688.6171875
4.46 1729.41015625
4.48 1739.72265625
4.5 1750.29296875
4.52 1760.86328125
4.54 1770.40234375
4.56 1780.45703125
4.58 1791.02734375
4.6 1800.82421875
4.62 1886.16015625
4.64 1778.75390625
4.66 1778.75390625
4.68 1541.078125
4.7 1331.32421875
4.72 1292.47265625
4.74 1208.09765625
4.76 1016.1328125
};
\addplot [, color1, dotted, forget plot]
table {%
0 275.5
0.02 277.921875
0.04 277.921875
0.06 277.921875
0.08 277.921875
0.1 277.921875
0.12 277.921875
0.14 277.921875
0.16 277.921875
0.18 277.921875
0.2 277.921875
0.22 277.921875
0.24 277.921875
0.26 277.921875
0.28 277.921875
0.3 277.921875
0.32 277.921875
0.34 277.921875
0.36 277.921875
0.38 277.921875
0.4 277.921875
0.42 277.921875
0.44 277.921875
0.46 295.84765625
0.48 362.62109375
0.5 430.68359375
0.52 486.11328125
0.54 542.31640625
0.56 439.7421875
0.58 496.203125
0.6 552.921875
0.62 479.58984375
0.64 424.66796875
0.66 425.95703125
0.68 425.95703125
0.7 425.95703125
0.72 425.95703125
0.74 425.95703125
0.76 425.95703125
0.78 425.95703125
0.8 425.95703125
0.82 425.95703125
0.84 425.95703125
0.86 425.95703125
0.88 425.95703125
0.9 425.95703125
0.92 425.95703125
0.94 425.95703125
0.96 425.95703125
0.98 425.95703125
1 425.95703125
1.02 425.95703125
1.04 425.95703125
1.06 425.95703125
1.08 448.12890625
1.1 515.16015625
1.12 582.44921875
1.14 637.62109375
1.16 693.56640625
1.18 590.9765625
1.2 647.4375
1.22 704.15625
1.24 637.234375
1.26 572.046875
1.28 574.0625
1.3 574.0625
1.32 574.0625
1.34 574.0625
1.36 574.0625
1.38 574.0625
1.4 574.0625
1.42 574.0625
1.44 574.0625
1.46 574.0625
1.48 574.0625
1.5 574.0625
1.52 574.0625
1.54 574.0625
1.56 574.0625
1.58 574.0625
1.6 574.0625
1.62 574.0625
1.64 574.0625
1.66 574.0625
1.68 574.0625
1.7 574.0625
1.72 622.2734375
1.74 628.26953125
1.76 635.0546875
1.78 635.890625
1.8 644.4765625
1.82 644.4765625
1.84 644.4765625
1.86 645.27734375
1.88 645.27734375
1.9 645.27734375
1.92 645.27734375
1.94 645.27734375
1.96 645.27734375
1.98 645.27734375
2 753.38671875
2.02 883.29296875
2.04 1031.7890625
2.06 1087.734375
2.08 1119.4453125
2.1 1150.8984375
2.12 1071.23046875
2.14 1247.05859375
2.16 1363.58984375
2.18 1343.125
2.2 1417.62890625
2.22 1446.76171875
2.24 1530.5390625
2.26 1609.171875
2.28 1637.7890625
2.3 1639.078125
2.32 1587.99609375
2.34 1740.36328125
2.36 1728.69140625
2.38 1780.79296875
2.4 1861.2890625
2.42 1873.08203125
2.44 1934.015625
2.46 1862.08203125
2.48 1862.08203125
2.5 1868.15625
2.52 2034.703125
2.54 2192.2265625
2.56 2462.4140625
2.58 2296.3125
2.6 2511.328125
2.62 2511.328125
2.64 2511.328125
2.66 2511.328125
2.68 2511.328125
2.7 2511.328125
2.72 2511.328125
2.74 2511.328125
2.76 2511.328125
2.78 2284.15234375
2.8 1878.96484375
2.82 1966.36328125
2.84 1971.7578125
2.86 1994.1484375
2.88 2118.9296875
2.9 2241.90625
2.92 2428.5625
2.94 2379.48828125
2.96 2463.65234375
2.98 2565.48828125
3 2565.48828125
3.02 2565.48828125
3.04 2565.48828125
3.06 2565.48828125
3.08 2565.48828125
3.1 2565.48828125
3.12 2565.48828125
3.14 2565.48828125
3.16 1923.7890625
3.18 1859.1484375
3.2 1878.7421875
3.22 1898.078125
3.24 1916.640625
3.26 1978.71875
3.28 1978.71875
3.3 1978.71875
3.32 1978.71875
3.34 1939.91015625
3.36 1983.99609375
3.38 2029.11328125
3.4 2073.97265625
3.42 2119.08984375
3.44 2163.43359375
3.46 2208.03515625
3.48 2251.86328125
3.5 2466.87890625
3.52 2223.4765625
3.54 2493.1484375
3.56 2517.8984375
3.58 2517.8984375
3.6 2517.8984375
3.62 2517.8984375
3.64 2517.8984375
3.66 2517.8984375
3.68 2517.8984375
3.7 2517.8984375
3.72 2404.37109375
3.74 1761.2109375
3.76 1702.6796875
3.78 1710.15625
3.8 1772.77734375
3.82 1772.77734375
3.84 1732.3359375
3.86 1776.9375
3.88 1822.3125
3.9 1866.140625
3.92 2009.2265625
3.94 2048.53125
3.96 2048.53125
3.98 2048.53125
4 2048.53125
4.02 2048.53125
4.04 1598.328125
4.06 1654.26953125
4.08 1764.65234375
4.1 1802.52734375
4.12 1713.32421875
4.14 1827.79296875
4.16 1865.43359375
4.18 1814.82421875
4.2 1862.78125
4.22 1862.78125
4.24 1760.6796875
4.26 1613.01171875
4.28 1608.140625
4.3 1642.171875
4.32 1676.9765625
4.34 1642.89453125
4.36 1750.06640625
4.38 1809.10546875
4.4 1809.10546875
4.42 1809.10546875
4.44 1704.37109375
4.46 1717.68359375
4.48 1727.73828125
4.5 1737.79296875
4.52 1748.10546875
4.54 1758.67578125
4.56 1768.73046875
4.58 1778.78515625
4.6 1788.58203125
4.62 1867.98828125
4.64 1767.02734375
4.66 1767.02734375
4.68 1567.234375
4.7 1314.18359375
4.72 1274.30078125
4.74 1196.11328125
4.76 1004.15625
};
\addplot [, color1, dotted, forget plot]
table {%
0 275.76953125
0.02 278.3828125
0.04 278.3828125
0.06 278.3828125
0.08 278.3828125
0.1 278.3828125
0.12 278.3828125
0.14 278.3828125
0.16 278.3828125
0.18 278.3828125
0.2 278.3828125
0.22 278.3828125
0.24 278.3828125
0.26 278.3828125
0.28 278.3828125
0.3 278.3828125
0.32 278.3828125
0.34 278.3828125
0.36 278.3828125
0.38 278.3828125
0.4 278.3828125
0.42 278.3828125
0.44 278.3828125
0.46 292.9453125
0.48 359.203125
0.5 427.5234375
0.52 483.7265625
0.54 538.640625
0.56 435.9765625
0.58 492.1796875
0.6 547.609375
0.62 463.95703125
0.64 494.9921875
0.66 426.0703125
0.68 426.0703125
0.7 426.0703125
0.72 426.0703125
0.74 426.0703125
0.76 426.0703125
0.78 426.0703125
0.8 426.0703125
0.82 426.0703125
0.84 426.0703125
0.86 426.0703125
0.88 426.0703125
0.9 426.0703125
0.92 426.0703125
0.94 426.0703125
0.96 426.0703125
0.98 426.0703125
1 426.0703125
1.02 426.0703125
1.04 426.0703125
1.06 426.0703125
1.08 430.96875
1.1 496.453125
1.12 563.484375
1.14 621.75
1.16 678.2109375
1.18 576.12890625
1.2 630.0078125
1.22 684.6640625
1.24 587.4140625
1.26 719.59375
1.28 572.40234375
1.3 574.4296875
1.32 574.4296875
1.34 574.4296875
1.36 574.4296875
1.38 574.4296875
1.4 574.4296875
1.42 574.4296875
1.44 574.4296875
1.46 574.4296875
1.48 574.4296875
1.5 574.4296875
1.52 574.4296875
1.54 574.4296875
1.56 574.4296875
1.58 574.4296875
1.6 574.4296875
1.62 574.4296875
1.64 574.4296875
1.66 574.4296875
1.68 574.4296875
1.7 574.4296875
1.72 592.734375
1.74 618.61328125
1.76 634.6796875
1.78 636.31640625
1.8 642.53515625
1.82 644.8984375
1.84 644.8984375
1.86 645.02734375
1.88 646.46875
1.9 646.46875
1.92 646.46875
1.94 646.46875
1.96 646.46875
1.98 646.46875
2 646.46875
2.02 845.07421875
2.04 953.38671875
2.06 1070.68359375
2.08 1101.87890625
2.1 1132.81640625
2.12 1058.12890625
2.14 1194.1640625
2.16 1347.5625
2.18 1299.4453125
2.2 1417.3359375
2.22 1455.75
2.24 1455.9453125
2.26 1618.80859375
2.28 1647.16796875
2.3 1663.66796875
2.32 1663.66796875
2.34 1705.13671875
2.36 1759.27734375
2.38 1753.08203125
2.4 1821.4765625
2.42 1870.18359375
2.44 1923.78515625
2.46 1942.890625
2.48 1870.95703125
2.5 1871.265625
2.52 1960.27734375
2.54 2127.59765625
2.56 2337.45703125
2.58 2254.9375
2.6 2435.359375
2.62 2520.1796875
2.64 2520.1796875
2.66 2520.1796875
2.68 2520.1796875
2.7 2520.1796875
2.72 2520.1796875
2.74 2520.1796875
2.76 2520.1796875
2.78 2520.1796875
2.8 1893.234375
2.82 1906.63671875
2.84 1980.609375
2.86 1980.609375
2.88 2054.3046875
2.9 2178.5703125
2.92 2302.8359375
2.94 2556.78125
2.96 2320.91015625
2.98 2574.59765625
3 2574.59765625
3.02 2574.59765625
3.04 2574.59765625
3.06 2574.59765625
3.08 2574.59765625
3.1 2574.59765625
3.12 2574.59765625
3.14 2574.59765625
3.16 2408.52734375
3.18 1890.265625
3.2 1876.5078125
3.22 1896.1015625
3.24 1915.6953125
3.26 1886.25
3.28 1987.828125
3.3 1987.828125
3.32 1987.828125
3.34 1950.03125
3.36 1970.93359375
3.38 2014.76171875
3.4 2059.87890625
3.42 2104.73828125
3.44 2147.79296875
3.46 2192.39453125
3.48 2238.02734375
3.5 2332.64453125
3.52 2549.81640625
3.54 2359.6875
3.56 2526.4921875
3.58 2526.4921875
3.6 2526.4921875
3.62 2526.4921875
3.64 2526.4921875
3.66 2526.4921875
3.68 2526.4921875
3.7 2526.4921875
3.72 2526.4921875
3.74 1975.3125
3.76 1661.51953125
3.78 1709.46875
3.8 1696.8046875
3.82 1720.78125
3.84 1720.78125
3.86 1705.5390625
3.88 1750.3984375
3.9 1795.7734375
3.92 1846.3046875
3.94 1900.3046875
3.96 1996.7265625
3.98 1996.7265625
4 1996.7265625
4.02 1996.7265625
4.04 1683.3828125
4.06 1564.56640625
4.08 1654.02734375
4.1 1739.12109375
4.12 1750.72265625
4.14 1760.51953125
4.16 1797.64453125
4.18 1846.37109375
4.2 1810.71875
4.22 1810.71875
4.24 1772.93359375
4.26 1579.4609375
4.28 1542.4140625
4.3 1576.1875
4.32 1610.734375
4.34 1645.5390625
4.36 1596.16796875
4.38 1757.30078125
4.4 1757.30078125
4.42 1757.30078125
4.44 1757.30078125
4.46 1633.65234375
4.48 1672.06640625
4.5 1681.86328125
4.52 1692.17578125
4.54 1701.97265625
4.56 1712.28515625
4.58 1722.85546875
4.6 1732.91015625
4.62 1742.19140625
4.64 1713.16015625
4.66 1714.96484375
4.68 1714.96484375
4.7 1447.15625
4.72 1240.390625
4.74 1158.3125
4.76 952.30859375
};
\addplot [, color1, dotted, forget plot]
table {%
0 276
0.02 278.55859375
0.04 278.55859375
0.06 278.55859375
0.08 278.55859375
0.1 278.55859375
0.12 278.55859375
0.14 278.55859375
0.16 278.55859375
0.18 278.55859375
0.2 278.55859375
0.22 278.55859375
0.24 278.55859375
0.26 278.55859375
0.28 278.55859375
0.3 278.55859375
0.32 278.55859375
0.34 278.55859375
0.36 278.55859375
0.38 278.55859375
0.4 278.55859375
0.42 278.55859375
0.44 278.55859375
0.46 307.81640625
0.48 375.10546875
0.5 440.58984375
0.52 496.01953125
0.54 551.96484375
0.56 449.08984375
0.58 506.06640625
0.6 562.26953125
0.62 500.171875
0.64 426.29296875
0.66 426.29296875
0.68 426.29296875
0.7 426.29296875
0.72 426.29296875
0.74 426.29296875
0.76 426.29296875
0.78 426.29296875
0.8 426.29296875
0.82 426.29296875
0.84 426.29296875
0.86 426.29296875
0.88 426.29296875
0.9 426.29296875
0.92 426.29296875
0.94 426.29296875
0.96 426.29296875
0.98 426.29296875
1 426.29296875
1.02 426.29296875
1.04 426.29296875
1.06 426.29296875
1.08 459.29296875
1.1 526.58203125
1.12 591.29296875
1.14 646.72265625
1.16 702.66796875
1.18 600.06640625
1.2 656.78515625
1.22 712.98828125
1.24 655.81640625
1.26 572.625
1.28 574.6484375
1.3 574.6484375
1.32 574.6484375
1.34 574.6484375
1.36 574.6484375
1.38 574.6484375
1.4 574.6484375
1.42 574.6484375
1.44 574.6484375
1.46 574.6484375
1.48 574.6484375
1.5 574.6484375
1.52 574.6484375
1.54 574.6484375
1.56 574.6484375
1.58 574.6484375
1.6 574.6484375
1.62 574.6484375
1.64 574.6484375
1.66 574.6484375
1.68 574.6484375
1.7 574.6484375
1.72 628.7890625
1.74 634.7734375
1.76 636.61328125
1.78 638.82421875
1.8 645.19140625
1.82 645.19140625
1.84 645.19140625
1.86 645.5
1.88 647.54296875
1.9 647.54296875
1.92 647.54296875
1.94 647.54296875
1.96 647.54296875
1.98 647.54296875
2 752.04296875
2.02 875.77734375
2.04 1030.203125
2.06 1088.7265625
2.08 1120.6953125
2.1 1151.6328125
2.12 1076.87109375
2.14 1247.5390625
2.16 1364.84375
2.18 1368.09765625
2.2 1444.20703125
2.22 1473.33984375
2.24 1554.93359375
2.26 1636.40234375
2.28 1663.98828125
2.3 1663.98828125
2.32 1640.23046875
2.34 1759.59765625
2.36 1753.40625
2.38 1798.74609375
2.4 1879.52734375
2.42 1894.9765625
2.44 1952.09765625
2.46 1880.1640625
2.48 1880.1640625
2.5 1902.0234375
2.52 2069.0859375
2.54 2237.953125
2.56 2507.8828125
2.58 2339.46875
2.6 2529.734375
2.62 2529.734375
2.64 2529.734375
2.66 2529.734375
2.68 2529.734375
2.7 2529.734375
2.72 2529.734375
2.74 2529.734375
2.76 2529.734375
2.78 2205.8515625
2.8 1900.98046875
2.82 1936.28515625
2.84 1990.421875
2.86 2026.4765625
2.88 2151
2.9 2276.0390625
2.92 2488.4765625
2.94 2276.65234375
2.96 2520.73046875
2.98 2583.89453125
3 2583.89453125
3.02 2583.89453125
3.04 2583.89453125
3.06 2583.89453125
3.08 2583.89453125
3.1 2583.89453125
3.12 2583.89453125
3.14 2583.89453125
3.16 1970.8125
3.18 1880.1328125
3.2 1899.7265625
3.22 1919.0625
3.24 1902.6015625
3.26 1997.125
3.28 1997.125
3.3 1997.125
3.32 1997.125
3.34 1971.20703125
3.36 2016.06640625
3.38 2060.66796875
3.4 2105.26953125
3.42 2149.87109375
3.44 2195.24609375
3.46 2240.36328125
3.48 2296.05078125
3.5 2565.72265625
3.52 2320
3.54 2536.5625
3.56 2536.5625
3.58 2536.5625
3.6 2536.5625
3.62 2536.5625
3.64 2536.5625
3.66 2536.5625
3.68 2536.5625
3.7 2536.5625
3.72 2189.1640625
3.74 1753.1484375
3.76 1719.0234375
3.78 1673.87890625
3.8 1730.59375
3.82 1730.59375
3.84 1706.84375
3.86 1752.21875
3.88 1796.3046875
3.9 1841.6796875
3.92 1850.5625
3.94 2006.28125
3.96 2006.28125
3.98 2006.28125
4 2006.28125
4.02 1892.82421875
4.04 1556.59375
4.06 1642.95703125
4.08 1695.82421875
4.1 1760.27734375
4.12 1762.33984375
4.14 1799.46484375
4.16 1836.58984375
4.18 1820.2734375
4.2 1820.53125
4.22 1820.53125
4.24 1557.76171875
4.26 1506.59375
4.28 1578.265625
4.3 1612.8125
4.32 1647.6171875
4.34 1595.02734375
4.36 1767.11328125
4.38 1767.11328125
4.4 1767.11328125
4.42 1767.11328125
4.44 1579.26953125
4.46 1678.78515625
4.48 1688.83984375
4.5 1698.89453125
4.52 1709.20703125
4.54 1719.26171875
4.56 1729.31640625
4.58 1740.14453125
4.6 1749.68359375
4.62 1654.13671875
4.64 1724.26171875
4.66 1724.26171875
4.68 1394.05859375
4.7 1324.26953125
4.72 1249.83984375
4.74 1057.86328125
};
\addplot [, color1, dotted, forget plot]
table {%
0 275.63671875
0.02 278.16796875
0.04 278.16796875
0.06 278.16796875
0.08 278.16796875
0.1 278.16796875
0.12 278.16796875
0.14 278.16796875
0.16 278.16796875
0.18 278.16796875
0.2 278.16796875
0.22 278.16796875
0.24 278.16796875
0.26 278.16796875
0.28 278.16796875
0.3 278.16796875
0.32 278.16796875
0.34 278.16796875
0.36 278.16796875
0.38 278.16796875
0.4 278.16796875
0.42 278.16796875
0.44 278.16796875
0.46 295.5546875
0.48 361.8125
0.5 429.6171875
0.52 485.5625
0.54 542.28125
0.56 439.41015625
0.58 495.35546875
0.6 551.04296875
0.62 473.77734375
0.64 424.34375
0.66 426.14453125
0.68 426.14453125
0.7 426.14453125
0.72 426.14453125
0.74 426.14453125
0.76 426.14453125
0.78 426.14453125
0.8 426.14453125
0.82 426.14453125
0.84 426.14453125
0.86 426.14453125
0.88 426.14453125
0.9 426.14453125
0.92 426.14453125
0.94 426.14453125
0.96 426.14453125
0.98 426.14453125
1 426.14453125
1.02 426.14453125
1.04 426.14453125
1.06 426.14453125
1.08 428.20703125
1.1 488.27734375
1.12 556.33984375
1.14 615.63671875
1.16 671.58203125
1.18 646.6796875
1.2 623.90625
1.22 678.5625
1.24 575.95703125
1.26 705.421875
1.28 572.23046875
1.3 574.2734375
1.32 574.2734375
1.34 574.2734375
1.36 574.2734375
1.38 574.2734375
1.4 574.2734375
1.42 574.2734375
1.44 574.2734375
1.46 574.2734375
1.48 574.2734375
1.5 574.2734375
1.52 574.2734375
1.54 574.2734375
1.56 574.2734375
1.58 574.2734375
1.6 574.2734375
1.62 574.2734375
1.64 574.2734375
1.66 574.2734375
1.68 574.2734375
1.7 574.2734375
1.72 574.2734375
1.74 621.1953125
1.76 622.2265625
1.78 635.30078125
1.8 636.2578125
1.82 644.84765625
1.84 644.84765625
1.86 644.84765625
1.88 646.8671875
1.9 646.8671875
1.92 646.8671875
1.94 646.8671875
1.96 646.8671875
1.98 646.8671875
2 646.8671875
2.02 756.78125
2.04 886.859375
2.06 1035.61328125
2.08 1088.98046875
2.1 1119.91796875
2.12 1151.37109375
2.14 1073
2.16 1247.5390625
2.18 1364.84375
2.2 1368.09375
2.22 1444.40234375
2.24 1473.27734375
2.26 1554.86328125
2.28 1636.33203125
2.3 1663.91796875
2.32 1663.91796875
2.34 1641.71484375
2.36 1759.53515625
2.38 1753.328125
2.4 1800.25390625
2.42 1880.22265625
2.44 1897.38671875
2.46 1952.21875
2.48 1880.28515625
2.5 1880.28515625
2.52 1902.86328125
2.54 2068.63671875
2.56 2235.95703125
2.58 2505.11328125
2.6 2339.5234375
2.62 2529.53125
2.64 2529.53125
2.66 2529.53125
2.68 2529.53125
2.7 2529.53125
2.72 2529.53125
2.74 2529.53125
2.76 2529.53125
2.78 2529.53125
2.8 2129.8828125
2.82 1904.90234375
2.84 1936.08203125
2.86 1990.21875
2.88 2034.265625
2.9 2158.7890625
2.92 2283.828125
2.94 2498.328125
2.96 2264.00390625
2.98 2533.16015625
3 2583.94921875
3.02 2583.94921875
3.04 2583.94921875
3.06 2583.94921875
3.08 2583.94921875
3.1 2583.94921875
3.12 2583.94921875
3.14 2583.94921875
3.16 2583.94921875
3.18 1946.14453125
3.2 1881.21875
3.22 1901.0703125
3.24 1920.1484375
3.26 1911.1640625
3.28 1996.921875
3.3 1996.921875
3.32 1996.921875
3.34 1996.921875
3.36 1971.77734375
3.38 2016.89453125
3.4 2062.26953125
3.42 2107.38671875
3.44 2152.24609375
3.46 2197.36328125
3.48 2241.70703125
3.5 2314.66796875
3.52 2584.33984375
3.54 2338.875
3.56 2536.6171875
3.58 2536.6171875
3.6 2536.6171875
3.62 2536.6171875
3.64 2536.6171875
3.66 2536.6171875
3.68 2536.6171875
3.7 2536.6171875
3.72 2536.6171875
3.74 2099.0859375
3.76 1729.25390625
3.78 1719.078125
3.8 1678.83203125
3.82 1730.90625
3.84 1730.90625
3.86 1707.9296875
3.88 1752.7890625
3.9 1798.1640625
3.92 1843.28125
3.94 1863.765625
3.96 2006.59375
3.98 2006.59375
4 2006.59375
4.02 2006.59375
4.04 1844.78125
4.06 1562.83203125
4.08 1648.94140625
4.1 1713.15234375
4.12 1760.58984375
4.14 1764.97265625
4.16 1802.61328125
4.18 1839.73828125
4.2 1820.328125
4.22 1820.328125
4.24 1820.328125
4.26 1559.9375
4.28 1523.6640625
4.3 1580.640625
4.32 1615.4453125
4.34 1649.9921875
4.36 1562.984375
4.38 1767.16796875
4.4 1767.16796875
4.42 1767.16796875
4.44 1767.16796875
4.46 1597.37109375
4.48 1679.61328125
4.5 1689.41015625
4.52 1699.98046875
4.54 1709.77734375
4.56 1720.08984375
4.58 1730.40234375
4.6 1739.94140625
4.62 1750.25390625
4.64 1659.34765625
4.66 1724.31640625
4.68 1724.31640625
4.7 1399.26953125
4.72 1328.44921875
4.74 1250.15234375
4.76 1058.1796875
};
\addplot [, color1, dotted, forget plot]
table {%
0 275.51953125
0.02 277.96484375
0.04 277.96484375
0.06 277.96484375
0.08 277.96484375
0.1 277.96484375
0.12 277.96484375
0.14 277.96484375
0.16 277.96484375
0.18 277.96484375
0.2 277.96484375
0.22 277.96484375
0.24 277.96484375
0.26 277.96484375
0.28 277.96484375
0.3 277.96484375
0.32 277.96484375
0.34 277.96484375
0.36 277.96484375
0.38 277.96484375
0.4 277.96484375
0.42 277.96484375
0.44 277.96484375
0.46 289.953125
0.48 356.2109375
0.5 424.7890625
0.52 480.9921875
0.54 536.421875
0.56 433.05078125
0.58 489.25390625
0.6 545.71484375
0.62 461.609375
0.64 515.91796875
0.66 425.71875
0.68 425.71875
0.7 425.71875
0.72 425.71875
0.74 425.71875
0.76 425.71875
0.78 425.71875
0.8 425.71875
0.82 425.71875
0.84 425.71875
0.86 425.71875
0.88 425.71875
0.9 425.71875
0.92 425.71875
0.94 425.71875
0.96 425.71875
0.98 425.71875
1 425.71875
1.02 425.71875
1.04 425.71875
1.06 425.71875
1.08 427.0078125
1.1 483.7265625
1.12 552.3046875
1.14 612.6328125
1.16 668.578125
1.18 713.77734375
1.2 619.34375
1.22 674.7734375
1.24 608.6171875
1.26 695.2421875
1.28 572.0546875
1.3 574.07421875
1.32 574.07421875
1.34 574.07421875
1.36 574.07421875
1.38 574.07421875
1.4 574.07421875
1.42 574.07421875
1.44 574.07421875
1.46 574.07421875
1.48 574.07421875
1.5 574.07421875
1.52 574.07421875
1.54 574.07421875
1.56 574.07421875
1.58 574.07421875
1.6 574.07421875
1.62 574.07421875
1.64 574.07421875
1.66 574.07421875
1.68 574.07421875
1.7 574.07421875
1.72 574.07421875
1.74 630.53515625
1.76 633.90625
1.78 635.82421875
1.8 638.1640625
1.82 644.41015625
1.84 644.41015625
1.86 644.41015625
1.88 644.72265625
1.9 644.72265625
1.92 644.72265625
1.94 644.72265625
1.96 644.72265625
1.98 644.72265625
2 644.72265625
2.02 790.75390625
2.04 947.890625
2.06 1062.61328125
2.08 1094.06640625
2.1 1125.00390625
2.12 1155.94140625
2.14 1116.48828125
2.16 1284.58203125
2.18 1368.37109375
2.2 1367.75390625
2.22 1446.640625
2.24 1476.03125
2.26 1566.89453125
2.28 1637.53515625
2.3 1663.57421875
2.32 1663.57421875
2.34 1655.296875
2.36 1759.19140625
2.38 1753
2.4 1800.76171875
2.42 1872.8984375
2.44 1893.77734375
2.46 1945.61328125
2.48 1873.6796875
2.5 1873.6796875
2.52 1904.7578125
2.54 2071.8203125
2.56 2245.5859375
2.58 2515.2578125
2.6 2349.93359375
2.62 2522.92578125
2.64 2522.92578125
2.66 2522.92578125
2.68 2522.92578125
2.7 2522.92578125
2.72 2522.92578125
2.74 2522.92578125
2.76 2522.92578125
2.78 2522.92578125
2.8 2123.27734375
2.82 1898.296875
2.84 1929.9921875
2.86 1983.87109375
2.88 2028.17578125
2.9 2151.92578125
2.92 2275.67578125
2.94 2487.85546875
2.96 2253.27734375
2.98 2521.9140625
3 2577.6015625
3.02 2577.6015625
3.04 2577.6015625
3.06 2577.6015625
3.08 2577.6015625
3.1 2577.6015625
3.12 2577.6015625
3.14 2577.6015625
3.16 2577.6015625
3.18 1974.83203125
3.2 1874.09765625
3.22 1893.69140625
3.24 1913.28515625
3.26 1899.40234375
3.28 1990.57421875
3.3 1990.57421875
3.32 1990.57421875
3.34 1990.57421875
3.36 1961.3046875
3.38 2004.875
3.4 2049.734375
3.42 2094.8515625
3.44 2140.2265625
3.46 2184.828125
3.48 2229.6875
3.5 2266.5546875
3.52 2516.890625
3.54 2272.97265625
3.56 2530.01171875
3.58 2530.01171875
3.6 2530.01171875
3.62 2530.01171875
3.64 2530.01171875
3.66 2530.01171875
3.68 2530.01171875
3.7 2530.01171875
3.72 2530.01171875
3.74 2317.375
3.76 1721.84765625
3.78 1715.30859375
3.8 1723.30078125
3.82 1784.6328125
3.84 1784.6328125
3.86 1751.41015625
3.88 1796.01171875
3.9 1839.83984375
3.92 1885.47265625
3.94 2059.23828125
3.96 2060.38671875
3.98 2060.38671875
4 2060.38671875
4.02 2060.38671875
4.04 2060.38671875
4.06 1549.55859375
4.08 1612.9765625
4.1 1651.66796875
4.12 1753.7578125
4.14 1686.46875
4.16 1782.6328125
4.18 1818.7265625
4.2 1733.0625
4.22 1814.01171875
4.24 1814.01171875
4.26 1726.34765625
4.28 1535.25
4.3 1560.66015625
4.32 1595.46484375
4.34 1630.01171875
4.36 1604.1796875
4.38 1712.3828125
4.4 1760.3359375
4.42 1760.3359375
4.44 1760.3359375
4.46 1633.60546875
4.48 1668.9140625
4.5 1678.96875
4.52 1689.5390625
4.54 1699.3359375
4.56 1709.1328125
4.58 1719.4453125
4.6 1729.5
4.62 1739.5546875
4.64 1803.4921875
4.66 1718
4.68 1718
4.7 1518.2109375
4.72 1258.45703125
4.74 1214.96484375
4.76 1147.34765625
4.78 955.44140625
};
\addplot [, color1, dotted, forget plot]
table {%
0 275.2890625
0.02 277.8125
0.04 277.8125
0.06 277.8125
0.08 277.8125
0.1 277.8125
0.12 277.8125
0.14 277.8125
0.16 277.8125
0.18 277.8125
0.2 277.8125
0.22 277.8125
0.24 277.8125
0.26 277.8125
0.28 277.8125
0.3 277.8125
0.32 277.8125
0.34 277.8125
0.36 277.8125
0.38 277.8125
0.4 277.8125
0.42 277.8125
0.44 277.8125
0.46 299.07421875
0.48 365.84765625
0.5 433.13671875
0.52 489.08203125
0.54 545.80078125
0.56 443.15234375
0.58 499.35546875
0.6 554.52734375
0.62 481.40234375
0.64 424.7265625
0.66 425.7578125
0.68 425.7578125
0.7 425.7578125
0.72 425.7578125
0.74 425.7578125
0.76 425.7578125
0.78 425.7578125
0.8 425.7578125
0.82 425.7578125
0.84 425.7578125
0.86 425.7578125
0.88 425.7578125
0.9 425.7578125
0.92 425.7578125
0.94 425.7578125
0.96 425.7578125
0.98 425.7578125
1 425.7578125
1.02 425.7578125
1.04 425.7578125
1.06 425.7578125
1.08 445.8671875
1.1 512.8984375
1.12 579.9296875
1.14 635.875
1.16 692.3359375
1.18 589.23046875
1.2 644.40234375
1.22 700.08984375
1.24 625.03515625
1.26 571.58984375
1.28 574.109375
1.3 574.109375
1.32 574.109375
1.34 574.109375
1.36 574.109375
1.38 574.109375
1.4 574.109375
1.42 574.109375
1.44 574.109375
1.46 574.109375
1.48 574.109375
1.5 574.109375
1.52 574.109375
1.54 574.109375
1.56 574.109375
1.58 574.109375
1.6 574.109375
1.62 574.109375
1.64 574.109375
1.66 574.109375
1.68 574.109375
1.7 574.109375
1.72 599.890625
1.74 620.1171875
1.76 633.93359375
1.78 635.72265625
1.8 641.96484375
1.82 644.30078125
1.84 644.30078125
1.86 644.30078125
1.88 645.9765625
1.9 645.9765625
1.92 645.9765625
1.94 645.9765625
1.96 645.9765625
1.98 645.9765625
2 645.9765625
2.02 645.9765625
2.04 645.9765625
2.06 768.91015625
2.08 952.890625
2.1 1071.9921875
2.12 1102.9296875
2.14 1133.8671875
2.16 1075.16015625
2.18 1229.5078125
2.2 1350.421875
2.22 1273.49609375
2.24 1429.265625
2.26 1458.65625
2.28 1468.32421875
2.3 1620.43359375
2.32 1649.30859375
2.34 1663.23046875
2.36 1663.23046875
2.38 1717.078125
2.4 1758.84375
2.42 1752.640625
2.44 1826.203125
2.46 1869.45703125
2.48 1937.796875
2.5 1942.14453125
2.52 1870.2109375
2.54 1870.76953125
2.56 1974.7265625
2.58 2142.5625
2.6 2361.703125
2.62 2195.61328125
2.64 2463.734375
2.66 2519.421875
2.68 2519.421875
2.7 2519.421875
2.72 2519.421875
2.74 2519.421875
2.76 2519.421875
2.78 2519.421875
2.8 2519.421875
2.82 2519.421875
2.84 1871.53515625
2.86 1913.87109375
2.88 1980.109375
2.9 1942.2421875
2.92 2070.8203125
2.94 2190.703125
2.96 2295.375
2.98 2443.1015625
3 2375.83984375
3.02 2472.51953125
3.04 2573.83984375
3.06 2573.83984375
3.08 2573.83984375
3.1 2573.83984375
3.12 2573.83984375
3.14 2573.83984375
3.16 2573.83984375
3.18 2573.83984375
3.2 2573.83984375
3.22 1925.95703125
3.24 1866.984375
3.26 1887.09375
3.28 1906.4296875
3.3 1925.765625
3.32 1987.0703125
3.34 1987.0703125
3.36 1987.0703125
3.38 1987.0703125
3.4 1953.67578125
3.42 1998.79296875
3.44 2044.42578125
3.46 2087.99609375
3.48 2133.62890625
3.5 2178.48828125
3.52 2223.34765625
3.54 2262.53515625
3.56 2525.76171875
3.58 2281.5859375
3.6 2525.9921875
3.62 2525.9921875
3.64 2525.9921875
3.66 2525.9921875
3.68 2525.9921875
3.7 2525.9921875
3.72 2525.9921875
3.74 2525.9921875
3.76 2525.9921875
3.78 2260.93359375
3.8 1721.953125
3.82 1708.453125
3.84 1708.453125
3.86 1720.28125
3.88 1720.28125
3.9 1692.6640625
3.92 1737.0078125
3.94 1782.640625
3.96 1826.984375
3.98 1844.796875
4 1996.2265625
4.02 1996.2265625
4.04 1996.2265625
4.06 1996.2265625
4.08 1920.65234375
4.1 1546.0234375
4.12 1628.26171875
4.14 1673.13671875
4.16 1750.22265625
4.18 1745.32421875
4.2 1787.60546875
4.22 1825.24609375
4.24 1799.90625
4.26 1810.21875
4.28 1810.21875
4.3 1566.3671875
4.32 1489.0625
4.34 1567.4375
4.36 1601.7265625
4.38 1636.2734375
4.4 1622.85546875
4.42 1756.80078125
4.44 1756.80078125
4.46 1756.80078125
4.48 1756.80078125
4.5 1557.61328125
4.52 1668.21484375
4.54 1678.26953125
4.56 1688.83984375
4.58 1698.63671875
4.6 1708.69140625
4.62 1718.48828125
4.64 1729.57421875
4.66 1739.62890625
4.68 1640.73046875
4.7 1714.46484375
4.72 1714.46484375
4.74 1382.71484375
4.76 1313.95703125
4.78 1239.52734375
4.8 1047.55078125
};
\addplot [, color1, dotted, forget plot]
table {%
0 275.54296875
0.02 278.0625
0.04 278.0625
0.06 278.0625
0.08 278.0625
0.1 278.0625
0.12 278.0625
0.14 278.0625
0.16 278.0625
0.18 278.0625
0.2 278.0625
0.22 278.0625
0.24 278.0625
0.26 278.0625
0.28 278.0625
0.3 278.0625
0.32 278.0625
0.34 278.0625
0.36 278.0625
0.38 278.0625
0.4 278.0625
0.42 278.0625
0.44 278.0625
0.46 297.26953125
0.48 364.30078125
0.5 432.10546875
0.52 487.79296875
0.54 543.73828125
0.56 440.55859375
0.58 497.01953125
0.6 552.96484375
0.62 479.64453125
0.64 424.96875
0.66 426
0.68 426
0.7 426
0.72 426
0.74 426
0.76 426
0.78 426
0.8 426
0.82 426
0.84 426
0.86 426
0.88 426
0.9 426
0.92 426
0.94 426
0.96 426
0.98 426
1 426
1.02 426
1.04 426
1.06 426
1.08 438.1171875
1.1 504.890625
1.12 573.2109375
1.14 629.15625
1.16 685.1015625
1.18 581.99609375
1.2 638.71484375
1.22 694.40234375
1.24 613.09765625
1.26 604.42578125
1.28 574.109375
1.3 574.109375
1.32 574.109375
1.34 574.109375
1.36 574.109375
1.38 574.109375
1.4 574.109375
1.42 574.109375
1.44 574.109375
1.46 574.109375
1.48 574.109375
1.5 574.109375
1.52 574.109375
1.54 574.109375
1.56 574.109375
1.58 574.109375
1.6 574.109375
1.62 574.109375
1.64 574.109375
1.66 574.109375
1.68 574.109375
1.7 574.109375
1.72 607.625
1.74 631.69921875
1.76 634.0625
1.78 635.80859375
1.8 643.68359375
1.82 644.39453125
1.84 644.39453125
1.86 644.84375
1.88 644.84375
1.9 644.84375
1.92 644.84375
1.94 644.84375
1.96 644.84375
1.98 644.84375
2 650.5546875
2.02 769.2265625
2.04 953.04296875
2.06 1076.015625
2.08 1107.46875
2.1 1138.40625
2.12 1102.90625
2.14 1246.94140625
2.16 1354.19140625
2.18 1303.0390625
2.2 1432.44921875
2.22 1460.80859375
2.24 1489.81640625
2.26 1623.62109375
2.28 1652.23828125
2.3 1663.32421875
2.32 1625.625
2.34 1737.79296875
2.36 1721.1015625
2.38 1755.83203125
2.4 1832.58984375
2.42 1872.60546875
2.44 1930.67578125
2.46 1945.37109375
2.48 1873.4375
2.5 1874.05859375
2.52 1996.58984375
2.54 2164.16796875
2.56 2399.55078125
2.58 2233.96875
2.6 2503.125
2.62 2522.9765625
2.64 2522.9765625
2.66 2522.9765625
2.68 2522.9765625
2.7 2522.9765625
2.72 2522.9765625
2.74 2522.9765625
2.76 2522.9765625
2.78 2522.9765625
2.8 1876.69140625
2.82 1924.90234375
2.84 1983.40625
2.86 1958.1015625
2.88 2097.3203125
2.9 2222.6171875
2.92 2374.7265625
2.94 2593.125
2.96 2409.04296875
2.98 2576.87890625
3 2576.87890625
3.02 2576.87890625
3.04 2576.87890625
3.06 2576.87890625
3.08 2576.87890625
3.1 2576.87890625
3.12 2576.87890625
3.14 2576.87890625
3.16 2122.609375
3.18 1838.0546875
3.2 1885.4921875
3.22 1904.828125
3.24 1924.6796875
3.26 1973.609375
3.28 1990.109375
3.3 1990.109375
3.32 1990.109375
3.34 1942.27734375
3.36 1991.77734375
3.38 2036.89453125
3.4 2080.72265625
3.42 2126.09765625
3.44 2170.95703125
3.46 2216.33203125
3.48 2261.44921875
3.5 2459.96484375
3.52 2215.7890625
3.54 2485.4609375
3.56 2529.03125
3.58 2529.03125
3.6 2529.03125
3.62 2529.03125
3.64 2529.03125
3.66 2529.03125
3.68 2529.03125
3.7 2529.03125
3.72 2491.26953125
3.74 1833.0859375
3.76 1713.5546875
3.78 1720.515625
3.8 1784.16796875
3.82 1784.16796875
3.84 1742.4375
3.86 1787.296875
3.88 1831.8984375
3.9 1877.015625
3.92 2015.4609375
3.94 2059.921875
3.96 2059.921875
3.98 2059.921875
4 2059.921875
4.02 2059.921875
4.04 1551.66796875
4.06 1624.37109375
4.08 1773.90234375
4.1 1777.53515625
4.12 1676.98828125
4.14 1801.25390625
4.16 1839.15234375
4.18 1820.5390625
4.2 1837.2734375
4.22 1837.2734375
4.24 1728.46875
4.26 1617.1328125
4.28 1582.1171875
4.3 1616.1484375
4.32 1650.953125
4.34 1611.45703125
4.36 1720.43359375
4.38 1783.85546875
4.4 1783.85546875
4.42 1783.85546875
4.44 1708.1328125
4.46 1691.66015625
4.48 1702.48828125
4.5 1712.02734375
4.52 1722.08203125
4.54 1732.91015625
4.56 1742.96484375
4.58 1752.76171875
4.6 1763.07421875
4.62 1826.49609375
4.64 1741.51953125
4.66 1741.51953125
4.68 1579.609375
4.7 1277.84765625
4.72 1233.83984375
4.74 1171.12109375
4.76 979.15625
};
\addplot [, color0, dashed]
table {%
0 3232.25546875
5.01 3232.25546875
};
\addlegendentry{expensive: $3232 \pm 32$ MB}
\addplot [, color1, dashed]
table {%
0 2600.866015625
5.01 2600.866015625
};
\addlegendentry{optimized: $2601 \pm 23$ MB}
\addplot [, color2, dashed]
table {%
0 1957.40390625
5.01 1957.40390625
};
\addlegendentry{baseline: $1957 \pm 12$ MB}
\end{axis}

\end{tikzpicture}

    \tikzexternaldisable
    \label{cockpit::fig:memory-benchmark-cifar100}
  \end{subfigure}
  \caption{\textbf{Memory consumption and savings with hooks} during one
    forward-backward step on a CPU for different \deepobs problems. We compare
    three settings; i) without \cockpit (baseline); ii) \cockpit with
    \robustInlinecode{GradHist1d} with \backpack (expensive); iii) \cockpit with
    \robustInlinecode{GradHist1d} with \backpack and additional hooks
    (optimized). Peak memory consumptions are highlighted by horizontal dashed
    bars and shown in the legend. Shaded areas, if visible, fill two standard
    deviations above and below the mean value, all of them result from ten
    independent runs. Dotted lines indicate individual runs. Our optimized
    approach allows to free obsolete tensors during backpropagation and thereby
    reduces memory consumption. From top to bottom: the effect is less
    pronounced for architectures that concentrate the majority of parameters in
    a single layer (\subfigref{cockpit::fig:memory-benchmark-fmnist} $3,274,634$
    total, $3,211,264$ largest layer) and increases for more balanced networks
    \subfigref{cockpit::fig:memory-benchmark-mnist} $1,336,610$ total, $784,000$
    largest layer, \subfigref{cockpit::fig:memory-benchmark-cifar10}: $895,210$
    total, $589,824$ largest layer).}
  \label{cockpit::fig:memory-benchmark}
\end{figure}

\captionsetup[subfigure]{justification=centering, singlelinecheck=true}

%%% Local Variables:
%%% mode: latex
%%% TeX-master: "../thesis"
%%% End:
