\ggn and Hessian are predominantly used to locally approximate the loss by a
quadratic model $q$ (see \Cref{vivit::eq:quadratic_model}). Even if the curvature's
eigenspace is completely preserved in spite of the approximations, they can
still alter the curvature \textit{magnitude} along the eigenvectors.

\subsubsection{Procedure}

\Cref{vivit::tab:cases_curvature_noise} shows the cases considered in
this experiment.

\begin{table*}[ht]
  \centering
  \caption{ \textbf{Considered cases for approximation of curvature:} We use a
    different set of cases for the approximation of the \ggn{} depending on the
    test problem. For the test problems with $C=10$, we use $M=1$ \mc-sample,
    for the \cifarhun \allcnnc test problem ($C=100$), we use $M=10$ \mc-samples
    in order to reduce the computational costs by the same factor. }
  \label{vivit::tab:cases_curvature_noise}
  \vspace{1ex}
  \begin{footnotesize}
    \begin{tabular}{llll}
      \toprule
      Problem
      & Cases \\
      \midrule
      \makecell[tl]{
      \fmnist \twoctwod \\
      \cifarten \threecthreed and \\
      \cifarten \resnetthirtytwo}
      & \makecell[tl]{
        \textbf{mb, exact} with mini-batch size $N = 128$\\
      \textbf{mb, mc} with $N=128$ and $M=1$ \mc{}-sample\\
      \textbf{sub, exact} using $16$ samples from the mini-batch\\
      \textbf{sub, mc} using $16$ samples from the mini-batch and $M=1$ \mc{}-sample
      }
      \\
      \midrule
      \cifarhun \allcnnc
      & \makecell[tl]{
        \textbf{mb, exact} with mini-batch size $N = 128$\\
      \textbf{mb, mc} with $N=128$ and $M=10$ \mc{}-samples\\
      \textbf{sub, exact} using $16$ samples from the mini-batch\\
      \textbf{sub, mc} using $16$ samples from the mini-batch and $M=10$ \mc{}-samples
      }
      \\
      \bottomrule
    \end{tabular}
  \end{footnotesize}
\end{table*}

Due to the large computational effort for evaluating the full-batch directional
derivatives, a subset of the checkpoints from \Cref{vivit::sec:ggn_vs_hessian}
is used for two problems: we use every second checkpoint for \cifarten
\resnetthirtytwo and every forth checkpoint for \cifarhun \allcnnc.

For each checkpoint and case, we compute the top-$C$ eigenvectors
$\{\ve_k\}_{k=1}^C$ of the \ggn approximation $\mG^{(\text{ap})}$ either with
\vivit or using an iterative matrix-free approach (as in
\Cref{vivit::sec:eigenspace_noise}). The second-order directional derivative of the
corresponding quadratic model along direction $\ve_k$ is then given by
$\lambda_k^{(\text{ap})} = \ve_k^\top \mG^{(\text{ap})} \ve_k$ (see
\Cref{vivit::eq:directional-derivatives}). As a reference, we compute the full-batch
\ggn $\mG^{(\text{fb})}$ and the resulting directional derivatives along the
same eigenvectors $\{\ve_k\}_{k=1}^C$, \ie $\lambda_k^{(\text{fb})} =
\ve_k^\top \mG^{(\text{fb})} \ve_k$. The average (over all $C$ directions)
relative error is given by
\begin{equation*}
  \epsilon = \frac{1}{C} \sum_{k=1}^C \frac{
    \lvert \lambda_k^{(\text{ap})} - \lambda_k^{(\text{fb})} \rvert
  }{\lambda_k^{(\text{fb})}} \, .
\end{equation*}
The procedure above is repeated on $3$ mini-batches from the training data (\ie
we obtain $3$ average relative errors for every checkpoint and case) -- except
for the \cifarhun \allcnnc test problem, where we perform only a single run to
keep the computational effort manageable.

\subsubsection{Results}

\Cref{vivit::fig:curvature_noise} shows the results. We observe similar results
as in \Cref{vivit::sec:eigenspace_noise}: with increasing computational effort,
the approximated directional derivatives become more precise and the overall
approximation quality decreases over the course of the optimization. For the
\resnetthirtytwo architecture, the average errors are particularly large.

\begin{figure*}[p]
  \centering
  \begin{minipage}[t]{0.495\linewidth}
    \centering
    {\footnotesize \sgd}
  \end{minipage}\hfill
  \begin{minipage}[t]{0.495\linewidth}
    \centering
    {\footnotesize \adam}
  \end{minipage}

  \begin{subfigure}[t]{\linewidth}
    \centering
    \caption{\fmnist \twoctwod}
    \begin{minipage}{0.50\linewidth}
      \centering
      % defines the pgfplots style "eigspacedefault"
\pgfkeys{/pgfplots/eigspacedefault/.style={
    width=1.03\linewidth,
    height=\goldenRatioInv*1.03*\linewidth,
    every axis plot/.append style={line width = 1pt},
    tick pos = left,
    ylabel near ticks,
    xlabel near ticks,
    xtick align = inside,
    ytick align = inside,
    legend cell align = left,
    legend columns = 1,
    legend pos = north east,
    legend style = {
      fill opacity = 0.9,
      text opacity = 1,
      font = \tiny,
      % column sep=0.1cm,
    },
    legend image post style={scale=2},
    xticklabel style = {font = \small},
    xlabel style = {font = \small},
    axis line style = {black},
    yticklabel style = {font = \small},
    ylabel style = {font = \small},
    title style = {font = \small},
    grid = major,
    grid style = {dashed}
  }
}

\pgfkeys{/pgfplots/eigspacedefaultapp/.style={
    eigspacedefault,
    height=0.6\linewidth,
    legend columns = 2,
  }
}

\pgfkeys{/pgfplots/eigspacenolegend/.style={
    legend image post style = {scale=0},
    legend style = {
      fill opacity = 0,
      draw opacity = 0,
      text opacity = 0,
      font = \small,
      at={(1, 1.025)},
      anchor=south east,
      column sep=0.25cm,
    },
  }
}
%%% Local Variables:
%%% mode: latex
%%% TeX-master: "../main"
%%% End:

      \pgfkeys{/pgfplots/zmystyle/.style={
          eigspacedefaultapp,
        }}
      \tikzexternalenable
      % This file was created by tikzplotlib v0.9.7.
\begin{tikzpicture}

\definecolor{color0}{rgb}{0.274509803921569,0.6,0.564705882352941}
\definecolor{color1}{rgb}{0.870588235294118,0.623529411764706,0.0862745098039216}
\definecolor{color2}{rgb}{0.501960784313725,0.184313725490196,0.6}

\begin{axis}[
axis line style={white!10!black},
legend columns=2,
legend style={fill opacity=0.8, draw opacity=1, text opacity=1, at={(0.97,0.03)}, anchor=south east, draw=white!80!black},
log basis x={10},
tick pos=left,
xlabel={epoch (log scale)},
xmajorgrids,
xmin=0.794328234724281, xmax=125.892541179417,
xmode=log,
ylabel={av. rel. error (log scale)},
ymajorgrids,
ymin=0.0129898560586076, ymax=382.170771933262,
ymode=log,
zmystyle
]
\addplot [, black, opacity=0.6, mark=*, mark size=0.5, mark options={solid}, only marks]
table {%
1 0.0351262073304685
1.04615384615385 0.117368382550381
1.0974358974359 0.332830087977687
1.14871794871795 0.172888293469627
1.2025641025641 0.280240632610728
1.26153846153846 0.402328977902066
1.32051282051282 0.248482170214628
1.38461538461538 0.180777412702239
1.44871794871795 0.277962362869217
1.51794871794872 0.294495584492285
1.58974358974359 0.345311269095769
1.66666666666667 0.341081276455096
1.74615384615385 0.401836479953737
1.82820512820513 0.406836753469868
1.91538461538462 0.291705241781577
2.00769230769231 0.373056699035261
2.1025641025641 0.39245023853267
2.20512820512821 0.400841957576626
2.30769230769231 0.245120107881536
2.41794871794872 0.295694755158932
2.53333333333333 0.544851333703763
2.65384615384615 0.433253748119535
2.78205128205128 0.534519085247783
2.91282051282051 0.471504256903435
3.05384615384615 0.605206758480856
3.1974358974359 0.582687073931798
3.35128205128205 1.15897949013726
3.51025641025641 0.558867101887748
3.67692307692308 0.710013213942829
3.85128205128205 0.580418055309316
4.03589743589744 1.0499784024624
4.22820512820513 1.06504309864556
4.42820512820513 1.27154496023177
4.64102564102564 1.28455045543329
4.86153846153846 1.60933643327705
5.09230769230769 2.39327478100793
5.33589743589744 1.34767916127237
5.58974358974359 2.26574709191488
5.85641025641026 1.84937245387021
6.13589743589744 1.9043969833596
6.42564102564103 2.85167470118379
6.73333333333333 1.58309846310343
7.05384615384615 2.97774595885425
7.38974358974359 3.25024354879336
7.74102564102564 2.89674049693418
8.11025641025641 2.30493229633105
8.4974358974359 2.67936068540252
8.9 2.30837148288972
9.32564102564103 4.2353946669187
9.76923076923077 4.16813539720868
10.2333333333333 4.17562282494711
10.7205128205128 5.35932264508914
11.2307692307692 4.19968187487231
11.7666666666667 3.23826641147675
12.3282051282051 5.29582823571468
12.9153846153846 7.19428212033536
13.5282051282051 6.04573783119228
14.174358974359 8.91088906721037
14.8487179487179 10.8292548153193
15.5564102564103 12.3148961866455
16.2974358974359 16.0455243171386
17.0717948717949 8.45521899855219
17.8846153846154 12.1130224816733
18.7358974358974 17.0689360661066
19.6282051282051 5.30325766850223
20.5641025641026 7.25868513076228
21.5435897435897 15.1838153955398
22.5692307692308 11.9981275696484
23.6435897435897 17.6845806353174
24.7692307692308 27.3720028237114
25.9487179487179 18.6813140935605
27.1846153846154 23.1744359906437
28.4794871794872 29.8401225967567
29.8358974358974 7.35169653813072
31.2564102564103 10.9779696180947
32.7435897435897 10.7820191545402
34.3025641025641 10.2099134152306
35.9358974358974 11.716694987499
37.648717948718 13.3484336877154
39.4410256410256 8.32676252666177
41.3179487179487 10.5805687557644
43.2871794871795 11.9165660552354
45.3487179487179 12.1677035292667
47.5076923076923 7.38519791805624
49.7692307692308 11.6784272420363
52.1384615384615 8.50334162670589
54.6205128205128 11.4762913463695
57.2230769230769 8.16399286960931
59.9461538461538 8.8146283491102
62.8025641025641 11.5544994743032
65.7923076923077 8.3786718172677
68.925641025641 10.1775599026885
72.2076923076923 9.70034769335658
75.6461538461539 10.7470027931458
79.2461538461538 10.3138546580804
83.0205128205128 10.3638027099249
86.974358974359 10.1207504961325
91.1153846153846 9.28258972048315
95.4538461538462 8.85799217051232
100 9.49889024408192
};
\addlegendentry{mb 128, exact}
\addplot [, black, opacity=0.6, mark=*, mark size=0.5, mark options={solid}, only marks, forget plot]
table {%
1 0.0207359976759041
1.04615384615385 0.0795830965105317
1.0974358974359 0.182904551187448
1.14871794871795 0.22057851178678
1.2025641025641 0.189000299212437
1.26153846153846 0.282572858511966
1.32051282051282 0.25659737282214
1.38461538461538 0.33216724352362
1.44871794871795 0.279171678001551
1.51794871794872 0.238536199054596
1.58974358974359 0.193966084174655
1.66666666666667 0.257260807316936
1.74615384615385 0.286567843757066
1.82820512820513 0.543592929889056
1.91538461538462 0.332548324060965
2.00769230769231 0.272518461658853
2.1025641025641 0.325548721458789
2.20512820512821 0.455499395988265
2.30769230769231 0.344689037192781
2.41794871794872 0.193412650517241
2.53333333333333 0.437580082985971
2.65384615384615 0.246371473801309
2.78205128205128 1.00827533222994
2.91282051282051 0.431021780442789
3.05384615384615 0.256090562156801
3.1974358974359 0.916663231337216
3.35128205128205 0.607765094344867
3.51025641025641 0.855994939418426
3.67692307692308 0.444167260454524
3.85128205128205 0.573561009901973
4.03589743589744 0.658155698839127
4.22820512820513 0.957329094574624
4.42820512820513 0.755611144505169
4.64102564102564 0.819685007836538
4.86153846153846 1.40140790616532
5.09230769230769 0.938334046585685
5.33589743589744 1.03165811067738
5.58974358974359 1.21389059706876
5.85641025641026 1.132050223
6.13589743589744 1.61979420489816
6.42564102564103 1.54842815214732
6.73333333333333 1.26338517705013
7.05384615384615 2.07087905999354
7.38974358974359 1.70020793977559
7.74102564102564 2.54872884089231
8.11025641025641 2.20148253439194
8.4974358974359 2.36566828343546
8.9 2.88863865069178
9.32564102564103 2.42614507003159
9.76923076923077 3.32981498754312
10.2333333333333 2.93629730840056
10.7205128205128 4.40345065033191
11.2307692307692 4.28767383489524
11.7666666666667 4.40929577081637
12.3282051282051 5.27419870787745
12.9153846153846 4.16036972946393
13.5282051282051 7.46993807220465
14.174358974359 8.39721795032555
14.8487179487179 4.59835777315297
15.5564102564103 9.04984864701387
16.2974358974359 7.81679941672984
17.0717948717949 9.37057361127805
17.8846153846154 9.89539916460368
18.7358974358974 12.2624539491399
19.6282051282051 8.64499629074318
20.5641025641026 6.50839691786916
21.5435897435897 23.2820957930678
22.5692307692308 12.3991676679291
23.6435897435897 13.8833773070962
24.7692307692308 6.79454262358585
25.9487179487179 19.8616524571523
27.1846153846154 9.55076787273993
28.4794871794872 6.62990331021153
29.8358974358974 7.35172457889361
31.2564102564103 13.8261093431471
32.7435897435897 12.3189931672849
34.3025641025641 10.5336316031156
35.9358974358974 10.3191742727079
37.648717948718 9.42611596775674
39.4410256410256 10.704787543131
41.3179487179487 9.36165798669604
43.2871794871795 9.48864184492159
45.3487179487179 10.1547697683526
47.5076923076923 11.3078609720657
49.7692307692308 9.33931327968502
52.1384615384615 8.76730396360302
54.6205128205128 9.31375785684376
57.2230769230769 10.6575322381965
59.9461538461538 9.02015070600267
62.8025641025641 9.35733617559023
65.7923076923077 9.44679758735497
68.925641025641 9.07381027171389
72.2076923076923 9.47719473512312
75.6461538461539 8.68261791819489
79.2461538461538 8.71689790971132
83.0205128205128 9.24484349054695
86.974358974359 9.34155425547195
91.1153846153846 9.5746113588626
95.4538461538462 9.42934885763034
100 8.94955245148845
};
\addplot [, black, opacity=0.6, mark=*, mark size=0.5, mark options={solid}, only marks, forget plot]
table {%
1 0.0290523275455448
1.04615384615385 0.0754928002298071
1.0974358974359 0.177789225075698
1.14871794871795 0.0971360193443167
1.2025641025641 0.126602210948458
1.26153846153846 0.118038657936116
1.32051282051282 0.139199271940722
1.38461538461538 0.0823576183293548
1.44871794871795 0.14127781048119
1.51794871794872 0.146631298318643
1.58974358974359 0.155750132720494
1.66666666666667 0.157064795517691
1.74615384615385 0.223869873528316
1.82820512820513 0.276095061218088
1.91538461538462 0.307474605164762
2.00769230769231 0.267631866160566
2.1025641025641 0.302135718633776
2.20512820512821 0.770263296482987
2.30769230769231 0.210602234022038
2.41794871794872 0.372598769991978
2.53333333333333 0.278475601051402
2.65384615384615 0.658222270463608
2.78205128205128 0.301898020997618
2.91282051282051 0.42089354949213
3.05384615384615 0.454255427221705
3.1974358974359 0.574331411433671
3.35128205128205 0.660100367615782
3.51025641025641 0.500746353678422
3.67692307692308 1.00785084155716
3.85128205128205 0.604831618721445
4.03589743589744 1.06734652851034
4.22820512820513 0.740811022312474
4.42820512820513 1.23267468906076
4.64102564102564 0.844672156485248
4.86153846153846 1.35559374938549
5.09230769230769 1.18231589484376
5.33589743589744 1.45932668493446
5.58974358974359 1.72102127100181
5.85641025641026 1.36042642066157
6.13589743589744 1.53439676848489
6.42564102564103 1.4778095050868
6.73333333333333 1.18132673511343
7.05384615384615 1.79084200672065
7.38974358974359 1.7453752105069
7.74102564102564 1.81672668355087
8.11025641025641 2.86037308344636
8.4974358974359 2.30934123071787
8.9 1.95584789723864
9.32564102564103 2.67233855308109
9.76923076923077 2.65504730879894
10.2333333333333 2.07550490053955
10.7205128205128 3.22630617234813
11.2307692307692 2.0999791991513
11.7666666666667 2.67035991113995
12.3282051282051 5.12812471084201
12.9153846153846 3.26983348105654
13.5282051282051 5.78164789185406
14.174358974359 4.80921159837561
14.8487179487179 8.54162095216993
15.5564102564103 3.28436591526379
16.2974358974359 3.09649841236202
17.0717948717949 8.7966680599859
17.8846153846154 6.7457976528822
18.7358974358974 9.97735016553684
19.6282051282051 8.74885493580152
20.5641025641026 6.95454065929939
21.5435897435897 19.5249854929307
22.5692307692308 10.9279964133409
23.6435897435897 20.0713165107262
24.7692307692308 8.16696764316673
25.9487179487179 13.9600251405656
27.1846153846154 14.0324842361857
28.4794871794872 21.1878618010674
29.8358974358974 13.3740710271348
31.2564102564103 11.9598770054296
32.7435897435897 15.2053883638936
34.3025641025641 10.7606640597336
35.9358974358974 8.74159732979116
37.648717948718 8.30205757978363
39.4410256410256 7.79657815360492
41.3179487179487 7.66174341622131
43.2871794871795 6.50660365016109
45.3487179487179 7.80196182889189
47.5076923076923 7.22524376367878
49.7692307692308 7.68726843367918
52.1384615384615 8.74170284547059
54.6205128205128 6.52288820195929
57.2230769230769 6.52986317077049
59.9461538461538 7.617725024225
62.8025641025641 7.08202116154472
65.7923076923077 7.00947946351037
68.925641025641 6.61088365703544
72.2076923076923 6.40819667943481
75.6461538461539 7.54375621420964
79.2461538461538 6.8118386918188
83.0205128205128 5.65718489250557
86.974358974359 6.27204485243495
91.1153846153846 5.90492957578569
95.4538461538462 6.1526546147858
100 6.04206570813901
};
\addplot [, color0, opacity=0.6, mark=diamond*, mark size=0.5, mark options={solid}, only marks]
table {%
1 0.227035006767813
1.04615384615385 0.316513438229382
1.0974358974359 0.259522430412647
1.14871794871795 0.936102821892269
1.2025641025641 1.4295196674988
1.26153846153846 1.45250819604232
1.32051282051282 1.28643806841221
1.38461538461538 1.87805496838567
1.44871794871795 1.41153734130421
1.51794871794872 2.12307746980996
1.58974358974359 2.6276237501323
1.66666666666667 2.99691237883876
1.74615384615385 2.76218942127386
1.82820512820513 2.76341564212595
1.91538461538462 2.83458565777057
2.00769230769231 3.0215151886905
2.1025641025641 2.56237768863075
2.20512820512821 2.98871637020433
2.30769230769231 2.14954190396027
2.41794871794872 2.90053733662978
2.53333333333333 3.41159361467424
2.65384615384615 3.04044552816868
2.78205128205128 3.39157326483331
2.91282051282051 4.05127139480311
3.05384615384615 5.20253577646584
3.1974358974359 5.56576387579576
3.35128205128205 8.98806291634682
3.51025641025641 3.06400492912264
3.67692307692308 7.93650367318318
3.85128205128205 3.74405795470555
4.03589743589744 6.83484408307626
4.22820512820513 5.45137855367746
4.42820512820513 10.1902139031629
4.64102564102564 7.17699468041263
4.86153846153846 7.38035747571018
5.09230769230769 9.59004980155486
5.33589743589744 6.51544989722917
5.58974358974359 11.5821833314531
5.85641025641026 13.7781696051138
6.13589743589744 5.12471006447061
6.42564102564103 13.9875239390353
6.73333333333333 14.8282564772351
7.05384615384615 11.2698207093361
7.38974358974359 8.83301453950666
7.74102564102564 15.3622495096943
8.11025641025641 12.8983196339952
8.4974358974359 9.59663859415697
8.9 13.9138058259472
9.32564102564103 6.26458646232718
9.76923076923077 9.80631971371472
10.2333333333333 15.4360858991247
10.7205128205128 15.6829080800089
11.2307692307692 8.5010697224318
11.7666666666667 7.89988898352573
12.3282051282051 5.36368301696937
12.9153846153846 16.9450760455376
13.5282051282051 4.91545937769922
14.174358974359 32.3433171899427
14.8487179487179 11.3453989879522
15.5564102564103 16.7252021735804
16.2974358974359 22.4191358287319
17.0717948717949 10.4143395999126
17.8846153846154 7.66626694936072
18.7358974358974 21.4431226597518
19.6282051282051 10.7444021310423
20.5641025641026 9.53009940287667
21.5435897435897 25.2006413917215
22.5692307692308 15.8756515705463
23.6435897435897 6.68972841384296
24.7692307692308 92.9680262701513
25.9487179487179 44.7630561380522
27.1846153846154 51.6763292132504
28.4794871794872 32.0947868345698
29.8358974358974 16.2013349373851
31.2564102564103 22.1517730482709
32.7435897435897 22.7633959264474
34.3025641025641 9.83084872190777
35.9358974358974 4.78115279563797
37.648717948718 4.43633834804378
39.4410256410256 3.86897852218698
41.3179487179487 3.77782666305185
43.2871794871795 2.83563237618336
45.3487179487179 2.50840527654053
47.5076923076923 3.29219866125127
49.7692307692308 2.27860819403844
52.1384615384615 1.75569672852264
54.6205128205128 1.70871423641892
57.2230769230769 1.84706022124641
59.9461538461538 1.45257418390222
62.8025641025641 1.43640531392962
65.7923076923077 1.47171616970297
68.925641025641 1.23449844862805
72.2076923076923 1.29660216277022
75.6461538461539 1.26924850262016
79.2461538461538 1.06373726341734
83.0205128205128 1.11532991477553
86.974358974359 1.05291116085904
91.1153846153846 1.02799145582854
95.4538461538462 1.00398656957761
100 1.01532684384803
};
\addlegendentry{sub 16, exact}
\addplot [, color0, opacity=0.6, mark=diamond*, mark size=0.5, mark options={solid}, only marks, forget plot]
table {%
1 0.052327790859718
1.04615384615385 0.296790156443512
1.0974358974359 0.303737971011001
1.14871794871795 0.45785711947313
1.2025641025641 0.805686635863101
1.26153846153846 1.16281599900165
1.32051282051282 0.680441919909761
1.38461538461538 0.897865434857611
1.44871794871795 1.57567375874034
1.51794871794872 1.14279251647204
1.58974358974359 1.01104956423309
1.66666666666667 1.58691974738703
1.74615384615385 1.66730993697997
1.82820512820513 2.33573602541458
1.91538461538462 3.42866745313405
2.00769230769231 2.44493045398006
2.1025641025641 1.9816251427842
2.20512820512821 2.7185443740389
2.30769230769231 3.35522411443298
2.41794871794872 1.89725634028519
2.53333333333333 2.42539947823816
2.65384615384615 2.93631512328694
2.78205128205128 2.9820121518607
2.91282051282051 3.58038656576925
3.05384615384615 2.70890278488555
3.1974358974359 3.88238632761734
3.35128205128205 2.43223418038328
3.51025641025641 3.42426489400592
3.67692307692308 4.14842458191167
3.85128205128205 4.7316280127708
4.03589743589744 4.39060251864009
4.22820512820513 4.96966903985558
4.42820512820513 3.62806745940589
4.64102564102564 4.78961091687021
4.86153846153846 4.23811506478464
5.09230769230769 5.20828504330529
5.33589743589744 5.78661519707367
5.58974358974359 5.74817875448717
5.85641025641026 5.25083779489485
6.13589743589744 4.30911090672459
6.42564102564103 8.17423553214266
6.73333333333333 7.85242216935763
7.05384615384615 12.0554495616832
7.38974358974359 9.29155581802421
7.74102564102564 10.5989020887115
8.11025641025641 16.7541208779785
8.4974358974359 10.9661984830708
8.9 14.2236573669781
9.32564102564103 10.4685636396504
9.76923076923077 11.1466609472933
10.2333333333333 24.6052611602909
10.7205128205128 7.10930788678699
11.2307692307692 13.8045479211704
11.7666666666667 20.1398881166513
12.3282051282051 22.8849510989542
12.9153846153846 18.3418323016204
13.5282051282051 23.0395390818058
14.174358974359 34.4770852539846
14.8487179487179 28.9939480409208
15.5564102564103 32.8660748438335
16.2974358974359 22.669013688294
17.0717948717949 15.4029739493053
17.8846153846154 19.4940941561023
18.7358974358974 30.8623482885348
19.6282051282051 20.0187569116705
20.5641025641026 23.001991881685
21.5435897435897 57.1645226548115
22.5692307692308 28.1311984002385
23.6435897435897 25.8477553927726
24.7692307692308 3.54247674638049
25.9487179487179 31.5683049903923
27.1846153846154 7.02181265643416
28.4794871794872 28.23293677481
29.8358974358974 31.459967933527
31.2564102564103 22.0762761938725
32.7435897435897 59.46815571912
34.3025641025641 38.4440604824272
35.9358974358974 44.7047857120106
37.648717948718 42.2433675006449
39.4410256410256 40.8032595961519
41.3179487179487 39.2696536131614
43.2871794871795 42.3073790107959
45.3487179487179 39.0589466220239
47.5076923076923 42.725231431672
49.7692307692308 35.3025420707427
52.1384615384615 29.6074918764858
54.6205128205128 42.4080331734749
57.2230769230769 41.5786898986033
59.9461538461538 34.6471838488206
62.8025641025641 36.7594405710222
65.7923076923077 35.8063440917305
68.925641025641 37.4744891363576
72.2076923076923 36.2743960802514
75.6461538461539 31.7681105714739
79.2461538461538 32.4945122773391
83.0205128205128 39.4187815603848
86.974358974359 35.3341970710486
91.1153846153846 35.6583653705796
95.4538461538462 33.6177272253783
100 33.0564364401049
};
\addplot [, color0, opacity=0.6, mark=diamond*, mark size=0.5, mark options={solid}, only marks, forget plot]
table {%
1 0.0650363340881994
1.04615384615385 0.122492682678379
1.0974358974359 0.42111898107273
1.14871794871795 0.77862546270054
1.2025641025641 0.914055532040184
1.26153846153846 0.629553173926892
1.32051282051282 1.13868961588268
1.38461538461538 1.13966083948498
1.44871794871795 0.972064811565223
1.51794871794872 2.07409398699614
1.58974358974359 1.82272025491366
1.66666666666667 2.2198849698793
1.74615384615385 1.85326129759853
1.82820512820513 1.35747985083365
1.91538461538462 2.94629262421461
2.00769230769231 2.34232681484741
2.1025641025641 2.49089927258148
2.20512820512821 3.13449945120047
2.30769230769231 2.8519641670383
2.41794871794872 6.5621057926663
2.53333333333333 3.1720528333558
2.65384615384615 3.76926905456517
2.78205128205128 2.62457995389594
2.91282051282051 2.57734275237208
3.05384615384615 4.57273755876622
3.1974358974359 3.83729305970314
3.35128205128205 3.12715930098923
3.51025641025641 4.06937827148304
3.67692307692308 3.71719506970228
3.85128205128205 8.89445898907101
4.03589743589744 3.88091446234543
4.22820512820513 3.28232828614362
4.42820512820513 4.39809442280849
4.64102564102564 3.74575495917626
4.86153846153846 5.83351361523351
5.09230769230769 2.43082947596214
5.33589743589744 5.658163062142
5.58974358974359 3.86898922270418
5.85641025641026 3.66442258374917
6.13589743589744 8.58016906893012
6.42564102564103 3.89952325678009
6.73333333333333 2.70868503377735
7.05384615384615 2.27474843513471
7.38974358974359 3.47875758344445
7.74102564102564 3.3150952730257
8.11025641025641 7.15675173082912
8.4974358974359 1.74400598071944
8.9 6.43926825377893
9.32564102564103 5.9351636022688
9.76923076923077 6.52568991625954
10.2333333333333 9.108482427163
10.7205128205128 4.21261315614523
11.2307692307692 3.59722095643824
11.7666666666667 5.51648341400205
12.3282051282051 11.3716085346281
12.9153846153846 3.18235129114247
13.5282051282051 3.70464881053494
14.174358974359 6.89144094204638
14.8487179487179 2.55294257809662
15.5564102564103 1.75361333634313
16.2974358974359 6.74104789676432
17.0717948717949 2.46773996137706
17.8846153846154 3.7984055684841
18.7358974358974 5.29502873211401
19.6282051282051 2.29043942216737
20.5641025641026 0.924347875109709
21.5435897435897 1.00778363838488
22.5692307692308 1.30862524783337
23.6435897435897 1.60530430493743
24.7692307692308 0.731949875951791
25.9487179487179 15.7098286208557
27.1846153846154 0.888661237480128
28.4794871794872 0.888661909282776
29.8358974358974 0.975775553823788
31.2564102564103 0.966896521130526
32.7435897435897 0.929679309808527
34.3025641025641 0.966823711388097
35.9358974358974 0.968320077768071
37.648717948718 0.97182173062881
39.4410256410256 0.968728625995498
41.3179487179487 0.976363404289427
43.2871794871795 0.972561393763009
45.3487179487179 0.975775375418455
47.5076923076923 0.978189330576957
49.7692307692308 0.984149619640478
52.1384615384615 0.988311390586426
54.6205128205128 0.982955806686697
57.2230769230769 0.985917006296336
59.9461538461538 0.989239983198757
62.8025641025641 0.990951537974418
65.7923076923077 0.989225834149366
68.925641025641 0.990087728115746
72.2076923076923 0.991124740129312
75.6461538461539 0.993265602102015
79.2461538461538 0.99350921481086
83.0205128205128 0.991970916270351
86.974358974359 0.993433290366261
91.1153846153846 0.993648928252822
95.4538461538462 0.993878850856773
100 0.994742437092949
};
\addplot [, color1, opacity=0.6, mark=square*, mark size=0.5, mark options={solid}, only marks]
table {%
1 0.274275646056653
1.04615384615385 0.207433623474843
1.0974358974359 0.450267354757209
1.14871794871795 0.401715111174062
1.2025641025641 0.575378719795101
1.26153846153846 0.613624270873701
1.32051282051282 0.681573106699016
1.38461538461538 0.640346814370809
1.44871794871795 0.734353206048627
1.51794871794872 1.26929239691943
1.58974358974359 1.02360469824558
1.66666666666667 1.11158108599862
1.74615384615385 1.74406300341153
1.82820512820513 0.910197032373361
1.91538461538462 1.36746007482501
2.00769230769231 1.11169550185824
2.1025641025641 1.09514735807755
2.20512820512821 1.26664667870776
2.30769230769231 1.54170546310873
2.41794871794872 1.79807105556023
2.53333333333333 1.4651500688252
2.65384615384615 2.11018753334949
2.78205128205128 1.72265459374738
2.91282051282051 1.54922627081623
3.05384615384615 1.32364579503062
3.1974358974359 2.91412386697396
3.35128205128205 2.91339811758843
3.51025641025641 2.47799937950582
3.67692307692308 1.86699774043409
3.85128205128205 2.97503648210074
4.03589743589744 2.18928365189432
4.22820512820513 1.98278623449487
4.42820512820513 2.72805903488343
4.64102564102564 1.53029838981101
4.86153846153846 3.95031996497591
5.09230769230769 2.86547893404938
5.33589743589744 2.67431922356331
5.58974358974359 3.06114851474542
5.85641025641026 2.90432370296946
6.13589743589744 3.81079248166111
6.42564102564103 4.00625530553825
6.73333333333333 3.60566317360668
7.05384615384615 4.2018210518276
7.38974358974359 3.97673432648526
7.74102564102564 4.78991939641324
8.11025641025641 2.00597903048145
8.4974358974359 4.23987970215517
8.9 4.29874596878291
9.32564102564103 6.36241086623458
9.76923076923077 6.16438594635969
10.2333333333333 7.91210872721234
10.7205128205128 6.55926085005722
11.2307692307692 7.35408198676028
11.7666666666667 11.8404403014737
12.3282051282051 6.21447675763577
12.9153846153846 7.44699760096373
13.5282051282051 3.68108869310406
14.174358974359 5.26798089704197
14.8487179487179 6.84754212581025
15.5564102564103 7.32196238160133
16.2974358974359 24.2162985295087
17.0717948717949 4.26032354567459
17.8846153846154 6.15587121374268
18.7358974358974 11.9392131542261
19.6282051282051 3.73545002522824
20.5641025641026 1.5140111704385
21.5435897435897 8.46978579307157
22.5692307692308 4.37130991091759
23.6435897435897 20.2273764454817
24.7692307692308 14.9633963266436
25.9487179487179 3.1303135878696
27.1846153846154 19.3368645897673
28.4794871794872 9.68789327683537
29.8358974358974 0.887174140467861
31.2564102564103 1.77080438835039
32.7435897435897 1.83334936490668
34.3025641025641 1.85043208215285
35.9358974358974 1.91209505611093
37.648717948718 1.77135752975223
39.4410256410256 1.4805562450074
41.3179487179487 1.63931122075057
43.2871794871795 1.42597003795212
45.3487179487179 1.46808263087411
47.5076923076923 1.18884414026934
49.7692307692308 1.58526313045002
52.1384615384615 1.22635353301722
54.6205128205128 1.32403186807427
57.2230769230769 1.19430742881197
59.9461538461538 1.27349004353571
62.8025641025641 1.36140752346847
65.7923076923077 1.16885216009587
68.925641025641 1.12292789642017
72.2076923076923 1.33429903062414
75.6461538461539 1.40101304075307
79.2461538461538 1.29974259636643
83.0205128205128 1.11873499367994
86.974358974359 1.16984918194625
91.1153846153846 1.214350500352
95.4538461538462 1.1613585445778
100 1.17531903208341
};
\addlegendentry{mb 128, mc 1}
\addplot [, color1, opacity=0.6, mark=square*, mark size=0.5, mark options={solid}, only marks, forget plot]
table {%
1 0.228876404880271
1.04615384615385 0.31913022052949
1.0974358974359 0.410908129620621
1.14871794871795 0.599136430770323
1.2025641025641 0.916782800735936
1.26153846153846 0.926948750062787
1.32051282051282 1.16941691712139
1.38461538461538 0.907210474412074
1.44871794871795 0.661275541137784
1.51794871794872 1.04828599525337
1.58974358974359 1.26681448645196
1.66666666666667 1.36026657489929
1.74615384615385 2.1445229305316
1.82820512820513 0.870837115180862
1.91538461538462 1.2583385637646
2.00769230769231 1.14914747653363
2.1025641025641 1.9357853986283
2.20512820512821 1.66798889085177
2.30769230769231 1.7019442188803
2.41794871794872 1.21498808937181
2.53333333333333 2.03640466208507
2.65384615384615 1.07151625842721
2.78205128205128 2.25745888901918
2.91282051282051 0.843886802820085
3.05384615384615 1.14790145553715
3.1974358974359 0.936913169860007
3.35128205128205 0.716553450252825
3.51025641025641 1.39653058543548
3.67692307692308 1.74062537127311
3.85128205128205 1.84011289114377
4.03589743589744 1.70322261417488
4.22820512820513 1.27826081479093
4.42820512820513 1.08793747542588
4.64102564102564 1.9820824925288
4.86153846153846 4.502956229928
5.09230769230769 2.68354942792337
5.33589743589744 3.01780386862842
5.58974358974359 1.43610196666318
5.85641025641026 3.78371654617022
6.13589743589744 1.76147446413917
6.42564102564103 2.48412838144186
6.73333333333333 0.913572421900736
7.05384615384615 6.17709548789902
7.38974358974359 1.64926724905279
7.74102564102564 6.59280355011244
8.11025641025641 3.14103516130395
8.4974358974359 6.32005255787062
8.9 2.56159961563819
9.32564102564103 3.98243432093077
9.76923076923077 10.2683317317896
10.2333333333333 4.81036162538757
10.7205128205128 4.50416047457881
11.2307692307692 2.31630824372995
11.7666666666667 3.60592438260727
12.3282051282051 3.63112156497281
12.9153846153846 1.98445579557239
13.5282051282051 4.68288541964691
14.174358974359 5.55455098998099
14.8487179487179 8.00939501541848
15.5564102564103 5.87969518271087
16.2974358974359 6.02602021623589
17.0717948717949 2.4465202079945
17.8846153846154 1.73629725289958
18.7358974358974 3.99483933470908
19.6282051282051 6.21779272461191
20.5641025641026 4.20356202876272
21.5435897435897 7.90754862755186
22.5692307692308 9.78504895456456
23.6435897435897 2.80310495399707
24.7692307692308 5.14447157543744
25.9487179487179 8.67076818164045
27.1846153846154 2.68270391811919
28.4794871794872 12.2076156809682
29.8358974358974 17.3401857184574
31.2564102564103 20.3343518503202
32.7435897435897 9.96157368145688
34.3025641025641 21.8718667675142
35.9358974358974 19.5492734666394
37.648717948718 21.7879100128546
39.4410256410256 23.7001763089885
41.3179487179487 24.008961137157
43.2871794871795 22.6676185713639
45.3487179487179 25.5920041717601
47.5076923076923 25.1025232918606
49.7692307692308 27.1518446158491
52.1384615384615 30.1494355267798
54.6205128205128 24.6748330195051
57.2230769230769 25.4907960103539
59.9461538461538 28.72919945219
62.8025641025641 28.3238007411717
65.7923076923077 28.2849387122383
68.925641025641 28.0544389990873
72.2076923076923 28.3013987466004
75.6461538461539 31.2846334212587
79.2461538461538 30.9477350024482
83.0205128205128 27.6150235354192
86.974358974359 29.8922668303241
91.1153846153846 30.0532712524457
95.4538461538462 31.137637928434
100 31.3948624093268
};
\addplot [, color1, opacity=0.6, mark=square*, mark size=0.5, mark options={solid}, only marks, forget plot]
table {%
1 0.174203257980185
1.04615384615385 0.327084341246436
1.0974358974359 0.423487316815831
1.14871794871795 0.311638287993696
1.2025641025641 0.461827208854058
1.26153846153846 0.851457790103776
1.32051282051282 0.788110906863266
1.38461538461538 1.081386640447
1.44871794871795 0.616498502605671
1.51794871794872 1.22608860632173
1.58974358974359 0.776969485351515
1.66666666666667 1.75006749903566
1.74615384615385 1.24000882378609
1.82820512820513 2.05536056872161
1.91538461538462 2.71419385037973
2.00769230769231 2.12986897545367
2.1025641025641 1.22549605773666
2.20512820512821 1.54173466086781
2.30769230769231 1.7804175151772
2.41794871794872 1.84361722489193
2.53333333333333 2.32177662140562
2.65384615384615 2.52961846458004
2.78205128205128 2.50854917338186
2.91282051282051 3.08298258091442
3.05384615384615 1.40354443398435
3.1974358974359 3.02013982987127
3.35128205128205 1.49596373452716
3.51025641025641 5.25626217216074
3.67692307692308 4.69163634619645
3.85128205128205 4.74635218836805
4.03589743589744 2.25453077339371
4.22820512820513 3.48736575510035
4.42820512820513 2.96476000471252
4.64102564102564 8.18040498333097
4.86153846153846 4.21335393786753
5.09230769230769 3.28211268568416
5.33589743589744 2.7610597930401
5.58974358974359 3.58398470862231
5.85641025641026 3.18800604659465
6.13589743589744 5.38847441635724
6.42564102564103 7.74221325755826
6.73333333333333 6.53694927364954
7.05384615384615 5.26330934580777
7.38974358974359 5.59839057096638
7.74102564102564 4.08712803564708
8.11025641025641 3.78413051903199
8.4974358974359 2.70964941716815
8.9 7.31906920827611
9.32564102564103 4.93091389562132
9.76923076923077 7.41562940777612
10.2333333333333 8.09592502846233
10.7205128205128 4.8981704812134
11.2307692307692 5.48800452076175
11.7666666666667 13.63317663664
12.3282051282051 7.6366818758671
12.9153846153846 12.5908807365735
13.5282051282051 7.34163781178564
14.174358974359 11.5804927430809
14.8487179487179 12.4931126420532
15.5564102564103 13.0104942966318
16.2974358974359 6.23038642769622
17.0717948717949 9.068485558814
17.8846153846154 23.7820123748994
18.7358974358974 21.9526647198011
19.6282051282051 36.8040109514016
20.5641025641026 5.73429391878129
21.5435897435897 18.278995363997
22.5692307692308 0.549562718279521
23.6435897435897 2.41058579643219
24.7692307692308 4.32109055849199
25.9487179487179 5.75141860028017
27.1846153846154 2.55531895126377
28.4794871794872 7.50066861395022
29.8358974358974 2.15121367356501
31.2564102564103 8.84156962468399
32.7435897435897 5.41979043643467
34.3025641025641 2.87746866206774
35.9358974358974 2.08447604446931
37.648717948718 1.98154818162617
39.4410256410256 1.82555658898162
41.3179487179487 1.70519813403658
43.2871794871795 1.45353058961962
45.3487179487179 1.84829147741345
47.5076923076923 1.58842976243246
49.7692307692308 1.82294765917897
52.1384615384615 2.36833968158253
54.6205128205128 1.48601073500094
57.2230769230769 1.4630783365944
59.9461538461538 1.90022508780103
62.8025641025641 1.65210878866211
65.7923076923077 1.6675003103864
68.925641025641 1.51217795024602
72.2076923076923 1.46145052581046
75.6461538461539 1.83150354189024
79.2461538461538 1.56181922279914
83.0205128205128 1.24462930966052
86.974358974359 1.40646572595021
91.1153846153846 1.30679664407336
95.4538461538462 1.37455399198407
100 1.32617823974345
};
\addplot [, color2, opacity=0.6, mark=triangle*, mark size=0.5, mark options={solid,rotate=180}, only marks]
table {%
1 1.36272058157939
1.04615384615385 1.48993785895236
1.0974358974359 1.71781665510187
1.14871794871795 2.92366743180457
1.2025641025641 2.85770564530756
1.26153846153846 5.06638083675658
1.32051282051282 2.72443033816649
1.38461538461538 2.63184179683782
1.44871794871795 5.50614028775545
1.51794871794872 2.86807941056231
1.58974358974359 5.744822032282
1.66666666666667 2.99434821793119
1.74615384615385 4.86463536282151
1.82820512820513 2.87862401502914
1.91538461538462 5.20470446450866
2.00769230769231 6.26693144595028
2.1025641025641 5.62525799415648
2.20512820512821 2.09077334252039
2.30769230769231 4.44708733088337
2.41794871794872 3.93760916622263
2.53333333333333 6.13721808420857
2.65384615384615 3.88096070835773
2.78205128205128 10.6290458037776
2.91282051282051 5.05977257230803
3.05384615384615 9.80466878868898
3.1974358974359 8.51139315388769
3.35128205128205 14.2179160147715
3.51025641025641 7.19828617920183
3.67692307692308 6.35039018279636
3.85128205128205 1.81064419807254
4.03589743589744 2.79849246788672
4.22820512820513 3.7930985288606
4.42820512820513 15.9035469613989
4.64102564102564 4.28323674761786
4.86153846153846 31.3159068483163
5.09230769230769 17.0797888597683
5.33589743589744 4.87892134487343
5.58974358974359 17.9918689700714
5.85641025641026 9.1617102492061
6.13589743589744 15.3981453052923
6.42564102564103 10.1196753801284
6.73333333333333 15.5880064341807
7.05384615384615 2.44062932109436
7.38974358974359 2.74123092788599
7.74102564102564 16.1116169334752
8.11025641025641 2.29709930631457
8.4974358974359 2.02634073442918
8.9 14.5251727157429
9.32564102564103 5.12835086355098
9.76923076923077 9.54027120038598
10.2333333333333 6.73012630658521
10.7205128205128 1.59534884560772
11.2307692307692 6.34447669852422
11.7666666666667 7.38686561502284
12.3282051282051 1.24547017858662
12.9153846153846 22.6418317647505
13.5282051282051 12.5727341017938
14.174358974359 2.48736391129778
14.8487179487179 18.7343277500116
15.5564102564103 6.08359177863452
16.2974358974359 32.3841603612684
17.0717948717949 2.14294322353968
17.8846153846154 0.901404714460617
18.7358974358974 17.4186425877619
19.6282051282051 1.13095366960343
20.5641025641026 1.1570926324197
21.5435897435897 1.55965948480901
22.5692307692308 0.968477228847536
23.6435897435897 0.875810602579673
24.7692307692308 32.2521530430717
25.9487179487179 2.31567695139143
27.1846153846154 83.962149792395
28.4794871794872 4.24034825350576
29.8358974358974 1.2264597196908
31.2564102564103 0.940237631376431
32.7435897435897 1.74239612491985
34.3025641025641 0.931665793701455
35.9358974358974 0.974068907921356
37.648717948718 0.975823846688799
39.4410256410256 0.982470011603991
41.3179487179487 0.981910022150547
43.2871794871795 0.99147455327616
45.3487179487179 0.993277109737669
47.5076923076923 0.986349376586111
49.7692307692308 0.99436661959003
52.1384615384615 0.997454662956935
54.6205128205128 0.996934922531487
57.2230769230769 0.996268156697645
59.9461538461538 0.998265732583671
62.8025641025641 0.998222608347622
65.7923076923077 0.998152660065129
68.925641025641 0.998957926554677
72.2076923076923 0.998849749223457
75.6461538461539 0.998836101762345
79.2461538461538 0.99925746671666
83.0205128205128 0.999334202652745
86.974358974359 0.999438502038625
91.1153846153846 0.999496767510559
95.4538461538462 0.999534792659702
100 0.999564427765288
};
\addlegendentry{sub 16, mc 1}
\addplot [, color2, opacity=0.6, mark=triangle*, mark size=0.5, mark options={solid,rotate=180}, only marks, forget plot]
table {%
1 1.30727301944784
1.04615384615385 0.684058146311858
1.0974358974359 1.73398729829122
1.14871794871795 1.92805622654194
1.2025641025641 1.39890786821088
1.26153846153846 2.05419060840114
1.32051282051282 1.53875191878754
1.38461538461538 3.40305685297066
1.44871794871795 1.61114456904638
1.51794871794872 3.35464963737471
1.58974358974359 1.9380069187476
1.66666666666667 3.08743766078126
1.74615384615385 3.48883576614889
1.82820512820513 2.29221909166878
1.91538461538462 1.75299893148092
2.00769230769231 2.76205435726332
2.1025641025641 2.1088529157797
2.20512820512821 2.07088242230925
2.30769230769231 3.38926638140721
2.41794871794872 2.29191769759646
2.53333333333333 1.90515676841933
2.65384615384615 3.69892527747458
2.78205128205128 1.76615643255597
2.91282051282051 3.01141719050454
3.05384615384615 4.03307897415727
3.1974358974359 2.93781089161111
3.35128205128205 4.66587399660292
3.51025641025641 6.04865989531739
3.67692307692308 3.37185161061931
3.85128205128205 3.93086615138631
4.03589743589744 1.71923466773
4.22820512820513 2.93548735979607
4.42820512820513 3.53541377679375
4.64102564102564 5.18207138245533
4.86153846153846 4.74114041466963
5.09230769230769 9.11992226727932
5.33589743589744 11.4060947334817
5.58974358974359 0.946811917279299
5.85641025641026 6.65873333499714
6.13589743589744 4.44239856557415
6.42564102564103 2.23752054603048
6.73333333333333 1.66510895932732
7.05384615384615 12.0626264985621
7.38974358974359 5.0338694012171
7.74102564102564 4.30744117997765
8.11025641025641 16.9615866379745
8.4974358974359 6.62339904346535
8.9 7.15766133677512
9.32564102564103 14.4882498276329
9.76923076923077 34.0430567012264
10.2333333333333 15.8142592412021
10.7205128205128 17.248485914963
11.2307692307692 9.77118579693578
11.7666666666667 14.4084990755201
12.3282051282051 13.6619401586504
12.9153846153846 8.40603430511737
13.5282051282051 26.0510085683531
14.174358974359 37.1286933665923
14.8487179487179 28.5905702399354
15.5564102564103 21.3670997502418
16.2974358974359 17.9548209317171
17.0717948717949 4.54980554305894
17.8846153846154 4.65726791293186
18.7358974358974 12.1761569997543
19.6282051282051 39.8479852646225
20.5641025641026 7.24145774616095
21.5435897435897 23.754204274342
22.5692307692308 63.3667537710769
23.6435897435897 12.3221445767128
24.7692307692308 0.924864096242508
25.9487179487179 34.9948574640667
27.1846153846154 0.891410119544645
28.4794871794872 93.212352588078
29.8358974358974 129.922675608661
31.2564102564103 130.711255653108
32.7435897435897 67.405653748609
34.3025641025641 156.703411158974
35.9358974358974 143.288634705201
37.648717948718 163.584788638217
39.4410256410256 172.689026882652
41.3179487179487 181.52625445957
43.2871794871795 170.488518514549
45.3487179487179 189.356730422155
47.5076923076923 182.349399147557
49.7692307692308 204.581497937727
52.1384615384615 226.620336585263
54.6205128205128 187.679205544007
57.2230769230769 190.625638498689
59.9461538461538 218.837343988946
62.8025641025641 216.319046510783
65.7923076923077 216.01096299377
68.925641025641 214.818764238861
72.2076923076923 215.724118265591
75.6461538461539 239.40701551046
79.2461538461538 236.834724170978
83.0205128205128 nan
86.974358974359 nan
91.1153846153846 nan
95.4538461538462 235.949679820454
100 nan
};
\addplot [, color2, opacity=0.6, mark=triangle*, mark size=0.5, mark options={solid,rotate=180}, only marks, forget plot]
table {%
1 1.67792765331684
1.04615384615385 2.02074174160216
1.0974358974359 1.85143395786432
1.14871794871795 4.85672921540076
1.2025641025641 1.59588660643541
1.26153846153846 4.1529935146196
1.32051282051282 3.30759898680838
1.38461538461538 4.77504833752661
1.44871794871795 4.33885346249826
1.51794871794872 3.78576689689778
1.58974358974359 1.61629301271408
1.66666666666667 4.26335836007048
1.74615384615385 5.44060475828895
1.82820512820513 3.08187596273456
1.91538461538462 5.5515752073002
2.00769230769231 5.93237585307205
2.1025641025641 3.21205162243742
2.20512820512821 3.66227641920114
2.30769230769231 6.14675531005805
2.41794871794872 6.70237514537012
2.53333333333333 3.24491498743741
2.65384615384615 5.52205386504533
2.78205128205128 3.95218526500149
2.91282051282051 3.72586891064921
3.05384615384615 3.23494610551876
3.1974358974359 9.19039811050503
3.35128205128205 4.17405424437683
3.51025641025641 6.78580880893377
3.67692307692308 6.80191600795478
3.85128205128205 3.98404873831457
4.03589743589744 2.63705908979859
4.22820512820513 7.76230835092954
4.42820512820513 7.69469948743786
4.64102564102564 3.1239890633589
4.86153846153846 4.16039238590485
5.09230769230769 1.28338922462601
5.33589743589744 2.71274971501325
5.58974358974359 3.57925196616434
5.85641025641026 0.976741591868937
6.13589743589744 4.21144336413233
6.42564102564103 3.94838421331557
6.73333333333333 1.56029284027433
7.05384615384615 2.94647174759664
7.38974358974359 2.17533987908667
7.74102564102564 4.45966818438342
8.11025641025641 3.01588521935701
8.4974358974359 1.33379987775397
8.9 7.51007949555732
9.32564102564103 1.41189826147353
9.76923076923077 4.31546005615469
10.2333333333333 5.1279512528297
10.7205128205128 1.44619456175759
11.2307692307692 0.814372476491599
11.7666666666667 4.7330700282335
12.3282051282051 2.93286008839934
12.9153846153846 1.69451290668953
13.5282051282051 1.22336771946248
14.174358974359 2.77503837321428
14.8487179487179 0.981039408459802
15.5564102564103 0.934997286903345
16.2974358974359 1.17223977391154
17.0717948717949 0.92490545332403
17.8846153846154 1.27667169343624
18.7358974358974 0.927739373905511
19.6282051282051 20.3211356653181
20.5641025641026 0.997751597092812
21.5435897435897 0.997947342994244
22.5692307692308 0.993757981748462
23.6435897435897 0.992960420262797
24.7692307692308 0.998581824407395
25.9487179487179 43.0026796300101
27.1846153846154 0.998600601142337
28.4794871794872 0.999859145800717
29.8358974358974 0.99978779726976
31.2564102564103 0.99999889843031
32.7435897435897 0.999989260419765
34.3025641025641 0.999997878954849
35.9358974358974 0.999998368767689
37.648717948718 0.999998413716468
39.4410256410256 0.999997610254913
41.3179487179487 0.999998716598778
43.2871794871795 0.999998179466316
45.3487179487179 0.999998640700166
47.5076923076923 0.999998935111843
49.7692307692308 0.999999487303639
52.1384615384615 0.999999775595182
54.6205128205128 0.999999326398968
57.2230769230769 0.999999580196499
59.9461538461538 0.999999781402747
62.8025641025641 0.999999856000554
65.7923076923077 0.999999774567762
68.925641025641 0.999999813210065
72.2076923076923 0.999999850049604
75.6461538461539 0.999999920007785
79.2461538461538 0.999999929290669
83.0205128205128 0.999999870123452
86.974358974359 0.999999922314703
91.1153846153846 0.999999923745816
95.4538461538462 0.99999992762644
100 0.999999949377613
};
\end{axis}

\end{tikzpicture}

      \tikzexternaldisable
    \end{minipage}\hfill
    \begin{minipage}{0.50\linewidth}
      \centering
      % defines the pgfplots style "eigspacedefault"
\pgfkeys{/pgfplots/eigspacedefault/.style={
    width=1.03\linewidth,
    height=\goldenRatioInv*1.03*\linewidth,
    every axis plot/.append style={line width = 1pt},
    tick pos = left,
    ylabel near ticks,
    xlabel near ticks,
    xtick align = inside,
    ytick align = inside,
    legend cell align = left,
    legend columns = 1,
    legend pos = north east,
    legend style = {
      fill opacity = 0.9,
      text opacity = 1,
      font = \tiny,
      % column sep=0.1cm,
    },
    legend image post style={scale=2},
    xticklabel style = {font = \small},
    xlabel style = {font = \small},
    axis line style = {black},
    yticklabel style = {font = \small},
    ylabel style = {font = \small},
    title style = {font = \small},
    grid = major,
    grid style = {dashed}
  }
}

\pgfkeys{/pgfplots/eigspacedefaultapp/.style={
    eigspacedefault,
    height=0.6\linewidth,
    legend columns = 2,
  }
}

\pgfkeys{/pgfplots/eigspacenolegend/.style={
    legend image post style = {scale=0},
    legend style = {
      fill opacity = 0,
      draw opacity = 0,
      text opacity = 0,
      font = \small,
      at={(1, 1.025)},
      anchor=south east,
      column sep=0.25cm,
    },
  }
}
%%% Local Variables:
%%% mode: latex
%%% TeX-master: "../main"
%%% End:

      \pgfkeys{/pgfplots/zmystyle/.style={
          eigspacedefaultapp,
        }}
      \tikzexternalenable
      % This file was created by tikzplotlib v0.9.7.
\begin{tikzpicture}

\definecolor{color0}{rgb}{0.274509803921569,0.6,0.564705882352941}
\definecolor{color1}{rgb}{0.870588235294118,0.623529411764706,0.0862745098039216}
\definecolor{color2}{rgb}{0.501960784313725,0.184313725490196,0.6}

\begin{axis}[
axis line style={white!10!black},
legend columns=2,
legend style={fill opacity=0.8, draw opacity=1, text opacity=1, at={(0.97,0.03)}, anchor=south east, draw=white!80!black},
log basis x={10},
tick pos=left,
xlabel={epoch (log scale)},
xmajorgrids,
xmin=0.794328234724281, xmax=125.892541179417,
xmode=log,
ylabel={av. rel. error (log scale)},
ymajorgrids,
ymin=0.00405880024879513, ymax=45.3220560247346,
ymode=log,
zmystyle
]
\addplot [, black, opacity=0.6, mark=*, mark size=0.5, mark options={solid}, only marks]
table {%
1 0.0198810451389297
1.04615384615385 0.0312718835317677
1.0974358974359 0.0411081759585323
1.14871794871795 0.0662622254393964
1.2025641025641 0.0472311998216718
1.26153846153846 0.0726323060297243
1.32051282051282 0.085097832233068
1.38461538461538 0.0875988814700559
1.44871794871795 0.0880336990139601
1.51794871794872 0.0935679759143635
1.58974358974359 0.117557085431165
1.66666666666667 0.138367738241827
1.74615384615385 0.143763053692959
1.82820512820513 0.149020276377624
1.91538461538462 0.10220270092033
2.00769230769231 0.146969559921581
2.1025641025641 0.180687987052484
2.20512820512821 0.197632069708014
2.30769230769231 0.157222900665102
2.41794871794872 0.16517152522414
2.53333333333333 0.206253992525264
2.65384615384615 0.230650485774643
2.78205128205128 0.337765079052956
2.91282051282051 0.168889771204794
3.05384615384615 0.174637100887941
3.1974358974359 0.23715436226045
3.35128205128205 0.213860368246833
3.51025641025641 0.234734703557337
3.67692307692308 0.251307737283523
3.85128205128205 0.291063003942446
4.03589743589744 0.232921498744789
4.22820512820513 0.220860881177696
4.42820512820513 0.249896669140032
4.64102564102564 0.269292257302126
4.86153846153846 0.294868832607944
5.09230769230769 0.190682525782168
5.33589743589744 0.197741879812318
5.58974358974359 0.21135771266412
5.85641025641026 0.198828447178304
6.13589743589744 0.28195114311032
6.42564102564103 0.174098525480257
6.73333333333333 0.245618163969512
7.05384615384615 0.298914813117619
7.38974358974359 0.227553586648616
7.74102564102564 0.303715033425136
8.11025641025641 0.210319260456201
8.4974358974359 0.258486228549664
8.9 0.197649941287551
9.32564102564103 0.283304536276713
9.76923076923077 0.218143768620902
10.2333333333333 0.238795895020029
10.7205128205128 0.420632894279687
11.2307692307692 0.251820353482581
11.7666666666667 0.368965018391623
12.3282051282051 0.335293898759609
12.9153846153846 0.282977660626199
13.5282051282051 0.396335201627295
14.174358974359 0.335855207144244
14.8487179487179 0.445908607303794
15.5564102564103 0.504438307523102
16.2974358974359 0.523337386502118
17.0717948717949 0.590061173926648
17.8846153846154 0.561067548539683
18.7358974358974 0.731673290747603
19.6282051282051 0.516778223100629
20.5641025641026 0.815377097984998
21.5435897435897 0.716358151522121
22.5692307692308 1.03098722052079
23.6435897435897 1.0673614484693
24.7692307692308 1.1442575104017
25.9487179487179 0.833664431561671
27.1846153846154 1.34692056487511
28.4794871794872 1.38289172284556
29.8358974358974 1.51765931343388
31.2564102564103 1.64021795031257
32.7435897435897 1.13154106192631
34.3025641025641 1.08586476091974
35.9358974358974 2.52523531794198
37.648717948718 2.10632075290087
39.4410256410256 2.59829257232618
41.3179487179487 2.43989818888396
43.2871794871795 2.71984937316495
45.3487179487179 3.54750754129008
47.5076923076923 3.4520479794498
49.7692307692308 3.32831980408473
52.1384615384615 3.10769540018117
54.6205128205128 4.32403347022914
57.2230769230769 5.55510791189318
59.9461538461538 6.51349995195175
62.8025641025641 3.86066335071297
65.7923076923077 1.57455668805495
68.925641025641 3.31236464899414
72.2076923076923 2.02913767982582
75.6461538461539 4.50814089796017
79.2461538461538 6.46367108904072
83.0205128205128 2.90475046166579
86.974358974359 7.28226991964941
91.1153846153846 2.57820434039059
95.4538461538462 3.68655226419994
100 6.55651481264116
};
\addlegendentry{mb 128, exact}
\addplot [, black, opacity=0.6, mark=*, mark size=0.5, mark options={solid}, only marks, forget plot]
table {%
1 0.0100508375779495
1.04615384615385 0.0285317076593556
1.0974358974359 0.0579722434423199
1.14871794871795 0.0918112916402571
1.2025641025641 0.0826267227540798
1.26153846153846 0.0840604207179099
1.32051282051282 0.0742799506667483
1.38461538461538 0.108953030384255
1.44871794871795 0.117084287907708
1.51794871794872 0.109457388392687
1.58974358974359 0.128079322016811
1.66666666666667 0.129835123458248
1.74615384615385 0.12947510650389
1.82820512820513 0.151727986818916
1.91538461538462 0.136759661081131
2.00769230769231 0.173727798783826
2.1025641025641 0.188500105603106
2.20512820512821 0.161053609643188
2.30769230769231 0.120963198627732
2.41794871794872 0.169585844701795
2.53333333333333 0.16611972127608
2.65384615384615 0.186219925766753
2.78205128205128 0.183137312395906
2.91282051282051 0.159427739733555
3.05384615384615 0.204028176288927
3.1974358974359 0.162684491726823
3.35128205128205 0.184373544784272
3.51025641025641 0.177473612578583
3.67692307692308 0.19074801469158
3.85128205128205 0.233766306283141
4.03589743589744 0.218785421398963
4.22820512820513 0.236639543283659
4.42820512820513 0.226205489010189
4.64102564102564 0.19860417088084
4.86153846153846 0.210377294932977
5.09230769230769 0.230428263779206
5.33589743589744 0.163482195113867
5.58974358974359 0.180328885692728
5.85641025641026 0.192105649038382
6.13589743589744 0.188294256110901
6.42564102564103 0.193738917060014
6.73333333333333 0.20943081356593
7.05384615384615 0.218525074292402
7.38974358974359 0.155438461303218
7.74102564102564 0.216896753032971
8.11025641025641 0.23546509918515
8.4974358974359 0.151450341608068
8.9 0.18291651994314
9.32564102564103 0.170779828996019
9.76923076923077 0.276903550194696
10.2333333333333 0.229002358237941
10.7205128205128 0.19275148644549
11.2307692307692 0.207715574444722
11.7666666666667 0.216764817799257
12.3282051282051 0.339935922527051
12.9153846153846 0.413528024062436
13.5282051282051 0.28145773041212
14.174358974359 0.415760070913872
14.8487179487179 0.299648300676693
15.5564102564103 0.569375924684554
16.2974358974359 0.313439643517339
17.0717948717949 0.414527564099092
17.8846153846154 0.304006454195328
18.7358974358974 0.587323157360087
19.6282051282051 0.596119773065579
20.5641025641026 0.395651373501048
21.5435897435897 0.75687521628491
22.5692307692308 0.479955241777455
23.6435897435897 0.591135150162068
24.7692307692308 0.44308852590877
25.9487179487179 0.908728952162651
27.1846153846154 0.521381665883547
28.4794871794872 0.71345858942766
29.8358974358974 0.924363631724976
31.2564102564103 0.830860158623733
32.7435897435897 1.3423146492814
34.3025641025641 0.879895327784737
35.9358974358974 0.780387263411599
37.648717948718 1.28447259567814
39.4410256410256 0.768360827367967
41.3179487179487 1.34311629703507
43.2871794871795 1.4328393946931
45.3487179487179 1.84795626191182
47.5076923076923 0.751999838412848
49.7692307692308 0.434849596187653
52.1384615384615 1.15825550611208
54.6205128205128 0.689849202931852
57.2230769230769 0.449691108752751
59.9461538461538 1.51251269782159
62.8025641025641 3.09950695575225
65.7923076923077 0.825769368753227
68.925641025641 0.219024505370509
72.2076923076923 0.840128883309506
75.6461538461539 0.69458453503927
79.2461538461538 0.509553381195843
83.0205128205128 1.32646721270942
86.974358974359 0.584897802707003
91.1153846153846 0.501597991194824
95.4538461538462 0.737093182950426
100 0.531718621510623
};
\addplot [, black, opacity=0.6, mark=*, mark size=0.5, mark options={solid}, only marks, forget plot]
table {%
1 0.0062000283439144
1.04615384615385 0.0264973573131877
1.0974358974359 0.0492406660592975
1.14871794871795 0.0839483554718766
1.2025641025641 0.113457136599074
1.26153846153846 0.0766638836202558
1.32051282051282 0.11003368424914
1.38461538461538 0.119393630029584
1.44871794871795 0.139241997069781
1.51794871794872 0.177221534608756
1.58974358974359 0.178940107755004
1.66666666666667 0.121781899649384
1.74615384615385 0.132539895299595
1.82820512820513 0.129615826646139
1.91538461538462 0.0887441918990191
2.00769230769231 0.151996037739241
2.1025641025641 0.135489396703091
2.20512820512821 0.116991497135156
2.30769230769231 0.0869906268244851
2.41794871794872 0.127879490468381
2.53333333333333 0.11887544911059
2.65384615384615 0.128845318957303
2.78205128205128 0.128283951190434
2.91282051282051 0.112233909715696
3.05384615384615 0.0915774119761065
3.1974358974359 0.147800528196273
3.35128205128205 0.168247559229073
3.51025641025641 0.1357516983994
3.67692307692308 0.132503705991366
3.85128205128205 0.158802721361461
4.03589743589744 0.141775121613749
4.22820512820513 0.213584411129573
4.42820512820513 0.183613191462771
4.64102564102564 0.11969124823857
4.86153846153846 0.141014627923639
5.09230769230769 0.190948115250021
5.33589743589744 0.231081228439242
5.58974358974359 0.208330813320706
5.85641025641026 0.205657577310279
6.13589743589744 0.130972780864554
6.42564102564103 0.218872164041043
6.73333333333333 0.212595624775541
7.05384615384615 0.183930767170672
7.38974358974359 0.211073354101878
7.74102564102564 0.2977490835426
8.11025641025641 0.249911585070129
8.4974358974359 0.146671740091751
8.9 0.183309253823951
9.32564102564103 0.221943704053573
9.76923076923077 0.175842087665546
10.2333333333333 0.166720597582533
10.7205128205128 0.236350998249719
11.2307692307692 0.17383266293045
11.7666666666667 0.197889647582102
12.3282051282051 0.242428823560109
12.9153846153846 0.230527484698374
13.5282051282051 0.252747883273741
14.174358974359 0.315256524274162
14.8487179487179 0.247260082820699
15.5564102564103 0.311691266337345
16.2974358974359 0.395868488520442
17.0717948717949 0.523511422317233
17.8846153846154 0.386776849500396
18.7358974358974 0.251554120579496
19.6282051282051 0.573205122217316
20.5641025641026 0.598206554381486
21.5435897435897 0.602876491896791
22.5692307692308 0.475112384926119
23.6435897435897 0.706719049031841
24.7692307692308 0.866561555642224
25.9487179487179 0.861575595283367
27.1846153846154 0.602164319435198
28.4794871794872 0.818051853389619
29.8358974358974 0.91182528334284
31.2564102564103 0.956995975260477
32.7435897435897 1.38289814421317
34.3025641025641 0.977657963676835
35.9358974358974 1.13763431302244
37.648717948718 1.43106587605091
39.4410256410256 0.867572227934018
41.3179487179487 1.63745279156313
43.2871794871795 1.87141018955946
45.3487179487179 2.56303726224533
47.5076923076923 1.59956530993317
49.7692307692308 2.03719952480799
52.1384615384615 1.38910037914735
54.6205128205128 1.84159515751664
57.2230769230769 2.84554027998052
59.9461538461538 1.70176392211799
62.8025641025641 1.3589494450561
65.7923076923077 1.63866559393782
68.925641025641 1.66333133579854
72.2076923076923 0.999318619248265
75.6461538461539 1.35890448759411
79.2461538461538 4.59485838097344
83.0205128205128 3.04497404657887
86.974358974359 0.556783993679935
91.1153846153846 2.30332894662385
95.4538461538462 3.98251551324129
100 1.00358674997415
};
\addplot [, color0, opacity=0.6, mark=diamond*, mark size=0.5, mark options={solid}, only marks]
table {%
1 0.110119925821807
1.04615384615385 0.146144335632971
1.0974358974359 0.214761642789691
1.14871794871795 0.225102845729453
1.2025641025641 0.266756951876773
1.26153846153846 0.308533562752924
1.32051282051282 0.278361838760739
1.38461538461538 0.321502253680442
1.44871794871795 0.335848645407708
1.51794871794872 0.302425822114389
1.58974358974359 0.384160727938741
1.66666666666667 0.299886337787575
1.74615384615385 0.308501272239401
1.82820512820513 0.396532188887747
1.91538461538462 0.446788836382854
2.00769230769231 0.424365213415068
2.1025641025641 0.45407711983157
2.20512820512821 0.487403737488134
2.30769230769231 0.380323761670547
2.41794871794872 0.436475980216679
2.53333333333333 0.535113626692305
2.65384615384615 0.438464422790879
2.78205128205128 0.384469203726523
2.91282051282051 0.482185301211905
3.05384615384615 0.481022141667236
3.1974358974359 0.385675203634334
3.35128205128205 0.595174451634071
3.51025641025641 0.556234623846736
3.67692307692308 0.548943210288408
3.85128205128205 0.526312140324987
4.03589743589744 0.398339021342121
4.22820512820513 0.468968831423352
4.42820512820513 0.423054940248519
4.64102564102564 0.746220514771956
4.86153846153846 0.347379938259253
5.09230769230769 0.582194467817629
5.33589743589744 0.445417965293713
5.58974358974359 0.575066475406796
5.85641025641026 0.47141534626028
6.13589743589744 0.282563520005487
6.42564102564103 0.435991353396146
6.73333333333333 0.37056569866727
7.05384615384615 0.535719350495582
7.38974358974359 0.500379074510457
7.74102564102564 0.441794922523291
8.11025641025641 0.409560295819433
8.4974358974359 0.58120854599374
8.9 0.352105963025041
9.32564102564103 0.462672919214911
9.76923076923077 0.53201759041331
10.2333333333333 0.486895122576845
10.7205128205128 0.350817380620001
11.2307692307692 0.353589918844854
11.7666666666667 0.542270749509376
12.3282051282051 0.528615557409515
12.9153846153846 0.688157070594779
13.5282051282051 0.760158120389272
14.174358974359 0.538931448772747
14.8487179487179 0.721845663151091
15.5564102564103 0.775259842050675
16.2974358974359 1.06776703752695
17.0717948717949 0.80586974445923
17.8846153846154 0.733632570440781
18.7358974358974 0.763498954665459
19.6282051282051 1.12327758633276
20.5641025641026 1.1462753146868
21.5435897435897 0.832836015239825
22.5692307692308 1.46768342895398
23.6435897435897 2.07000272447184
24.7692307692308 1.98461295243368
25.9487179487179 1.55457896403807
27.1846153846154 1.4437520643523
28.4794871794872 1.04563716318674
29.8358974358974 1.40994152668487
31.2564102564103 1.70787181866676
32.7435897435897 2.15731771632964
34.3025641025641 1.29010893378494
35.9358974358974 1.69012711807883
37.648717948718 0.938552588120481
39.4410256410256 0.955499851982998
41.3179487179487 1.48926746155322
43.2871794871795 1.14489878311323
45.3487179487179 2.20145314968553
47.5076923076923 2.76024586659041
49.7692307692308 3.40077339733973
52.1384615384615 2.9293801017785
54.6205128205128 3.72276427306528
57.2230769230769 1.63228859852107
59.9461538461538 2.18884065584054
62.8025641025641 3.58606603813917
65.7923076923077 0.80507298978603
68.925641025641 1.82949867536532
72.2076923076923 1.96846761936133
75.6461538461539 1.24678075377776
79.2461538461538 0.698061863149631
83.0205128205128 0.958781535341168
86.974358974359 1.16288579867389
91.1153846153846 0.646500552308439
95.4538461538462 0.710384218465669
100 0.753410065330058
};
\addlegendentry{sub 16, exact}
\addplot [, color0, opacity=0.6, mark=diamond*, mark size=0.5, mark options={solid}, only marks, forget plot]
table {%
1 0.252375827303188
1.04615384615385 0.207235630935308
1.0974358974359 0.263412368428133
1.14871794871795 0.329386183527384
1.2025641025641 0.358564691873958
1.26153846153846 0.36550097877262
1.32051282051282 0.406929743693464
1.38461538461538 0.406536780785224
1.44871794871795 0.21643446907721
1.51794871794872 0.41997787280034
1.58974358974359 0.436533936655289
1.66666666666667 0.405302328002827
1.74615384615385 0.366633756631219
1.82820512820513 0.366654104058438
1.91538461538462 0.409439871236007
2.00769230769231 0.323878107925613
2.1025641025641 0.418797393138693
2.20512820512821 0.324320159570111
2.30769230769231 0.623177351450498
2.41794871794872 0.402818349980471
2.53333333333333 0.43419956436059
2.65384615384615 0.481821252940685
2.78205128205128 0.824951072509737
2.91282051282051 0.423455742861515
3.05384615384615 0.505917150554669
3.1974358974359 0.715030028985791
3.35128205128205 0.414869355358035
3.51025641025641 0.520073586524223
3.67692307692308 0.705971072069563
3.85128205128205 0.687708779106374
4.03589743589744 0.610009753372444
4.22820512820513 0.872171700434215
4.42820512820513 0.588453575800805
4.64102564102564 0.703693374614373
4.86153846153846 0.569930785969381
5.09230769230769 0.866772510655339
5.33589743589744 0.470752004520193
5.58974358974359 0.634356428302849
5.85641025641026 0.597759238058933
6.13589743589744 0.652876725660567
6.42564102564103 0.561558438643314
6.73333333333333 0.418568276092174
7.05384615384615 0.536523234711956
7.38974358974359 0.441579899260064
7.74102564102564 0.437045893349236
8.11025641025641 0.65550255079202
8.4974358974359 1.12985660441912
8.9 0.64134066201357
9.32564102564103 0.653444903381156
9.76923076923077 0.774872958972466
10.2333333333333 0.682079352130888
10.7205128205128 0.784115765233989
11.2307692307692 0.78595584728257
11.7666666666667 0.503917877437679
12.3282051282051 0.741352684680335
12.9153846153846 0.84169414642731
13.5282051282051 0.983594992466669
14.174358974359 0.647695819107705
14.8487179487179 1.00806270407288
15.5564102564103 1.01161722619511
16.2974358974359 0.594876434561599
17.0717948717949 1.03776774432905
17.8846153846154 1.00199405710433
18.7358974358974 1.4406222756057
19.6282051282051 1.33423937888183
20.5641025641026 1.28936581074796
21.5435897435897 1.03772465654509
22.5692307692308 0.757583675079615
23.6435897435897 1.62519994725237
24.7692307692308 1.21695107265271
25.9487179487179 1.10652120799404
27.1846153846154 1.16127887718828
28.4794871794872 1.01478471120428
29.8358974358974 1.7677742158049
31.2564102564103 0.832303599070226
32.7435897435897 1.89433069865864
34.3025641025641 2.0194668266678
35.9358974358974 0.976699993290024
37.648717948718 2.38580902210458
39.4410256410256 1.87530198421726
41.3179487179487 1.72944035085473
43.2871794871795 2.82007283893513
45.3487179487179 2.3242050778089
47.5076923076923 3.18931871290219
49.7692307692308 0.916128229116418
52.1384615384615 1.28508649704932
54.6205128205128 1.4123798047559
57.2230769230769 3.40399250521127
59.9461538461538 1.94291066998022
62.8025641025641 1.70364851461828
65.7923076923077 0.864793223298632
68.925641025641 2.26985371549055
72.2076923076923 0.869742482218593
75.6461538461539 0.98021073858527
79.2461538461538 0.885007206870667
83.0205128205128 0.918403450636297
86.974358974359 0.851675734422651
91.1153846153846 1.00762530738694
95.4538461538462 1.0105895414091
100 1.08539663410586
};
\addplot [, color0, opacity=0.6, mark=diamond*, mark size=0.5, mark options={solid}, only marks, forget plot]
table {%
1 0.0619597363043891
1.04615384615385 0.0692204390514654
1.0974358974359 0.149882150960914
1.14871794871795 0.150952562295058
1.2025641025641 0.196618128571242
1.26153846153846 0.320518990046133
1.32051282051282 0.335455244444235
1.38461538461538 0.407417032413841
1.44871794871795 0.35540029239526
1.51794871794872 0.559772028559124
1.58974358974359 0.620317331132565
1.66666666666667 0.601103036273523
1.74615384615385 0.50157884266402
1.82820512820513 0.48878917342009
1.91538461538462 0.538137362501238
2.00769230769231 0.472266852417207
2.1025641025641 0.545770117827897
2.20512820512821 0.339737602417038
2.30769230769231 0.330635455734328
2.41794871794872 0.30687500961769
2.53333333333333 0.377178376818018
2.65384615384615 0.302594376246544
2.78205128205128 0.283783632552274
2.91282051282051 0.409380260187655
3.05384615384615 0.369353487145298
3.1974358974359 0.541623337955991
3.35128205128205 0.305143844607175
3.51025641025641 0.305167380476233
3.67692307692308 0.274397756649328
3.85128205128205 0.224092285751866
4.03589743589744 0.274580322341738
4.22820512820513 0.374171767745319
4.42820512820513 0.306169708823397
4.64102564102564 0.687243810632159
4.86153846153846 0.365335435190926
5.09230769230769 0.425498245821647
5.33589743589744 0.722936828751701
5.58974358974359 0.443935355276587
5.85641025641026 0.775857731287763
6.13589743589744 0.382559195225929
6.42564102564103 0.415262467520079
6.73333333333333 0.388846440982461
7.05384615384615 0.385670159932329
7.38974358974359 0.409710285811571
7.74102564102564 0.505444354859523
8.11025641025641 0.543893408213112
8.4974358974359 0.475489677436612
8.9 0.663807431657467
9.32564102564103 0.69762181567897
9.76923076923077 0.70556315997723
10.2333333333333 0.890671840849484
10.7205128205128 0.769345018495502
11.2307692307692 0.380312703263984
11.7666666666667 0.898011181879619
12.3282051282051 0.666332036758637
12.9153846153846 0.851595297783766
13.5282051282051 0.947423812062227
14.174358974359 0.928278300127923
14.8487179487179 1.03922496889541
15.5564102564103 1.20173594843079
16.2974358974359 0.913049855358561
17.0717948717949 0.546948108267579
17.8846153846154 1.08787995055339
18.7358974358974 1.73913027910487
19.6282051282051 0.698382983961668
20.5641025641026 0.78454875546226
21.5435897435897 1.45321696530509
22.5692307692308 1.15532666997036
23.6435897435897 1.34874326801065
24.7692307692308 1.81315410753162
25.9487179487179 2.0067904234174
27.1846153846154 1.46286090750894
28.4794871794872 1.62197291788847
29.8358974358974 2.15042948356842
31.2564102564103 1.46969602700362
32.7435897435897 1.77451116246482
34.3025641025641 0.735387157118059
35.9358974358974 0.771686258348682
37.648717948718 1.25752245356318
39.4410256410256 2.5050333734099
41.3179487179487 1.30369087187919
43.2871794871795 2.05552139623536
45.3487179487179 1.75260811856153
47.5076923076923 1.06624481264012
49.7692307692308 2.36649695022236
52.1384615384615 1.75192686973201
54.6205128205128 1.57381526142356
57.2230769230769 0.913661387520289
59.9461538461538 2.5582689166393
62.8025641025641 1.89450705037783
65.7923076923077 0.687243574922086
68.925641025641 0.917770038323983
72.2076923076923 0.967828654421489
75.6461538461539 0.80028077670705
79.2461538461538 1.72159986356384
83.0205128205128 3.63512468207851
86.974358974359 1.02949955524406
91.1153846153846 0.821159459275944
95.4538461538462 0.829286566434632
100 0.783869032621559
};
\addplot [, color1, opacity=0.6, mark=square*, mark size=0.5, mark options={solid}, only marks]
table {%
1 0.196748291410971
1.04615384615385 0.0658902080582907
1.0974358974359 0.142246556792341
1.14871794871795 0.149998029775988
1.2025641025641 0.254471478707359
1.26153846153846 0.276624228218962
1.32051282051282 0.337743939029347
1.38461538461538 0.501541580218208
1.44871794871795 0.297596940927338
1.51794871794872 0.24481269923233
1.58974358974359 0.289435692595212
1.66666666666667 0.339494988201304
1.74615384615385 0.362038975159474
1.82820512820513 0.648185985590638
1.91538461538462 0.416475482039576
2.00769230769231 0.234471092703414
2.1025641025641 0.426076404959509
2.20512820512821 0.211352087162719
2.30769230769231 0.595724861158648
2.41794871794872 0.454550242396621
2.53333333333333 0.28995520508159
2.65384615384615 0.342042046056358
2.78205128205128 0.39651142267665
2.91282051282051 0.482943236157026
3.05384615384615 0.495796221245312
3.1974358974359 0.827599188124673
3.35128205128205 0.918260542481324
3.51025641025641 0.340782979185031
3.67692307692308 0.673300375279211
3.85128205128205 0.335850248472237
4.03589743589744 0.407286727643356
4.22820512820513 0.471029876520584
4.42820512820513 0.501271263992726
4.64102564102564 0.961302547850109
4.86153846153846 0.575651438045373
5.09230769230769 0.436038452869061
5.33589743589744 0.356775630267288
5.58974358974359 0.356958927611789
5.85641025641026 0.3128547903644
6.13589743589744 0.60810471410956
6.42564102564103 0.54710466600228
6.73333333333333 0.589021431617153
7.05384615384615 0.752594077453466
7.38974358974359 0.60496046906294
7.74102564102564 0.504256498457595
8.11025641025641 0.649473967151288
8.4974358974359 0.64552491876192
8.9 0.592984775028888
9.32564102564103 0.919460408511316
9.76923076923077 0.643567111258052
10.2333333333333 1.1854968270275
10.7205128205128 0.909178055348954
11.2307692307692 1.24875688591161
11.7666666666667 1.4562984758918
12.3282051282051 1.6688932234972
12.9153846153846 1.63250321419741
13.5282051282051 1.51339737205995
14.174358974359 1.60175282737387
14.8487179487179 1.12939601218875
15.5564102564103 2.10343653856733
16.2974358974359 1.63363913319303
17.0717948717949 2.21565541899295
17.8846153846154 2.03470182992406
18.7358974358974 1.07170721812297
19.6282051282051 1.70183054199855
20.5641025641026 2.15947574594267
21.5435897435897 1.97805090680327
22.5692307692308 2.05340154802501
23.6435897435897 3.11111480042602
24.7692307692308 4.21291661131181
25.9487179487179 2.82187447001326
27.1846153846154 3.08871418923675
28.4794871794872 2.56650111047091
29.8358974358974 1.95831197364511
31.2564102564103 2.58922824908092
32.7435897435897 1.20043850393473
34.3025641025641 2.02738701492601
35.9358974358974 2.54313023893478
37.648717948718 1.96072633410275
39.4410256410256 1.2873052758263
41.3179487179487 3.18512872834067
43.2871794871795 2.46301007313498
45.3487179487179 4.19840374128569
47.5076923076923 2.20421978977158
49.7692307692308 3.57360582926389
52.1384615384615 5.08442457182096
54.6205128205128 3.84695693839249
57.2230769230769 6.94190856510857
59.9461538461538 2.21887659330918
62.8025641025641 0.661772494889806
65.7923076923077 9.94329158309315
68.925641025641 12.1042312821384
72.2076923076923 3.52812640512061
75.6461538461539 14.0986600661284
79.2461538461538 2.36439721888066
83.0205128205128 4.71819071043
86.974358974359 13.1898824331489
91.1153846153846 0.994303174516682
95.4538461538462 22.3445714578746
100 21.6921789825929
};
\addlegendentry{mb 128, mc 1}
\addplot [, color1, opacity=0.6, mark=square*, mark size=0.5, mark options={solid}, only marks, forget plot]
table {%
1 0.234576915860748
1.04615384615385 0.188542962789029
1.0974358974359 0.185625599617996
1.14871794871795 0.289878208511341
1.2025641025641 0.256272777039496
1.26153846153846 0.196032287227123
1.32051282051282 0.195746081077971
1.38461538461538 0.363894661132945
1.44871794871795 0.378014539412325
1.51794871794872 0.285244947961885
1.58974358974359 0.229046600461067
1.66666666666667 0.212217396545728
1.74615384615385 0.309728373212679
1.82820512820513 0.324266184662042
1.91538461538462 0.320100955732314
2.00769230769231 0.311869747867778
2.1025641025641 0.281493630856142
2.20512820512821 0.360254498463165
2.30769230769231 0.22445182113529
2.41794871794872 0.312289614756783
2.53333333333333 0.391803002245164
2.65384615384615 0.270930255780866
2.78205128205128 0.34862916637171
2.91282051282051 0.234252084270726
3.05384615384615 0.473089665244548
3.1974358974359 0.342226094830442
3.35128205128205 0.219279888517209
3.51025641025641 0.281363536648553
3.67692307692308 0.403613495080422
3.85128205128205 0.304068712855739
4.03589743589744 0.444640173366946
4.22820512820513 0.399834329301413
4.42820512820513 0.551540801770313
4.64102564102564 0.46126201995076
4.86153846153846 0.625966906623269
5.09230769230769 0.366936668503192
5.33589743589744 0.337203213244368
5.58974358974359 0.333298838738787
5.85641025641026 0.275065147080801
6.13589743589744 0.327794498734932
6.42564102564103 0.262390878790917
6.73333333333333 0.311614272162496
7.05384615384615 0.257271676742042
7.38974358974359 0.438123799023279
7.74102564102564 0.283887870825292
8.11025641025641 0.614702057332021
8.4974358974359 0.460259421457892
8.9 0.247828080465395
9.32564102564103 0.264471263972426
9.76923076923077 0.539849460325333
10.2333333333333 0.30808783954107
10.7205128205128 0.321386883914547
11.2307692307692 0.345247067759556
11.7666666666667 0.335000947413994
12.3282051282051 0.252899726596039
12.9153846153846 0.294234415584621
13.5282051282051 0.565004927032476
14.174358974359 0.408320016141764
14.8487179487179 0.804166840874703
15.5564102564103 0.811590628408738
16.2974358974359 0.864391822257542
17.0717948717949 1.15961771550403
17.8846153846154 0.868346487461115
18.7358974358974 1.44547276546941
19.6282051282051 0.955201665222257
20.5641025641026 1.10731522510428
21.5435897435897 0.9695963709029
22.5692307692308 1.71241511243311
23.6435897435897 0.873704602737696
24.7692307692308 0.896345496140274
25.9487179487179 1.5957592523442
27.1846153846154 1.38902566534879
28.4794871794872 0.933935112883297
29.8358974358974 1.95125280618081
31.2564102564103 2.59964419462701
32.7435897435897 1.41987353290846
34.3025641025641 1.03653705045109
35.9358974358974 1.44632123306852
37.648717948718 1.57574157938082
39.4410256410256 1.41797565156827
41.3179487179487 1.70099270633013
43.2871794871795 0.610277621865441
45.3487179487179 2.37586726776078
47.5076923076923 1.71949083105791
49.7692307692308 0.975997714285923
52.1384615384615 0.942278429505821
54.6205128205128 0.952870877614103
57.2230769230769 0.854794621688309
59.9461538461538 0.665138014674854
62.8025641025641 0.801949425419631
65.7923076923077 1.67592655560942
68.925641025641 0.96608335646422
72.2076923076923 0.970655736451222
75.6461538461539 0.977870113738152
79.2461538461538 0.963771938987434
83.0205128205128 3.10100463658737
86.974358974359 3.7737587865269
91.1153846153846 0.987802998324297
95.4538461538462 0.969143256521976
100 0.98095717808494
};
\addplot [, color1, opacity=0.6, mark=square*, mark size=0.5, mark options={solid}, only marks, forget plot]
table {%
1 0.167369942924613
1.04615384615385 0.130586590166012
1.0974358974359 0.155508351956239
1.14871794871795 0.225106712395443
1.2025641025641 0.360319231631453
1.26153846153846 0.287284675305128
1.32051282051282 0.2109817205776
1.38461538461538 0.338270470217013
1.44871794871795 0.271147386039278
1.51794871794872 0.509699489406212
1.58974358974359 0.298963452634817
1.66666666666667 0.376905606370353
1.74615384615385 0.228805832867204
1.82820512820513 0.460504303287647
1.91538461538462 0.429124123257634
2.00769230769231 0.203070337165153
2.1025641025641 0.244223563969386
2.20512820512821 0.353745415557567
2.30769230769231 0.278930934760572
2.41794871794872 0.149095055880809
2.53333333333333 0.252064163706813
2.65384615384615 0.174561386463134
2.78205128205128 0.347173259272421
2.91282051282051 0.36800616534338
3.05384615384615 0.391738226120787
3.1974358974359 0.283392416349687
3.35128205128205 0.343551836322299
3.51025641025641 0.338469477534785
3.67692307692308 0.338342741035987
3.85128205128205 0.336702435826389
4.03589743589744 0.51589316766155
4.22820512820513 0.35079215525923
4.42820512820513 0.582189971831329
4.64102564102564 0.316757618107517
4.86153846153846 0.662662675369199
5.09230769230769 0.521148119369966
5.33589743589744 0.736752675575649
5.58974358974359 0.938227116613544
5.85641025641026 0.373126474629199
6.13589743589744 0.503867034557616
6.42564102564103 0.580373759826988
6.73333333333333 0.337887225974017
7.05384615384615 0.746252575310267
7.38974358974359 0.427261645825222
7.74102564102564 0.848558863297141
8.11025641025641 0.674898247364292
8.4974358974359 0.528751018438214
8.9 0.663435054663372
9.32564102564103 0.501013554032934
9.76923076923077 0.326896797678286
10.2333333333333 0.486364036993059
10.7205128205128 0.788123134529527
11.2307692307692 0.911389126833597
11.7666666666667 1.54586318801465
12.3282051282051 0.523061050862441
12.9153846153846 1.44304035157337
13.5282051282051 0.603981820342285
14.174358974359 0.974821601170955
14.8487179487179 1.01423701239651
15.5564102564103 1.22178944364637
16.2974358974359 2.00398580166456
17.0717948717949 1.52094411439211
17.8846153846154 0.783961518968333
18.7358974358974 0.44540548819439
19.6282051282051 0.942149652021697
20.5641025641026 1.41785511055485
21.5435897435897 1.83640579494769
22.5692307692308 1.69980730730342
23.6435897435897 1.96321988886951
24.7692307692308 2.9273673022425
25.9487179487179 2.14325691552222
27.1846153846154 1.39950362842122
28.4794871794872 2.02002054142034
29.8358974358974 1.75977829567057
31.2564102564103 1.44002974636042
32.7435897435897 1.12888307679044
34.3025641025641 3.08222695290679
35.9358974358974 1.01810838327321
37.648717948718 1.4673715617285
39.4410256410256 1.02972733523508
41.3179487179487 1.49407054643671
43.2871794871795 1.00828115507649
45.3487179487179 1.54019782095899
47.5076923076923 0.851548007292695
49.7692307692308 0.503209153437956
52.1384615384615 1.14446708088252
54.6205128205128 2.02952669029947
57.2230769230769 2.77465613651026
59.9461538461538 1.24185485478928
62.8025641025641 0.957094334145342
65.7923076923077 3.00796270083124
68.925641025641 0.786263184465268
72.2076923076923 3.85597260789448
75.6461538461539 0.903423068250318
79.2461538461538 3.08822600381979
83.0205128205128 1.20642883512471
86.974358974359 0.945671028823733
91.1153846153846 1.18249397895966
95.4538461538462 2.32163106956178
100 0.92589797924422
};
\addplot [, color2, opacity=0.6, mark=triangle*, mark size=0.5, mark options={solid,rotate=180}, only marks]
table {%
1 1.68986428932641
1.04615384615385 0.780035432387425
1.0974358974359 0.672763101628935
1.14871794871795 0.876155561584267
1.2025641025641 0.970379867835855
1.26153846153846 0.906265410311543
1.32051282051282 0.902865209520595
1.38461538461538 1.0955414529683
1.44871794871795 0.741365005854127
1.51794871794872 0.953769217348697
1.58974358974359 0.9655239056533
1.66666666666667 0.949858413165417
1.74615384615385 0.909934855051455
1.82820512820513 1.22328725060802
1.91538461538462 0.997579974842547
2.00769230769231 1.0090770605053
2.1025641025641 1.1094253650689
2.20512820512821 1.05534212806555
2.30769230769231 1.34997441433719
2.41794871794872 0.839848492975969
2.53333333333333 1.39011849310896
2.65384615384615 1.71641296956963
2.78205128205128 1.51443090073938
2.91282051282051 1.64548873237809
3.05384615384615 0.998661836704496
3.1974358974359 2.56676255098128
3.35128205128205 1.12249760465163
3.51025641025641 1.1734675194382
3.67692307692308 1.28155538194844
3.85128205128205 1.18725874821687
4.03589743589744 1.66668423513491
4.22820512820513 1.86820865892622
4.42820512820513 1.67258890292017
4.64102564102564 1.70818070564339
4.86153846153846 1.91095118895407
5.09230769230769 1.87553143150423
5.33589743589744 1.47144408436273
5.58974358974359 1.58452149982751
5.85641025641026 1.82355873605708
6.13589743589744 2.37905075567797
6.42564102564103 1.90636978743337
6.73333333333333 2.19043478485009
7.05384615384615 2.384163459449
7.38974358974359 1.76517138174006
7.74102564102564 2.02212966855073
8.11025641025641 1.70367559800326
8.4974358974359 0.733236198007241
8.9 2.04721152233555
9.32564102564103 0.814272190935948
9.76923076923077 1.40820107389541
10.2333333333333 2.80957649325389
10.7205128205128 2.86999634987153
11.2307692307692 2.71870723606892
11.7666666666667 0.808284052845075
12.3282051282051 3.07918595769885
12.9153846153846 2.01722068334087
13.5282051282051 2.47333030977116
14.174358974359 2.81177713395621
14.8487179487179 0.81990524761776
15.5564102564103 1.96765666920345
16.2974358974359 3.23906288360149
17.0717948717949 2.20026295143865
17.8846153846154 2.82283107244653
18.7358974358974 1.25907842359912
19.6282051282051 1.99029507854964
20.5641025641026 2.85024390610171
21.5435897435897 3.20236151118017
22.5692307692308 3.50274920185676
23.6435897435897 5.13298365633575
24.7692307692308 3.87598266957448
25.9487179487179 6.51395326080265
27.1846153846154 5.35924308867222
28.4794871794872 3.4497728743885
29.8358974358974 1.77551601825083
31.2564102564103 2.46323329421437
32.7435897435897 2.25279862363694
34.3025641025641 2.83477391777503
35.9358974358974 2.78048127419381
37.648717948718 0.933550002605744
39.4410256410256 0.868014360962394
41.3179487179487 4.69392621025191
43.2871794871795 5.96404683196826
45.3487179487179 6.54905186953187
47.5076923076923 3.49743624070847
49.7692307692308 29.6697308569012
52.1384615384615 4.42235057481835
54.6205128205128 1.00095996009049
57.2230769230769 0.949646212236672
59.9461538461538 0.865153931561811
62.8025641025641 1.27878037179068
65.7923076923077 0.98672504772553
68.925641025641 0.923982409136797
72.2076923076923 0.950825559244132
75.6461538461539 0.985597021424532
79.2461538461538 0.996522845178657
83.0205128205128 0.999989242713799
86.974358974359 0.996033643200395
91.1153846153846 0.999371839100338
95.4538461538462 0.998742145860413
100 0.999530493437012
};
\addlegendentry{sub 16, mc 1}
\addplot [, color2, opacity=0.6, mark=triangle*, mark size=0.5, mark options={solid,rotate=180}, only marks, forget plot]
table {%
1 1.73215013615221
1.04615384615385 0.522769923213006
1.0974358974359 0.824865422005137
1.14871794871795 1.50796752597096
1.2025641025641 1.37877978487555
1.26153846153846 0.823469916141659
1.32051282051282 0.789469581812417
1.38461538461538 0.871036201226773
1.44871794871795 0.824939024412528
1.51794871794872 0.797467874967892
1.58974358974359 1.27090732746582
1.66666666666667 0.732454357131717
1.74615384615385 0.757501299608087
1.82820512820513 0.642021645967579
1.91538461538462 1.11736322675063
2.00769230769231 0.661220521422198
2.1025641025641 0.722426862848801
2.20512820512821 1.017938109682
2.30769230769231 0.78226519378975
2.41794871794872 0.788911335782979
2.53333333333333 1.73944112740437
2.65384615384615 0.904321942872258
2.78205128205128 0.787402139391016
2.91282051282051 0.844255800310849
3.05384615384615 1.04472762188937
3.1974358974359 0.947199869991023
3.35128205128205 0.736369762815278
3.51025641025641 0.788163152172573
3.67692307692308 0.822064877589877
3.85128205128205 0.848539161323432
4.03589743589744 1.03758302667562
4.22820512820513 2.25795880474736
4.42820512820513 0.950650004021782
4.64102564102564 0.843248674879489
4.86153846153846 0.88268240362928
5.09230769230769 0.965184805947957
5.33589743589744 0.803179156073884
5.58974358974359 1.03049192664782
5.85641025641026 0.972278362182226
6.13589743589744 1.09861390182071
6.42564102564103 0.822728867145574
6.73333333333333 1.10448850573393
7.05384615384615 0.677891074236204
7.38974358974359 0.775532752992079
7.74102564102564 1.14482057522408
8.11025641025641 0.988463429952666
8.4974358974359 1.08529734175493
8.9 1.20984361679905
9.32564102564103 1.21475681365382
9.76923076923077 1.13723428809493
10.2333333333333 0.937901394090769
10.7205128205128 0.8215339771076
11.2307692307692 1.19111697864916
11.7666666666667 1.34807798995566
12.3282051282051 0.987587030607967
12.9153846153846 0.998637931057568
13.5282051282051 0.935280672350092
14.174358974359 1.17036456317171
14.8487179487179 0.881527022597959
15.5564102564103 1.06601205402348
16.2974358974359 0.918421861453269
17.0717948717949 1.00427843222286
17.8846153846154 1.21086426681109
18.7358974358974 1.63549607006784
19.6282051282051 1.65143770726602
20.5641025641026 0.985435068435415
21.5435897435897 0.882071901879878
22.5692307692308 0.884903712054796
23.6435897435897 0.814163305895884
24.7692307692308 0.812718112798972
25.9487179487179 0.869049417949477
27.1846153846154 1.22830876077078
28.4794871794872 0.889110875697689
29.8358974358974 0.830312290297316
31.2564102564103 0.956826895509922
32.7435897435897 0.793768750340478
34.3025641025641 0.857949608489493
35.9358974358974 1.00932143952127
37.648717948718 0.911298709457604
39.4410256410256 0.828616368842092
41.3179487179487 1.36536277333834
43.2871794871795 0.903437955019031
45.3487179487179 0.88090598630947
47.5076923076923 5.7700951869418
49.7692307692308 0.97913151512954
52.1384615384615 0.953580373833082
54.6205128205128 1.04255045643718
57.2230769230769 1.43405670628052
59.9461538461538 1.090552265741
62.8025641025641 0.901879548345988
65.7923076923077 0.991388053925515
68.925641025641 0.917892370351244
72.2076923076923 0.999579900139241
75.6461538461539 0.990956275677114
79.2461538461538 0.99991491914032
83.0205128205128 0.999899249821999
86.974358974359 0.999575516241044
91.1153846153846 0.997913228285354
95.4538461538462 0.996397980745984
100 0.997286722217838
};
\addplot [, color2, opacity=0.6, mark=triangle*, mark size=0.5, mark options={solid,rotate=180}, only marks, forget plot]
table {%
1 1.93265924082106
1.04615384615385 0.579518446449613
1.0974358974359 0.531123192287425
1.14871794871795 0.841045480007394
1.2025641025641 0.708950485763759
1.26153846153846 0.661598123253062
1.32051282051282 0.497027937500969
1.38461538461538 0.554765846319773
1.44871794871795 0.89366983406535
1.51794871794872 1.06164837444625
1.58974358974359 0.724962202101014
1.66666666666667 0.816436541645901
1.74615384615385 0.752118889948172
1.82820512820513 0.810546722876651
1.91538461538462 0.634002852219045
2.00769230769231 0.647217127830926
2.1025641025641 0.785053046706894
2.20512820512821 0.840057051434566
2.30769230769231 0.876555061814223
2.41794871794872 0.739925269496516
2.53333333333333 1.1622084523171
2.65384615384615 0.92444653412728
2.78205128205128 1.24069434658461
2.91282051282051 0.752374467288769
3.05384615384615 1.21066330340189
3.1974358974359 1.14279043893288
3.35128205128205 0.752128248975352
3.51025641025641 0.799774711844937
3.67692307692308 1.30542585821639
3.85128205128205 1.91100341562683
4.03589743589744 1.35310097081491
4.22820512820513 1.5192365559787
4.42820512820513 1.79137552368352
4.64102564102564 1.10811191423319
4.86153846153846 1.73735045424343
5.09230769230769 1.30779422185644
5.33589743589744 1.58086801803132
5.58974358974359 3.22123214345629
5.85641025641026 2.25978640955617
6.13589743589744 1.57819755225132
6.42564102564103 0.811270275754453
6.73333333333333 0.752684090830172
7.05384615384615 1.61266115149825
7.38974358974359 1.65134638422618
7.74102564102564 1.82787685120245
8.11025641025641 1.88691202323859
8.4974358974359 1.77194002604317
8.9 1.55238404177577
9.32564102564103 1.75039015059808
9.76923076923077 2.17964029755232
10.2333333333333 2.07552475254991
10.7205128205128 1.21302711921288
11.2307692307692 0.846929849274288
11.7666666666667 2.86528529633413
12.3282051282051 2.36759350815516
12.9153846153846 3.33377815597087
13.5282051282051 1.70009441590103
14.174358974359 2.7175037274975
14.8487179487179 1.23762715591174
15.5564102564103 1.92668755006972
16.2974358974359 4.20199805326304
17.0717948717949 1.53375652311186
17.8846153846154 3.21131017908837
18.7358974358974 1.83647904121104
19.6282051282051 0.880873425227975
20.5641025641026 1.37560169021813
21.5435897435897 3.84290402957179
22.5692307692308 2.75044399170227
23.6435897435897 1.9378513886497
24.7692307692308 4.42961498341771
25.9487179487179 2.18475317737839
27.1846153846154 1.66227178088583
28.4794871794872 3.3700731965563
29.8358974358974 7.04292512006194
31.2564102564103 1.96350966895975
32.7435897435897 0.905406463820725
34.3025641025641 1.01688079929887
35.9358974358974 0.95662474192304
37.648717948718 0.90732484856035
39.4410256410256 0.958356613888199
41.3179487179487 0.949287824368532
43.2871794871795 0.843245766561629
45.3487179487179 2.59844403029897
47.5076923076923 0.89545936677598
49.7692307692308 0.852208492218904
52.1384615384615 0.954797738115746
54.6205128205128 1.17870553238927
57.2230769230769 0.986407711535851
59.9461538461538 1.02154512655448
62.8025641025641 0.950552393567836
65.7923076923077 0.998750286601416
68.925641025641 0.992878671151718
72.2076923076923 0.997344928440113
75.6461538461539 0.999152631749154
79.2461538461538 0.968567429101795
83.0205128205128 0.93449808677362
86.974358974359 0.994473514078149
91.1153846153846 0.999755757853634
95.4538461538462 0.998743171825449
100 0.999674991303106
};
\end{axis}

\end{tikzpicture}

      \tikzexternaldisable
    \end{minipage}
  \end{subfigure}

  \begin{subfigure}[t]{\linewidth}
    \centering
    \caption{\cifarten \threecthreed}
    \begin{minipage}{0.50\linewidth}
      \centering
      % defines the pgfplots style "eigspacedefault"
\pgfkeys{/pgfplots/eigspacedefault/.style={
    width=1.03\linewidth,
    height=\goldenRatioInv*1.03*\linewidth,
    every axis plot/.append style={line width = 1pt},
    tick pos = left,
    ylabel near ticks,
    xlabel near ticks,
    xtick align = inside,
    ytick align = inside,
    legend cell align = left,
    legend columns = 1,
    legend pos = north east,
    legend style = {
      fill opacity = 0.9,
      text opacity = 1,
      font = \tiny,
      % column sep=0.1cm,
    },
    legend image post style={scale=2},
    xticklabel style = {font = \small},
    xlabel style = {font = \small},
    axis line style = {black},
    yticklabel style = {font = \small},
    ylabel style = {font = \small},
    title style = {font = \small},
    grid = major,
    grid style = {dashed}
  }
}

\pgfkeys{/pgfplots/eigspacedefaultapp/.style={
    eigspacedefault,
    height=0.6\linewidth,
    legend columns = 2,
  }
}

\pgfkeys{/pgfplots/eigspacenolegend/.style={
    legend image post style = {scale=0},
    legend style = {
      fill opacity = 0,
      draw opacity = 0,
      text opacity = 0,
      font = \small,
      at={(1, 1.025)},
      anchor=south east,
      column sep=0.25cm,
    },
  }
}
%%% Local Variables:
%%% mode: latex
%%% TeX-master: "../main"
%%% End:

      \pgfkeys{/pgfplots/zmystyle/.style={
          eigspacedefaultapp,
          eigspacenolegend,
        }}
      \tikzexternalenable
      \vspace{-6ex}
      % This file was created by tikzplotlib v0.9.7.
\begin{tikzpicture}

\definecolor{color0}{rgb}{0.274509803921569,0.6,0.564705882352941}
\definecolor{color1}{rgb}{0.870588235294118,0.623529411764706,0.0862745098039216}
\definecolor{color2}{rgb}{0.501960784313725,0.184313725490196,0.6}

\begin{axis}[
axis line style={white!10!black},
legend columns=2,
legend style={fill opacity=0.8, draw opacity=1, text opacity=1, at={(0.03,0.97)}, anchor=north west, draw=white!80!black},
log basis x={10},
tick pos=left,
xlabel={epoch (log scale)},
xmajorgrids,
xmin=0.794328234724281, xmax=125.892541179417,
xmode=log,
ylabel={av. rel. error (log scale)},
ymajorgrids,
ymin=0.0104885033112232, ymax=421.718329540383,
ymode=log,
zmystyle
]
\addplot [, black, opacity=0.6, mark=*, mark size=0.5, mark options={solid}, only marks]
table {%
1 0.0234843385732728
1.04487179487179 0.0236386092806035
1.09615384615385 0.016982447882705
1.1474358974359 0.021694150620875
1.20192307692308 0.0246032792699513
1.25961538461538 0.0364435768071171
1.32051282051282 0.0246658657390732
1.38461538461538 0.0310773905404485
1.44871794871795 0.0330142618626171
1.51923076923077 0.0411431772984558
1.58974358974359 0.0255503403636606
1.66666666666667 0.0607394434353838
1.74679487179487 0.0421125052972792
1.83012820512821 0.0667462348949838
1.91666666666667 0.0700501543798385
2.00641025641026 0.0585591913267597
2.1025641025641 0.0864457810004235
2.20512820512821 0.0639926353091992
2.30769230769231 0.0528159119754292
2.41987179487179 0.0623354178687622
2.53525641025641 0.125818543721746
2.65384615384615 0.0693107517548196
2.78205128205128 0.0984677730441775
2.91346153846154 0.101913359148904
3.05128205128205 0.115372293617463
3.19871794871795 0.0764009695055928
3.34935897435897 0.120291884932914
3.50961538461538 0.148696119810136
3.67628205128205 0.130580304113751
3.8525641025641 0.122598742183174
4.03525641025641 0.124350105254516
4.2275641025641 0.158478341392425
4.42948717948718 0.121897226558674
4.64102564102564 0.175173172940954
4.86217948717949 0.124129373051892
5.09294871794872 0.203041462054759
5.33653846153846 0.133969244881818
5.58974358974359 0.156724256059186
5.85576923076923 0.17755886379068
6.13461538461539 0.236355488563716
6.42628205128205 0.196253013709404
6.73397435897436 0.248515529218314
7.05448717948718 0.235925760163614
7.38782051282051 0.259766898611254
7.74038461538461 0.19625070723164
8.10897435897436 0.287846472061976
8.49679487179487 0.223221151863399
8.90064102564103 0.265810678106018
9.32371794871795 0.224826637210783
9.76923076923077 0.229934220758385
10.2339743589744 0.34330124728858
10.7211538461538 0.354085843600059
11.2307692307692 0.386618922683861
11.7660256410256 0.426541658296724
12.3269230769231 0.395405487110444
12.9134615384615 0.386036079620534
13.5288461538462 0.526675707039009
14.1730769230769 0.384656631006126
14.849358974359 0.531604630982247
15.5544871794872 0.586516999355601
16.2948717948718 0.586160191051342
17.0705128205128 0.672197321915248
17.8846153846154 0.624825443954634
18.7371794871795 0.557365266474034
19.6282051282051 0.60818307930048
20.5641025641026 0.759208970868744
21.5416666666667 0.692483173025598
22.5673076923077 1.09892376332674
23.6442307692308 0.932025825013697
24.7692307692308 0.898922644186148
25.9487179487179 1.09899243962943
27.1826923076923 0.977650759644138
28.4775641025641 1.08107320089883
29.8333333333333 1.10187067011074
31.2564102564103 1.12405937333874
32.7435897435897 1.41256523230947
34.3044871794872 1.55679744748683
35.9358974358974 1.45945045710113
37.6474358974359 1.67261981755448
39.4391025641026 2.07068316998916
41.3173076923077 3.00495409635117
43.2852564102564 2.91598349635038
45.3461538461538 3.04519686362065
47.5064102564103 3.17935999122266
49.7692307692308 5.10106042355035
52.1378205128205 5.42917106800764
54.6217948717949 3.36281765583731
57.2211538461538 6.51401889452528
59.9455128205128 5.47736112682511
62.8012820512821 8.47975719863571
65.7916666666667 4.87153404753597
68.9230769230769 11.6846925361435
72.2051282051282 9.03930870623534
75.6442307692308 12.4853194780866
79.2467948717949 11.1398463446452
83.0192307692308 12.7969563837352
86.974358974359 11.7611353744875
91.1153846153846 16.2279472408526
95.4519230769231 16.450301313131
100 23.3341163231738
};
\addlegendentry{mb 128, exact}
\addplot [, black, opacity=0.6, mark=*, mark size=0.5, mark options={solid}, only marks, forget plot]
table {%
1 0.0414492775023186
1.04487179487179 0.0336650535556261
1.09615384615385 0.0351726737689849
1.1474358974359 0.0270218954635068
1.20192307692308 0.0334232225283053
1.25961538461538 0.0332032438734216
1.32051282051282 0.0492826919234895
1.38461538461538 0.034805849246171
1.44871794871795 0.0401117932084099
1.51923076923077 0.0287885520360222
1.58974358974359 0.0533153984894304
1.66666666666667 0.0350127808197824
1.74679487179487 0.0596806434415458
1.83012820512821 0.0516958973465193
1.91666666666667 0.0538003202802435
2.00641025641026 0.055773946811275
2.1025641025641 0.0598308229275157
2.20512820512821 0.0534787667640114
2.30769230769231 0.0795580708956879
2.41987179487179 0.0672184747256048
2.53525641025641 0.032448308871596
2.65384615384615 0.0789904780454776
2.78205128205128 0.0778120134213774
2.91346153846154 0.0792457508881905
3.05128205128205 0.0564480574327223
3.19871794871795 0.0785971322556398
3.34935897435897 0.0845396080014712
3.50961538461538 0.0937785591672037
3.67628205128205 0.0566872940250422
3.8525641025641 0.0707108977080718
4.03525641025641 0.0581342143048823
4.2275641025641 0.0609282988448069
4.42948717948718 0.10112987493914
4.64102564102564 0.147143989270698
4.86217948717949 0.0965341492295994
5.09294871794872 0.158577336545058
5.33653846153846 0.125390990290477
5.58974358974359 0.123871573071302
5.85576923076923 0.0984958075580986
6.13461538461539 0.113992033102985
6.42628205128205 0.103157497570141
6.73397435897436 0.189952668409184
7.05448717948718 0.168005746726972
7.38782051282051 0.13831380231228
7.74038461538461 0.120555183221354
8.10897435897436 0.184802030761406
8.49679487179487 0.28111111777191
8.90064102564103 0.236684727572579
9.32371794871795 0.296253877303799
9.76923076923077 0.333349701021511
10.2339743589744 0.201349702598937
10.7211538461538 0.325774596364705
11.2307692307692 0.332020644203909
11.7660256410256 0.335804279502766
12.3269230769231 0.327865624166411
12.9134615384615 0.447694498125725
13.5288461538462 0.325873489361299
14.1730769230769 0.492814735364315
14.849358974359 0.371186523835327
15.5544871794872 0.603698055196234
16.2948717948718 0.630632806040666
17.0705128205128 0.586175667128831
17.8846153846154 0.617040510108828
18.7371794871795 0.779671869872001
19.6282051282051 0.725615335591917
20.5641025641026 0.743610485299993
21.5416666666667 0.476052460786837
22.5673076923077 0.739668383417909
23.6442307692308 1.04553671576824
24.7692307692308 1.12249987311173
25.9487179487179 1.30093779849569
27.1826923076923 1.19929804870502
28.4775641025641 1.26744355731128
29.8333333333333 1.44761268891787
31.2564102564103 1.34378638070728
32.7435897435897 1.8196692269347
34.3044871794872 1.87103367985946
35.9358974358974 1.92851303278909
37.6474358974359 1.52136003570249
39.4391025641026 2.06043706764987
41.3173076923077 2.83228280057376
43.2852564102564 2.25690462694628
45.3461538461538 3.6621400455135
47.5064102564103 3.27910922398659
49.7692307692308 3.83826964719309
52.1378205128205 5.06238848866851
54.6217948717949 4.48019018268773
57.2211538461538 6.39667710711139
59.9455128205128 5.94953490626717
62.8012820512821 7.20243180162605
65.7916666666667 10.9584988338736
68.9230769230769 5.94550479793721
72.2051282051282 8.48197157723505
75.6442307692308 11.1709992873706
79.2467948717949 10.3557005504793
83.0192307692308 12.2324311829483
86.974358974359 16.9114182338829
91.1153846153846 17.0063794752739
95.4519230769231 17.8653624874673
100 25.6366452771834
};
\addplot [, black, opacity=0.6, mark=*, mark size=0.5, mark options={solid}, only marks, forget plot]
table {%
1 0.0726767836338234
1.04487179487179 0.0588643401382577
1.09615384615385 0.0544725152958042
1.1474358974359 0.0361049566880415
1.20192307692308 0.0236695463466175
1.25961538461538 0.0482016999656274
1.32051282051282 0.0358414315122598
1.38461538461538 0.0434804674167734
1.44871794871795 0.0283809091960422
1.51923076923077 0.042986266271672
1.58974358974359 0.0248914816330174
1.66666666666667 0.0403912969661595
1.74679487179487 0.0477140723578488
1.83012820512821 0.0632234166669965
1.91666666666667 0.0733692237102183
2.00641025641026 0.0690877717294525
2.1025641025641 0.0902278817179661
2.20512820512821 0.0777361666514706
2.30769230769231 0.0687101682065956
2.41987179487179 0.0864941592757174
2.53525641025641 0.0686582910308447
2.65384615384615 0.0836117564610582
2.78205128205128 0.106125495174694
2.91346153846154 0.105840569143404
3.05128205128205 0.0741471229820445
3.19871794871795 0.0855534730799616
3.34935897435897 0.0661432709620053
3.50961538461538 0.0879561961590966
3.67628205128205 0.16516715856164
3.8525641025641 0.0882703998478171
4.03525641025641 0.140740006085227
4.2275641025641 0.131424562410458
4.42948717948718 0.250121269975511
4.64102564102564 0.212689730096479
4.86217948717949 0.179548808309984
5.09294871794872 0.217862538814972
5.33653846153846 0.137523284482749
5.58974358974359 0.190008210003822
5.85576923076923 0.202922187383078
6.13461538461539 0.182695269764095
6.42628205128205 0.132450853211059
6.73397435897436 0.230981132085173
7.05448717948718 0.222872596366245
7.38782051282051 0.227967553822108
7.74038461538461 0.200322135927439
8.10897435897436 0.221224364782548
8.49679487179487 0.253339664217208
8.90064102564103 0.22734462205843
9.32371794871795 0.254955539885746
9.76923076923077 0.338589489792153
10.2339743589744 0.415441383124656
10.7211538461538 0.378176760745255
11.2307692307692 0.530495436954903
11.7660256410256 0.43816354941738
12.3269230769231 0.605763991869743
12.9134615384615 0.47501307143015
13.5288461538462 0.596667398935174
14.1730769230769 0.521522090387169
14.849358974359 0.61435692164138
15.5544871794872 0.645613411505028
16.2948717948718 0.767375110239016
17.0705128205128 0.709327533870429
17.8846153846154 0.724670279347825
18.7371794871795 0.898299470405983
19.6282051282051 0.873284367590523
20.5641025641026 0.991462468685117
21.5416666666667 1.1192107946348
22.5673076923077 1.00579282086765
23.6442307692308 0.987388119930643
24.7692307692308 1.39439188826882
25.9487179487179 1.38973377565664
27.1826923076923 1.19134394204546
28.4775641025641 1.56017018889659
29.8333333333333 1.30427437196929
31.2564102564103 1.45175385525819
32.7435897435897 2.17393018185406
34.3044871794872 1.90782439627711
35.9358974358974 1.71101455421462
37.6474358974359 2.03289359961454
39.4391025641026 2.02069165413429
41.3173076923077 4.4091779578122
43.2852564102564 2.3232173878253
45.3461538461538 2.85048202547266
47.5064102564103 4.69383918896014
49.7692307692308 6.62067505815237
52.1378205128205 5.18965873260628
54.6217948717949 3.20698776224484
57.2211538461538 6.3227937448444
59.9455128205128 7.05126085210752
62.8012820512821 7.51059897353879
65.7916666666667 9.05679523444787
68.9230769230769 8.53537972373212
72.2051282051282 11.6345194866959
75.6442307692308 12.0322599354045
79.2467948717949 13.2467179647526
83.0192307692308 13.48367670193
86.974358974359 17.1623589788571
91.1153846153846 16.5799155116058
95.4519230769231 15.9994371019476
100 29.5560418920271
};
\addplot [, color0, opacity=0.6, mark=diamond*, mark size=0.5, mark options={solid}, only marks]
table {%
1 0.613958715788197
1.04487179487179 0.565375266430926
1.09615384615385 0.412076307368984
1.1474358974359 0.335936569828528
1.20192307692308 0.246646557325577
1.25961538461538 0.249938125650781
1.32051282051282 0.312944154422721
1.38461538461538 0.292191690874937
1.44871794871795 0.228512628507189
1.51923076923077 0.3708360388316
1.58974358974359 0.330627347572136
1.66666666666667 0.426491580557368
1.74679487179487 0.431186592911823
1.83012820512821 0.467761033286412
1.91666666666667 0.461063880613179
2.00641025641026 0.414975295934846
2.1025641025641 0.549591705240449
2.20512820512821 0.398586903246952
2.30769230769231 0.495363192073838
2.41987179487179 0.482144055644413
2.53525641025641 0.713158067544253
2.65384615384615 0.340487537849521
2.78205128205128 0.518401693325471
2.91346153846154 0.750846288560091
3.05128205128205 0.710929635255953
3.19871794871795 0.768233411040885
3.34935897435897 0.940923431172446
3.50961538461538 0.763974977816603
3.67628205128205 1.23779479704315
3.8525641025641 1.22698924381464
4.03525641025641 1.28243539003604
4.2275641025641 1.19857781077802
4.42948717948718 1.64327660312798
4.64102564102564 1.53011232791519
4.86217948717949 1.40539521368259
5.09294871794872 2.02776979517058
5.33653846153846 1.98097651278103
5.58974358974359 1.54316629422852
5.85576923076923 2.24189933385006
6.13461538461539 2.15891778122939
6.42628205128205 1.86792779138353
6.73397435897436 3.19047546697468
7.05448717948718 2.44987743864331
7.38782051282051 2.43502992096411
7.74038461538461 3.29279277616816
8.10897435897436 3.27937574585779
8.49679487179487 3.03229915996094
8.90064102564103 2.9443174251067
9.32371794871795 3.71090226929944
9.76923076923077 2.99677908854358
10.2339743589744 4.60894527178166
10.7211538461538 3.80805406570065
11.2307692307692 4.03889247133291
11.7660256410256 3.98280678749066
12.3269230769231 4.51857558075025
12.9134615384615 6.80413997858958
13.5288461538462 4.95184437958587
14.1730769230769 4.54562990350347
14.849358974359 6.84575793134688
15.5544871794872 7.22665471066896
16.2948717948718 5.76896770966241
17.0705128205128 6.47898354016193
17.8846153846154 6.07930296480957
18.7371794871795 6.26915008875657
19.6282051282051 6.96879022363761
20.5641025641026 8.1713112582245
21.5416666666667 9.38763126099212
22.5673076923077 11.2469854381356
23.6442307692308 9.91106748800989
24.7692307692308 8.15815395993778
25.9487179487179 8.87848255914614
27.1826923076923 9.64279412769669
28.4775641025641 15.029874191674
29.8333333333333 26.7865227755404
31.2564102564103 11.4773860875301
32.7435897435897 9.98770056578772
34.3044871794872 10.9455539042416
35.9358974358974 18.1473078522187
37.6474358974359 11.8459270951789
39.4391025641026 15.5749362336453
41.3173076923077 30.8056974678337
43.2852564102564 24.5252715449812
45.3461538461538 34.2884758057978
47.5064102564103 36.7949010854199
49.7692307692308 55.1198173395612
52.1378205128205 59.775792449143
54.6217948717949 26.4018691797776
57.2211538461538 59.9673428239662
59.9455128205128 51.0622327954242
62.8012820512821 60.4951382170886
65.7916666666667 22.6567711306631
68.9230769230769 63.1319141855486
72.2051282051282 25.1289123626122
75.6442307692308 72.0882639207047
79.2467948717949 56.8448923693609
83.0192307692308 62.2631640913517
86.974358974359 38.2545655816469
91.1153846153846 99.9342429064719
95.4519230769231 20.8851370716604
100 58.6318469177098
};
\addlegendentry{sub 16, exact}
\addplot [, color0, opacity=0.6, mark=diamond*, mark size=0.5, mark options={solid}, only marks, forget plot]
table {%
1 0.652981929053975
1.04487179487179 0.621183087972533
1.09615384615385 0.320129329802377
1.1474358974359 0.228477298372892
1.20192307692308 0.153623042867769
1.25961538461538 0.156357349901541
1.32051282051282 0.19088519633867
1.38461538461538 0.16135561203913
1.44871794871795 0.173383753214449
1.51923076923077 0.187891347519202
1.58974358974359 0.204235115876035
1.66666666666667 0.207620531478588
1.74679487179487 0.178644199774671
1.83012820512821 0.178901995499729
1.91666666666667 0.164622551602112
2.00641025641026 0.176136669953611
2.1025641025641 0.200118026328471
2.20512820512821 0.169219413179278
2.30769230769231 0.268547045961747
2.41987179487179 0.300640782520717
2.53525641025641 0.22436772412166
2.65384615384615 0.337809482292288
2.78205128205128 0.256289833716671
2.91346153846154 0.282093497186612
3.05128205128205 0.383812159148513
3.19871794871795 0.387452066527612
3.34935897435897 0.658059503832164
3.50961538461538 0.511765728090048
3.67628205128205 0.984055738008315
3.8525641025641 0.74869645244542
4.03525641025641 0.635849300110628
4.2275641025641 0.894307628372227
4.42948717948718 1.05645420413417
4.64102564102564 1.22051197819353
4.86217948717949 1.22407439973999
5.09294871794872 1.26820003577994
5.33653846153846 1.4790058530776
5.58974358974359 1.16788387889911
5.85576923076923 1.58245608051789
6.13461538461539 1.82049547210393
6.42628205128205 1.64057903691737
6.73397435897436 1.66994899165445
7.05448717948718 2.78598604083776
7.38782051282051 2.48239505406646
7.74038461538461 2.33908566713892
8.10897435897436 2.70710653016016
8.49679487179487 3.01856459855141
8.90064102564103 2.57385313302042
9.32371794871795 3.19822561945401
9.76923076923077 3.71450761629853
10.2339743589744 3.7954770455483
10.7211538461538 2.89872304185951
11.2307692307692 4.74779579933192
11.7660256410256 3.82587628407137
12.3269230769231 4.9326536832088
12.9134615384615 4.29428159874194
13.5288461538462 3.20883560786759
14.1730769230769 4.38354025426584
14.849358974359 4.31681960068647
15.5544871794872 5.79980052176517
16.2948717948718 5.22355443726645
17.0705128205128 3.45686502929676
17.8846153846154 5.92620384123021
18.7371794871795 6.61550520701323
19.6282051282051 6.92881608522054
20.5641025641026 7.80824598367894
21.5416666666667 5.80064626815859
22.5673076923077 8.23983561526135
23.6442307692308 7.97436689652923
24.7692307692308 9.79608211131807
25.9487179487179 11.3311887762897
27.1826923076923 7.9764329556281
28.4775641025641 12.4144223701548
29.8333333333333 9.74828964141215
31.2564102564103 11.1873828314713
32.7435897435897 12.1716015770005
34.3044871794872 9.71085311656772
35.9358974358974 9.26952231091373
37.6474358974359 12.5918244197914
39.4391025641026 15.6375004081655
41.3173076923077 13.9045948847977
43.2852564102564 21.9605000172368
45.3461538461538 20.7804976631141
47.5064102564103 11.3919993213722
49.7692307692308 24.5340490801425
52.1378205128205 25.1642628890703
54.6217948717949 30.9273847963672
57.2211538461538 15.3378609466514
59.9455128205128 27.0212290851281
62.8012820512821 45.5785950580589
65.7916666666667 26.5422663550607
68.9230769230769 16.1535382546625
72.2051282051282 22.9783312505619
75.6442307692308 24.7086956130627
79.2467948717949 26.9357184573198
83.0192307692308 16.2292364337204
86.974358974359 18.1125654806756
91.1153846153846 31.6859828802992
95.4519230769231 50.3356069338583
100 59.9748492270728
};
\addplot [, color0, opacity=0.6, mark=diamond*, mark size=0.5, mark options={solid}, only marks, forget plot]
table {%
1 0.705176895098423
1.04487179487179 0.785218782859852
1.09615384615385 0.530180192015705
1.1474358974359 0.46650797347886
1.20192307692308 0.402328464460734
1.25961538461538 0.459499270358694
1.32051282051282 0.47440320394706
1.38461538461538 0.49653401969794
1.44871794871795 0.305452943605546
1.51923076923077 0.371434306808561
1.58974358974359 0.400786882042445
1.66666666666667 0.760409366111529
1.74679487179487 0.507581130232457
1.83012820512821 0.566564078192052
1.91666666666667 0.711654446741642
2.00641025641026 0.456954816614775
2.1025641025641 0.746731569490849
2.20512820512821 0.62128968865225
2.30769230769231 0.702124726345957
2.41987179487179 0.884932626250153
2.53525641025641 0.632814767239323
2.65384615384615 1.20818315736767
2.78205128205128 0.93896397827413
2.91346153846154 0.692762842616822
3.05128205128205 0.635633942543763
3.19871794871795 0.677766547205428
3.34935897435897 0.621711092124142
3.50961538461538 1.00002703577365
3.67628205128205 0.87552093686373
3.8525641025641 0.930070831451499
4.03525641025641 1.22474129939649
4.2275641025641 1.03459618886879
4.42948717948718 1.38763875694733
4.64102564102564 1.49192744667878
4.86217948717949 1.7663123955166
5.09294871794872 1.77125011209562
5.33653846153846 1.45744791522292
5.58974358974359 1.35176615434338
5.85576923076923 1.68946982190625
6.13461538461539 1.64847473890499
6.42628205128205 1.50763089487947
6.73397435897436 2.15797282655253
7.05448717948718 2.1079585324697
7.38782051282051 1.9770589031795
7.74038461538461 2.07557831918733
8.10897435897436 3.80690903180521
8.49679487179487 2.87218083606678
8.90064102564103 3.26267661489253
9.32371794871795 3.22491600228857
9.76923076923077 2.3042753342439
10.2339743589744 3.14225679202487
10.7211538461538 3.97083240476185
11.2307692307692 4.83842474161135
11.7660256410256 5.14030905649036
12.3269230769231 4.46895560359191
12.9134615384615 5.71969055546462
13.5288461538462 9.65535128836897
14.1730769230769 6.10111185253888
14.849358974359 6.68313705250704
15.5544871794872 5.31821777306395
16.2948717948718 7.22205720557338
17.0705128205128 7.63682974754604
17.8846153846154 5.81534344704374
18.7371794871795 7.217600572887
19.6282051282051 5.80598040648679
20.5641025641026 8.52106054269293
21.5416666666667 7.15776998236398
22.5673076923077 6.336086594931
23.6442307692308 6.01967944874659
24.7692307692308 9.94222023597382
25.9487179487179 12.2349145161222
27.1826923076923 11.4745766521119
28.4775641025641 9.83135642974995
29.8333333333333 13.3193951322352
31.2564102564103 10.0623418537838
32.7435897435897 12.4774111904125
34.3044871794872 12.3500628728134
35.9358974358974 13.2497777457895
37.6474358974359 14.4071044003383
39.4391025641026 14.856230061308
41.3173076923077 16.9375959127169
43.2852564102564 18.0130900365357
45.3461538461538 15.4386356902547
47.5064102564103 20.5919066160867
49.7692307692308 19.3504581081293
52.1378205128205 23.779071083185
54.6217948717949 19.8415484376579
57.2211538461538 7.47194160360598
59.9455128205128 15.2387741400107
62.8012820512821 44.8638494767736
65.7916666666667 13.7093624641061
68.9230769230769 26.4997273076184
72.2051282051282 30.0672064318534
75.6442307692308 32.2531720248709
79.2467948717949 36.2104115967795
83.0192307692308 17.3564154354282
86.974358974359 45.875567152831
91.1153846153846 47.0711058193563
95.4519230769231 50.7963158997904
100 31.1250023323678
};
\addplot [, color1, opacity=0.6, mark=square*, mark size=0.5, mark options={solid}, only marks]
table {%
1 0.357452606829476
1.04487179487179 0.286416457057956
1.09615384615385 0.19860310276121
1.1474358974359 0.0937263182844644
1.20192307692308 0.10095118793647
1.25961538461538 0.152528671705292
1.32051282051282 0.0975683113132908
1.38461538461538 0.165236072269952
1.44871794871795 0.104200154947196
1.51923076923077 0.190837730710877
1.58974358974359 0.0949641657052943
1.66666666666667 0.134729821344459
1.74679487179487 0.11517809594782
1.83012820512821 0.185330783442671
1.91666666666667 0.228599085378758
2.00641025641026 0.0961593113458212
2.1025641025641 0.214336492809076
2.20512820512821 0.153582573952672
2.30769230769231 0.134031517547076
2.41987179487179 0.131676722531829
2.53525641025641 0.268977419239694
2.65384615384615 0.230315052182639
2.78205128205128 0.172300806135139
2.91346153846154 0.219034052968615
3.05128205128205 0.303743500151104
3.19871794871795 0.220558492097705
3.34935897435897 0.422342055492769
3.50961538461538 0.28118548952515
3.67628205128205 0.344621588369623
3.8525641025641 0.425240090668523
4.03525641025641 0.448617169580119
4.2275641025641 0.444303517171478
4.42948717948718 0.46161517770236
4.64102564102564 0.623397796341657
4.86217948717949 0.360530028638964
5.09294871794872 0.531423886872396
5.33653846153846 0.519070891243114
5.58974358974359 0.687541364291743
5.85576923076923 0.674949862104435
6.13461538461539 0.622022320216033
6.42628205128205 0.55723110961806
6.73397435897436 1.05819279391545
7.05448717948718 1.12268035076527
7.38782051282051 0.955716832459496
7.74038461538461 1.33129916717617
8.10897435897436 1.05453755124915
8.49679487179487 1.12749238988911
8.90064102564103 0.986783197269412
9.32371794871795 1.26222328090832
9.76923076923077 1.78043491688758
10.2339743589744 1.61569315981748
10.7211538461538 1.54237793988742
11.2307692307692 1.69873070342896
11.7660256410256 1.69302428411611
12.3269230769231 1.81906023910444
12.9134615384615 2.07541244454343
13.5288461538462 1.83332925759615
14.1730769230769 2.97616164343926
14.849358974359 2.73452091338414
15.5544871794872 3.50076474581849
16.2948717948718 2.75419499377791
17.0705128205128 2.60988728179949
17.8846153846154 4.45178572055478
18.7371794871795 3.05253161317501
19.6282051282051 3.58415938690192
20.5641025641026 3.42527206658581
21.5416666666667 3.52670266753761
22.5673076923077 4.92511993959965
23.6442307692308 4.64487833342676
24.7692307692308 4.79707568333363
25.9487179487179 5.1146938071736
27.1826923076923 4.10479670770526
28.4775641025641 4.81205683422148
29.8333333333333 5.45007764619347
31.2564102564103 5.50222410527528
32.7435897435897 7.23598632442799
34.3044871794872 7.18076895948258
35.9358974358974 9.47724814748251
37.6474358974359 7.42159611829509
39.4391025641026 8.48445965417812
41.3173076923077 10.9374514574563
43.2852564102564 10.1102918134491
45.3461538461538 10.7144708947907
47.5064102564103 11.4233395427428
49.7692307692308 17.0305942099347
52.1378205128205 22.106190050036
54.6217948717949 18.7283607630078
57.2211538461538 30.2214686354926
59.9455128205128 29.1944592334643
62.8012820512821 23.9866250989769
65.7916666666667 33.2255088684984
68.9230769230769 19.5450543745719
72.2051282051282 22.73727808837
75.6442307692308 30.2549300781906
79.2467948717949 32.5867615144558
83.0192307692308 47.7816274318917
86.974358974359 40.7854826771797
91.1153846153846 40.5920323297746
95.4519230769231 33.4428618573318
100 69.9306274767768
};
\addlegendentry{mb 128, mc 1}
\addplot [, color1, opacity=0.6, mark=square*, mark size=0.5, mark options={solid}, only marks, forget plot]
table {%
1 0.434599751828721
1.04487179487179 0.293305957942881
1.09615384615385 0.187315199528322
1.1474358974359 0.172380158306008
1.20192307692308 0.175424408391272
1.25961538461538 0.171291812117265
1.32051282051282 0.21025674216701
1.38461538461538 0.140236480767063
1.44871794871795 0.127657946671558
1.51923076923077 0.415043512280851
1.58974358974359 0.18318029729598
1.66666666666667 0.146956321178345
1.74679487179487 0.193201370952481
1.83012820512821 0.11758348445966
1.91666666666667 0.2543308242925
2.00641025641026 0.226319157669166
2.1025641025641 0.217840708148435
2.20512820512821 0.136068996737864
2.30769230769231 0.119489253958424
2.41987179487179 0.159106497666516
2.53525641025641 0.171635020991784
2.65384615384615 0.222412071943003
2.78205128205128 0.355854072571646
2.91346153846154 0.497663693438345
3.05128205128205 0.362747461578335
3.19871794871795 0.28972345143231
3.34935897435897 0.285865283648714
3.50961538461538 0.61029554016885
3.67628205128205 0.374692978194514
3.8525641025641 0.243124713227301
4.03525641025641 0.566016956353495
4.2275641025641 0.580839119504029
4.42948717948718 0.531303369991753
4.64102564102564 1.19233645607993
4.86217948717949 0.559534636371344
5.09294871794872 0.732677695370175
5.33653846153846 0.714199965532974
5.58974358974359 0.915281674979569
5.85576923076923 1.15954796545404
6.13461538461539 0.861951612990994
6.42628205128205 0.760069235086839
6.73397435897436 1.27209216533506
7.05448717948718 0.895276496825665
7.38782051282051 1.30574857938347
7.74038461538461 1.35183452671589
8.10897435897436 1.49139042712277
8.49679487179487 1.44376024413082
8.90064102564103 1.68527103199553
9.32371794871795 1.62261005077942
9.76923076923077 2.05006296379905
10.2339743589744 2.2174269385116
10.7211538461538 2.08619153789602
11.2307692307692 1.99596631927974
11.7660256410256 2.21300752425945
12.3269230769231 2.0738552954195
12.9134615384615 2.28243569278005
13.5288461538462 3.299566913582
14.1730769230769 2.40885807328227
14.849358974359 2.51721791052913
15.5544871794872 2.92169808826962
16.2948717948718 3.68108361739139
17.0705128205128 3.69202489701563
17.8846153846154 3.39765336389202
18.7371794871795 4.69590250054151
19.6282051282051 3.0459565972412
20.5641025641026 2.63251336144216
21.5416666666667 4.84627865549104
22.5673076923077 4.82614160975272
23.6442307692308 6.78491248732005
24.7692307692308 4.80317680582068
25.9487179487179 3.79409547600441
27.1826923076923 5.99156951569465
28.4775641025641 3.78203522731004
29.8333333333333 4.4961149642758
31.2564102564103 6.74580471574074
32.7435897435897 9.52108437016004
34.3044871794872 7.19480547969565
35.9358974358974 6.36620806033634
37.6474358974359 8.48011982718784
39.4391025641026 13.1294463587869
41.3173076923077 10.5847326168084
43.2852564102564 12.5756037648225
45.3461538461538 10.2571262275537
47.5064102564103 19.5306007692415
49.7692307692308 15.0517569747965
52.1378205128205 24.848339121632
54.6217948717949 16.4466272279992
57.2211538461538 14.9713175804279
59.9455128205128 16.6410651660807
62.8012820512821 11.1862930136817
65.7916666666667 21.158426053134
68.9230769230769 28.5247265283498
72.2051282051282 20.2383820714825
75.6442307692308 36.7013094617641
79.2467948717949 40.379496250751
83.0192307692308 26.3079205415466
86.974358974359 40.2119471809323
91.1153846153846 40.3782668050466
95.4519230769231 33.8564481884161
100 30.5712799910826
};
\addplot [, color1, opacity=0.6, mark=square*, mark size=0.5, mark options={solid}, only marks, forget plot]
table {%
1 0.341353894145073
1.04487179487179 0.349582143731829
1.09615384615385 0.241849505440899
1.1474358974359 0.152854137285041
1.20192307692308 0.123437946846385
1.25961538461538 0.0969310098028491
1.32051282051282 0.136679465336398
1.38461538461538 0.0856706879377822
1.44871794871795 0.131721499993181
1.51923076923077 0.171230021019403
1.58974358974359 0.157302043458576
1.66666666666667 0.170310402268007
1.74679487179487 0.218675377067935
1.83012820512821 0.34840527059167
1.91666666666667 0.290390931853821
2.00641025641026 0.209691080320259
2.1025641025641 0.329216263364386
2.20512820512821 0.279641398203538
2.30769230769231 0.15925157793584
2.41987179487179 0.196367634080154
2.53525641025641 0.325900672936468
2.65384615384615 0.169399954124992
2.78205128205128 0.341759355894447
2.91346153846154 0.329482016152775
3.05128205128205 0.562550362463487
3.19871794871795 0.291775647643241
3.34935897435897 0.273830800683642
3.50961538461538 0.606945295265928
3.67628205128205 0.54859656026836
3.8525641025641 0.528763077519752
4.03525641025641 0.505961875856101
4.2275641025641 0.776751925140548
4.42948717948718 0.609332983360486
4.64102564102564 0.840155128744303
4.86217948717949 0.453782403573069
5.09294871794872 0.897511460549944
5.33653846153846 0.412364646317803
5.58974358974359 0.899801959296069
5.85576923076923 0.918812029584625
6.13461538461539 0.670511926080714
6.42628205128205 0.724189116225991
6.73397435897436 0.845456703625641
7.05448717948718 0.958566844159504
7.38782051282051 0.862760675134679
7.74038461538461 0.939556083375443
8.10897435897436 1.27145405079241
8.49679487179487 0.899200144794261
8.90064102564103 0.978074637954591
9.32371794871795 1.37880078888499
9.76923076923077 1.49379006903734
10.2339743589744 1.69396913320979
10.7211538461538 1.72598355424194
11.2307692307692 2.61673351818139
11.7660256410256 2.40382609286014
12.3269230769231 2.06795423869014
12.9134615384615 2.2004808614533
13.5288461538462 3.18142141962102
14.1730769230769 2.78392447436587
14.849358974359 2.12363502430483
15.5544871794872 2.24178339776917
16.2948717948718 2.70691568841749
17.0705128205128 2.8009368342867
17.8846153846154 3.01816948256586
18.7371794871795 3.65786964810829
19.6282051282051 3.5713758240799
20.5641025641026 4.41173110029125
21.5416666666667 5.70251989199867
22.5673076923077 4.61469480399579
23.6442307692308 4.37116164901375
24.7692307692308 5.00951503118089
25.9487179487179 6.33112831792611
27.1826923076923 7.0210053858908
28.4775641025641 6.25779322982121
29.8333333333333 6.45381848665665
31.2564102564103 8.62596923749617
32.7435897435897 8.8515657008481
34.3044871794872 6.93250275526002
35.9358974358974 12.1125677266471
37.6474358974359 10.1584479170635
39.4391025641026 7.95863494855985
41.3173076923077 8.42334384002054
43.2852564102564 13.1273658522858
45.3461538461538 11.2650943951755
47.5064102564103 11.8258655025565
49.7692307692308 16.813552401553
52.1378205128205 17.1814379134424
54.6217948717949 14.8747583225437
57.2211538461538 29.0379384602919
59.9455128205128 17.9432491049712
62.8012820512821 25.1102877814879
65.7916666666667 17.3989020430459
68.9230769230769 30.5731973034334
72.2051282051282 29.7319916639931
75.6442307692308 39.9118000454373
79.2467948717949 26.9152637582766
83.0192307692308 27.6711544206535
86.974358974359 63.3890828080745
91.1153846153846 33.7553461561698
95.4519230769231 26.9646048627517
100 36.7069571492296
};
\addplot [, color2, opacity=0.6, mark=triangle*, mark size=0.5, mark options={solid,rotate=180}, only marks]
table {%
1 4.52527736791423
1.04487179487179 3.48965939025133
1.09615384615385 3.13992340746364
1.1474358974359 2.49647460559014
1.20192307692308 2.43122333181587
1.25961538461538 2.23340800458572
1.32051282051282 2.97912199925338
1.38461538461538 2.21040014831487
1.44871794871795 1.54549465221607
1.51923076923077 2.29557023806448
1.58974358974359 1.6877491801342
1.66666666666667 2.87158573368029
1.74679487179487 2.03278993665996
1.83012820512821 2.37952583969416
1.91666666666667 1.80013803672678
2.00641025641026 2.45748988867814
2.1025641025641 2.06155099505751
2.20512820512821 1.72785205413953
2.30769230769231 1.70758966591426
2.41987179487179 1.85208757416761
2.53525641025641 2.43542482225568
2.65384615384615 2.24339202174
2.78205128205128 1.69721469176133
2.91346153846154 2.45589547313914
3.05128205128205 2.0146206473425
3.19871794871795 2.1936490638847
3.34935897435897 4.16548124914194
3.50961538461538 2.65760252123898
3.67628205128205 3.9057493305277
3.8525641025641 5.13965054115391
4.03525641025641 4.3592722218322
4.2275641025641 5.06207448199689
4.42948717948718 5.34944244854862
4.64102564102564 3.95209435607952
4.86217948717949 4.24721422778856
5.09294871794872 2.52357260396196
5.33653846153846 6.94368039399751
5.58974358974359 8.0490136406021
5.85576923076923 5.02707022138034
6.13461538461539 6.05062071447269
6.42628205128205 3.86077019427344
6.73397435897436 10.3846613460274
7.05448717948718 10.2438111355544
7.38782051282051 7.36336232779815
7.74038461538461 11.7865915443441
8.10897435897436 12.7553940192093
8.49679487179487 13.6744274308427
8.90064102564103 8.5038133564079
9.32371794871795 12.8445612089037
9.76923076923077 8.6119050815279
10.2339743589744 12.1070186360426
10.7211538461538 14.5755234409546
11.2307692307692 12.2397628697367
11.7660256410256 11.6163918502717
12.3269230769231 13.5006948076975
12.9134615384615 5.71723308213811
13.5288461538462 10.3825204850124
14.1730769230769 17.1189480069058
14.849358974359 18.7398462069522
15.5544871794872 10.3417004230146
16.2948717948718 16.4103870334112
17.0705128205128 15.4606745104524
17.8846153846154 20.6699076723479
18.7371794871795 13.4008782781975
19.6282051282051 11.7841511373654
20.5641025641026 15.8993438154215
21.5416666666667 13.7995814958098
22.5673076923077 25.3443349973826
23.6442307692308 18.723623397689
24.7692307692308 21.9929788960676
25.9487179487179 14.5219264962742
27.1826923076923 26.3598788202131
28.4775641025641 23.1791535163445
29.8333333333333 28.5352338941639
31.2564102564103 26.466593383818
32.7435897435897 35.1591337592913
34.3044871794872 24.613357627868
35.9358974358974 45.9426116622497
37.6474358974359 17.1237596503316
39.4391025641026 15.3737730465335
41.3173076923077 60.8614181174801
43.2852564102564 28.1345709357659
45.3461538461538 23.26163174889
47.5064102564103 31.3107612648974
49.7692307692308 81.7424724072524
52.1378205128205 123.248839251802
54.6217948717949 47.4483313098618
57.2211538461538 53.9698474249294
59.9455128205128 123.508127450687
62.8012820512821 26.4463841541162
65.7916666666667 127.271718807824
68.9230769230769 60.5628584311922
72.2051282051282 40.7860662223688
75.6442307692308 77.1995284046673
79.2467948717949 62.824523070372
83.0192307692308 151.880174324278
86.974358974359 155.220672293086
91.1153846153846 31.2556755904725
95.4519230769231 50.5514131978859
100 260.456803773997
};
\addlegendentry{sub 16, mc 1}
\addplot [, color2, opacity=0.6, mark=triangle*, mark size=0.5, mark options={solid,rotate=180}, only marks, forget plot]
table {%
1 3.73993427849712
1.04487179487179 4.37969346366231
1.09615384615385 3.50559954240573
1.1474358974359 3.29924757059808
1.20192307692308 1.48753995924418
1.25961538461538 0.878619045222662
1.32051282051282 1.09182447333685
1.38461538461538 1.41958118003238
1.44871794871795 1.512484564328
1.51923076923077 1.25314475595656
1.58974358974359 1.01476948507721
1.66666666666667 1.02140017740672
1.74679487179487 0.882493593626952
1.83012820512821 1.20228263953241
1.91666666666667 1.63320484206123
2.00641025641026 0.946030711339409
2.1025641025641 1.3747101192319
2.20512820512821 0.835498484160748
2.30769230769231 1.70131040610007
2.41987179487179 1.86851141184349
2.53525641025641 2.68302624060519
2.65384615384615 2.72777137605631
2.78205128205128 3.05546767260272
2.91346153846154 2.03203741462894
3.05128205128205 2.45567373084906
3.19871794871795 2.70994096715747
3.34935897435897 3.34325547577413
3.50961538461538 4.36201936233662
3.67628205128205 4.06402382884615
3.8525641025641 5.67727078180935
4.03525641025641 4.70601406492457
4.2275641025641 4.66369615842765
4.42948717948718 2.62174544447643
4.64102564102564 5.43887742504124
4.86217948717949 4.44854139141872
5.09294871794872 5.3810963798723
5.33653846153846 6.27763434008147
5.58974358974359 6.3103547956828
5.85576923076923 6.218428290036
6.13461538461539 7.43811432466027
6.42628205128205 5.46998167973956
6.73397435897436 4.85890443763906
7.05448717948718 5.05934361894326
7.38782051282051 7.53315931611677
7.74038461538461 8.88062386647856
8.10897435897436 9.76736337533608
8.49679487179487 10.8257467638749
8.90064102564103 10.8580359167655
9.32371794871795 9.17157791272819
9.76923076923077 11.2999931016415
10.2339743589744 12.7658768701348
10.7211538461538 11.5097686796924
11.2307692307692 9.03633152521487
11.7660256410256 12.4275347713638
12.3269230769231 10.9066977129399
12.9134615384615 10.3279520040083
13.5288461538462 14.047928357319
14.1730769230769 11.5337622274178
14.849358974359 10.8486124023777
15.5544871794872 8.02630835947863
16.2948717948718 10.317247184572
17.0705128205128 14.2545521155373
17.8846153846154 16.0943799017629
18.7371794871795 16.0938429297932
19.6282051282051 12.6155703579661
20.5641025641026 15.2932530476808
21.5416666666667 7.326784190895
22.5673076923077 14.0269455430548
23.6442307692308 9.13116402709098
24.7692307692308 11.8826481721092
25.9487179487179 17.937262675072
27.1826923076923 12.2791248498514
28.4775641025641 15.3050990814724
29.8333333333333 8.19194279902863
31.2564102564103 18.3953932532389
32.7435897435897 7.21184484413991
34.3044871794872 4.72434470678542
35.9358974358974 7.45121913757698
37.6474358974359 9.92293521847684
39.4391025641026 17.6228854029951
41.3173076923077 14.2563756641113
43.2852564102564 27.5762972934667
45.3461538461538 18.7086655085664
47.5064102564103 4.31232971018852
49.7692307692308 29.0717573692824
52.1378205128205 12.9149979463346
54.6217948717949 24.6831280265365
57.2211538461538 13.8563041232901
59.9455128205128 18.5646008222259
62.8012820512821 41.3845763493634
65.7916666666667 29.7867083639388
68.9230769230769 25.8686464797351
72.2051282051282 26.9103686046171
75.6442307692308 31.3809901685574
79.2467948717949 52.2299776283062
83.0192307692308 4.37307808382972
86.974358974359 29.6511890582759
91.1153846153846 34.6047960230715
95.4519230769231 48.8931767770091
100 13.0505470582581
};
\addplot [, color2, opacity=0.6, mark=triangle*, mark size=0.5, mark options={solid,rotate=180}, only marks, forget plot]
table {%
1 4.42176206004835
1.04487179487179 4.36798250419352
1.09615384615385 3.43384955337547
1.1474358974359 2.92349431575511
1.20192307692308 1.35980666597115
1.25961538461538 1.59001148720031
1.32051282051282 1.67132099556491
1.38461538461538 1.79615987935832
1.44871794871795 1.77947043625291
1.51923076923077 2.47254301934675
1.58974358974359 2.9167436750457
1.66666666666667 2.52808005391887
1.74679487179487 3.24009481855852
1.83012820512821 3.32611391325613
1.91666666666667 3.55555033567029
2.00641025641026 3.09842831074487
2.1025641025641 3.53814070821305
2.20512820512821 3.14745667102966
2.30769230769231 3.91033317117329
2.41987179487179 2.77888070435284
2.53525641025641 2.6116875457042
2.65384615384615 3.09497998776381
2.78205128205128 4.40989205991336
2.91346153846154 3.03584953136062
3.05128205128205 4.2111103458286
3.19871794871795 3.49263943873925
3.34935897435897 3.73201842453361
3.50961538461538 7.83646725527756
3.67628205128205 4.55507057659787
3.8525641025641 6.22723173991911
4.03525641025641 6.66096612831519
4.2275641025641 5.90112183150215
4.42948717948718 5.40916151915434
4.64102564102564 8.18282010955679
4.86217948717949 5.99268938965175
5.09294871794872 9.99645058975519
5.33653846153846 6.20220559372166
5.58974358974359 6.61345739931251
5.85576923076923 12.6308123618207
6.13461538461539 4.92714507830146
6.42628205128205 6.85514870527022
6.73397435897436 7.20466598577236
7.05448717948718 8.41290463409201
7.38782051282051 10.5841197641147
7.74038461538461 6.49120975534813
8.10897435897436 10.1552431612935
8.49679487179487 9.69618198795954
8.90064102564103 8.24881715773494
9.32371794871795 7.48061164139098
9.76923076923077 8.59484395856705
10.2339743589744 8.32999020048047
10.7211538461538 11.2795647251879
11.2307692307692 7.99759475497993
11.7660256410256 10.5164324276331
12.3269230769231 8.38975527741882
12.9134615384615 16.2604455261952
13.5288461538462 12.1235013439153
14.1730769230769 9.33335513802219
14.849358974359 21.0993764283542
15.5544871794872 11.6877077653608
16.2948717948718 9.75800905230884
17.0705128205128 15.253640786479
17.8846153846154 11.5587393903461
18.7371794871795 19.2751864066578
19.6282051282051 14.4394774220218
20.5641025641026 28.1025956798156
21.5416666666667 15.8558693389364
22.5673076923077 8.58469765189554
23.6442307692308 4.06015585744426
24.7692307692308 15.7241854900885
25.9487179487179 33.625354526659
27.1826923076923 19.9754190937573
28.4775641025641 24.4172744822715
29.8333333333333 27.6867813095585
31.2564102564103 13.4304125569643
32.7435897435897 37.1441436776933
34.3044871794872 13.759917744549
35.9358974358974 4.90437272657633
37.6474358974359 16.1094256980974
39.4391025641026 9.36445209718578
41.3173076923077 5.15462338682103
43.2852564102564 16.7378613747633
45.3461538461538 13.5418727075096
47.5064102564103 26.6612432899466
49.7692307692308 12.5669650195287
52.1378205128205 22.3736316488649
54.6217948717949 14.3713574194507
57.2211538461538 25.3020559387075
59.9455128205128 27.1412276548056
62.8012820512821 53.974788792004
65.7916666666667 1.47414417781042
68.9230769230769 34.4028918067215
72.2051282051282 28.5716880136872
75.6442307692308 96.3406499026454
79.2467948717949 76.5660388614058
83.0192307692308 1.22753097880231
86.974358974359 21.5849259438207
91.1153846153846 7.07023606344991
95.4519230769231 17.9879290981212
100 3.32930454138747
};
\end{axis}

\end{tikzpicture}

      \tikzexternaldisable
    \end{minipage}\hfill
    \begin{minipage}{0.50\linewidth}
      \centering
      % defines the pgfplots style "eigspacedefault"
\pgfkeys{/pgfplots/eigspacedefault/.style={
    width=1.03\linewidth,
    height=\goldenRatioInv*1.03*\linewidth,
    every axis plot/.append style={line width = 1pt},
    tick pos = left,
    ylabel near ticks,
    xlabel near ticks,
    xtick align = inside,
    ytick align = inside,
    legend cell align = left,
    legend columns = 1,
    legend pos = north east,
    legend style = {
      fill opacity = 0.9,
      text opacity = 1,
      font = \tiny,
      % column sep=0.1cm,
    },
    legend image post style={scale=2},
    xticklabel style = {font = \small},
    xlabel style = {font = \small},
    axis line style = {black},
    yticklabel style = {font = \small},
    ylabel style = {font = \small},
    title style = {font = \small},
    grid = major,
    grid style = {dashed}
  }
}

\pgfkeys{/pgfplots/eigspacedefaultapp/.style={
    eigspacedefault,
    height=0.6\linewidth,
    legend columns = 2,
  }
}

\pgfkeys{/pgfplots/eigspacenolegend/.style={
    legend image post style = {scale=0},
    legend style = {
      fill opacity = 0,
      draw opacity = 0,
      text opacity = 0,
      font = \small,
      at={(1, 1.025)},
      anchor=south east,
      column sep=0.25cm,
    },
  }
}
%%% Local Variables:
%%% mode: latex
%%% TeX-master: "../main"
%%% End:

      \pgfkeys{/pgfplots/zmystyle/.style={
          eigspacedefaultapp,
          eigspacenolegend,
        }}
      \tikzexternalenable
      \vspace{-6ex}
      % This file was created by tikzplotlib v0.9.7.
\begin{tikzpicture}

\definecolor{color0}{rgb}{0.274509803921569,0.6,0.564705882352941}
\definecolor{color1}{rgb}{0.870588235294118,0.623529411764706,0.0862745098039216}
\definecolor{color2}{rgb}{0.501960784313725,0.184313725490196,0.6}

\begin{axis}[
axis line style={white!10!black},
legend columns=2,
legend style={fill opacity=0.8, draw opacity=1, text opacity=1, at={(0.97,0.03)}, anchor=south east, draw=white!80!black},
log basis x={10},
tick pos=left,
xlabel={epoch (log scale)},
xmajorgrids,
xmin=0.794328234724281, xmax=125.892541179417,
xmode=log,
ylabel={av. rel. error (log scale)},
ymajorgrids,
ymin=0.00597486331503854, ymax=41.4758374908701,
ymode=log,
zmystyle
]
\addplot [, black, opacity=0.6, mark=*, mark size=0.5, mark options={solid}, only marks]
table {%
1 0.0481647262601491
1.04487179487179 0.00893182323326879
1.09615384615385 0.0298083044245325
1.1474358974359 0.049198126390951
1.20192307692308 0.0510053706775294
1.25961538461538 0.0418976321842765
1.32051282051282 0.0663091045647227
1.38461538461538 0.0558354814596189
1.44871794871795 0.0586924257918554
1.51923076923077 0.0492259562177687
1.58974358974359 0.0699166121912698
1.66666666666667 0.0686064093586022
1.74679487179487 0.0698231047483374
1.83012820512821 0.045245578596117
1.91666666666667 0.0429826604230815
2.00641025641026 0.0907018303426103
2.1025641025641 0.0680257716553892
2.20512820512821 0.0855641108711658
2.30769230769231 0.140227225493913
2.41987179487179 0.162529712130487
2.53525641025641 0.056584984515374
2.65384615384615 0.11991007355384
2.78205128205128 0.119005828567258
2.91346153846154 0.0743725661378498
3.05128205128205 0.0984585530010522
3.19871794871795 0.113368738571839
3.34935897435897 0.187047139479857
3.50961538461538 0.1065703917823
3.67628205128205 0.128250047119625
3.8525641025641 0.0721062979406337
4.03525641025641 0.133597395431405
4.2275641025641 0.0989574406412751
4.42948717948718 0.115905553308511
4.64102564102564 0.128925201993619
4.86217948717949 0.165896634435186
5.09294871794872 0.105604334136269
5.33653846153846 0.118167157497179
5.58974358974359 0.129219690406259
5.85576923076923 0.118486035541256
6.13461538461539 0.123753277458868
6.42628205128205 0.121904666834513
6.73397435897436 0.196179170777401
7.05448717948718 0.127325474318148
7.38782051282051 0.0977311042801887
7.74038461538461 0.189354435117945
8.10897435897436 0.16536212615189
8.49679487179487 0.0695220100623556
8.90064102564103 0.157244831108039
9.32371794871795 0.107492913640135
9.76923076923077 0.125604083453168
10.2339743589744 0.330753226118148
10.7211538461538 0.281782967599471
11.2307692307692 0.199138341944303
11.7660256410256 0.137346500232812
12.3269230769231 0.356621697276953
12.9134615384615 0.297600802897003
13.5288461538462 0.474171344855285
14.1730769230769 0.398518125491093
14.849358974359 0.485133459543862
15.5544871794872 0.499572754891822
16.2948717948718 0.456124381884197
17.0705128205128 0.408940509942538
17.8846153846154 0.229808109621134
18.7371794871795 0.571362016569626
19.6282051282051 0.486974663110807
20.5641025641026 0.453445844428226
21.5416666666667 0.56488613565146
22.5673076923077 0.316182053807334
23.6442307692308 0.55254413110865
24.7692307692308 0.676610012095492
25.9487179487179 0.789501394971077
27.1826923076923 0.823471122732481
28.4775641025641 0.67490308731494
29.8333333333333 0.845427543258372
31.2564102564103 1.38989933953536
32.7435897435897 1.77396938305129
34.3044871794872 0.985763239305584
35.9358974358974 1.07630090205091
37.6474358974359 1.5142911917766
39.4391025641026 1.1833764143495
41.3173076923077 1.7935669569951
43.2852564102564 1.59228784153754
45.3461538461538 1.49615313984892
47.5064102564103 1.67259493206422
49.7692307692308 2.11789422151055
52.1378205128205 1.24123082918853
54.6217948717949 2.60845834250429
57.2211538461538 3.04412990092827
59.9455128205128 3.01567982246452
62.8012820512821 1.41219416064172
65.7916666666667 1.51590945929594
68.9230769230769 2.10245967224271
72.2051282051282 2.5046951998959
75.6442307692308 1.59127302014321
79.2467948717949 2.32020875433393
83.0192307692308 1.38346883430171
86.974358974359 3.57938289311748
91.1153846153846 4.00719874478096
95.4519230769231 2.54450711984279
100 4.66460058119178
};
\addlegendentry{mb 128, exact}
\addplot [, black, opacity=0.6, mark=*, mark size=0.5, mark options={solid}, only marks, forget plot]
table {%
1 0.0366536329593465
1.04487179487179 0.0252432968722369
1.09615384615385 0.0313323811825261
1.1474358974359 0.0399346389542062
1.20192307692308 0.0424675707118637
1.25961538461538 0.0809099438636526
1.32051282051282 0.0473757106166367
1.38461538461538 0.0696870120102163
1.44871794871795 0.0653411567230017
1.51923076923077 0.0446589093606474
1.58974358974359 0.0652717392767021
1.66666666666667 0.0954005265841757
1.74679487179487 0.0644653228427255
1.83012820512821 0.0968874849292754
1.91666666666667 0.0829426911264143
2.00641025641026 0.0780643971338453
2.1025641025641 0.086733543291354
2.20512820512821 0.104334673417747
2.30769230769231 0.129836139341399
2.41987179487179 0.119131582052523
2.53525641025641 0.0784556775748679
2.65384615384615 0.148982075433839
2.78205128205128 0.101447865053365
2.91346153846154 0.133777540127076
3.05128205128205 0.141076715678726
3.19871794871795 0.13271662363503
3.34935897435897 0.108736834771044
3.50961538461538 0.14855393188694
3.67628205128205 0.168491824625729
3.8525641025641 0.159710106699048
4.03525641025641 0.160964224939765
4.2275641025641 0.174440802710947
4.42948717948718 0.0973299562012622
4.64102564102564 0.190280755059139
4.86217948717949 0.159927901616508
5.09294871794872 0.130465566744418
5.33653846153846 0.140175152939389
5.58974358974359 0.123494332460069
5.85576923076923 0.207429403604048
6.13461538461539 0.126593794065392
6.42628205128205 0.135692166299784
6.73397435897436 0.160335761955486
7.05448717948718 0.125725005338038
7.38782051282051 0.14852180234765
7.74038461538461 0.133657832564994
8.10897435897436 0.148000788609891
8.49679487179487 0.108117708720655
8.90064102564103 0.213210130363065
9.32371794871795 0.174374696672677
9.76923076923077 0.15274833055381
10.2339743589744 0.0928795646435957
10.7211538461538 0.259528330206696
11.2307692307692 0.213179676587549
11.7660256410256 0.152838928252677
12.3269230769231 0.197188059745517
12.9134615384615 0.266028272697386
13.5288461538462 0.250793847763522
14.1730769230769 0.299849396089218
14.849358974359 0.330692113844124
15.5544871794872 0.309543594186351
16.2948717948718 0.187358445012317
17.0705128205128 0.25670894249221
17.8846153846154 0.361501485417538
18.7371794871795 0.344725465709121
19.6282051282051 0.429855011951289
20.5641025641026 0.238334813101791
21.5416666666667 0.29329005050519
22.5673076923077 0.228146687589614
23.6442307692308 0.418026933728069
24.7692307692308 0.609303955881061
25.9487179487179 0.471270939096792
27.1826923076923 0.508674222865127
28.4775641025641 0.441291434297093
29.8333333333333 0.511424944618731
31.2564102564103 0.596441444665098
32.7435897435897 0.613659716124804
34.3044871794872 0.67950667833063
35.9358974358974 0.88012596982642
37.6474358974359 1.10181907370781
39.4391025641026 0.968645885065341
41.3173076923077 0.997723198878946
43.2852564102564 1.11212569781421
45.3461538461538 1.56715771337753
47.5064102564103 0.734011629978347
49.7692307692308 0.971948616374934
52.1378205128205 2.43130213075099
54.6217948717949 1.79110232224185
57.2211538461538 1.66921740524976
59.9455128205128 1.51273176569921
62.8012820512821 2.45157187393574
65.7916666666667 1.0348219250848
68.9230769230769 2.55801777094604
72.2051282051282 2.426631078049
75.6442307692308 3.61458934738115
79.2467948717949 5.13474800646262
83.0192307692308 3.33103583835029
86.974358974359 4.93043946690507
91.1153846153846 5.07199164175575
95.4519230769231 3.62059054107935
100 5.70772387836216
};
\addplot [, black, opacity=0.6, mark=*, mark size=0.5, mark options={solid}, only marks, forget plot]
table {%
1 0.0444660141201666
1.04487179487179 0.0545835757051577
1.09615384615385 0.0743715743159385
1.1474358974359 0.0879104418792352
1.20192307692308 0.0760180613321268
1.25961538461538 0.0743029043925961
1.32051282051282 0.070477510212472
1.38461538461538 0.074145893168814
1.44871794871795 0.0701863851681623
1.51923076923077 0.0939359647174406
1.58974358974359 0.0710724249601814
1.66666666666667 0.0890449937241027
1.74679487179487 0.0842544157588987
1.83012820512821 0.0716298273949814
1.91666666666667 0.0948236151374263
2.00641025641026 0.0839427895611165
2.1025641025641 0.0769449315245072
2.20512820512821 0.0885745403528112
2.30769230769231 0.0751605822136459
2.41987179487179 0.0753710954985524
2.53525641025641 0.0701952492933095
2.65384615384615 0.103513447599466
2.78205128205128 0.100272403541854
2.91346153846154 0.0957864903969283
3.05128205128205 0.0683374234763366
3.19871794871795 0.0828560872766387
3.34935897435897 0.117296830211463
3.50961538461538 0.0849743875056799
3.67628205128205 0.0891784796760344
3.8525641025641 0.0700662920827926
4.03525641025641 0.10362873146049
4.2275641025641 0.0837798969503819
4.42948717948718 0.0912073670837568
4.64102564102564 0.112940793609045
4.86217948717949 0.111166130014405
5.09294871794872 0.111972890955712
5.33653846153846 0.0858233159333932
5.58974358974359 0.0943433000737588
5.85576923076923 0.0864941695782179
6.13461538461539 0.0823596606007852
6.42628205128205 0.0931753893379287
6.73397435897436 0.0978968324151293
7.05448717948718 0.0576971508067005
7.38782051282051 0.088054904841047
7.74038461538461 0.0729535927073132
8.10897435897436 0.0739147450116969
8.49679487179487 0.0576455932144269
8.90064102564103 0.141081096602459
9.32371794871795 0.151850379087118
9.76923076923077 0.125793732955918
10.2339743589744 0.192282171438471
10.7211538461538 0.0962363312098022
11.2307692307692 0.198326349640548
11.7660256410256 0.1130797265532
12.3269230769231 0.106042660317627
12.9134615384615 0.141219923698201
13.5288461538462 0.187801686115597
14.1730769230769 0.1669245888253
14.849358974359 0.262820537740248
15.5544871794872 0.233210124936074
16.2948717948718 0.288734466376503
17.0705128205128 0.360905244472489
17.8846153846154 0.304813797550955
18.7371794871795 0.497721333429803
19.6282051282051 0.473429965787322
20.5641025641026 0.286205577968948
21.5416666666667 0.241755271685148
22.5673076923077 0.432443243988704
23.6442307692308 0.421561674130571
24.7692307692308 0.735429487267614
25.9487179487179 0.530974504768497
27.1826923076923 0.617282202361386
28.4775641025641 0.480990585557772
29.8333333333333 0.596316669451817
31.2564102564103 0.646896250344171
32.7435897435897 0.91311103914561
34.3044871794872 0.890800480905825
35.9358974358974 1.21801000525108
37.6474358974359 1.06714919458076
39.4391025641026 1.35765847116207
41.3173076923077 0.965307948776297
43.2852564102564 1.12796337783559
45.3461538461538 1.62718110777152
47.5064102564103 1.17082370489644
49.7692307692308 0.987357455506011
52.1378205128205 1.62594288828514
54.6217948717949 1.63569750270337
57.2211538461538 1.06378672973088
59.9455128205128 3.47422852458056
62.8012820512821 2.27545025579505
65.7916666666667 1.89235372046896
68.9230769230769 2.04795681475484
72.2051282051282 3.23270128023378
75.6442307692308 1.56266537795744
79.2467948717949 1.8413999074967
83.0192307692308 3.34484268917849
86.974358974359 2.50994353354435
91.1153846153846 2.95981149676043
95.4519230769231 3.15266806444522
100 4.74666652451393
};
\addplot [, color0, opacity=0.6, mark=diamond*, mark size=0.5, mark options={solid}, only marks]
table {%
1 0.330974532577985
1.04487179487179 0.0763483207437193
1.09615384615385 0.136094921786323
1.1474358974359 0.0821377487478444
1.20192307692308 0.105791056741542
1.25961538461538 0.153348801669394
1.32051282051282 0.198160076825187
1.38461538461538 0.222339402106514
1.44871794871795 0.185579328336012
1.51923076923077 0.245378739494771
1.58974358974359 0.333551007710308
1.66666666666667 0.22052038517152
1.74679487179487 0.294833950602902
1.83012820512821 0.317566434019981
1.91666666666667 0.276578273631185
2.00641025641026 0.45221202875416
2.1025641025641 0.272158008307026
2.20512820512821 0.294173084070565
2.30769230769231 0.55658344043632
2.41987179487179 0.754018729900758
2.53525641025641 0.390252553068908
2.65384615384615 0.597593247004023
2.78205128205128 0.458648498756627
2.91346153846154 0.248109609528067
3.05128205128205 0.293135902454659
3.19871794871795 0.569825907706979
3.34935897435897 0.5142391089821
3.50961538461538 0.272661902884058
3.67628205128205 0.434824296177735
3.8525641025641 0.709378944550307
4.03525641025641 0.432081260241085
4.2275641025641 0.430933941668571
4.42948717948718 0.403338510479716
4.64102564102564 0.560344283612859
4.86217948717949 0.608120714525116
5.09294871794872 0.660916346850049
5.33653846153846 0.915459756217095
5.58974358974359 0.922410572217166
5.85576923076923 0.834127288579097
6.13461538461539 0.935646815174901
6.42628205128205 0.630829711104193
6.73397435897436 1.53440603611342
7.05448717948718 0.55913555768627
7.38782051282051 1.27007104315965
7.74038461538461 1.43680049651881
8.10897435897436 1.05971012535529
8.49679487179487 1.6870288171103
8.90064102564103 1.58623545540474
9.32371794871795 1.98597997285567
9.76923076923077 1.02930226157195
10.2339743589744 2.84399720797084
10.7211538461538 2.01275564885554
11.2307692307692 1.53088251548829
11.7660256410256 2.40480437500352
12.3269230769231 2.25981961257556
12.9134615384615 2.30476892736641
13.5288461538462 1.88906477888546
14.1730769230769 2.23200023480859
14.849358974359 3.62364214114656
15.5544871794872 2.85601242966928
16.2948717948718 4.33260912150781
17.0705128205128 4.87079177724888
17.8846153846154 1.53988227463408
18.7371794871795 3.88588007363551
19.6282051282051 3.31265331300446
20.5641025641026 3.91801068511319
21.5416666666667 2.74257909625741
22.5673076923077 3.67450437645457
23.6442307692308 3.39577670643235
24.7692307692308 4.12827510664355
25.9487179487179 3.17243932444984
27.1826923076923 4.42219462696815
28.4775641025641 5.39852960546088
29.8333333333333 1.94038045452642
31.2564102564103 3.51243939464175
32.7435897435897 2.82353301174978
34.3044871794872 3.05726323576334
35.9358974358974 3.61892052990175
37.6474358974359 4.55199357660914
39.4391025641026 3.53771970557758
41.3173076923077 8.80370089054874
43.2852564102564 2.98062629687169
45.3461538461538 6.60923170192452
47.5064102564103 6.46469489206567
49.7692307692308 8.9962749670868
52.1378205128205 6.08789842808706
54.6217948717949 3.41949792477839
57.2211538461538 7.11025303626032
59.9455128205128 4.30714651800206
62.8012820512821 7.10980897926797
65.7916666666667 1.19857700364901
68.9230769230769 2.67515119231207
72.2051282051282 0.707438843307918
75.6442307692308 1.82195895034907
79.2467948717949 0.737969094671479
83.0192307692308 2.89757778715561
86.974358974359 1.65412225130328
91.1153846153846 8.88516849271167
95.4519230769231 7.40273314559867
100 1.55124721220694
};
\addlegendentry{sub 16, exact}
\addplot [, color0, opacity=0.6, mark=diamond*, mark size=0.5, mark options={solid}, only marks, forget plot]
table {%
1 0.16906627790583
1.04487179487179 0.0664099002553621
1.09615384615385 0.0873534167103208
1.1474358974359 0.0997760546457215
1.20192307692308 0.167750091803684
1.25961538461538 0.39619033952895
1.32051282051282 0.326396985310562
1.38461538461538 0.449879277356578
1.44871794871795 0.305761345984296
1.51923076923077 0.446920289977702
1.58974358974359 0.33781230900998
1.66666666666667 0.389806981526774
1.74679487179487 0.381676204714222
1.83012820512821 0.503108564842765
1.91666666666667 0.568359756494032
2.00641025641026 0.588769747324244
2.1025641025641 0.422350326968882
2.20512820512821 0.589863058617933
2.30769230769231 0.491788976289277
2.41987179487179 0.400508098257859
2.53525641025641 0.846532728501852
2.65384615384615 0.50407339508426
2.78205128205128 0.453052410527197
2.91346153846154 1.01992261932302
3.05128205128205 0.583733076672159
3.19871794871795 0.472965405785942
3.34935897435897 0.836941676040715
3.50961538461538 0.869862556585618
3.67628205128205 0.899925480403117
3.8525641025641 0.864649827164414
4.03525641025641 0.684284615436887
4.2275641025641 0.930596515602938
4.42948717948718 0.674506422746884
4.64102564102564 1.22753662266165
4.86217948717949 1.04306714412524
5.09294871794872 1.55405426189282
5.33653846153846 1.41135751173015
5.58974358974359 1.08075318026962
5.85576923076923 1.42252608590465
6.13461538461539 1.38554793034983
6.42628205128205 1.46262400538629
6.73397435897436 1.34819996350102
7.05448717948718 1.34923225954777
7.38782051282051 1.8865700153175
7.74038461538461 2.14501632083284
8.10897435897436 2.0327632914125
8.49679487179487 1.59122683939144
8.90064102564103 1.72444957273398
9.32371794871795 2.27571563356696
9.76923076923077 1.94517265320788
10.2339743589744 1.82671169988985
10.7211538461538 1.74173119364363
11.2307692307692 2.24438115956503
11.7660256410256 1.41142996119447
12.3269230769231 1.83369839902896
12.9134615384615 1.69242512152441
13.5288461538462 1.7686680098991
14.1730769230769 2.14877823498303
14.849358974359 2.38451785614147
15.5544871794872 2.59965744609026
16.2948717948718 2.35779634484153
17.0705128205128 2.11876332929041
17.8846153846154 2.85918935055623
18.7371794871795 1.33249552254665
19.6282051282051 3.44715920588557
20.5641025641026 1.54523704505113
21.5416666666667 2.06050016890377
22.5673076923077 1.81214765160614
23.6442307692308 1.42111082983451
24.7692307692308 1.81793188377388
25.9487179487179 4.08739662191254
27.1826923076923 2.09987775395731
28.4775641025641 3.17972424444143
29.8333333333333 2.13508354665855
31.2564102564103 2.06818343771886
32.7435897435897 1.35532283318404
34.3044871794872 2.94072733732187
35.9358974358974 1.69228376830531
37.6474358974359 2.3160755375561
39.4391025641026 1.8605384628008
41.3173076923077 1.97752207934982
43.2852564102564 1.73255555664562
45.3461538461538 3.58406405701702
47.5064102564103 3.21703918822005
49.7692307692308 1.36932927654228
52.1378205128205 4.68250283673085
54.6217948717949 5.72436157692068
57.2211538461538 5.9379253285953
59.9455128205128 5.28898881048618
62.8012820512821 9.23042921792035
65.7916666666667 0.862893757775296
68.9230769230769 13.4832762231986
72.2051282051282 4.1839857962748
75.6442307692308 7.80940700789848
79.2467948717949 5.29799929768064
83.0192307692308 10.5704673351998
86.974358974359 2.92442958641217
91.1153846153846 2.47158825784863
95.4519230769231 11.0901200719616
100 6.50223220993256
};
\addplot [, color0, opacity=0.6, mark=diamond*, mark size=0.5, mark options={solid}, only marks, forget plot]
table {%
1 0.174053793507957
1.04487179487179 0.0819708625019508
1.09615384615385 0.0913261036977374
1.1474358974359 0.163315507938045
1.20192307692308 0.0702096792559588
1.25961538461538 0.206282662735312
1.32051282051282 0.185822065595469
1.38461538461538 0.22662630157049
1.44871794871795 0.155281769672314
1.51923076923077 0.3078060012622
1.58974358974359 0.269762274924179
1.66666666666667 0.336248849796297
1.74679487179487 0.251211128341119
1.83012820512821 0.480386563331101
1.91666666666667 0.349048326149268
2.00641025641026 0.419592321035152
2.1025641025641 0.456981904286333
2.20512820512821 0.423387340053478
2.30769230769231 0.481268684732365
2.41987179487179 0.57593996154639
2.53525641025641 0.529922101359996
2.65384615384615 0.518066065626318
2.78205128205128 0.563604430719331
2.91346153846154 0.551634071175184
3.05128205128205 0.623500309040249
3.19871794871795 0.599491493921606
3.34935897435897 0.599798301597054
3.50961538461538 0.663611432251999
3.67628205128205 0.824107104617882
3.8525641025641 0.950805159899122
4.03525641025641 1.10722578026027
4.2275641025641 0.919091596641466
4.42948717948718 0.803721030008229
4.64102564102564 0.492741318463844
4.86217948717949 0.769792986893818
5.09294871794872 0.697274755517075
5.33653846153846 1.00214430404669
5.58974358974359 1.1240945876121
5.85576923076923 1.69988971835453
6.13461538461539 1.34187180121053
6.42628205128205 1.19436440064734
6.73397435897436 1.12195323591621
7.05448717948718 1.07646520571312
7.38782051282051 1.47497962176924
7.74038461538461 1.71343584892378
8.10897435897436 1.6796761402667
8.49679487179487 1.11330376049819
8.90064102564103 1.36260605589753
9.32371794871795 1.56066861445381
9.76923076923077 1.75414181964781
10.2339743589744 1.8019143045655
10.7211538461538 2.17833140560615
11.2307692307692 2.17818546295068
11.7660256410256 2.12305558729835
12.3269230769231 1.80110438279599
12.9134615384615 1.49948674267637
13.5288461538462 2.52471803639908
14.1730769230769 1.87715432552605
14.849358974359 1.34659610949503
15.5544871794872 1.82191978261265
16.2948717948718 1.95857516432376
17.0705128205128 2.78187298060939
17.8846153846154 2.84521719376886
18.7371794871795 4.71938547034506
19.6282051282051 2.87425393353845
20.5641025641026 3.64428786764689
21.5416666666667 2.63760237373949
22.5673076923077 3.75862076546827
23.6442307692308 3.52189904135026
24.7692307692308 2.86614613131886
25.9487179487179 2.89259312333234
27.1826923076923 4.71950487367748
28.4775641025641 4.60917171951671
29.8333333333333 4.2347391688192
31.2564102564103 3.73826733824549
32.7435897435897 5.75244617051907
34.3044871794872 5.31421461396553
35.9358974358974 6.36290726956407
37.6474358974359 5.68976738275254
39.4391025641026 6.0862465034242
41.3173076923077 6.45197874559855
43.2852564102564 6.83268053935087
45.3461538461538 5.86181839063042
47.5064102564103 6.98594386168871
49.7692307692308 4.10091920915744
52.1378205128205 9.49393612811615
54.6217948717949 6.76061113470484
57.2211538461538 3.46117367281755
59.9455128205128 12.4729089777767
62.8012820512821 8.96627080462092
65.7916666666667 9.9831761897469
68.9230769230769 7.1120650013214
72.2051282051282 5.83204681300014
75.6442307692308 5.63544454434823
79.2467948717949 4.04618024508043
83.0192307692308 9.72568038706876
86.974358974359 4.51853870369501
91.1153846153846 4.74219839575057
95.4519230769231 16.1003661997277
100 9.49314256145172
};
\addplot [, color1, opacity=0.6, mark=square*, mark size=0.5, mark options={solid}, only marks]
table {%
1 0.29631924222818
1.04487179487179 0.10736132930992
1.09615384615385 0.175231188148394
1.1474358974359 0.165526967946877
1.20192307692308 0.18918280623407
1.25961538461538 0.209388812953391
1.32051282051282 0.176453956142419
1.38461538461538 0.13401926602561
1.44871794871795 0.135330723716228
1.51923076923077 0.16962268963043
1.58974358974359 0.137589542310855
1.66666666666667 0.232669722958018
1.74679487179487 0.175602381940541
1.83012820512821 0.226943544616396
1.91666666666667 0.247301794784497
2.00641025641026 0.240605635331139
2.1025641025641 0.0908480901949281
2.20512820512821 0.326428715073817
2.30769230769231 0.151991922961267
2.41987179487179 0.353504601522228
2.53525641025641 0.228998378198869
2.65384615384615 0.228338050967701
2.78205128205128 0.204919962112473
2.91346153846154 0.242092181467069
3.05128205128205 0.329484362792237
3.19871794871795 0.478278611658819
3.34935897435897 0.221776768998718
3.50961538461538 0.392523219624435
3.67628205128205 0.269118246700704
3.8525641025641 0.291522958424218
4.03525641025641 0.315196918117569
4.2275641025641 0.262232173665524
4.42948717948718 0.424346689876122
4.64102564102564 0.29488784525823
4.86217948717949 0.275668253098777
5.09294871794872 0.37618200520587
5.33653846153846 0.371117266070045
5.58974358974359 0.291276639299402
5.85576923076923 0.300354088345348
6.13461538461539 0.167979029802667
6.42628205128205 0.271760961179414
6.73397435897436 0.521169052593977
7.05448717948718 0.219495135577623
7.38782051282051 0.397141366638824
7.74038461538461 0.331044744279962
8.10897435897436 0.31567372876141
8.49679487179487 0.247894456798542
8.90064102564103 0.315164030064086
9.32371794871795 0.429234753774052
9.76923076923077 0.629184866533671
10.2339743589744 0.505627655372286
10.7211538461538 0.57071685985986
11.2307692307692 0.541586824598519
11.7660256410256 0.908000085394528
12.3269230769231 0.915786134844896
12.9134615384615 0.31561241057814
13.5288461538462 0.647330659789976
14.1730769230769 0.634845710810815
14.849358974359 1.20514320908963
15.5544871794872 1.55836837936122
16.2948717948718 0.630365338805715
17.0705128205128 0.472704938482278
17.8846153846154 0.677462697415379
18.7371794871795 1.22092819315481
19.6282051282051 0.82062878885633
20.5641025641026 0.741305009559229
21.5416666666667 0.808376617420328
22.5673076923077 0.335130241754716
23.6442307692308 0.83469699037397
24.7692307692308 1.10246799616643
25.9487179487179 1.61710302374493
27.1826923076923 1.96212531896562
28.4775641025641 1.18583802053848
29.8333333333333 1.97515079061986
31.2564102564103 2.07839750324043
32.7435897435897 1.73745002183718
34.3044871794872 2.0495945506166
35.9358974358974 2.221907107799
37.6474358974359 1.39863323031076
39.4391025641026 2.61062639091893
41.3173076923077 2.55233615621182
43.2852564102564 3.20804294574068
45.3461538461538 3.1071779968871
47.5064102564103 3.45030001196084
49.7692307692308 2.39880803330899
52.1378205128205 3.9296010496512
54.6217948717949 3.55715326075663
57.2211538461538 5.4021052842334
59.9455128205128 5.68822538135436
62.8012820512821 5.84978281727238
65.7916666666667 2.59658679616195
68.9230769230769 3.02419834830912
72.2051282051282 2.09799227920311
75.6442307692308 1.57094816314055
79.2467948717949 7.72616629488236
83.0192307692308 2.85119296366364
86.974358974359 2.02793410601231
91.1153846153846 5.49711697740367
95.4519230769231 3.61206275362241
100 4.80166296133084
};
\addlegendentry{mb 128, mc 1}
\addplot [, color1, opacity=0.6, mark=square*, mark size=0.5, mark options={solid}, only marks, forget plot]
table {%
1 0.369650619310464
1.04487179487179 0.193457829481574
1.09615384615385 0.150315071661681
1.1474358974359 0.0892669941847719
1.20192307692308 0.0505866790468222
1.25961538461538 0.150316066567346
1.32051282051282 0.232827492313542
1.38461538461538 0.103509595521156
1.44871794871795 0.230889177976985
1.51923076923077 0.493665324849689
1.58974358974359 0.531797788101524
1.66666666666667 0.127325621836017
1.74679487179487 0.337004470534017
1.83012820512821 0.332102414569717
1.91666666666667 0.267590913224829
2.00641025641026 0.281082399767335
2.1025641025641 0.2387679208875
2.20512820512821 0.534266659971488
2.30769230769231 0.312192552108486
2.41987179487179 0.182348028270291
2.53525641025641 0.192656372303306
2.65384615384615 0.444303493403893
2.78205128205128 0.225315811069394
2.91346153846154 0.152380088952251
3.05128205128205 0.36878644475898
3.19871794871795 0.407757312904984
3.34935897435897 0.441840049463423
3.50961538461538 0.461124170423943
3.67628205128205 0.272931948126399
3.8525641025641 0.46871740009067
4.03525641025641 0.252129672307004
4.2275641025641 0.38740392321018
4.42948717948718 0.54412889738202
4.64102564102564 0.442703890341149
4.86217948717949 0.714525793060323
5.09294871794872 0.553010273013517
5.33653846153846 0.42679824625036
5.58974358974359 0.624001322292482
5.85576923076923 0.583690504617427
6.13461538461539 0.491138040934336
6.42628205128205 0.667066974841042
6.73397435897436 0.679422428951169
7.05448717948718 0.333236454826544
7.38782051282051 0.536072146462685
7.74038461538461 0.655307718154813
8.10897435897436 0.564289362670691
8.49679487179487 0.60834758583015
8.90064102564103 0.818067388350327
9.32371794871795 0.814775970886048
9.76923076923077 0.781578395040117
10.2339743589744 0.704388344640055
10.7211538461538 0.572642443101464
11.2307692307692 0.88952244019597
11.7660256410256 0.768654632073563
12.3269230769231 0.658831932158828
12.9134615384615 1.23548272591689
13.5288461538462 0.798258761887479
14.1730769230769 1.02706121185688
14.849358974359 1.49813441569936
15.5544871794872 1.14330421610286
16.2948717948718 0.975360726204534
17.0705128205128 1.06750582543101
17.8846153846154 1.35098994572558
18.7371794871795 1.77613257387662
19.6282051282051 2.04541188307628
20.5641025641026 1.08191804942396
21.5416666666667 2.2243079428027
22.5673076923077 1.5467608070308
23.6442307692308 2.84037194150962
24.7692307692308 1.19756313093757
25.9487179487179 1.7424682341748
27.1826923076923 2.82645011100722
28.4775641025641 1.37435589986102
29.8333333333333 1.36821291466911
31.2564102564103 1.73286266345066
32.7435897435897 1.67558393610549
34.3044871794872 3.20617429264536
35.9358974358974 2.95396755694841
37.6474358974359 1.83009823412779
39.4391025641026 2.92028209322183
41.3173076923077 2.13439534639129
43.2852564102564 2.80641276325195
45.3461538461538 3.18650842092397
47.5064102564103 1.83995543315741
49.7692307692308 2.38072295533859
52.1378205128205 7.01024289183509
54.6217948717949 2.96730454186263
57.2211538461538 1.26761725829015
59.9455128205128 3.97711517061271
62.8012820512821 4.78963040934914
65.7916666666667 2.77499263339979
68.9230769230769 5.92646930471862
72.2051282051282 4.64587870683695
75.6442307692308 3.64710284550426
79.2467948717949 6.38692554830251
83.0192307692308 10.2937582251772
86.974358974359 5.49585063918833
91.1153846153846 5.36116941731256
95.4519230769231 5.81525155224683
100 4.580181008266
};
\addplot [, color1, opacity=0.6, mark=square*, mark size=0.5, mark options={solid}, only marks, forget plot]
table {%
1 0.256632796552155
1.04487179487179 0.0894536672175192
1.09615384615385 0.10448217422256
1.1474358974359 0.144869247205118
1.20192307692308 0.13473814927014
1.25961538461538 0.131174025033824
1.32051282051282 0.193041257118067
1.38461538461538 0.129724671955957
1.44871794871795 0.13509084178082
1.51923076923077 0.143691406957827
1.58974358974359 0.123552803163759
1.66666666666667 0.102421302939904
1.74679487179487 0.120775892825652
1.83012820512821 0.151257292248905
1.91666666666667 0.121965721446286
2.00641025641026 0.137316296057666
2.1025641025641 0.222347679154478
2.20512820512821 0.209723895043885
2.30769230769231 0.231196358833672
2.41987179487179 0.246885046183677
2.53525641025641 0.283639883842729
2.65384615384615 0.264807322202554
2.78205128205128 0.260527348164957
2.91346153846154 0.288828607313827
3.05128205128205 0.257040550512897
3.19871794871795 0.375887987232095
3.34935897435897 0.255638865667796
3.50961538461538 0.19242981093214
3.67628205128205 0.267763444920624
3.8525641025641 0.272422224897641
4.03525641025641 0.219927519789324
4.2275641025641 0.226985178827012
4.42948717948718 0.497597245796077
4.64102564102564 0.2502871185142
4.86217948717949 0.155226355654943
5.09294871794872 0.301619212215286
5.33653846153846 0.465784299766812
5.58974358974359 0.428569153994954
5.85576923076923 0.33147488789614
6.13461538461539 0.465202103102581
6.42628205128205 0.329999664706886
6.73397435897436 0.505823820460514
7.05448717948718 0.256195452577493
7.38782051282051 0.481620027141005
7.74038461538461 0.544693639837169
8.10897435897436 0.744650346991055
8.49679487179487 0.409700676107629
8.90064102564103 0.519677648459471
9.32371794871795 0.7030038947392
9.76923076923077 0.663083443380396
10.2339743589744 0.480885561594335
10.7211538461538 0.618510810524575
11.2307692307692 0.870655372061272
11.7660256410256 0.551699262171006
12.3269230769231 0.767099003010659
12.9134615384615 0.785916924495134
13.5288461538462 1.59549547193802
14.1730769230769 1.41596890867834
14.849358974359 0.989916491880786
15.5544871794872 1.23529646494249
16.2948717948718 1.21528782872227
17.0705128205128 1.17776657616274
17.8846153846154 1.24051539735067
18.7371794871795 1.76889174886151
19.6282051282051 1.24292580716916
20.5641025641026 1.92678877061292
21.5416666666667 2.40878073553638
22.5673076923077 1.16643466957795
23.6442307692308 1.41939126239482
24.7692307692308 2.37021558933617
25.9487179487179 2.19101447407825
27.1826923076923 1.78880786588367
28.4775641025641 2.2547475878615
29.8333333333333 2.81904230082807
31.2564102564103 2.74687925849761
32.7435897435897 2.00827985223626
34.3044871794872 1.71770148292682
35.9358974358974 2.88010241984294
37.6474358974359 3.22439254056182
39.4391025641026 3.10873636609561
41.3173076923077 2.04293393076521
43.2852564102564 3.00796908456016
45.3461538461538 3.49688092720105
47.5064102564103 2.2090671637316
49.7692307692308 4.99788193742409
52.1378205128205 4.18199840259105
54.6217948717949 4.45296717487763
57.2211538461538 2.34226275407922
59.9455128205128 2.81741178833831
62.8012820512821 3.48206572617722
65.7916666666667 5.05055387223092
68.9230769230769 7.35076752315853
72.2051282051282 4.17991697190874
75.6442307692308 0.924375137864136
79.2467948717949 0.900398460768346
83.0192307692308 5.66368468775072
86.974358974359 2.33733442533049
91.1153846153846 2.26155083880299
95.4519230769231 2.0770354705942
100 2.41804236879183
};
\addplot [, color2, opacity=0.6, mark=triangle*, mark size=0.5, mark options={solid,rotate=180}, only marks]
table {%
1 2.7116957130773
1.04487179487179 1.90703524060776
1.09615384615385 1.57896057261964
1.1474358974359 0.629697548078325
1.20192307692308 0.37868808857228
1.25961538461538 0.509934736898811
1.32051282051282 0.4238400101238
1.38461538461538 0.500704324023029
1.44871794871795 0.496733578027076
1.51923076923077 0.829724488320554
1.58974358974359 0.551979747557748
1.66666666666667 0.875437493432824
1.74679487179487 0.999398925967993
1.83012820512821 0.630318687476502
1.91666666666667 1.46135157723796
2.00641025641026 1.92366226645603
2.1025641025641 1.1359672252209
2.20512820512821 1.56643159100116
2.30769230769231 3.17347707863881
2.41987179487179 1.94966373210486
2.53525641025641 2.49271169865796
2.65384615384615 2.47798663253022
2.78205128205128 3.33242544146202
2.91346153846154 1.34844079004776
3.05128205128205 2.51352344986792
3.19871794871795 2.69500791719761
3.34935897435897 1.95193040406625
3.50961538461538 2.80332211602695
3.67628205128205 1.68498315195879
3.8525641025641 2.70452533968547
4.03525641025641 4.15240359561922
4.2275641025641 1.44596798026842
4.42948717948718 2.54048811461058
4.64102564102564 2.15268758617432
4.86217948717949 2.67885062113273
5.09294871794872 2.24801434306822
5.33653846153846 2.13210494725286
5.58974358974359 3.27691431321632
5.85576923076923 3.08638450497077
6.13461538461539 1.53724735385821
6.42628205128205 2.21658948348522
6.73397435897436 2.37442151515454
7.05448717948718 2.23528979161447
7.38782051282051 3.07429503915801
7.74038461538461 2.62922131749555
8.10897435897436 1.9551178955598
8.49679487179487 2.87288990702237
8.90064102564103 2.67675344762236
9.32371794871795 2.95301191416148
9.76923076923077 1.39774764129334
10.2339743589744 4.15091453316246
10.7211538461538 3.38823197387609
11.2307692307692 3.41880725225426
11.7660256410256 3.11006386878405
12.3269230769231 3.08157310312532
12.9134615384615 1.72597765539131
13.5288461538462 1.55340431032058
14.1730769230769 2.64206068466862
14.849358974359 4.36774262667594
15.5544871794872 4.34699585921154
16.2948717948718 2.62428947161731
17.0705128205128 3.54037588941924
17.8846153846154 2.55062569289541
18.7371794871795 3.21164430724795
19.6282051282051 2.83031122126425
20.5641025641026 5.73684752522929
21.5416666666667 3.0585903850992
22.5673076923077 1.5695795768435
23.6442307692308 4.09153943119933
24.7692307692308 2.0811845183465
25.9487179487179 2.24652162480168
27.1826923076923 2.99168030156605
28.4775641025641 3.77917627819228
29.8333333333333 2.12321937868936
31.2564102564103 3.33859774158384
32.7435897435897 1.99376787077307
34.3044871794872 2.52253181785097
35.9358974358974 4.15540284101793
37.6474358974359 3.00278530237905
39.4391025641026 7.35431927183273
41.3173076923077 9.81425148500532
43.2852564102564 2.57534575083468
45.3461538461538 2.12982931291239
47.5064102564103 5.54714686816915
49.7692307692308 4.95330669280461
52.1378205128205 12.4674502355422
54.6217948717949 5.77273365523049
57.2211538461538 8.38334310621624
59.9455128205128 10.4180210179661
62.8012820512821 10.1932705476276
65.7916666666667 0.934497986247502
68.9230769230769 0.748881162568975
72.2051282051282 3.31766602039932
75.6442307692308 0.902262317569775
79.2467948717949 0.99786984921471
83.0192307692308 2.59922253426038
86.974358974359 0.901039062513488
91.1153846153846 12.6855658845583
95.4519230769231 2.42609834526837
100 0.956583893254537
};
\addlegendentry{sub 16, mc 1}
\addplot [, color2, opacity=0.6, mark=triangle*, mark size=0.5, mark options={solid,rotate=180}, only marks, forget plot]
table {%
1 4.35081286153224
1.04487179487179 1.08235070912253
1.09615384615385 1.131508709449
1.1474358974359 2.07775802740634
1.20192307692308 0.840056478112608
1.25961538461538 1.30931428579238
1.32051282051282 1.67336176290969
1.38461538461538 1.3046935033301
1.44871794871795 1.34938207314804
1.51923076923077 2.8306867717532
1.58974358974359 2.14499431080101
1.66666666666667 1.52078405784138
1.74679487179487 1.54193047645152
1.83012820512821 1.44929319288329
1.91666666666667 1.78672215357197
2.00641025641026 1.01310849435383
2.1025641025641 1.35022622310837
2.20512820512821 3.1405678910742
2.30769230769231 1.99991783369318
2.41987179487179 1.51189497639448
2.53525641025641 2.21555109487404
2.65384615384615 1.96489378703476
2.78205128205128 1.64887078449868
2.91346153846154 3.06518290435502
3.05128205128205 1.73543880322095
3.19871794871795 2.02801711717054
3.34935897435897 2.6194712730649
3.50961538461538 3.07394458700035
3.67628205128205 2.33406434928733
3.8525641025641 4.34809708798051
4.03525641025641 2.02166023022471
4.2275641025641 3.98540837261387
4.42948717948718 2.97589353799155
4.64102564102564 2.83585445964612
4.86217948717949 2.62585337256691
5.09294871794872 3.59068301014706
5.33653846153846 4.59987208977493
5.58974358974359 5.1106397677182
5.85576923076923 3.11322471366266
6.13461538461539 2.22535361504885
6.42628205128205 3.61063901026721
6.73397435897436 3.6231127059787
7.05448717948718 3.62830831488696
7.38782051282051 2.2544817763188
7.74038461538461 2.39118793179761
8.10897435897436 3.94408836344962
8.49679487179487 4.37886301376789
8.90064102564103 3.30474333227141
9.32371794871795 2.87750456564839
9.76923076923077 2.90772957886653
10.2339743589744 3.23504462183126
10.7211538461538 2.42764601728468
11.2307692307692 3.89062620590816
11.7660256410256 1.4612341923841
12.3269230769231 2.83651088205716
12.9134615384615 6.66777155350589
13.5288461538462 5.68078959686743
14.1730769230769 5.2883423180007
14.849358974359 5.55809357435579
15.5544871794872 2.85986458189497
16.2948717948718 1.92998101584042
17.0705128205128 2.48395847685102
17.8846153846154 3.81158028668517
18.7371794871795 5.02432423626169
19.6282051282051 1.90028211398142
20.5641025641026 1.2167185488374
21.5416666666667 1.87801983261648
22.5673076923077 0.617857729740007
23.6442307692308 13.8274212729639
24.7692307692308 1.87040949960709
25.9487179487179 5.58214457428819
27.1826923076923 1.75646172351699
28.4775641025641 2.30198482718069
29.8333333333333 1.89362525260038
31.2564102564103 0.97281201591509
32.7435897435897 0.968933213642541
34.3044871794872 7.12887038308265
35.9358974358974 0.890299470029166
37.6474358974359 1.16894577531908
39.4391025641026 1.31867759295091
41.3173076923077 2.26980599232545
43.2852564102564 1.94677319803464
45.3461538461538 2.76299302385901
47.5064102564103 2.72572760497999
49.7692307692308 1.38686540291208
52.1378205128205 25.1786043991923
54.6217948717949 13.9916384602569
57.2211538461538 3.5500088311549
59.9455128205128 6.01724528827773
62.8012820512821 8.27741912353565
65.7916666666667 0.960553805580065
68.9230769230769 27.7448907588834
72.2051282051282 4.10815180516745
75.6442307692308 10.3827015264474
79.2467948717949 3.13950213856203
83.0192307692308 12.1883115768627
86.974358974359 0.879739784109395
91.1153846153846 0.90616700257359
95.4519230769231 6.35238722001628
100 6.19461717751649
};
\addplot [, color2, opacity=0.6, mark=triangle*, mark size=0.5, mark options={solid,rotate=180}, only marks, forget plot]
table {%
1 4.11246733989087
1.04487179487179 1.37608066272715
1.09615384615385 0.574445231055064
1.1474358974359 0.524725286521152
1.20192307692308 1.00612320895476
1.25961538461538 0.960353397029881
1.32051282051282 0.887949615060299
1.38461538461538 0.522538997457813
1.44871794871795 1.52221498575457
1.51923076923077 0.710546095711496
1.58974358974359 1.8952397782817
1.66666666666667 1.00348555304369
1.74679487179487 1.19676535020382
1.83012820512821 1.41551106968496
1.91666666666667 1.15678595112191
2.00641025641026 1.45710516428558
2.1025641025641 1.38336734113655
2.20512820512821 2.28541035035709
2.30769230769231 2.3470994791556
2.41987179487179 1.95651599076017
2.53525641025641 1.68261203915296
2.65384615384615 2.43119819322224
2.78205128205128 2.39899523913215
2.91346153846154 2.34728107855437
3.05128205128205 2.10520157570355
3.19871794871795 2.63479235318043
3.34935897435897 1.75759503541934
3.50961538461538 3.00963448919563
3.67628205128205 1.93904757630806
3.8525641025641 2.20769337528601
4.03525641025641 3.38253388898875
4.2275641025641 2.85506230313854
4.42948717948718 2.3238501872745
4.64102564102564 1.69673445752045
4.86217948717949 2.07566460996241
5.09294871794872 2.5821360769898
5.33653846153846 3.10699958136328
5.58974358974359 3.48032369302354
5.85576923076923 5.66768917166026
6.13461538461539 4.06919438741776
6.42628205128205 3.80318956249517
6.73397435897436 2.52600757973226
7.05448717948718 3.78371109043262
7.38782051282051 3.53647154659243
7.74038461538461 2.49563603187693
8.10897435897436 3.12128749067478
8.49679487179487 3.45022383962765
8.90064102564103 3.57385466336142
9.32371794871795 2.02607392461357
9.76923076923077 5.17140342705738
10.2339743589744 3.71649379032778
10.7211538461538 5.69035233852923
11.2307692307692 4.45850819545831
11.7660256410256 3.71021381220023
12.3269230769231 4.20927215164236
12.9134615384615 6.32769202148305
13.5288461538462 5.189879944409
14.1730769230769 10.3792173527951
14.849358974359 8.89862460193634
15.5544871794872 6.08720149362675
16.2948717948718 8.95445384213697
17.0705128205128 8.4987347342987
17.8846153846154 9.87200729628242
18.7371794871795 11.5597785713877
19.6282051282051 8.0278618705219
20.5641025641026 13.1039895067954
21.5416666666667 18.4843928573121
22.5673076923077 7.5591521167924
23.6442307692308 2.06691898414041
24.7692307692308 7.62990571744805
25.9487179487179 11.6872972927978
27.1826923076923 7.87969997678322
28.4775641025641 11.3980619713202
29.8333333333333 9.68351711866695
31.2564102564103 6.37338428509207
32.7435897435897 5.191419225532
34.3044871794872 8.45416786115819
35.9358974358974 9.37550213533927
37.6474358974359 5.89254408678864
39.4391025641026 3.56671366826976
41.3173076923077 1.92398831219579
43.2852564102564 11.7890682289695
45.3461538461538 4.89873815375987
47.5064102564103 3.11056800845247
49.7692307692308 4.25641507879181
52.1378205128205 9.77502444337106
54.6217948717949 10.1883876811023
57.2211538461538 2.08691639483671
59.9455128205128 8.27768805205355
62.8012820512821 2.30771681467226
65.7916666666667 11.1364499167429
68.9230769230769 20.6018083720938
72.2051282051282 2.9457408210621
75.6442307692308 1.03384550681274
79.2467948717949 1.83796231412653
83.0192307692308 14.0305751275991
86.974358974359 5.19602453913498
91.1153846153846 7.83144416931065
95.4519230769231 7.95097838108238
100 4.06247792458178
};
\end{axis}

\end{tikzpicture}

      \tikzexternaldisable
    \end{minipage}
  \end{subfigure}

  \begin{subfigure}[t]{\linewidth}
    \centering
    \caption{\cifarten \resnetthirtytwo}
    \begin{minipage}{0.50\linewidth}
      \centering
      % defines the pgfplots style "eigspacedefault"
\pgfkeys{/pgfplots/eigspacedefault/.style={
    width=1.03\linewidth,
    height=\goldenRatioInv*1.03*\linewidth,
    every axis plot/.append style={line width = 1pt},
    tick pos = left,
    ylabel near ticks,
    xlabel near ticks,
    xtick align = inside,
    ytick align = inside,
    legend cell align = left,
    legend columns = 1,
    legend pos = north east,
    legend style = {
      fill opacity = 0.9,
      text opacity = 1,
      font = \tiny,
      % column sep=0.1cm,
    },
    legend image post style={scale=2},
    xticklabel style = {font = \small},
    xlabel style = {font = \small},
    axis line style = {black},
    yticklabel style = {font = \small},
    ylabel style = {font = \small},
    title style = {font = \small},
    grid = major,
    grid style = {dashed}
  }
}

\pgfkeys{/pgfplots/eigspacedefaultapp/.style={
    eigspacedefault,
    height=0.6\linewidth,
    legend columns = 2,
  }
}

\pgfkeys{/pgfplots/eigspacenolegend/.style={
    legend image post style = {scale=0},
    legend style = {
      fill opacity = 0,
      draw opacity = 0,
      text opacity = 0,
      font = \small,
      at={(1, 1.025)},
      anchor=south east,
      column sep=0.25cm,
    },
  }
}
%%% Local Variables:
%%% mode: latex
%%% TeX-master: "../main"
%%% End:

      \pgfkeys{/pgfplots/zmystyle/.style={
          eigspacedefaultapp,
          eigspacenolegend,
        }}
      \tikzexternalenable
      \vspace{-6ex}
      % This file was created by tikzplotlib v0.9.7.
\begin{tikzpicture}

\definecolor{color0}{rgb}{0.274509803921569,0.6,0.564705882352941}
\definecolor{color1}{rgb}{0.870588235294118,0.623529411764706,0.0862745098039216}
\definecolor{color2}{rgb}{0.501960784313725,0.184313725490196,0.6}

\begin{axis}[
axis line style={white!10!black},
legend columns=2,
legend style={fill opacity=0.8, draw opacity=1, text opacity=1, at={(0.5,0.09)}, anchor=south, draw=white!80!black},
log basis x={10},
tick pos=left,
xlabel={epoch (log scale)},
xmajorgrids,
xmin=0.771323165184619, xmax=233.365219825747,
xmode=log,
ylabel={av. rel. error (log scale)},
ymajorgrids,
ymin=0.614028895697881, ymax=3646.32136007637,
ymode=log,
zmystyle
]
\addplot [, black, opacity=0.6, mark=*, mark size=0.5, mark options={solid}, only marks]
table {%
1 1
1.10897435897436 17.150220001374
1.23076923076923 16.7223589695998
1.36858974358974 25.7627246993579
1.51923076923077 19.6825680526765
1.68910256410256 48.3084829507448
1.875 41.890276443334
2.08333333333333 32.8331765906524
2.31410256410256 68.261641635909
2.57051282051282 59.6895932497821
2.8525641025641 125.507456776111
3.16987179487179 177.862627296658
3.51923076923077 145.850614282062
3.91025641025641 154.497189607174
4.34294871794872 175.085861313401
4.82371794871795 168.993056845832
5.35576923076923 201.119573742795
5.94871794871795 215.889212843291
6.60576923076923 243.579216797906
7.33653846153846 183.099438852269
8.15064102564103 247.577800951989
9.05128205128205 202.262461909509
10.0512820512821 178.336245300692
11.1634615384615 189.092041023203
12.400641025641 199.060094141537
13.7724358974359 185.18705640176
15.2948717948718 178.40474640856
16.9871794871795 234.264587770832
18.8653846153846 219.641661819419
20.9519230769231 212.832174298898
23.2692307692308 293.061964334182
25.8429487179487 192.661578337709
28.7019230769231 190.669727660856
31.8782051282051 174.003204327199
35.4038461538462 243.035982328891
39.3205128205128 184.267071164596
43.6698717948718 284.519771442313
48.5 221.532677797229
53.8653846153846 238.609339971943
59.8237179487179 234.114117478313
66.4391025641026 179.285206906596
73.7884615384615 275.348847026958
81.9519230769231 196.311568539055
91.0160256410256 336.293177056236
101.083333333333 298.434866091999
112.262820512821 101.320737980378
124.682692307692 393.349850640006
138.474358974359 245.22092157295
153.788461538462 422.441023807854
170.801282051282 422.964701923811
180 231.65915309193
};
\addlegendentry{mb 128, exact}
\addplot [, black, opacity=0.6, mark=*, mark size=0.5, mark options={solid}, only marks, forget plot]
table {%
1 21.166344680563
1.10897435897436 4.41277916875808
1.23076923076923 20.7119253613348
1.36858974358974 35.3646477680041
1.51923076923077 23.3351641532384
1.68910256410256 58.4394725690616
1.875 46.8701153385143
2.08333333333333 43.9106099996915
2.31410256410256 104.964796031138
2.57051282051282 84.1669775583704
2.8525641025641 118.352904414814
3.16987179487179 164.1799429271
3.51923076923077 161.791954047749
3.91025641025641 203.256536632031
4.34294871794872 204.192284231071
4.82371794871795 212.427487541598
5.35576923076923 245.44170361867
5.94871794871795 211.160262239465
6.60576923076923 217.035514573967
7.33653846153846 218.21575258679
8.15064102564103 192.768737482971
9.05128205128205 208.111194854373
10.0512820512821 169.385270619152
11.1634615384615 164.769948793564
12.400641025641 158.535725375954
13.7724358974359 167.381788297519
15.2948717948718 157.623495803522
16.9871794871795 158.153282835858
18.8653846153846 179.643068487034
20.9519230769231 175.883291881283
23.2692307692308 178.054806247239
25.8429487179487 177.626689072214
28.7019230769231 174.048045099948
31.8782051282051 212.676952673813
35.4038461538462 170.624138434692
39.3205128205128 196.750044019746
43.6698717948718 172.749093326799
48.5 178.111027978513
53.8653846153846 179.735868621253
59.8237179487179 204.692164367206
66.4391025641026 284.042896067093
73.7884615384615 201.474202339233
81.9519230769231 245.624321002881
91.0160256410256 159.358735798804
101.083333333333 251.989856015
112.262820512821 255.991228143572
124.682692307692 273.544819768869
138.474358974359 216.999559580195
153.788461538462 233.962179140072
170.801282051282 405.968467304719
180 175.231498590961
};
\addplot [, black, opacity=0.6, mark=*, mark size=0.5, mark options={solid}, only marks, forget plot]
table {%
1 20.2538187685851
1.10897435897436 3.93298435662741
1.23076923076923 19.8341840430638
1.36858974358974 28.4063885492169
1.51923076923077 22.8985364663486
1.68910256410256 61.7139901996479
1.875 43.17616203618
2.08333333333333 47.2893941744526
2.31410256410256 98.472638955907
2.57051282051282 66.4153721785886
2.8525641025641 103.465843008194
3.16987179487179 162.362017277045
3.51923076923077 166.990452316192
3.91025641025641 185.722915226703
4.34294871794872 191.415524140569
4.82371794871795 200.100756393565
5.35576923076923 223.80711406175
5.94871794871795 186.141555515877
6.60576923076923 210.284536986696
7.33653846153846 199.064075670203
8.15064102564103 188.10364531824
9.05128205128205 186.325221210505
10.0512820512821 171.497322240237
11.1634615384615 157.839937595322
12.400641025641 174.791794611128
13.7724358974359 173.049093129801
15.2948717948718 137.322188282372
16.9871794871795 163.945700581568
18.8653846153846 171.41973923685
20.9519230769231 152.16367556305
23.2692307692308 165.704493471243
25.8429487179487 170.786611813441
28.7019230769231 169.074965024682
31.8782051282051 149.811494318549
35.4038461538462 193.51559259773
39.3205128205128 154.051525819337
43.6698717948718 154.243352192448
48.5 235.732961031828
53.8653846153846 180.632922133881
59.8237179487179 256.767264567761
66.4391025641026 336.844636493721
73.7884615384615 176.132348845259
81.9519230769231 213.751689132818
91.0160256410256 176.786103950867
101.083333333333 324.739050626239
112.262820512821 293.451113886347
124.682692307692 360.799164711337
138.474358974359 224.330818226112
153.788461538462 414.656186951109
170.801282051282 334.315405466302
180 254.096234484591
};
\addplot [, color0, opacity=0.6, mark=diamond*, mark size=0.5, mark options={solid}, only marks]
table {%
1 nan
1.10897435897436 28.3113870878695
1.23076923076923 8.17387432851413
1.36858974358974 13.2008599841932
1.51923076923077 15.7339280860089
1.68910256410256 37.7720306206038
1.875 30.431911639513
2.08333333333333 27.2271294753048
2.31410256410256 64.2049199866509
2.57051282051282 50.2688183799386
2.8525641025641 140.362246399643
3.16987179487179 188.104416310181
3.51923076923077 194.880032801231
3.91025641025641 142.452828977527
4.34294871794872 169.727853468562
4.82371794871795 208.265499795998
5.35576923076923 261.502305492445
5.94871794871795 277.768800073352
6.60576923076923 282.362297968851
7.33653846153846 212.691718000693
8.15064102564103 302.817028033418
9.05128205128205 305.366978859563
10.0512820512821 207.044817895579
11.1634615384615 255.571441667969
12.400641025641 227.784085171634
13.7724358974359 325.92638724708
15.2948717948718 199.514325219319
16.9871794871795 295.966405498739
18.8653846153846 365.989423973164
20.9519230769231 503.280251783646
23.2692307692308 879.872650689014
25.8429487179487 358.027959501884
28.7019230769231 216.61046557144
31.8782051282051 297.262674878384
35.4038461538462 220.897479704915
39.3205128205128 137.729242099248
43.6698717948718 387.813775677461
48.5 159.627352628904
53.8653846153846 560.830862874233
59.8237179487179 92.4723201104655
66.4391025641026 230.050839443292
73.7884615384615 201.479748372557
81.9519230769231 517.127674827816
91.0160256410256 90.7489907909107
101.083333333333 346.644754693163
112.262820512821 17.5620096512104
124.682692307692 808.607420432428
138.474358974359 236.176714462894
153.788461538462 25.5440626214454
170.801282051282 16.2972606935794
180 74.8734588461042
};
\addlegendentry{sub 16, exact}
\addplot [, color0, opacity=0.6, mark=diamond*, mark size=0.5, mark options={solid}, only marks, forget plot]
table {%
1 40.939935624551
1.10897435897436 5.15903456965372
1.23076923076923 25.9497579017909
1.36858974358974 32.7492163393546
1.51923076923077 24.7291889230989
1.68910256410256 34.3443724445917
1.875 40.2333585518082
2.08333333333333 39.9334320436869
2.31410256410256 114.60386336591
2.57051282051282 80.0315375087127
2.8525641025641 131.250061574534
3.16987179487179 131.259231006347
3.51923076923077 227.310398830036
3.91025641025641 278.513369983405
4.34294871794872 250.621545079506
4.82371794871795 295.72133716468
5.35576923076923 323.373081604845
5.94871794871795 312.810051669774
6.60576923076923 411.643084490424
7.33653846153846 294.908446030208
8.15064102564103 312.202633872023
9.05128205128205 455.461328930271
10.0512820512821 315.102064550567
11.1634615384615 218.10383289825
12.400641025641 286.001430147732
13.7724358974359 319.234089652644
15.2948717948718 425.9350130169
16.9871794871795 295.893262941566
18.8653846153846 415.660342937989
20.9519230769231 282.539203573744
23.2692307692308 352.029579427222
25.8429487179487 555.475400130483
28.7019230769231 528.670921205476
31.8782051282051 416.07669971568
35.4038461538462 349.547459774529
39.3205128205128 488.034766391807
43.6698717948718 559.095417663899
48.5 533.183975207364
53.8653846153846 589.31684150977
59.8237179487179 405.331311650109
66.4391025641026 370.236301014281
73.7884615384615 488.783169234188
81.9519230769231 697.509496859645
91.0160256410256 94.6996694254602
101.083333333333 149.889318689589
112.262820512821 272.324100737228
124.682692307692 518.869637015926
138.474358974359 504.360945635095
153.788461538462 219.007090682555
170.801282051282 519.886630197086
180 316.338010474295
};
\addplot [, color0, opacity=0.6, mark=diamond*, mark size=0.5, mark options={solid}, only marks, forget plot]
table {%
1 34.1483250760657
1.10897435897436 8.90029963364873
1.23076923076923 8.58262967214618
1.36858974358974 13.072269897342
1.51923076923077 15.1051563369683
1.68910256410256 35.265260192419
1.875 27.6801469945651
2.08333333333333 46.2732241759307
2.31410256410256 70.8389314766712
2.57051282051282 63.1944985333947
2.8525641025641 70.9060127264288
3.16987179487179 85.3503166917835
3.51923076923077 101.363226278597
3.91025641025641 129.889577325351
4.34294871794872 119.540607214752
4.82371794871795 155.266276285346
5.35576923076923 121.243693083885
5.94871794871795 185.44465045913
6.60576923076923 203.288624078112
7.33653846153846 153.500442573679
8.15064102564103 149.688863285103
9.05128205128205 230.810733341946
10.0512820512821 152.513474276849
11.1634615384615 173.795083200536
12.400641025641 241.624444274344
13.7724358974359 270.40409313508
15.2948717948718 286.545544650093
16.9871794871795 266.835369505206
18.8653846153846 223.519048293358
20.9519230769231 316.732170120454
23.2692307692308 203.937108679262
25.8429487179487 277.139556464687
28.7019230769231 225.824608758702
31.8782051282051 167.603132316412
35.4038461538462 202.80830937401
39.3205128205128 310.344557200729
43.6698717948718 338.854932764992
48.5 292.696716538142
53.8653846153846 427.172935498221
59.8237179487179 630.092884548733
66.4391025641026 667.168627076347
73.7884615384615 297.982771847536
81.9519230769231 751.035257198351
91.0160256410256 615.978060635908
101.083333333333 448.450403943581
112.262820512821 212.309058793042
124.682692307692 283.657489143835
138.474358974359 414.250585912887
153.788461538462 736.082046519178
170.801282051282 382.910141382864
180 302.542556429892
};
\addplot [, color1, opacity=0.6, mark=square*, mark size=0.5, mark options={solid}, only marks]
table {%
1 nan
1.10897435897436 16.7974960137468
1.23076923076923 10.995021996438
1.36858974358974 22.2519657327283
1.51923076923077 21.6417066167902
1.68910256410256 54.2694187429895
1.875 33.5631297318141
2.08333333333333 26.9651056811978
2.31410256410256 72.3716090417034
2.57051282051282 74.8440952631218
2.8525641025641 137.534607417104
3.16987179487179 162.684657951167
3.51923076923077 166.552871677932
3.91025641025641 151.414644230417
4.34294871794872 168.90831552852
4.82371794871795 190.067855457598
5.35576923076923 189.518353347721
5.94871794871795 271.841947560036
6.60576923076923 226.589475463773
7.33653846153846 227.027472352884
8.15064102564103 285.346949478875
9.05128205128205 185.33538752607
10.0512820512821 174.231891883791
11.1634615384615 271.376884423274
12.400641025641 238.80168410193
13.7724358974359 195.172033008119
15.2948717948718 209.958845876931
16.9871794871795 345.392906938526
18.8653846153846 263.318934843226
20.9519230769231 284.222069568662
23.2692307692308 238.033050108884
25.8429487179487 192.48977614889
28.7019230769231 277.056133636018
31.8782051282051 281.584246519023
35.4038461538462 400.192117036121
39.3205128205128 162.325566955495
43.6698717948718 581.610929579306
48.5 266.590494865783
53.8653846153846 232.434389353399
59.8237179487179 315.939121413485
66.4391025641026 194.716789518246
73.7884615384615 158.305687397748
81.9519230769231 195.767691358734
91.0160256410256 304.80700001641
101.083333333333 137.906084888998
112.262820512821 8.22980071038465
124.682692307692 303.996515168892
138.474358974359 45.476386192874
153.788461538462 578.417000703924
170.801282051282 418.476055248872
180 83.1236804455117
};
\addlegendentry{mb 128, mc 1}
\addplot [, color1, opacity=0.6, mark=square*, mark size=0.5, mark options={solid}, only marks, forget plot]
table {%
1 38.8107351126997
1.10897435897436 21.7512368225602
1.23076923076923 26.3730436103315
1.36858974358974 23.1322419823573
1.51923076923077 18.6136476194768
1.68910256410256 54.2040553518903
1.875 33.9545616438678
2.08333333333333 61.9081495222702
2.31410256410256 114.825695350282
2.57051282051282 88.5861105662467
2.8525641025641 145.129491402348
3.16987179487179 174.243301160362
3.51923076923077 184.444887060257
3.91025641025641 222.149466005563
4.34294871794872 291.650238392598
4.82371794871795 339.252234251664
5.35576923076923 301.686070096179
5.94871794871795 253.52885880534
6.60576923076923 267.598747810244
7.33653846153846 271.518666319283
8.15064102564103 262.068018013279
9.05128205128205 284.085884624202
10.0512820512821 242.917534211506
11.1634615384615 238.12465955563
12.400641025641 164.02014359715
13.7724358974359 318.576891668792
15.2948717948718 258.420534909735
16.9871794871795 215.6318818452
18.8653846153846 253.12472832075
20.9519230769231 303.499009339495
23.2692307692308 419.297886100474
25.8429487179487 277.237841371692
28.7019230769231 380.848376574711
31.8782051282051 411.37883041581
35.4038461538462 325.069549092865
39.3205128205128 419.716155125788
43.6698717948718 352.463041838739
48.5 362.859607702842
53.8653846153846 563.57304121615
59.8237179487179 627.217642790258
66.4391025641026 552.550426019247
73.7884615384615 449.347265466328
81.9519230769231 412.035409888623
91.0160256410256 154.287711510768
101.083333333333 777.370431046932
112.262820512821 391.262127696875
124.682692307692 328.597145061212
138.474358974359 512.391415988259
153.788461538462 1180.90853099725
170.801282051282 903.245481902853
180 497.394849274264
};
\addplot [, color1, opacity=0.6, mark=square*, mark size=0.5, mark options={solid}, only marks, forget plot]
table {%
1 34.7249573374945
1.10897435897436 4.82662898286716
1.23076923076923 12.5506790434129
1.36858974358974 20.5345090245795
1.51923076923077 19.6163609849554
1.68910256410256 33.9972249268979
1.875 27.3535285026925
2.08333333333333 40.2573211799719
2.31410256410256 112.356663754933
2.57051282051282 71.5851472261087
2.8525641025641 102.55792326472
3.16987179487179 149.927262408877
3.51923076923077 169.477510247013
3.91025641025641 200.805043100304
4.34294871794872 269.699513485848
4.82371794871795 190.648755029314
5.35576923076923 239.811683834575
5.94871794871795 194.537564896875
6.60576923076923 284.902932592423
7.33653846153846 208.701293898874
8.15064102564103 196.632962410963
9.05128205128205 205.954274646477
10.0512820512821 171.550653247394
11.1634615384615 169.861315299569
12.400641025641 262.773371747516
13.7724358974359 186.390856131183
15.2948717948718 142.516515559854
16.9871794871795 190.198461976634
18.8653846153846 225.895752580949
20.9519230769231 230.237115323089
23.2692307692308 224.327272201503
25.8429487179487 217.929063375775
28.7019230769231 177.525571818054
31.8782051282051 96.9814892148009
35.4038461538462 251.588957856501
39.3205128205128 212.705109810183
43.6698717948718 310.77218403208
48.5 435.919146638138
53.8653846153846 256.748335150596
59.8237179487179 318.928191130752
66.4391025641026 298.000989976352
73.7884615384615 209.649290540847
81.9519230769231 321.060773066568
91.0160256410256 273.121780463004
101.083333333333 321.969243050491
112.262820512821 112.480400149812
124.682692307692 604.884764095096
138.474358974359 139.955050014554
153.788461538462 322.523170990464
170.801282051282 182.41029783257
180 86.8472801047237
};
\addplot [, color2, opacity=0.6, mark=triangle*, mark size=0.5, mark options={solid,rotate=180}, only marks]
table {%
1 nan
1.10897435897436 14.509630867842
1.23076923076923 7.00390853781071
1.36858974358974 8.26435729777169
1.51923076923077 23.3278020091839
1.68910256410256 57.1786692858051
1.875 42.0164850072221
2.08333333333333 37.4510670397686
2.31410256410256 67.1329061766058
2.57051282051282 59.149154210375
2.8525641025641 114.430772637507
3.16987179487179 290.670437757623
3.51923076923077 251.61675468757
3.91025641025641 156.173004048565
4.34294871794872 200.270763110445
4.82371794871795 282.092876621285
5.35576923076923 265.096611053036
5.94871794871795 211.702444963247
6.60576923076923 231.44746328473
7.33653846153846 205.204230834107
8.15064102564103 269.226628777626
9.05128205128205 302.311424235992
10.0512820512821 217.568891490709
11.1634615384615 352.674939818868
12.400641025641 172.282830751351
13.7724358974359 394.690941289423
15.2948717948718 134.105843750939
16.9871794871795 279.198801175196
18.8653846153846 411.275378563765
20.9519230769231 1068.95510329866
23.2692307692308 718.825266959305
25.8429487179487 394.606428975181
28.7019230769231 161.278690288153
31.8782051282051 233.725442643824
35.4038461538462 319.668631880008
39.3205128205128 11.9948809466611
43.6698717948718 341.999784266471
48.5 16.1172961561434
53.8653846153846 574.785057827103
59.8237179487179 275.525159100981
66.4391025641026 50.2834170612341
73.7884615384615 67.5797269585506
81.9519230769231 625.347424284307
91.0160256410256 5.74152992953197
101.083333333333 52.712903812959
112.262820512821 0.91142111960763
124.682692307692 254.019758997141
138.474358974359 16.3273092789789
153.788461538462 0.912741027879409
170.801282051282 0.92333181566932
180 2.7486076322756
};
\addlegendentry{sub 16, mc 1}
\addplot [, color2, opacity=0.6, mark=triangle*, mark size=0.5, mark options={solid,rotate=180}, only marks, forget plot]
table {%
1 94.3310254394453
1.10897435897436 5.19121991882767
1.23076923076923 19.3881457192843
1.36858974358974 13.3549438520993
1.51923076923077 14.776924401762
1.68910256410256 38.2652527513283
1.875 25.000638280497
2.08333333333333 53.2113775109653
2.31410256410256 120.699615548356
2.57051282051282 81.3825952711294
2.8525641025641 153.461877250424
3.16987179487179 117.186219526103
3.51923076923077 144.764246681957
3.91025641025641 252.636439926085
4.34294871794872 205.317609310855
4.82371794871795 190.046548612961
5.35576923076923 202.604697870699
5.94871794871795 480.242785094069
6.60576923076923 414.704259863067
7.33653846153846 297.456147068136
8.15064102564103 673.505742741057
9.05128205128205 520.177350033555
10.0512820512821 417.06575805311
11.1634615384615 85.2582383193905
12.400641025641 131.912275645648
13.7724358974359 1342.14957087367
15.2948717948718 321.11185780061
16.9871794871795 206.540909865943
18.8653846153846 694.770280342491
20.9519230769231 250.15077979247
23.2692307692308 1621.26700807539
25.8429487179487 900.534266771339
28.7019230769231 347.786492246631
31.8782051282051 477.213948581002
35.4038461538462 335.200068215931
39.3205128205128 593.775296716766
43.6698717948718 704.720686816479
48.5 500.51611817992
53.8653846153846 513.1457150951
59.8237179487179 211.465742358752
66.4391025641026 78.7675496209201
73.7884615384615 745.073015960328
81.9519230769231 999.946988004207
91.0160256410256 8.15163380315797
101.083333333333 10.4290020667065
112.262820512821 8.90968432465678
124.682692307692 911.396986307161
138.474358974359 473.396332050284
153.788461538462 24.244590211916
170.801282051282 1024.98844002098
180 44.8251929905591
};
\addplot [, color2, opacity=0.6, mark=triangle*, mark size=0.5, mark options={solid,rotate=180}, only marks, forget plot]
table {%
1 110.588135183992
1.10897435897436 5.75041706301623
1.23076923076923 9.63370349277899
1.36858974358974 13.4614760357091
1.51923076923077 17.2173313542314
1.68910256410256 36.1195224213527
1.875 32.3028246981363
2.08333333333333 60.5151585817493
2.31410256410256 92.8389564612643
2.57051282051282 113.621300738631
2.8525641025641 120.015816325708
3.16987179487179 117.259682488123
3.51923076923077 153.537309556105
3.91025641025641 140.392545066306
4.34294871794872 363.939898119633
4.82371794871795 170.907926291386
5.35576923076923 325.339372792875
5.94871794871795 296.351050668311
6.60576923076923 440.395641742208
7.33653846153846 255.525605498378
8.15064102564103 465.578440884324
9.05128205128205 352.23378140913
10.0512820512821 84.8581610452538
11.1634615384615 104.905532452811
12.400641025641 104.967584678751
13.7724358974359 267.526915326636
15.2948717948718 248.231011319321
16.9871794871795 511.712007458927
18.8653846153846 207.8746824641
20.9519230769231 695.120520196293
23.2692307692308 358.824575785223
25.8429487179487 382.992996567883
28.7019230769231 76.9391908456318
31.8782051282051 139.906260179923
35.4038461538462 147.347629027635
39.3205128205128 837.332313619578
43.6698717948718 256.602893923323
48.5 206.580681082603
53.8653846153846 918.145639704184
59.8237179487179 504.570026273599
66.4391025641026 397.362160306998
73.7884615384615 99.9240710928153
81.9519230769231 441.308929288458
91.0160256410256 236.963493855833
101.083333333333 840.015681163497
112.262820512821 25.6463535777296
124.682692307692 2456.54465309205
138.474358974359 377.790135282012
153.788461538462 415.513665087704
170.801282051282 30.2760132713183
180 59.4039087714222
};
\end{axis}

\end{tikzpicture}

      \tikzexternaldisable
    \end{minipage}\hfill
    \begin{minipage}{0.50\linewidth}
      \centering
      % defines the pgfplots style "eigspacedefault"
\pgfkeys{/pgfplots/eigspacedefault/.style={
    width=1.03\linewidth,
    height=\goldenRatioInv*1.03*\linewidth,
    every axis plot/.append style={line width = 1pt},
    tick pos = left,
    ylabel near ticks,
    xlabel near ticks,
    xtick align = inside,
    ytick align = inside,
    legend cell align = left,
    legend columns = 1,
    legend pos = north east,
    legend style = {
      fill opacity = 0.9,
      text opacity = 1,
      font = \tiny,
      % column sep=0.1cm,
    },
    legend image post style={scale=2},
    xticklabel style = {font = \small},
    xlabel style = {font = \small},
    axis line style = {black},
    yticklabel style = {font = \small},
    ylabel style = {font = \small},
    title style = {font = \small},
    grid = major,
    grid style = {dashed}
  }
}

\pgfkeys{/pgfplots/eigspacedefaultapp/.style={
    eigspacedefault,
    height=0.6\linewidth,
    legend columns = 2,
  }
}

\pgfkeys{/pgfplots/eigspacenolegend/.style={
    legend image post style = {scale=0},
    legend style = {
      fill opacity = 0,
      draw opacity = 0,
      text opacity = 0,
      font = \small,
      at={(1, 1.025)},
      anchor=south east,
      column sep=0.25cm,
    },
  }
}
%%% Local Variables:
%%% mode: latex
%%% TeX-master: "../main"
%%% End:

      \pgfkeys{/pgfplots/zmystyle/.style={
          eigspacedefaultapp,
          eigspacenolegend,
        }}
      \tikzexternalenable
      \vspace{-6ex}
      % This file was created by tikzplotlib v0.9.7.
\begin{tikzpicture}

\definecolor{color0}{rgb}{0.274509803921569,0.6,0.564705882352941}
\definecolor{color1}{rgb}{0.870588235294118,0.623529411764706,0.0862745098039216}
\definecolor{color2}{rgb}{0.501960784313725,0.184313725490196,0.6}

\begin{axis}[
axis line style={white!10!black},
legend columns=2,
legend style={fill opacity=0.8, draw opacity=1, text opacity=1, at={(0.5,0.09)}, anchor=south, draw=white!80!black},
log basis x={10},
tick pos=left,
xlabel={epoch (log scale)},
xmajorgrids,
xmin=0.771323165184619, xmax=233.365219825747,
xmode=log,
ylabel={av. rel. error (log scale)},
ymajorgrids,
ymin=0.663898557426992, ymax=1515.70204778941,
ymode=log,
zmystyle
]
\addplot [, black, opacity=0.6, mark=*, mark size=0.5, mark options={solid}, only marks]
table {%
1 1
1.10897435897436 13.4029228024081
1.23076923076923 11.6505306467795
1.36858974358974 5.67814344319601
1.51923076923077 9.90883231813371
1.68910256410256 10.9629038960984
1.875 18.5806784488648
2.08333333333333 13.5583916294779
2.31410256410256 8.15220998269563
2.57051282051282 12.3421729870286
2.8525641025641 25.6985520060942
3.16987179487179 18.3198232273842
3.51923076923077 20.0075503910402
3.91025641025641 9.6707905051115
4.34294871794872 13.5056267714956
4.82371794871795 15.3037456071284
5.35576923076923 10.4164655937532
5.94871794871795 16.9427983863259
6.60576923076923 10.5669949423151
7.33653846153846 8.43281693316208
8.15064102564103 14.9717359978528
9.05128205128205 8.45364028842063
10.0512820512821 11.4247624313202
11.1634615384615 13.6412992833401
12.400641025641 13.1179701512758
13.7724358974359 15.0147011770222
15.2948717948718 11.2727112928529
16.9871794871795 16.7446061893995
18.8653846153846 19.810643512251
20.9519230769231 20.0310307256073
23.2692307692308 16.6598582253852
25.8429487179487 15.4825343009851
28.7019230769231 43.6635667777565
31.8782051282051 27.4161412015356
35.4038461538462 28.5073044348771
39.3205128205128 42.146297911087
43.6698717948718 42.5765406866749
48.5 29.0142670045382
53.8653846153846 46.6474968825005
59.8237179487179 31.728754999319
66.4391025641026 57.5970749259841
73.7884615384615 46.4428986831361
81.9519230769231 64.8565284109462
91.0160256410256 80.5447875247586
101.083333333333 67.2923547087785
112.262820512821 115.102407620773
124.682692307692 100.946396590215
138.474358974359 109.268595172768
153.788461538462 36.9048315089024
170.801282051282 68.7451319464772
180 100.997039811919
};
\addlegendentry{mb 128, exact}
\addplot [, black, opacity=0.6, mark=*, mark size=0.5, mark options={solid}, only marks, forget plot]
table {%
1 19.86499103271
1.10897435897436 18.5326656355209
1.23076923076923 11.735502961732
1.36858974358974 8.22725876352447
1.51923076923077 9.38361943525938
1.68910256410256 11.3301712225476
1.875 9.08746494637451
2.08333333333333 13.9081991076122
2.31410256410256 14.9945620707811
2.57051282051282 15.8693111409085
2.8525641025641 15.0951606058069
3.16987179487179 13.4059779309144
3.51923076923077 16.5731367016869
3.91025641025641 13.6763532234302
4.34294871794872 9.09189328768214
4.82371794871795 10.266838919134
5.35576923076923 16.6390133019449
5.94871794871795 12.542014305081
6.60576923076923 11.8353557879811
7.33653846153846 18.4374439258081
8.15064102564103 11.1333362013525
9.05128205128205 8.86139551663086
10.0512820512821 8.27947377714771
11.1634615384615 11.2974757541125
12.400641025641 10.9077892361597
13.7724358974359 8.96570603280396
15.2948717948718 10.6716813318361
16.9871794871795 10.5681663120209
18.8653846153846 10.5747579182262
20.9519230769231 12.0906508352301
23.2692307692308 15.3630477897618
25.8429487179487 24.8427927143516
28.7019230769231 17.9611724404273
31.8782051282051 24.5835130639193
35.4038461538462 21.9868782678903
39.3205128205128 34.5116856430753
43.6698717948718 28.1201158365506
48.5 27.8836895836264
53.8653846153846 36.9268232023906
59.8237179487179 24.8834390593787
66.4391025641026 33.6753083283222
73.7884615384615 37.3326214594912
81.9519230769231 44.3867982152446
91.0160256410256 49.1003764156856
101.083333333333 58.9246510731903
112.262820512821 88.8197677616312
124.682692307692 55.4251083357803
138.474358974359 47.8046174043572
153.788461538462 62.8240705624946
170.801282051282 79.958134605329
180 39.0252295665849
};
\addplot [, black, opacity=0.6, mark=*, mark size=0.5, mark options={solid}, only marks, forget plot]
table {%
1 19.9981039636166
1.10897435897436 15.4314994912078
1.23076923076923 9.99021301144797
1.36858974358974 7.70431104366731
1.51923076923077 8.75949367445939
1.68910256410256 11.6002919669248
1.875 9.21223070777104
2.08333333333333 11.6170050776433
2.31410256410256 9.80549545168902
2.57051282051282 7.99674260518551
2.8525641025641 14.0750313857935
3.16987179487179 10.1073452364003
3.51923076923077 8.08021606709185
3.91025641025641 11.5879208767914
4.34294871794872 7.14244684720074
4.82371794871795 9.71132617881587
5.35576923076923 12.3492665388249
5.94871794871795 6.67891841925489
6.60576923076923 7.50753767238263
7.33653846153846 12.8971604517952
8.15064102564103 7.5116898719837
9.05128205128205 8.47545570703558
10.0512820512821 7.78055551555252
11.1634615384615 11.6208770485455
12.400641025641 10.6246481450811
13.7724358974359 8.47662341586302
15.2948717948718 12.126014140465
16.9871794871795 9.64924936511512
18.8653846153846 13.9392715419
20.9519230769231 13.702545322838
23.2692307692308 10.9493998174239
25.8429487179487 16.1918931832984
28.7019230769231 14.2131925131621
31.8782051282051 14.5379807815722
35.4038461538462 21.8722856030711
39.3205128205128 32.7621273922342
43.6698717948718 21.7363194745009
48.5 31.7476657637218
53.8653846153846 24.5188851920189
59.8237179487179 39.0320798142566
66.4391025641026 44.0169003710493
73.7884615384615 55.9498759880168
81.9519230769231 67.1494692331575
91.0160256410256 39.1342711126205
101.083333333333 71.7439960995001
112.262820512821 69.5727443528473
124.682692307692 84.9455183897304
138.474358974359 66.747333983758
153.788461538462 43.29984537275
170.801282051282 117.515857219758
180 77.0666410726577
};
\addplot [, color0, opacity=0.6, mark=diamond*, mark size=0.5, mark options={solid}, only marks]
table {%
1 nan
1.10897435897436 13.1658521861087
1.23076923076923 13.087091000234
1.36858974358974 8.63359359192524
1.51923076923077 11.1398920832036
1.68910256410256 11.1893987169421
1.875 13.8657406752943
2.08333333333333 12.6349540840464
2.31410256410256 12.7434854510095
2.57051282051282 15.5145599859813
2.8525641025641 38.9799085554433
3.16987179487179 27.81967919429
3.51923076923077 34.9894559170485
3.91025641025641 15.3713778812207
4.34294871794872 31.651119631827
4.82371794871795 28.9732623245593
5.35576923076923 19.5496295373561
5.94871794871795 32.6954612445869
6.60576923076923 30.3722348000714
7.33653846153846 24.5286614698506
8.15064102564103 28.4868545195667
9.05128205128205 30.8135883859246
10.0512820512821 23.22430438604
11.1634615384615 31.4096836844123
12.400641025641 43.8698936497675
13.7724358974359 37.0193417297858
15.2948717948718 28.2654900527456
16.9871794871795 29.1079517981268
18.8653846153846 43.0921142271621
20.9519230769231 50.8971197350581
23.2692307692308 39.4562604484911
25.8429487179487 55.6118732515501
28.7019230769231 67.9694233675997
31.8782051282051 73.2294104519876
35.4038461538462 54.8274892242764
39.3205128205128 123.847521934268
43.6698717948718 102.937160164325
48.5 106.180250519159
53.8653846153846 173.786013617411
59.8237179487179 97.471567977325
66.4391025641026 161.946262314413
73.7884615384615 89.6153277173058
81.9519230769231 35.7065230120337
91.0160256410256 308.451723913848
101.083333333333 161.165764760265
112.262820512821 26.03180937365
124.682692307692 177.832961196231
138.474358974359 350.359849554181
153.788461538462 34.3565454645961
170.801282051282 49.0445947840579
180 8.1725981204529
};
\addlegendentry{sub 16, exact}
\addplot [, color0, opacity=0.6, mark=diamond*, mark size=0.5, mark options={solid}, only marks, forget plot]
table {%
1 32.7764156986643
1.10897435897436 26.0276499939331
1.23076923076923 9.03413614103307
1.36858974358974 7.2640959755415
1.51923076923077 6.3134369337628
1.68910256410256 7.19163243865837
1.875 13.3906452571422
2.08333333333333 12.9579767493733
2.31410256410256 14.236488495831
2.57051282051282 13.8393658587092
2.8525641025641 11.714152505041
3.16987179487179 11.3210909309755
3.51923076923077 9.97238164520075
3.91025641025641 8.46541881302951
4.34294871794872 11.5673626371033
4.82371794871795 8.85030361116548
5.35576923076923 20.6685802523271
5.94871794871795 12.8907051267086
6.60576923076923 14.4652105065988
7.33653846153846 13.1997940385089
8.15064102564103 16.5205465074379
9.05128205128205 9.67590754367607
10.0512820512821 9.63350355825852
11.1634615384615 11.4901890541942
12.400641025641 21.2102531848434
13.7724358974359 23.0152415121686
15.2948717948718 11.835348277554
16.9871794871795 20.1525437535889
18.8653846153846 31.2783414170039
20.9519230769231 34.1521403547728
23.2692307692308 38.0637380811534
25.8429487179487 51.8927334034986
28.7019230769231 52.8892556609922
31.8782051282051 51.7335578896091
35.4038461538462 72.2941669579502
39.3205128205128 114.491909936912
43.6698717948718 91.9462418413709
48.5 80.2112411382082
53.8653846153846 141.530340065511
59.8237179487179 126.033618623605
66.4391025641026 137.09756920287
73.7884615384615 128.410543736484
81.9519230769231 157.553804287017
91.0160256410256 111.575067603848
101.083333333333 98.6063496033785
112.262820512821 121.990594660626
124.682692307692 144.384714257566
138.474358974359 111.867050361198
153.788461538462 39.8006135655481
170.801282051282 271.530058540623
180 16.1271999499737
};
\addplot [, color0, opacity=0.6, mark=diamond*, mark size=0.5, mark options={solid}, only marks, forget plot]
table {%
1 36.3949614658494
1.10897435897436 20.5585793741612
1.23076923076923 11.0985759762551
1.36858974358974 6.68564400669866
1.51923076923077 6.41198996800699
1.68910256410256 10.3159771385883
1.875 10.0264347017088
2.08333333333333 10.7596038492423
2.31410256410256 9.590871010358
2.57051282051282 7.91382482417978
2.8525641025641 16.0019457761242
3.16987179487179 6.09611802943382
3.51923076923077 8.02245219553924
3.91025641025641 10.346859230378
4.34294871794872 9.58412451087998
4.82371794871795 15.0345540061753
5.35576923076923 12.2565147442057
5.94871794871795 8.78201155233413
6.60576923076923 18.5640450776868
7.33653846153846 10.3976915728926
8.15064102564103 9.94612976281985
9.05128205128205 8.49490655119406
10.0512820512821 11.4045175744299
11.1634615384615 14.4163152207401
12.400641025641 14.6800878242971
13.7724358974359 21.4929528390154
15.2948717948718 46.9761257970717
16.9871794871795 11.4855572252529
18.8653846153846 59.4228244963428
20.9519230769231 29.1600399229158
23.2692307692308 40.9315479220343
25.8429487179487 36.3280932734291
28.7019230769231 26.1687036563463
31.8782051282051 50.6419502258898
35.4038461538462 75.803162193966
39.3205128205128 96.6199290755062
43.6698717948718 90.6637467306801
48.5 90.4202559488873
53.8653846153846 122.277264737333
59.8237179487179 152.81801515637
66.4391025641026 124.931577216037
73.7884615384615 238.946434652871
81.9519230769231 187.104956904144
91.0160256410256 123.074911464132
101.083333333333 67.9306714131621
112.262820512821 235.623010297897
124.682692307692 201.746003079119
138.474358974359 243.630255953512
153.788461538462 169.084009874727
170.801282051282 191.378330953086
180 150.93500844065
};
\addplot [, color1, opacity=0.6, mark=square*, mark size=0.5, mark options={solid}, only marks]
table {%
1 nan
1.10897435897436 12.8757060970452
1.23076923076923 13.7462998766728
1.36858974358974 7.40809907341916
1.51923076923077 11.6914288834173
1.68910256410256 8.50779803623384
1.875 14.3991850052886
2.08333333333333 13.0091638087007
2.31410256410256 9.99743429180458
2.57051282051282 11.7159625152475
2.8525641025641 30.6204728206522
3.16987179487179 17.2948457423068
3.51923076923077 26.6021518924702
3.91025641025641 17.8644799853403
4.34294871794872 18.7251447224672
4.82371794871795 14.9910156994492
5.35576923076923 18.4010474156059
5.94871794871795 32.3458420105649
6.60576923076923 18.5580730748978
7.33653846153846 12.4511886599545
8.15064102564103 11.6159437222588
9.05128205128205 15.0839623876677
10.0512820512821 26.3224711520041
11.1634615384615 31.2873700619086
12.400641025641 20.8855325353804
13.7724358974359 32.6395667880102
15.2948717948718 18.4147911450713
16.9871794871795 14.4469003983811
18.8653846153846 27.0288888307314
20.9519230769231 35.5884219876575
23.2692307692308 62.5057910581535
25.8429487179487 27.3013833855236
28.7019230769231 47.8783502437228
31.8782051282051 83.4588837345243
35.4038461538462 48.7895835448111
39.3205128205128 77.5417905019672
43.6698717948718 78.9014561538223
48.5 48.3494598320317
53.8653846153846 97.3333408883128
59.8237179487179 61.228852857324
66.4391025641026 87.5917459864431
73.7884615384615 82.6157889935443
81.9519230769231 142.566531341468
91.0160256410256 158.631790659673
101.083333333333 45.2228353624914
112.262820512821 183.576443798606
124.682692307692 199.959307805689
138.474358974359 78.8942910065245
153.788461538462 6.51417824272448
170.801282051282 83.4587779216811
180 74.9672796408132
};
\addlegendentry{mb 128, mc 1}
\addplot [, color1, opacity=0.6, mark=square*, mark size=0.5, mark options={solid}, only marks, forget plot]
table {%
1 33.453997692702
1.10897435897436 17.4433673638845
1.23076923076923 13.5603673749017
1.36858974358974 10.1177582351648
1.51923076923077 8.47880054037151
1.68910256410256 11.2723316228117
1.875 13.7905791688505
2.08333333333333 12.0414803007926
2.31410256410256 12.6180514328979
2.57051282051282 11.5673818712525
2.8525641025641 17.1408361923198
3.16987179487179 10.2065541371706
3.51923076923077 13.296455997899
3.91025641025641 23.9397502908551
4.34294871794872 8.86929941799636
4.82371794871795 6.73772481446632
5.35576923076923 12.8969182797932
5.94871794871795 8.69040369550247
6.60576923076923 11.6200505686102
7.33653846153846 25.9973536883294
8.15064102564103 35.5432400609081
9.05128205128205 11.7906461179761
10.0512820512821 12.5032072144932
11.1634615384615 16.4534421985478
12.400641025641 24.5660591844965
13.7724358974359 10.2646303533531
15.2948717948718 20.3299953268064
16.9871794871795 32.4520650099933
18.8653846153846 29.2939299295488
20.9519230769231 17.6677449216022
23.2692307692308 43.1367083281562
25.8429487179487 41.3886786713465
28.7019230769231 69.1034666458257
31.8782051282051 62.8491855898031
35.4038461538462 32.8147641434892
39.3205128205128 68.5769830034641
43.6698717948718 60.1086104000492
48.5 52.8225449626496
53.8653846153846 95.1671767138772
59.8237179487179 70.3495296988572
66.4391025641026 66.5189055384019
73.7884615384615 81.9370486199284
81.9519230769231 214.659394013973
91.0160256410256 69.790059995994
101.083333333333 78.3418600303252
112.262820512821 94.2285954169805
124.682692307692 131.972746490937
138.474358974359 55.9313235323867
153.788461538462 78.5519793877264
170.801282051282 238.740939561578
180 119.073334333062
};
\addplot [, color1, opacity=0.6, mark=square*, mark size=0.5, mark options={solid}, only marks, forget plot]
table {%
1 39.8523432071853
1.10897435897436 17.6134034998463
1.23076923076923 9.66188833698565
1.36858974358974 7.69870480939991
1.51923076923077 9.8424763456349
1.68910256410256 11.4600062153751
1.875 13.7849066891362
2.08333333333333 11.6597836259487
2.31410256410256 12.3466614793697
2.57051282051282 8.34381619910121
2.8525641025641 14.3038675459936
3.16987179487179 10.9918277980781
3.51923076923077 6.20969999217315
3.91025641025641 7.58505895099865
4.34294871794872 6.70968917780022
4.82371794871795 7.20419046868194
5.35576923076923 8.83673754672371
5.94871794871795 9.11160331981979
6.60576923076923 9.81093819496613
7.33653846153846 6.90725271796515
8.15064102564103 12.3037862294081
9.05128205128205 6.41802739350397
10.0512820512821 6.19323227542391
11.1634615384615 11.2526597264086
12.400641025641 13.9221919330718
13.7724358974359 19.4265735481779
15.2948717948718 21.0345226883901
16.9871794871795 26.9240634229906
18.8653846153846 23.8795771379283
20.9519230769231 24.1323079906362
23.2692307692308 12.9886050356393
25.8429487179487 21.1927327557149
28.7019230769231 32.3656859010927
31.8782051282051 23.4225093225
35.4038461538462 31.5347599316925
39.3205128205128 76.7469042179671
43.6698717948718 46.7754937717643
48.5 74.4026085745087
53.8653846153846 63.4009119957239
59.8237179487179 97.1156262044878
66.4391025641026 47.6414535011298
73.7884615384615 150.937726431944
81.9519230769231 152.220884828897
91.0160256410256 46.9231408990348
101.083333333333 118.670157704281
112.262820512821 59.9297903630679
124.682692307692 170.239731728895
138.474358974359 122.666346711348
153.788461538462 66.3711171011729
170.801282051282 86.2420723093309
180 85.8434504753698
};
\addplot [, color2, opacity=0.6, mark=triangle*, mark size=0.5, mark options={solid,rotate=180}, only marks]
table {%
1 nan
1.10897435897436 15.7322749735992
1.23076923076923 54.8797248662365
1.36858974358974 16.4891488275739
1.51923076923077 33.1141874325902
1.68910256410256 21.8130866361589
1.875 19.3672027502615
2.08333333333333 17.1315130963177
2.31410256410256 31.3794803428643
2.57051282051282 31.655698048641
2.8525641025641 72.5614715048705
3.16987179487179 35.4735358093557
3.51923076923077 89.3421337700102
3.91025641025641 19.4148067058509
4.34294871794872 51.4831947270156
4.82371794871795 31.8685129632322
5.35576923076923 47.3431081402529
5.94871794871795 82.4126022934301
6.60576923076923 52.0373493301814
7.33653846153846 39.4307209549518
8.15064102564103 18.2487411172107
9.05128205128205 110.795515940951
10.0512820512821 48.1966452854129
11.1634615384615 92.1636901661917
12.400641025641 102.28312844542
13.7724358974359 79.9519635734032
15.2948717948718 23.8851000649477
16.9871794871795 23.3990486490441
18.8653846153846 66.1230075994896
20.9519230769231 102.502784240367
23.2692307692308 106.580372367875
25.8429487179487 70.4424264609114
28.7019230769231 65.3803553024792
31.8782051282051 78.9684795871842
35.4038461538462 52.4159622511009
39.3205128205128 144.984262040267
43.6698717948718 175.522935749655
48.5 102.327603582073
53.8653846153846 365.00018240549
59.8237179487179 149.265513454324
66.4391025641026 132.279474764523
73.7884615384615 262.936246467298
81.9519230769231 102.571331993929
91.0160256410256 401.275903340723
101.083333333333 113.172951472779
112.262820512821 1.46048650514464
124.682692307692 375.033448725279
138.474358974359 161.438029861641
153.788461538462 1.06244795032409
170.801282051282 188.234604032343
180 0.943542281817456
};
\addlegendentry{sub 16, mc 1}
\addplot [, color2, opacity=0.6, mark=triangle*, mark size=0.5, mark options={solid,rotate=180}, only marks, forget plot]
table {%
1 84.7291782200611
1.10897435897436 27.1406842811932
1.23076923076923 11.3543322179983
1.36858974358974 7.57528371447151
1.51923076923077 8.71729157527103
1.68910256410256 7.33791258919529
1.875 18.6863894158968
2.08333333333333 12.6384312059286
2.31410256410256 10.8897032571585
2.57051282051282 13.5413302670285
2.8525641025641 12.0688611468808
3.16987179487179 21.1035906947686
3.51923076923077 11.278851248805
3.91025641025641 8.91307115965171
4.34294871794872 11.2365512663809
4.82371794871795 11.6870700562862
5.35576923076923 22.2028120905859
5.94871794871795 21.727467692031
6.60576923076923 26.5264589366411
7.33653846153846 15.9493050066059
8.15064102564103 19.7244157370604
9.05128205128205 26.0945171962731
10.0512820512821 42.1503178848032
11.1634615384615 13.2060083257226
12.400641025641 17.992341605113
13.7724358974359 22.2500077993867
15.2948717948718 20.8729373924132
16.9871794871795 79.3028122949008
18.8653846153846 100.877952255594
20.9519230769231 53.338614009232
23.2692307692308 45.7160050630074
25.8429487179487 71.1583432274222
28.7019230769231 286.507038321416
31.8782051282051 122.722915749062
35.4038461538462 105.098587071279
39.3205128205128 212.3395634198
43.6698717948718 186.648455138151
48.5 155.518686736492
53.8653846153846 322.425093513284
59.8237179487179 237.491248082365
66.4391025641026 317.955076381743
73.7884615384615 433.949752466793
81.9519230769231 1066.48363555923
91.0160256410256 172.269416906428
101.083333333333 353.450205752881
112.262820512821 17.3139424427689
124.682692307692 268.519162252131
138.474358974359 167.462477778862
153.788461538462 3.31259996930363
170.801282051282 865.256514934318
180 0.948485494810853
};
\addplot [, color2, opacity=0.6, mark=triangle*, mark size=0.5, mark options={solid,rotate=180}, only marks, forget plot]
table {%
1 97.9714497016913
1.10897435897436 37.2966028875917
1.23076923076923 13.0080350460434
1.36858974358974 9.64480736840153
1.51923076923077 30.3862047332136
1.68910256410256 16.6826585499975
1.875 37.0825328119306
2.08333333333333 32.927182708655
2.31410256410256 30.646187280144
2.57051282051282 27.0886476275727
2.8525641025641 48.3614689485454
3.16987179487179 14.3836656419033
3.51923076923077 10.9755435638561
3.91025641025641 13.615359166138
4.34294871794872 17.8613829154709
4.82371794871795 15.0550382318972
5.35576923076923 18.3737060472745
5.94871794871795 27.9905759236318
6.60576923076923 37.2650116563854
7.33653846153846 20.6186170471706
8.15064102564103 38.1503651711202
9.05128205128205 18.1236901749394
10.0512820512821 17.5371584838202
11.1634615384615 25.7706521345145
12.400641025641 56.6477915484654
13.7724358974359 53.4915305299283
15.2948717948718 45.2314986978077
16.9871794871795 10.2613754159644
18.8653846153846 72.9340184799
20.9519230769231 28.9115834430054
23.2692307692308 39.4984618956181
25.8429487179487 37.1672609732933
28.7019230769231 59.6072035134815
31.8782051282051 95.108008481996
35.4038461538462 59.2690510993115
39.3205128205128 226.382514457684
43.6698717948718 220.534437290129
48.5 171.419055064845
53.8653846153846 141.332486057454
59.8237179487179 209.297083228769
66.4391025641026 155.131675164681
73.7884615384615 520.992343045785
81.9519230769231 237.687578351909
91.0160256410256 175.83495174538
101.083333333333 158.396925666211
112.262820512821 333.498811589813
124.682692307692 440.328902560564
138.474358974359 318.86013258194
153.788461538462 389.623087361892
170.801282051282 228.872683133066
180 18.9228661046844
};
\end{axis}

\end{tikzpicture}

      \tikzexternaldisable
    \end{minipage}
  \end{subfigure}

  \begin{subfigure}[t]{\linewidth}
    \centering
    \caption{\cifarhun \allcnnc}
    \begin{minipage}{0.50\linewidth}
      \centering
      % defines the pgfplots style "eigspacedefault"
\pgfkeys{/pgfplots/eigspacedefault/.style={
    width=1.03\linewidth,
    height=\goldenRatioInv*1.03*\linewidth,
    every axis plot/.append style={line width = 1pt},
    tick pos = left,
    ylabel near ticks,
    xlabel near ticks,
    xtick align = inside,
    ytick align = inside,
    legend cell align = left,
    legend columns = 1,
    legend pos = north east,
    legend style = {
      fill opacity = 0.9,
      text opacity = 1,
      font = \tiny,
      % column sep=0.1cm,
    },
    legend image post style={scale=2},
    xticklabel style = {font = \small},
    xlabel style = {font = \small},
    axis line style = {black},
    yticklabel style = {font = \small},
    ylabel style = {font = \small},
    title style = {font = \small},
    grid = major,
    grid style = {dashed}
  }
}

\pgfkeys{/pgfplots/eigspacedefaultapp/.style={
    eigspacedefault,
    height=0.6\linewidth,
    legend columns = 2,
  }
}

\pgfkeys{/pgfplots/eigspacenolegend/.style={
    legend image post style = {scale=0},
    legend style = {
      fill opacity = 0,
      draw opacity = 0,
      text opacity = 0,
      font = \small,
      at={(1, 1.025)},
      anchor=south east,
      column sep=0.25cm,
    },
  }
}
%%% Local Variables:
%%% mode: latex
%%% TeX-master: "../main"
%%% End:

      \pgfkeys{/pgfplots/zmystyle/.style={
          eigspacedefaultapp,
          eigspacenolegend,
        }}
      \tikzexternalenable
      \vspace{-6ex}
      % This file was created by tikzplotlib v0.9.7.
\begin{tikzpicture}

\definecolor{color0}{rgb}{0.274509803921569,0.6,0.564705882352941}
\definecolor{color1}{rgb}{0.870588235294118,0.623529411764706,0.0862745098039216}
\definecolor{color2}{rgb}{0.501960784313725,0.184313725490196,0.6}

\begin{axis}[
axis line style={white!10!black},
legend style={fill opacity=0.8, draw opacity=1, text opacity=1, at={(0.03,0.97)}, anchor=north west, draw=white!80!black},
log basis x={10},
tick pos=left,
xlabel={epoch (log scale)},
xmajorgrids,
xmin=0.746099240306814, xmax=469.106495613199,
xmode=log,
ylabel={av. rel. error (log scale)},
ymajorgrids,
ymin=0.0622575701156013, ymax=21.8692423655614,
ymode=log,
zmystyle
]
\addplot [, black, opacity=0.6, mark=*, mark size=0.5, mark options={solid}, only marks]
table {%
1 0.737452839443649
1.26282051282051 0.74479327170042
1.6025641025641 0.32846551230474
2.03205128205128 0.239586357173377
2.57692307692308 0.258605723682562
3.26282051282051 0.165422047230875
4.13461538461539 0.0812649264963345
5.23717948717949 0.107677469438104
6.64102564102564 0.139921217434693
8.41025641025641 0.154642781806713
10.6602564102564 0.243001066748309
13.5064102564103 0.329373987151286
17.1153846153846 0.609253270923549
21.6858974358974 0.88463333583301
27.4807692307692 1.12251118345973
34.8205128205128 1.41054763346452
44.1153846153846 2.18795592497761
55.8974358974359 3.25935324656937
70.8269230769231 3.7278454115808
89.7435897435897 4.22356088542468
113.711538461538 6.0991969014331
144.076923076923 6.54556792905002
182.551282051282 6.68554380618034
231.301282051282 6.64629799053202
293.070512820513 7.31121511094237
350 6.76491846412083
};
\addlegendentry{mb 128, exact}
\addplot [, color0, opacity=0.6, mark=diamond*, mark size=0.5, mark options={solid}, only marks]
table {%
1 0.688618468613691
1.26282051282051 0.734878529905612
1.6025641025641 0.43308403502225
2.03205128205128 0.262275625596565
2.57692307692308 0.249791267991574
3.26282051282051 0.13796408116023
4.13461538461539 0.129198620480029
5.23717948717949 0.232314965983945
6.64102564102564 0.249589250318994
8.41025641025641 0.340316241198062
10.6602564102564 0.541996645268246
13.5064102564103 0.638596235173853
17.1153846153846 1.21368656024406
21.6858974358974 1.63716324953191
27.4807692307692 2.45022083951963
34.8205128205128 3.18820051889067
44.1153846153846 4.85441631962645
55.8974358974359 8.42596428526263
70.8269230769231 5.94512728182668
89.7435897435897 7.23946733964662
113.711538461538 10.145532327309
144.076923076923 7.24636735751918
182.551282051282 9.36694083678201
231.301282051282 14.1282345353095
293.070512820513 7.69383923972402
350 9.34685456412716
};
\addlegendentry{sub 16, exact}
\addplot [, color1, opacity=0.6, mark=square*, mark size=0.5, mark options={solid}, only marks]
table {%
1 0.641129734792194
1.26282051282051 0.603199790468971
1.6025641025641 0.278691547043248
2.03205128205128 0.324073609290395
2.57692307692308 0.39996334157014
3.26282051282051 0.314913871462396
4.13461538461539 0.146983576631965
5.23717948717949 0.225500343432844
6.64102564102564 0.296579388321829
8.41025641025641 0.352067300791752
10.6602564102564 0.428622934320231
13.5064102564103 0.680490482886327
17.1153846153846 0.929913885627347
21.6858974358974 1.33324284758796
27.4807692307692 1.59919909657594
34.8205128205128 2.08819812303848
44.1153846153846 3.3105563850894
55.8974358974359 4.53775019176077
70.8269230769231 5.40757671384735
89.7435897435897 6.07394717518404
113.711538461538 7.95189353087407
144.076923076923 8.94901468831447
182.551282051282 9.22828208322507
231.301282051282 9.51076555807624
293.070512820513 9.83216229700428
350 9.00519368229288
};
\addlegendentry{mb 128, mc 10}
\addplot [, color2, opacity=0.6, mark=triangle*, mark size=0.5, mark options={solid,rotate=180}, only marks]
table {%
1 0.62734526559326
1.26282051282051 0.662321721130723
1.6025641025641 0.560197949726718
2.03205128205128 0.502218867851249
2.57692307692308 0.74596059428226
3.26282051282051 0.622804526616274
4.13461538461539 0.6883479457494
5.23717948717949 0.839614075351206
6.64102564102564 0.904155965391075
8.41025641025641 1.00485487995797
10.6602564102564 1.05006300016676
13.5064102564103 1.26843260356375
17.1153846153846 2.27519765635138
21.6858974358974 3.05099093419294
27.4807692307692 3.73566347008288
34.8205128205128 4.5954836192938
44.1153846153846 7.39572078147804
55.8974358974359 10.538624294905
70.8269230769231 8.67637286818439
89.7435897435897 9.8782173645485
113.711538461538 11.7650449138584
144.076923076923 10.1921674457008
182.551282051282 11.9481337310287
231.301282051282 16.7541638028853
293.070512820513 10.9206256670983
350 11.7112797100718
};
\addlegendentry{sub 16, mc 10}
\end{axis}

\end{tikzpicture}

      \tikzexternaldisable
    \end{minipage}\hfill
    \begin{minipage}{0.50\linewidth}
      \centering
      % defines the pgfplots style "eigspacedefault"
\pgfkeys{/pgfplots/eigspacedefault/.style={
    width=1.03\linewidth,
    height=\goldenRatioInv*1.03*\linewidth,
    every axis plot/.append style={line width = 1pt},
    tick pos = left,
    ylabel near ticks,
    xlabel near ticks,
    xtick align = inside,
    ytick align = inside,
    legend cell align = left,
    legend columns = 1,
    legend pos = north east,
    legend style = {
      fill opacity = 0.9,
      text opacity = 1,
      font = \tiny,
      % column sep=0.1cm,
    },
    legend image post style={scale=2},
    xticklabel style = {font = \small},
    xlabel style = {font = \small},
    axis line style = {black},
    yticklabel style = {font = \small},
    ylabel style = {font = \small},
    title style = {font = \small},
    grid = major,
    grid style = {dashed}
  }
}

\pgfkeys{/pgfplots/eigspacedefaultapp/.style={
    eigspacedefault,
    height=0.6\linewidth,
    legend columns = 2,
  }
}

\pgfkeys{/pgfplots/eigspacenolegend/.style={
    legend image post style = {scale=0},
    legend style = {
      fill opacity = 0,
      draw opacity = 0,
      text opacity = 0,
      font = \small,
      at={(1, 1.025)},
      anchor=south east,
      column sep=0.25cm,
    },
  }
}
%%% Local Variables:
%%% mode: latex
%%% TeX-master: "../main"
%%% End:

      \pgfkeys{/pgfplots/zmystyle/.style={
          eigspacedefaultapp,
          eigspacenolegend,
        }}
      \tikzexternalenable
      \vspace{-6ex}
      % This file was created by tikzplotlib v0.9.7.
\begin{tikzpicture}

\definecolor{color0}{rgb}{0.274509803921569,0.6,0.564705882352941}
\definecolor{color1}{rgb}{0.870588235294118,0.623529411764706,0.0862745098039216}
\definecolor{color2}{rgb}{0.501960784313725,0.184313725490196,0.6}

\begin{axis}[
axis line style={white!10!black},
legend style={fill opacity=0.8, draw opacity=1, text opacity=1, at={(0.97,0.03)}, anchor=south east, draw=white!80!black},
log basis x={10},
tick pos=left,
xlabel={epoch (log scale)},
xmajorgrids,
xmin=0.746099240306814, xmax=469.106495613199,
xmode=log,
ylabel={av. rel. error (log scale)},
ymajorgrids,
ymin=0.119997587807708, ymax=3.09566647918761,
ymode=log,
zmystyle
]
\addplot [, black, opacity=0.6, mark=*, mark size=0.5, mark options={solid}, only marks]
table {%
1 0.805314431378719
1.26282051282051 0.300950954842856
1.6025641025641 0.190279251478634
2.03205128205128 0.182398342672557
2.57692307692308 0.179221858045622
3.26282051282051 0.175741750847941
4.13461538461539 0.166732145784981
5.23717948717949 0.197913092587593
6.64102564102564 0.172775943080721
8.41025641025641 0.139102622607451
10.6602564102564 0.144043609513481
13.5064102564103 0.243554682954986
17.1153846153846 0.29417711711213
21.6858974358974 0.423567609290669
27.4807692307692 0.379322146958845
34.8205128205128 0.409647993232912
44.1153846153846 0.52949759172365
55.8974358974359 0.752220007490846
70.8269230769231 0.966021311124253
89.7435897435897 1.00579918814335
113.711538461538 1.19974102007187
144.076923076923 1.08813032377181
182.551282051282 1.22959355312331
231.301282051282 1.1473607011687
293.070512820513 0.916352317156926
350 0.739416167676223
};
\addlegendentry{mb 128, exact}
\addplot [, color0, opacity=0.6, mark=diamond*, mark size=0.5, mark options={solid}, only marks]
table {%
1 0.945030488535629
1.26282051282051 0.320822929586115
1.6025641025641 0.302336827232013
2.03205128205128 0.330868998836008
2.57692307692308 0.372525100513785
3.26282051282051 0.328640622148736
4.13461538461539 0.389820133068207
5.23717948717949 0.400765084619607
6.64102564102564 0.447345220788167
8.41025641025641 0.410345575584058
10.6602564102564 0.583759384379201
13.5064102564103 0.697697105485114
17.1153846153846 0.809555165020619
21.6858974358974 1.10463390277468
27.4807692307692 0.955293820405729
34.8205128205128 0.962380311189213
44.1153846153846 1.18859789178068
55.8974358974359 1.5426038302628
70.8269230769231 1.60096449394279
89.7435897435897 1.38211869198352
113.711538461538 1.59274893457378
144.076923076923 1.32478619168887
182.551282051282 1.04687631913737
231.301282051282 1.55118682236089
293.070512820513 2.3197848703041
350 2.10375444298379
};
\addlegendentry{sub 16, exact}
\addplot [, color1, opacity=0.6, mark=square*, mark size=0.5, mark options={solid}, only marks]
table {%
1 0.674202390078793
1.26282051282051 0.250027787616856
1.6025641025641 0.231676615579921
2.03205128205128 0.266532667259341
2.57692307692308 0.211191330228927
3.26282051282051 0.23682378207553
4.13461538461539 0.21035262932446
5.23717948717949 0.265891596298788
6.64102564102564 0.19822945549189
8.41025641025641 0.170805706570317
10.6602564102564 0.219374722284877
13.5064102564103 0.430705949329764
17.1153846153846 0.462988895095582
21.6858974358974 0.71638662786421
27.4807692307692 0.561334991541044
34.8205128205128 0.66048529789233
44.1153846153846 0.793441791637311
55.8974358974359 1.05502782017124
70.8269230769231 1.34173215393285
89.7435897435897 1.33160364769956
113.711538461538 1.69309224202163
144.076923076923 1.45434170044291
182.551282051282 1.77202048646335
231.301282051282 1.54972130984587
293.070512820513 1.27071932273066
350 1.01952811840488
};
\addlegendentry{mb 128, mc 10}
\addplot [, color2, opacity=0.6, mark=triangle*, mark size=0.5, mark options={solid,rotate=180}, only marks]
table {%
1 0.734303879474974
1.26282051282051 0.88073162277117
1.6025641025641 0.866803735722558
2.03205128205128 0.937755160760827
2.57692307692308 0.704615242920529
3.26282051282051 0.701462528913144
4.13461538461539 0.675621461338762
5.23717948717949 0.704487848041704
6.64102564102564 0.78220160492253
8.41025641025641 0.595180051105274
10.6602564102564 0.99887621313946
13.5064102564103 1.2946906486948
17.1153846153846 1.54737034308573
21.6858974358974 1.92572475481443
27.4807692307692 1.59310752722597
34.8205128205128 1.86502619332993
44.1153846153846 1.8377790234553
55.8974358974359 2.30840508701845
70.8269230769231 2.67049249824709
89.7435897435897 1.9804217245161
113.711538461538 2.31856106201585
144.076923076923 1.78667633719033
182.551282051282 1.67892860964531
231.301282051282 2.12171602000133
293.070512820513 2.01859060853167
350 2.07495852551672
};
\addlegendentry{sub 16, mc 10}
\end{axis}

\end{tikzpicture}

      \tikzexternaldisable
    \end{minipage}
  \end{subfigure}

  \caption{ \textbf{\bfvivit{}'s versus full-batch quadratic model.} Comparison
    between approximations to the quadratic model and the full-batch model in
    terms of the average relative error for the directional curvature during
    training for all test problems. } \label{vivit::fig:curvature_noise}
\end{figure*}

%%% Local Variables:
%%% mode: latex
%%% TeX-master: "../../thesis"
%%% End:


%%% Local Variables:
%%% mode: latex
%%% TeX-master: "../thesis"
%%% End:
