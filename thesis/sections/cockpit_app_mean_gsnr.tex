\subsection{Gradient Signal-to-noise Ratio (\robustInlinecode{MeanGSNR})}
\label{cockpit::app:mean-gsnr}
The gradient signal-to-noise ratio $\mathrm{GSNR}(\left[ \vtheta \right]_d) \in
\sR$ for a single parameter $\left[ \vtheta \right]_d$ is defined as
\begin{align}
  \label{cockpit::eq:gsnr-definition}
  \mathrm{GSNR}(\left[ \vtheta \right]_d)
  =
  \frac{
  \E_{(\vx, \vy)\sim P}\left[
  \left[
  \nabla_\vtheta\ell(f_{\vtheta}(\vx), \vy)
  \right]_d
  \right]^2
  }{
  \Var_{(\vx, \vy)\sim P}\left[
  \left[
  \nabla_\vtheta \ell(f_{\vtheta}(\vx), \vy)
  \right]_d
  \right]
  }
  =
  \frac{
  \left[
  \vg_{P}(\vtheta)
  \right]^2_d
  }{
  \left[
  \mSigma_P(\vtheta)
  \right]_{d,d}
  }\,.
\end{align}
\begin{subequations}
  \citet{liu2020understanding} use it to explain generalization properties of models
  in the early training phase. We apply their estimation to mini-batches,
  \begin{align}
    \label{cockpit::eq:gsnr-mini-batch}
    \mathrm{GSNR}(\left[ \vtheta \right]_d)
    &\approx
      \frac{
      \left[
      \vg_{\sB}(\vtheta)
      \right]_d^2
      }{
      \frac{|\sB| - 1}{|\sB|}
      \left[
      \hat{\mSigma}_{\sB}(\vtheta)
      \right]_{d,d}
      }
      =
      \frac{
      \left[
      \vg_{\sB}(\vtheta)
      \right]_d^2
      }{
      \frac{1}{|\sB|}
      \left(
      \sum_{n\in \sB}
      \left[
      \vg_n(\vtheta)
      \right]_d^2
      \right)
      -
      \left[
      \vg_{\sB}(\vtheta)
      \right]_d^2
      }\,.
  \end{align}
  Inspired by \citet{liu2020understanding}, \cockpit's \inlinecode{MeanGSNR} computes the average
  GSNR over all parameters,
  \begin{equation}
    \label{cockpit::eq:mean-gsnr-feature-table}
    \frac{1}{D}\sum_{j=1}^D \mathrm{GSNR}(\left[ \vtheta \right]_j)\,.
  \end{equation}
\end{subequations}

\subsubsection{Usage}

The \gsnr describes the gradient noise level which is
influenced, among other things, by the batch size. Using the \gsnr, perhaps in
combination with the gradient tests or the \cabs criterion could provide
practitioners a clearer picture of suitable batch sizes for their particular
problem. As shown by \citet{liu2020understanding}, the \gsnr is also linked to generalization
of neural networks.

%%% Local Variables:
%%% mode: latex
%%% TeX-master: "../thesis"
%%% End:
