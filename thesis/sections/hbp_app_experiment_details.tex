\subsubsection{Fully-connected Neural Network}
The same model as in~\cite{wei2018bdapch} (see
\Cref{hbp::subtable:modelArchitectures1}) is used to extend the experiment
performed therein. The weights of each linear layer are initialized with the
Xavier method of~\citet{glorot2010XavierInit}. Bias terms are intialized to
zero. Backpropagation of the Hessian uses approximation
\Cref{hbp::equ:hessians_batch_average_approximation} of
\Cref{hbp::equ:kfraAndBdapch} to compute the curvature blocks
$\average{\gradsquared{\mW^{(l)}}\ell}$ and
$\average{\gradsquared{\vb^{(l)}}\ell }$.

\begin{table*}[t]
  \caption{\textbf{Model architectures under consideration.} We use
    \robustInlinecode{Conv2d(in\_channels, out\_channels, kernel\_size,
      padding)}, \robustInlinecode{ZeroPad2d(padding\_left, padding\_right,
      padding\_top, padding\_bottom)}, \robustInlinecode{Linear(in\_features,
      out\_features)}, and \robustInlinecode{MaxPool2d(kernel\_size, stride)} as
    patterns to describe module hyperparameters. Convolution strides are always
    one. \subfigref{hbp::subtable:modelArchitectures1} FCNN used to extend the
    experiment in~\citet{wei2018bdapch} (3\,846\,810 parameters).
    \subfigref{hbp::subtable:modelArchitectures2} CNN architecture (1\,099\,226
    parameters). \subfigref{hbp::subtable:modelArchitectures3} DeepOBS 3c3d test
    problem with three convolutional and three dense layers (895\,210
    parameters). ReLU activation functions are replaced by sigmoids.}
  \label{hbp::table:modelArchitectures}
  \begin{subfigure}[t]{0.30\linewidth}
    \centering
    \caption{FCNN
      (\Cref{hbp::fig:experiment_fcnn})}\label{hbp::subtable:modelArchitectures1}
    \begin{footnotesize}
      \begin{tabular}[t]{lll}
        \toprule
        \# & Module
        \\
        \midrule
        1 & \inlinecode{Flatten()}
        \\
        2 & \inlinecode{Linear(3072, 1024)}
        \\
        3 & \inlinecode{Sigmoid()}
        \\
        4 & \inlinecode{Linear(1024, 512)}
        \\
        5 & \inlinecode{Sigmoid()}
        \\
        6 & \inlinecode{Linear(512, 256)}
        \\
        7 & \inlinecode{Sigmoid()}
        \\
        8 & \inlinecode{Linear(256, 128)}
        \\
        9 & \inlinecode{Sigmoid()}
        \\
        10 & \inlinecode{Linear(128, 64)}
        \\
        11 & \inlinecode{Sigmoid()}
        \\
        12 & \inlinecode{Linear(64, 32)}
        \\
        13 & \inlinecode{Sigmoid()}
        \\
        14 & \inlinecode{Linear(32, 16)}
        \\
        15 & \inlinecode{Sigmoid()}
        \\
        16 & \inlinecode{Linear(16, 10)}
        \\
        \bottomrule
      \end{tabular}
    \end{footnotesize}
  \end{subfigure}
  \hfill
  \begin{subfigure}[t]{0.30\linewidth}
    \centering
    \caption{CNN
      (\Cref{hbp::fig:experiment_cnn})}\label{hbp::subtable:modelArchitectures2}
    \begin{footnotesize}
      \begin{tabular}[t]{lll}
        \toprule
        \# & Module
        \\
        \midrule
        1 & \inlinecode{Conv2d(3, 16, 3, 1)}
        \\
        2 & \inlinecode{Sigmoid()}
        \\
        3 & \inlinecode{Conv2d(16, 16, 3, 1)}
        \\
        4 & \inlinecode{Sigmoid()}
        \\
        5 & \inlinecode{MaxPool2d(2, 2)}
        \\
        6 & \inlinecode{Conv2d(16, 32, 3, 1)}
        \\
        7 & \inlinecode{Sigmoid()}
        \\
        8 & \inlinecode{Conv2d(32, 32, 3, 1)}
        \\
        9 & \inlinecode{Sigmoid()}
        \\
        10 & \inlinecode{MaxPool2d(2, 2)}
        \\
        11 & \inlinecode{Flatten()}
        \\
        12 & \inlinecode{Linear(2048, 512)}
        \\
        13 & \inlinecode{Sigmoid()}
        \\
        14 & \inlinecode{Linear(512, 64)}
        \\
        15 & \inlinecode{Sigmoid()}
        \\
        16 & \inlinecode{Linear(64, 10)}
        \\
        \bottomrule
      \end{tabular}
    \end{footnotesize}
  \end{subfigure}
  \hfill
  \begin{subfigure}[t]{0.36\linewidth}
    \centering
    \caption{\DeepOBS \threecthreed
      (\Cref{hbp::fig:experiment_c3d3})}\label{hbp::subtable:modelArchitectures3}
    \begin{footnotesize}
      \begin{tabular}[t]{lll}
        \toprule
        \# & Module
        \\
        \midrule
        1 & \inlinecode{Conv2d(3, 64, 5, 0)}
        \\
        2 & \inlinecode{Sigmoid()}
        \\
        3 & \inlinecode{ZeroPad2d(0, 1, 0, 1)}
        \\
        4 & \inlinecode{MaxPool2d(3, 2)}
        \\
        5 & \inlinecode{Conv2d(64, 96, 3, 0)}
        \\
        6 & \inlinecode{Sigmoid()}
        \\
        7 & \inlinecode{ZeroPad2d(0, 1, 0, 1)}
        \\
        8 & \inlinecode{MaxPool2d(3, 2)}
        \\
        9 & \inlinecode{ZeroPad2d(1, 1, 1, 1)}
        \\
        10 & \inlinecode{Conv2d(96, 128, 3, 0)}
        \\
        11 & \inlinecode{Sigmoid()}
        \\
        12 & \inlinecode{ZeroPad2d(0, 1, 0, 1)}
        \\
        13 & \inlinecode{MaxPool2d(3, 2)}
        \\
        14 & \inlinecode{Flatten()}
        \\
        15 & \inlinecode{Linear(1152, 512)}
        \\
        16 & \inlinecode{Sigmoid()}
        \\
        17 & \inlinecode{Linear(512, 256)}
        \\
        18 & \inlinecode{Sigmoid()}
        \\
        19 & \inlinecode{Linear(256, 10)}
        \\
        \bottomrule
      \end{tabular}
    \end{footnotesize}
  \end{subfigure}
\end{table*}

Hyperparameters are chosen as follows to obtain consistent results with the
original work: all runs shown in \Cref{hbp::fig:experiment_fcnn} use a batch
size of $|\sB| = 500$. For SGD, the learning rate is assigned to $\eta = 0.1$
with momentum $\rho=0.9$. Block-splitting experiments with the second-order
method use the PCH-abs. All runs were performed with a learning rate $\eta =
0.1$ and a regularization strength of $\alpha = 0.02$. For the convergence
criterion of CG, the maximum number of iterations is restricted to $n_\text{CG}
= 50$; convergence is reached at a relative tolerance $\epsilon_{\text{CG}} =
0.1$.

\subsubsection{Convolutional Neural Net}

The CNN architecture shown in \Cref{hbp::subtable:modelArchitectures2} is
trained on a hyperparameter grid. Runs with smallest final training loss are
selected to rerun on different random seeds. The curves in
\Cref{hbp::subfig:experiment_cnn2} represent mean values and standard deviations
for ten different realizations over the random seed. All layer parameters were
initialized with \pytorch's default.

For the first-order methods (SGD, Adam), we considered batch sizes $|\sB| \in
\left\{100, 200, 500\right\}$. For SGD, momentum $\rho$ was tuned over the set
$\left\{0, 0.45, 0.9\right\}$. Although we varied the learning rate over a large
range of values $\eta \in \left\{ 10^{-3}, 10^{-2}, 0.1,1, 10 \right\}$, losses
kept plateauing and did not decrease. In particular, the loss even increased for
the large learning rates. For Adam, we only vary the learning rate $\eta \in
\left\{ 10^{-4}, 10^{-3}, 10^{-2}, 0.1,1, 10 \right\}$.

As second-order methods scale better to large batch sizes, we used $|\sB| \in
\left\{200, 500, 1000\right\}$ for them. The CG convergence parameters were set
to $n_\text{CG} = 200$ and $\epsilon_{\text{CG}} = 0.1$. For all curvature
matrices, we varied the learning rates over $\eta \in \left\{ 0.05, 0.1,0.2
\right\}$ and $\alpha \in \left\{10^{-4}, 10^{-3}, 10^{-2} \right\}$.

To compare with another second-order method, we experimented with a public
PyTorch implementation of the KFAC optimizer
\citep{martens2015optimizing,grosse2016kronecker}
(\href{https://github.com/alecwangcq/KFAC-Pytorch}{\texttt{github.com/alecwangcq/KFAC-Pytorch}}).
All hyperparameters were kept at their default setting. The learning rate was
varied over $\eta \in \left\{ 10^{-4}, 10^{-3}, 10^{-2}, 0.1,1, 10 \right\}$.

The hyperparameters of results shown in \Cref{hbp::fig:experiment_cnn} read as
follows:
\begin{itemize}
\item SGD ($|\sB|= 100, \eta = 10^{-3}, \rho=0.9$). The particular choice of
  these hyperparameters is artificial. This run is representative for SGD, which
  does not achieve any progress at all.

\item Adam ($|\sB|=100, \eta = 10^{-3}$)

\item KFAC ($|\sB|=500, \eta = 0.1$)

\item PCH-abs ($|\sB|=1000, \eta = 0.2, \alpha = 10^{-3}$),\\ PCH-clip
  ($|\sB|=1000, \eta = 0.1, \alpha = 10^{-4}$)

\item GGN, $\alpha_1$ ($|\sB|=1000, \eta = 0.1, \alpha = 10^{-4}$). This run
  does not yield the minimum training loss on the grid; it is shown to
  illustrate that the second-order method can escape the flat regions in early
  stages.

\item GGN, $\alpha_2$ ($|\sB|=1000, \eta = 0.1, \alpha = 10^{-3}$). Compared to
  $\alpha_1$, the second-order method requires more iterations to escape the
  initial plateau, caused by the larger regularization strength. However, this
  leads to improved robustness against noise in later training stages.
\end{itemize}

\paragraph{Additional experiment:} Another experiment conducted with HBP
considers the 3c3d architecture (\Cref{hbp::subtable:modelArchitectures3}) of
DeepOBS \citep{schneider2019deepobs} on CIFAR-10. ReLU activations are replaced
by sigmoids to make the problem more challenging. The hyperparameter grid is
chosen identically to the CNN experiment above, and results are summarized in
\Cref{hbp::fig:experiment_c3d3}. In particular, the hyperparameter settings for
each competitor are:
\begin{itemize}
\item SGD ($|\sB|= 100, \eta = 1, \rho =0$)
\item Adam ($|\sB|=100, \eta = 10^{-3}$)
\item PCH-abs ($|\sB|=500, \eta = 0.1, \alpha = 10^{-3}$),\\ PCH-clip ($|\sB|=500, \eta = 0.1, \alpha = 10^{-2}$)
\item GGN ($|\sB|=500, \eta = 0.05, \alpha = 10^{-3}$)
\end{itemize}

\begin{figure*}
  \centering
  \footnotesize

  \begin{minipage}{0.49\linewidth}
    \setlength{\figwidth}{1.08\linewidth}
    \setlength{\figheight}{0.66\figwidth}
    % Training loss plot
    \HBPresetPGFStyle
    % modify style: do not show legend
    \pgfkeys{/pgfplots/mystyle/.style={
        HBPoriginal,
        HBPnolegend,
        ymax = 2.4,
      }}
    \vspace{-\baselineskip}
    \tikzexternalenable
    % This file was created by matplotlib2tikz v0.6.18.
\begin{tikzpicture}

\definecolor{color0}{rgb}{0.992156862745098,0.552941176470588,0.235294117647059}
\definecolor{color1}{rgb}{0.741176470588235,0,0.149019607843137}
\definecolor{color2}{rgb}{0.145098039215686,0.203921568627451,0.580392156862745}
\definecolor{color3}{rgb}{0.172549019607843,0.498039215686275,0.72156862745098}
\definecolor{color4}{rgb}{0.254901960784314,0.713725490196078,0.768627450980392}

\begin{axis}[
legend entries={{SGD},{Adam},{PCH-abs},{PCH-clip},{GGN}},
mystyle,
xlabel={epoch},
ylabel={train loss},
]
\path [draw=white, fill=color0, opacity=0.3] (axis cs:0.002,8.15762960790896)
--(axis cs:0.002,5.5575114977429)
--(axis cs:0.252,2.30409309216702)
--(axis cs:0.502,2.3038114755644)
--(axis cs:0.752,2.30304209723446)
--(axis cs:1.002,2.30399714267121)
--(axis cs:1.252,2.30440756057284)
--(axis cs:1.502,2.30369602970501)
--(axis cs:1.752,2.30334128255162)
--(axis cs:2.002,2.30357982830532)
--(axis cs:2.252,2.30429105403193)
--(axis cs:2.502,2.30325331208264)
--(axis cs:2.752,2.30322751321083)
--(axis cs:3.002,2.3032031308766)
--(axis cs:3.252,2.30282290329259)
--(axis cs:3.502,2.30318961619533)
--(axis cs:3.752,2.30374926394854)
--(axis cs:4.002,2.30373804471363)
--(axis cs:4.252,2.30342043690537)
--(axis cs:4.502,2.30362684762452)
--(axis cs:4.752,2.30327155702092)
--(axis cs:5.002,2.30377897017975)
--(axis cs:5.252,2.30351530913657)
--(axis cs:5.502,2.3045367971634)
--(axis cs:5.752,2.30334715448246)
--(axis cs:6.002,2.30358166211224)
--(axis cs:6.252,2.30356041294715)
--(axis cs:6.502,2.30399872444559)
--(axis cs:6.752,2.30388654941263)
--(axis cs:7.002,2.30333841538665)
--(axis cs:7.252,2.29459622347869)
--(axis cs:7.502,2.2272709798023)
--(axis cs:7.752,2.21272307397359)
--(axis cs:8.002,2.18925271788205)
--(axis cs:8.252,2.154298301839)
--(axis cs:8.502,2.13634919653915)
--(axis cs:8.752,2.0987016183287)
--(axis cs:9.002,2.11651199416443)
--(axis cs:9.252,2.09122497999338)
--(axis cs:9.502,2.03994849285969)
--(axis cs:9.752,2.02201711329634)
--(axis cs:10.002,1.96194344403655)
--(axis cs:10.252,1.92860769848005)
--(axis cs:10.502,1.95003610479243)
--(axis cs:10.752,1.88440995633866)
--(axis cs:11.002,1.87581333990853)
--(axis cs:11.252,1.81721419278799)
--(axis cs:11.502,1.81045030855925)
--(axis cs:11.752,1.82512818251279)
--(axis cs:12.002,1.7583517513787)
--(axis cs:12.252,1.73944890399724)
--(axis cs:12.502,1.71333690319563)
--(axis cs:12.752,1.70647316790894)
--(axis cs:13.002,1.76456137687486)
--(axis cs:13.252,1.66240169381766)
--(axis cs:13.502,1.65099786588028)
--(axis cs:13.752,1.57560420475556)
--(axis cs:14.002,1.61205606721348)
--(axis cs:14.252,1.5512887512783)
--(axis cs:14.502,1.54068952733425)
--(axis cs:14.752,1.48748894398232)
--(axis cs:15.002,1.46995796858256)
--(axis cs:15.252,1.44449177170314)
--(axis cs:15.502,1.46204705651049)
--(axis cs:15.752,1.43486894703923)
--(axis cs:16.002,1.39960691743015)
--(axis cs:16.252,1.39077461976234)
--(axis cs:16.502,1.33251335508841)
--(axis cs:16.752,1.33289032941156)
--(axis cs:17.002,1.26047451418704)
--(axis cs:17.252,1.28194587584963)
--(axis cs:17.502,1.22866761178562)
--(axis cs:17.752,1.20431648723635)
--(axis cs:18.002,1.20944652042478)
--(axis cs:18.252,1.1522138242729)
--(axis cs:18.502,1.14899507178684)
--(axis cs:18.752,1.12517361462391)
--(axis cs:19.002,1.0906751777914)
--(axis cs:19.252,1.08456533396758)
--(axis cs:19.502,1.04485206247723)
--(axis cs:19.752,1.04141477139645)
--(axis cs:19.752,2.12477519718952)
--(axis cs:19.752,2.12477519718952)
--(axis cs:19.502,2.12468221543872)
--(axis cs:19.252,2.15259558712923)
--(axis cs:19.002,2.18545435406896)
--(axis cs:18.752,2.23669464766705)
--(axis cs:18.502,2.25935077057461)
--(axis cs:18.252,2.26619869422839)
--(axis cs:18.002,2.26413143196017)
--(axis cs:17.752,2.2735850191876)
--(axis cs:17.502,2.27338867693355)
--(axis cs:17.252,2.30153655651579)
--(axis cs:17.002,2.29453501778775)
--(axis cs:16.752,2.30724328989692)
--(axis cs:16.502,2.32168345563394)
--(axis cs:16.252,2.3420711634713)
--(axis cs:16.002,2.35511970705868)
--(axis cs:15.752,2.3651138152975)
--(axis cs:15.502,2.35687720362897)
--(axis cs:15.252,2.36204208945713)
--(axis cs:15.002,2.36960180104787)
--(axis cs:14.752,2.36845385367851)
--(axis cs:14.502,2.37682811087223)
--(axis cs:14.252,2.38198439725838)
--(axis cs:14.002,2.38179595209652)
--(axis cs:13.752,2.39808594741272)
--(axis cs:13.502,2.39010542563126)
--(axis cs:13.252,2.39827979231211)
--(axis cs:13.002,2.37454840152938)
--(axis cs:12.752,2.38378104351684)
--(axis cs:12.502,2.38613640632128)
--(axis cs:12.252,2.38423314193935)
--(axis cs:12.002,2.38510780710824)
--(axis cs:11.752,2.37974652375553)
--(axis cs:11.502,2.38085878586976)
--(axis cs:11.252,2.37818568762125)
--(axis cs:11.002,2.36978719357688)
--(axis cs:10.752,2.36843435346817)
--(axis cs:10.502,2.35685192716711)
--(axis cs:10.252,2.3663236655794)
--(axis cs:10.002,2.35815033075898)
--(axis cs:9.752,2.37628022042101)
--(axis cs:9.502,2.37174784579388)
--(axis cs:9.252,2.37414896524283)
--(axis cs:9.002,2.36990498944)
--(axis cs:8.752,2.3806085127443)
--(axis cs:8.502,2.3648198174379)
--(axis cs:8.252,2.36145410904967)
--(axis cs:8.002,2.36134351930057)
--(axis cs:7.752,2.35294503210551)
--(axis cs:7.502,2.34511681087763)
--(axis cs:7.252,2.31031219517671)
--(axis cs:7.002,2.30616928839448)
--(axis cs:6.752,2.30673037296591)
--(axis cs:6.502,2.30574472762656)
--(axis cs:6.252,2.30537074702599)
--(axis cs:6.002,2.30678244120502)
--(axis cs:5.752,2.30640411771908)
--(axis cs:5.502,2.30600983941802)
--(axis cs:5.252,2.3052368938511)
--(axis cs:5.002,2.30553675896148)
--(axis cs:4.752,2.3067425001385)
--(axis cs:4.502,2.30562198126342)
--(axis cs:4.252,2.30713893122817)
--(axis cs:4.002,2.30633399584424)
--(axis cs:3.752,2.30675598316755)
--(axis cs:3.502,2.30645632267796)
--(axis cs:3.252,2.30665819297511)
--(axis cs:3.002,2.3063885916118)
--(axis cs:2.752,2.30616889677758)
--(axis cs:2.502,2.30646043303454)
--(axis cs:2.252,2.30602226612798)
--(axis cs:2.002,2.30623615069858)
--(axis cs:1.752,2.30581360941615)
--(axis cs:1.502,2.30925658412555)
--(axis cs:1.252,2.306418550267)
--(axis cs:1.002,2.30544795239105)
--(axis cs:0.752,2.30640666947391)
--(axis cs:0.502,2.30626948184832)
--(axis cs:0.252,2.30619819811618)
--(axis cs:0.002,8.15762960790896)
--cycle;

\path [draw=white, fill=color1, opacity=0.3] (axis cs:0.002,2.470506631226)
--(axis cs:0.002,2.35757449975545)
--(axis cs:0.252,2.02883793108422)
--(axis cs:0.502,1.89097750705442)
--(axis cs:0.752,1.74897300899535)
--(axis cs:1.002,1.66978791403059)
--(axis cs:1.252,1.60024181679835)
--(axis cs:1.502,1.53610550876088)
--(axis cs:1.752,1.51046336246094)
--(axis cs:2.002,1.44044785480512)
--(axis cs:2.252,1.39811806184802)
--(axis cs:2.502,1.35772812178482)
--(axis cs:2.752,1.31326129557783)
--(axis cs:3.002,1.30069222767492)
--(axis cs:3.252,1.27130267351396)
--(axis cs:3.502,1.20776406294676)
--(axis cs:3.752,1.2077326947851)
--(axis cs:4.002,1.17260368064337)
--(axis cs:4.252,1.16676732679873)
--(axis cs:4.502,1.13035606112567)
--(axis cs:4.752,1.09809590051946)
--(axis cs:5.002,1.06814399676262)
--(axis cs:5.252,1.05178986705887)
--(axis cs:5.502,1.02951338339733)
--(axis cs:5.752,0.983511222191722)
--(axis cs:6.002,0.993046961707462)
--(axis cs:6.252,0.951836287545286)
--(axis cs:6.502,0.938169661403322)
--(axis cs:6.752,0.932861821155138)
--(axis cs:7.002,0.911029911598996)
--(axis cs:7.252,0.912512533532453)
--(axis cs:7.502,0.871362094066718)
--(axis cs:7.752,0.859362802511021)
--(axis cs:8.002,0.848189429894962)
--(axis cs:8.252,0.819200837020108)
--(axis cs:8.502,0.824010130884221)
--(axis cs:8.752,0.795082662024887)
--(axis cs:9.002,0.78479767218421)
--(axis cs:9.252,0.778987095831397)
--(axis cs:9.502,0.754557503711785)
--(axis cs:9.752,0.734226292443927)
--(axis cs:10.002,0.727234316006789)
--(axis cs:10.252,0.707230748999139)
--(axis cs:10.502,0.70364062331079)
--(axis cs:10.752,0.678814612700526)
--(axis cs:11.002,0.670705875015134)
--(axis cs:11.252,0.65800602585091)
--(axis cs:11.502,0.633060542464016)
--(axis cs:11.752,0.62042069487785)
--(axis cs:12.002,0.617659116498238)
--(axis cs:12.252,0.600654201387805)
--(axis cs:12.502,0.5883377311103)
--(axis cs:12.752,0.572769855421711)
--(axis cs:13.002,0.567022737290037)
--(axis cs:13.252,0.542753690664287)
--(axis cs:13.502,0.555230382563349)
--(axis cs:13.752,0.548669286506749)
--(axis cs:14.002,0.516820214299796)
--(axis cs:14.252,0.503388209376112)
--(axis cs:14.502,0.486285530510702)
--(axis cs:14.752,0.486881280900194)
--(axis cs:15.002,0.469143529852424)
--(axis cs:15.252,0.452580479755395)
--(axis cs:15.502,0.443796890268023)
--(axis cs:15.752,0.430844037880073)
--(axis cs:16.002,0.415833732548367)
--(axis cs:16.252,0.416216006853336)
--(axis cs:16.502,0.396660424678131)
--(axis cs:16.752,0.391725154504218)
--(axis cs:17.002,0.391931663813888)
--(axis cs:17.252,0.359511275134318)
--(axis cs:17.502,0.361077733649265)
--(axis cs:17.752,0.334022148634243)
--(axis cs:18.002,0.333892690554084)
--(axis cs:18.252,0.310381026806926)
--(axis cs:18.502,0.321024240801285)
--(axis cs:18.752,0.303700535809651)
--(axis cs:19.002,0.285663896115288)
--(axis cs:19.252,0.259399754306376)
--(axis cs:19.502,0.266864997274766)
--(axis cs:19.752,0.252764687195486)
--(axis cs:19.752,0.314383250221544)
--(axis cs:19.752,0.314383250221544)
--(axis cs:19.502,0.320356666915527)
--(axis cs:19.252,0.321181165674626)
--(axis cs:19.002,0.332230705706612)
--(axis cs:18.752,0.342677111113414)
--(axis cs:18.502,0.363188700130035)
--(axis cs:18.252,0.384757338223363)
--(axis cs:18.002,0.394110608682213)
--(axis cs:17.752,0.381266687129688)
--(axis cs:17.502,0.416276178942669)
--(axis cs:17.252,0.425160007633932)
--(axis cs:17.002,0.445569778618516)
--(axis cs:16.752,0.440104673519693)
--(axis cs:16.502,0.447634826691346)
--(axis cs:16.252,0.468330380342251)
--(axis cs:16.002,0.486052933272709)
--(axis cs:15.752,0.500010723720422)
--(axis cs:15.502,0.527106750717466)
--(axis cs:15.252,0.508300962076194)
--(axis cs:15.002,0.518753049651589)
--(axis cs:14.752,0.542128228186415)
--(axis cs:14.502,0.541346473511897)
--(axis cs:14.252,0.557853310075029)
--(axis cs:14.002,0.557887949438455)
--(axis cs:13.752,0.601819912654781)
--(axis cs:13.502,0.599789937852148)
--(axis cs:13.252,0.592877799089436)
--(axis cs:13.002,0.639227310393679)
--(axis cs:12.752,0.627210653859447)
--(axis cs:12.502,0.674182749045644)
--(axis cs:12.252,0.664826698422986)
--(axis cs:12.002,0.675608824976677)
--(axis cs:11.752,0.694999479720844)
--(axis cs:11.502,0.69422963488126)
--(axis cs:11.252,0.72396179288612)
--(axis cs:11.002,0.747296694660311)
--(axis cs:10.752,0.76304926984113)
--(axis cs:10.502,0.757002191325441)
--(axis cs:10.252,0.757017359864692)
--(axis cs:10.002,0.778495192393174)
--(axis cs:9.752,0.824850743936841)
--(axis cs:9.502,0.800627814281379)
--(axis cs:9.252,0.831001903535363)
--(axis cs:9.002,0.823972495643396)
--(axis cs:8.752,0.879334464153854)
--(axis cs:8.502,0.870598426340053)
--(axis cs:8.252,0.933432996865085)
--(axis cs:8.002,0.910200496061764)
--(axis cs:7.752,0.92600437640591)
--(axis cs:7.502,0.957819839336297)
--(axis cs:7.252,0.955292577875781)
--(axis cs:7.002,0.996062099375888)
--(axis cs:6.752,1.00952420999568)
--(axis cs:6.502,0.995886584877348)
--(axis cs:6.252,1.0297891382703)
--(axis cs:6.002,1.07579786928238)
--(axis cs:5.752,1.10743192356023)
--(axis cs:5.502,1.09075844239308)
--(axis cs:5.252,1.10671244941605)
--(axis cs:5.002,1.13164376301827)
--(axis cs:4.752,1.1684080295724)
--(axis cs:4.502,1.19325522217663)
--(axis cs:4.252,1.20591946462126)
--(axis cs:4.002,1.25525989338464)
--(axis cs:3.752,1.25173988993932)
--(axis cs:3.502,1.30394071572451)
--(axis cs:3.252,1.31024084359877)
--(axis cs:3.002,1.3550175396667)
--(axis cs:2.752,1.39301707623309)
--(axis cs:2.502,1.42481389233719)
--(axis cs:2.252,1.4496372510716)
--(axis cs:2.002,1.55710876007068)
--(axis cs:1.752,1.56052858280578)
--(axis cs:1.502,1.60000913624339)
--(axis cs:1.252,1.6405370585526)
--(axis cs:1.002,1.72329437089678)
--(axis cs:0.752,1.85070774852723)
--(axis cs:0.502,1.95777148681918)
--(axis cs:0.252,2.06755694158118)
--(axis cs:0.002,2.470506631226)
--cycle;

\path [draw=white, fill=color2, opacity=0.3] (axis cs:0.01,2.36699213977356)
--(axis cs:0.01,2.32974014286499)
--(axis cs:0.26,1.88857511433894)
--(axis cs:0.51,1.43810072104175)
--(axis cs:0.76,1.18758121654715)
--(axis cs:1.01,1.01086577760814)
--(axis cs:1.26,0.903539590241144)
--(axis cs:1.51,0.820212008624459)
--(axis cs:1.76,0.767308023717379)
--(axis cs:2.01,0.710734501092641)
--(axis cs:2.26,0.674478844552644)
--(axis cs:2.51,0.636835038253833)
--(axis cs:2.76,0.606462426294632)
--(axis cs:3.01,0.578500179831332)
--(axis cs:3.26,0.556600811108852)
--(axis cs:3.51,0.529689546015215)
--(axis cs:3.76,0.51265884183655)
--(axis cs:4.01,0.485941678750965)
--(axis cs:4.26,0.465835846966729)
--(axis cs:4.51,0.456419127476761)
--(axis cs:4.76,0.434248650000529)
--(axis cs:5.01,0.419526117913102)
--(axis cs:5.26,0.403514645673378)
--(axis cs:5.51,0.38676630607104)
--(axis cs:5.76,0.376382391896361)
--(axis cs:6.01,0.363181285676362)
--(axis cs:6.26,0.344115899799312)
--(axis cs:6.51,0.34266968703148)
--(axis cs:6.76,0.333010231584814)
--(axis cs:7.01,0.313323478631516)
--(axis cs:7.26,0.308193316558873)
--(axis cs:7.51,0.298858650791685)
--(axis cs:7.76,0.2876018803812)
--(axis cs:8.01,0.281025052862875)
--(axis cs:8.26,0.268365707961545)
--(axis cs:8.51,0.266525018730959)
--(axis cs:8.76,0.253356191232333)
--(axis cs:9.01,0.247892486231579)
--(axis cs:9.26,0.239986402699679)
--(axis cs:9.51,0.235512944605488)
--(axis cs:9.76,0.230265231156031)
--(axis cs:10.01,0.216390447619823)
--(axis cs:10.26,0.213720718138521)
--(axis cs:10.51,0.208555934986069)
--(axis cs:10.76,0.200553404956521)
--(axis cs:11.01,0.199577167699623)
--(axis cs:11.26,0.185364453003112)
--(axis cs:11.51,0.183797774333774)
--(axis cs:11.76,0.186510289048533)
--(axis cs:12.01,0.170990366880288)
--(axis cs:12.26,0.168774909519314)
--(axis cs:12.51,0.166293144135468)
--(axis cs:12.76,0.162967474678033)
--(axis cs:13.01,0.15434830913617)
--(axis cs:13.26,0.153772829614279)
--(axis cs:13.51,0.14889131246126)
--(axis cs:13.76,0.146817694988236)
--(axis cs:14.01,0.137684735536961)
--(axis cs:14.26,0.137584427252886)
--(axis cs:14.51,0.133767967521919)
--(axis cs:14.76,0.135247051644261)
--(axis cs:15.01,0.129248404548642)
--(axis cs:15.26,0.123000068069088)
--(axis cs:15.51,0.118453923101373)
--(axis cs:15.76,0.121924861623944)
--(axis cs:16.01,0.114872227341547)
--(axis cs:16.26,0.114883381955662)
--(axis cs:16.51,0.109067076978667)
--(axis cs:16.76,0.110810197280142)
--(axis cs:17.01,0.107972657278671)
--(axis cs:17.26,0.10386405057774)
--(axis cs:17.51,0.100930505045612)
--(axis cs:17.76,0.102437410105814)
--(axis cs:18.01,0.0966001022486557)
--(axis cs:18.26,0.0921575302266323)
--(axis cs:18.51,0.0893164008484624)
--(axis cs:18.76,0.0917734485814347)
--(axis cs:19.01,0.0879304590442929)
--(axis cs:19.26,0.0836715993448958)
--(axis cs:19.51,0.0823102992313432)
--(axis cs:19.76,0.0844818713675453)
--(axis cs:19.76,0.0970092726339386)
--(axis cs:19.76,0.0970092726339386)
--(axis cs:19.51,0.0983364168269111)
--(axis cs:19.26,0.0965822282269731)
--(axis cs:19.01,0.100644126274272)
--(axis cs:18.76,0.102660465543531)
--(axis cs:18.51,0.101170174814627)
--(axis cs:18.26,0.106438363156394)
--(axis cs:18.01,0.106803544995321)
--(axis cs:17.76,0.111475397299181)
--(axis cs:17.51,0.110821769586842)
--(axis cs:17.26,0.113190684546372)
--(axis cs:17.01,0.117558168276654)
--(axis cs:16.76,0.123825476183202)
--(axis cs:16.51,0.121455917718427)
--(axis cs:16.26,0.124012102849445)
--(axis cs:16.01,0.128967607288942)
--(axis cs:15.76,0.133719310806094)
--(axis cs:15.51,0.138581518595271)
--(axis cs:15.26,0.139077334999454)
--(axis cs:15.01,0.141053543939117)
--(axis cs:14.76,0.151122027707165)
--(axis cs:14.51,0.149372393787133)
--(axis cs:14.26,0.149953952177885)
--(axis cs:14.01,0.154154110073658)
--(axis cs:13.76,0.165179596457496)
--(axis cs:13.51,0.163290655035612)
--(axis cs:13.26,0.16254269913614)
--(axis cs:13.01,0.170309804581863)
--(axis cs:12.76,0.183516378781325)
--(axis cs:12.51,0.18331263372422)
--(axis cs:12.26,0.184247078753353)
--(axis cs:12.01,0.18861796146215)
--(axis cs:11.76,0.20247319605412)
--(axis cs:11.51,0.195078637581052)
--(axis cs:11.26,0.204653455808458)
--(axis cs:11.01,0.217252811720085)
--(axis cs:10.76,0.217310780615149)
--(axis cs:10.51,0.223440851608322)
--(axis cs:10.26,0.228690886385138)
--(axis cs:10.01,0.236003194448086)
--(axis cs:9.76,0.246571095562299)
--(axis cs:9.51,0.250413421246868)
--(axis cs:9.26,0.260000242879659)
--(axis cs:9.01,0.266427449209438)
--(axis cs:8.76,0.271448257849088)
--(axis cs:8.51,0.289429515084425)
--(axis cs:8.26,0.290185752542987)
--(axis cs:8.01,0.297014873427637)
--(axis cs:7.76,0.304451854874995)
--(axis cs:7.51,0.318472746741731)
--(axis cs:7.26,0.330074301858867)
--(axis cs:7.01,0.339609797068099)
--(axis cs:6.76,0.348657775788996)
--(axis cs:6.51,0.355529340745239)
--(axis cs:6.26,0.374134464265858)
--(axis cs:6.01,0.384435887518524)
--(axis cs:5.76,0.401972509179002)
--(axis cs:5.51,0.409184470267537)
--(axis cs:5.26,0.426866956375496)
--(axis cs:5.01,0.441326862700606)
--(axis cs:4.76,0.460003465487524)
--(axis cs:4.51,0.479790080296448)
--(axis cs:4.26,0.492733280354515)
--(axis cs:4.01,0.509398602973965)
--(axis cs:3.76,0.535270540487669)
--(axis cs:3.51,0.558899728868056)
--(axis cs:3.26,0.576104758158421)
--(axis cs:3.01,0.603603203709775)
--(axis cs:2.76,0.635233383069686)
--(axis cs:2.51,0.663432658126782)
--(axis cs:2.26,0.704551120848052)
--(axis cs:2.01,0.749677011759074)
--(axis cs:1.76,0.80772027656796)
--(axis cs:1.51,0.868272981972312)
--(axis cs:1.26,0.948186023829749)
--(axis cs:1.01,1.07712617528798)
--(axis cs:0.76,1.25333665683542)
--(axis cs:0.51,1.49746990044873)
--(axis cs:0.26,1.94855197516149)
--(axis cs:0.01,2.36699213977356)
--cycle;

\path [draw=white, fill=color3, opacity=0.3] (axis cs:0.01,2.37599376174964)
--(axis cs:0.01,2.33553287055932)
--(axis cs:0.26,2.30365207833696)
--(axis cs:0.51,2.30187024415745)
--(axis cs:0.76,2.06162494663876)
--(axis cs:1.01,1.92541129158857)
--(axis cs:1.26,1.75662449149994)
--(axis cs:1.51,1.63009100427716)
--(axis cs:1.76,1.52248657166245)
--(axis cs:2.01,1.40978536705831)
--(axis cs:2.26,1.31914728829227)
--(axis cs:2.51,1.22760441475215)
--(axis cs:2.76,1.14378934202889)
--(axis cs:3.01,1.08177568293806)
--(axis cs:3.26,1.02279473498375)
--(axis cs:3.51,0.97063559834235)
--(axis cs:3.76,0.919493671668144)
--(axis cs:4.01,0.86706482933162)
--(axis cs:4.26,0.828845671845318)
--(axis cs:4.51,0.799230497224259)
--(axis cs:4.76,0.758889047731282)
--(axis cs:5.01,0.725570132612012)
--(axis cs:5.26,0.693522551636795)
--(axis cs:5.51,0.667918632285724)
--(axis cs:5.76,0.647011244923086)
--(axis cs:6.01,0.632168146637108)
--(axis cs:6.26,0.610853617304019)
--(axis cs:6.51,0.59253950312946)
--(axis cs:6.76,0.573794074987382)
--(axis cs:7.01,0.565444837028575)
--(axis cs:7.26,0.547916398336351)
--(axis cs:7.51,0.525476876285573)
--(axis cs:7.76,0.517199710238877)
--(axis cs:8.01,0.504450440699622)
--(axis cs:8.26,0.491319744557969)
--(axis cs:8.51,0.481720203256068)
--(axis cs:8.76,0.466253267021852)
--(axis cs:9.01,0.451612099969894)
--(axis cs:9.26,0.442853841755525)
--(axis cs:9.51,0.431502206610197)
--(axis cs:9.76,0.425834913733839)
--(axis cs:10.01,0.409294142847798)
--(axis cs:10.26,0.40168574687931)
--(axis cs:10.51,0.390271037942722)
--(axis cs:10.76,0.384653999384686)
--(axis cs:11.01,0.381043324992139)
--(axis cs:11.26,0.369190313943012)
--(axis cs:11.51,0.357075328624734)
--(axis cs:11.76,0.3500641640574)
--(axis cs:12.01,0.341064309274063)
--(axis cs:12.26,0.33519516102224)
--(axis cs:12.51,0.333623041604572)
--(axis cs:12.76,0.319974816142579)
--(axis cs:13.01,0.311459874363233)
--(axis cs:13.26,0.304410338414354)
--(axis cs:13.51,0.299582611367955)
--(axis cs:13.76,0.301450559375219)
--(axis cs:14.01,0.289216443123651)
--(axis cs:14.26,0.277503687566916)
--(axis cs:14.51,0.275217279572451)
--(axis cs:14.76,0.27182764321534)
--(axis cs:15.01,0.268355498441509)
--(axis cs:15.26,0.259863282156486)
--(axis cs:15.51,0.255974357657783)
--(axis cs:15.76,0.250873337452187)
--(axis cs:16.01,0.247156928866777)
--(axis cs:16.26,0.237601575536243)
--(axis cs:16.51,0.239431534739687)
--(axis cs:16.76,0.232167205757234)
--(axis cs:17.01,0.22731234678611)
--(axis cs:17.26,0.222088588584049)
--(axis cs:17.51,0.212551490738866)
--(axis cs:17.76,0.212168359860712)
--(axis cs:18.01,0.210153934043274)
--(axis cs:18.26,0.206906754397724)
--(axis cs:18.51,0.198358951145907)
--(axis cs:18.76,0.200737037278237)
--(axis cs:19.01,0.189135099968741)
--(axis cs:19.26,0.184406759534893)
--(axis cs:19.51,0.188366706279489)
--(axis cs:19.76,0.185744090244176)
--(axis cs:19.76,0.198600556090472)
--(axis cs:19.76,0.198600556090472)
--(axis cs:19.51,0.20820701608303)
--(axis cs:19.26,0.204078484381618)
--(axis cs:19.01,0.213445200004746)
--(axis cs:18.76,0.211244646691261)
--(axis cs:18.51,0.223437138264875)
--(axis cs:18.26,0.225928991413738)
--(axis cs:18.01,0.227936765867843)
--(axis cs:17.76,0.235930037394232)
--(axis cs:17.51,0.235354762002947)
--(axis cs:17.26,0.236973245632069)
--(axis cs:17.01,0.248816481739433)
--(axis cs:16.76,0.25577936118212)
--(axis cs:16.51,0.256578646568106)
--(axis cs:16.26,0.259653213339337)
--(axis cs:16.01,0.266080240518179)
--(axis cs:15.76,0.26995331817411)
--(axis cs:15.51,0.2784176594683)
--(axis cs:15.26,0.281136368798714)
--(axis cs:15.01,0.281889238229939)
--(axis cs:14.76,0.290599269569236)
--(axis cs:14.51,0.293053272585905)
--(axis cs:14.26,0.303967380100091)
--(axis cs:14.01,0.311240080771612)
--(axis cs:13.76,0.319295022013255)
--(axis cs:13.51,0.327452189876781)
--(axis cs:13.26,0.322752732025939)
--(axis cs:13.01,0.331440145520877)
--(axis cs:12.76,0.341411798895339)
--(axis cs:12.51,0.357160649801678)
--(axis cs:12.26,0.361329512678092)
--(axis cs:12.01,0.36351009687294)
--(axis cs:11.76,0.377064073428486)
--(axis cs:11.51,0.383554684125892)
--(axis cs:11.26,0.389729246966259)
--(axis cs:11.01,0.400634988024752)
--(axis cs:10.76,0.407473705235675)
--(axis cs:10.51,0.416886353367016)
--(axis cs:10.26,0.426577178268701)
--(axis cs:10.01,0.433039764041164)
--(axis cs:9.76,0.44865833949721)
--(axis cs:9.51,0.451815257979876)
--(axis cs:9.26,0.464654531504973)
--(axis cs:9.01,0.473926499521225)
--(axis cs:8.76,0.48412034681253)
--(axis cs:8.51,0.505281388664785)
--(axis cs:8.26,0.511611653595336)
--(axis cs:8.01,0.528281598990396)
--(axis cs:7.76,0.539232191216048)
--(axis cs:7.51,0.55798471829603)
--(axis cs:7.26,0.57379625053221)
--(axis cs:7.01,0.580319740360188)
--(axis cs:6.76,0.599394527936965)
--(axis cs:6.51,0.620936737505459)
--(axis cs:6.26,0.642991394883939)
--(axis cs:6.01,0.657429972621772)
--(axis cs:5.76,0.688262378543405)
--(axis cs:5.51,0.710156848175396)
--(axis cs:5.26,0.736440321822067)
--(axis cs:5.01,0.758559892297961)
--(axis cs:4.76,0.798770709406016)
--(axis cs:4.51,0.830996687071396)
--(axis cs:4.26,0.867447205351947)
--(axis cs:4.01,0.910038570898788)
--(axis cs:3.76,0.967774776207833)
--(axis cs:3.51,1.01115443166024)
--(axis cs:3.26,1.0702522163102)
--(axis cs:3.01,1.13302977703813)
--(axis cs:2.76,1.20554697217247)
--(axis cs:2.51,1.28909327811895)
--(axis cs:2.26,1.385522044212)
--(axis cs:2.01,1.47078866858622)
--(axis cs:1.76,1.59101149619338)
--(axis cs:1.51,1.71752120981128)
--(axis cs:1.26,1.86033055038544)
--(axis cs:1.01,2.00210469199297)
--(axis cs:0.76,2.25336719508488)
--(axis cs:0.51,2.30502548872219)
--(axis cs:0.26,2.30498711424421)
--(axis cs:0.01,2.37599376174964)
--cycle;

\path [draw=white, fill=color4, opacity=0.3] (axis cs:0.01,2.40240061830944)
--(axis cs:0.01,2.34963442731434)
--(axis cs:0.26,2.12084202106922)
--(axis cs:0.51,1.68342093080893)
--(axis cs:0.76,1.34386847620727)
--(axis cs:1.01,1.13284235366308)
--(axis cs:1.26,0.983553697784973)
--(axis cs:1.51,0.883890117421757)
--(axis cs:1.76,0.810269477528126)
--(axis cs:2.01,0.750588585501047)
--(axis cs:2.26,0.7076543459643)
--(axis cs:2.51,0.670146178538274)
--(axis cs:2.76,0.647042547917607)
--(axis cs:3.01,0.619247359808136)
--(axis cs:3.26,0.581418418957777)
--(axis cs:3.51,0.569164904803001)
--(axis cs:3.76,0.558972358995922)
--(axis cs:4.01,0.532218142987255)
--(axis cs:4.26,0.504491473520845)
--(axis cs:4.51,0.491132863683409)
--(axis cs:4.76,0.489153441673517)
--(axis cs:5.01,0.470685272377593)
--(axis cs:5.26,0.433092826842803)
--(axis cs:5.51,0.433095271747016)
--(axis cs:5.76,0.420908435863079)
--(axis cs:6.01,0.412813291996213)
--(axis cs:6.26,0.378110387276166)
--(axis cs:6.51,0.374894759656075)
--(axis cs:6.76,0.369719659290834)
--(axis cs:7.01,0.360616364374997)
--(axis cs:7.26,0.328520595397607)
--(axis cs:7.51,0.331058543684549)
--(axis cs:7.76,0.311108937036979)
--(axis cs:8.01,0.315120212748998)
--(axis cs:8.26,0.28484449629336)
--(axis cs:8.51,0.284062081332746)
--(axis cs:8.76,0.274975843880046)
--(axis cs:9.01,0.278497254241997)
--(axis cs:9.26,0.250160575863029)
--(axis cs:9.51,0.243754094760091)
--(axis cs:9.76,0.244133994444705)
--(axis cs:10.01,0.238838429820833)
--(axis cs:10.26,0.212735117700372)
--(axis cs:10.51,0.2125430651684)
--(axis cs:10.76,0.185904346842549)
--(axis cs:11.01,0.194535541845862)
--(axis cs:11.26,0.173624111381191)
--(axis cs:11.51,0.169639866819592)
--(axis cs:11.76,0.170515372513711)
--(axis cs:12.01,0.135329023124423)
--(axis cs:12.26,0.139056325523232)
--(axis cs:12.51,0.14130959433529)
--(axis cs:12.76,0.14148587808614)
--(axis cs:13.01,0.135360093873011)
--(axis cs:13.26,0.135169135375424)
--(axis cs:13.51,0.132604398584847)
--(axis cs:13.76,0.138336852183668)
--(axis cs:14.01,0.134225148277075)
--(axis cs:14.26,0.129857781215406)
--(axis cs:14.51,0.124140572987987)
--(axis cs:14.76,0.120813596987651)
--(axis cs:15.01,0.117269235348578)
--(axis cs:15.26,0.118269371699329)
--(axis cs:15.51,0.111086871327627)
--(axis cs:15.76,0.110821217485497)
--(axis cs:16.01,0.0744792628711576)
--(axis cs:16.26,0.0778872120780338)
--(axis cs:16.51,0.0798775997159398)
--(axis cs:16.76,0.0792456835754755)
--(axis cs:17.01,0.0889636108657087)
--(axis cs:17.26,0.0857507613121151)
--(axis cs:17.51,0.070802548403166)
--(axis cs:17.76,0.0802666233440225)
--(axis cs:18.01,0.0818773953897655)
--(axis cs:18.26,0.0800181706128358)
--(axis cs:18.51,0.0836153695601876)
--(axis cs:18.76,0.0844514612737971)
--(axis cs:19.01,0.0796419958261946)
--(axis cs:19.26,0.0828825110530382)
--(axis cs:19.51,0.0815493441699702)
--(axis cs:19.76,0.081617576719308)
--(axis cs:19.76,0.0993230513324498)
--(axis cs:19.76,0.0993230513324498)
--(axis cs:19.51,0.103736545932416)
--(axis cs:19.26,0.102813229527521)
--(axis cs:19.01,0.109337007007172)
--(axis cs:18.76,0.111132734578483)
--(axis cs:18.51,0.117165808914239)
--(axis cs:18.26,0.126034424764133)
--(axis cs:18.01,0.132858453811914)
--(axis cs:17.76,0.143335316930216)
--(axis cs:17.51,0.165690522020437)
--(axis cs:17.26,0.14148987414254)
--(axis cs:17.01,0.154840590618018)
--(axis cs:16.76,0.173060643648302)
--(axis cs:16.51,0.183187739908942)
--(axis cs:16.26,0.194865145941318)
--(axis cs:16.01,0.207930178599942)
--(axis cs:15.76,0.154472061804703)
--(axis cs:15.51,0.159602543113959)
--(axis cs:15.26,0.157298575628762)
--(axis cs:15.01,0.175871786975984)
--(axis cs:14.76,0.172163870191647)
--(axis cs:14.51,0.170951616323995)
--(axis cs:14.26,0.17661296458652)
--(axis cs:14.01,0.19868219165346)
--(axis cs:13.76,0.218634042153033)
--(axis cs:13.51,0.203601656102653)
--(axis cs:13.26,0.205097315864162)
--(axis cs:13.01,0.236258156139387)
--(axis cs:12.76,0.24415571167464)
--(axis cs:12.51,0.264685843167571)
--(axis cs:12.26,0.270310049684669)
--(axis cs:12.01,0.325483777998243)
--(axis cs:11.76,0.252502984643996)
--(axis cs:11.51,0.221868757138041)
--(axis cs:11.26,0.234071295294148)
--(axis cs:11.01,0.261020588563379)
--(axis cs:10.76,0.289532462816456)
--(axis cs:10.51,0.228183522007923)
--(axis cs:10.26,0.238443391296592)
--(axis cs:10.01,0.250478972422781)
--(axis cs:9.76,0.261018707886838)
--(axis cs:9.51,0.269259777744143)
--(axis cs:9.26,0.272888966444825)
--(axis cs:9.01,0.308515352617211)
--(axis cs:8.76,0.307886014726292)
--(axis cs:8.51,0.319126159195361)
--(axis cs:8.26,0.321227053360647)
--(axis cs:8.01,0.350637127443796)
--(axis cs:7.76,0.374020693528664)
--(axis cs:7.51,0.349200487372808)
--(axis cs:7.26,0.369243938360556)
--(axis cs:7.01,0.382077125414966)
--(axis cs:6.76,0.394294972218946)
--(axis cs:6.51,0.408268513678428)
--(axis cs:6.26,0.40810581808234)
--(axis cs:6.01,0.429085894376544)
--(axis cs:5.76,0.448108235317646)
--(axis cs:5.51,0.454696367819406)
--(axis cs:5.26,0.464470702171785)
--(axis cs:5.01,0.49085906847133)
--(axis cs:4.76,0.508131924384833)
--(axis cs:4.51,0.519431773500734)
--(axis cs:4.26,0.525950132523924)
--(axis cs:4.01,0.551674965380665)
--(axis cs:3.76,0.585797548001759)
--(axis cs:3.51,0.592941978722848)
--(axis cs:3.26,0.60204701416295)
--(axis cs:3.01,0.638230328981232)
--(axis cs:2.76,0.662206936621425)
--(axis cs:2.51,0.69086566037469)
--(axis cs:2.26,0.726066481138361)
--(axis cs:2.01,0.774333308572439)
--(axis cs:1.76,0.841660818892925)
--(axis cs:1.51,0.921051262602199)
--(axis cs:1.26,1.03527521876471)
--(axis cs:1.01,1.20798172585047)
--(axis cs:0.76,1.44680981034516)
--(axis cs:0.51,1.81444999127969)
--(axis cs:0.26,2.27987838450939)
--(axis cs:0.01,2.40240061830944)
--cycle;

\addplot [, color0]
table [row sep=\\]{%
0.002	6.85757055282593 \\
0.252	2.3051456451416 \\
0.502	2.30504047870636 \\
0.752	2.30472438335419 \\
1.002	2.30472254753113 \\
1.252	2.30541305541992 \\
1.502	2.30647630691528 \\
1.752	2.30457744598389 \\
2.002	2.30490798950195 \\
2.252	2.30515666007996 \\
2.502	2.30485687255859 \\
2.752	2.3046982049942 \\
3.002	2.3047958612442 \\
3.252	2.30474054813385 \\
3.502	2.30482296943665 \\
3.752	2.30525262355804 \\
4.002	2.30503602027893 \\
4.252	2.30527968406677 \\
4.502	2.30462441444397 \\
4.752	2.30500702857971 \\
5.002	2.30465786457062 \\
5.252	2.30437610149384 \\
5.502	2.30527331829071 \\
5.752	2.30487563610077 \\
6.002	2.30518205165863 \\
6.252	2.30446557998657 \\
6.502	2.30487172603607 \\
6.752	2.30530846118927 \\
7.002	2.30475385189056 \\
7.252	2.3024542093277 \\
7.502	2.28619389533997 \\
7.752	2.28283405303955 \\
8.002	2.27529811859131 \\
8.252	2.25787620544434 \\
8.502	2.25058450698853 \\
8.752	2.2396550655365 \\
9.002	2.24320849180222 \\
9.252	2.2326869726181 \\
9.502	2.20584816932678 \\
9.752	2.19914866685867 \\
10.002	2.16004688739777 \\
10.252	2.14746568202972 \\
10.502	2.15344401597977 \\
10.752	2.12642215490341 \\
11.002	2.12280026674271 \\
11.252	2.09769994020462 \\
11.502	2.09565454721451 \\
11.752	2.10243735313416 \\
12.002	2.07172977924347 \\
12.252	2.06184102296829 \\
12.502	2.04973665475845 \\
12.752	2.04512710571289 \\
13.002	2.06955488920212 \\
13.252	2.03034074306488 \\
13.502	2.02055164575577 \\
13.752	1.98684507608414 \\
14.002	1.996926009655 \\
14.252	1.96663657426834 \\
14.502	1.95875881910324 \\
14.752	1.92797139883041 \\
15.002	1.91977988481522 \\
15.252	1.90326693058014 \\
15.502	1.90946213006973 \\
15.752	1.89999138116837 \\
16.002	1.87736331224442 \\
16.252	1.86642289161682 \\
16.502	1.82709840536118 \\
16.752	1.82006680965424 \\
17.002	1.7775047659874 \\
17.252	1.79174121618271 \\
17.502	1.75102814435959 \\
17.752	1.73895075321198 \\
18.002	1.73678897619247 \\
18.252	1.70920625925064 \\
18.502	1.70417292118073 \\
18.752	1.68093413114548 \\
19.002	1.63806476593018 \\
19.252	1.6185804605484 \\
19.502	1.58476713895798 \\
19.752	1.58309498429298 \\
};
\addplot [, color1]
table [row sep=\\]{%
0.002	2.41404056549072 \\
0.252	2.0481974363327 \\
0.502	1.9243744969368 \\
0.752	1.79984037876129 \\
1.002	1.69654114246368 \\
1.252	1.62038943767548 \\
1.502	1.56805732250214 \\
1.752	1.53549597263336 \\
2.002	1.4987783074379 \\
2.252	1.42387765645981 \\
2.502	1.391271007061 \\
2.752	1.35313918590546 \\
3.002	1.32785488367081 \\
3.252	1.29077175855637 \\
3.502	1.25585238933563 \\
3.752	1.22973629236221 \\
4.002	1.21393178701401 \\
4.252	1.18634339570999 \\
4.502	1.16180564165115 \\
4.752	1.13325196504593 \\
5.002	1.09989387989044 \\
5.252	1.07925115823746 \\
5.502	1.0601359128952 \\
5.752	1.04547157287598 \\
6.002	1.03442241549492 \\
6.252	0.990812712907791 \\
6.502	0.967028123140335 \\
6.752	0.971193015575409 \\
7.002	0.953546005487442 \\
7.252	0.933902555704117 \\
7.502	0.914590966701508 \\
7.752	0.892683589458466 \\
8.002	0.879194962978363 \\
8.252	0.876316916942596 \\
8.502	0.847304278612137 \\
8.752	0.837208563089371 \\
9.002	0.804385083913803 \\
9.252	0.80499449968338 \\
9.502	0.777592658996582 \\
9.752	0.779538518190384 \\
10.002	0.752864754199982 \\
10.252	0.732124054431915 \\
10.502	0.730321407318115 \\
10.752	0.720931941270828 \\
11.002	0.709001284837723 \\
11.252	0.690983909368515 \\
11.502	0.663645088672638 \\
11.752	0.657710087299347 \\
12.002	0.646633970737457 \\
12.252	0.632740449905395 \\
12.502	0.631260240077972 \\
12.752	0.599990254640579 \\
13.002	0.603125023841858 \\
13.252	0.567815744876862 \\
13.502	0.577510160207748 \\
13.752	0.575244599580765 \\
14.002	0.537354081869125 \\
14.252	0.530620759725571 \\
14.502	0.513816002011299 \\
14.752	0.514504754543304 \\
15.002	0.493948289752007 \\
15.252	0.480440720915794 \\
15.502	0.485451820492744 \\
15.752	0.465427380800247 \\
16.002	0.450943332910538 \\
16.252	0.442273193597794 \\
16.502	0.422147625684738 \\
16.752	0.415914914011955 \\
17.002	0.418750721216202 \\
17.252	0.392335641384125 \\
17.502	0.388676956295967 \\
17.752	0.357644417881966 \\
18.002	0.364001649618149 \\
18.252	0.347569182515144 \\
18.502	0.34210647046566 \\
18.752	0.323188823461533 \\
19.002	0.30894730091095 \\
19.252	0.290290459990501 \\
19.502	0.293610832095146 \\
19.752	0.283573968708515 \\
};
\addplot [, color2]
table [row sep=\\]{%
0.01	2.34836614131927 \\
0.26	1.91856354475021 \\
0.51	1.46778531074524 \\
0.76	1.22045893669128 \\
1.01	1.04399597644806 \\
1.26	0.925862807035446 \\
1.51	0.844242495298386 \\
1.76	0.78751415014267 \\
2.01	0.730205756425857 \\
2.26	0.689514982700348 \\
2.51	0.650133848190308 \\
2.76	0.620847904682159 \\
3.01	0.591051691770554 \\
3.26	0.566352784633636 \\
3.51	0.544294637441635 \\
3.76	0.523964691162109 \\
4.01	0.497670140862465 \\
4.26	0.479284563660622 \\
4.51	0.468104603886604 \\
4.76	0.447126057744026 \\
5.01	0.430426490306854 \\
5.26	0.415190801024437 \\
5.51	0.397975388169289 \\
5.76	0.389177450537682 \\
6.01	0.373808586597443 \\
6.26	0.359125182032585 \\
6.51	0.349099513888359 \\
6.76	0.340834003686905 \\
7.01	0.326466637849808 \\
7.26	0.31913380920887 \\
7.51	0.308665698766708 \\
7.76	0.296026867628098 \\
8.01	0.289019963145256 \\
8.26	0.279275730252266 \\
8.51	0.277977266907692 \\
8.76	0.26240222454071 \\
9.01	0.257159967720509 \\
9.26	0.249993322789669 \\
9.51	0.242963182926178 \\
9.76	0.238418163359165 \\
10.01	0.226196821033955 \\
10.26	0.221205802261829 \\
10.51	0.215998393297195 \\
10.76	0.208932092785835 \\
11.01	0.208414989709854 \\
11.26	0.195008954405785 \\
11.51	0.189438205957413 \\
11.76	0.194491742551327 \\
12.01	0.179804164171219 \\
12.26	0.176510994136333 \\
12.51	0.174802888929844 \\
12.76	0.173241926729679 \\
13.01	0.162329056859016 \\
13.26	0.15815776437521 \\
13.51	0.156090983748436 \\
13.76	0.155998645722866 \\
14.01	0.145919422805309 \\
14.26	0.143769189715385 \\
14.51	0.141570180654526 \\
14.76	0.143184539675713 \\
15.01	0.135150974243879 \\
15.26	0.131038701534271 \\
15.51	0.128517720848322 \\
15.76	0.127822086215019 \\
16.01	0.121919917315245 \\
16.26	0.119447742402554 \\
16.51	0.115261497348547 \\
16.76	0.117317836731672 \\
17.01	0.112765412777662 \\
17.26	0.108527367562056 \\
17.51	0.105876137316227 \\
17.76	0.106956403702497 \\
18.01	0.101701823621988 \\
18.26	0.0992979466915131 \\
18.51	0.0952432878315449 \\
18.76	0.0972169570624828 \\
19.01	0.0942872926592827 \\
19.26	0.0901269137859344 \\
19.51	0.0903233580291271 \\
19.76	0.090745572000742 \\
};
\addplot [, color3, dashed]
table [row sep=\\]{%
0.01	2.35576331615448 \\
0.26	2.30431959629059 \\
0.51	2.30344786643982 \\
0.76	2.15749607086182 \\
1.01	1.96375799179077 \\
1.26	1.80847752094269 \\
1.51	1.67380610704422 \\
1.76	1.55674903392792 \\
2.01	1.44028701782227 \\
2.26	1.35233466625214 \\
2.51	1.25834884643555 \\
2.76	1.17466815710068 \\
3.01	1.1074027299881 \\
3.26	1.04652347564697 \\
3.51	0.990895015001297 \\
3.76	0.943634223937988 \\
4.01	0.888551700115204 \\
4.26	0.848146438598633 \\
4.51	0.815113592147827 \\
4.76	0.778829878568649 \\
5.01	0.742065012454987 \\
5.26	0.714981436729431 \\
5.51	0.68903774023056 \\
5.76	0.667636811733246 \\
6.01	0.64479905962944 \\
6.26	0.626922506093979 \\
6.51	0.606738120317459 \\
6.76	0.586594301462174 \\
7.01	0.572882288694382 \\
7.26	0.56085632443428 \\
7.51	0.541730797290802 \\
7.76	0.528215950727463 \\
8.01	0.516366019845009 \\
8.26	0.501465699076653 \\
8.51	0.493500795960426 \\
8.76	0.475186806917191 \\
9.01	0.46276929974556 \\
9.26	0.453754186630249 \\
9.51	0.441658732295036 \\
9.76	0.437246626615524 \\
10.01	0.421166953444481 \\
10.26	0.414131462574005 \\
10.51	0.403578695654869 \\
10.76	0.396063852310181 \\
11.01	0.390839156508446 \\
11.26	0.379459780454636 \\
11.51	0.370315006375313 \\
11.76	0.363564118742943 \\
12.01	0.352287203073502 \\
12.26	0.348262336850166 \\
12.51	0.345391845703125 \\
12.76	0.330693307518959 \\
13.01	0.321450009942055 \\
13.26	0.313581535220146 \\
13.51	0.313517400622368 \\
13.76	0.310372790694237 \\
14.01	0.300228261947632 \\
14.26	0.290735533833504 \\
14.51	0.284135276079178 \\
14.76	0.281213456392288 \\
15.01	0.275122368335724 \\
15.26	0.2704998254776 \\
15.51	0.267196008563042 \\
15.76	0.260413327813149 \\
16.01	0.256618584692478 \\
16.26	0.24862739443779 \\
16.51	0.248005090653896 \\
16.76	0.243973283469677 \\
17.01	0.238064414262772 \\
17.26	0.229530917108059 \\
17.51	0.223953126370907 \\
17.76	0.224049198627472 \\
18.01	0.219045349955559 \\
18.26	0.216417872905731 \\
18.51	0.210898044705391 \\
18.76	0.205990841984749 \\
19.01	0.201290149986744 \\
19.26	0.194242621958256 \\
19.51	0.198286861181259 \\
19.76	0.192172323167324 \\
};
\addplot [, color4, dash pattern=on 1pt off 3pt on 3pt off 3pt]
table [row sep=\\]{%
0.01	2.37601752281189 \\
0.26	2.20036020278931 \\
0.51	1.74893546104431 \\
0.76	1.39533914327621 \\
1.01	1.17041203975677 \\
1.26	1.00941445827484 \\
1.51	0.902470690011978 \\
1.76	0.825965148210525 \\
2.01	0.762460947036743 \\
2.26	0.716860413551331 \\
2.51	0.680505919456482 \\
2.76	0.654624742269516 \\
3.01	0.628738844394684 \\
3.26	0.591732716560364 \\
3.51	0.581053441762924 \\
3.76	0.57238495349884 \\
4.01	0.54194655418396 \\
4.26	0.515220803022385 \\
4.51	0.505282318592072 \\
4.76	0.498642683029175 \\
5.01	0.480772170424461 \\
5.26	0.448781764507294 \\
5.51	0.443895819783211 \\
5.76	0.434508335590363 \\
6.01	0.420949593186378 \\
6.26	0.393108102679253 \\
6.51	0.391581636667252 \\
6.76	0.38200731575489 \\
7.01	0.371346744894981 \\
7.26	0.348882266879082 \\
7.51	0.340129515528679 \\
7.76	0.342564815282822 \\
8.01	0.332878670096397 \\
8.26	0.303035774827004 \\
8.51	0.301594120264053 \\
8.76	0.291430929303169 \\
9.01	0.293506303429604 \\
9.26	0.261524771153927 \\
9.51	0.256506936252117 \\
9.76	0.252576351165771 \\
10.01	0.244658701121807 \\
10.26	0.225589254498482 \\
10.51	0.220363293588161 \\
10.76	0.237718404829502 \\
11.01	0.22777806520462 \\
11.26	0.203847703337669 \\
11.51	0.195754311978817 \\
11.76	0.211509178578854 \\
12.01	0.230406400561333 \\
12.26	0.20468318760395 \\
12.51	0.20299771875143 \\
12.76	0.19282079488039 \\
13.01	0.185809125006199 \\
13.26	0.170133225619793 \\
13.51	0.16810302734375 \\
13.76	0.17848544716835 \\
14.01	0.166453669965267 \\
14.26	0.153235372900963 \\
14.51	0.147546094655991 \\
14.76	0.146488733589649 \\
15.01	0.146570511162281 \\
15.26	0.137783973664045 \\
15.51	0.135344707220793 \\
15.76	0.1326466396451 \\
16.01	0.14120472073555 \\
16.26	0.136376179009676 \\
16.51	0.131532669812441 \\
16.76	0.126153163611889 \\
17.01	0.121902100741863 \\
17.26	0.113620317727327 \\
17.51	0.118246535211802 \\
17.76	0.111800970137119 \\
18.01	0.10736792460084 \\
18.26	0.103026297688484 \\
18.51	0.100390589237213 \\
18.76	0.0977920979261398 \\
19.01	0.0944895014166832 \\
19.26	0.0928478702902794 \\
19.51	0.0926429450511932 \\
19.76	0.0904703140258789 \\
};
\end{axis}

\end{tikzpicture}
    \tikzexternaldisable
  \end{minipage}
  \hfill
  \begin{minipage}{0.49\linewidth}
    \centering
    \setlength{\figwidth}{1.08\linewidth}
    \setlength{\figheight}{0.66\figwidth}
    % Test accuracy plot
    \HBPresetPGFStyle
    % modify style: show legend on lower right
    \pgfkeys{/pgfplots/mystyle/.style={
        HBPoriginal,
        ymin=0.1,
        legend pos = south east,
      }}
    \tikzexternalenable
    % This file was created by matplotlib2tikz v0.6.18.
\begin{tikzpicture}

\definecolor{color0}{rgb}{0.992156862745098,0.552941176470588,0.235294117647059}
\definecolor{color1}{rgb}{0.741176470588235,0,0.149019607843137}
\definecolor{color2}{rgb}{0.145098039215686,0.203921568627451,0.580392156862745}
\definecolor{color3}{rgb}{0.172549019607843,0.498039215686275,0.72156862745098}
\definecolor{color4}{rgb}{0.254901960784314,0.713725490196078,0.768627450980392}

\begin{axis}[
legend entries={{SGD},{Adam},{PCH-abs},{PCH-clip},{GGN}},
mystyle,
xlabel={epoch},
ylabel={test acc},
]
\path [draw=white, fill=color0, opacity=0.3] (axis cs:0.002,0.100000001490116)
--(axis cs:0.002,0.100000001490116)
--(axis cs:0.252,0.100000001490116)
--(axis cs:0.502,0.100000001490116)
--(axis cs:0.752,0.100000001490116)
--(axis cs:1.002,0.100000001490116)
--(axis cs:1.252,0.100000001490116)
--(axis cs:1.502,0.100000001490116)
--(axis cs:1.752,0.100000001490116)
--(axis cs:2.002,0.100000001490116)
--(axis cs:2.252,0.100000001490116)
--(axis cs:2.502,0.100000001490116)
--(axis cs:2.752,0.100000001490116)
--(axis cs:3.002,0.100000001490116)
--(axis cs:3.252,0.100000001490116)
--(axis cs:3.502,0.100000001490116)
--(axis cs:3.752,0.100000001490116)
--(axis cs:4.002,0.100000001490116)
--(axis cs:4.252,0.100000001490116)
--(axis cs:4.502,0.0999135105306346)
--(axis cs:4.752,0.100000001490116)
--(axis cs:5.002,0.100000001490116)
--(axis cs:5.252,0.100000001490116)
--(axis cs:5.502,0.0981685993295657)
--(axis cs:5.752,0.100000001490116)
--(axis cs:6.002,0.100000001490116)
--(axis cs:6.252,0.100000001490116)
--(axis cs:6.502,0.100000001490116)
--(axis cs:6.752,0.0978860678416025)
--(axis cs:7.002,0.100000001490116)
--(axis cs:7.252,0.0892636123851167)
--(axis cs:7.502,0.0815773949001103)
--(axis cs:7.752,0.0845701366176757)
--(axis cs:8.002,0.0751905073891284)
--(axis cs:8.252,0.0724689404358537)
--(axis cs:8.502,0.0730496278990181)
--(axis cs:8.752,0.0667495322682451)
--(axis cs:9.002,0.0692214381971412)
--(axis cs:9.252,0.0655222123051677)
--(axis cs:9.502,0.0686128881692949)
--(axis cs:9.752,0.0671187591258603)
--(axis cs:10.002,0.0756372899084608)
--(axis cs:10.252,0.0730604108920516)
--(axis cs:10.502,0.0765594122949661)
--(axis cs:10.752,0.074683232719535)
--(axis cs:11.002,0.0717558122615111)
--(axis cs:11.252,0.0673059347985902)
--(axis cs:11.502,0.0692389887925196)
--(axis cs:11.752,0.0685311718103293)
--(axis cs:12.002,0.0659241200283013)
--(axis cs:12.252,0.0651261462327637)
--(axis cs:12.502,0.0642267128271787)
--(axis cs:12.752,0.0655582271715739)
--(axis cs:13.002,0.0718240830606737)
--(axis cs:13.252,0.0608310611053148)
--(axis cs:13.502,0.0684514588915451)
--(axis cs:13.752,0.0644474557833202)
--(axis cs:14.002,0.0702582045729933)
--(axis cs:14.252,0.0624717273985783)
--(axis cs:14.502,0.0736921816680606)
--(axis cs:14.752,0.075669341068932)
--(axis cs:15.002,0.0764810510460309)
--(axis cs:15.252,0.0783990176834137)
--(axis cs:15.502,0.0797153364346104)
--(axis cs:15.752,0.0768251379288659)
--(axis cs:16.002,0.0793252951389533)
--(axis cs:16.252,0.0887629133718383)
--(axis cs:16.502,0.0945848878975896)
--(axis cs:16.752,0.10400816414262)
--(axis cs:17.002,0.109295316281497)
--(axis cs:17.252,0.11140514838066)
--(axis cs:17.502,0.117289473557271)
--(axis cs:17.752,0.117702586074348)
--(axis cs:18.002,0.119547862339322)
--(axis cs:18.252,0.120511509181887)
--(axis cs:18.502,0.122500230078517)
--(axis cs:18.752,0.141886847334903)
--(axis cs:19.002,0.152694002428013)
--(axis cs:19.252,0.164230188902636)
--(axis cs:19.502,0.18007083570101)
--(axis cs:19.752,0.18155649547833)
--(axis cs:19.752,0.610523489663425)
--(axis cs:19.752,0.610523489663425)
--(axis cs:19.502,0.609049167880037)
--(axis cs:19.252,0.592249819461088)
--(axis cs:19.002,0.594545995554966)
--(axis cs:18.752,0.581773144883114)
--(axis cs:18.502,0.570379768605413)
--(axis cs:18.252,0.568968487082094)
--(axis cs:18.002,0.550812142562564)
--(axis cs:17.752,0.547577422241692)
--(axis cs:17.502,0.546430532193385)
--(axis cs:17.252,0.525254843360521)
--(axis cs:17.002,0.533424691614926)
--(axis cs:16.752,0.50585184183215)
--(axis cs:16.502,0.503315116787335)
--(axis cs:16.252,0.486057083581172)
--(axis cs:16.002,0.480554706704022)
--(axis cs:15.752,0.462974862261869)
--(axis cs:15.502,0.455004656739466)
--(axis cs:15.252,0.46238097823486)
--(axis cs:15.002,0.445378947954995)
--(axis cs:14.752,0.449370667058757)
--(axis cs:14.502,0.428447814324123)
--(axis cs:14.252,0.418868276360091)
--(axis cs:14.002,0.398721799320382)
--(axis cs:13.752,0.410732542793321)
--(axis cs:13.502,0.386808539251174)
--(axis cs:13.252,0.377888941665109)
--(axis cs:13.002,0.353615922897407)
--(axis cs:12.752,0.362861777077045)
--(axis cs:12.502,0.365513289924172)
--(axis cs:12.252,0.35357385292085)
--(axis cs:12.002,0.347235879223064)
--(axis cs:11.752,0.321928827631771)
--(axis cs:11.502,0.324921015167613)
--(axis cs:11.252,0.321194073188432)
--(axis cs:11.002,0.298484187628816)
--(axis cs:10.752,0.302856761103039)
--(axis cs:10.502,0.265980589574236)
--(axis cs:10.252,0.273539591158348)
--(axis cs:10.002,0.255202713581987)
--(axis cs:9.752,0.233801243036596)
--(axis cs:9.502,0.222587114858621)
--(axis cs:9.252,0.210297791144749)
--(axis cs:9.002,0.185778562994952)
--(axis cs:8.752,0.190210472299855)
--(axis cs:8.502,0.175450373531493)
--(axis cs:8.252,0.176031060994658)
--(axis cs:8.002,0.155509493731439)
--(axis cs:7.752,0.132529866744026)
--(axis cs:7.502,0.135462608936045)
--(axis cs:7.252,0.121116390471138)
--(axis cs:7.002,0.100000001490116)
--(axis cs:6.752,0.104553934719013)
--(axis cs:6.502,0.100000001490116)
--(axis cs:6.252,0.100000001490116)
--(axis cs:6.002,0.100000001490116)
--(axis cs:5.752,0.100000001490116)
--(axis cs:5.502,0.100951403655435)
--(axis cs:5.252,0.100000001490116)
--(axis cs:5.002,0.100000001490116)
--(axis cs:4.752,0.100000001490116)
--(axis cs:4.502,0.100166492313699)
--(axis cs:4.252,0.100000001490116)
--(axis cs:4.002,0.100000001490116)
--(axis cs:3.752,0.100000001490116)
--(axis cs:3.502,0.100000001490116)
--(axis cs:3.252,0.100000001490116)
--(axis cs:3.002,0.100000001490116)
--(axis cs:2.752,0.100000001490116)
--(axis cs:2.502,0.100000001490116)
--(axis cs:2.252,0.100000001490116)
--(axis cs:2.002,0.100000001490116)
--(axis cs:1.752,0.100000001490116)
--(axis cs:1.502,0.100000001490116)
--(axis cs:1.252,0.100000001490116)
--(axis cs:1.002,0.100000001490116)
--(axis cs:0.752,0.100000001490116)
--(axis cs:0.502,0.100000001490116)
--(axis cs:0.252,0.100000001490116)
--(axis cs:0.002,0.100000001490116)
--cycle;

\path [draw=white, fill=color1, opacity=0.3] (axis cs:0.002,0.100000001490116)
--(axis cs:0.002,0.100000001490116)
--(axis cs:0.252,0.210901740945397)
--(axis cs:0.502,0.273698574664893)
--(axis cs:0.752,0.320314492251709)
--(axis cs:1.002,0.373893180182346)
--(axis cs:1.252,0.393318266195911)
--(axis cs:1.502,0.411263603040927)
--(axis cs:1.752,0.423665205128021)
--(axis cs:2.002,0.436879104050083)
--(axis cs:2.252,0.47119395014546)
--(axis cs:2.502,0.477805981720476)
--(axis cs:2.752,0.4866442531608)
--(axis cs:3.002,0.503983751307384)
--(axis cs:3.252,0.517773700951912)
--(axis cs:3.502,0.515113311482864)
--(axis cs:3.752,0.538388514152059)
--(axis cs:4.002,0.532065272703209)
--(axis cs:4.252,0.55310492954414)
--(axis cs:4.502,0.555973123020354)
--(axis cs:4.752,0.565047590019122)
--(axis cs:5.002,0.576847973587051)
--(axis cs:5.252,0.583938016677124)
--(axis cs:5.502,0.591970178369443)
--(axis cs:5.752,0.586870763457673)
--(axis cs:6.002,0.589583378767362)
--(axis cs:6.252,0.606516032129761)
--(axis cs:6.502,0.616202774260937)
--(axis cs:6.752,0.611840444055273)
--(axis cs:7.002,0.616742196546125)
--(axis cs:7.252,0.626025432794976)
--(axis cs:7.502,0.629902896465474)
--(axis cs:7.752,0.638070403594162)
--(axis cs:8.002,0.639362554181445)
--(axis cs:8.252,0.633137349672335)
--(axis cs:8.502,0.651327790519931)
--(axis cs:8.752,0.646354377031384)
--(axis cs:9.002,0.663505777491371)
--(axis cs:9.252,0.659070312928351)
--(axis cs:9.502,0.6684752152385)
--(axis cs:9.752,0.659885411704825)
--(axis cs:10.002,0.667878340631236)
--(axis cs:10.252,0.676990994069605)
--(axis cs:10.502,0.676090665846701)
--(axis cs:10.752,0.672186544273622)
--(axis cs:11.002,0.67745341975898)
--(axis cs:11.252,0.678898958645558)
--(axis cs:11.502,0.685888270377198)
--(axis cs:11.752,0.687133480917541)
--(axis cs:12.002,0.692726069287105)
--(axis cs:12.252,0.690766424047622)
--(axis cs:12.502,0.68791772515585)
--(axis cs:12.752,0.702178901498366)
--(axis cs:13.002,0.699264241855694)
--(axis cs:13.252,0.705045303518812)
--(axis cs:13.502,0.701630769432322)
--(axis cs:13.752,0.700687955659255)
--(axis cs:14.002,0.711288023943135)
--(axis cs:14.252,0.708324133643633)
--(axis cs:14.502,0.707305946166079)
--(axis cs:14.752,0.708411263291638)
--(axis cs:15.002,0.71604430517778)
--(axis cs:15.252,0.711127119249976)
--(axis cs:15.502,0.708814118418406)
--(axis cs:15.752,0.713323726941774)
--(axis cs:16.002,0.712751984063219)
--(axis cs:16.252,0.718176045933759)
--(axis cs:16.502,0.719678668887507)
--(axis cs:16.752,0.71731346729294)
--(axis cs:17.002,0.715455012943961)
--(axis cs:17.252,0.71249921228777)
--(axis cs:17.502,0.715683795918074)
--(axis cs:17.752,0.72236201521449)
--(axis cs:18.002,0.718893455281389)
--(axis cs:18.252,0.720753172911583)
--(axis cs:18.502,0.722192691623078)
--(axis cs:18.752,0.723100821348571)
--(axis cs:19.002,0.724517615440067)
--(axis cs:19.252,0.723074184937257)
--(axis cs:19.502,0.718158024849314)
--(axis cs:19.752,0.723457827677433)
--(axis cs:19.752,0.736042199025448)
--(axis cs:19.752,0.736042199025448)
--(axis cs:19.502,0.738901978431326)
--(axis cs:19.252,0.739805830435973)
--(axis cs:19.002,0.739662389156643)
--(axis cs:18.752,0.735639162686921)
--(axis cs:18.502,0.734487284363403)
--(axis cs:18.252,0.737466820202888)
--(axis cs:18.002,0.733026541457045)
--(axis cs:17.752,0.73669799807973)
--(axis cs:17.502,0.730896208774004)
--(axis cs:17.252,0.733520771003856)
--(axis cs:17.002,0.726984996173166)
--(axis cs:16.752,0.732786525459135)
--(axis cs:16.502,0.732521302788366)
--(axis cs:16.252,0.728183969935381)
--(axis cs:16.002,0.730868006285597)
--(axis cs:15.752,0.728256282518675)
--(axis cs:15.502,0.727825881924917)
--(axis cs:15.252,0.726872868352258)
--(axis cs:15.002,0.724315692571548)
--(axis cs:14.752,0.725708736117084)
--(axis cs:14.502,0.72637406367584)
--(axis cs:14.252,0.72333585094499)
--(axis cs:14.002,0.720171987062266)
--(axis cs:13.752,0.712852038103715)
--(axis cs:13.502,0.716589226066335)
--(axis cs:13.252,0.718334701840837)
--(axis cs:13.002,0.709555755454945)
--(axis cs:12.752,0.712621110613298)
--(axis cs:12.502,0.713482268740634)
--(axis cs:12.252,0.71363356698784)
--(axis cs:12.002,0.708373922511296)
--(axis cs:11.752,0.708146522630127)
--(axis cs:11.502,0.704971750737151)
--(axis cs:11.252,0.699761052645946)
--(axis cs:11.002,0.697666564390953)
--(axis cs:10.752,0.696273455287688)
--(axis cs:10.502,0.692669335812692)
--(axis cs:10.252,0.692789009954901)
--(axis cs:10.002,0.687941671461355)
--(axis cs:9.752,0.684974593673899)
--(axis cs:9.502,0.681404812532496)
--(axis cs:9.252,0.674229693461267)
--(axis cs:9.002,0.67663423143502)
--(axis cs:8.752,0.672305619955005)
--(axis cs:8.502,0.667012212970517)
--(axis cs:8.252,0.666962666921598)
--(axis cs:8.002,0.656697447668683)
--(axis cs:7.752,0.657069600563858)
--(axis cs:7.502,0.653537109313792)
--(axis cs:7.252,0.640774577409339)
--(axis cs:7.002,0.644277796282245)
--(axis cs:6.752,0.640279562029169)
--(axis cs:6.502,0.638157222057881)
--(axis cs:6.252,0.635023958772186)
--(axis cs:6.002,0.624856621289859)
--(axis cs:5.752,0.620989241444213)
--(axis cs:5.502,0.610569813963015)
--(axis cs:5.252,0.601221988415497)
--(axis cs:5.002,0.599212035415635)
--(axis cs:4.752,0.586932416675671)
--(axis cs:4.502,0.581606878333864)
--(axis cs:4.252,0.563495095156025)
--(axis cs:4.002,0.565834724531136)
--(axis cs:3.752,0.553271484741679)
--(axis cs:3.502,0.555466688202423)
--(axis cs:3.252,0.531986306906364)
--(axis cs:3.002,0.521236243476017)
--(axis cs:2.752,0.516735753390942)
--(axis cs:2.502,0.498654016171904)
--(axis cs:2.252,0.482366050512575)
--(axis cs:2.002,0.467320895520763)
--(axis cs:1.752,0.447994794957809)
--(axis cs:1.502,0.431196397474057)
--(axis cs:1.252,0.40804173298393)
--(axis cs:1.002,0.38918681473457)
--(axis cs:0.752,0.358265511725113)
--(axis cs:0.502,0.30660142266482)
--(axis cs:0.252,0.246358262740555)
--(axis cs:0.002,0.100000001490116)
--cycle;

\path [draw=white, fill=color2, opacity=0.3] (axis cs:0.01,0.100000001490116)
--(axis cs:0.01,0.100000001490116)
--(axis cs:0.26,0.27773133285456)
--(axis cs:0.51,0.448526288323671)
--(axis cs:0.76,0.53990648608281)
--(axis cs:1.01,0.605474585910646)
--(axis cs:1.26,0.648743353816249)
--(axis cs:1.51,0.676642764496457)
--(axis cs:1.76,0.694198008266957)
--(axis cs:2.01,0.709357113341105)
--(axis cs:2.26,0.722241699587659)
--(axis cs:2.51,0.727412707833917)
--(axis cs:2.76,0.735275794847441)
--(axis cs:3.01,0.739546597610106)
--(axis cs:3.26,0.74646312881642)
--(axis cs:3.51,0.748385027225861)
--(axis cs:3.76,0.75671830567305)
--(axis cs:4.01,0.75719978890656)
--(axis cs:4.26,0.75934490055761)
--(axis cs:4.51,0.760409501562246)
--(axis cs:4.76,0.763515175112501)
--(axis cs:5.01,0.767598669191345)
--(axis cs:5.26,0.764877930411277)
--(axis cs:5.51,0.76563111913318)
--(axis cs:5.76,0.766535507815889)
--(axis cs:6.01,0.767354465586348)
--(axis cs:6.26,0.769324311937506)
--(axis cs:6.51,0.769652444733372)
--(axis cs:6.76,0.769113554186928)
--(axis cs:7.01,0.769671984824641)
--(axis cs:7.26,0.769701606554246)
--(axis cs:7.51,0.77024022171953)
--(axis cs:7.76,0.770034215133161)
--(axis cs:8.01,0.771412106983199)
--(axis cs:8.26,0.770769260084389)
--(axis cs:8.51,0.768632755964286)
--(axis cs:8.76,0.770571556350054)
--(axis cs:9.01,0.771223031615339)
--(axis cs:9.26,0.771333744735487)
--(axis cs:9.51,0.772535021769186)
--(axis cs:9.76,0.769511155803491)
--(axis cs:10.01,0.770705567463388)
--(axis cs:10.26,0.771361630401248)
--(axis cs:10.51,0.771391363985562)
--(axis cs:10.76,0.771261664565552)
--(axis cs:11.01,0.772144416843252)
--(axis cs:11.26,0.771245095751496)
--(axis cs:11.51,0.770154887228972)
--(axis cs:11.76,0.769343817063719)
--(axis cs:12.01,0.770814881705033)
--(axis cs:12.26,0.773044785643695)
--(axis cs:12.51,0.770003770049302)
--(axis cs:12.76,0.769670636921468)
--(axis cs:13.01,0.770719934668345)
--(axis cs:13.26,0.77266280443742)
--(axis cs:13.51,0.770283438349673)
--(axis cs:13.76,0.768855481175092)
--(axis cs:14.01,0.769180121397605)
--(axis cs:14.26,0.770499571179865)
--(axis cs:14.51,0.771309041437028)
--(axis cs:14.76,0.768479841748537)
--(axis cs:15.01,0.768888699442753)
--(axis cs:15.26,0.769556562098597)
--(axis cs:15.51,0.77040733723366)
--(axis cs:15.76,0.769179707412537)
--(axis cs:16.01,0.768018600141334)
--(axis cs:16.26,0.77038170445672)
--(axis cs:16.51,0.770374393483201)
--(axis cs:16.76,0.768513206140041)
--(axis cs:17.01,0.769853845092436)
--(axis cs:17.26,0.76866441672694)
--(axis cs:17.51,0.769130600822717)
--(axis cs:17.76,0.768492689818343)
--(axis cs:18.01,0.769668638723037)
--(axis cs:18.26,0.77021948918162)
--(axis cs:18.51,0.769509081355928)
--(axis cs:18.76,0.767971662152256)
--(axis cs:19.01,0.769280083498144)
--(axis cs:19.26,0.769385044220832)
--(axis cs:19.51,0.770244769663517)
--(axis cs:19.76,0.767741434649409)
--(axis cs:19.76,0.77455856315714)
--(axis cs:19.76,0.77455856315714)
--(axis cs:19.51,0.775455220132168)
--(axis cs:19.26,0.774754948969934)
--(axis cs:19.01,0.773539905229426)
--(axis cs:18.76,0.774128314387356)
--(axis cs:18.51,0.774610896595122)
--(axis cs:18.26,0.775460521137136)
--(axis cs:18.01,0.775211358053544)
--(axis cs:17.76,0.774967297822038)
--(axis cs:17.51,0.774029384719581)
--(axis cs:17.26,0.775495579782612)
--(axis cs:17.01,0.774626168754915)
--(axis cs:16.76,0.774006792410374)
--(axis cs:16.51,0.77708561418434)
--(axis cs:16.26,0.776398309581366)
--(axis cs:16.01,0.774661389196587)
--(axis cs:15.76,0.774020284767334)
--(axis cs:15.51,0.776212674095991)
--(axis cs:15.26,0.777003439274694)
--(axis cs:15.01,0.775791311829677)
--(axis cs:14.76,0.774040180643736)
--(axis cs:14.51,0.776210940424087)
--(axis cs:14.26,0.775700430059911)
--(axis cs:14.01,0.775919888043771)
--(axis cs:13.76,0.774844534369799)
--(axis cs:13.51,0.775536547516874)
--(axis cs:13.26,0.77715717761443)
--(axis cs:13.01,0.776840065751272)
--(axis cs:12.76,0.774909357757029)
--(axis cs:12.51,0.777076222245009)
--(axis cs:12.26,0.777855196808697)
--(axis cs:12.01,0.778185107327712)
--(axis cs:11.76,0.774096179178804)
--(axis cs:11.51,0.777725118607515)
--(axis cs:11.26,0.777314915635375)
--(axis cs:11.01,0.777655597652598)
--(axis cs:10.76,0.777658346954834)
--(axis cs:10.51,0.776668624036289)
--(axis cs:10.26,0.778938359775907)
--(axis cs:10.01,0.777494419948111)
--(axis cs:9.76,0.776448853294562)
--(axis cs:9.51,0.778424970639567)
--(axis cs:9.26,0.777846274166338)
--(axis cs:9.01,0.779056967163958)
--(axis cs:8.76,0.777328440884291)
--(axis cs:8.51,0.775327243120186)
--(axis cs:8.26,0.777230765664817)
--(axis cs:8.01,0.776647904880509)
--(axis cs:7.76,0.776885786373644)
--(axis cs:7.51,0.775579764147017)
--(axis cs:7.26,0.776878386216903)
--(axis cs:7.01,0.776848009911076)
--(axis cs:6.76,0.776126430802239)
--(axis cs:6.51,0.775307553397427)
--(axis cs:6.26,0.77487569120962)
--(axis cs:6.01,0.775525533097581)
--(axis cs:5.76,0.772644472938963)
--(axis cs:5.51,0.772448881744201)
--(axis cs:5.26,0.770462068787636)
--(axis cs:5.01,0.772361321786896)
--(axis cs:4.76,0.767864834538683)
--(axis cs:4.51,0.765310486783855)
--(axis cs:4.26,0.765095113804725)
--(axis cs:4.01,0.764260207793726)
--(axis cs:3.76,0.760461684141708)
--(axis cs:3.51,0.757834991637817)
--(axis cs:3.26,0.753276863802141)
--(axis cs:3.01,0.750373410572419)
--(axis cs:2.76,0.744064198152589)
--(axis cs:2.51,0.738627295465796)
--(axis cs:2.26,0.729838311780139)
--(axis cs:2.01,0.720542901059377)
--(axis cs:1.76,0.706421999247996)
--(axis cs:1.51,0.690597223353732)
--(axis cs:1.26,0.663136650589727)
--(axis cs:1.01,0.626525419334562)
--(axis cs:0.76,0.56559352059291)
--(axis cs:0.51,0.46625372642967)
--(axis cs:0.26,0.303288664742181)
--(axis cs:0.01,0.100000001490116)
--cycle;

\path [draw=white, fill=color3, opacity=0.3] (axis cs:0.01,0.100000001490116)
--(axis cs:0.01,0.100000001490116)
--(axis cs:0.26,0.0956970686035555)
--(axis cs:0.51,0.0885020706874143)
--(axis cs:0.76,0.143134487698432)
--(axis cs:1.01,0.259293834491925)
--(axis cs:1.26,0.312563731620839)
--(axis cs:1.51,0.371136780353167)
--(axis cs:1.76,0.410575320454171)
--(axis cs:2.01,0.451912662567)
--(axis cs:2.26,0.484558507177029)
--(axis cs:2.51,0.524830644703665)
--(axis cs:2.76,0.554162586509955)
--(axis cs:3.01,0.580894091878799)
--(axis cs:3.26,0.601377256530064)
--(axis cs:3.51,0.621520662917818)
--(axis cs:3.76,0.641807855371775)
--(axis cs:4.01,0.658952313334299)
--(axis cs:4.26,0.672989290486272)
--(axis cs:4.51,0.686371019931315)
--(axis cs:4.76,0.696676130614684)
--(axis cs:5.01,0.702864013005697)
--(axis cs:5.26,0.707709961452162)
--(axis cs:5.51,0.715165331480877)
--(axis cs:5.76,0.720203375925305)
--(axis cs:6.01,0.727289588845615)
--(axis cs:6.26,0.729885735636767)
--(axis cs:6.51,0.7336459327602)
--(axis cs:6.76,0.737983762644117)
--(axis cs:7.01,0.73926618042689)
--(axis cs:7.26,0.740564535748472)
--(axis cs:7.51,0.744892621634123)
--(axis cs:7.76,0.748645077188084)
--(axis cs:8.01,0.750739935065224)
--(axis cs:8.26,0.749545363873383)
--(axis cs:8.51,0.75447479877463)
--(axis cs:8.76,0.755850943115515)
--(axis cs:9.01,0.758844396476319)
--(axis cs:9.26,0.758294976678344)
--(axis cs:9.51,0.760251649628274)
--(axis cs:9.76,0.758108099622883)
--(axis cs:10.01,0.760021970805328)
--(axis cs:10.26,0.760839139173707)
--(axis cs:10.51,0.762621215019501)
--(axis cs:10.76,0.760388746568886)
--(axis cs:11.01,0.762611564791191)
--(axis cs:11.26,0.763849225336281)
--(axis cs:11.51,0.764387290251501)
--(axis cs:11.76,0.763219667099601)
--(axis cs:12.01,0.766636960896811)
--(axis cs:12.26,0.764833479387459)
--(axis cs:12.51,0.764814520974747)
--(axis cs:12.76,0.765295134739052)
--(axis cs:13.01,0.767854463707034)
--(axis cs:13.26,0.768734556145079)
--(axis cs:13.51,0.766912495719002)
--(axis cs:13.76,0.767776735733967)
--(axis cs:14.01,0.764810350132499)
--(axis cs:14.26,0.767939248527986)
--(axis cs:14.51,0.768132840692782)
--(axis cs:14.76,0.769737066297665)
--(axis cs:15.01,0.768048686889004)
--(axis cs:15.26,0.767370632533483)
--(axis cs:15.51,0.76804508486469)
--(axis cs:15.76,0.768680976566864)
--(axis cs:16.01,0.76878673517966)
--(axis cs:16.26,0.769878249963327)
--(axis cs:16.51,0.768157029771455)
--(axis cs:16.76,0.767911756561356)
--(axis cs:17.01,0.769261463822906)
--(axis cs:17.26,0.770345117289906)
--(axis cs:17.51,0.768762951558256)
--(axis cs:17.76,0.768245672797325)
--(axis cs:18.01,0.768518165781413)
--(axis cs:18.26,0.767352125942973)
--(axis cs:18.51,0.767267734955483)
--(axis cs:18.76,0.769266942043049)
--(axis cs:19.01,0.768141216284101)
--(axis cs:19.26,0.770639546894357)
--(axis cs:19.51,0.767950406144146)
--(axis cs:19.76,0.770514366506283)
--(axis cs:19.76,0.774425640226658)
--(axis cs:19.76,0.774425640226658)
--(axis cs:19.51,0.774949607779499)
--(axis cs:19.26,0.77562045238085)
--(axis cs:19.01,0.776098769658739)
--(axis cs:18.76,0.777013048630016)
--(axis cs:18.51,0.77449225669033)
--(axis cs:18.26,0.775907876193257)
--(axis cs:18.01,0.775041838452901)
--(axis cs:17.76,0.77505432405555)
--(axis cs:17.51,0.776057047182894)
--(axis cs:17.26,0.776334892075176)
--(axis cs:17.01,0.774438539801056)
--(axis cs:16.76,0.773288249446792)
--(axis cs:16.51,0.776562964296691)
--(axis cs:16.26,0.775201732317404)
--(axis cs:16.01,0.77351325070596)
--(axis cs:15.76,0.773619021239685)
--(axis cs:15.51,0.774994916527674)
--(axis cs:15.26,0.775469347591945)
--(axis cs:15.01,0.775071327778507)
--(axis cs:14.76,0.774122940034733)
--(axis cs:14.51,0.775527168214536)
--(axis cs:14.26,0.775720760379332)
--(axis cs:14.01,0.77502964477488)
--(axis cs:13.76,0.772803286601085)
--(axis cs:13.51,0.772307503594353)
--(axis cs:13.26,0.775065440230958)
--(axis cs:13.01,0.775025523055967)
--(axis cs:12.76,0.773524857803215)
--(axis cs:12.51,0.77082548032225)
--(axis cs:12.26,0.772046525018516)
--(axis cs:12.01,0.772383029070535)
--(axis cs:11.76,0.771340321875924)
--(axis cs:11.51,0.769312722909204)
--(axis cs:11.26,0.770250782674583)
--(axis cs:11.01,0.770048454606545)
--(axis cs:10.76,0.767411253145012)
--(axis cs:10.51,0.770218774642669)
--(axis cs:10.26,0.76850088958911)
--(axis cs:10.01,0.767138029995758)
--(axis cs:9.76,0.766551903562389)
--(axis cs:9.51,0.76458835958422)
--(axis cs:9.26,0.764465009245423)
--(axis cs:9.01,0.763555613155791)
--(axis cs:8.76,0.762289039107996)
--(axis cs:8.51,0.759985216522306)
--(axis cs:8.26,0.758174629717926)
--(axis cs:8.01,0.756760093545005)
--(axis cs:7.76,0.7559349198746)
--(axis cs:7.51,0.753227388265017)
--(axis cs:7.26,0.75165546070386)
--(axis cs:7.01,0.747293806641287)
--(axis cs:6.76,0.746356237508471)
--(axis cs:6.51,0.743794066820183)
--(axis cs:6.26,0.736234257096235)
--(axis cs:6.01,0.737530426584835)
--(axis cs:5.76,0.730396616349933)
--(axis cs:5.51,0.727634653927906)
--(axis cs:5.26,0.717530030689561)
--(axis cs:5.01,0.713915993879832)
--(axis cs:4.76,0.706023852505281)
--(axis cs:4.51,0.696708973793508)
--(axis cs:4.26,0.685670704115931)
--(axis cs:4.01,0.673907679646658)
--(axis cs:3.76,0.659192143674551)
--(axis cs:3.51,0.639199315891539)
--(axis cs:3.26,0.621062735897762)
--(axis cs:3.01,0.599305912698838)
--(axis cs:2.76,0.573997389495599)
--(axis cs:2.51,0.546769339465342)
--(axis cs:2.26,0.511861494806614)
--(axis cs:2.01,0.475827333151003)
--(axis cs:1.76,0.433664677409598)
--(axis cs:1.51,0.394843218950651)
--(axis cs:1.26,0.350536266614864)
--(axis cs:1.01,0.285046168044849)
--(axis cs:0.76,0.244965513517503)
--(axis cs:0.51,0.131377932049631)
--(axis cs:0.26,0.108282934693773)
--(axis cs:0.01,0.100000001490116)
--cycle;

\path [draw=white, fill=color4, opacity=0.3] (axis cs:0.01,0.100000001490116)
--(axis cs:0.01,0.100000001490116)
--(axis cs:0.26,0.153370320816462)
--(axis cs:0.51,0.348642033247212)
--(axis cs:0.76,0.469961916645781)
--(axis cs:1.01,0.561831202133071)
--(axis cs:1.26,0.619711722160963)
--(axis cs:1.51,0.659492604931096)
--(axis cs:1.76,0.683731179007438)
--(axis cs:2.01,0.701850851254803)
--(axis cs:2.26,0.711019753538625)
--(axis cs:2.51,0.720651603094563)
--(axis cs:2.76,0.72653561979008)
--(axis cs:3.01,0.732619900123202)
--(axis cs:3.26,0.740428631537493)
--(axis cs:3.51,0.740298267179334)
--(axis cs:3.76,0.740649762708614)
--(axis cs:4.01,0.747805865860788)
--(axis cs:4.26,0.747339811440758)
--(axis cs:4.51,0.74873422280364)
--(axis cs:4.76,0.752653255320423)
--(axis cs:5.01,0.753281106192487)
--(axis cs:5.26,0.754450044359999)
--(axis cs:5.51,0.759223118806347)
--(axis cs:5.76,0.758100828720073)
--(axis cs:6.01,0.762808867144601)
--(axis cs:6.26,0.760809257389597)
--(axis cs:6.51,0.760157301654655)
--(axis cs:6.76,0.761701635843436)
--(axis cs:7.01,0.763549110264656)
--(axis cs:7.26,0.766685763933253)
--(axis cs:7.51,0.765844250102989)
--(axis cs:7.76,0.761498707349524)
--(axis cs:8.01,0.765196496230003)
--(axis cs:8.26,0.766584354312115)
--(axis cs:8.51,0.770302937537586)
--(axis cs:8.76,0.768230471429888)
--(axis cs:9.01,0.767464763406538)
--(axis cs:9.26,0.774359065918548)
--(axis cs:9.51,0.771861201632745)
--(axis cs:9.76,0.77276710835717)
--(axis cs:10.01,0.773855410419844)
--(axis cs:10.26,0.773079709522326)
--(axis cs:10.51,0.774165034871641)
--(axis cs:10.76,0.767541975567759)
--(axis cs:11.01,0.771744163628646)
--(axis cs:11.26,0.774219367618164)
--(axis cs:11.51,0.776269118784656)
--(axis cs:11.76,0.769979456554505)
--(axis cs:12.01,0.760521960303543)
--(axis cs:12.26,0.77009172464892)
--(axis cs:12.51,0.768067885795034)
--(axis cs:12.76,0.77312994873242)
--(axis cs:13.01,0.772373331780757)
--(axis cs:13.26,0.776025022700394)
--(axis cs:13.51,0.775793825727983)
--(axis cs:13.76,0.772404423045442)
--(axis cs:14.01,0.77417647417655)
--(axis cs:14.26,0.776371980691637)
--(axis cs:14.51,0.775960980888116)
--(axis cs:14.76,0.777679072488395)
--(axis cs:15.01,0.775452732116576)
--(axis cs:15.26,0.778392355557575)
--(axis cs:15.51,0.777976696108691)
--(axis cs:15.76,0.77920059541921)
--(axis cs:16.01,0.77203763157026)
--(axis cs:16.26,0.774316094532906)
--(axis cs:16.51,0.774380480084511)
--(axis cs:16.76,0.776222854404095)
--(axis cs:17.01,0.777905098654842)
--(axis cs:17.26,0.777894023244936)
--(axis cs:17.51,0.774857293665751)
--(axis cs:17.76,0.776878863263394)
--(axis cs:18.01,0.779148222867709)
--(axis cs:18.26,0.778333732345531)
--(axis cs:18.51,0.777733930558353)
--(axis cs:18.76,0.779093143617908)
--(axis cs:19.01,0.77812443842609)
--(axis cs:19.26,0.779130291833204)
--(axis cs:19.51,0.777955423406959)
--(axis cs:19.76,0.778820695922849)
--(axis cs:19.76,0.782259294940951)
--(axis cs:19.76,0.782259294940951)
--(axis cs:19.51,0.782144559808372)
--(axis cs:19.26,0.78250969658919)
--(axis cs:19.01,0.783795572617458)
--(axis cs:18.76,0.782666864717206)
--(axis cs:18.51,0.782426062613339)
--(axis cs:18.26,0.783166268608143)
--(axis cs:18.01,0.782331786688108)
--(axis cs:17.76,0.783221131872867)
--(axis cs:17.51,0.784782696663992)
--(axis cs:17.26,0.783645984346311)
--(axis cs:17.01,0.783174904129887)
--(axis cs:16.76,0.78261714622533)
--(axis cs:16.51,0.785219515528587)
--(axis cs:16.26,0.784783925398887)
--(axis cs:16.01,0.783502381743033)
--(axis cs:15.76,0.782499414880472)
--(axis cs:15.51,0.782843299294599)
--(axis cs:15.26,0.782507641199932)
--(axis cs:15.01,0.782807267635469)
--(axis cs:14.76,0.783200920950326)
--(axis cs:14.51,0.781839002136481)
--(axis cs:14.26,0.782428013300214)
--(axis cs:14.01,0.781683544553614)
--(axis cs:13.76,0.781235573006346)
--(axis cs:13.51,0.782946170228438)
--(axis cs:13.26,0.783114968106185)
--(axis cs:13.01,0.783366667036687)
--(axis cs:12.76,0.78341005170627)
--(axis cs:12.51,0.783732138714396)
--(axis cs:12.26,0.784228265032324)
--(axis cs:12.01,0.78511804337764)
--(axis cs:11.76,0.783320554603485)
--(axis cs:11.51,0.783310885430584)
--(axis cs:11.26,0.782580640201965)
--(axis cs:11.01,0.781895808581284)
--(axis cs:10.76,0.78219802897173)
--(axis cs:10.51,0.781054937261996)
--(axis cs:10.26,0.782420297153395)
--(axis cs:10.01,0.77806458631859)
--(axis cs:9.76,0.779492885195991)
--(axis cs:9.51,0.778838795315497)
--(axis cs:9.26,0.779560940833466)
--(axis cs:9.01,0.776215236898637)
--(axis cs:8.76,0.775549533548292)
--(axis cs:8.51,0.774577083080852)
--(axis cs:8.26,0.776795655815907)
--(axis cs:8.01,0.773563510197762)
--(axis cs:7.76,0.775281292383447)
--(axis cs:7.51,0.771975743392952)
--(axis cs:7.26,0.774214239976812)
--(axis cs:7.01,0.77295087876809)
--(axis cs:6.76,0.768158359998544)
--(axis cs:6.51,0.767542717228097)
--(axis cs:6.26,0.771030733226247)
--(axis cs:6.01,0.768191127133353)
--(axis cs:5.76,0.763959196018239)
--(axis cs:5.51,0.765416892981068)
--(axis cs:5.26,0.766109969887895)
--(axis cs:5.01,0.762458883088213)
--(axis cs:4.76,0.759126756333477)
--(axis cs:4.51,0.757825769032908)
--(axis cs:4.26,0.755940184477516)
--(axis cs:4.01,0.755394152545126)
--(axis cs:3.76,0.748610255163243)
--(axis cs:3.51,0.745681730932391)
--(axis cs:3.26,0.744591374165479)
--(axis cs:3.01,0.741960113628781)
--(axis cs:2.76,0.733184395735738)
--(axis cs:2.51,0.728288399823681)
--(axis cs:2.26,0.717580247796519)
--(axis cs:2.01,0.706589142603535)
--(axis cs:1.76,0.693068809739205)
--(axis cs:1.51,0.669527394572994)
--(axis cs:1.26,0.63662825701079)
--(axis cs:1.01,0.587688789741624)
--(axis cs:0.76,0.510638088742479)
--(axis cs:0.51,0.397177964540263)
--(axis cs:0.26,0.239549680034693)
--(axis cs:0.01,0.100000001490116)
--cycle;

\addplot [, color0]
table [row sep=\\]{%
0.002	0.100000001490116 \\
0.252	0.100000001490116 \\
0.502	0.100000001490116 \\
0.752	0.100000001490116 \\
1.002	0.100000001490116 \\
1.252	0.100000001490116 \\
1.502	0.100000001490116 \\
1.752	0.100000001490116 \\
2.002	0.100000001490116 \\
2.252	0.100000001490116 \\
2.502	0.100000001490116 \\
2.752	0.100000001490116 \\
3.002	0.100000001490116 \\
3.252	0.100000001490116 \\
3.502	0.100000001490116 \\
3.752	0.100000001490116 \\
4.002	0.100000001490116 \\
4.252	0.100000001490116 \\
4.502	0.100040001422167 \\
4.752	0.100000001490116 \\
5.002	0.100000001490116 \\
5.252	0.100000001490116 \\
5.502	0.0995600014925003 \\
5.752	0.100000001490116 \\
6.002	0.100000001490116 \\
6.252	0.100000001490116 \\
6.502	0.100000001490116 \\
6.752	0.101220001280308 \\
7.002	0.100000001490116 \\
7.252	0.105190001428127 \\
7.502	0.108520001918077 \\
7.752	0.108550001680851 \\
8.002	0.115350000560284 \\
8.252	0.124250000715256 \\
8.502	0.124250000715256 \\
8.752	0.12848000228405 \\
9.002	0.127500000596046 \\
9.252	0.137910001724958 \\
9.502	0.145600001513958 \\
9.752	0.150460001081228 \\
10.002	0.165420001745224 \\
10.252	0.1733000010252 \\
10.502	0.171270000934601 \\
10.752	0.188769996911287 \\
11.002	0.185119999945164 \\
11.252	0.194250003993511 \\
11.502	0.197080001980066 \\
11.752	0.19522999972105 \\
12.002	0.206579999625683 \\
12.252	0.209349999576807 \\
12.502	0.214870001375675 \\
12.752	0.21421000212431 \\
13.002	0.21272000297904 \\
13.252	0.219360001385212 \\
13.502	0.22762999907136 \\
13.752	0.237589999288321 \\
14.002	0.234490001946688 \\
14.252	0.240670001879334 \\
14.502	0.251069997996092 \\
14.752	0.262520004063845 \\
15.002	0.260929999500513 \\
15.252	0.270389997959137 \\
15.502	0.267359996587038 \\
15.752	0.269900000095367 \\
16.002	0.279940000921488 \\
16.252	0.287409998476505 \\
16.502	0.298950002342462 \\
16.752	0.304930002987385 \\
17.002	0.321360003948212 \\
17.252	0.31832999587059 \\
17.502	0.331860002875328 \\
17.752	0.33264000415802 \\
18.002	0.335180002450943 \\
18.252	0.34473999813199 \\
18.502	0.346439999341965 \\
18.752	0.361829996109009 \\
19.002	0.373619998991489 \\
19.252	0.378240004181862 \\
19.502	0.394560001790523 \\
19.752	0.396039992570877 \\
};
\addplot [, color1]
table [row sep=\\]{%
0.002	0.100000001490116 \\
0.252	0.228630001842976 \\
0.502	0.290149998664856 \\
0.752	0.339290001988411 \\
1.002	0.381539997458458 \\
1.252	0.40067999958992 \\
1.502	0.421230000257492 \\
1.752	0.435830000042915 \\
2.002	0.452099999785423 \\
2.252	0.476780000329018 \\
2.502	0.48822999894619 \\
2.752	0.501690003275871 \\
3.002	0.512609997391701 \\
3.252	0.524880003929138 \\
3.502	0.535289999842644 \\
3.752	0.545829999446869 \\
4.002	0.548949998617172 \\
4.252	0.558300012350082 \\
4.502	0.568790000677109 \\
4.752	0.575990003347397 \\
5.002	0.588030004501343 \\
5.252	0.59258000254631 \\
5.502	0.601269996166229 \\
5.752	0.603930002450943 \\
6.002	0.60722000002861 \\
6.252	0.620769995450974 \\
6.502	0.627179998159409 \\
6.752	0.626060003042221 \\
7.002	0.630509996414185 \\
7.252	0.633400005102158 \\
7.502	0.641720002889633 \\
7.752	0.64757000207901 \\
8.002	0.648030000925064 \\
8.252	0.650050008296967 \\
8.502	0.659170001745224 \\
8.752	0.659329998493195 \\
9.002	0.670070004463196 \\
9.252	0.666650003194809 \\
9.502	0.674940013885498 \\
9.752	0.672430002689362 \\
10.002	0.677910006046295 \\
10.252	0.684890002012253 \\
10.502	0.684380000829697 \\
10.752	0.684229999780655 \\
11.002	0.687559992074966 \\
11.252	0.689330005645752 \\
11.502	0.695430010557175 \\
11.752	0.697640001773834 \\
12.002	0.7005499958992 \\
12.252	0.702199995517731 \\
12.502	0.700699996948242 \\
12.752	0.707400006055832 \\
13.002	0.704409998655319 \\
13.252	0.711690002679825 \\
13.502	0.709109997749329 \\
13.752	0.706769996881485 \\
14.002	0.715730005502701 \\
14.252	0.715829992294312 \\
14.502	0.716840004920959 \\
14.752	0.717059999704361 \\
15.002	0.720179998874664 \\
15.252	0.718999993801117 \\
15.502	0.718320000171661 \\
15.752	0.720790004730225 \\
16.002	0.721809995174408 \\
16.252	0.72318000793457 \\
16.502	0.726099985837936 \\
16.752	0.725049996376038 \\
17.002	0.721220004558563 \\
17.252	0.723009991645813 \\
17.502	0.723290002346039 \\
17.752	0.72953000664711 \\
18.002	0.725959998369217 \\
18.252	0.729109996557236 \\
18.502	0.72833998799324 \\
18.752	0.729369992017746 \\
19.002	0.732090002298355 \\
19.252	0.731440007686615 \\
19.502	0.72853000164032 \\
19.752	0.72975001335144 \\
};
\addplot [, color2]
table [row sep=\\]{%
0.01	0.100000001490116 \\
0.26	0.29050999879837 \\
0.51	0.457390007376671 \\
0.76	0.55275000333786 \\
1.01	0.616000002622604 \\
1.26	0.655940002202988 \\
1.51	0.683619993925095 \\
1.76	0.700310003757477 \\
2.01	0.714950007200241 \\
2.26	0.726040005683899 \\
2.51	0.733020001649857 \\
2.76	0.739669996500015 \\
3.01	0.744960004091263 \\
3.26	0.74986999630928 \\
3.51	0.753110009431839 \\
3.76	0.758589994907379 \\
4.01	0.760729998350143 \\
4.26	0.762220007181168 \\
4.51	0.76285999417305 \\
4.76	0.765690004825592 \\
5.01	0.76997999548912 \\
5.26	0.767669999599457 \\
5.51	0.76904000043869 \\
5.76	0.769589990377426 \\
6.01	0.771439999341965 \\
6.26	0.772100001573563 \\
6.51	0.772479999065399 \\
6.76	0.772619992494583 \\
7.01	0.773259997367859 \\
7.26	0.773289996385574 \\
7.51	0.772909992933273 \\
7.76	0.773460000753403 \\
8.01	0.774030005931854 \\
8.26	0.774000012874603 \\
8.51	0.771979999542236 \\
8.76	0.773949998617172 \\
9.01	0.775139999389648 \\
9.26	0.774590009450912 \\
9.51	0.775479996204376 \\
9.76	0.772980004549026 \\
10.01	0.77409999370575 \\
10.26	0.775149995088577 \\
10.51	0.774029994010925 \\
10.76	0.774460005760193 \\
11.01	0.774900007247925 \\
11.26	0.774280005693436 \\
11.51	0.773940002918243 \\
11.76	0.771719998121262 \\
12.01	0.774499994516373 \\
12.26	0.775449991226196 \\
12.51	0.773539996147156 \\
12.76	0.772289997339249 \\
13.01	0.773780000209808 \\
13.26	0.774909991025925 \\
13.51	0.772909992933273 \\
13.76	0.771850007772446 \\
14.01	0.772550004720688 \\
14.26	0.773100000619888 \\
14.51	0.773759990930557 \\
14.76	0.771260011196137 \\
15.01	0.772340005636215 \\
15.26	0.773280000686645 \\
15.51	0.773310005664825 \\
15.76	0.771599996089935 \\
16.01	0.771339994668961 \\
16.26	0.773390007019043 \\
16.51	0.773730003833771 \\
16.76	0.771259999275207 \\
17.01	0.772240006923676 \\
17.26	0.772079998254776 \\
17.51	0.771579992771149 \\
17.76	0.77172999382019 \\
18.01	0.77243999838829 \\
18.26	0.772840005159378 \\
18.51	0.772059988975525 \\
18.76	0.771049988269806 \\
19.01	0.771409994363785 \\
19.26	0.772069996595383 \\
19.51	0.772849994897842 \\
19.76	0.771149998903275 \\
};
\addplot [, color3, dashed]
table [row sep=\\]{%
0.01	0.100000001490116 \\
0.26	0.101990001648664 \\
0.51	0.109940001368523 \\
0.76	0.194050000607967 \\
1.01	0.272170001268387 \\
1.26	0.331549999117851 \\
1.51	0.382989999651909 \\
1.76	0.422119998931885 \\
2.01	0.463869997859001 \\
2.26	0.498210000991821 \\
2.51	0.535799992084503 \\
2.76	0.564079988002777 \\
3.01	0.590100002288818 \\
3.26	0.611219996213913 \\
3.51	0.630359989404678 \\
3.76	0.650499999523163 \\
4.01	0.666429996490479 \\
4.26	0.679329997301102 \\
4.51	0.691539996862412 \\
4.76	0.701349991559982 \\
5.01	0.708390003442764 \\
5.26	0.712619996070862 \\
5.51	0.721399992704391 \\
5.76	0.725299996137619 \\
6.01	0.732410007715225 \\
6.26	0.733059996366501 \\
6.51	0.738719999790192 \\
6.76	0.742170000076294 \\
7.01	0.743279993534088 \\
7.26	0.746109998226166 \\
7.51	0.74906000494957 \\
7.76	0.752289998531342 \\
8.01	0.753750014305115 \\
8.26	0.753859996795654 \\
8.51	0.757230007648468 \\
8.76	0.759069991111755 \\
9.01	0.761200004816055 \\
9.26	0.761379992961884 \\
9.51	0.762420004606247 \\
9.76	0.762330001592636 \\
10.01	0.763580000400543 \\
10.26	0.764670014381409 \\
10.51	0.766419994831085 \\
10.76	0.763899999856949 \\
11.01	0.766330009698868 \\
11.26	0.767050004005432 \\
11.51	0.766850006580353 \\
11.76	0.767279994487762 \\
12.01	0.769509994983673 \\
12.26	0.768440002202988 \\
12.51	0.767820000648499 \\
12.76	0.769409996271133 \\
13.01	0.7714399933815 \\
13.26	0.771899998188019 \\
13.51	0.769609999656677 \\
13.76	0.770290011167526 \\
14.01	0.76991999745369 \\
14.26	0.771830004453659 \\
14.51	0.771830004453659 \\
14.76	0.771930003166199 \\
15.01	0.771560007333756 \\
15.26	0.771419990062714 \\
15.51	0.771520000696182 \\
15.76	0.771149998903275 \\
16.01	0.77114999294281 \\
16.26	0.772539991140366 \\
16.51	0.772359997034073 \\
16.76	0.770600003004074 \\
17.01	0.771850001811981 \\
17.26	0.773340004682541 \\
17.51	0.772409999370575 \\
17.76	0.771649998426437 \\
18.01	0.771780002117157 \\
18.26	0.771630001068115 \\
18.51	0.770879995822906 \\
18.76	0.773139995336533 \\
19.01	0.77211999297142 \\
19.26	0.773129999637604 \\
19.51	0.771450006961823 \\
19.76	0.77247000336647 \\
};
\addplot [, color4, dash pattern=on 1pt off 3pt on 3pt off 3pt]
table [row sep=\\]{%
0.01	0.100000001490116 \\
0.26	0.196460000425577 \\
0.51	0.372909998893738 \\
0.76	0.49030000269413 \\
1.01	0.574759995937347 \\
1.26	0.628169989585876 \\
1.51	0.664509999752045 \\
1.76	0.688399994373322 \\
2.01	0.704219996929169 \\
2.26	0.714300000667572 \\
2.51	0.724470001459122 \\
2.76	0.729860007762909 \\
3.01	0.737290006875992 \\
3.26	0.742510002851486 \\
3.51	0.742989999055862 \\
3.76	0.744630008935928 \\
4.01	0.751600009202957 \\
4.26	0.751639997959137 \\
4.51	0.753279995918274 \\
4.76	0.75589000582695 \\
5.01	0.75786999464035 \\
5.26	0.760280007123947 \\
5.51	0.762320005893707 \\
5.76	0.761030012369156 \\
6.01	0.765499997138977 \\
6.26	0.765919995307922 \\
6.51	0.763850009441376 \\
6.76	0.76492999792099 \\
7.01	0.768249994516373 \\
7.26	0.770450001955032 \\
7.51	0.768909996747971 \\
7.76	0.768389999866486 \\
8.01	0.769380003213882 \\
8.26	0.771690005064011 \\
8.51	0.772440010309219 \\
8.76	0.77189000248909 \\
9.01	0.771840000152588 \\
9.26	0.776960003376007 \\
9.51	0.775349998474121 \\
9.76	0.776129996776581 \\
10.01	0.775959998369217 \\
10.26	0.77775000333786 \\
10.51	0.777609986066818 \\
10.76	0.774870002269745 \\
11.01	0.776819986104965 \\
11.26	0.778400003910065 \\
11.51	0.77979000210762 \\
11.76	0.776650005578995 \\
12.01	0.772820001840591 \\
12.26	0.777159994840622 \\
12.51	0.775900012254715 \\
12.76	0.778270000219345 \\
13.01	0.777869999408722 \\
13.26	0.77956999540329 \\
13.51	0.77936999797821 \\
13.76	0.776819998025894 \\
14.01	0.777930009365082 \\
14.26	0.779399996995926 \\
14.51	0.778899991512299 \\
14.76	0.78043999671936 \\
15.01	0.779129999876022 \\
15.26	0.780449998378754 \\
15.51	0.780409997701645 \\
15.76	0.780850005149841 \\
16.01	0.777770006656647 \\
16.26	0.779550009965897 \\
16.51	0.779799997806549 \\
16.76	0.779420000314712 \\
17.01	0.780540001392364 \\
17.26	0.780770003795624 \\
17.51	0.779819995164871 \\
17.76	0.780049997568131 \\
18.01	0.780740004777908 \\
18.26	0.780750000476837 \\
18.51	0.780079996585846 \\
18.76	0.780880004167557 \\
19.01	0.780960005521774 \\
19.26	0.780819994211197 \\
19.51	0.780049991607666 \\
19.76	0.7805399954319 \\
};
\end{axis}

\end{tikzpicture}
    \tikzexternaldisable
  \end{minipage}
  \HBPresetPGFStyle
  \vspace{-2ex}
  \caption{\textbf{Comparison of SGD, Adam, and Newton-style methods with
      different exact curvature matrix-vector products (HBP)}. The architecture
    is the DeepOBS 3c3d network \citep{schneider2019deepobs} with sigmoid
    activations (\Cref{hbp::subtable:modelArchitectures3}).}
  \label{hbp::fig:experiment_c3d3}
\end{figure*}

%%% Local Variables:
%%% mode: latex
%%% TeX-master: "../thesis"
%%% End:
