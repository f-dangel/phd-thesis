\subsection{The \gnn's Eigenvalues \& the Gram
  Matrix}\label{vivit::sec:relation-ggn-gram-eigenvalues}

For \Cref{vivit::eq:ggn-eigenvalues}, consider the left hand side of the \ggn's
characteristic polynomial $\det(\mG - \lambda \mI_D) = 0$. Inserting the \vivit
factorization (\Cref{vivit::eq:ggn-factorization}) and using the matrix determinant
lemma yields
\begin{align*}
  &\det\left(
  - \lambda \mI_D + \mG
  \right)
  \\
  &\qquad =
    \det\!\left(
    - \lambda \mI_D + \mV\mV^\top
    \right)
    \explainmath{(Low-rank structure (\ref{vivit::eq:ggn-factorization}))}
  \\
  &\qquad=
    \det\!\left(
    \mI_{NC} + \mV^\top (-\lambda \mI_D)^{-1} \mV
    \right)
    \det\!\left(
    -\lambda \mI_D
    \right)
    \explainmath{(Matrix determinant lemma)}
  \\
  &\qquad=
    \det\!\left(
    \mI_{NC} - \frac{1}{\lambda} \mV^\top \mV
    \right)
    (-\lambda)^D
    \explainmath{\empty}
  \\
  &\qquad=
    \left(-\frac{1}{\lambda}\right)^{NC}
    \det\!\left(
    \mV^\top \mV - \lambda \mI_{NC}
    \right)
    (-\lambda)^D
    \explainmath{\empty}
  \\
  &\qquad=
    (-\lambda)^{D - NC}
    \det\!\left(
    \mGtilde - \lambda \mI_{NC}
    \right)\,.
    \explainmath{(Gram matrix)}
\end{align*}
Setting the above expression to zero reveals that the \ggn's spectrum decomposes
into $D-NC$ zero eigenvalues and the Gram matrix spectrum obtained from
$\det(\mGtilde - \lambda \mI_{NC} )= 0$.

\subsection{Relation Between \ggn \& Gram Matrix Eigenvectors}
\label{vivit::sec:relation-ggn-gram-eigenvectors}

Assume the nontrivial Gram matrix spectrum $\tilde{\sS}_+ := \{(\lambda_k,
\vetilde_k)\:|\: \lambda_k \neq 0, \mGtilde \vetilde_k = \lambda_k \vetilde_k
\}_{k=1}^K$ with orthonormal eigenvectors $\vetilde_j^\top \vetilde_k =
\delta_{j,k}$ ($\delta$ is the Kronecker delta) and $K = \mathrm{rank}(\mG)$. We
now show that $\ve_k = \nicefrac{1}{\sqrt{\lambda_k}} \mV \vetilde_k$ are
normalized eigenvectors of $\mG$ and inherit orthogonality from $\vetilde_k$.

To see the first, consider right-multiplication of the \ggn with $\ve_k$, then
expand the low-rank structure,
\begin{align*}
  \mG \ve_k
  &=
    \frac{1}{\sqrt{\lambda_k}}
    \mV \mV^\top  \mV \vetilde_k
    \explainmath{(\Cref{vivit::eq:ggn-factorization} and definition of $\ve_k$)}
  \\
  &=
    \frac{1}{\sqrt{\lambda_k}}
    \mV \mGtilde \vetilde_k
    \explainmath{(Gram matrix)}
  \\
  &=
    \lambda_k\frac{1}{\sqrt{\lambda_k}}
    \mV \vetilde_k
    \explainmath{(Eigenvector property of $\vetilde_k$)}
  \\
  &= \lambda_k \ve_k\,.
\end{align*}
Orthonormality of the $\ve_k$ results from the Gram matrix eigenvector
orthonormality,
\begin{align*}
  \ve_j^\top \ve_{k}
  &=
    \left(
    \frac{1}{\sqrt{\lambda_j}} \vetilde_j^\top \mV ^\top
    \right)
    \left(
    \frac{1}{\sqrt{\lambda_k}} \mV \vetilde_k
    \right)
    \explainmath{(Definition of $\ve_j, \ve_k$)}
  \\
  &=
    \frac{1}{\sqrt{\lambda_j\lambda_k}} \vetilde_j^\top \mGtilde \vetilde_k
    \explainmath{(Gram matrix)}
  \\
  &=
    \frac{\lambda_k}{\sqrt{\lambda_j\lambda_k}} \vetilde_j^\top \vetilde_k
    \explainmath{(Eigenvector property of $\vetilde_k$)}
  \\
  &=
    \delta_{j,k}\,.
    \explainmath{(Orthonormality)}
\end{align*}

%%% Local Variables:
%%% mode: latex
%%% TeX-master: "../thesis"
%%% End:
