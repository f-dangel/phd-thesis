\begin{figure*}[!t]
  \centering
  \begin{subfigure}[t]{1.0\linewidth}
    \caption{One datum}\label{subfig:background::gradientBackpropagation1}
    \centering\resizebox{\linewidth}{!}{
      {\footnotesize
        % basic setting of a fully-connected neural network with data flow for
% forward pass

\begin{tikzpicture}
  % first two layers
  \node (in1)
  [inner sep=0]
  {\tikz \drawMessagesWithArrows{$\vz^{(0)}$}{ }{ }{\hNodeDistance};};
  \node (layer1)
  [anchor=south west, inner sep=0]
  at (in1.south east)
  {\tikz \drawModuleWithParams{$f^{(1)}_{\vtheta^{(1)}}$}{16}{$\vtheta^{(1)}$}{$\grad{\vtheta^{(1)}}\ell$}{ };};
  \node (out1)
  [inner sep=0, anchor=south west]
  at (layer1.south east)
  {\tikz \drawMessagesWithArrows{$\vz^{(1)}$}{$\grad{\vz^{(1)}}\ell$}{ }{\hNodeDistance};};
  \node (layer2)
  [inner sep=0pt, anchor=south west]
  at (out1.south east)
  {\tikz \drawModuleWithParams{$f^{(2)}_{\vtheta^{(2)}}$}{16}{$\vtheta^{(2)}$}{$\grad{\vtheta^{(2)}}\ell$}{ };};

  % dots with messages
  \node (in2)
  [inner sep=0, anchor=south west]
  at (layer2.south east)
  {\tikz \drawMessagesWithArrows{$\vz^{(2)}$}{$\grad{\vz^{(2)}}\ell$}{ }{\hNodeDistance};};
  \node (dots)
  [xshift=0.75ex, inner sep=0pt, anchor=west]
  at (in2.east)
  {$\dots$};

  \node (inLast)
  [xshift=0.75ex, inner sep=0pt, anchor=west]
  at (dots.east)
  {\tikz \drawMessagesWithArrows{$\vz^{(L-1)}$}{$\grad{\vz^{(L-1)}}\ell$}{ }{\hNodeDistance};};

  \node (layerLast)
  [anchor=south west, inner sep=0]
  at (inLast.south east)
  {\tikz \drawModuleWithParams{$f^{(L)}_{\vtheta^{(L)}}$}{16}{$\vtheta^{(L)}$}{$\grad{\vtheta^{(L)}}\ell$}{ };};
  \node (outLast)
  [inner sep=0, anchor=south west]
  at (layerLast.south east)
  {\tikz \drawMessagesWithArrows{$\vz^{(L)}$}{$\grad{\vz^{(L)}}\ell$}{ }{\hNodeDistance};};

  % loss layer
  \node (lossLayer)
  [inner sep=0pt, anchor=south west]
  at (outLast.south east)
  {\tikz\drawModuleNoParams{$\ell$}{5};};
  \node (loss)
  [inner sep=0, anchor=south west]
  at (lossLayer.south east)
  {\tikz \drawMessagesWithArrows{$\ell$}{ }{ }{\hNodeDistance};};

  % label
  \node (label)
  [inner sep=0, anchor=south west, rotate=-90, xshift = -96, yshift = -50]
  at (lossLayer.south east)
  {\tikz \drawMessagesWithArrows{$\vy$}{ }{ }{\hNodeDistance};};
\end{tikzpicture}

%%% Local Variables:
%%% mode: latex
%%% TeX-master: "../../thesis"
%%% End:

      }}
  \end{subfigure}
  \begin{subfigure}[t]{1.0\linewidth}
    \caption{Batched data, arbitrary transformations}\label{subfig:background::gradientBackpropagation2}
    \centering\resizebox{\linewidth}{!}{
      {\footnotesize
        % basic setting of a fully-connected neural network with data flow for
% forward pass

\begin{tikzpicture}
  % first layer
  \node (out1)
  [inner sep=0, anchor=south west]
  {\tikz \drawMessagesWithArrows{$\mZ^{(0)}$}{ }{ }{\hNodeDistance};};
  \node (layer2)
  [inner sep=0pt, anchor=south west]
  at (out1.south east)
  {\tikz \drawModuleWithParams{$F^{(1)}_{\vtheta^{(1)}}$}{16}{$\vtheta^{(1)}$}{$\grad{\vtheta^{(1)}}\gL$}{ };};

  % dots with messages
  \node (in2)
  [inner sep=0, anchor=south west]
  at (layer2.south east)
  {\tikz \drawMessagesWithArrows{$\mZ^{(1)}$}{$\grad{\mZ^{(1)}}\gL$}{ }{\hNodeDistance};};
  \node (dots)
  [xshift=0.75ex, inner sep=0pt, anchor=west]
  at (in2.east)
  {$\dots$};

  \node (inLast)
  [xshift=0.75ex, inner sep=0pt, anchor=west]
  at (dots.east)
  {\tikz \drawMessagesWithArrows{$\mZ^{(L-1)}$}{$\grad{\mZ^{(L-1)}}\gL$}{ }{\hNodeDistance};};

  \node (layerLast)
  [anchor=south west, inner sep=0]
  at (inLast.south east)
  {\tikz \drawModuleWithParams{$F^{(L)}_{\vtheta^{(L)}}$}{16}{$\vtheta^{(L)}$}{$\grad{\vtheta^{(L)}}\gL$}{ };};
  \node (outLast)
  [inner sep=0, anchor=south west]
  at (layerLast.south east)
  {\tikz \drawMessagesWithArrows{$\mZ^{(L)}$}{$\grad{\mZ^{(L)}}\gL$}{ }{\hNodeDistance};};

  % loss layer
  \node (lossLayer)
  [inner sep=0pt, anchor=south west]
  at (outLast.south east)
  {\tikz\drawModuleNoParams{$\vell$}{5};};
  \node (losses)
  [inner sep=0, anchor=south west]
  at (lossLayer.south east)
  {\tikz \drawMessagesWithArrows{$\vell$}{$\grad{\vell}\gL$}{ }{\hNodeDistance};};

  % label
  \node (label)
  [inner sep=0, anchor=south west, rotate=-90, xshift = -96, yshift = -50]
  at (lossLayer.south east)
  {\tikz \drawMessagesWithArrows{$\mY$}{ }{ }{\hNodeDistance};};

  % Reduction layer
  \node (reductionLayer)
  [inner sep=0pt, anchor=south west]
  at (losses.south east)
  {\tikz\drawModuleNoParams{$\gL$}{5};};
  \node (reducedLoss)
  [inner sep=0, anchor=south west]
  at (reductionLayer.south east)
  {\tikz \drawMessagesWithArrows{$\gL$}{ }{ }{\hNodeDistance};};

\end{tikzpicture}

%%% Local Variables:
%%% mode: latex
%%% TeX-master: "../../thesis"
%%% End:

      }}
  \end{subfigure}
  \begin{subfigure}[t]{1.0\linewidth}
    \caption{Batched data, batched instructions}\label{subfig:background::gradientBackpropagation3}
    \centering\resizebox{\linewidth}{!}{
      {\footnotesize
        % basic setting of a fully-connected neural network with data flow for
% forward pass

\begin{tikzpicture}
  % first two layers
  % \node (in1)
  % [inner sep=0]
  % {\tikz \drawMessagesWithArrows{$\{\vz_{n}^{(0)}\}$}{ }{ }{\hNodeDistance};};
  % \node (layer1)
  % [anchor=south west, inner sep=0]
  % at (in1.south east)
  % {\tikz \drawModuleWithParams{$\vf^{(1)}_{\vtheta^{(1)}}$}{16}{$\vtheta^{(1)}$}{$\grad{\vtheta^{(1)}}\gL$}{ };};
  \node (out1)
  [inner sep=0, anchor=south west]
  % at (layer1.south east)
  {\tikz \drawMessagesWithArrows{\scalebox{0.8}{$\{\vz_{n}^{(0)}\}$}}{ }{ }{\hNodeDistance};};
  \node (layer2)
  [inner sep=0pt, anchor=south west]
  at (out1.south east)
  {\tikz \drawModuleWithParams{$\vf^{(1)}_{\vtheta^{(1)}}$}{16}{$\vtheta^{(1)}$}{$\grad{\vtheta^{(1)}}\gL$}{ };};

  % dots with messages
  \node (in2)
  [inner sep=0, anchor=south west]
  at (layer2.south east)
  {\tikz \drawMessagesWithArrows{\scalebox{0.8}{$\{\vz_{n}^{(2)}\}$}}{\scalebox{0.75}{$\{\grad{\vz_{n}^{(2)}}\gL\}$}}{ }{\hNodeDistance};};
  \node (dots)
  [xshift=0.75ex, inner sep=0pt, anchor=west]
  at (in2.east)
  {$\dots$};

  \node (inLast)
  [xshift=0.75ex, inner sep=0pt, anchor=west]
  at (dots.east)
  {\tikz \drawMessagesWithArrows{\,\scalebox{0.8}{$\{\vz_{n}^{(L-1)}\}$}\,}{\,\scalebox{0.75}{$\{\grad{\vz_{n}^{(L-1)}}\gL\}$}\,}{ }{\hNodeDistance};};

  \node (layerLast)
  [anchor=south west, inner sep=0]
  at (inLast.south east)
  {\tikz \drawModuleWithParams{$\vf^{(L)}_{\vtheta^{(L)}}$}{16}{$\vtheta^{(L)}$}{$\grad{\vtheta^{(L)}}\gL$}{ };};
  \node (outLast)
  [inner sep=0, anchor=south west]
  at (layerLast.south east)
  {\tikz \drawMessagesWithArrows{\scalebox{0.8}{$\{\vz_{n}^{(L)}\}$}}{\scalebox{0.75}{$\{\grad{\vz_{n}^{(L)}}\gL\}$}}{ }{\hNodeDistance};};

  % loss layer
  \node (lossLayer)
  [inner sep=0pt, anchor=south west]
  at (outLast.south east)
  {\tikz\drawModuleNoParams{$\vell$}{5};};
  \node (losses)
  [inner sep=0, anchor=south west]
  at (lossLayer.south east)
  {\tikz \drawMessagesWithArrows{\scalebox{1.0}{$\{\ell_n\}$}}{\scalebox{1.0}{$\{\grad{\ell_n}\gL\}$}}{ }{\hNodeDistance};};

  % label
  \node (label)
  [inner sep=0, anchor=south west, rotate=-90, xshift = -96, yshift = -50]
  at (lossLayer.south east)
  {\tikz \drawMessagesWithArrows{\scalebox{1.0}{$\{\vy_n\}$}}{ }{ }{\hNodeDistance};};

  % Reduction layer
  \node (reductionLayer)
  [inner sep=0pt, anchor=south west]
  at (losses.south east)
  {\tikz\drawModuleNoParams{$\gL$}{5};};
  \node (reducedLoss)
  [inner sep=0, anchor=south west]
  at (reductionLayer.south east)
  {\tikz \drawMessagesWithArrows{$\gL$}{ }{ }{\hNodeDistance};};

\end{tikzpicture}

%%% Local Variables:
%%% mode: latex
%%% TeX-master: "../../thesis"
%%% End:

      }}
  \end{subfigure}
  \caption{\textbf{Gradient backpropagation and (un)awareness of per-sample
      structure in many ML libraries.}
    \subfigref{subfig:background::gradientBackpropagation1} Computation graph of
    of a neural network's loss on a single datum $(\vx, \vy)$. Gradients are
    backpropagated through the graph as described by
    \Cref{def:background::GradientBackpropagation} to obtain
    $\grad{\vtheta}\ell$.
    \subfigref{subfig:background::gradientBackpropagation2} To exploit
    parallelism in the computations, multiple data are stacked into matrices
    $(\mX, \mY)$ which are then processed by a sequence of matrix-to-matrix
    functions $F_{\vtheta^{(l)}}^{(l)}$ into a batch of losses $\vell$, and
    reduced into a scalar $\gL$ via mean reduction. AD in popular ML libraries
    like \pytorch tracks variables on the level of batched tensors. Therefore,
    operations are allowed to build up dependencies between data---such that
    $\gL = \nicefrac{1}{|\sB|} \sum_{n} [\vell(\mX, \mY, \vtheta)]_n$ where each
    component of $\vell$ may depend on \emph{all} data (batch
    normalization~\cite{ioffe2015batch} is such a case)---without breaking
    gradient backpropagation. ML libraries implement VJPS for the
    matrix-to-matrix functions $F_{\vtheta^{(l)}}^{(l)}$. This loses structure
    for operations that treat inputs independently along the batch axis.
    \subfigref{subfig:background::gradientBackpropagation3} The empirical risk
    on a mini-batch (\Cref{eq:background::miniBatchRisk} is such a case: all
    operations in the graph process inputs independently and with the same
    instructions along the batch axis. The following connections to the single
    datum case \subfigref{subfig:background::gradientBackpropagation1} hold:
    $F_{\vtheta^{(l)}}^{(l)} \leftrightarrow \vf_{\vtheta^{(l)}}^{(l)} =
    \vmap(f_{\vtheta^{(l)}}^{(l)}), \vell = \vmap(\ell)$ with $\vmap$ from
    \Cref{def:background::vmap}. Due to the more general support of AD in ML
    libraries for graphs of the form
    \subfigref{subfig:background::gradientBackpropagation2}, their VJPs cannot
    be accessed per-sample.}\label{fig:background::gradientBackpropagation}
\end{figure*}

%%% Local Variables:
%%% mode: latex
%%% TeX-master: "../../thesis"
%%% End:
