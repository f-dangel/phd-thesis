Broadly speaking, ML seeks to find a well-performing algorithm for a given task
from ``experience''. This thesis considers supervised deep learning, where
``experience'' is given through annotated examples in form of a dataset. A datum
$(\vx, \vy)$ consists of \emph{input features} $\vx$ and \emph{targets}, or
\emph{labels}, $\vy$. The algorithm, or \emph{model}, is a deep neural network
(\Cref{sec:background::DeepNeuralNetworks}) that tries to predict $\vy$ from
$\vx$ and is selected by minimizing a performance criterion on the available
data (the empirical risk, \Cref{sec:background::SupervisedLearning}) using
optimization methods which rely on automatically computed derivatives
(\Cref{sec:background::GradientBackpropagation}). This chapter reviews these
components, highlighting their structure \wrt implementation in ML libraries.
For a broader introduction to deep learning, see \eg \cite{goodfellow2016deep}.

\section{Empirical Risk Minimization}\label{sec:background::SupervisedLearning}
``Learning'' is connected to optimization by the risk minimization paradigm. The
idea is to define a performance metric that assesses the quality of the model's
prediction. Then, learning happens by maximizing that performance metric.
Conversely, one can specify a metric for the prediction's error
(also referred to as \emph{risk}), and minimize the latter.


For example, an intuitive way to assess performance for a classification task is
\emph{accuracy}, the ratio of correct and total predictions (the error would be
the ratio of incorrect and total predictions). However, such direct performance
measures are hard to optimize with derivative-based methods\sidenote{\eg the
  accuracy on one datum is either 0 or 1, and hence its derivative vanishes
  everywhere it is defined.} and must be substituted by a surrogate function
that approximates the original performance measure, but is easier to optimize.
In the deep learning terminology, these surrogates are commonly referred to as
\emph{loss functions}. This section expands on risk minimization, its
characteristic properties in deep learning, and its probabilistic
interpretation.

\subsection{Notation \& Mathematical
  Details}\label{sec:background::empiricalRiskMinimization}

\subsubsection{Risk \& Empirical Risk}

Consider supervised learning with the goal to learn the functional relation
between inputs $\vx \in \sX$ and targets $\vy \in \sY$
(\Cref{fig:background::ModelLossFunctionSplit}). The mapping is described by a
model $f_{\vtheta}: \sX \to \sF$ with adjustable parameters $\vtheta \in
\sTheta$ that produces predictions $\vf := f_{\vtheta}(\vx) \in \sF$ for an
input $\vx$. A prediction's error is assessed through a convex loss function
$\ell: \sF \times \sY \to \sR$.

\begin{figure}[tb]
  \centering \tikzexternalenable
  \resizebox{!}{3.5cm}{% Model-loss function split with data flow
\input{figures/background/model_loss_split_command}

\begin{tikzpicture}
  \drawModelLossSplit{$\vx$}{$\vf$}{$\vy$}{$\ell(\vf, \vy)$}
\end{tikzpicture}

%%% Local Variables:
%%% mode: latex
%%% TeX-master: "../../thesis"
%%% End:
}
  \tikzexternaldisable
  \caption{\textbf{Components of supervised learning:} The goal is to infer
    parameters $\vtheta$ of a model $f_{\vtheta}$, that relates inputs $\vx$ to
    predictions $\vf$, by minimizing a loss function $\ell$ between the
    prediction $\vf$ and label $\vy$.}
  \label{fig:background::ModelLossFunctionSplit}
\end{figure}

\emph{Regression} (\Cref{ex:background::Regression}) and \emph{classification}
(\Cref{ex:background::Classification}) are two common tasks that will be used
frequently in later chapters. The remainder of this text sets $\sX = \sR^{M}$
and $\sF = \sR^{C}$, with input and prediction space dimensions $M, C$,
respectively ($C$ corresponds to the number of classes for classification).
Because the discussion focuses on neural networks as model, it sets $\sTheta =
\sR^{D}$ in what follows.

\begin{figure}[!b]
  \begin{example}[\textbf{Least squares regression \& square
      loss}]\label{ex:background::Regression}
    Regression associates features in $\sX = \sR^M$ with targets in $\sY =
    \sR^C$. A prediction in $\sF = \sR^{C}$ compares to its ground truth via the
    mean squared error\sidenote{%
      There exist different conventions for the normalization factor. This text
      adapts the implementation of
      \inlinecode{\href{https://pytorch.org/docs/1.11/generated/torch.nn.MSELoss.html\#torch.nn.MSELoss}{MSELoss}}
      (with \inlinecode{reduction="mean"} mode) in \pytorch for consistency with
      the code presented in later chapters. Normalizing by $\nicefrac{1}{C}$ is
      also close to what the name, mean squared error, suggests.}
    \begin{align}\label{eq:background::squareLoss}
      \ell(\vf, \vy)
      =
      \frac{1}{C} \sum_{c=1}^C (\evy_c - \evf_c)^2
      =
      \frac{1}{C} \lVert \vy - \vf  \rVert_{2}^{2}
    \end{align}
  \end{example}
  \begin{example}[\textbf{$C$-class classification \& softmax cross-entropy
      loss}]\label{ex:background::Classification}
    Classification assigns features in $\sX = \sR^{M}$ to classes in $\sY =
    \left\{ 1, \dots, C \right\}$ using a model to $\sF = \sR^C$. The softmax
    cross-entropy loss\sidenote{%
      Sometimes, the softmax is considered part of the model rather than the
      loss function. This text assigns it to the loss function, in line with the
      \pytorch implementation of
      \inlinecode{\href{https://pytorch.org/docs/1.11/generated/torch.nn.CrossEntropyLoss.html\#torch.nn.CrossEntropyLoss}{CrossEntropyLoss}},
      that combines softmax and cross-entropy, which is numerically more
      stable~\cite[][Chapter 4]{goodfellow2016deep}.}%
    maps the model's prediction to a probability distribution over classes, then
    uses cross-entropy to compare it with the ground truth,
    \begin{align}
      \label{eq:background::softmaxCrossEntropyLoss}
      \ell(\vf, y) = - \log([p(\vf)]_{y})
      =
      - {\log p(\vf)}^\top \onehot(y)
    \end{align}
    where $p(\vf) = \softmax(\vf)$ and $[\onehot(y)]_{c} = \delta_{c,y}$.
  \end{example}
\end{figure}

% \paragraph{Risk minimization:}
Assume that the population of input-target pairs,
\ie the \emph{data-generating process}, follows a distribution $\pdata(\vx,
\vy)$. A model's expected risk under this distribution is defined as
\begin{subequations}
  \begin{align}
    \label{eq:background::expectedRisk}
    \begin{split}
      \gL_{\pdata}(\vtheta)
      &:=
        \E_{\pdata(\vx, \vy)}
        \left[
        \ell(f_{\vtheta}(\vx), \vy)
        \right]
      \\
      &=
        \iint_{\sX,\sY}
        \ell(f_{\vtheta}(\vx), \vy)
        \pdata(\vx, \vy)
        \,\diff\vx\,\diff\vy\,.
    \end{split}
  \end{align}
  The incentive for a model to perform well is to achieve a small expected risk.
  Therefore, training a model is minimizing \Cref{eq:background::expectedRisk},
  \begin{equation}
    \label{eq:background::minimizeExpectedRisk}
    \minimize_{\vtheta} \gL_{\pdata}(\vtheta)\,.
  \end{equation}
\end{subequations}

% \paragraph{Empirical approximation:}
But in practice, \Cref{eq:background::minimizeExpectedRisk} is inaccessible
because the data-generating process $\pdata(\vx, \vy)$ is unknown. Instead, the
problem is approximated through an \iid dataset $\sD = \{ (\vx_n, \vy_n) \in \sX
\times \sY \}_n$ of labeled data collected from $\pdata(\vx,\vy)$. $\pdata(\vx,
\vy)$ is approximated by the empirical distribution
\begin{subequations}\label{eq:background::supervisedLearning}
  \begin{equation}
    \label{eq:background::empiricalDistribution}
    p_{\sD}(\vx, \vy)
    =
    \frac{1}{|\sD|}
    \sum_{(\vx_{n}, \vy_{n}) \in \sD}
    \delta(\vx - \vx_{n}) \delta(\vy - \vy_{n})
  \end{equation}
  implied by $\sD$. The model's empirical risk on $\sD$ follows from
  substituting $\pdata$ with $p_{\sD}$ in \Cref{eq:background::expectedRisk},
  which yields
  \begin{equation}
    \label{eq:background::empiricalRisk}
    \gL_{\sD}(\vtheta)
    :=
    \gL_{p_{\sD}}(\vtheta)
    =
    \frac{1}{|\sD|}
    \sum_{(\vx_{n}, \vy_{n}) \in \sD}
    \ell(f_{\vtheta}(\vx_{n}), \vy_{n})\,.
  \end{equation}
  In practice, learning happens by minimizing an empirical risk,
  \begin{equation}
    \label{eq:background::empiricalRiskMinimization}
    \minimize_{\vtheta} \gL_{\sD}(\vtheta)\,.
  \end{equation}
\end{subequations}

\subsubsection{Challenges for Optimization}

Training a model relies on optimization algorithms which are initialized at some
$\vtheta_{0}\in \sTheta$ and seek to iteratively improve the solution to
\Cref{eq:background::empiricalRiskMinimization}. At an iteration $t$, the
optimizer is given access to information about the objective at $\vtheta_t$,
like derivatives (\Cref{sec:background::GradientBackpropagation}), to deduct a
step $\vtheta_{t+1} \leftarrow \vtheta_t$, potentially using additional
information from past observations. In general, the more local information is
available, the more powerful a single update step can be. But richer information
is often more costly to compute. The large-scale nature of deep learning poses
challenges on the accessible information:
\begin{itemize}

\item \textbf{Big data:} deep learning datasets are often large because more
  data gives better approximations of the true distribution $\pdata$ via
  $p_{\sD}$, and thus better task performance. The simultaneous computation of
  all per-datum losses for $\gL_{\sD}(\vtheta)$ does not fit into memory (for
  many contemporary tasks, it is even infeasible to hold $\sD$ in
  memory\sidenote{%
    \Eg \imagenet~\cite{deng2009imagenet} consists of 1,281,167 training images.
    The Kaggle download requires roughly 166\,GB of memory.%
  }). Still, to obtain $\gL_{\sD}$, one could sequentially compute and reduce
  its summands on data chunks of manageable size. In practice, it is more common
  though to approximate $\gL_{\sD}$ on a randomly drawn small subset, a
  mini-batch, $\sB \subseteq \sD$ with $|\sB| \ll |\sD|$. While this avoids a
  computationally expensive full sweep over the data, subsampling introduces
  \emph{noise} in the loss (\Cref{sec:background::MiniBatching}). This noise is
  inherited by the computed information supplied to the optimizer. Algorithms
  must therefore take into account the stochastic nature of their observations.

\item \textbf{Very large models:} the parameter space dimension $D$ of DNNs
  usually exceeds the (already large) amount of data $|\sD|$, \ie $D \gg |\sD|$.
  It affects the complexity to store and compute information, such as
  derivatives of the loss \wrt $\vtheta$, and makes it more challenging to
  efficiently work with higher-order information\sidenote{%
    For example, interesting quantities for an optimizer are the loss
    landscape's local slope (gradient) and curvature (Hessian,
    \Cref{def:background::Hessian}). While the gradient is cheap to compute and
    store ($D$ elements), holding the $D\times D$ Hessian in memory is
    infeasible.%
  }. Therefore, working with higher-order information relies on implicit schemes
  (\eg matrix-free Hessian-vector products~\cite{pearlmutter1994fast}) or
  light-weight structured approximations (\eg through Kronecker
  products~\cite{martens2015optimizing}). Such approaches are usually technical
  and challenging to implement. They also add significant computational work
  that must be compensated for by an improved update step quality in optimizers.
\end{itemize}
In addition to these computational aspects, there exist other challenges:
\begin{itemize}
\item \textbf{Non-convexity:} although the loss function $\ell$ is convex \wrt
  the model's output $\vf$, convexity does not carry through to the model's
  parameters\sidenote[][-2.5\baselineskip]{Convexity of $\ell$ in $\vf$ is still
    useful to construct PSD approximations to the Hessian, see
    \Cref{sec:background::ggn}.}. This is because DNNs $f_{\vtheta}$ are highly
  non-linear, and therefore generally non-convex, in $\vtheta$. Local minima of
  convex functions are global minima, meaning that local improvement gets us
  closer to a global solution. But non-convex problems like
  \Cref{eq:background::empiricalRiskMinimization} have multiple local minima
  that need not be global. Through local improvements, an optimizer can arrive
  at one of these local minima, but with no path of improvement to a global
  solution. This depends on various aspects, such as the update rule,
  hyperparameters, initialization, \etc.

\item \textbf{Generalization:} learning is not pure optimization. While
  minimizing the empirical risk \Cref{eq:background::empiricalRiskMinimization}
  improves the model's performance on the collected data, the actual goal is to
  obtain good performance on new unseen data, \ie generalize well. To achieve
  generalization, it is crucial to prevent optimization to overfit specifics in
  the data.
  % train-test split
  It is common practice to split the data into three disjoint sets $\Dtrain$,
  $\Dvalid$, and $\Dtest$. The train set's empirical risk
  $\gL_{\Dtrain}(\vtheta)$ is minimized, and the validation loss
  $\gL_{\Dvalid}(\vtheta)$ serves to identify hyperparameters that lead to
  generalization on $\Dvalid$. The held-out examples in the test set $\Dtest$
  are used to assess generalization to new data.
  % data augmentation
  Another way to improve generalization is to use more data during training.
  Data augmentation~\cite{shorten2019survey} allows for cheap generation of new
  examples without collecting new data.
  % regularization
  Sometimes, it may be desirable to penalize model properties by adding a
  regularization term to the objective in
  \Cref{eq:background::empiricalRiskMinimization}.
\end{itemize}

\subsection{Batching \& Noise}\label{sec:background::MiniBatching}

Due to the large-scale nature of $\sD$ and $f_{\vtheta}$ in the empirical risk,
\Cref{eq:background::empiricalRisk} is usually stochastically approximated
through a mini-batch $\sB \subseteq \sD$, and assessed through a mini-batch loss
\begin{equation}
  \label{eq:background::miniBatchRisk}
  \gL_{\sB}(\vtheta)
  =
  \frac{1}{|\sB|}
  \sum_{(\vx_n, \vy_n) \in \sB}
  \ell(f_{\vtheta}(\vx_n), \vy_n)\,.
\end{equation}
In the following, per-sample predictions and losses will often be abbreviated as
$\vf_n := f_{\vtheta}(\vx_n)$ and $\ell_n := \ell(f_{\vtheta}(\vx_n), \vy_n)$.

\subsubsection{Batched Computations}

Evaluation of the mini-batch loss in \Cref{eq:background::miniBatchRisk} can be
parallelized (\Cref{subfig:background::BatchedComputation1}). All mini-batch
features $\{ \vx_n \}$ are mapped to predictions $\{ \vf_n \}$ by the \emph{same
  instructions $f_{\vtheta}$}, and compared in parallel with the labels $\{
\vy_n \}$, resulting in the per-sample losses $\{ \ell_n\}$. Hardware
accelerators can use this structure to achieve significant speed-up of
evaluating the loss, or its derivatives
(\Cref{sec:background::GradientBackpropagation}).

\begin{figure*}[t]
  \centering
  \begin{subfigure}[t]{0.495\linewidth}
    \centering
    \caption{Same instructions, multiple data}
    \label{subfig:background::BatchedComputation1}
    \tikzexternalenable
    \resizebox{!}{3.5cm}{\input{figures/background/model_loss_split_command}

\begin{tikzpicture}
  \foreach \x in {1,2,3} {
    \pgfmathsetmacro{\xshiftcm}{-0.2 * (\x - 1)}
    \pgfmathsetmacro{\yshiftcm}{-0.6 * (\x - 1)}
    \begin{scope}[xshift = \xshiftcm cm, yshift = \yshiftcm cm, opacity = 0.95]
      \drawModelLossSplit{$\vx_{\x}$}{$\vf_{\x}$}{$\vy_{\x}$}{$\ell_{\x}$}
    \end{scope}
   }
\end{tikzpicture}

%%% Local Variables:
%%% mode: latex
%%% TeX-master: "../../thesis"
%%% End:
}
    \tikzexternaldisable
  \end{subfigure}
  \hfill
  \begin{subfigure}[t]{0.495\linewidth}
    \centering
    \caption{Batched operations}
    \label{subfig:background::BatchedComputation2}
    \tikzexternalenable
    \resizebox{!}{3.5cm}{\input{figures/background/model_loss_split_command}

\begin{tikzpicture}
  \drawModelLossSplit{%
    $\left\{\vx_n\right\}$}{%
    $\left\{\vf_n\right\}$}{%
    $\left\{\vy_n\right\}$}{%
    $\left\{\ell_n\right\}$}
\end{tikzpicture}

%%% Local Variables:
%%% mode: latex
%%% TeX-master: "../../thesis"
%%% End:
}
    \tikzexternaldisable
  \end{subfigure}
  \caption{\textbf{Different visualizations of empirical risk computation
      graphs.} \subfigref{subfig:background::BatchedComputation1}, The empirical
    risk is a weighted sum of per-sample losses that are computed independently
    and with the same instructions, allowing for efficient parallelization.
    \subfigref{subfig:background::BatchedComputation2} Many ML libraries
    natively support batched operations for efficiency. In code, this batching
    is often assumed, but not explicitly expressed. The $\vmap$ concept allows
    to make batching explicit: $f_{\vtheta}$ and $\ell$ in
    \subfigref{subfig:background::BatchedComputation2} correspond to vectorized
    versions $\vmap(f)$ and $\vmap(\ell)$ from
    \subfigref{subfig:background::BatchedComputation1}.}
  \label{fig:background::BatchedComputation}
\end{figure*}

This single-instruction-multiple-data structure of the loss is often baked into
ML libraries. Many of their operations natively support batched behavior, \ie
accept stacked inputs and assume one, usually the first, axis to correspond to a
batch axis. The operation is then applied to all slices along the batch axis
(\Cref{subfig:background::BatchedComputation2}).

The concept of \inlinecode{map} in functional
programming~\cite{hughes1989functional} formalizes applying a function to a
collection of inputs, which can be seen as a function transformation. Batched
operations can be understood as transformations of the original operation.
Recently developed ML libraries~\cite{bradbury2018jax,he2021functorch} make
batching explicit by providing a $\inlinecode{vmap}$ interface to automatically
vectorize, \ie parallelize, the application of \inlinecode{map}.

\begin{definition}[\textbf{vmap}]\label{def:background::vmap} Let $f: \sX \to
  \sY, x \mapsto f(x)$ denote a function. The vectorized \inlinecode{map},
  $\vmap(f)$, of $f$ \wrt its argument $x$ is a function that accepts a
  collection of inputs and maps each item by $f$, resulting in a collection of
  outputs,
  \begin{align*}
    \vmap:\quad (\sX \to \sY)
    \to (\sX^{N} \to \sY^{N})
    \quad \forall\,N \in \sN
    \shortintertext{such that}
    \vmap(f)(\{ x_1, x_2, \dots, x_N\})
    =
    \{f(x_1), f(x_2), \dots, f(x_N)\}\,.
  \end{align*}
  Collections are often abbreviated as $\{x_n\} := \{ x_1, x_2, \dots, x_N\}$
  and $\{f(x_n)\} := \{f(x_1), f(x_2), \dots, f(x_N)\}$. Multi-variate functions
  can be mapped \wrt to a subset of arguments.
\end{definition}

Making batching explicit with \inlinecode{map}, the mini-batch loss computation
uses a vectorized model $\vmap(f_{\vtheta})$ \wrt the input features
$\vx$, such that
\begin{subequations}
  \begin{align}
    \vmap(f_{\vtheta})(\{ \vx_n\}) &= \{ \vf_{n}(\vtheta) \}\,.
  \end{align}
  Per-sample losses, which are reduced into the mini-batch loss, are
  \begin{align}
    \{ \ell_n(\vtheta) \}
    &=
      \vmap(\ell)(\{ \vf_{n}(\vtheta) \}, \{ \vy_n \})\,.
  \end{align}
\end{subequations}
Numerically, collections of vectors like $\{\vx_n\}$ \etc are represented as
matrices, and more generally, collections of $r$-dimensional arrays are stacked
into a $(r+1)$-dimensional arrays with an additional batch axis (\eg, $\{ \vx_n
\}$ is a $|\sB| \times M$ matrix). These arrays can then be efficiently
processed in hardware accelerators.

\subsubsection{Noise}

While
%
\marginnote[*-12]{
  \begin{center}
    \tikzexternalenable
    \resizebox{\linewidth}{!}{% #1 input
% #2 output
% #3 label
% #4 loss
\newcommand{\drawModelLossSplit}[4]{
  \node (in)
  [inner sep=0]
  {\tikz \drawMessagesWithArrows{#1}{ }{ }{\hNodeDistance};};
  \node (layer1)
  [anchor=south west, inner sep=0]
  at (in.south east)
  {\tikz \drawModuleWithParams{$f_{\vtheta}$}{16}{$\vtheta$}{ }{ };};
  \node (out)
  [inner sep=0, anchor=south west]
  at (layer1.south east)
  {\tikz \drawMessagesWithArrows{#2}{ }{ }{\hNodeDistance};};
  % loss layer
  \node (lossLayer)
  [inner sep=0pt, anchor=south west]
  at (out.south east)
  {\tikz\drawModuleNoParams{$\ell$}{5};};
  \node (loss)
  [inner sep=0, anchor=south west]
  at (lossLayer.south east)
  {\tikz \drawMessagesWithArrows{#4}{ }{ }{\hNodeDistance};};
  \node (label)
  [inner sep=0, anchor=south west, rotate=-90, xshift = -96, yshift = -50]
  at (lossLayer.south east)
  {\tikz \drawMessagesWithArrows{#3}{ }{ }{\hNodeDistance};};
}
%%% Local Variables:
%%% mode: latex
%%% TeX-master: "../../thesis"
%%% End:


\begin{tikzpicture}
  \pgfmathsetseed{42}
  \foreach \x in {1,2,3,4,5} {
    \pgfmathsetmacro{\xshiftcm}{-0.2 * (\x - 1)}
    \pgfmathsetmacro{\yshiftcm}{-0.6 * (\x - 1)}
    \pgfmathrandominteger{\opacitySwitch}{0}{10}
    \ifthenelse{\opacitySwitch > 6}{%
      \pgfmathsetmacro{\opacity}{0.15}}{%
      \pgfmathsetmacro{\opacity}{0.95}}
    % \node [opacity=\opacity] at (\x, 0) {\opacitySwitch};
    \begin{scope}[xshift = \xshiftcm cm, yshift = \yshiftcm cm, opacity = \opacity]
      \drawModelLossSplit{$\vx_{\x}$}{$\vf_{\x}$}{$\vy_{\x}$}{$\ell_{\x}$}
    \end{scope}
  }
\end{tikzpicture}

%%% Local Variables:
%%% mode: latex
%%% TeX-master: "../../thesis"
%%% End:
}
    \tikzexternaldisable
  \end{center}
  \captionof{figure}{\textbf{Illustration of stochastic sub-sampling:} The
    computation graph of the empirical risk (with five data in this example) is
    only evaluated on a subset of data (two in this example) to save
    computations. While this preserves parallelization, the transparent parts
    are not evaluated, which introduces noise.}\label{fig:background::MiniBatch}
}%
the sum structure in the loss \Cref{eq:background::empiricalRisk} can be
efficiently parallelized, it can also be used for stochastic approximation via
sub-sampling (\Cref{fig:background::MiniBatch}). The mini-batch loss $\gL_\sB$
in \Cref{eq:background::miniBatchRisk} is an estimator of the empirical risk
$\gL_\sD$ implied by the stochastic sampling procedure of $\sB$. Increasing the
batch size makes this estimator more precise but more costly to compute.


% Batch size = 1
To see this cost-accuracy trade-off, consider the loss of a single datum
$\ell_n$ where $n$ is uniformly drawn from $\{1, \dots, |\sD|\}$.
Then, $\ell_n$ is a random variable, implied by the sampling distribution $n \sim
p(n) = \gU(\{1, \dots, |\sD|\})$, and an unbiased estimator of the empirical risk,
\begin{subequations}
  \begin{align}\label{eq:background::unbiasedEstimator}
    \E_{p(n)}[\ell_n(\vtheta)]
    &=
      \sum_{n=1}^{|\sD|} p(n) \ell_n(\vtheta)
      =
      \gL_{\sD}(\vtheta)\,,
  \end{align}
  with variance
  \begin{align}
    \label{eq:background::unbiasedEstimatorVariance}
    \begin{split}
      \sigma^{2}
      :=
      \Var_{p(n)}[\ell_n(\vtheta)]
      &=
        \E_{p(n)}\left[(\ell_n(\vtheta) - \gL_{\sD}(\vtheta))^2\right]
      \\
      &=
        \E_{p(n)}\left[\ell_n(\vtheta)^2\right]
        -
        \gL_{\sD}(\vtheta)^2\,.
    \end{split}
  \end{align}
\end{subequations}
% Batch size B
Next, consider a batch with $\sB|$ randomly drawn samples $\vn = (n_1, \dots,
n_{|\sB|})$ from a joint distribution $p(\vn)$, \ie the mean of $\ell_{n_1},
\dots, \ell_{n_{|\sB|}}$,
\begin{subequations}
  \begin{align}
    \gL_{\vn}(\vtheta)
    &=
      \frac{1}{|\sB|}
      \sum_{i=1}^{|\sB|} \ell_{n_i}(\vtheta)\,.
  \end{align}
  If samples are drawn uniformly \iid ($p(\vn) = \prod_{i=1}^{|\sB|} p(n_i)$
  with $p(n_i) = \gU(\{1, \dots, |\sD|\})$), this estimator is also unbiased,
  \begin{align}
    \label{eq:background::unbiasedBatchEstimator}
    \E_{p(\vn)}[\gL_\vn(\vtheta)]
    =
    \frac{1}{|\sB|}
    \sum_{i=1}^{|\sB|} \E_{p(n_i)}[\ell_{n_i}(\vtheta)]
    =
    \gL_{\sD}(\vtheta)\,,
  \end{align}
  but has smaller variance than the single-sample estimator:
  \begin{align}
    \label{eq:background::varianceBatchEstimator}
    \Var_{p(\vn)}[\gL_{\vn}(\vtheta)]
    =
    \E_{p(\vn)}[\gL_\vn(\vtheta)^2] - \gL_{\sD}(\vtheta)^2
    =
    \frac{\sigma^{2}}{|\sB|}\,.
  \end{align}
  Using more samples, \ie a larger $|\sB|$, decreases the variance, and thereby
  reduces noise in the mini-batch loss.
\end{subequations}
The central limit theorem~\cite{fischer2010history} connects the mini-batch
estimator to a normal distribution. As the mini-batch size $|\sB|$ approaches
infinity, $\gL_{\vn}$ converges to a normal distribution with mean and variance
from
\Cref{eq:background::unbiasedBatchEstimator,eq:background::varianceBatchEstimator}
\begin{align}\label{eq:background::CentralLimitTheorem}
  \lim_{|\sB| \to \infty}:
  \quad
  \gL_{\vn}
  \sim
  \gN(
  \giventhat{
  \gL_{\vn}
  }
  {
  \gL_{\sD},
  \nicefrac{\sigma^{2}}{|\sB|}
  }
  )\,.
\end{align}
While applications rely on finite batch sizes, it is sometimes useful to assume
that this Gaussian distribution holds approximately.

Noise in the mini-batch loss propagates to other quantities, like the mini-batch
gradient $\grad{\vtheta}\gL_{\sB}(\vtheta)$, that are used for applications like
training. Therefore, it represents a fundamental challenge for deep learning
methods. It can be assessed through higher-order statistical moments such as the
variance (the centered second moment, see
\Cref{sec:background::gradientCovariance,cockpit::app:instruments} for
examples). Motivated by the central limit theorem
\Cref{eq:background::CentralLimitTheorem}---if the Gaussian approximation holds
sufficiently well---only second-order statistical moments are required.

Like higher-order derivatives of multi-variate functions, higher-order
statistical moments of multi-variate random variables such as the mini-batch
gradient $\grad{\vtheta}\gL_{\sB}(\vtheta) \in \sR^D$ (first moment) scale
exponentially with dimension $D$ (\eg the $D \times D$ gradient covariance
matrix, \Cref{sec:background::gradientCovariance}). Therefore, they are
challenging to work with using naive approaches.

\subsection{Probabilistic
  Interpretation}\label{sec:background::ProbabilisticInterpretation}

Risk-based supervised learning is often connected to learning the unknown true
distribution $\pdata(\vx, \vy)$ through a model distribution $p_{\vtheta}(\vx,
\vy)$ and data $\sD$, using estimation techniques to finding a good set of
parameters that minimizes a measure of dissimilarity between $\pdata$ and
$p_{\vtheta}$. The following section links the loss function $\ell$ and the
model $f_{\vtheta}$ to probabilistic objects. This will be helpful to identify
additional structure in the risk minimization problem and use it for structural
approximation of higher-order information (\eg the Fisher information matrix,
\Cref{sec:background::naturalGradientDescent}), and to motivate probabilistic
applications (\eg Laplace approximations,
\Cref{sec:background::LaplaceApproximation}).

\subsubsection{Connections to Maximum Likelihood Estimation (MLE)}
The KL-divergence between the true and the model distribution,
\begin{subequations}
  \begin{align}
    \label{eq:background::KLDivergence}
    \begin{split}
      &\KLdiv{\pdata(\vx, \vy)}{p_{\vtheta}(\vx, \vy)}
      \\
      &\quad =
        \E_{\pdata(\vx, \vy)}
        \left[
        \log\pdata(\vx, \vy)
        -
        \log p_{\vtheta}(\vx, \vy)
        \right]\,,
    \end{split}
  \end{align}
  can be used to measure their dissimilarity. Minimizing the above expression
  over $\vtheta$, and dropping parameter-independent terms leads to
  \begin{align}
    \label{eq:background::minimizeNegativeLogProbability}
    \begin{split}
      &\minimize_{\vtheta}
        \KLdiv{\pdata(\vx, \vy)}{p_{\vtheta}(\vx, \vy)}
      \\
      \Leftrightarrow
      &\minimize_{\vtheta}
        \E_{\pdata(\vx, \vy)}
        \left[
        - \log p_{\vtheta}(\vx, \vy)
        \right]\,.
    \end{split}
  \end{align}
  $\pdata(\vx, \vy)$ is inaccessible and therefore empirically approximated
  through data $\sD = \{ (\vx_n, \vy_n) \}_n$. Under the \iid assumption in the
  data, $\pdata$ is replaced with the empirical distribution $p_{\sD}$ from
  \Cref{eq:background::empiricalDistribution} and yields the accessible
  optimization task
  \begin{align}
    \label{eq:background::minimizeEmpiricalNegativeLogProbability}
    \minimize_{\vtheta}
    \frac{1}{|\sD|}
    \sum_{(\vx_{n}, \vy_{n}) \in \sD} - \log p_{\vtheta}(\vx_n, \vy_n)
  \end{align}
  This expression resembles the empirical risk minimization
  \Cref{eq:background::empiricalRiskMinimization} with a specific loss function
  $\ell$ that produces the negative log-probability of a datum $(\vx_n,\vy_n)$.
\end{subequations}

\begin{subequations}
  Supervised learning only processes features $\vx$ to predict labels $\vy$. The
  probabilistic model $p_{\vtheta}(\vx, \vy)$ thus has a more special form, in
  that it only parameterizes the likelihood $\giventhat{\vy}{\vx}$,
  \begin{align}
    \label{eq:background::modelOnlyLikelihood}
    p_{\vtheta}(\vx, \vy) = p_{\vtheta}(\giventhat{\vy}{\vx}) p(\vx)\,.
  \end{align}
  Since only the likelihood contains parameters,
  \Cref{eq:background::minimizeNegativeLogProbability} simplifies to minimizing
  the expected negative log-likelihood
  \begin{align*}
    &\minimize_{\vtheta}
      \E_{\pdata(\vx, \vy)}
      \left[
      - \log p_{\vtheta}(\giventhat{\vy}{\vx}) - \log p(\vx)
      \right]
    \\
    \Leftrightarrow
    &\minimize_{\vtheta}
      \E_{\pdata(\vx, \vy)}
      \left[
      - \log p_{\vtheta}(\giventhat{\vy}{\vx})
      \right]
  \end{align*}
  and with empirical approximation through \iid data as \Cref{eq:background::minimizeEmpiricalNegativeLogProbability},
  \begin{align}
    \label{eq:background::mnimizeEmpiricalNegativeLogLikelihood}
    \minimize_{\vtheta}
    \frac{1}{|\sD|}
    \sum_{(\vx_{n}, \vy_{n}) \in \sD}
    - \log p_{\vtheta}(\giventhat{\vy_n}{\vx_n})\,.
  \end{align}
\end{subequations}
This minimization problem corresponds to MLE\sidenote{%
  \label{side:background::MLE}
  Finding the negative log-likelihood's minimum,
  \Cref{eq:background::mnimizeEmpiricalNegativeLogLikelihood}, is equivalent to
  finding the maximum of the \iid data's likelihood
  \begin{align*}
    p(\giventhat{\sD}{\vtheta})
    =
    \prod_{(\vx_n, \vy_n)\in\sD}
    p_{\vtheta}(\giventhat{\vy_n}{\vx_n})
    p(\vx_n)\,.
  \end{align*}
  The MLE satisfies
  \begin{align*}
    \vtheta_{\text{MLE}}
    &= \argmax p(\giventhat{\sD}{\vtheta})
    \\
    &= \argmax \log p(\giventhat{\sD}{\vtheta})
    \\
    &= \argmin - \log p(\giventhat{\sD}{\vtheta})\,.
  \end{align*}
  Inserting $p(\giventhat{\vtheta}{\sD})$ and dropping parameter-independent
  terms recovers the problem
  \Cref{eq:background::mnimizeEmpiricalNegativeLogLikelihood}.} with a
statistical model $p_{\vtheta}(\giventhat{\vy}{\vx})$. This is a specific form
of empirical risk minimization where the model's prediction $f_{\vtheta}(\vx)$
parameterizes a likelihood $q$ for $\giventhat{\vy}{\vf}$, and a negative
log-likelihood loss function $\ell$, \ie
\begin{subequations}
  \label{eq:background::connectionsMLEandEmpiricalRiskMinimization}
  \begin{align}
    p_{\vtheta}(\giventhat{\vy}{\vx}) &= q(\giventhat{\vy}{f_{\vtheta}(\vx)})\,,
    \\
    \ell(\vf, \vy) &= - \log q(\giventhat{\vy}{\vf})\,.
  \end{align}
\end{subequations}
Both the square loss and softmax cross-entropy loss
(\Cref{ex:background::Regression,ex:background::Classification}) have such a
probabilistic interpretation, see
\Cref{ex:background::probabilisticInterpretationMSELoss,ex:background::probabilisticInterpretationCrossEntropyLoss}.

\begin{example}[\textbf{Probabilistic interpretation of square
    loss}]\label{ex:background::probabilisticInterpretationMSELoss} The square
  loss \Cref{eq:background::squareLoss} is the negative log-likelihood of a
  Gaussian centered around the model's prediction with diagonal constant
  covariance,
  \begin{align*}
    \ell(\vf, \vy) &= -\log q(\giventhat{\vy}{\vf})
    \\
    \text{with}\quad
    q(\giventhat{\vy}{f_{\vtheta}(\vx)}) &= \gN(\giventhat{\vy}{\vmu, \mSigma})
  \end{align*}
  where%
  \sidenote[][-4\baselineskip]{Inserting mean and covariance into the negative
    log-probability yields,
    \begin{align*}
      -& \log \gN(\vy; \vmu, \mSigma)
      \\
       &=
         \nicefrac{1}{2} (\vy - \vmu)^{\top} \mSigma^{-1} (\vy - \vmu)
      \\
       &\phantom{=}
         + \nicefrac{1}{2}
         \left[
         \log\det\mSigma
         + C \log 2\pi
         \right]\,,
    \end{align*}
    \ie the square loss \Cref{eq:background::squareLoss} up to a
    $\vtheta$-independent term which does not affect optimization.}
  %
  $\vmu = f_{\vtheta}(\vx)$ and $\mSigma = \nicefrac{C}{2} \mI$.
\end{example}

\begin{example}[\textbf{Probabilistic interpretation of softmax cross-entropy
    loss}]\label{ex:background::probabilisticInterpretationCrossEntropyLoss}
  The softmax cross-entropy loss \Cref{eq:background::softmaxCrossEntropyLoss}
  is the negative log-likelihood of a multinomial distribution parameterized by
  the softmax probabilities,
  \begin{align*}
    \ell(\vf, y) &= -\log q(\giventhat{y}{\vf})
    \\
    \text{with}\quad
    q(\giventhat{y}{f_{\vtheta}(\vx)}) &= \Cat(y; \vp)
  \end{align*}
  where%
  \sidenote[][-4\baselineskip]{ With $\evp_{c}$ denoting the probability to observe class $y=c$,
    \begin{align*}
      - \log \Cat(y; \vp) = -\log \evp_{y}\,,
    \end{align*}
    which is the softmax cross-entropy loss \Cref{eq:background::softmaxCrossEntropyLoss}.}
  %
  $\vp = \softmax(f_{\vtheta}(\vx))$.
\end{example}

Following the maximum likelihood principle results in the MLE parameter
$\vtheta_{\text{MLE}}$ which satisfies
\Cref{eq:background::mnimizeEmpiricalNegativeLogLikelihood} and gives rise to
the distribution $p_{\vtheta_{\text{MLE}}}(\giventhat{\vy}{\vx}) =
q(\giventhat{\vy}{f_{\vtheta_{\text{MLE}}}(\vx)})$ as an approximation to $\pdata(\giventhat{\vy}{\vx})$.

\subsubsection{Connections to Maximum A Posteriori (MAP) Estimation}
MLE maximizes the likelihood $ p(\giventhat{\sD}{\vtheta})$. In a probabilistic
formulation with a prior $p(\vtheta)$ over the parameters and evidence $p(\sD) =
\int_{\sTheta} p(\giventhat{\sD}{\vtheta}) p(\vtheta)\,\diff \vtheta$ for the
data, one can instead consider the posterior
\begin{align}
  \label{eq:background::BayesRule}
  p(\giventhat{\vtheta}{\sD})
  =
  \frac{
  p(\giventhat{\sD}{\vtheta}) p(\vtheta)
  }{
  p(\sD)
  }\,
\end{align}
which corrects the prior with data observations. Such a posterior is useful to
form probabilistic beliefs over predictions $\vy_{\star}$ for new inputs
$\vx_{\star}$,
\begin{align}
  \label{eq:background::BayesianPrediction}
  \begin{split}
    p(\giventhat{\vy_{\star}}{\vx_{\star}, \sD})
    &=
      \int_{\sTheta} p(\giventhat{\vy_{\star}}{\vx_{\star}, \sD, \vtheta})
      p(\giventhat{\vtheta}{\vx_{\star}, \sD})\,\diff \vtheta
    \\
    &=
      \int_{\sTheta} p(\vy_{\star} | \vx_{\star}, \vtheta)
      p(\giventhat{\vtheta}{\sD})\,\diff \vtheta\,.
  \end{split}
\end{align}
However, this requires integration over the posterior, which itself is almost
always intractable and thus needs to be approximated.

The MAP principle approximates the posterior with a delta distribution around
the posterior mode, \ie the point of maximum posterior density,
\begin{align}
  \label{eq:background::MAP}
  p(\giventhat{\vtheta}{\sD})
  \approx
  \delta(\vtheta - \vtheta_{\text{MAP}})
  \quad\text{where}\quad
  \vtheta_{\text{MAP}}
  =
  \argmax_{\vtheta} p(\giventhat{\vtheta}{\sD})\,.
\end{align}
It is connected to empirical risk minimization with a regularization term that
results from the prior $p(\vtheta)$. To see this, reformulate
\Cref{eq:background::MAP} to minimize the negative log-posterior, expand Baye's
rule (\Cref{eq:background::BayesRule}) and neglect the parameter-independent
evidence term. Then, apply the same assumptions as for MLE\sidenote{%
  Mathematically, they translate into
  \begin{align*}
    \begin{split}
      &p(\giventhat{\sD}{\vtheta})
      \\
      &\quad =
        \prod_{n}
        p(\giventhat{\vx_{n}, \vy_{n}}{\vtheta})
      \\
      &\quad =
        \prod_{n}
        p(\giventhat{\vy_{n}}{\vx_{n}, \vtheta}) p(\giventhat{\vx_{n}}{\vtheta})
      \\
      &\quad =
        \prod_{n}
        p(\giventhat{\vy_{n}}{\vx_n, \vtheta}) p(\vx_{n})\,
    \end{split}
  \end{align*}
  with slightly different notation for $\vtheta$ in comparison to the MLE
  discussion, as it is now treated probabilistically. }, \ie \iid data and
$\vtheta$ only parameterizing the likelihood $\giventhat{\vy}{\vx}$. This yields
\begin{align}
  \label{eq:background::MAPRelationToEmpiricalRiskMinimization}
  \begin{split}
    \vtheta_{\text{MAP}}
    &= \argmin_{\vtheta} -\log p(\giventhat{\vtheta}{\sD})
    \\
    &= \argmin_{\vtheta} -\log p(\giventhat{\sD}{\vtheta}) - \log p(\vtheta)
    \\
    &= \argmin_{\vtheta} \sum_{(\vx_n,\vy_n) \in \sD}
      -\log p(\giventhat{\vy_n}{\vx_n, \vtheta})
      -\log p(\vtheta)
    \\
    &= \argmin_{\vtheta}
      \frac{1}{|\sD|}
      \sum_{(\vx_n,\vy_n) \in \sD}
      -\log p(\giventhat{\vy_n}{\vx_n, \vtheta})
      -\frac{\log p(\vtheta)}{|\sD|}\,.
  \end{split}
\end{align}
In analogy to \Cref{eq:background::connectionsMLEandEmpiricalRiskMinimization},
the first term is an empirical risk
(\Cref{eq:background::mnimizeEmpiricalNegativeLogLikelihood}) with negative
log-likelihood loss of a distribution $q$ for targets given model predictions,
\begin{align}\label{eq:background::connectionsMAPandEmpiricalRiskMinimization}
  p(\giventhat{\vy}{\vx,\vtheta})
  = q(\giventhat{\vy}{f_{\vtheta}(\vx)})
  \quad\text{and}\quad
  \ell(\vf, \vy)
  = - \log q(\giventhat{\vy}{\vf})\,.
\end{align}
However, this risk is extended by a regularization term from the prior,
\begin{align}\label{eq:background::MAPLoss}
  \vtheta_{\text{MAP}}
  =
  \argmin_{\vtheta}
  \gL_\sD(\vtheta) + r(\vtheta)
  \quad
  \text{where}
  \quad
  r(\vtheta) = - \frac{\log p(\vtheta)}{|\sD|}\,.
\end{align}
\Cref{eq:background::MAPRelationToEmpiricalRiskMinimization,eq:background::MAPLoss,eq:background::BayesRule}
connect the posterior with the loss via $ \log p(\giventhat{\sD}{\vtheta}) = -
|\sD| \gL_{\sD}(\vtheta)$ and $\log p(\vtheta) = - |\sD| r(\vtheta)$,
\begin{align}\label{eq:background::RelationPosteriorLossRegularization}
  \begin{split}
    p(\giventhat{\vtheta}{\sD})
    &=
      \frac{
      \exp
      \left[
      \log p(\giventhat{\sD}{\vtheta})
      + \log p(\vtheta)
      \right]
      }{
      p(\sD)
      }
    \\
    &=
      \frac{
      \exp\left\{
      -|\sD| \left[
      \gL_{\sD}(\vtheta) + r(\vtheta)
      \right]
      \right\}
      }{p(\sD)}\,.
  \end{split}
\end{align}
It underlines the aforementioned challenges to track the posterior. The
exponent is non-linear in $\vtheta$, and $p(\sD)$ requires computing an
integral.

\Cref{eq:background::RelationPosteriorLossRegularization} gives rise to
posterior approximations that go beyond a delta distribution. The Laplace
approximation~\cite{laplace1774memoire}
(\Cref{sec:background::LaplaceApproximation}) also starts with the MAP estimate,
but uses a quadratic Taylor expansion of the log-posterior around
$\vtheta_{\text{MAP}}$ to approximate the posterior by a Gaussian. This
quadratic expansion requires higher-order information in form of second-order
derivatives, presented in \Cref{chap:background::HigherOrder}.

%%% Local Variables:
%%% mode: latex
%%% TeX-master: "../thesis"
%%% End:


\section{Neural Networks}\label{sec:background::DeepNeuralNetworks}
The previous section focused on structure in the empirical risk---like its sum
structure, and interpretations of risk-based learning for specific loss
functions---without assumptions about the model $f_{\vtheta}$. This section
introduces structure in the model. While DNNs are generally highly
over-parameterized, they usually rely on relatively simple components and
construction principles.

This manuscript considers sequential feedforward neural networks that consist of
layers. They comprise ``classic'' architectures like multi-layer perceptrons
(MLPs), convolutional neural networks (CNNs), and established architectures like
VGG~\cite{simonyan2015deep}, and ResNets~\cite{he2016deep}. Most of the
discussion in this text also applies to other architectures, but would require
heavier notation.

\subsection{Layer-wise Notation}

\tikzexternalenable
\begin{figure*}[!t]
  \centering \resizebox{\linewidth}{!}{ {\footnotesize
      % basic setting of a fully-connected neural network with data flow for
% forward pass

\begin{tikzpicture}
  % first two layers
  \node (in1)
  [inner sep=0]
  {\tikz \drawMessagesWithArrows{$\vz^{(0)}$}{ }{ }{\hNodeDistance};};
  \node (layer1)
  [anchor=south west, inner sep=0]
  at (in1.south east)
  {\tikz \drawModuleWithParams{$f^{(1)}_{\vtheta^{(1)}}$}{16}{$\vtheta^{(1)}$}{ }{ };};
  \node (out1)
  [inner sep=0, anchor=south west]
  at (layer1.south east)
  {\tikz \drawMessagesWithArrows{$\vz^{(1)}$}{ }{ }{\hNodeDistance};};
  \node (layer2)
  [inner sep=0pt, anchor=south west]
  at (out1.south east)
  {\tikz \drawModuleWithParams{$f^{(2)}_{\vtheta^{(2)}}$}{16}{$\vtheta^{(2)}$}{ }{ };};

  % dots with messages
  \node (in2)
  [inner sep=0, anchor=south west]
  at (layer2.south east)
  {\tikz \drawMessagesWithArrows{$\vz^{(2)}$}{ }{ }{\hNodeDistance};};
  \node (dots)
  [xshift=0.75ex, inner sep=0pt, anchor=west]
  at (in2.east)
  {$\dots$};

  \node (inLast)
  [xshift=0.75ex, inner sep=0pt, anchor=west]
  at (dots.east)
  {\tikz \drawMessagesWithArrows{$\vz^{(L-1)}$}{ }{ }{\hNodeDistance};};

  \node (layerLast)
  [anchor=south west, inner sep=0]
  at (inLast.south east)
  {\tikz \drawModuleWithParams{$f^{(L)}_{\vtheta^{(L)}}$}{16}{$\vtheta^{(L)}$}{ }{ };};
  \node (outLast)
  [inner sep=0, anchor=south west]
  at (layerLast.south east)
  {\tikz \drawMessagesWithArrows{$\vz^{(L)}$}{ }{ }{\hNodeDistance};};
\end{tikzpicture}

%%% Local Variables:
%%% mode: latex
%%% TeX-master: "../../thesis"
%%% End:
}}
  \caption{\textbf{Forward pass of a sequential feedforward neural network
      (\Cref{eq:background::neuralNetwork}).} The computational graph indicates
    the data flow and dependencies of intermediate
    variables.}\label{fig:background::neuralNetwork}
\end{figure*}
\tikzexternaldisable

Sequential feedforward neural networks of depth $L$ consist of \emph{modules},
or \emph{layers}, $f^{(l)}_{\vtheta^{(l)}}, l = 1, \ldots, L$, stacked on top of
each other such that
\begin{subequations}\label{eq:background::neuralNetworkAndModule}
  \begin{align}
    \label{eq:background::neuralNetwork}
    f_{\vtheta}
    =
    f^{(L)}_{\vtheta^{(L)}}
    \circ
    f^{(L-1)}_{\vtheta^{(L-1)}}
    \circ
    \ldots
    \circ
    f^{(1)}_{\vtheta^{(1)}}
  \end{align}
  They map input features $\vx =: \vz^{(0)}$ to predictions $f_{\vtheta}(\vx) =:
  \vz^{(L)}$ via a sequence of intermediate hidden features $\vz^{(1)}, \dots,
  \vz^{(L-1)}$. In a forward pass, a module $f^{(l)}_{\vtheta^{(l)}}$ receives the
  parental output $\vz^{(l-1)} \in \sR^{h^{(l-1)}}$ and applies an operation with
  (potentially empty) parameters $\vtheta^{(l)} \in \sR^{d^{(l)}}$,
  \begin{align}
    \label{eq:background::Module}
    \vz^{(l)}
    =
    f^{(l)}_{\vtheta^{(l)}}(\vz^{(l-1)})\,.
  \end{align}
\end{subequations}
The
output features $\vz^{(l)} \in \sR^{h^{(l)}}$ serve as input to the next layer
$l+1$. This builds up dependencies in form of the computational graph shown in
\Cref{fig:background::neuralNetwork} that maps the leaf nodes $\vz^{(0)}$ and
$\vtheta^{(1)}, \dots, \vtheta^{(L)}$ to the prediction $\vz^{(L)}$. The neural
network parameters are often treated as a single vector $\vtheta \in \sR^{D}$
which results from layer-wise concatenation,%
\marginnote{%
  \begin{definition}[\textbf{Tensor flattening}]\label{def:background::Flattening}
    Let $\tA \in \sR^{n_1 \times n_2 \times \dots, \times n_m}$ denote a tensor
    of rank $m$ with dimensions $n_1, n_2, \dots, n_{m}$.
    The flattened tensor $\vec(\tA) \in \sR^{n_1 n_2 \cdots n_m}$ is a vector
    that concatenates $\tA$'s elements in a
    first-index-varies-fastest fashion,
    \begin{equation}\label{eq:background::Flattening}
      \vec(\tA)
      =
      \begin{pmatrix}
        \etens{A}_{1,1,1,\dots,1}
        \\
        \etens{A}_{2,1,1,\dots,1}
        \\
        \vdots
        \\
        \etens{A}_{n_1,1,1,\dots,1}
        \\
        \etens{A}_{1,2,1,\dots,1}
        \\
        \vdots
        \\
        \etens{A}_{n_1,2,1,\dots,1}
        \\
        \vdots
        \\
        \etens{A}_{n_1, n_2, n_3, \dots, n_m}
      \end{pmatrix}\,.
    \end{equation}
    The matrix case $m=2$ corresponds to column-stacking. Flattening a vector
    $\va$ leaves it unaffected, \ie $\vec(\va) = \va$.
  \end{definition}}%
\begin{align}
  \label{eq:background::neuralNetworkParameters}
  \vtheta
  =
  \begin{pmatrix}
    \vtheta^{(1)}
    \\
    \vtheta^{(2)}
    \\
    \vdots
    \\
    \vtheta^{(L)}
  \end{pmatrix}\,.
\end{align}
To simplify the presentation, \Cref{eq:background::Module} assumes vector-shaped
quantities. However, many neural networks process higher-dimensional data like images,
represented by tensors. Sometimes, the tensor structure is convenient to work
with. One can convert between the tensor and vector view without loss of
generality by introducing conventions for tensor flattening
(\Cref{def:background::Flattening}) and reshaping
(\Cref{def:background::TensorReshape})%
\marginnote{%
  \begin{definition}[\textbf{Vector reshaping}]\label{def:background::TensorReshape}
    Let $\va\in \sR^{n_1 n_2 \cdots n_m}$ be a vector. Reshaping that vector
    into a rank-$m$ tensor of shape $S = (n_1, n_2, \dots, n_m)$ happens by
    filling $\va$'s elements into the tensor in a first-index-varies-fastest
    fashion,
    \begin{subequations}\label{eq:background::TensorReshape}
      \begin{align}
        \tA &= \reshape_{S}(\va)
      \end{align}
      with elements
      \begin{align}
        \begin{split}
          \etA_{1, 1, 1, \dots, 1} &= \eva_1\,,
          \\
          \etA_{2, 1, 1, \dots, 1} &= \eva_2\,,
          \\
                                   &\vdots
          \\
          \etA_{n_1, 1, 1, \dots, 1} &= \eva_{n_1}\,,
          \\
          \etA_{1, 2, 1, \dots, 1} &= \eva_{n_1 + 1}\,,
          \\
                                   &\vdots
          \\
          \etA_{n_1, 2, 1, \dots, 1} &= \eva_{2 n_1}\,,
          \\
                                   &\vdots
          \\
          \hspace{-2ex}\etA_{n_1, n_2, n_3, \dots, n_m} &= \eva_{n_1 n_2 n_3 \cdots n_m}\,.
        \end{split}
      \end{align}
    \end{subequations}
    The matrix case $m=2$ corresponds to column-filling. With tensor flattening
    (\Cref{def:background::Flattening}) this allows to define tensor reshaping:
    A tensor $\tB_1$ of shape $S_1$ is rearranged into any tensor $\tB_2$ of
    compatible shape $S_2$ by first flattening, then reshaping it, \ie $\tB_2 :=
    \reshape_{S_2}(\tB_{1}) := \reshape_{S_2}(\vec \tB_{1})$.
  \end{definition}
  }%
%
; after all, multi-dimensional arrays are represented in a vector format in
memory.

However, there exist different flattening conventions. Implementations often
favor row-major ordering. This manuscript uses (the more common in literature)
column-major order as it allows for elegant generalizations of derivative
concepts for multi-variate functions, like the Jacobian
(\Cref{hbp::def:generalizedJacobian}), the Hessian
(\Cref{hbp::def:generalizedHessian}), and their chain rules
(\Cref{hbp::the:chainRuleJacobians,hbp::the:chainRuleHessians}). To translate
analytical results into implementations, it is crucial to be aware of these
differing conventions.

\subsection{Modularity \& Common
  Operations}\label{sec:background::CommonOperations}

An important strength of deep learning is its \emph{modularity}. ML libraries
provide a large number of operations, or modules, that can be combined in almost
arbitrary ways through function composition, like in
\Cref{eq:background::neuralNetwork}. Training the resulting models with
first-order methods remains simple because their gradient can be automatically
computed via AD (\Cref{sec:background::GradientBackpropagation}). New operations
can easily be added because its implementation is decoupled to the modular
level.

% What is a module?
Modules are vaguely defined. Often, multiple operations that form a logical
processing unit in a neural network are grouped into a single module, \eg an MLP
layer combines affine transformation and elementwise activation (see below). In
extreme cases, even an entire neural network can be considered a single module
that can be used in other neural networks; \eg the neural network in
\Cref{fig:background::ModelLossFunctionSplit} resembles a single layer in
\Cref{fig:background::neuralNetwork} and could act as one layer in a larger
network.

% Which perspective are we choosing?
For theoretical analyses, it is preferable to consider units with a small number
of operations as modules. This, however, is inconvenient for constructing large
architectures, where many operations are grouped into higher-level units. This
text adapts a rather fine-grained view on modules that is close to their
implementation in ML libraries like \pytorch.

A common categorization for modules distinguishes trainable functions with
parameters, and parameter-free operations. \Cref{tab:background::forward} lists
the forward passes of common operations that will be illustrated in the
following presentation of network architectures. To distringuish more clearly
between input and output of an operation, the notation uses the symbols $\vx,
\vz$ for module input and output instead of $\vz^{(l-1)}, \vz^{(l)}$, and
neglects the layer superscript for the parameters, writing $\vtheta$ instead of
$\vtheta^{(l)}$.

\begin{table}[!t]
  \caption{\textbf{Forward pass for common modules used in feedforward
      networks.} Input and output are denoted $\vx, \vz$ rather than $\vz^{(l)},
    \vz^{(l+1)}$ to avoid clutter. $\mI$ is the identity matrix. Bold upper-case
    symbols ($\mW, \mX, \mZ, \dots$) denote matrices and bold upper-case sans
    serif symbols ($\tW, \tX, \tZ, \dots$) denote tensors. See
    \Cref{hbp::sec:examples_fcnn,hbp::sec:examples_loss,hbp::sec:examples_cnn}
    for details, and
    \Cref{tab:background::Jacobians,hbp::table:backpropEquations} for extended
    versions of this table for the backward, and Hessian backward, pass.}
  \label{tab:background::forward}
  \centering
  \begin{footnotesize}
    \begin{tabular}{ll}
      \toprule
      \textbf{OPERATION} & \textbf{FORWARD}
      \\
      \midrule
      % matrix-vector multiplication
      Matrix-vector multiplication & $\vz(\vx, \mW) = \mW\vx$
      \\
      % matrix-matrix multiplication
      Matrix-matrix multiplication & $\mZ(\mX, \mW) = \mW\mX$
      \\
      % addition
      Addition & $\vz(\vx, \vb) = \vx + \vb$
      \\
      % elementwise activation
      Elementwise activation & $\vz(\vx) = \vphi(\vx)$\,,\ \text{s.t.} $z_i(\vx) = \phi(x_i)$
      \\
      \midrule
      % residual unit/skip-connection
      Skip-connection & $\vz(\vx, \vtheta) = \vx + \vs(\vx, \vtheta)$
      \\
      \midrule
      % reshape/view operation
      Reshape/view & $\tZ(\tX)= \mathrm{reshape}(\tX)$
      \\
      % extraction operator
      Index select/map $\pi$ & $\vz(\vx) = \mPi \vx\, ,$ $\emPi_{j,\pi(j)} = 1\,, $
      \\
      % convolution
      Convolution & $\tZ(\tX, \tW) = \tX \star \tW$\,,
      \\
                         & $\mZ(\mW, \llbracket\tX\rrbracket) = \mW \llbracket \tX \rrbracket$\,,
      \\
      \midrule
      % square loss
      Square loss & $\ell(\vf, \vy) = \nicefrac{1}{C} (\vy-\vf)^\top (\vy - \vf)$ \\
      % cross-entropy loss
      Softmax cross-entropy & $\ell(\vf, y) = - \onehot(y)^\top \log\left[\vp(\vf)\right]$
      \\
      \bottomrule
    \end{tabular}
  \end{footnotesize}
\end{table}

\subsubsection{Deep Linear Networks \& Multi-layer Perceptrons (MLPs)}

\emph{Linear layers} process inputs $\vx$ by affine transformation, \ie multiplication
with a weight matrix $\mW$, followed by addition of a bias vector $\vb$,
\begin{align}\label{eq:background::LinearLayer}
  \vz = \mW \vx + \vb
  \qquad
  \text{where}
  \qquad
  \vtheta =
  \begin{pmatrix}
    {(\vec \mW)}^{\top} & {\vb}^{\top}
  \end{pmatrix}^{\top}\,.
\end{align}
They are also referred to as \emph{fully-connected} layers, because each output
$z_i$ depends on all inputs $\vx$ through $\mW_{i,:}$ and $\evb_i$.

\emph{Deep linear networks} %
\marginnote{%
  \begin{example}[\textbf{Deep linear
      network}]\label{ex:background::deepLinearNetwork}
    In notation of \Cref{eq:background::neuralNetworkAndModule}, a deep linear
    network of depth $L$ reads
    \begin{align*}
      \vz^{(l)}
      &=
        \mW^{(l)} \vz^{(\ell-1)} + \vb^{(l)}
        \qquad
        \shortintertext{where}
        \qquad
        \vtheta^{(l)}
      &=
        \begin{pmatrix}
          {\left(\vec \mW^{(l)}\right)}^{\top} & {\vb^{(l)}}^{\top}
        \end{pmatrix}^{\top}\,,
      \\
      l &= 1, \dots, L\,.
    \end{align*}
  \end{example}
}%
(\Cref{ex:background::deepLinearNetwork}) consist of
only linear layers and are of interest for analytical
studies~\cite[\eg][]{saxe2014exact,mulayoff2020unique,bernacchia2018exact} as
they are somewhat tractable. They describe a linear feature map, \ie a linear
function \wrt the input $\vz^{(0)}$, that is non-linear in the parameters.
Therefore, such networks are as expressive as a single linear layer, but highly
overparameterized.

A common technique to turn a deep linear network into a non-linear feature map
is to interlace affine transformations with non-linear
activations~\cite{rosenblatt1958perceptron}. An \emph{activation layer} $\vphi$
acts elementwise on its input, \ie applies the same function $\phi$ to each
input element,
\begin{align*}
  \vz = \vphi(\vx)\qquad \text{such that}\qquad \evz_i = \phi(\evx_i)\,.
\end{align*}
There exist many activations (ReLU, sigmoid, $\tanh$, \etc \cite[Chapter
6]{goodfellow2016deep}), and recent works proposing new choices (\eg squared
ReLU~\citep{so2021searching}).

\emph{Multi-layer perceptrons} %
\marginnote{%
  \begin{example}[\textbf{Multi-layer perceptron (MLP)}]\label{ex:background::MLP}
    In terms of \Cref{eq:background::neuralNetworkAndModule}, an MLP of depth
    $L$ reads
    \begin{align*}
      \vz^{(l)}
      &=
        \vphi^{(l)}\left( \mW^{(l)} \vz^{(\ell-1)} + \vb^{(l)}\right)\,,
        \shortintertext{where}
        \vtheta^{(l)}
      &=
        \begin{pmatrix}
          {\left(\vec \mW^{(l)}\right)}^{\top} & {\vb^{(l)}}^{\top}
        \end{pmatrix}^{\top}\,,
      \\
      l &= 1, \dots, L\,,
      \\
      \vphi^{(L)} &= \mathrm{id}
    \end{align*}
    and $\mathrm{id}$ denotes the identity.
  \end{example}
}%
(MLPs, \Cref{ex:background::MLP}) combine affine transformation and activation
in a layer. Activation functions $\vphi^{(l)}$ may vary between layers, but are
often chosen identically, with no activation in the last layer. One way to
interpret this design is that the mapping $\vz^{(0)} \mapsto \vz^{(L-1)}$ acts
as non-linear feature transformation, and the last layer $\vz^{(L-1)} \mapsto
\vz^{(L)}$ as linear classifier for the learned features.

\subsubsection{Convolutional Neural Networks (CNNs)}

CNNs represent an important neural network architecture revolution and were the
first class of deep neural network to beat ``classical'' methods on the
\imagenet
competition~\cite{deng2009imagenet,krizhevsky2012imagenet,russakovsky2015imagenet}.
Broadly speaking, such networks contain convolutional layers with trainable
parameters.

\paragraph{Convolutional layers:} Convolutions process multi-channel input
features such as images and are parameterized by a kernel that can be imagined
as a filter for patterns like edges, corners, \etc During the convolution
operation, the kernel slides over the input features and produces an output
element by contraction with the overlapping elements of the image. In most
cases, each output channel is also shifted by a bias parameter which will be
neglected in this presentation for simplicity (detailed discussion in
\Cref{hbp::sec:examples_cnn}, example in \Cref{hbp::equ:convolutionExample}).
Because the kernel moves over the image, it can detect patterns irrespective of
their position. The process can be adjusted with various hyperparameters, such
as stride, padding, groups, dilation
(see~\cite{dumoulin2016ConvolutionArithmeticGuide} for a visual guide).

In contrast to the linear layer (\Cref{eq:background::LinearLayer}) where each
output is connected to all inputs via independent rows of the weight matrix, the
parameters in the kernel are shared across all outputs. Therefore, convolutions
usually require less parameters than fully-connected layers.

Nevertheless, both layers are related because convolution is a linear operation
and can therefore be regarded as matrix multiplication. Due to the weight
sharing, this matrix is structured by the kernel elements\sidenote{\Eg, in the
  one-dimensional case, the convolution of two vectors can be computed by
  expanding one into a Toeplitz matrix, and multiplying that onto the second
  vector~\cite{wiki2022toeplitz}.}. Alternatively, one can stack patches---input
elements that overlap with the kernel at each stage---into columns of a matrix,
which yields the unfolded input, denoted by $\llbracket \tX \rrbracket$ in
\Cref{tab:background::forward}. Then, convolution is a matrix-matrix product
between a matrix reshape of the kernel and the unfolded
input~\cite{chellapilla2006HighPerformanceCNN}
(see~\Cref{hbp::subfig:convolutionIllustration3} for an illustration).

\paragraph{Padding \& pooling layers:} Convolutions are often combined with
other modules. \emph{Padding layers} add pixels around the outer dimensions of
an image, which helps to reduce information loss at the image boundaries during
convolution. \emph{Pooling layers} down-sample images by summarizing patches of
pixels and reduce the number of hidden features. Similar to convolution, pooling
considers patches of an input image. Two common summary operations are
per-channel averaging and taking the per-channel maximum. They give rise to
maximum and average pooling.

One can interpret padding and pooling as scatter operations, realized by
multiplication with a sparse, binary matrix $\mPi$ (compare
\Cref{tab:background::forward}:
\begin{itemize}
\item For padding, $\mPi$ does not dependent on the input, but only its shape
  and the hyperparameters. A row is empty if its index corresponds to the padded
  area, and otherwise contains a one at the element's index to be copied from
  the input.

\item For maximum pooling, $\mPi$ depends on the input. Each row contains a one
  at the index of the element with maximum value in the patch.

\item For average pooling, $\mPi$ does not dependent on the input, but only its
  shape and the hyperparameters. Each row contains the inverse patch size at
  indices of the elements in the current patch.
\end{itemize}

\subsubsection{Residual Networks (ResNets)}

The inclusion of skip (or residual) connections~\cite{he2016deep} represents
another revolution in the design of CNNs, and enabled training of deeper
architectures, with 100 or even 1000 layers. This lead to improved performance
of such CNNs on tasks like
\imagenet~\cite{deng2009imagenet,russakovsky2015imagenet}. Skip connections
branch off a hidden feature and feed it back after the residual block $\vs(\vx,
\vtheta)$,
\begin{align*}
  \vz = \vx + \vs(\vx, \vtheta)\,.
\end{align*}
This can be seen as learning a small modification $\vs(\vx, \vtheta)$---a
residual function---for $\vx$; hence the name \emph{residual connection}.

\subsubsection{Closing Remarks \& Sources of Non-linearity}

This short overview of common neural network layers and architectures is, of
course, incomplete. Other famous layers include
dropout~\cite{srivastava2014dropout},
recursive~\cite{hochreither1997lstm,cho2014properties,elman1990finding},
normalization~\cite{ioffe2015batch,wu2019group}, attention
layers~\cite{vaswani2017attention}, \etc

An interesting observation about the operations in
\Cref{tab:background::forward} is that most of them are linear (linear,
convolution, padding, and average pooling layers) or piece-wise linear (maximum
pooling and ReLU activation layer) \wrt both their input and parameters. This
implies that their second- and higher-order derivatives vanish. Non-linearity is
often only introduced by activation layers (elementwise) and loss functions
(after the model's forward pass). While the properties of these components are
not inherited by the entire neural network, they give rise to structure in a
network's higher-order derivatives, see \eg \Cref{chap:hbp}.


%%% Local Variables:
%%% mode: latex
%%% TeX-master: "../thesis"
%%% End:


\section{Automatic
  Differentiation}\label{sec:background::GradientBackpropagation}
Together with empirical risk minimization and neural networks, the last
ingredient in the ML pipeline, \Cref{alg:background::trainingLoop}, is computing
the gradient. Contemporary methods to train neural networks
(\Cref{sec:background::PopularOptimizers}) rely on this quantity. ML libraries
compute it via their built-in automatic differentiation (AD), built around the
famous backpropagation algorithm~\cite{rumelhart1986learning}, described in more
detail here.

\subsection{Foundations}\label{sec:background::ADFoundations}

\subsubsection{An Example \& Path Interpretation}

Given a program that evaluates a function, AD produces a program to evaluate its
derivative. The general idea is to consider the function as composition of
atomic operations (\eg addition, multiplication, \dots) with known derivatives.
To automatically compute the derivatives, one needs to track the dependencies
between intermediate variables and combine their derivatives using the chain
rule.

\tikzexternalenable
\definecolor{autodiffSketchForwardFill}{RGB}{255,255,204}
\definecolor{autodiffSketchBackwardFill}{RGB}{161,218,180}

\tikzset{autodiffSketchForward/.style={
    font=\footnotesize,
    fill=autodiffSketchForwardFill,
    thick,
    rounded corners=1ex,
    draw=black,
    minimum width=3.5ex,
    minimum height=3ex,
  }
}

\tikzset{autodiffSketchBackward/.style={
    font=\footnotesize,
    fill=autodiffSketchBackwardFill,
    thick,
    rounded corners=1ex,
    draw=black,
    minimum width=3.5ex,
    minimum height=3ex,
  }
}

\tikzset{autodiffSketchArrow/.style={
    ->,
    >=stealth,
    ultra thick,
    black,
  }
}

\newcommand{\autodiffSketchDrawNodes}{
  \node [anchor = south west, autodiffSketchForward] (z1)
  at (0, 0) {$z^{(1)}$};
  \node [anchor = south east, autodiffSketchForward] (z2)
  at (\figwidth pt, 0) {$z^{(2)}$};
  \node [anchor = center, autodiffSketchForward] (z3)
  at (0.5*\figwidth pt, 0.25*\figheight pt) {$z^{(3)}$};
  \node [anchor = west, autodiffSketchForward] (z4)
  at (0, 0.575*\figheight pt) {$z^{(4)}$};
  \node [anchor = east, autodiffSketchForward] (z5)
  at (\figwidth pt, 0.425*\figheight pt) {$z^{(5)}$};
  \node [anchor = east, autodiffSketchForward] (z6)
  at (\figwidth pt, 0.75*\figheight pt) {$z^{(6)}$};
  \node [anchor = north, autodiffSketchForward] (z7)
  at (0.5*\figwidth pt, \figheight pt) {$z^{(7)}$};
}

\newcommand{\autodiffSketchConnectNodes}{
  \draw[autodiffSketchArrow] (z1.north) to (z4.south);
  \draw[autodiffSketchArrow] (z2.north west) to (z3.south east);
  \draw[autodiffSketchArrow] (z2.north) to (z5.south);
  \draw[autodiffSketchArrow] (z3.north east) to (z5.south west);
  \draw[autodiffSketchArrow] (z4.north) to (z7.south west);
  \draw[autodiffSketchArrow] (z4.north east) to (z6.south west);
  \draw[autodiffSketchArrow] (z5.north) to (z6.south);
  \draw[autodiffSketchArrow] (z6.north west) to (z7.south east);
}

\newcommand{\autodiffSketchDrawBackwardNodes}{
  \draw[autodiffSketchArrow, draw=none] (z1.north) to
  node [midway, autodiffSketchBackward, xshift=2ex] {$\exp(z^{(1)})$}
  (z4.south);
  \draw[autodiffSketchArrow, draw=none] (z2.north west) to
  node [midway, autodiffSketchBackward] {$\cos(z^{(2)})$}
  (z3.south east);
  \draw[autodiffSketchArrow, draw=none] (z2.north) to
  node [midway, autodiffSketchBackward] {$1$}
  (z5.south);
  \draw[autodiffSketchArrow, draw=none] (z3.north east) to
  node [midway, autodiffSketchBackward] {$1$}
  (z5.south west);
  \draw[autodiffSketchArrow, draw=none] (z4.north)
  to
  node [midway, autodiffSketchBackward] {$1$}
  (z7.south west);
  \draw[autodiffSketchArrow, draw=none] (z4.north east) to
  node [midway, autodiffSketchBackward] {$z^{(5)}$}
  (z6.south west);
  \draw[autodiffSketchArrow, draw=none] (z5.north) to
  node [midway, autodiffSketchBackward] {$z^{(4)}$}
  (z6.south);
  \draw[autodiffSketchArrow, draw=none] (z6.north west) to
  node [midway, autodiffSketchBackward] {$1$}
  (z7.south east);
}

\newcommand{\autodiffSketchDefineSizeClip}{
  \pgfmathsetmacro{\figwidth}{0.95\linewidth}
  \pgfmathsetmacro{\figheight}{1.3\linewidth}
  \clip (0,0) rectangle (\figwidth pt, \figheight pt);
}

\begin{figure}
  \centering
  \begin{subfigure}[t]{0.45\linewidth}
    \centering
    \caption{Function as Python program}
    \label{subfig:background::AutodiffSketch1}
\begin{lstlisting}[language=Python]
from math import exp, sin

def z7(z1: float, z2: float):
    """Example function."""

    # intermediate variables
    z3 = sin(z2)
    z5 = z3 + z2
    z4 = exp(z1)
    z6 = z4 * z5

    # output variable
    z7 = z4 + z6

    return z7
\end{lstlisting}
  \end{subfigure}
  \hfill
  \begin{subfigure}[t]{0.45\linewidth}
    \centering
    \caption{Computation graph}
    \label{subfig:background::AutodiffSketch2}
    \begin{tikzpicture}
      \autodiffSketchDefineSizeClip
      \autodiffSketchDrawNodes
      \autodiffSketchConnectNodes

      \node [font=\footnotesize, anchor=west] at (z4.east) {$\exp(z^{(1)})$};
      \node [font=\footnotesize, anchor=north east] at (z3.south) {$\sin(z^{(2)})$};
      \node [font=\footnotesize, anchor=east] at (z5.west) {$z^{(3)} + z^{(2)}$};
      \node [font=\footnotesize, anchor=east] at (z6.west) {$z^{(4)} \cdot z^{(5)}$};
      \node [font=\footnotesize, anchor=west] at (z7.east) {$z^{(4)} + z^{(6)}$};
    \end{tikzpicture}
  \end{subfigure}

  \vspace{1ex}

  \begin{subfigure}[t]{0.45\linewidth}
    \centering
    \caption{Local derivatives}
    \label{subfig:background::AutodiffSketch3}
    \begin{tikzpicture}
      \autodiffSketchDefineSizeClip
      \autodiffSketchDrawNodes
      \autodiffSketchConnectNodes
      \autodiffSketchDrawBackwardNodes
    \end{tikzpicture}
  \end{subfigure}
  \hfill
  \begin{subfigure}[t]{0.45\linewidth}
    \centering
    \caption{Bauer paths}
    \label{subfig:background::AutodiffSketch4}
    \begin{tikzpicture}
      \autodiffSketchDefineSizeClip

      \begin{scope}[transparency group, opacity=0.2]
        \autodiffSketchDrawNodes
        \autodiffSketchConnectNodes
      \end{scope}
      \begin{scope}[transparency group, opacity=0.66]
        \autodiffSketchDrawBackwardNodes
      \end{scope}

      \begin{pgfonlayer}{background}
        \draw [maincolor, line width = 2.5pt] plot [smooth] coordinates
        {($(z7.east)!0.8!(z7.west)$)
          ($(z4.east)!0.66!(z4.west)$)
          ($(z1.east)!0.66!(z1.west)$)};

        \draw [secondcolor, line width = 2.5pt] plot [smooth] coordinates
        {($(z7.east)!0.6!(z7.west)$)
          ($(z6.east)!0.75!(z6.west)$)
          ($(z4.east)!0.33!(z4.west)$)
          ($(z1.east)!0.33!(z1.west)$)};

        \draw [thirdcolor, line width = 2.5pt] plot [smooth] coordinates
        {($(z7.east)!0.4!(z7.west)$)
          ($(z6.east)!0.5!(z6.west)$)
          ($(z5.east)!0.66!(z5.west)$)
          (z3)
          ($(z2.east)!0.66!(z2.west)$)};

        \draw [darkgraycolor, line width = 2.5pt] plot [smooth] coordinates
        {($(z7.east)!0.2!(z7.west)$)
          ($(z6.east)!0.25!(z6.west)$)
          ($(z5.east)!0.33!(z5.west)$)
          ($(z2.east)!0.33!(z2.west)$)};
      \end{pgfonlayer}
    \end{tikzpicture}
  \end{subfigure}
  \caption{\textbf{Basic AD principles~\cite{oktay2021randomized}.}
    \subfigref{subfig:background::AutodiffSketch1} Example input program
    represented by Python code. The atomic operations combined via
    composition are addition, multiplication, exponential function, and sine.
    The result $z^{(7)}$ is computed from the inputs $z^{(1)}, z^{(2)}$ through
    intermediates $z^{(3)}, z^{(4)}, z^{(5)}, z^{(6)}$,
    \begin{align*}
      z^{(7)}
      &=
        \exp(z^{(1)})
      \\
      &\phantom{=}\,
        + \exp(z^{(1)}) \left( \sin(z^{(2)}) + z^{(2)} \right)\,.
    \end{align*}
    \subfigref{subfig:background::AutodiffSketch2}
    Representation as computation graph to track dependencies between the
    intermediate variables on the level of atomic operations.
    \subfigref{subfig:background::AutodiffSketch3} Computing derivatives relies
    on local derivatives $\nicefrac{\partial z^{(j)}}{\partial z^{(i)}}$ on edges
    $(z^{(i)}, z^{(j)})$, which need to be accumulated according to the chain rule.
    \subfigref{subfig:background::AutodiffSketch4} Interpretation of the chain
    rule as sum over path products. Computing the derivatives of a node \wrt to
    another node in the graph requires summing the path product of local
    derivatives for all paths that connect them. In detail:
    \begin{align*}
      \frac{\partial z^{(7)}}{\partial z^{(1)}}
      &=
        \textcolor{maincolor}{
        \frac{\partial z^{(7)}}{\partial z^{(4)}}
        \frac{\partial z^{(4)}}{\partial z^{(1)}}
        }
        +
        \textcolor{secondcolor}{
        \frac{\partial z^{(7)}}{\partial z^{(6)}}
        \frac{\partial z^{(6)}}{\partial z^{(4)}}
        \frac{\partial z^{(4)}}{\partial z^{(1)}}
        }
      \\
      &=
        \textcolor{maincolor}{
        \exp(z^{(1)})
        }
        +
        \textcolor{secondcolor}{
        z^{(5)}
        \exp(z^{(1)})
        }
      \\
      &=
        \textcolor{maincolor}{
        \exp(z^{(1)})
        }
      \\
      &\phantom{=}\,
        +
        \textcolor{secondcolor}{
        \left( \sin(z^{(2)}) + z^{(2)} \right)
        \exp(z^{(1)})
        }\,,
      \\
      \frac{\partial z^{(7)}}{\partial z^{(2)}}
      &=
        \textcolor{thirdcolor}{
        \frac{\partial z^{(7)}}{\partial z^{(6)}}
        \frac{\partial z^{(6)}}{\partial z^{(5)}}
        \frac{\partial z^{(5)}}{\partial z^{(3)}}
        \frac{\partial z^{(3)}}{\partial z^{(2)}}
        }
      \\
      &\phantom{=}\,
        +
        \textcolor{darkgraycolor}{
        \frac{\partial z^{(7)}}{\partial z^{(6)}}
        \frac{\partial z^{(6)}}{\partial z^{(5)}}
        \frac{\partial z^{(5)}}{\partial z^{(2)}}
        }
      \\
      &=
        \textcolor{thirdcolor}{
        z^{(4)}
        \cos(z^{(2)})
        }
        +
        \textcolor{darkgraycolor}{
        z^{(4)}
        }
      \\
      &=
        \textcolor{thirdcolor}{
        \exp(z^{(1)})
        \cos(z^{(2)})
        }
        +
        \textcolor{darkgraycolor}{
        \exp(z^{(1)})
        }\,.
    \end{align*}}\label{fig:background::AutodiffSketch}
\end{figure}

%%% Local Variables:
%%% mode: latex
%%% TeX-master: "../../thesis"
%%% End:

\tikzexternaldisable

\Cref{fig:background::AutodiffSketch} illustrates the basic principles of AD for
an example from~\cite{oktay2021randomized}. The starting point is a function
defined by code (\Cref{subfig:background::AutodiffSketch1}). Such a function
transforms input to output variables through atomic operations and builds up
intermediate variables along its execution. The relation between these
operations are described by a directed graph $\gG = (\gV, \gE)$ with a set of
nodes $\gV = \{z^{(1)}, z^{(2)}, \dots\}$ and a set of edges $\gE =
\{(z^{(i_1)}, z^{(j_1)}), (z^{(i_2)}, z^{(j_2)}), \dots\}$ where $(z^{(i)},
z^{(j)})$ denotes a directed edge from node $z^{(i)}$ to $z^{(j)}$
(\Cref{subfig:background::AutodiffSketch2}).

To compute derivatives, the local derivatives on edges
(\Cref{subfig:background::AutodiffSketch3}) are combined according to the chain
rule. For $\nicefrac{\partial z^{(j)}}{ \partial z^{(i)}}$ between two variables
in the graph, all paths connecting them need to be considered. A path between
two nodes $z^{(i)}$ and $z^{(j)}$ is a sequence of edges that connect them:
starting from $z^{(i)}$, following the edges in a path leads to $z^{(j)}$. Let
$[z^{(i)} \to z^{(j)}]$ denote the set of paths connecting $z^{(i)}$ to
$z^{(j)}$. Then the derivative is the sum of path products of local derivatives
(\Cref{subfig:background::AutodiffSketch4}),
\begin{align}
  \label{eq:background::BauersFormula}
  \frac{\partial z^{(j)}}{\partial z^{(i)}}
  =
  \sum_{p \in [z^{(i)} \to z^{(j)}]}
  \prod_{(z^{(k)}, z^{(l)}) \in p}
  \frac{\partial z^{(l)}}{\partial z^{(k)}}\,.
\end{align}
The path formulation goes back to~\citet{bauer1974computational}.

\subsubsection{The Jacobian Matrix \& Its Chain Rule}
For the computation graph of a neural network's empirical risk, tracking
dependencies between variables at a scalar level would result in a considerable
book-keeping overhead due to the large number of connections. This can be
circumvented by using vector-valued (or tensor-valued) nodes $ \vz^{(1)},
\vz^{(2)}, \dots$ and tracking edges between vectors (or tensors), see
\Cref{fig:background::neuralNetworkLoss}. However, this requires a
generalization of \Cref{eq:background::BauersFormula} to multi-variate nodes.
The accumulation is efficiently expressed as matrix multiplication by arranging
partial derivatives into \emph{Jacobians}.

\begin{definition}[\textbf{Jacobian}]\label{def:background::JacobianVectorVector}
  Let $\vb:\mathbb{R}^{n} \to \mathbb{R}^{m}, \va \mapsto \vb(\va)$ be a
  differentiable vector-to-vector function. The Jacobian $\jac_{\va} \vb(\va)$
  of $\vb$ \wrt $\va$ is an $m \times n$ matrix with partial derivatives,
  \begin{align}
    \label{eq:background::JacobianVector}
    \jac_{\va} \vb(\va) = \frac{\partial \vb(\va)}{\partial \va^\top}\,,
    \quad\text{with}\quad
    [\jac_{\va} \vb(\va)]_{i,j}
    &=
      \frac{
      \partial \evb_i(\va)
      }{
      \partial \eva_j
      }\,.
  \end{align}
  Matrix- and tensor-valued functions require flattening their arguments into
  vectors\sidenote[][-15em]{\Cref{def:background::JacobianVectorVector} assumes
    vector-valued functions. With the flattening convention
    \Cref{def:background::Flattening}, it generalizes to tensor-valued functions
    as follows:\vspace{1ex}
    \begin{definition}[\textbf{Generalized Jacobian}]\label{hbp::def:generalizedJacobian}
      Let $\mB:\mathbb{R}^{n\times q}\to\mathbb{R}^{m\times p},
      \mA\mapsto\mB(\mA)$ be a differentiable matrix-to-matrix function. The
      Jacobian $\jac_{\mA} \mB(\mA)$ of $\mB$ \wrt $\mA$ is an $mp \times nq$
      matrix
      \begin{subequations}\label{hbp::equ:generalizedJacobian}
        \begin{align}
          \jac_{\mA} \mB(\mA) &= \frac{\partial \vec \mB(\mA)}{\partial (\vec \mA)^\top}
                                \shortintertext{with entries}
                                [\jac_{\mA} \mB(\mA)]_{i,j}
          &=
            \frac{
            \partial \left[ \vec \mB(\mA)\right]_i
            }{
            \partial \left[\vec \mA\right]_j
            }
        \end{align}
      \end{subequations}
      and the flattening operation $\vec$ from \Cref{def:background::Flattening}
      \citep[Chapter 9.4]{magnus1999MatrixDifferentialCalculus}. The analogous
      tensor case $(\mA, \mB) \to (\tA, \tB)$ requires lengthy notation and is
      therefore omitted.
    \end{definition}
    In the context of neural networks, the most common occurrences of
    \Cref{hbp::def:generalizedJacobian} involve vector-to-vector functions $f:
    \mathbb{R}^n \to \mathbb{R}^m, \vx \mapsto f(\vx)$ with
    \begin{align*}
      \jac_{\vx} f(\vx) = \frac{\partial f(\vx)}{\partial \vx^\top}\,.
    \end{align*}
    For instance, $\vx$ can be considered the input or bias vector of a layer
    applying an affine transformation. Other cases involve matrix-to-vector mappings
    $f: \mathbb{R}^{n\times q} \to \mathbb{R}^m, \mX \mapsto f(\mX)$ with
    \begin{align*}
      \jac_{\mX} f(\mX) = \frac{\partial f(\mX)}{\partial (\vec \mX)^\top}\,,
    \end{align*}
    where $\mX$ might correspond to the $\mathbb{R}^{m\times q}$ weight matrix
    of a linear layer. See \Cref{tab:background::Jacobians} for an overview.
  }.
  For a vector-to-scalar function $b(\va)$, \ie $m=1$, the Jacobian has one row
  that contains the gradient, $[\jac_{\va} b(\va)]^{\top} = \grad{\va}b$. The
  gradient will often be denoted by $\vg$. \Eg $\vg_{\pdata}(\vtheta) :=
  \grad{\vtheta} \gL_{\pdata}(\vtheta)$ for the gradient of the population risk
  \Cref{eq:background::expectedRisk}, and $\vg_{\sD}(\vtheta) := \grad{\vtheta}
  \gL_{\sD}(\vtheta)$ for the gradient of the empirical risk
  \Cref{eq:background::empiricalRisk} on a dataset $\sD$ (with $\sD = \Dtrain, \sB$
  for the train loss and the mini-batch gradient).
\end{definition}

In the vector-valued case, one must accumulate Jacobians through matrix
multiplies instead of scalar multiplications to compute derivatives,
\begin{align}\label{eq:background::BauersFormulaVector}
  \jac_{\vz^{(j)}}\vz^{(i)}
  =
  \sum_{p \in [\vz^{(i)} \to \vz^{(j)}]}
  \prod_{(\vz^{(k)}, \vz^{(l)}) \in p}
  \jac_{\vz^{(k)}}\vz^{(l)}(\vz^{(k)})\,.
\end{align}
The product term generalizes the chain rule to vector-valued functions.

\begin{theorem}[\textbf{Jacobian chain
    rule}]\label{def:background::JacobianChainRuleVector} Let $\vb: \mathbb{R}^{n} \to
  \mathbb{R}^{m}, \va \mapsto \vb(\va)$ and $\vc: \mathbb{R}^{m} \to
  \mathbb{R}^{r}, \vb \mapsto \vc(\vb)$ be differentiable vector-to-vector
  functions. Consider their composition $\vd = \vc \circ \vb: \mathbb{R}^{n} \to
  \mathbb{R}^{r}, \va \mapsto \vd(\va) = \vc(\vb(\va))$. The composition's
  Jacobian $\jac_{\va} \vd(\va) \in \sR^{r \times n}$ is related to the composite
  Jacobians via
  \begin{align}
    \label{eq:background::JacobianChainRuleVector}
    \jac_{\va} \vd(\va)=  \left[ \jac_{\vb} \vc(\vb) \right] \jac_{\va} \vb(\va)\,.
  \end{align}
  This can be generalized to tensor-valued functions\sidenote[][1em]{Proper
    arrangement of partial derivatives leads to a generalized Jacobian chain
    rule for matrices/tensors:\vspace{1ex}
    \begin{theorem}[\textbf{Generalized Jacobian chain rule}]
      \label{hbp::the:chainRuleJacobians}
      Let $\mB: \mathbb{R}^{n\times q} \to \mathbb{R}^{m \times p}$ and $\mC:
      \mathbb{R}^{m\times p} \to \mathbb{R}^{r\times s}$ be differentiable
      matrix-to-matrix functions. Let $\mD = \mC \circ \mB: \mathbb{R}^{n\times q}
      \to \mathbb{R}^{r\times s}, \mA \to \mD(\mA) = \mC(\mB(\mA))$ be their
      composition. Then,
      \begin{align}
        \label{hbp::equ:chainRuleJacobians}
        \jac_{\mA} \mD(\mA)=  \left[ \jac_{\mB} \mC(\mB) \right] \jac_{\mA} \mB(\mA)
      \end{align}
      with the generalized Jacobian \Cref{hbp::def:generalizedJacobian}
      \citep[Chapter 5.15]{magnus1999MatrixDifferentialCalculus}. The tensor case
      $(\mD, \mC, \mB, \mA) \to (\tD, \tC, \tB, \tA)$ is analogous.
    \end{theorem}
  }.
\end{theorem}

\subsubsection{Jacobian Accumulation (Automatic Differentiation Modes)}

Given the computation graph $\gG$ of a function to be differentiated,
\Cref{eq:background::BauersFormulaVector} describes the operations that need to
be performed. But there are different schedules for carrying out these
computations, with differing performance: \eg, it is possible to share
accumulated derivative products between paths that share subpaths (like paths
\tikz[baseline=-0.5ex]{\draw[fill=thirdcolor, draw=none] circle (0.75ex);} and
\tikz[baseline=-0.5ex]{\draw[fill=darkgraycolor, draw=none] circle (0.75ex);} in
\Cref{subfig:background::AutodiffSketch4}). And for a single path in the vector
case, the optimal contraction order of the Jacobian matrix chain (one summand of
\Cref{eq:background::BauersFormulaVector}) depends on the dimension of the nodes
(\Cref{eq:background::MatrixChain}). The following Jacobian accumulation
schedules are of specific interest for AD:
\begin{itemize}
\item \textbf{Forward accumulation, or forward mode AD,} starts at the leafs,
  \ie the nodes \wrt which the function is differentiated. Jacobians are
  accumulated in the same order as the function evaluation.
\item \textbf{Reverse accumulation, or reverse mode
    AD~\cite{griewank2012invented,linnainmaa1976taylor},} starts at the root,
  \ie the variable that is differentiated. Jacobians are accumulated from root
  to leaf nodes, traversing the graph backwards. This is often called a
  \emph{backward pass}.
\item \textbf{Optimal Jacobian accumulation} computes derivatives according to
  the optimal schedule which usually traverses the computation graph in a
  nontrivial fashion. For arbitrary computation graphs, finding this schedule is
  NP-hard \cite{naumann2008optimal}.
\end{itemize}
Due to the specific structure of computation graphs in deep learning, reverse
mode AD is often more practical than forward accumulation. This is outlined in
in the following section, that illustrates reverse mode for differentiation of a
neural network's loss \wrt its parameters, leading to the famous backpropagation
algorithm~\cite{rumelhart1986learning}.

\subsection{Gradient Backpropagation}\label{sec:background:GradientBackpropagationInMLLibraries}

Gradient backpropagation~\cite{rumelhart1986learning} enables efficient
differentiation of the training objective in deep learning. It is the central
algorithm of popular ML libraries with built-in AD. This section presents
backpropagation for chain-structured computation graphs (see~\cite[Chapter
6]{goodfellow2016deep} for the general case) like the loss of a sequential
feedforward neural network. Starting from the loss of a single datum, the goal
is to show that ML libraries combine AD and batching to maximize efficiency. But
this limits their functionality to computing the gradient, ignoring \eg the
per-sample structure in the loss. Alleviating this limitation to compute richer
information (\Cref{chap:background::HigherOrder}) using the existing
implementation of gradient backpropagation is a main goal of \Cref{part:papers}
in this thesis.

\tikzexternalenable
\begin{figure*}[t]
  \centering \resizebox{\linewidth}{!}{ {\footnotesize
      % basic setting of a fully-connected neural network with data flow for
% forward pass

\begin{tikzpicture}
  % first two layers
  \node (in1)
  [inner sep=0]
  {\tikz \drawMessagesWithArrows{$\vz^{(0)}$}{ }{ }{\hNodeDistance};};
  \node (layer1)
  [anchor=south west, inner sep=0]
  at (in1.south east)
  {\tikz \drawModuleWithParams{$f^{(1)}_{\vtheta^{(1)}}$}{16}{$\vtheta^{(1)}$}{ }{ };};
  \node (out1)
  [inner sep=0, anchor=south west]
  at (layer1.south east)
  {\tikz \drawMessagesWithArrows{$\vz^{(1)}$}{ }{ }{\hNodeDistance};};
  \node (layer2)
  [inner sep=0pt, anchor=south west]
  at (out1.south east)
  {\tikz \drawModuleWithParams{$f^{(2)}_{\vtheta^{(2)}}$}{16}{$\vtheta^{(2)}$}{ }{ };};

  % dots with messages
  \node (in2)
  [inner sep=0, anchor=south west]
  at (layer2.south east)
  {\tikz \drawMessagesWithArrows{$\vz^{(2)}$}{ }{ }{\hNodeDistance};};
  \node (dots)
  [xshift=0.75ex, inner sep=0pt, anchor=west]
  at (in2.east)
  {$\dots$};

  \node (inLast)
  [xshift=0.75ex, inner sep=0pt, anchor=west]
  at (dots.east)
  {\tikz \drawMessagesWithArrows{$\vz^{(L-1)}$}{ }{ }{\hNodeDistance};};

  \node (layerLast)
  [anchor=south west, inner sep=0]
  at (inLast.south east)
  {\tikz \drawModuleWithParams{$f^{(L)}_{\vtheta^{(L)}}$}{16}{$\vtheta^{(L)}$}{ }{ };};
  \node (outLast)
  [inner sep=0, anchor=south west]
  at (layerLast.south east)
  {\tikz \drawMessagesWithArrows{$\vz^{(L)}$}{ }{ }{\hNodeDistance};};

  % loss layer
  \node (lossLayer)
  [inner sep=0pt, anchor=south west]
  at (outLast.south east)
  {\tikz\drawModuleNoParams{$\ell$}{5};};
  \node (loss)
  [inner sep=0, anchor=south west]
  at (lossLayer.south east)
  {\tikz \drawMessagesWithArrows{$\ell$}{ }{ }{\hNodeDistance};};

  % label
  \node (label)
  [inner sep=0, anchor=south west, rotate=-90, xshift = -96, yshift = -50]
  at (lossLayer.south east)
  {\tikz \drawMessagesWithArrows{$\vy$}{ }{ }{\hNodeDistance};};
\end{tikzpicture}

%%% Local Variables:
%%% mode: latex
%%% TeX-master: "../../thesis"
%%% End:
}}
  \caption{\textbf{Computation graph of a sequential feedforward neural
      network's loss for a single datum from
      \Cref{eq:background::neuralNetworkLoss}.}}\label{fig:background::neuralNetworkLoss}
\end{figure*}
\tikzexternaldisable

\subsubsection{Loss of a Single Datum}
Consider the loss implied by a single datum $(\vx, \vy)$, a loss function
$\ell$, and a neural network $f_{\vtheta}$ depicted in
\Cref{fig:background::neuralNetworkLoss},
\begin{subequations}
  \begin{align}
    \label{eq:background::neuralNetworkLoss}
    \ell(\vtheta)
    =
    \ell(f_{\vtheta}(\vx), \vy)
    \qquad
    \text{with}
    \qquad
    f_{\vtheta}
    =
    f^{(L)}_{\vtheta^{(L)}}
    \circ
    f^{(L-1)}_{\vtheta^{(L-1)}}
    \circ
    \ldots
    \circ
    f^{(1)}_{\vtheta^{(1)}}\,.
  \end{align}
  Its computation graph $\gG = (\gV, \gE)$ has nodes
  \begin{align}
    \gV &=
          \left\{
          \vtheta^{(l)}
          \right\}_{l=1}^L
          \cup
          \left\{
          \vz^{(l)}
          \right\}_{l=0}^L
          \cup
          \left\{
          \vy, \ell
          \right\}
          \shortintertext{and edges}
          \gE &=
                \left\{
                (\vz^{(l-1)}, \vz^{(l)})
                \right\}_{l=1}^L
                \cup
                \left\{
                (\vtheta^{(l)}, \vz^{(l)})
                \right\}_{l=1}^L
                \cup
                \left\{
                (\vz^{(L)}, \ell), (\vy, \ell)
                \right\}\,.
  \end{align}
\end{subequations}
Each edge implies a Jacobian, categorized as one of the following:
\begin{itemize}
\item The \emph{input-output} Jacobian $\jac_{\vz^{(l-1)}}\vz^{(l)}(\vz^{(l-1)})$ of
  a module $l$.
\item The \emph{parameter-output} Jacobian
  $\jac_{\vtheta^{(l)}}\vz^{(l)}(\vtheta^{(l)})$ of a module $l$.
\item The \emph{prediction-loss} Jacobian $\jac_{\vz^{(L)}}\ell(\vz^{(L)})$ has
  one column and will be written as gradient of the loss \wrt the model
  prediction, $\jac_{\vz^{(L)}}\ell(\vz^{(L)}) = [\grad{\vz^{(L)}}
  \ell(\vz^{(L)})]^{\top}$.
\end{itemize}
The goal is to compute \emph{parameter-loss} Jacobians, \ie gradients of the
loss \wrt parameters $\jac_{\vtheta^{(l)}}\ell(\vtheta^{(l)}) =
[\grad{\vtheta^{(l)}} \ell(\vtheta^{(l)})]^{\top}$ for all layers $l=1,\dots,
L$.

First, consider only the gradient $\grad{\vtheta^{(l)}} \ell$ of one parameter
$\vtheta^{(l)}$. \Cref{eq:background::BauersFormulaVector} requires identifying
all paths connecting $\vtheta^{(l)}$ to $\ell$. Due to the computation graph's
chain structure, this is only a single path,
\begin{subequations}
  \begin{gather}
    [\vtheta^{(l)} \to \ell]
    =
    \left\{
      p
    \right\}
    \shortintertext{with}
    p
    =
    \left(
      \left( \vtheta^{(l)}, \vz^{(l)} \right),
      \left( \vz^{(l)}, \vz^{(l+1)} \right),
      \dots
      \left( \vz^{(L-1)}, \vz^{(L)} \right),
      \left( \vz^{(L)}, \ell \right)
    \right)\,.
  \end{gather}
  Plugging this into \Cref{eq:background::BauersFormulaVector} simplifies to
  \begin{align}
    \label{eq:background::ParameterJacobian}
    \underbrace{
    \jac_{\vtheta^{(l)}}\ell(\vtheta)
    }_{
    1 \times d^{(l)}
    }
    =
    \underbrace{
    \left[
    \jac_{\vz^{(L)}}\ell
    \right]
    }_{
    1 \times h^{(L)}
    }
    \underbrace{
    \left[
    \jac_{\vz^{(L-1)}}\vz^{(L)}
    \right]
    }_{
    h^{(L)} \times h^{(L-1)}
    }
    \cdots
    \underbrace{
    \left[
    \jac_{\vz^{(l)}}\vz^{(l+1)}
    \right]
    }_{
    h^{(l+1)} \times h^{(l)}
    }
    \underbrace{
    \left[
    \jac_{\vtheta^{(l)}}\vz^{(l)}
    \right]
    }_{
    h^{(l)} \times d^{(l)}
    }
    \,,
    \intertext{and in gradient notation}
    \label{eq:background::ParameterGradient}
    \underbrace{
    \grad{\vtheta^{(l)}}\ell(\vtheta)
    }_{
    d^{(l)}
    }
    =
    \underbrace{
    \left[
    \jac_{\vtheta^{(l)}}\vz^{(l)}
    \right]^{\top}
    }_{
    d^{(l)}
    \times
    h^{(l)}
    }
    \underbrace{
    \left[
    \jac_{\vz^{(l)}}\vz^{(l+1)}
    \right]^{\top}
    }_{
    h^{(l)}
    \times
    h^{(l+1)}
    }
    \cdots
    \underbrace{
    \left[
    \jac_{\vz^{(L-1)}}\vz^{(L)}
    \right]^{\top}
    }_{
    h^{(L-1)}
    \times
    h^{(L)}
    }
    \underbrace{
    \grad{\vz^{(L)}}\ell
    }_{
    h^{(L)}
    }
    \,.
  \end{align}
\end{subequations}
In comparison to the general formulation
\Cref{eq:background::BauersFormulaVector}, the rather simple graphs of a neural
network's loss yield much simpler expressions
(\Cref{eq:background::ParameterJacobian,eq:background::ParameterGradient}) that
are a result of the Jacobian chain rule
(\Cref{def:background::JacobianChainRuleVector}) applied to the loss
\Cref{eq:background::neuralNetworkLoss}. They also illustrate the impact of
contraction order on performance due to the connection to matrix
chains\sidenote[][-15em]{Assuming no cost to compute a Jacobian, the optimal
  Jacobian contraction of
  \Cref{eq:background::ParameterGradient,eq:background::ParameterJacobian} are
  matrix chain problems that can be solved with dynamic programming: given $n$
  matrices $\mA_1, \mA_2, \dots, \mA_n$, the task is to find the optimal
  contraction schedule of $\mA_1 \mA_2 \cdots \mA_n$. This is crucial for
  performance, as this example from \citet[Chapter 15.2]{cormen2001introduction}
  illustrates:
  \begin{example}[\textbf{Matrix chain contraction}]\label{eq:background::MatrixChain}
    Let $\mA_1 \in \sR^{10 \times 100}, \mA_2 \in \sR^{100 \times 5}, \mA_3 \in
    \sR^{5 \times 50}$. There are two schedules to evaluate the chain $\mA_1
    \mA_2 \mA_3$ (cost for addition neglected for simplicity):
    \begin{itemize}[leftmargin=*]
    \item \textbf{$(\mA_1 \mA_2)\mA_3$:} $\mB_1 = \mA_1 \mA_2 \in \sR^{10 \times
        5}$ costs 100 multiplications per element ($5,000$ in total). $\mB_1
      \mA_3 \in \sR^{10 \times 50}$ costs 5 multiplications per element ($2,500$
      in total).
    \item \textbf{$\mA_1 (\mA_2 \mA_3)$:} $\mB_2 = \mA_2 \mA_3 \in \sR^{100
        \times 50}$ costs 5 multiplications per element ($25,000$ in total).
      $\mA_1 \mB_2 \in \sR^{10 \times 50}$ costs 100 multiplications per element
      ($50,000$ in total).
    \end{itemize}
    The order $(\mA_1 \mA_2) \mA_3$ uses 10x fewer operations ($7,500$ versus
    $75,000$).
  \end{example}
}%
, mentioned in \Cref{sec:background::ADFoundations}.

In forward mode, the matrix chain \Cref{eq:background::ParameterGradient} would
be evaluated from left to right, starting with a $d^{(l)} \times h^{(l)}$
Jacobian that is transformed into $d^{(l)} \times h^{(l')}$ matrices where $l' >
l$. Since the parameter count $d^{(l)}$ in a layer is large in DNNs, these
intermediate matrices are costly to store.

In reverse mode, \Cref{eq:background::ParameterGradient} is evaluated from right
to left, starting with a $h^{(L)}$-dimensional vector that is transformed into
vectors of dimension $h^{(l')}$ with $L > l' \ge l$. These intermediate
accumulations require less memory than forward mode.

Each approach has drawbacks, however. Reverse mode uses a more efficient matrix
multiplication order, but the entire graph must have been evaluated and stored,
or re-computed, to construct the Jacobians. While this is not needed for forward
mode, forward accumulation starts with the Jacobian of an edge $(\vtheta^{(l)},
\vz^{(l)})$ that is not shared with paths for other parameters. Therefore,
intermediate accumulations can not be reused for other gradients. This is
another crucial property of reverse mode, which does allow reuse of intermediate
results: consider the accumulations to obtain the gradient $\grad{\vtheta}\ell$
of all layers,
\begin{align*}
  \setlength{\arraycolsep}{1.4pt} % auto-reverts outside the environment
  \begin{array}{cccrc}
    \grad{\vtheta^{(1)}}\ell
    &
      =
    &
      \left[
      \jac_{\vtheta^{(1)}} \vz^{(1)}
      \right]^{\top}
    &
      \left[
      \jac_{\vz^{(1)}} \vz^{(2)}
      \right]^{\top}
      \left[
      \jac_{\vz^{(2)}} \vz^{(3)}
      \right]^{\top}
      \cdots
      \left[
      \jac_{\vz^{(L-1)}} \vz^{(L)}
      \right]^{\top}
    &
      \grad{\vz^{(L)}} \ell
    \\
    \grad{\vtheta^{(2)}}\ell
    &
      =
    &
      \left[
      \jac_{\vtheta^{(2)}} \vz^{(2)}
      \right]^{\top}
    &
      \left[
      \jac_{\vz^{(2)}} \vz^{(3)}
      \right]^{\top}
      \cdots
      \left[
      \jac_{\vz^{(L-1)}} \vz^{(L)}
      \right]^{\top}
    &
      \grad{\vz^{(L)}} \ell
    \\
    \vdots
    &
      =
    &
      \vdots
    &
      \ddots
      \phantom{
      \left[
      \jac_{\vz^{(L-1)}} \vz^{(L)}
      \right]^{\top}
      }
    &
      \vdots
    \\
    \grad{\vtheta^{(L-1)}}\ell
    &
      =
    &
      \left[
      \jac_{\vtheta^{(L-1)}} \vz^{(L-1)}
      \right]^{\top}
    &
      \left[
      \jac_{\vz^{(L-1)}} \vz^{(L)}
      \right]^{\top}
    &
      \grad{\vz^{(L)}} \ell
    \\
    \grad{\vtheta^{(L)}}\ell
    &
      =
    &
      \left[
      \jac_{\vtheta^{(L)}} \vz^{(L)}
      \right]^{\top}
    &
    %
    &
      \grad{\vz^{(L)}} \ell
  \end{array}
\end{align*}
The paths for any two parameters $\vtheta^{(l_1)}, \vtheta^{(l_2)}$ with $l_2 >
l_1$ share edges
\begin{align*}
  \left( \vz^{(l_2)}, \vz^{(l_2+1)} \right),
  \dots,
  \left( \vz^{(L-1)}, \vz^{(L)} \right),
  \left( \vz^{(L)}, \ell \right)\,.
\end{align*}
Therefore, their matrix chains share the accumulated gradient
\begin{align*}
  \grad{\vz^{(l_2)}}\ell
  =
  \left[
  \jac_{\vz^{(l_2)}}\vz^{(l_2+1)}
  \right]^{\top}
  \left[
  \jac_{\vz^{(l_2+1)}}\vz^{(l_2+2)}
  \right]^{\top}
  \cdots
  \left[
  \jac_{vz^{(L-1)}}\vz^{(L)}
  \right]^{\top}
  \grad{\vz^{(L)}}\ell\,.
\end{align*}
This gradient \wrt hidden features is passed backwards through the graph and
used by a layer, before updating it and passing it to the next. The
interpretation of the described accumulation scheme is therefore known as
gradient backpropagation algorithm \cite{rumelhart1986learning}:%

\begin{definition}[\textbf{Gradient backpropagation for sequential feedforward
    neural networks}]
  \label{def:background::GradientBackpropagation}
  Given the computation graph of a loss $\ell(f_{\vtheta}(\vx), \vy)$ implied by
  a datum $(\vx, \vy)$, a loss function $\ell$, and a sequential feedforward
  neural network $f_{\vtheta}$ from \Cref{eq:background::neuralNetworkLoss},
  gradient backpropagation recovers the gradient vector $\grad{\vtheta}\ell =
  \begin{pmatrix}
    (\grad{\vtheta^{(1)}}\ell)^{\top},
    (\grad{\vtheta^{(2)}}\ell)^{\top},
    \dots,
    (\grad{\vtheta^{(L)}}\ell)^{\top}
  \end{pmatrix}^{\top}
  $ in stages by passing gradients backward through the graph (\Cref{subfig:background::gradientBackpropagation1}):
  \begin{itemize}
  \item Initialize the backpropagated vector with $\grad{\vz^{(L)}}\ell$ at $L$.
  \item For layer $l = L, \dots, 1$
    \begin{enumerate}
    \item Receive the output gradient $\grad{\vz^{(l)}}\ell$.
    \item Recover the parameter gradient $\grad{\vtheta^{(l)}}\ell =
      \left[ \jac_{\vtheta^{(l)}}\vz^{(l)}\right]^{\top}\grad{\vz^{(l)}}\ell$\,.
    \item Compute the input gradient $\grad{\vz^{(l-1)}}\ell = \left[
        \jac_{\vtheta^{(l)}}\vz^{(l)}\right]^{\top}\grad{\vz^{(l)}}\ell$\,.
    \item Free $\grad{\vz^{(l)}}\ell$ and send $\grad{\vz^{(l-1)}}\ell$ to layer
      $l-1$.
    \end{enumerate}
  \end{itemize}
\end{definition}
\Cref{def:background::GradientBackpropagation} is \emph{modular}, as mentioned
earlier in \Cref{sec:background::CommonOperations}: it only relies on local
derivatives. Supporting a new operation only requires specifying its forward
pass and the vector-Jacobian products (VJPs%
\sidenote[][-3.5\baselineskip]{During backpropagation, the transposed Jacobian is
  right-multiplied onto the backpropagated gradient vector
  (\Cref{eq:background::ParameterGradient} from right to left). One can see this
  as left-multiplying the Jacobian to a column vector
  (\Cref{eq:background::ParameterJacobian} from left to right), \ie computing a
  vector-Jacobian product. Forward accumulation instead left-multiplies the
  transposed Jacobian onto a matrix (\Cref{eq:background::ParameterGradient}
  from left to right). One can view this as right-multiplying the Jacobian onto
  the transposed matrix (\Cref{eq:background::ParameterJacobian} from right to
  left). This requires multiple Jacobian-vector products (JVPs), or a
  Jacobian-matrix product (JMP).}) %
with its input-output and parameter-output Jacobians. Given functionality to
create and traverse computation graphs, implementations of backpropagation are
very extensible due to its abstraction to the modular level.

Backpropagation itself performs a VJP with the network's parameter-output
Jacobian $\jac_{\vtheta} \vz^{(L)}$. Choosing the vector to be
$\grad{\vz^{(L)}}\ell$ yields the gradient $\grad{\vtheta} \ell =
[\jac_{\vtheta} \vz^{(L)}]^{\top} \grad{\vz^{(L)}}\ell$. But one can also
compute VJPs $[\jac_{\vtheta} \vz^{(L)}]^{\top} \vv$ with arbitrary vectors
$\vv$. The model's parameter-output Jacobian is crucial for computing
higher-order information (\Cref{chap:background::HigherOrder,part:papers}).

Although backpropagation only requires VJPs, the Jacobian matrix is an
interesting object for analytical studies into its structure, and for efficient
implementation of functionality that goes beyond computing gradients (\eg
matrix-Jacobian products (MJPs)). \Cref{tab:background::Jacobians} contains the
Jacobians of the common operations in neural networks from
\Cref{sec:background::CommonOperations}. These Jacobians are conveniently
obtained using matrix differential
calculus~\cite{magnus1999MatrixDifferentialCalculus}, presented in
\Cref{hbp::sec:matrixDifferentialCalculus}.

\begin{table*}[t]
  \centering
  \caption{\textbf{Jacobians (\Cref{hbp::def:generalizedJacobian}) for common
      modules in feedforward networks.} Input and output are denoted $\vx, \vz$
    rather than $\vz^{(l)}, \vz^{(l+1)}$ to avoid clutter. $\mI$ is the identity
    matrix. Matrices use bold upper-case symbols ($\mW, \mX, \mZ, \dots$),
    tensors use bold upper-case sans serif symbols ($\tW, \tX, \tZ, \dots$).
    Most Jacobians can be elegantly derived with matrix differential calculus,
    see \Cref{hbp::sec:matrixDifferentialCalculus} for
    details.}\label{tab:background::Jacobians}
  \begin{footnotesize}
    \begin{tabular}{llll}
      \toprule
      \textbf{OPERATION} & \textbf{FORWARD} & \textbf{JACOBIAN} (\Cref{hbp::def:generalizedJacobian}) & \textbf{DETAILS}
      \\
      \midrule
      % matrix-vector multiplication
      Matrix-vector multiplication & $\vz(\vx, \mW) = \mW\vx$ & $\jac_{\vx}\vz = \mW$\,,  & \Cref{hbp::subsec:linearLayerBackwardPass}
      \\
                         & & $\jac_{\mW} \vz = \vx^{\top} \otimes \mI$
      \\
      % matrix-matrix multiplication
      Matrix-matrix multiplication & $\mZ(\mX, \mW) = \mW\mX$ & $\jac_{\mX} \mZ = \mI \otimes
                                                                \mW$\,, & \Cref{hbp::subsec:linearLayerBackwardPass}
      \\
                         & & $\jac_{\mW} \mZ = \mX^\top \otimes \mI$
      \\
      % addition
      Addition & $\vz(\vx, \vb) = \vx + \vb$ & $\jac_{\vx}\vz = \jac_{\vb} \vz =\mI $ & \Cref{hbp::subsec:linearLayerBackwardPass}
      \\
      % elementwise activation
      Elementwise activation & $\vz(\vx) = \vphi(\vx)$\,,\ \text{s.t.} & $\jac_{\vx}\vz = \diag[\vphi'(\vx)]$ & \Cref{hbp::subsec:activationBackwardPass}
      \\
                         & $z_i(\vx) = \phi(x_i)$ &
      \\
      \midrule
      % residual unit/skip-connection
      Skip-connection & $\vz(\vx, \vtheta) = \vx + \vs(\vx, \vtheta)$ & $\jac_{\vx}\vz =
                                                                        \mI + \jac_{\vx}\vs$\,, & \Cref{hbp::subsec:skipconnectionBackwardPass}
      \\
                         & & $\jac_{\vtheta}\vz = \jac_{\vtheta} \vs $
      \\
      \midrule
      % reshape/view operation
      Reshape/view & $\tZ(\tX)=
                     \mathrm{reshape}(\tX)$ & $\jac_{\tX}\tZ = \mI$ & \Cref{hbp::subsec:HBPReshape}
      \\
      % extraction operator
      Index select/map $\pi$ & $\vz(\vx) = \mPi \vx\, ,$ $\emPi_{j,\pi(j)} =
                               1 $ & $\jac_{\vx}\vz = \mPi$ %(same as matrix-vector mul.\ with $W$)
                                                                                                      & \Cref{hbp::subsec:HBPIndexSelect}
      \\
      % convolution
      Convolution & $\tZ(\tX, \tW) = \tX
                    \star \tW$\,, & $\jac_{\llbracket \tX \rrbracket}\tZ =
                                    \mI \otimes \mW$\,,
                                                                                                      & \Cref{hbp::subsec:convolutions}
      \\
                         & $\mZ(\mW, \llbracket\tX\rrbracket) = \mW
                           \llbracket \tX \rrbracket$ & $\jac_{\mW} \tZ = \llbracket
                                                        \tX \rrbracket^\top \otimes \mI$
      \\
      \midrule
      % square loss
      Square loss & $\ell(\vf, \vy) = \nicefrac{1}{C}(\vy-\vf)^\top (\vy - \vf)$ & $\jac_{\vf}\ell = 2(\vf - \vy)^{\top}$ & \Cref{hbp::subsec:mselossBackwardPass}
      \\
      % cross-entropy loss
      Softmax cross-entropy & $\ell(\vf, y) = -\onehot(y)^{\top} \log[\vp(\vf)]$ & $\jac_{\vf}\ell = (\vy - \vp(\vf))^{\top}$ & \Cref{hbp::subsec:crossentropylossBackwardPass}
      \\
      \bottomrule
    \end{tabular}
  \end{footnotesize}
\end{table*}

\tikzexternalenable
\begin{figure*}[!t]
  \centering
  \begin{subfigure}[t]{1.0\linewidth}
    \caption{One datum}\label{subfig:background::gradientBackpropagation1}
    \centering\resizebox{\linewidth}{!}{
      {\footnotesize
        % basic setting of a fully-connected neural network with data flow for
% forward pass

\begin{tikzpicture}
  % first two layers
  \node (in1)
  [inner sep=0]
  {\tikz \drawMessagesWithArrows{$\vz^{(0)}$}{ }{ }{\hNodeDistance};};
  \node (layer1)
  [anchor=south west, inner sep=0]
  at (in1.south east)
  {\tikz \drawModuleWithParams{$f^{(1)}_{\vtheta^{(1)}}$}{16}{$\vtheta^{(1)}$}{$\grad{\vtheta^{(1)}}\ell$}{ };};
  \node (out1)
  [inner sep=0, anchor=south west]
  at (layer1.south east)
  {\tikz \drawMessagesWithArrows{$\vz^{(1)}$}{$\grad{\vz^{(1)}}\ell$}{ }{\hNodeDistance};};
  \node (layer2)
  [inner sep=0pt, anchor=south west]
  at (out1.south east)
  {\tikz \drawModuleWithParams{$f^{(2)}_{\vtheta^{(2)}}$}{16}{$\vtheta^{(2)}$}{$\grad{\vtheta^{(2)}}\ell$}{ };};

  % dots with messages
  \node (in2)
  [inner sep=0, anchor=south west]
  at (layer2.south east)
  {\tikz \drawMessagesWithArrows{$\vz^{(2)}$}{$\grad{\vz^{(2)}}\ell$}{ }{\hNodeDistance};};
  \node (dots)
  [xshift=0.75ex, inner sep=0pt, anchor=west]
  at (in2.east)
  {$\dots$};

  \node (inLast)
  [xshift=0.75ex, inner sep=0pt, anchor=west]
  at (dots.east)
  {\tikz \drawMessagesWithArrows{$\vz^{(L-1)}$}{$\grad{\vz^{(L-1)}}\ell$}{ }{\hNodeDistance};};

  \node (layerLast)
  [anchor=south west, inner sep=0]
  at (inLast.south east)
  {\tikz \drawModuleWithParams{$f^{(L)}_{\vtheta^{(L)}}$}{16}{$\vtheta^{(L)}$}{$\grad{\vtheta^{(L)}}\ell$}{ };};
  \node (outLast)
  [inner sep=0, anchor=south west]
  at (layerLast.south east)
  {\tikz \drawMessagesWithArrows{$\vz^{(L)}$}{$\grad{\vz^{(L)}}\ell$}{ }{\hNodeDistance};};

  % loss layer
  \node (lossLayer)
  [inner sep=0pt, anchor=south west]
  at (outLast.south east)
  {\tikz\drawModuleNoParams{$\ell$}{5};};
  \node (loss)
  [inner sep=0, anchor=south west]
  at (lossLayer.south east)
  {\tikz \drawMessagesWithArrows{$\ell$}{ }{ }{\hNodeDistance};};

  % label
  \node (label)
  [inner sep=0, anchor=south west, rotate=-90, xshift = -96, yshift = -50]
  at (lossLayer.south east)
  {\tikz \drawMessagesWithArrows{$\vy$}{ }{ }{\hNodeDistance};};
\end{tikzpicture}

%%% Local Variables:
%%% mode: latex
%%% TeX-master: "../../thesis"
%%% End:

      }}
  \end{subfigure}
  \begin{subfigure}[t]{1.0\linewidth}
    \caption{Batched data, arbitrary transformations}\label{subfig:background::gradientBackpropagation2}
    \centering\resizebox{\linewidth}{!}{
      {\footnotesize
        \input{figures/background/backward_pass_batched}
      }}
  \end{subfigure}
  \begin{subfigure}[t]{1.0\linewidth}
    \caption{Batched data, batched instructions}\label{subfig:background::gradientBackpropagation3}
    \centering\resizebox{\linewidth}{!}{
      {\footnotesize
        % basic setting of a fully-connected neural network with data flow for
% forward pass

\begin{tikzpicture}
  % first two layers
  % \node (in1)
  % [inner sep=0]
  % {\tikz \drawMessagesWithArrows{$\{\vz_{n}^{(0)}\}$}{ }{ }{\hNodeDistance};};
  % \node (layer1)
  % [anchor=south west, inner sep=0]
  % at (in1.south east)
  % {\tikz \drawModuleWithParams{$\vf^{(1)}_{\vtheta^{(1)}}$}{16}{$\vtheta^{(1)}$}{$\grad{\vtheta^{(1)}}\gL$}{ };};
  \node (out1)
  [inner sep=0, anchor=south west]
  % at (layer1.south east)
  {\tikz \drawMessagesWithArrows{\scalebox{0.8}{$\{\vz_{n}^{(0)}\}$}}{ }{ }{\hNodeDistance};};
  \node (layer2)
  [inner sep=0pt, anchor=south west]
  at (out1.south east)
  {\tikz \drawModuleWithParams{$\vf^{(1)}_{\vtheta^{(1)}}$}{16}{$\vtheta^{(1)}$}{$\grad{\vtheta^{(1)}}\gL$}{ };};

  % dots with messages
  \node (in2)
  [inner sep=0, anchor=south west]
  at (layer2.south east)
  {\tikz \drawMessagesWithArrows{\scalebox{0.8}{$\{\vz_{n}^{(2)}\}$}}{\scalebox{0.75}{$\{\grad{\vz_{n}^{(2)}}\gL\}$}}{ }{\hNodeDistance};};
  \node (dots)
  [xshift=0.75ex, inner sep=0pt, anchor=west]
  at (in2.east)
  {$\dots$};

  \node (inLast)
  [xshift=0.75ex, inner sep=0pt, anchor=west]
  at (dots.east)
  {\tikz \drawMessagesWithArrows{\,\scalebox{0.8}{$\{\vz_{n}^{(L-1)}\}$}\,}{\,\scalebox{0.75}{$\{\grad{\vz_{n}^{(L-1)}}\gL\}$}\,}{ }{\hNodeDistance};};

  \node (layerLast)
  [anchor=south west, inner sep=0]
  at (inLast.south east)
  {\tikz \drawModuleWithParams{$\vf^{(L)}_{\vtheta^{(L)}}$}{16}{$\vtheta^{(L)}$}{$\grad{\vtheta^{(L)}}\gL$}{ };};
  \node (outLast)
  [inner sep=0, anchor=south west]
  at (layerLast.south east)
  {\tikz \drawMessagesWithArrows{\scalebox{0.8}{$\{\vz_{n}^{(L)}\}$}}{\scalebox{0.75}{$\{\grad{\vz_{n}^{(L)}}\gL\}$}}{ }{\hNodeDistance};};

  % loss layer
  \node (lossLayer)
  [inner sep=0pt, anchor=south west]
  at (outLast.south east)
  {\tikz\drawModuleNoParams{$\vell$}{5};};
  \node (losses)
  [inner sep=0, anchor=south west]
  at (lossLayer.south east)
  {\tikz \drawMessagesWithArrows{\scalebox{1.0}{$\{\ell_n\}$}}{\scalebox{1.0}{$\{\grad{\ell_n}\gL\}$}}{ }{\hNodeDistance};};

  % label
  \node (label)
  [inner sep=0, anchor=south west, rotate=-90, xshift = -96, yshift = -50]
  at (lossLayer.south east)
  {\tikz \drawMessagesWithArrows{\scalebox{1.0}{$\{\vy_n\}$}}{ }{ }{\hNodeDistance};};

  % Reduction layer
  \node (reductionLayer)
  [inner sep=0pt, anchor=south west]
  at (losses.south east)
  {\tikz\drawModuleNoParams{$\gL$}{5};};
  \node (reducedLoss)
  [inner sep=0, anchor=south west]
  at (reductionLayer.south east)
  {\tikz \drawMessagesWithArrows{$\gL$}{ }{ }{\hNodeDistance};};

\end{tikzpicture}

%%% Local Variables:
%%% mode: latex
%%% TeX-master: "../../thesis"
%%% End:

      }}
  \end{subfigure}
  \caption{\textbf{Gradient backpropagation and (un)awareness of per-sample
      structure in many ML libraries.}
    \subfigref{subfig:background::gradientBackpropagation1} Computation graph of
    of a neural network's loss on a single datum $(\vx, \vy)$. Gradients are
    backpropagated through the graph as described by
    \Cref{def:background::GradientBackpropagation} to obtain
    $\grad{\vtheta}\ell$.
    \subfigref{subfig:background::gradientBackpropagation2} To exploit
    parallelism in the computations, multiple data are stacked into matrices
    $(\mX, \mY)$ which are then processed by a sequence of matrix-to-matrix
    functions $F_{\vtheta^{(l)}}^{(l)}$ into a batch of losses $\vell$, and
    reduced into a scalar $\gL$ via mean reduction. AD in popular ML libraries
    like \pytorch tracks variables on the level of batched tensors. Therefore,
    operations are allowed to build up dependencies between data---such that
    $\gL = \nicefrac{1}{|\sB|} \sum_{n} [\vell(\mX, \mY, \vtheta)]_n$ where each
    component of $\vell$ may depend on \emph{all} data (batch
    normalization~\cite{ioffe2015batch} is such a case)---without breaking
    gradient backpropagation. ML libraries implement VJPS for the
    matrix-to-matrix functions $F_{\vtheta^{(l)}}^{(l)}$. This loses structure
    for operations that treat inputs independently along the batch axis.
    \subfigref{subfig:background::gradientBackpropagation3} The empirical risk
    on a mini-batch (\Cref{eq:background::miniBatchRisk} is such a case: all
    operations in the graph process inputs independently and with the same
    instructions along the batch axis. The following connections to the single
    datum case \subfigref{subfig:background::gradientBackpropagation1} hold:
    $F_{\vtheta^{(l)}}^{(l)} \leftrightarrow \vf_{\vtheta^{(l)}}^{(l)} =
    \vmap(f_{\vtheta^{(l)}}^{(l)}), \vell = \vmap(\ell)$ with $\vmap$ from
    \Cref{def:background::vmap}. Due to the more general support of AD in ML
    libraries for graphs of the form
    \subfigref{subfig:background::gradientBackpropagation2}, their VJPs cannot
    be accessed per-sample.}\label{fig:background::gradientBackpropagation}
\end{figure*}

%%% Local Variables:
%%% mode: latex
%%% TeX-master: "../../thesis"
%%% End:

\tikzexternaldisable

\subsubsection{Backpropagation \& Batching}

To see the interplay between AD and batching
(\Cref{sec:background::MiniBatching}), consider differentiation of the
mini-batch loss (\Cref{eq:background::miniBatchRisk}). Popular ML libraries like
\PyTorch construct the computation graph on the level of tensor-valued variables
with an additional batch axis.

\paragraph{General case:} Starting from the previous differentiation of a single
datum loss, the mini-batch scenario follows by the substitutions
\begin{subequations}\label{eq:background::ADBatchedCase}
  \begin{align}\label{eq:background::ADBatchedCaseSubstitutions}
    \begin{split}
      \vx =: \vz^{(0)}
      \quad&\leftrightarrow\quad
             \mX =: \mZ^{(0)} \in \sR^{|\sB| \times h^{(0)}}\,,
      \\
      \vy
      \quad&\leftrightarrow\quad
             \mY \in \sR^{|\sB| \times C}\,,
      \\
      \vz^{(\ell)}
      \quad&\leftrightarrow\quad
             \mZ^{(l)} \in \sR^{|\sB| \times h^{(l)}}\,,
      \\
      \ell
      \quad&\leftrightarrow\quad
             \vell \in  \sR^{|\sB|}\,,
    \end{split}
  \end{align}
  with stacked data $\mX= (\vx_1\, \vx_2\, \dots\, \vx_{|\sB|})$ and $ \mY =
  (\vy_1\, \vy_2\, \dots\, \vy_{|\sB|})$, and assuming matrix-to-matrix layer
  functions. To account for the reduction of per-sample losses $\vell$, the
  graph is extended by
  \begin{align}
    \gL(\ell) = \mean(\ell) = \frac{1}{|\sB|} \sum_{n=1}^{|\sB|} [\vell]_n\,.
  \end{align}
  The computation graph, shown in
  \Cref{subfig:background::gradientBackpropagation2}, is $\gG = (\gE, \gV)$ with
  nodes
  \begin{align}
    \gV &=
          \left\{
          \vtheta^{(l)}
          \right\}_{l=1}^L
          \cup
          \left\{
          \mZ^{(l)}
          \right\}_{l=0}^L
          \cup
          \left\{
          \mY, \vell, \gL
          \right\}
          \shortintertext{and edges}
          \begin{split}
            \gE &=
                  \left\{
                  (\mZ^{(l-1)}, \mZ^{(l)}),
                  \right\}_{l=1}^L
                  \cup
                  \left\{
                  (\vtheta^{(l)}, \mZ^{(l)})
                  \right\}_{l=1}^{L}
            \\
                &\phantom{=}\,
                  \cup
                  \left\{
                  (\mZ^{(L)}, \vell),
                  (\mY, \vell),
                  (\vell, \gL)
                  \right\}\,.
          \end{split}
  \end{align}
\end{subequations}
Gradient backpropagation (\Cref{def:background::GradientBackpropagation})
carries over the mini-batch loss graph \Cref{eq:background::ADBatchedCase} and
efficiently recovers the gradient $\grad{\vtheta}\gL$.

Popular libraries like \pytorch implement the required functionality, VJPs, for
the matrix-to-matrix functions that process inputs with a batch axis (see
\Cref{hbp::def:generalizedJacobian} for the Jacobian's generaliation to matrix
functions). Hence, operations $\mZ^{(l)} \mapsto \mZ^{(l+1)}$ are allowed to
create dependencies across the batch axis without breaking the gradient
computation\sidenote[][-4\baselineskip]{%
  \Eg batch normalization \cite{ioffe2015batch} introduces
  dependencies along the batch dimension by centering and re-scaling the input
  with statistics computed across the batch axis.}%
.

\paragraph{Per-sample structure:} This work focuses on empirical risks that are
averages over \emph{per-sample} losses (recall
\Cref{eq:background::empiricalRisk}),
\begin{align*}
  \gL_{\sB}(\vtheta)
  =
  \frac{1}{|\sB|}
  \sum_{(\vx_n, \vy_n) \in \sB}
  \ell(f_{\vtheta}(\vx_n), \vy_n)\,.
\end{align*}
Hence, all operations act independently, and using the same instructions, along
the batch dimension. All variables in the graph inherit this independence along
their batch axis,
\begin{subequations}\label{eq:background::matrixNotationBatchedVariables}
  \begin{align}
    \mZ^{(l)}
    &=
      \begin{pmatrix}
        \vz_1^{(l)} & \vz_2^{(l)} & \dots & \vz_{|\sB|}^{(l)}
      \end{pmatrix}\,,\qquad l=0,\dots,L\,,
    \\
    \vell
    &=
      \begin{pmatrix}
        \ell(f_{\vtheta}(\vx_1), \vy_1)
        \\
        \ell(f_{\vtheta}(\vx_2), \vy_2)
        \\
        \vdots
        \\
        \ell(f_{\vtheta}(\vx_{|\sB|}), \vy_{|\sB|})
      \end{pmatrix}\,.
  \end{align}
  For all layers $l=1,\dots,L$, this independence across samples implies block-diagonal input-output Jacobians,
  \begin{align}
    \label{eq:background::perSampleStructureHiddenJacobian}
    \jac_{\mZ^{(l-1)}} \mZ^{(l)}
    =
    \begin{pmatrix}
      \jac_{\vz_1^{(l-1)}} \vz_1^{(l)} & 0 & \dots & 0
      \\
      0 & \jac_{\vz_2^{(l-1)}} \vz_2^{(l)} & \ddots & \vdots
      \\
      \vdots &  \ddots & \ddots & 0
      \\
      0 & \dots & 0 & \jac_{\vz_{|\sB|}^{(l-1)}} \vz_{|\sB|}^{(l)},
    \end{pmatrix}
  \end{align}
  and per-sample block structure in the output-parameter Jacobian
  \begin{align}
    \label{eq:background::perSampleStructureParameterJacobian}
    \jac_{\vtheta^{(l)}} \mZ^{(l)}
    =
    \begin{pmatrix}
      \jac_{\vtheta^{(l)}} \vz_1^{(l)}
      &
        \jac_{\vtheta^{(l)}} \vz_2^{(l)}
      &
        \dots
      &
        \jac_{\vtheta^{(l)}} \vz_{|\sB|}^{(l)},
    \end{pmatrix}\,.
  \end{align}
\end{subequations}
Many implementations of backpropagation make it difficult to access this
per-sample structure. They only expose VJPs $\vv \mapsto [\jac_{\mZ^{(l)}}
\mZ^{(l+1)}]^{\top} \vv$ and $\vv \mapsto [\jac_{\vtheta^{(l)}}
\mZ^{(l+1)}]^{\top} \vv$ for
\Cref{eq:background::perSampleStructureHiddenJacobian,eq:background::perSampleStructureParameterJacobian}.
While this allows supporting AD of more general graphs than those of an
empirical risk (\Cref{eq:background::empiricalRisk}, it limits access to only
the \emph{average} gradient
\begin{align}\label{eq:backround::miniBatchGradientNotation}
  \vg_{\sB}(\vtheta) :=
  \grad{\vtheta}\gL_{\sB}(\vtheta) =
  \frac{1}{|\sB|} \sum_{(\vx_n, \vy_n) \in \sB}
  \grad{\vtheta}\ell(f_{\vtheta}(\vx_n), \vy_n)
\end{align}
when differentiating an empirical risk. Computing per-sample gradients is not
more demanding than computing the average gradient---the only difference is
taking the average---but their computation is not supported. More flexible
access to the Jacobians
\Cref{eq:background::perSampleStructureHiddenJacobian,eq:background::perSampleStructureParameterJacobian}
through per-sample VJPs (or MJPs), enables the computation of various
higher-order quantities. Their efficient realization will be the main focus of
\Cref{chap:backpack,chap:hbp,chap:vivit}.

To highlight independence across the batch axis in a computation graph, a set
notation will be preferred over the matrix notation
(\Cref{eq:background::ADBatchedCaseSubstitutions,eq:background::matrixNotationBatchedVariables})
in the following. \Eg the text uses $\{ \vz_n^{(l)} \}_n$, or just $\{
\vz_n^{(l)} \}$, instead of $\mZ^{(l)}$.
\Cref{subfig:background::gradientBackpropagation3} illustrates this set
notation\sidenote[][-10.5\baselineskip]{%
  This notation is closer to recent developments in AD for ML
  libraries~\cite{bradbury2018jax,he2021functorch} that separate batching and AD
  more clearly through vectorization via a $\vmap$ interface
  (\Cref{def:background::vmap}). A different way to arrive at the batched
  computation graph in \Cref{subfig:background::gradientBackpropagation3} is to
  start from the computation graph of a single datum's loss
  $\ell(f_{\vtheta}(\vx), \vy)$ and vectorize it to obtain the set of graphs $
  \{\ell(f_{\vtheta}(\vx_n), \vy_n)\}$, which can be stacked into a single graph
  that produces $\vell$. Appending the reduction node to this graph, one obtains
  the computation graph for the average loss.}.

%%% Local Variables:
%%% mode: latex
%%% TeX-master: "../thesis"
%%% End:


%%% Local Variables:
%%% mode: latex
%%% TeX-master: "../thesis"
%%% End:
