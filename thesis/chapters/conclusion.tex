Contemporary deep learning is powered by methods that solely rely on the
gradient. This is reflected in popular machine learning libraries which
prioritize its computation. However, it narrows research to focus on
gradient-based algorithms that are not agnostic to the empirical risk's
stochasticity and geometry beyond first order. To advance the field, we need to
explore the potential of higher-order information beyond the gradient. One main
hindrance to further explore its utility has been that it is complicated to
implement, which makes it difficult for practitioners to try it out. Therefore,
one major goal of this work was to ease experimentation with higher-order
information by making it as conveniently accessible as the gradient. This thesis
demonstrates that rich information beyond the gradient is \emph{affordable}, can
be made \emph{readily available} in existing machine learning libraries, and is
\emph{useful} to enable novel approaches for advancing deep learning:

\subsubsection{\ref{enum:background::Q1} \emph{Which information beyond the
    gradient is efficiently accessible?}}

\Cref{fig:conclusion::higher_order_information} provides an overview of the
higher-order information made accessible in this work: per-sample
gradients---whose empirical mean is the mini-batch gradient---can be explicitly
computed, or reduced into higher-order statistics like the gradient variance.
Light-weight structural approximations through diagonal and Kronecker matrices
enable the computation of per-layer second-order derivatives in the Hessian or
generalized Gauss-Newton. The generalized Gauss-Newton's outer product structure
allows to go beyond per-layer terms, even to compute with the full matrix, and
to access curvature noise.

\begin{figure}[!b]
  \centering
  \tikzexternalenable%
  % NOTE Cannot externalize. There is a node that contains another tikzpicture.
  % This requires a fix via savebox which is not picked up by externalize
  \tikzexternaldisable%
  % NOTE tikzpicture in node of another tikzpicture: how to screen of from
% inheriting style? (https://tex.stackexchange.com/a/23419)
\newsavebox\mybox
\begin{lrbox}{\mybox}
  % NOTE tikzpicture in node of another tikzpicture: how to screen of from
% inheriting style? (https://tex.stackexchange.com/a/23419)
\newsavebox\mybox
\begin{lrbox}{\mybox}
  % NOTE tikzpicture in node of another tikzpicture: how to screen of from
% inheriting style? (https://tex.stackexchange.com/a/23419)
\newsavebox\mybox
\begin{lrbox}{\mybox}
  \input{figures/background/higher_order_information}
\end{lrbox}

\begin{tikzpicture}
  % NOTE without the savebox solution, the inner sep would be inherited by the
  % sub-tikzpicture
  \node[inner sep=0pt] (higherOrder) {
    \usebox\mybox
  };

  \draw [draw=none, font=\footnotesize] (higherOrder.south west) to node [midway, above, yshift=0.065\linewidth, xshift=0.075\linewidth, anchor=south] (FirstOrder) {$\{ \lVert \grad{\vtheta}\ell_n \rVert_2\}, \, \sum_n (\grad{\vtheta}\ell_n)^{\odot 2}$} (higherOrder.north east);

  \draw [draw=none, font=\footnotesize] (higherOrder.south west) to node [midway, anchor=center, align=center, xshift = -0.0775\linewidth, yshift = 0.0275\linewidth] (DiagonalCurvature) {$\diag(\gradsquared{\vtheta}\gL),$\\ $\diag(\mG), \, \diag(\mG^{(\text{MC})})$} (higherOrder.north east);

  \draw [draw=none, font=\footnotesize] (higherOrder.south west) to node [midway, below, yshift=-0.05\linewidth, xshift=-0.14\linewidth, yshift = 0.0375\linewidth, anchor=north, align = center] (KroneckerCurvature) {$\{\mathrm{KFAC}(\gradsquared{\vtheta^{(l)}}\gL)\},$ \\ $\{\mathrm{KFLR}(\gradsquared{\vtheta^{(l)}}\gL)\}, \, \{\mathrm{KFRA}(\gradsquared{\vtheta^{(l)}}\gL)\}$} (higherOrder.north east);

  \node [anchor = north west, align=left, font=\footnotesize, yshift=0.0275\linewidth, xshift=-0.065\linewidth] (ViViTCurvature) at (DiagonalCurvature.east) {$\hspace{0.13\linewidth}\eig(\mG),$ \\ $\hspace{0.09\linewidth}\mG^{-1}\vv,$ \\ $\{\gamma_{n,k}, \lambda_{n,k}\}$};
\end{tikzpicture}

%%% Local Variables:
%%% mode: latex
%%% TeX-master: "../../thesis"
%%% End:

\end{lrbox}

\begin{tikzpicture}
  % NOTE without the savebox solution, the inner sep would be inherited by the
  % sub-tikzpicture
  \node[inner sep=0pt] (higherOrder) {
    \usebox\mybox
  };

  \draw [draw=none, font=\footnotesize] (higherOrder.south west) to node [midway, above, yshift=0.065\linewidth, xshift=0.075\linewidth, anchor=south] (FirstOrder) {$\{ \lVert \grad{\vtheta}\ell_n \rVert_2\}, \, \sum_n (\grad{\vtheta}\ell_n)^{\odot 2}$} (higherOrder.north east);

  \draw [draw=none, font=\footnotesize] (higherOrder.south west) to node [midway, anchor=center, align=center, xshift = -0.0775\linewidth, yshift = 0.0275\linewidth] (DiagonalCurvature) {$\diag(\gradsquared{\vtheta}\gL),$\\ $\diag(\mG), \, \diag(\mG^{(\text{MC})})$} (higherOrder.north east);

  \draw [draw=none, font=\footnotesize] (higherOrder.south west) to node [midway, below, yshift=-0.05\linewidth, xshift=-0.14\linewidth, yshift = 0.0375\linewidth, anchor=north, align = center] (KroneckerCurvature) {$\{\mathrm{KFAC}(\gradsquared{\vtheta^{(l)}}\gL)\},$ \\ $\{\mathrm{KFLR}(\gradsquared{\vtheta^{(l)}}\gL)\}, \, \{\mathrm{KFRA}(\gradsquared{\vtheta^{(l)}}\gL)\}$} (higherOrder.north east);

  \node [anchor = north west, align=left, font=\footnotesize, yshift=0.0275\linewidth, xshift=-0.065\linewidth] (ViViTCurvature) at (DiagonalCurvature.east) {$\hspace{0.13\linewidth}\eig(\mG),$ \\ $\hspace{0.09\linewidth}\mG^{-1}\vv,$ \\ $\{\gamma_{n,k}, \lambda_{n,k}\}$};
\end{tikzpicture}

%%% Local Variables:
%%% mode: latex
%%% TeX-master: "../../thesis"
%%% End:

\end{lrbox}

\begin{tikzpicture}
  % NOTE without the savebox solution, the inner sep would be inherited by the
  % sub-tikzpicture
  \node[inner sep=0pt] (higherOrder) {
    \usebox\mybox
  };

  \draw [draw=none, font=\footnotesize] (higherOrder.south west) to node [midway, above, yshift=0.065\linewidth, xshift=0.075\linewidth, anchor=south] (FirstOrder) {$\{ \lVert \grad{\vtheta}\ell_n \rVert_2\}, \, \sum_n (\grad{\vtheta}\ell_n)^{\odot 2}$} (higherOrder.north east);

  \draw [draw=none, font=\footnotesize] (higherOrder.south west) to node [midway, anchor=center, align=center, xshift = -0.0775\linewidth, yshift = 0.0275\linewidth] (DiagonalCurvature) {$\diag(\gradsquared{\vtheta}\gL),$\\ $\diag(\mG), \, \diag(\mG^{(\text{MC})})$} (higherOrder.north east);

  \draw [draw=none, font=\footnotesize] (higherOrder.south west) to node [midway, below, yshift=-0.05\linewidth, xshift=-0.14\linewidth, yshift = 0.0375\linewidth, anchor=north, align = center] (KroneckerCurvature) {$\{\mathrm{KFAC}(\gradsquared{\vtheta^{(l)}}\gL)\},$ \\ $\{\mathrm{KFLR}(\gradsquared{\vtheta^{(l)}}\gL)\}, \, \{\mathrm{KFRA}(\gradsquared{\vtheta^{(l)}}\gL)\}$} (higherOrder.north east);

  \node [anchor = north west, align=left, font=\footnotesize, yshift=0.0275\linewidth, xshift=-0.065\linewidth] (ViViTCurvature) at (DiagonalCurvature.east) {$\hspace{0.13\linewidth}\eig(\mG),$ \\ $\hspace{0.09\linewidth}\mG^{-1}\vv,$ \\ $\{\gamma_{n,k}, \lambda_{n,k}\}$};
\end{tikzpicture}

%%% Local Variables:
%%% mode: latex
%%% TeX-master: "../../thesis"
%%% End:

  \tikzexternaldisable%
  \caption{\textbf{Higher-order information made available by this work (extends
      \Cref{fig:background::higher_order_information}).} Quantities are roughly
    mapped onto the landscape spanned by the stochasticity and geometry of the
    loss. They can be categorized as (i) gradient statistics, (ii) diagonal
    curvature approximations, (iii) Kronecker-factored curvature approximations,
    and (iv) noise-aware curvature from the generalized Gauss-Newton's low-rank
    structure.}\label{fig:conclusion::higher_order_information}
\end{figure}

\subsubsection{\ref{enum:background::Q2} \emph{How to compute this
    information---conveniently, automatically, and efficiently---re-using the
    existing backpropagation implementation of ML frameworks?}}

All quantities presented in this thesis are phrased as extensions of a
standard backward pass.

Therefore, they can be computed \emph{efficiently} and at the same time as the
gradient (\Cref{fig:conclusion::extensionStandardBackwardPass}). Gradient
statistics share most computations with the gradient, and can recycle
information from the standard backward pass. Approximate second-order
derivatives require sending additional information through the computation
graph. In runtime, the additional work is often significantly reduced through
hardware parallelism.

\begin{figure*}[!t]
  \centering
  \tikzexternalenable%
  \resizebox{\linewidth}{!}{%
    \input{figures/conclusion/backward_pass_data_with_backpack}
  }%
  \tikzexternaldisable%
  \caption{\textbf{All information in
      \Cref{fig:conclusion::higher_order_information} is an extension of the
      standard backward pass for the gradient.} Forward
    ({\protect\tikz[baseline=-0.5ex]{\protect\draw[fill=forwardFill, draw=black,
        line width = 0.02mm] circle (0.75ex);}}) and backward
    ({\protect\tikz[baseline=-0.5ex]{\protect\draw[fill=backwardGradientFill,
        draw=none] circle (0.75ex);}}) pass are implemented by machine learning
    libraries. Minimal invasion by adding an extended backward pass
    ({\protect\tikz[baseline=-0.5ex]{\protect\draw[fill=backwardHessianFill,
        draw=none] circle (0.75ex);}}) allows to compute various other
    quantities. The specifics
    ({\protect\tikz[baseline=-0.75ex]{\protect\node[inner sep=0pt, draw=none]
        {\includegraphics[width=2.5ex]{../repos/backpack-paper/tex/logo/backpack_logo_github}};})}
    on what and how to backpropagate depend on the
    quantity of interest.}\label{fig:conclusion::extensionStandardBackwardPass}
\end{figure*}

\marginnote{
  \begin{center}
    \tikzexternalenable%
    \resizebox{\linewidth}{!}{%
      \begin{tikzpicture}
  \node
  [inner sep=0, anchor=south west]
  (input)
  {\tikz \drawMessagesWithArrows{\,\scalebox{0.8}{$\{\vz_{n}^{(l-1)}\}$}\,}{\,\scalebox{0.75}{$\{\grad{\vz_{n}^{(l-1)}}\gL\}$}\,}{ \includegraphics[width=3.7ex]{../repos/backpack-paper/tex/logo/backpack_logo_github} }{\hNodeDistance};};
  \node (layerLast)
  [anchor=south west, inner sep=0]
  at (input.south east)
  {\tikz \drawModuleWithParams{$\vf^{(l)}_{\vtheta^{(l)}}$}{16}{$\vtheta^{(l)}$}{$\grad{\vtheta^{(l)}}\gL$}{ \includegraphics[width=3.7ex]{../repos/backpack-paper/tex/logo/backpack_logo_github} };};
  \node (outLast)
  [inner sep=0, anchor=south west]
  at (layerLast.south east)
  {\tikz \drawMessagesWithArrows{\scalebox{0.8}{$\{\vz_{n}^{(l)}\}$}}{\scalebox{0.75}{$\{\grad{\vz_{n}^{(l)}}\gL\}$}}{ \includegraphics[width=3.7ex]{../repos/backpack-paper/tex/logo/backpack_logo_github} }{\hNodeDistance};};
\end{tikzpicture}

%%% Local Variables:
%%% mode: latex
%%% TeX-master: "../../thesis"
%%% End:

    }%
    \tikzexternaldisable%
  \end{center}
  \captionof{figure}{\textbf{Modularity is crucial for automation and
      extensibility.} Given a functioning backpropagation procedure, the
    extended backward pass in
    \Cref{fig:conclusion::extensionStandardBackwardPass} reduces to specifying a
    set of rules for each module and each quantity. This simplifies adding new
    operations, while preserving automatic
    computation.}\label{fig:conclusion::Modularity}
}

Their formulations vary in what, and how, objects are being backpropagated
through layers, and how the target quantities are extracted for each parameter.
As these rules are defined on a per-module basis
(\Cref{fig:conclusion::Modularity}), the approach is extensible and guarantees
\emph{automatic} computation.

From an implementation perspective, this is achieved through light-weight
extension of an existing gradient backpropagation implementation without
implementing a new framework. This enables easy integration into existing
machine learning libraries, and makes it \emph{convenient} for practitioners to
extend their code at minimal overhead.

\subsubsection{\ref{enum:background::Q3} \emph{How to use this information to
    advance gradient-based deep learning?}}

In addition to second-order optimization, various deep learning methods require
higher-order information: Laplace
approximations~\cite[\eg\!][]{daxberger2021laplace}, model compression
(pruning)~\cite[\eg\!][]{singh2020woodfisher}, differential
privacy~\cite[\eg\!][]{abadi2016deep}, importance
sampling~\cite[\eg\!][]{katharopoulos2018samples}, variance
adaptation~\cite[\eg\!][]{balles2022noise}, parameter
initialization~\cite[\eg\!][]{skorski2021revisiting}, batch size
adaptation~\cite[\eg\!][]{balles2017coupling}, \etc Some of them were briefly
outlined in this thesis to motivate the demand that this information be more
readily available. Easier access to this information enables more efficient and
creative research in these areas and helps establish the resulting methods
through user-friendly implementations.

As a use case, this thesis focused on enabling a deeper look into the inner
workings of neural networks during training through the lens of higher-order
information. This allows to identify common failure modes, which makes training
less painful, and thereby highlights the utility of higher-order information for
deep learning.

Going further, this work also underlines the unexplored nature of higher-order
information: the developed extended automatic differentiation functionality
enables novel efficient computational schemes that address shortcomings in
existing approaches. This allows to build more powerful local approximations of
the loss landscape which are agnostic to noise, curvature, and even noise in the
curvature. Such works help to identify important techniques to reduce run time,
make algorithms that rely on such information more competitive with
gradient-based methods, and shape the development of future machine learning
frameworks.

\section{Summary \& Impact}

\subsubsection{Extending Backpropagation to the Hessian}

Gradient backpropagation is efficient, automated, and extensible.
\Cref{chap:hbp} presented how---for sequential feedforward architectures---this
carries over to second-order derivatives: just like backpropagation recovers the
layer-wise gradients, Hessian backpropagation recovers the per-layer Hessians.
It fully aligns the computation of local Hessians with gradients and unifies the
view on block-diagonal curvature approximations like the block-diagonal
generalized Gauss-Newton~\cite{schraudolph2002fast},
Fisher~\cite{amari1998natural}, and its Kronecker-factorized
approximations~\cite{martens2015optimizing,grosse2016kronecker,botev2017practical,wei2018bdapch}.

\subsubsection{Packing More into Backprop}

The \backpack library, presented in \Cref{chap:backpack}, provides efficient
access to various deep learning quantities through implementing the insights on
extended backpropagation for the Hessian on top of PyTorch. During a standard
backward pass that computes the average gradient, it extracts (i) per-sample
gradients and gradient statistics, and (ii) approximate second-order derivatives
in the form of diagonal and Kronecker-factorized curvature. This often adds only
little overhead. \backpack easily integrates into existing code and simplifies
experimentation with the above quantities: it powers other libraries for
Bayesian applications with Laplace
approximations~\cite{daxberger2021laplace,immer2021improving},
out-of-distribution generalization~\cite{gulrajani2021in,rame2022fishr}, and
differential privacy \cite{yousefpour2021opacus}, as well as the follow-up works
in this thesis (\cockpit~\cite{schneider2021cockpit},
\vivit~\cite{dangel2022vivit}). More than two years after its release, the
library is still actively used, with
\href{https://pypistats.org/packages/backpack-for-pytorch}{multiple hundred
  downloads per week} at the time of writing (July 2022).

\subsubsection{Enabling a Closer Look Into Neural Nets}
Higher-order information as provided by BackPACK is valuable to guide neural
network training. Common methods for real-time training diagnostics, such as
monitoring the loss, are limited because they only indicate whether a model is
training, but not why. The \cockpit library, presented in \Cref{chap:cockpit},
enables a closer look into neural networks during training. The live-monitoring
tool visualizes established, recently proposed
\cite{mahsereci2017early,balles2017coupling,byrd2012sample,bollapragada2017adaptive,yao2020pyhessian,thomas2020interplay,liu2020understanding},
and novel summary statistics that are efficiently computed by BackPACK. It
allows to identify common bugs in the machine learning pipeline, such as
improper data pre-processing or vanishing gradients, but also to guide learning
rate selection, and to study implicit
regularization~\cite{mulayoff2020unique,ginsburg2020regularization}. This
showcases the potential of higher-order information to assist practitioners.

\subsubsection{Enabling Novel Ways to Compute with Curvature}

BackPACK's extended automatic differentiation functionality enables algorithmic
advances to tackle limitations of existing curvature proxies: diagonal or
Kronecker-factorized curvatures are (i) not agnostic to noise in the mini-batch,
(ii) strict approximations that do not become exact in any limit, and (iii)
restricted to the block diagonal. \vivit's quantities, presented in
\Cref{chap:vivit}, address this through the generalized Gauss-Newton's low-rank
structure, which allows for exact computation with the full---rather than
block-diagonal---matrix, and principled approximations to reduce cost in
exchange for less accuracy. \vivit enables efficient computation of spectral
properties, as well as directional gradients and curvatures on a per-sample
basis that quantify noise. Monitoring this noise through signal-to-noise ratios
helps understand its characteristics in deep learning~\cite{faghri2020study} and
to identify challenges for optimization and generalization from the interplay
between noise and curvature~\cite{thomas2020interplay}.

\section{Future Work}

Deep learning needs more than just the gradient. To leverage the full potential
of higher-order information, we need to (i) build more---and refine
existing---tools to (ii) study and better understand algorithmic challenges in
deep learning, and (iii) amplify the practicality of such next-generation
algorithms through user-friendly and highly-efficient implementations like
gradient-based methods.

\subsubsection{Extending Cockpit}

\Cref{chap:cockpit} demonstrated \cockpit's utility for identifying common bugs
in the machine learning pipeline. These failures were deliberately designed on
well-known, standardized machine learning problems, to illustrate \cockpit's
purpose. Since they were implicitly tuned over many years to work well with
currently popular methods, they rarely exhibit failure modes. \cockpit will be
even more useful for debugging unknown, non-standardized problems that have
\emph{not} undergone such tuning, and are therefore extremely likely to exhibit
failures. We want to establish \cockpit as a tool for practitioners facing such
problems and are looking forward to its first success stories ``in the wild''.

\cockpit also provides functionality for scientific analyses of neural networks.
Training such models is often wasteful in computation, \eg using large grid
searches whose computations are effectively discarded after identifying one
well-performing set of hyperparameters. \cockpit's summary statistics are
condensed and could be stored for a large number of training trajectories
corresponding to different hyperparameter settings. These trajectories could be
collected into a dataset that could serve for the analysis of optimization
algorithms to understand their implicit bias, to study properties of neural
networks that generalize well, or for meta-learning optimization strategies.

To further improve the meaningfulness and interpretability of \cockpit's
instruments, its control over parts of the network could be made more
customizable: currently, most quantities can be computed either on all
parameters, or per parameter. For very large networks, it will be more practical
to group parameters, and to compute and visualize \cockpit's instruments per
parameter group.

\subsubsection{Noise-aware Second-order Methods}

Newton steps are powerful, but their stability is strongly affected by noise:
one corrupted step might undo all previous progress. Improving their stability
is thus one key challenge to make them work in the mini-batch setting. To do
that, we need to quantify noise in the mini-batch. But popular curvature proxies
used in second-order methods are not noise-agnostic. Therefore, we need
curvature approximations that give access to noise, \eg through per-sample
information as provided through \vivit. Such information could be used to
develop noise-aware stabilizing mechanisms for Newton steps, like damping.

\subsubsection{Optimizing Run Time \& Advancing Automatic Differentiation}

In contrast to gradient-based methods, the run time performance of higher-order
methods can still be significantly improved: \eg BackPACK outperforms naive
implementations (like for-loops) and achieves practical overheads. But it relies
on PyTorch's Python API, which is sometimes not flexible enough, and therefore
realizes some functionality through less efficient workarounds. Recent advances
in automatic differentiation, like JAX~\cite{bradbury2018jax} and
functorch~\cite{he2021functorch}, rely on function transformation to achieve a
clean separation of automatic differentiation and batching, and allow for more
efficient implementations through just-in-time compilation.

Often, performance improvements are achieved through leveraging linear algebra,
like properties of the Kronecker product~\cite{loan2000ubiquitous} and matrix
decompositions~\cite[\eg\!][]{dangel2020backpack,dangel2022vivit}. Recent work
suggests that there is potential for further improvements, as a number of these
optimizations are not yet realized~\cite{sankaran2022benchmarking}. Therefore,
one direction would be to improve the automated optimization of operations in
second-order methods in these libraries. This would further reduce the overhead
of second-order methods stemming from poor implementation, and make these
performance gains widely available to the machine learning community.

%%% Local Variables:
%%% mode: latex
%%% TeX-master: "../thesis"
%%% End:
